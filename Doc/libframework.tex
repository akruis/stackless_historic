\section{Standard Module \sectcode{FrameWork}}
\stmodindex{FrameWork}
\label{module-FrameWork}

The \code{FrameWork} module contains classes that together provide a
framework for an interactive Macintosh application. The programmer
builds an application by creating subclasses that override various
methods of the bases classes, thereby implementing the functionality
wanted. Overriding functionality can often be done on various
different levels, i.e. to handle clicks in a single dialog window in a
non-standard way it is not necessary to override the complete event
handling.

The \code{FrameWork} is still very much work-in-progress, and the
documentation describes only the most important functionality, and not
in the most logical manner at that. Examine the source or the examples
for more details.

The \code{FrameWork} module defines the following functions:

\setindexsubitem{(in module FrameWork)}

\begin{funcdesc}{Application}{}
An object representing the complete application. See below for a
description of the methods. The default \code{__init__} routine
creates an empty window dictionary and a menu bar with an apple menu.
\end{funcdesc}

\begin{funcdesc}{MenuBar}{}
An object representing the menubar. This object is usually not created
by the user.
\end{funcdesc}

\begin{funcdesc}{Menu}{bar, title\optional{, after}}
An object representing a menu. Upon creation you pass the
\code{MenuBar} the menu appears in, the \var{title} string and a
position (1-based) \var{after} where the menu should appear (default:
at the end).
\end{funcdesc}

\begin{funcdesc}{MenuItem}{menu, title\optional{, shortcut, callback}}
Create a menu item object. The arguments are the menu to crate the
item it, the item title string and optionally the keyboard shortcut
and a callback routine. The callback is called with the arguments
menu-id, item number within menu (1-based), current front window and
the event record.

In stead of a callable object the callback can also be a string. In
this case menu selection causes the lookup of a method in the topmost
window and the application. The method name is the callback string
with \code{'domenu_'} prepended.

Calling the \code{MenuBar} \code{fixmenudimstate} method sets the
correct dimming for all menu items based on the current front window.
\end{funcdesc}

\begin{funcdesc}{Separator}{menu}
Add a separator to the end of a menu.
\end{funcdesc}

\begin{funcdesc}{SubMenu}{menu, label}
Create a submenu named \var{label} under menu \var{menu}. The menu
object is returned.
\end{funcdesc}

\begin{funcdesc}{Window}{parent}
Creates a (modeless) window. \var{Parent} is the application object to
which the window belongs. The window is not displayed until later.
\end{funcdesc}

\begin{funcdesc}{DialogWindow}{parent}
Creates a modeless dialog window.
\end{funcdesc}

\begin{funcdesc}{windowbounds}{width, height}
Return a \code{(left, top, right, bottom)} tuple suitable for creation
of a window of given width and height. The window will be staggered
with respect to previous windows, and an attempt is made to keep the
whole window on-screen. The window will however always be exact the
size given, so parts may be offscreen.
\end{funcdesc}

\begin{funcdesc}{setwatchcursor}{}
Set the mouse cursor to a watch.
\end{funcdesc}

\begin{funcdesc}{setarrowcursor}{}
Set the mouse cursor to an arrow.
\end{funcdesc}

\subsection{Application objects}
Application objects have the following methods, among others:

\setindexsubitem{(Application method)}

\begin{funcdesc}{makeusermenus}{}
Override this method if you need menus in your application. Append the
menus to \code{self.menubar}.
\end{funcdesc}

\begin{funcdesc}{getabouttext}{}
Override this method to return a text string describing your
application. Alternatively, override the \code{do_about} method for
more elaborate about messages.
\end{funcdesc}

\begin{funcdesc}{mainloop}{\optional{mask, wait}}
This routine is the main event loop, call it to set your application
rolling. \var{Mask} is the mask of events you want to handle,
\var{wait} is the number of ticks you want to leave to other
concurrent application (default 0, which is probably not a good
idea). While raising \code{self} to exit the mainloop is still
supported it is not recommended, call \code{self._quit} instead.

The event loop is split into many small parts, each of which can be
overridden. The default methods take care of dispatching events to
windows and dialogs, handling drags and resizes, Apple Events, events
for non-FrameWork windows, etc.

In general, all event handlers should return 1 if the event is fully
handled and 0 otherwise (because the front window was not a FrameWork
window, for instance). This is needed so that update events and such
can be passed on to other windows like the Sioux console window.
Calling \code{MacOS.HandleEvent} is not allowed within \var{our_dispatch}
or its callees, since this may result in an infinite loop if the
code is called through the python inner-loop event handler.
\end{funcdesc}

\begin{funcdesc}{asyncevents}{onoff}
Call this method with a nonzero parameter to enable
asynchronous event handling. This will tell the inner interpreter loop
to call the application event handler \var{async_dispatch} whenever events
are available. This will cause FrameWork window updates and the user
interface to remain working during long computations, but will slow the
interpreter down and may cause surprising results in non-reentrant code
(such as FrameWork itself). By default \var{async_dispatch} will immedeately
call \var{our_dispatch} but you may override this to handle only certain
events asynchronously. Events you do not handle will be passed to Sioux
and such.

The old on/off value is returned.
\end{funcdesc}

\begin{funcdesc}{_quit}{}
Terminate the event \code{mainloop} at the next convenient moment.
\end{funcdesc}

\begin{funcdesc}{do_char}{c, event}
The user typed character \var{c}. The complete details of the event
can be found in the \var{event} structure. This method can also be
provided in a \code{Window} object, which overrides the
application-wide handler if the window is frontmost.
\end{funcdesc}

\begin{funcdesc}{do_dialogevent}{event}
Called early in the event loop to handle modeless dialog events. The
default method simply dispatches the event to the relevant dialog (not
through the the \code{DialogWindow} object involved). Override if you
need special handling of dialog events (keyboard shortcuts, etc).
\end{funcdesc}

\begin{funcdesc}{idle}{event}
Called by the main event loop when no events are available. The
null-event is passed (so you can look at mouse position, etc).
\end{funcdesc}

\subsection{Window Objects}

Window objects have the following methods, among others:

\setindexsubitem{(Window method)}

\begin{funcdesc}{open}{}
Override this method to open a window. Store the MacOS window-id in
\code{self.wid} and call \code{self.do_postopen} to register the
window with the parent application.
\end{funcdesc}

\begin{funcdesc}{close}{}
Override this method to do any special processing on window
close. Call \code{self.do_postclose} to cleanup the parent state.
\end{funcdesc}

\begin{funcdesc}{do_postresize}{width, height, macoswindowid}
Called after the window is resized. Override if more needs to be done
than calling \code{InvalRect}.
\end{funcdesc}

\begin{funcdesc}{do_contentclick}{local, modifiers, event}
The user clicked in the content part of a window. The arguments are
the coordinates (window-relative), the key modifiers and the raw
event.
\end{funcdesc}

\begin{funcdesc}{do_update}{macoswindowid, event}
An update event for the window was received. Redraw the window.
\end{funcdesc}

\begin{funcdesc}{do_activate}{activate, event}
The window was activated (\code{activate==1}) or deactivated
(\code{activate==0}). Handle things like focus highlighting, etc.
\end{funcdesc}

\subsection{ControlsWindow Object}

ControlsWindow objects have the following methods besides those of
\code{Window} objects:

\setindexsubitem{(ControlsWindow method)}

\begin{funcdesc}{do_controlhit}{window, control, pcode, event}
Part \code{pcode} of control \code{control} was hit by the
user. Tracking and such has already been taken care of.
\end{funcdesc}

\subsection{ScrolledWindow Object}

ScrolledWindow objects are ControlsWindow objects with the following
extra methods:

\setindexsubitem{(ScrolledWindow method)}

\begin{funcdesc}{scrollbars}{\optional{wantx, wanty}}
Create (or destroy) horizontal and vertical scrollbars. The arguments
specify which you want (default: both). The scrollbars always have
minimum \code{0} and maximum \code{32767}.
\end{funcdesc}

\begin{funcdesc}{getscrollbarvalues}{}
You must supply this method. It should return a tuple \code{x, y}
giving the current position of the scrollbars (between \code{0} and
\code{32767}). You can return \code{None} for either to indicate the
whole document is visible in that direction.
\end{funcdesc}

\begin{funcdesc}{updatescrollbars}{}
Call this method when the document has changed. It will call
\code{getscrollbarvalues} and update the scrollbars.
\end{funcdesc}

\begin{funcdesc}{scrollbar_callback}{which, what, value}
Supplied by you and called after user interaction. \code{Which} will
be \code{'x'} or \code{'y'}, \code{what} will be \code{'-'},
\code{'--'}, \code{'set'}, \code{'++'} or \code{'+'}. For
\code{'set'}, \code{value} will contain the new scrollbar position.
\end{funcdesc}

\begin{funcdesc}{scalebarvalues}{absmin, absmax, curmin, curmax}
Auxiliary method to help you calculate values to return from
\code{getscrollbarvalues}. You pass document minimum and maximum value
and topmost (leftmost) and bottommost (rightmost) visible values and
it returns the correct number or \code{None}.
\end{funcdesc}

\begin{funcdesc}{do_activate}{onoff, event}
Takes care of dimming/highlighting scrollbars when a window becomes
frontmost vv. If you override this method call this one at the end of
your method.
\end{funcdesc}

\begin{funcdesc}{do_postresize}{width, height, window}
Moves scrollbars to the correct position. Call this method initially
if you override it.
\end{funcdesc}

\begin{funcdesc}{do_controlhit}{window, control, pcode, event}
Handles scrollbar interaction. If you override it call this method
first, a nonzero return value indicates the hit was in the scrollbars
and has been handled.
\end{funcdesc}

\subsection{DialogWindow Objects}

DialogWindow objects have the following methods besides those of
\code{Window} objects:

\setindexsubitem{(DialogWindow method)}

\begin{funcdesc}{open}{resid}
Create the dialog window, from the DLOG resource with id
\var{resid}. The dialog object is stored in \code{self.wid}.
\end{funcdesc}

\begin{funcdesc}{do_itemhit}{item, event}
Item number \var{item} was hit. You are responsible for redrawing
toggle buttons, etc.
\end{funcdesc}
