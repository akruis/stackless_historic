\section{\module{sha} ---
         SHA-1 message digest algorithm}

\declaremodule{builtin}{sha}
\modulesynopsis{NIST's secure hash algorithm, SHA.}
\sectionauthor{Fred L. Drake, Jr.}{fdrake@acm.org}


This module implements the interface to NIST's\index{NIST} secure hash 
algorithm,\index{Secure Hash Algorithm} known as SHA-1.  SHA-1 is an
improved version of the original SHA hash algorithm.  It is used in
the same way as the \refmodule{md5} module:\ use \function{new()}
to create an sha object, then feed this object with arbitrary strings
using the \method{update()} method, and at any point you can ask it
for the \dfn{digest} of the concatenation of the strings fed to it
so far.\index{checksum!SHA}  SHA-1 digests are 160 bits instead of
MD5's 128 bits.


\begin{funcdesc}{new}{\optional{string}}
  Return a new sha object.  If \var{string} is present, the method
  call \code{update(\var{string})} is made.
\end{funcdesc}


The following values are provided as constants in the module and as
attributes of the sha objects returned by \function{new()}:

\begin{datadesc}{blocksize}
  Size of the blocks fed into the hash function; this is always
  \code{1}.  This size is used to allow an arbitrary string to be
  hashed.
\end{datadesc}

\begin{datadesc}{digest_size}
  The size of the resulting digest in bytes.  This is always
  \code{20}.
\end{datadesc}


An sha object has the same methods as md5 objects:

\begin{methoddesc}[sha]{update}{arg}
Update the sha object with the string \var{arg}.  Repeated calls are
equivalent to a single call with the concatenation of all the
arguments: \code{m.update(a); m.update(b)} is equivalent to
\code{m.update(a+b)}.
\end{methoddesc}

\begin{methoddesc}[sha]{digest}{}
Return the digest of the strings passed to the \method{update()}
method so far.  This is a 20-byte string which may contain
non-\ASCII{} characters, including null bytes.
\end{methoddesc}

\begin{methoddesc}[sha]{hexdigest}{}
Like \method{digest()} except the digest is returned as a string of
length 40, containing only hexadecimal digits.  This may 
be used to exchange the value safely in email or other non-binary
environments.
\end{methoddesc}

\begin{methoddesc}[sha]{copy}{}
Return a copy (``clone'') of the sha object.  This can be used to
efficiently compute the digests of strings that share a common initial
substring.
\end{methoddesc}

\begin{seealso}
  \seetitle[http://csrc.nist.gov/publications/fips/fips180-2/fips180-2withchangenotice.pdf]
    {Secure Hash Standard}
    {The Secure Hash Algorithm is defined by NIST document FIPS
     PUB 180-2:
     \citetitle[http://csrc.nist.gov/publications/fips/fips180-2/fips180-2withchangenotice.pdf]
        {Secure Hash Standard}, published in August 2002.}

  \seetitle[http://csrc.nist.gov/encryption/tkhash.html]
           {Cryptographic Toolkit (Secure Hashing)}
           {Links from NIST to various information on secure hashing.}
\end{seealso}

