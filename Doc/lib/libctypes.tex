\ifx\locallinewidth\undefined\newlength{\locallinewidth}\fi
\setlength{\locallinewidth}{\linewidth}
\section{\module{ctypes} --- A foreign function library for Python.}
\declaremodule{standard}{ctypes}
\moduleauthor{Thomas Heller}{theller@python.net}
\modulesynopsis{A foreign function library for Python.}
\versionadded{2.5}

\code{ctypes} is a foreign function library for Python.


\subsection{ctypes tutorial\label{ctypes-ctypes-tutorial}}

This tutorial describes version 0.9.9 of \code{ctypes}.

Note: The code samples in this tutorial uses \code{doctest} to make sure
that they actually work.  Since some code samples behave differently
under Linux, Windows, or Mac OS X, they contain doctest directives in
comments.

Note: Quite some code samples references the ctypes \class{c{\_}int} type.
This type is an alias to the \class{c{\_}long} type on 32-bit systems.  So,
you should not be confused if \class{c{\_}long} is printed if you would
expect \class{c{\_}int} - they are actually the same type.


\subsubsection{Loading dynamic link libraries\label{ctypes-loading-dynamic-link-libraries}}

\code{ctypes} exports the \var{cdll}, and on Windows also \var{windll} and
\var{oledll} objects to load dynamic link libraries.

You load libraries by accessing them as attributes of these objects.
\var{cdll} loads libraries which export functions using the standard
\code{cdecl} calling convention, while \var{windll} libraries call
functions using the \code{stdcall} calling convention. \var{oledll} also
uses the \code{stdcall} calling convention, and assumes the functions
return a Windows \class{HRESULT} error code. The error code is used to
automatically raise \class{WindowsError} Python exceptions when the
function call fails.

Here are some examples for Windows, note that \code{msvcrt} is the MS
standard C library containing most standard C functions, and uses the
cdecl calling convention:
\begin{verbatim}
>>> from ctypes import *
>>> print windll.kernel32 # doctest: +WINDOWS
<WinDLL 'kernel32', handle ... at ...>
>>> print cdll.msvcrt # doctest: +WINDOWS
<CDLL 'msvcrt', handle ... at ...>
>>> libc = cdll.msvcrt # doctest: +WINDOWS
>>>
\end{verbatim}

Windows appends the usual '.dll' file suffix automatically.

On Linux, it is required to specify the filename \emph{including} the
extension to load a library, so attribute access does not work.
Either the \method{LoadLibrary} method of the dll loaders should be used,
or you should load the library by creating an instance of CDLL by
calling the constructor:
\begin{verbatim}
>>> cdll.LoadLibrary("libc.so.6") # doctest: +LINUX
<CDLL 'libc.so.6', handle ... at ...>
>>> libc = CDLL("libc.so.6")     # doctest: +LINUX
>>> libc                         # doctest: +LINUX
<CDLL 'libc.so.6', handle ... at ...>
>>>
\end{verbatim}
% XXX Add section for Mac OS X. 


\subsubsection{Accessing functions from loaded dlls\label{ctypes-accessing-functions-from-loaded-dlls}}

Functions are accessed as attributes of dll objects:
\begin{verbatim}
>>> from ctypes import *
>>> libc.printf
<_FuncPtr object at 0x...>
>>> print windll.kernel32.GetModuleHandleA # doctest: +WINDOWS
<_FuncPtr object at 0x...>
>>> print windll.kernel32.MyOwnFunction # doctest: +WINDOWS
Traceback (most recent call last):
  File "<stdin>", line 1, in ?
  File "ctypes.py", line 239, in __getattr__
    func = _StdcallFuncPtr(name, self)
AttributeError: function 'MyOwnFunction' not found
>>>
\end{verbatim}

Note that win32 system dlls like \code{kernel32} and \code{user32} often
export ANSI as well as UNICODE versions of a function. The UNICODE
version is exported with an \code{W} appended to the name, while the ANSI
version is exported with an \code{A} appended to the name. The win32
\code{GetModuleHandle} function, which returns a \emph{module handle} for a
given module name, has the following C prototype, and a macro is used
to expose one of them as \code{GetModuleHandle} depending on whether
UNICODE is defined or not:
\begin{verbatim}
/* ANSI version */
HMODULE GetModuleHandleA(LPCSTR lpModuleName);
/* UNICODE version */
HMODULE GetModuleHandleW(LPCWSTR lpModuleName);
\end{verbatim}

\var{windll} does not try to select one of them by magic, you must
access the version you need by specifying \code{GetModuleHandleA} or
\code{GetModuleHandleW} explicitely, and then call it with normal strings
or unicode strings respectively.

Sometimes, dlls export functions with names which aren't valid Python
identifiers, like \code{"??2@YAPAXI@Z"}. In this case you have to use
\code{getattr} to retrieve the function:
\begin{verbatim}
>>> getattr(cdll.msvcrt, "??2@YAPAXI@Z") # doctest: +WINDOWS
<_FuncPtr object at 0x...>
>>>
\end{verbatim}

On Windows, some dlls export functions not by name but by ordinal.
These functions can be accessed by indexing the dll object with the
odinal number:
\begin{verbatim}
>>> cdll.kernel32[1] # doctest: +WINDOWS
<_FuncPtr object at 0x...>
>>> cdll.kernel32[0] # doctest: +WINDOWS
Traceback (most recent call last):
  File "<stdin>", line 1, in ?
  File "ctypes.py", line 310, in __getitem__
    func = _StdcallFuncPtr(name, self)
AttributeError: function ordinal 0 not found
>>>
\end{verbatim}


\subsubsection{Calling functions\label{ctypes-calling-functions}}

You can call these functions like any other Python callable. This
example uses the \code{time()} function, which returns system time in
seconds since the \UNIX{} epoch, and the \code{GetModuleHandleA()} function,
which returns a win32 module handle.

This example calls both functions with a NULL pointer (\code{None} should
be used as the NULL pointer):
\begin{verbatim}
>>> print libc.time(None)
114...
>>> print hex(windll.kernel32.GetModuleHandleA(None)) # doctest: +WINDOWS
0x1d000000
>>>
\end{verbatim}

\code{ctypes} tries to protect you from calling functions with the wrong
number of arguments.  Unfortunately this only works on Windows.  It
does this by examining the stack after the function returns:
\begin{verbatim}
>>> windll.kernel32.GetModuleHandleA() # doctest: +WINDOWS
Traceback (most recent call last):
  File "<stdin>", line 1, in ?
ValueError: Procedure probably called with not enough arguments (4 bytes missing)
>>> windll.kernel32.GetModuleHandleA(0, 0) # doctest: +WINDOWS
Traceback (most recent call last):
  File "<stdin>", line 1, in ?
ValueError: Procedure probably called with too many arguments (4 bytes in excess)
>>>
\end{verbatim}

On Windows, \code{ctypes} uses win32 structured exception handling to
prevent crashes from general protection faults when functions are
called with invalid argument values:
\begin{verbatim}
>>> windll.kernel32.GetModuleHandleA(32) # doctest: +WINDOWS
Traceback (most recent call last):
  File "<stdin>", line 1, in ?
WindowsError: exception: access violation reading 0x00000020
>>>
\end{verbatim}

There are, however, enough ways to crash Python with \code{ctypes}, so
you should be careful anyway.

Python integers, strings and unicode strings are the only objects that
can directly be used as parameters in these function calls.

Before we move on calling functions with other parameter types, we
have to learn more about \code{ctypes} data types.


\subsubsection{Fundamental data types\label{ctypes-fundamental-data-types}}

\code{ctypes} defines a number of primitive C compatible data types :
\begin{quote}
\begin{tableiii}{l|l|l}{textrm}
{
ctypes type
}
{
C type
}
{
Python type
}
\lineiii{
\class{c{\_}char}
}
{
\code{char}
}
{
character
}
\lineiii{
\class{c{\_}byte}
}
{
\code{char}
}
{
integer
}
\lineiii{
\class{c{\_}ubyte}
}
{
\code{unsigned char}
}
{
integer
}
\lineiii{
\class{c{\_}short}
}
{
\code{short}
}
{
integer
}
\lineiii{
\class{c{\_}ushort}
}
{
\code{unsigned short}
}
{
integer
}
\lineiii{
\class{c{\_}int}
}
{
\code{int}
}
{
integer
}
\lineiii{
\class{c{\_}uint}
}
{
\code{unsigned int}
}
{
integer
}
\lineiii{
\class{c{\_}long}
}
{
\code{long}
}
{
integer
}
\lineiii{
\class{c{\_}ulong}
}
{
\code{unsigned long}
}
{
long
}
\lineiii{
\class{c{\_}longlong}
}
{
\code{{\_}{\_}int64} or
\code{long long}
}
{
long
}
\lineiii{
\class{c{\_}ulonglong}
}
{
\code{unsigned {\_}{\_}int64} or
\code{unsigned long long}
}
{
long
}
\lineiii{
\class{c{\_}float}
}
{
\code{float}
}
{
float
}
\lineiii{
\class{c{\_}double}
}
{
\code{double}
}
{
float
}
\lineiii{
\class{c{\_}char{\_}p}
}
{
\code{char *}
(NUL terminated)
}
{
string or
\code{None}
}
\lineiii{
\class{c{\_}wchar{\_}p}
}
{
\code{wchar{\_}t *}
(NUL terminated)
}
{
unicode or
\code{None}
}
\lineiii{
\class{c{\_}void{\_}p}
}
{
\code{void *}
}
{
integer or
\code{None}
}
\end{tableiii}
\end{quote}

All these types can be created by calling them with an optional
initializer of the correct type and value:
\begin{verbatim}
>>> c_int()
c_long(0)
>>> c_char_p("Hello, World")
c_char_p('Hello, World')
>>> c_ushort(-3)
c_ushort(65533)
>>>
\end{verbatim}

Since these types are mutable, their value can also be changed
afterwards:
\begin{verbatim}
>>> i = c_int(42)
>>> print i
c_long(42)
>>> print i.value
42
>>> i.value = -99
>>> print i.value
-99
>>>
\end{verbatim}

Assigning a new value to instances of the pointer types \class{c{\_}char{\_}p},
\class{c{\_}wchar{\_}p}, and \class{c{\_}void{\_}p} changes the \emph{memory location} they
point to, \emph{not the contents} of the memory block (of course not,
because Python strings are immutable):
\begin{verbatim}
>>> s = "Hello, World"
>>> c_s = c_char_p(s)
>>> print c_s
c_char_p('Hello, World')
>>> c_s.value = "Hi, there"
>>> print c_s
c_char_p('Hi, there')
>>> print s                 # first string is unchanged
Hello, World
>>>
\end{verbatim}

You should be careful, however, not to pass them to functions
expecting pointers to mutable memory. If you need mutable memory
blocks, ctypes has a \code{create{\_}string{\_}buffer} function which creates
these in various ways.  The current memory block contents can be
accessed (or changed) with the \code{raw} property, if you want to access
it as NUL terminated string, use the \code{string} property:
\begin{verbatim}
>>> from ctypes import *
>>> p = create_string_buffer(3)      # create a 3 byte buffer, initialized to NUL bytes
>>> print sizeof(p), repr(p.raw)
3 '\x00\x00\x00'
>>> p = create_string_buffer("Hello")      # create a buffer containing a NUL terminated string
>>> print sizeof(p), repr(p.raw)
6 'Hello\x00'
>>> print repr(p.value)
'Hello'
>>> p = create_string_buffer("Hello", 10)  # create a 10 byte buffer
>>> print sizeof(p), repr(p.raw)
10 'Hello\x00\x00\x00\x00\x00'
>>> p.value = "Hi"      
>>> print sizeof(p), repr(p.raw)
10 'Hi\x00lo\x00\x00\x00\x00\x00'
>>>
\end{verbatim}

The \code{create{\_}string{\_}buffer} function replaces the \code{c{\_}buffer}
function (which is still available as an alias), as well as the
\code{c{\_}string} function from earlier ctypes releases.  To create a
mutable memory block containing unicode characters of the C type
\code{wchar{\_}t} use the \code{create{\_}unicode{\_}buffer} function.


\subsubsection{Calling functions, continued\label{ctypes-calling-functions-continued}}

Note that printf prints to the real standard output channel, \emph{not} to
\code{sys.stdout}, so these examples will only work at the console
prompt, not from within \emph{IDLE} or \emph{PythonWin}:
\begin{verbatim}
>>> printf = libc.printf
>>> printf("Hello, %s\n", "World!")
Hello, World!
14
>>> printf("Hello, %S", u"World!")
Hello, World!
13
>>> printf("%d bottles of beer\n", 42)
42 bottles of beer
19
>>> printf("%f bottles of beer\n", 42.5)
Traceback (most recent call last):
  File "<stdin>", line 1, in ?
ArgumentError: argument 2: exceptions.TypeError: Don't know how to convert parameter 2
>>>
\end{verbatim}

As has been mentioned before, all Python types except integers,
strings, and unicode strings have to be wrapped in their corresponding
\code{ctypes} type, so that they can be converted to the required C data
type:
\begin{verbatim}
>>> printf("An int %d, a double %f\n", 1234, c_double(3.14))
Integer 1234, double 3.1400001049
31
>>>
\end{verbatim}


\subsubsection{Calling functions with your own custom data types\label{ctypes-calling-functions-with-own-custom-data-types}}

You can also customize \code{ctypes} argument conversion to allow
instances of your own classes be used as function arguments.
\code{ctypes} looks for an \member{{\_}as{\_}parameter{\_}} attribute and uses this as
the function argument. Of course, it must be one of integer, string,
or unicode:
\begin{verbatim}
>>> class Bottles(object):
...     def __init__(self, number):
...         self._as_parameter_ = number
...
>>> bottles = Bottles(42)
>>> printf("%d bottles of beer\n", bottles)
42 bottles of beer
19
>>>
\end{verbatim}

If you don't want to store the instance's data in the
\member{{\_}as{\_}parameter{\_}} instance variable, you could define a \code{property}
which makes the data avaiblable.


\subsubsection{Specifying the required argument types (function prototypes)\label{ctypes-specifying-required-argument-types}}

It is possible to specify the required argument types of functions
exported from DLLs by setting the \member{argtypes} attribute.

\member{argtypes} must be a sequence of C data types (the \code{printf}
function is probably not a good example here, because it takes a
variable number and different types of parameters depending on the
format string, on the other hand this is quite handy to experiment
with this feature):
\begin{verbatim}
>>> printf.argtypes = [c_char_p, c_char_p, c_int, c_double]
>>> printf("String '%s', Int %d, Double %f\n", "Hi", 10, 2.2)
String 'Hi', Int 10, Double 2.200000
37
>>>
\end{verbatim}

Specifying a format protects against incompatible argument types (just
as a prototype for a C function), and tries to convert the arguments
to valid types:
\begin{verbatim}
>>> printf("%d %d %d", 1, 2, 3)
Traceback (most recent call last):
  File "<stdin>", line 1, in ?
ArgumentError: argument 2: exceptions.TypeError: wrong type
>>> printf("%s %d %f", "X", 2, 3)
X 2 3.00000012
12
>>>
\end{verbatim}

If you have defined your own classes which you pass to function calls,
you have to implement a \method{from{\_}param} class method for them to be
able to use them in the \member{argtypes} sequence. The \method{from{\_}param}
class method receives the Python object passed to the function call,
it should do a typecheck or whatever is needed to make sure this
object is acceptable, and then return the object itself, it's
\member{{\_}as{\_}parameter{\_}} attribute, or whatever you want to pass as the C
function argument in this case. Again, the result should be an
integer, string, unicode, a \code{ctypes} instance, or something having
the \member{{\_}as{\_}parameter{\_}} attribute.


\subsubsection{Return types\label{ctypes-return-types}}

By default functions are assumed to return integers.  Other return
types can be specified by setting the \member{restype} attribute of the
function object.

Here is a more advanced example, it uses the strchr function, which
expects a string pointer and a char, and returns a pointer to a
string:
\begin{verbatim}
>>> strchr = libc.strchr
>>> strchr("abcdef", ord("d")) # doctest: +SKIP
8059983
>>> strchr.restype = c_char_p # c_char_p is a pointer to a string
>>> strchr("abcdef", ord("d"))
'def'
>>> print strchr("abcdef", ord("x"))
None
>>>
\end{verbatim}

If you want to avoid the \code{ord("x")} calls above, you can set the
\member{argtypes} attribute, and the second argument will be converted from
a single character Python string into a C char:
\begin{verbatim}
>>> strchr.restype = c_char_p
>>> strchr.argtypes = [c_char_p, c_char]
>>> strchr("abcdef", "d")
'def'
>>> strchr("abcdef", "def")
Traceback (most recent call last):
  File "<stdin>", line 1, in ?
ArgumentError: argument 2: exceptions.TypeError: one character string expected
>>> print strchr("abcdef", "x")
None
>>> strchr("abcdef", "d")
'def'
>>>
\end{verbatim}

XXX Mention the \member{errcheck} protocol...

You can also use a callable Python object (a function or a class for
example) as the \member{restype} attribute.  It will be called with the
\code{integer} the C function returns, and the result of this call will
be used as the result of your function call. This is useful to check
for error return values and automatically raise an exception:
\begin{verbatim}
>>> GetModuleHandle = windll.kernel32.GetModuleHandleA # doctest: +WINDOWS
>>> def ValidHandle(value):
...     if value == 0:
...         raise WinError()
...     return value
...
>>>
>>> GetModuleHandle.restype = ValidHandle # doctest: +WINDOWS
>>> GetModuleHandle(None) # doctest: +WINDOWS
486539264
>>> GetModuleHandle("something silly") # doctest: +WINDOWS
Traceback (most recent call last):
  File "<stdin>", line 1, in ?
  File "<stdin>", line 3, in ValidHandle
WindowsError: [Errno 126] The specified module could not be found.
>>>
\end{verbatim}

\code{WinError} is a function which will call Windows \code{FormatMessage()}
api to get the string representation of an error code, and \emph{returns}
an exception.  \code{WinError} takes an optional error code parameter, if
no one is used, it calls \function{GetLastError()} to retrieve it.


\subsubsection{Passing pointers (or: passing parameters by reference)\label{ctypes-passing-pointers}}

Sometimes a C api function expects a \emph{pointer} to a data type as
parameter, probably to write into the corresponding location, or if
the data is too large to be passed by value. This is also known as
\emph{passing parameters by reference}.

\code{ctypes} exports the \function{byref} function which is used to pass
parameters by reference.  The same effect can be achieved with the
\code{pointer} function, although \code{pointer} does a lot more work since
it constructs a real pointer object, so it is faster to use \function{byref}
if you don't need the pointer object in Python itself:
\begin{verbatim}
>>> i = c_int()
>>> f = c_float()
>>> s = create_string_buffer('\000' * 32)
>>> print i.value, f.value, repr(s.value)
0 0.0 ''
>>> libc.sscanf("1 3.14 Hello", "%d %f %s",
...             byref(i), byref(f), s)
3
>>> print i.value, f.value, repr(s.value)
1 3.1400001049 'Hello'
>>>
\end{verbatim}


\subsubsection{Structures and unions\label{ctypes-structures-unions}}

Structures and unions must derive from the \class{Structure} and \class{Union}
base classes which are defined in the \code{ctypes} module. Each subclass
must define a \member{{\_}fields{\_}} attribute.  \member{{\_}fields{\_}} must be a list of
\emph{2-tuples}, containing a \emph{field name} and a \emph{field type}.

The field type must be a \code{ctypes} type like \class{c{\_}int}, or any other
derived \code{ctypes} type: structure, union, array, pointer.

Here is a simple example of a POINT structure, which contains two
integers named \code{x} and \code{y}, and also shows how to initialize a
structure in the constructor:
\begin{verbatim}
>>> from ctypes import *
>>> class POINT(Structure):
...     _fields_ = [("x", c_int),
...                 ("y", c_int)]
...
>>> point = POINT(10, 20)
>>> print point.x, point.y
10 20
>>> point = POINT(y=5)
>>> print point.x, point.y
0 5
>>> POINT(1, 2, 3)
Traceback (most recent call last):
  File "<stdin>", line 1, in ?
ValueError: too many initializers
>>>
\end{verbatim}

You can, however, build much more complicated structures. Structures
can itself contain other structures by using a structure as a field
type.

Here is a RECT structure which contains two POINTs named \code{upperleft}
and \code{lowerright}
\begin{verbatim}
>>> class RECT(Structure):
...     _fields_ = [("upperleft", POINT),
...                 ("lowerright", POINT)]
...
>>> rc = RECT(point)
>>> print rc.upperleft.x, rc.upperleft.y
0 5
>>> print rc.lowerright.x, rc.lowerright.y
0 0
>>>
\end{verbatim}

Nested structures can also be initialized in the constructor in
several ways:
\begin{verbatim}
>>> r = RECT(POINT(1, 2), POINT(3, 4))
>>> r = RECT((1, 2), (3, 4))
\end{verbatim}

Fields descriptors can be retrieved from the \emph{class}, they are useful
for debugging because they can provide useful information:
\begin{verbatim}
>>> print POINT.x
<Field type=c_long, ofs=0, size=4>
>>> print POINT.y
<Field type=c_long, ofs=4, size=4>
>>>
\end{verbatim}


\subsubsection{Structure/union alignment and byte order\label{ctypes-structureunion-alignment-byte-order}}

By default, Structure and Union fields are aligned in the same way the
C compiler does it. It is possible to override this behaviour be
specifying a \member{{\_}pack{\_}} class attribute in the subclass
definition. This must be set to a positive integer and specifies the
maximum alignment for the fields. This is what \code{{\#}pragma pack(n)}
also does in MSVC.

\code{ctypes} uses the native byte order for Structures and Unions.  To
build structures with non-native byte order, you can use one of the
BigEndianStructure, LittleEndianStructure, BigEndianUnion, and
LittleEndianUnion base classes.  These classes cannot contain pointer
fields.


\subsubsection{Bit fields in structures and unions\label{ctypes-bit-fields-in-structures-unions}}

It is possible to create structures and unions containing bit fields.
Bit fields are only possible for integer fields, the bit width is
specified as the third item in the \member{{\_}fields{\_}} tuples:
\begin{verbatim}
>>> class Int(Structure):
...     _fields_ = [("first_16", c_int, 16),
...                 ("second_16", c_int, 16)]
...
>>> print Int.first_16
<Field type=c_long, ofs=0:0, bits=16>
>>> print Int.second_16
<Field type=c_long, ofs=0:16, bits=16>
>>>
\end{verbatim}


\subsubsection{Arrays\label{ctypes-arrays}}

Arrays are sequences, containing a fixed number of instances of the
same type.

The recommended way to create array types is by multiplying a data
type with a positive integer:
\begin{verbatim}
TenPointsArrayType = POINT * 10
\end{verbatim}

Here is an example of an somewhat artifical data type, a structure
containing 4 POINTs among other stuff:
\begin{verbatim}
>>> from ctypes import *
>>> class POINT(Structure):
...    _fields_ = ("x", c_int), ("y", c_int)
...
>>> class MyStruct(Structure):
...    _fields_ = [("a", c_int),
...                ("b", c_float),
...                ("point_array", POINT * 4)]
>>>
>>> print len(MyStruct().point_array)
4
>>>
\end{verbatim}

Instances are created in the usual way, by calling the class:
\begin{verbatim}
arr = TenPointsArrayType()
for pt in arr:
    print pt.x, pt.y
\end{verbatim}

The above code print a series of \code{0 0} lines, because the array
contents is initialized to zeros.

Initializers of the correct type can also be specified:
\begin{verbatim}
>>> from ctypes import *
>>> TenIntegers = c_int * 10
>>> ii = TenIntegers(1, 2, 3, 4, 5, 6, 7, 8, 9, 10)
>>> print ii
<c_long_Array_10 object at 0x...>
>>> for i in ii: print i,
...
1 2 3 4 5 6 7 8 9 10
>>>
\end{verbatim}


\subsubsection{Pointers\label{ctypes-pointers}}

XXX Rewrite this section.  Normally one only uses indexing, not the .contents
attribute!
List some recipes with pointers.  bool(ptr),  POINTER(tp)(), ...?

Pointer instances are created by calling the \code{pointer} function on a
\code{ctypes} type:
\begin{verbatim}
>>> from ctypes import *
>>> i = c_int(42)
>>> pi = pointer(i)
>>>
\end{verbatim}

Pointer instances have a \code{contents} attribute which returns the
object to which the pointer points, the \code{i} object above:
\begin{verbatim}
>>> pi.contents
c_long(42)
>>>
\end{verbatim}

Note that \code{ctypes} does not have OOR (original object return), it
constructs a new, equivalent object each time you retrieve an
attribute:
\begin{verbatim}
>>> pi.contents is i
False
>>> pi.contents is pi.contents
False
>>>
\end{verbatim}

Assigning another \class{c{\_}int} instance to the pointer's contents
attribute would cause the pointer to point to the memory location
where this is stored:
\begin{verbatim}
>>> pi.contents = c_int(99)
>>> pi.contents
c_long(99)
>>>
\end{verbatim}

Pointer instances can also be indexed with integers:
\begin{verbatim}
>>> pi[0]
99
>>>
\end{verbatim}

Assigning to an integer index changes the pointed to value:
\begin{verbatim}
>>> print i
c_long(99)
>>> pi[0] = 22
>>> print i
c_long(22)
>>>
\end{verbatim}

It is also possible to use indexes different from 0, but you must know
what you're doing, just as in C: You can access or change arbitrary
memory locations. Generally you only use this feature if you receive a
pointer from a C function, and you \emph{know} that the pointer actually
points to an array instead of a single item.


\subsubsection{Pointer classes/types\label{ctypes-pointer-classestypes}}

Behind the scenes, the \code{pointer} function does more than simply
create pointer instances, it has to create pointer \emph{types} first.
This is done with the \code{POINTER} function, which accepts any
\code{ctypes} type, and returns a new type:
\begin{verbatim}
>>> PI = POINTER(c_int)
>>> PI
<class 'ctypes.LP_c_long'>
>>> PI(42)
Traceback (most recent call last):
  File "<stdin>", line 1, in ?
TypeError: expected c_long instead of int
>>> PI(c_int(42))
<ctypes.LP_c_long object at 0x...>
>>>
\end{verbatim}


\subsubsection{Type conversions\label{ctypes-type-conversions}}

Usually, ctypes does strict type checking.  This means, if you have
\code{POINTER(c{\_}int)} in the \member{argtypes} list of a function or in the
\member{{\_}fields{\_}} of a structure definition, only instances of exactly the
same type are accepted.  There are some exceptions to this rule, where
ctypes accepts other objects.  For example, you can pass compatible
array instances instead of pointer types.  So, for \code{POINTER(c{\_}int)},
ctypes accepts an array of c{\_}int values:
\begin{verbatim}
>>> class Bar(Structure):
...     _fields_ = [("count", c_int), ("values", POINTER(c_int))]
...
>>> bar = Bar()
>>> print bar._objects
None
>>> bar.values = (c_int * 3)(1, 2, 3)
>>> print bar._objects
{'1': ({}, <ctypes._endian.c_long_Array_3 object at ...>)}
>>> bar.count = 3
>>> for i in range(bar.count):
...     print bar.values[i]
...
1
2
3
>>>
\end{verbatim}

To set a POINTER type field to \code{NULL}, you can assign \code{None}:
\begin{verbatim}
>>> bar.values = None
>>>
\end{verbatim}

XXX list other conversions...

Sometimes you have instances of incompatible types.  In \code{C}, you can
cast one type into another type.  \code{ctypes} provides a \code{cast}
function which can be used in the same way.  The Bar structure defined
above accepts \code{POINTER(c{\_}int)} pointers or \class{c{\_}int} arrays for its
\code{values} field, but not instances of other types:
\begin{verbatim}
>>> bar.values = (c_byte * 4)()
Traceback (most recent call last):
  File "<stdin>", line 1, in ?
TypeError: incompatible types, c_byte_Array_4 instance instead of LP_c_long instance
>>>
\end{verbatim}

For these cases, the \code{cast} function is handy.

The \code{cast} function can be used to cast a ctypes instance into a
pointer to a different ctypes data type.  \code{cast} takes two
parameters, a ctypes object that is or can be converted to a pointer
of some kind, and a ctypes pointer type.  It returns an instance of
the second argument, which references the same memory block as the
first argument:
\begin{verbatim}
>>> a = (c_byte * 4)()
>>> cast(a, POINTER(c_int))
<ctypes.LP_c_long object at ...>
>>>
\end{verbatim}

So, \code{cast} can be used to assign to the \code{values} field of \code{Bar}
the structure:
\begin{verbatim}
>>> bar = Bar()
>>> bar.values = cast((c_byte * 4)(), POINTER(c_int))
>>> print bar.values[0]
0
>>>
\end{verbatim}


\subsubsection{Incomplete Types\label{ctypes-incomplete-types}}

\emph{Incomplete Types} are structures, unions or arrays whose members are
not yet specified. In C, they are specified by forward declarations, which
are defined later:
\begin{verbatim}
struct cell; /* forward declaration */

struct {
    char *name;
    struct cell *next;
} cell;
\end{verbatim}

The straightforward translation into ctypes code would be this, but it
does not work:
\begin{verbatim}
>>> class cell(Structure):
...     _fields_ = [("name", c_char_p),
...                 ("next", POINTER(cell))]
...
Traceback (most recent call last):
  File "<stdin>", line 1, in ?
  File "<stdin>", line 2, in cell
NameError: name 'cell' is not defined
>>>
\end{verbatim}

because the new \code{class cell} is not available in the class statement
itself.  In \code{ctypes}, we can define the \code{cell} class and set the
\member{{\_}fields{\_}} attribute later, after the class statement:
\begin{verbatim}
>>> from ctypes import *
>>> class cell(Structure):
...     pass
...
>>> cell._fields_ = [("name", c_char_p),
...                  ("next", POINTER(cell))]
>>>
\end{verbatim}

Lets try it. We create two instances of \code{cell}, and let them point
to each other, and finally follow the pointer chain a few times:
\begin{verbatim}
>>> c1 = cell()
>>> c1.name = "foo"
>>> c2 = cell()
>>> c2.name = "bar"
>>> c1.next = pointer(c2)
>>> c2.next = pointer(c1)
>>> p = c1
>>> for i in range(8):
...     print p.name,
...     p = p.next[0]
...
foo bar foo bar foo bar foo bar
>>>    
\end{verbatim}


\subsubsection{Callback functions\label{ctypes-callback-functions}}

\code{ctypes} allows to create C callable function pointers from Python
callables. These are sometimes called \emph{callback functions}.

First, you must create a class for the callback function, the class
knows the calling convention, the return type, and the number and
types of arguments this function will receive.

The CFUNCTYPE factory function creates types for callback functions
using the normal cdecl calling convention, and, on Windows, the
WINFUNCTYPE factory function creates types for callback functions
using the stdcall calling convention.

Both of these factory functions are called with the result type as
first argument, and the callback functions expected argument types as
the remaining arguments.

I will present an example here which uses the standard C library's
\function{qsort} function, this is used to sort items with the help of a
callback function. \function{qsort} will be used to sort an array of
integers:
\begin{verbatim}
>>> IntArray5 = c_int * 5
>>> ia = IntArray5(5, 1, 7, 33, 99)
>>> qsort = libc.qsort
>>> qsort.restype = None
>>>
\end{verbatim}

\function{qsort} must be called with a pointer to the data to sort, the
number of items in the data array, the size of one item, and a pointer
to the comparison function, the callback. The callback will then be
called with two pointers to items, and it must return a negative
integer if the first item is smaller than the second, a zero if they
are equal, and a positive integer else.

So our callback function receives pointers to integers, and must
return an integer. First we create the \code{type} for the callback
function:
\begin{verbatim}
>>> CMPFUNC = CFUNCTYPE(c_int, POINTER(c_int), POINTER(c_int))
>>>
\end{verbatim}

For the first implementation of the callback function, we simply print
the arguments we get, and return 0 (incremental development ;-):
\begin{verbatim}
>>> def py_cmp_func(a, b):
...     print "py_cmp_func", a, b
...     return 0
...
>>>
\end{verbatim}

Create the C callable callback:
\begin{verbatim}
>>> cmp_func = CMPFUNC(py_cmp_func)
>>>
\end{verbatim}

And we're ready to go:
\begin{verbatim}
>>> qsort(ia, len(ia), sizeof(c_int), cmp_func) # doctest: +WINDOWS
py_cmp_func <ctypes.LP_c_long object at 0x00...> <ctypes.LP_c_long object at 0x00...>
py_cmp_func <ctypes.LP_c_long object at 0x00...> <ctypes.LP_c_long object at 0x00...>
py_cmp_func <ctypes.LP_c_long object at 0x00...> <ctypes.LP_c_long object at 0x00...>
py_cmp_func <ctypes.LP_c_long object at 0x00...> <ctypes.LP_c_long object at 0x00...>
py_cmp_func <ctypes.LP_c_long object at 0x00...> <ctypes.LP_c_long object at 0x00...>
py_cmp_func <ctypes.LP_c_long object at 0x00...> <ctypes.LP_c_long object at 0x00...>
py_cmp_func <ctypes.LP_c_long object at 0x00...> <ctypes.LP_c_long object at 0x00...>
py_cmp_func <ctypes.LP_c_long object at 0x00...> <ctypes.LP_c_long object at 0x00...>
py_cmp_func <ctypes.LP_c_long object at 0x00...> <ctypes.LP_c_long object at 0x00...>
py_cmp_func <ctypes.LP_c_long object at 0x00...> <ctypes.LP_c_long object at 0x00...>
>>>
\end{verbatim}

We know how to access the contents of a pointer, so lets redefine our callback:
\begin{verbatim}
>>> def py_cmp_func(a, b):
...     print "py_cmp_func", a[0], b[0]
...     return 0
...
>>> cmp_func = CMPFUNC(py_cmp_func)
>>>
\end{verbatim}

Here is what we get on Windows:
\begin{verbatim}
>>> qsort(ia, len(ia), sizeof(c_int), cmp_func) # doctest: +WINDOWS
py_cmp_func 7 1
py_cmp_func 33 1
py_cmp_func 99 1
py_cmp_func 5 1
py_cmp_func 7 5
py_cmp_func 33 5
py_cmp_func 99 5
py_cmp_func 7 99
py_cmp_func 33 99
py_cmp_func 7 33
>>>
\end{verbatim}

It is funny to see that on linux the sort function seems to work much
more efficient, it is doing less comparisons:
\begin{verbatim}
>>> qsort(ia, len(ia), sizeof(c_int), cmp_func) # doctest: +LINUX
py_cmp_func 5 1
py_cmp_func 33 99
py_cmp_func 7 33
py_cmp_func 5 7
py_cmp_func 1 7
>>>
\end{verbatim}

Ah, we're nearly done! The last step is to actually compare the two
items and return a useful result:
\begin{verbatim}
>>> def py_cmp_func(a, b):
...     print "py_cmp_func", a[0], b[0]
...     return a[0] - b[0]
...
>>>
\end{verbatim}

Final run on Windows:
\begin{verbatim}
>>> qsort(ia, len(ia), sizeof(c_int), CMPFUNC(py_cmp_func)) # doctest: +WINDOWS
py_cmp_func 33 7
py_cmp_func 99 33
py_cmp_func 5 99
py_cmp_func 1 99
py_cmp_func 33 7
py_cmp_func 1 33
py_cmp_func 5 33
py_cmp_func 5 7
py_cmp_func 1 7
py_cmp_func 5 1
>>>
\end{verbatim}

and on Linux:
\begin{verbatim}
>>> qsort(ia, len(ia), sizeof(c_int), CMPFUNC(py_cmp_func)) # doctest: +LINUX
py_cmp_func 5 1
py_cmp_func 33 99
py_cmp_func 7 33
py_cmp_func 1 7
py_cmp_func 5 7
>>>
\end{verbatim}

So, our array sorted now:
\begin{verbatim}
>>> for i in ia: print i,
...
1 5 7 33 99
>>>
\end{verbatim}

\textbf{Important note for callback functions:}

Make sure you keep references to CFUNCTYPE objects as long as they are
used from C code. ctypes doesn't, and if you don't, they may be
garbage collected, crashing your program when a callback is made.


\subsubsection{Accessing values exported from dlls\label{ctypes-accessing-values-exported-from-dlls}}

Sometimes, a dll not only exports functions, it also exports
values. An example in the Python library itself is the
\code{Py{\_}OptimizeFlag}, an integer set to 0, 1, or 2, depending on the
\programopt{-O} or \programopt{-OO} flag given on startup.

\code{ctypes} can access values like this with the \method{in{\_}dll} class
methods of the type.  \var{pythonapi} �s a predefined symbol giving
access to the Python C api:
\begin{verbatim}
>>> opt_flag = c_int.in_dll(pythonapi, "Py_OptimizeFlag")
>>> print opt_flag
c_long(0)
>>>
\end{verbatim}

If the interpreter would have been started with \programopt{-O}, the sample
would have printed \code{c{\_}long(1)}, or \code{c{\_}long(2)} if \programopt{-OO} would have
been specified.

An extended example which also demonstrates the use of pointers
accesses the \code{PyImport{\_}FrozenModules} pointer exported by Python.

Quoting the Python docs: \emph{This pointer is initialized to point to an
array of ``struct {\_}frozen`` records, terminated by one whose members
are all NULL or zero. When a frozen module is imported, it is searched
in this table. Third-party code could play tricks with this to provide
a dynamically created collection of frozen modules.}

So manipulating this pointer could even prove useful. To restrict the
example size, we show only how this table can be read with
\code{ctypes}:
\begin{verbatim}
>>> from ctypes import *
>>>
>>> class struct_frozen(Structure):
...     _fields_ = [("name", c_char_p),
...                 ("code", POINTER(c_ubyte)),
...                 ("size", c_int)]
...
>>>
\end{verbatim}

We have defined the \code{struct {\_}frozen} data type, so we can get the
pointer to the table:
\begin{verbatim}
>>> FrozenTable = POINTER(struct_frozen)
>>> table = FrozenTable.in_dll(pythonapi, "PyImport_FrozenModules")
>>>
\end{verbatim}

Since \code{table} is a \code{pointer} to the array of \code{struct{\_}frozen}
records, we can iterate over it, but we just have to make sure that
our loop terminates, because pointers have no size. Sooner or later it
would probably crash with an access violation or whatever, so it's
better to break out of the loop when we hit the NULL entry:
\begin{verbatim}
>>> for item in table:
...    print item.name, item.size
...    if item.name is None:
...        break
...
__hello__ 104
__phello__ -104
__phello__.spam 104
None 0
>>>
\end{verbatim}

The fact that standard Python has a frozen module and a frozen package
(indicated by the negative size member) is not wellknown, it is only
used for testing. Try it out with \code{import {\_}{\_}hello{\_}{\_}} for example.

XXX Describe how to access the \var{code} member fields, which contain
the byte code for the modules.


\subsubsection{Surprises\label{ctypes-surprises}}

There are some edges in \code{ctypes} where you may be expect something
else than what actually happens.

Consider the following example:
\begin{verbatim}
>>> from ctypes import *
>>> class POINT(Structure):
...     _fields_ = ("x", c_int), ("y", c_int)
...
>>> class RECT(Structure):
...     _fields_ = ("a", POINT), ("b", POINT)
...
>>> p1 = POINT(1, 2)
>>> p2 = POINT(3, 4)
>>> rc = RECT(p1, p2)
>>> print rc.a.x, rc.a.y, rc.b.x, rc.b.y
1 2 3 4
>>> # now swap the two points
>>> rc.a, rc.b = rc.b, rc.a
>>> print rc.a.x, rc.a.y, rc.b.x, rc.b.y
3 4 3 4
>>>
\end{verbatim}

Hm. We certainly expected the last statement to print \code{3 4 1 2}.
What happended? Here are the steps of the \code{rc.a, rc.b = rc.b, rc.a}
line above:
\begin{verbatim}
>>> temp0, temp1 = rc.b, rc.a
>>> rc.a = temp0
>>> rc.b = temp1
>>>
\end{verbatim}

Note that \code{temp0} and \code{temp1} are objects still using the internal
buffer of the \code{rc} object above. So executing \code{rc.a = temp0}
copies the buffer contents of \code{temp0} into \code{rc} 's buffer.  This,
in turn, changes the contents of \code{temp1}. So, the last assignment
\code{rc.b = temp1}, doesn't have the expected effect.

Keep in mind that retrieving subobjects from Structure, Unions, and
Arrays doesn't \emph{copy} the subobject, instead it retrieves a wrapper
object accessing the root-object's underlying buffer.

Another example that may behave different from what one would expect is this:
\begin{verbatim}
>>> s = c_char_p()
>>> s.value = "abc def ghi"
>>> s.value
'abc def ghi'
>>> s.value is s.value
False
>>>
\end{verbatim}

Why is it printing \code{False}?  ctypes instances are objects containing
a memory block plus some descriptors accessing the contents of the
memory.  Storing a Python object in the memory block does not store
the object itself, instead the \code{contents} of the object is stored.
Accessing the contents again constructs a new Python each time!


\subsubsection{Variable-sized data types\label{ctypes-variable-sized-data-types}}

\code{ctypes} provides some support for variable-sized arrays and
structures (this was added in version 0.9.9.7).

The \code{resize} function can be used to resize the memory buffer of an
existing ctypes object.  The function takes the object as first
argument, and the requested size in bytes as the second argument.  The
memory block cannot be made smaller than the natural memory block
specified by the objects type, a \code{ValueError} is raised if this is
tried:
\begin{verbatim}
>>> short_array = (c_short * 4)()
>>> print sizeof(short_array)
8
>>> resize(short_array, 4)
Traceback (most recent call last):
    ...
ValueError: minimum size is 8
>>> resize(short_array, 32)
>>> sizeof(short_array)
32
>>> sizeof(type(short_array))
8
>>>
\end{verbatim}

This is nice and fine, but how would one access the additional
elements contained in this array?  Since the type still only knows
about 4 elements, we get errors accessing other elements:
\begin{verbatim}
>>> short_array[:]
[0, 0, 0, 0]
>>> short_array[7]
Traceback (most recent call last):
    ...
IndexError: invalid index
>>>
\end{verbatim}

The solution is to use 1-element arrays; as a special case ctypes does
no bounds checking on them:
\begin{verbatim}
>>> short_array = (c_short * 1)()
>>> print sizeof(short_array)
2
>>> resize(short_array, 32)
>>> sizeof(short_array)
32
>>> sizeof(type(short_array))
2
>>> short_array[0:8]
[0, 0, 0, 0, 0, 0, 0, 0]
>>> short_array[7] = 42
>>> short_array[0:8]
[0, 0, 0, 0, 0, 0, 0, 42]
>>>
\end{verbatim}

Using 1-element arrays as variable sized fields in structures works as
well, but they should be used as the last field in the structure
definition.  This example shows a definition from the Windows header
files:
\begin{verbatim}
class SP_DEVICE_INTERFACE_DETAIL_DATA(Structure):
    _fields_ = [("cbSize", c_int),
                ("DevicePath", c_char * 1)]
\end{verbatim}

Another way to use variable-sized data types with \code{ctypes} is to use
the dynamic nature of Python, and (re-)define the data type after the
required size is already known, on a case by case basis.


\subsubsection{Bugs, ToDo and non-implemented things\label{ctypes-bugs-todo-non-implemented-things}}

Enumeration types are not implemented. You can do it easily yourself,
using \class{c{\_}int} as the base class.

\code{long double} is not implemented.
% Local Variables:
% compile-command: "make.bat"
% End: 


\subsection{ctypes reference\label{ctypes-ctypes-reference}}


\subsubsection{Finding shared libraries\label{ctypes-finding-shared-libraries}}

When programming in a compiled language, shared libraries are accessed
when compiling/linking a program, and when the program is run.

The purpose of the \code{find{\_}library} function is to locate a library in
a way similar to what the compiler does (on platforms with several
versions of a shared library the most recent should be loaded), while
the ctypes library loaders act like when a program is run, and call
the runtime loader directly.

The \code{ctypes.util} module provides a function which can help to
determine the library to load.

\begin{datadescni}{find_library(name)}
Try to find a library and return a pathname.  \var{name} is the
library name without any prefix like \var{lib}, suffix like \code{.so},
\code{.dylib} or version number (this is the form used for the posix
linker option \programopt{-l}).  If no library can be found, returns
\code{None}.
\end{datadescni}

The exact functionality is system dependend.

On Linux, \code{find{\_}library} tries to run external programs
(/sbin/ldconfig, gcc, and objdump) to find the library file.  It
returns the filename of the library file.  Here are sone examples:
\begin{verbatim}
>>> from ctypes.util import find_library
>>> find_library("m")
'libm.so.6'
>>> find_library("c")
'libc.so.6'
>>> find_library("bz2")
'libbz2.so.1.0'
>>>
\end{verbatim}

On OS X, \code{find{\_}library} tries several predefined naming schemes and
paths to locate the library, and returns a full pathname if successfull:
\begin{verbatim}
>>> from ctypes.util import find_library
>>> find_library("c")
'/usr/lib/libc.dylib'
>>> find_library("m")
'/usr/lib/libm.dylib'
>>> find_library("bz2")
'/usr/lib/libbz2.dylib'
>>> find_library("AGL")
'/System/Library/Frameworks/AGL.framework/AGL'
>>>
\end{verbatim}

On Windows, \code{find{\_}library} searches along the system search path,
and returns the full pathname, but since there is no predefined naming
scheme a call like \code{find{\_}library("c")} will fail and return
\code{None}.

If wrapping a shared library with \code{ctypes}, it \emph{may} be better to
determine the shared library name at development type, and hardcode
that into the wrapper module instead of using \code{find{\_}library} to
locate the library at runtime.


\subsubsection{Loading shared libraries\label{ctypes-loading-shared-libraries}}

There are several ways to loaded shared libraries into the Python
process.  One way is to instantiate one of the following classes:

\begin{classdesc}{CDLL}{name, mode=RTLD_LOCAL, handle=None}
Instances of this class represent loaded shared libraries.
Functions in these libraries use the standard C calling
convention, and are assumed to return \code{int}.
\end{classdesc}

\begin{classdesc}{OleDLL}{name, mode=RTLD_LOCAL, handle=None}
Windows only: Instances of this class represent loaded shared
libraries, functions in these libraries use the \code{stdcall}
calling convention, and are assumed to return the windows specific
\class{HRESULT} code.  \class{HRESULT} values contain information
specifying whether the function call failed or succeeded, together
with additional error code.  If the return value signals a
failure, an \class{WindowsError} is automatically raised.
\end{classdesc}

\begin{classdesc}{WinDLL}{name, mode=RTLD_LOCAL, handle=None}
Windows only: Instances of this class represent loaded shared
libraries, functions in these libraries use the \code{stdcall}
calling convention, and are assumed to return \code{int} by default.

On Windows CE only the standard calling convention is used, for
convenience the \class{WinDLL} and \class{OleDLL} use the standard calling
convention on this platform.
\end{classdesc}

The Python GIL is released before calling any function exported by
these libraries, and reaquired afterwards.

\begin{classdesc}{PyDLL}{name, mode=RTLD_LOCAL, handle=None}
Instances of this class behave like \class{CDLL} instances, except
that the Python GIL is \emph{not} released during the function call,
and after the function execution the Python error flag is checked.
If the error flag is set, a Python exception is raised.

Thus, this is only useful to call Python C api functions directly.
\end{classdesc}

All these classes can be instantiated by calling them with at least
one argument, the pathname of the shared library.  If you have an
existing handle to an already loaded shard library, it can be passed
as the \code{handle} named parameter, otherwise the underlying platforms
\code{dlopen} or \method{LoadLibrary} function is used to load the library
into the process, and to get a handle to it.

The \var{mode} parameter can be used to specify how the library is
loaded.  For details, consult the \code{dlopen(3)} manpage, on Windows,
\var{mode} is ignored.

\begin{datadescni}{RTLD_GLOBAL}
Flag to use as \var{mode} parameter.  On platforms where this flag
is not available, it is defined as the integer zero.
\end{datadescni}

\begin{datadescni}{RTLD_LOCAL}
Flag to use as \var{mode} parameter.  On platforms where this is not
available, it is the same as \var{RTLD{\_}GLOBAL}.
\end{datadescni}

Instances of these classes have no public methods, however
\method{{\_}{\_}getattr{\_}{\_}} and \method{{\_}{\_}getitem{\_}{\_}} have special behaviour: functions
exported by the shared library can be accessed as attributes of by
index.  Please note that both \method{{\_}{\_}getattr{\_}{\_}} and \method{{\_}{\_}getitem{\_}{\_}}
cache their result, so calling them repeatedly returns the same object
each time.

The following public attributes are available, their name starts with
an underscore to not clash with exported function names:

\begin{datadescni}{_handle: memberdesc}
The system handle used to access the library.
\end{datadescni}

\begin{datadescni}{_name: memberdesc}
The name of the library passed in the contructor.
\end{datadescni}

Shared libraries can also be loaded by using one of the prefabricated
objects, which are instances of the \class{LibraryLoader} class, either by
calling the \method{LoadLibrary} method, or by retrieving the library as
attribute of the loader instance.

\begin{classdesc}{LibraryLoader}{dlltype}
Class which loads shared libraries.  \code{dlltype} should be one
of the \class{CDLL}, \class{PyDLL}, \class{WinDLL}, or \class{OleDLL} types.

\method{{\_}{\_}getattr{\_}{\_}} has special behaviour: It allows to load a shared
library by accessing it as attribute of a library loader
instance.  The result is cached, so repeated attribute accesses
return the same library each time.
\end{classdesc}

\begin{methoddesc}{LoadLibrary}{name, mode=RTLD_LOCAL, handle=None}
Load a shared library into the process and return it.  This method
always creates a new instance of the library.  All three
parameters are passed to the constructor of the library object.
\end{methoddesc}

These prefabricated library loaders are available:

\begin{datadescni}{cdll}
Creates \class{CDLL} instances.
\end{datadescni}

\begin{datadescni}{windll}
Windows only: Creates \class{WinDLL} instances.
\end{datadescni}

\begin{datadescni}{oledll}
Windows only: Creates \class{OleDLL} instances.
\end{datadescni}

\begin{datadescni}{pydll}
Creates \class{PyDLL} instances.
\end{datadescni}

For accessing the C Python api directly, a ready-to-use Python shared
library object is available:

\begin{datadescni}{pythonapi}
An instance of \class{PyDLL} that exposes Python C api functions as
attributes.  Note that all these functions are assumed to return
integers, which is of course not always the truth, so you have to
assign the correct \member{restype} attribute to use these functions.
\end{datadescni}


\subsubsection{Foreign functions\label{ctypes-foreign-functions}}

As explained in the previous section, foreign functions can be
accessed as attributes of loaded shared libraries.  The function
objects created in this way by default accept any number of arguments,
accept any ctypes data instances as arguments, and return the default
result type specified by the library loader.  They are instances of a
private class:

\begin{classdesc*}{_FuncPtr}
Base class for C callable foreign functions.
\end{classdesc*}

Instances of foreign functions are also C compatible data types; they
represent C function pointers.

This behaviour can be customized by assigning to special attributes of
the foreign function object.

\begin{memberdesc}{restype}
Assign a ctypes type to specify the result type of the foreign
function.  Use \code{None} for \code{void} a function not returning
anything.

It is possible to assign a callable Python object that is not a
ctypes type, in this case the function is assumed to return an
integer, and the callable will be called with this integer,
allowing to do further processing or error checking.  Using this
is deprecated, for more flexible postprocessing or error checking
use a ctypes data type as \member{restype} and assign a callable to the
\member{errcheck} attribute.
\end{memberdesc}

\begin{memberdesc}{argtypes}
Assign a tuple of ctypes types to specify the argument types that
the function accepts.  Functions using the \code{stdcall} calling
convention can only be called with the same number of arguments as
the length of this tuple; functions using the C calling convention
accept additional, unspecified arguments as well.

When a foreign function is called, each actual argument is passed
to the \method{from{\_}param} class method of the items in the
\member{argtypes} tuple, this method allows to adapt the actual
argument to an object that the foreign function accepts.  For
example, a \class{c{\_}char{\_}p} item in the \member{argtypes} tuple will
convert a unicode string passed as argument into an byte string
using ctypes conversion rules.
\end{memberdesc}

\begin{memberdesc}{errcheck}
Assign a Python function or another callable to this attribute.
The callable will be called with three or more arguments:
\end{memberdesc}

\begin{funcdescni}{callable}{result, func, arguments}
\code{result} is what the foreign function returns, as specified by the
\member{restype} attribute.

\code{func} is the foreign function object itself, this allows to
reuse the same callable object to check or postprocess the results
of several functions.

\code{arguments} is a tuple containing the parameters originally
passed to the function call, this allows to specialize the
behaviour on the arguments used.

The object that this function returns will be returned from the
foreign function call, but it can also check the result value and
raise an exception if the foreign function call failed.
\end{funcdescni}

\begin{excdesc}{ArgumentError()}
This exception is raised when a foreign function call cannot
convert one of the passed arguments.
\end{excdesc}


\subsubsection{Function prototypes\label{ctypes-function-prototypes}}

Foreign functions can also be created by instantiating function
prototypes.  Function prototypes are similar to function prototypes in
C; they describe a function (return type, argument types, calling
convention) without defining an implementation.  The factory
functions must be called with the desired result type and the argument
types of the function.

\begin{funcdesc}{CFUNCTYPE}{restype, *argtypes}
The returned function prototype creates functions that use the
standard C calling convention.  The function will release the GIL
during the call.
\end{funcdesc}

\begin{funcdesc}{WINFUNCTYPE}{restype, *argtypes}
Windows only: The returned function prototype creates functions
that use the \code{stdcall} calling convention, except on Windows CE
where \function{WINFUNCTYPE} is the same as \function{CFUNCTYPE}.  The function
will release the GIL during the call.
\end{funcdesc}

\begin{funcdesc}{PYFUNCTYPE}{restype, *argtypes}
The returned function prototype creates functions that use the
Python calling convention.  The function will \emph{not} release the
GIL during the call.
\end{funcdesc}

Function prototypes created by the factory functions can be
instantiated in different ways, depending on the type and number of
the parameters in the call.

\begin{funcdescni}{prototype}{address}
Returns a foreign function at the specified address.
\end{funcdescni}

\begin{funcdescni}{prototype}{callable}
Create a C callable function (a callback function) from a Python
\code{callable}.
\end{funcdescni}

\begin{funcdescni}{prototype}{func_spec\optional{, paramflags}}
Returns a foreign function exported by a shared library.
\code{func{\_}spec} must be a 2-tuple \code{(name{\_}or{\_}ordinal, library)}.
The first item is the name of the exported function as string, or
the ordinal of the exported function as small integer.  The second
item is the shared library instance.
\end{funcdescni}

\begin{funcdescni}{prototype}{vtbl_index, name\optional{, paramflags\optional{, iid}}}
Returns a foreign function that will call a COM method.
\code{vtbl{\_}index} is the index into the virtual function table, a
small nonnegative integer. \var{name} is name of the COM method.
\var{iid} is an optional pointer to the interface identifier which
is used in extended error reporting.

COM methods use a special calling convention: They require a
pointer to the COM interface as first argument, in addition to
those parameters that are specified in the \member{argtypes} tuple.
\end{funcdescni}

The optional \var{paramflags} parameter creates foreign function
wrappers with much more functionality than the features described
above.

\var{paramflags} must be a tuple of the same length as \member{argtypes}.

Each item in this tuple contains further information about a
parameter, it must be a tuple containing 1, 2, or 3 items.

The first item is an integer containing flags for the parameter.

\begin{datadescni}{1}
Specifies an input parameter to the function.
\end{datadescni}

\begin{datadescni}{2}
Output parameter.  The foreign function fills in a value.
\end{datadescni}

\begin{datadescni}{4}
Input parameter which defaults to the integer zero.
\end{datadescni}

The optional second item is the parameter name as string.  If this is
specified, the foreign function can be called with named parameters.

The optional third item is the default value for this parameter.

This example demonstrates how to wrap the Windows \code{MessageBoxA}
function so that it supports default parameters and named arguments.
The C declaration from the windows header file is this:
\begin{verbatim}
WINUSERAPI int WINAPI
MessageBoxA(
    HWND hWnd ,
    LPCSTR lpText,
    LPCSTR lpCaption,
    UINT uType);
\end{verbatim}

Here is the wrapping with \code{ctypes}:
\begin{quote}
\begin{verbatim}>>> from ctypes import c_int, WINFUNCTYPE, windll
>>> from ctypes.wintypes import HWND, LPCSTR, UINT
>>> prototype = WINFUNCTYPE(c_int, HWND, LPCSTR, LPCSTR, c_uint)
>>> paramflags = (1, "hwnd", 0), (1, "text", "Hi"), (1, "caption", None), (1, "flags", 0)
>>> MessageBox = prototype(("MessageBoxA", windll.user32), paramflags)
>>>\end{verbatim}
\end{quote}

The MessageBox foreign function can now be called in these ways:
\begin{verbatim}
>>> MessageBox()
>>> MessageBox(text="Spam, spam, spam")
>>> MessageBox(flags=2, text="foo bar")
>>>
\end{verbatim}

A second example demonstrates output parameters.  The win32
\code{GetWindowRect} function retrieves the dimensions of a specified
window by copying them into \code{RECT} structure that the caller has to
supply.  Here is the C declaration:
\begin{verbatim}
WINUSERAPI BOOL WINAPI
GetWindowRect(
    HWND hWnd,
    LPRECT lpRect);
\end{verbatim}

Here is the wrapping with \code{ctypes}:
\begin{quote}
\begin{verbatim}>>> from ctypes import POINTER, WINFUNCTYPE, windll
>>> from ctypes.wintypes import BOOL, HWND, RECT
>>> prototype = WINFUNCTYPE(BOOL, HWND, POINTER(RECT))
>>> paramflags = (1, "hwnd"), (2, "lprect")
>>> GetWindowRect = prototype(("GetWindowRect", windll.user32), paramflags)
>>>\end{verbatim}
\end{quote}

Functions with output parameters will automatically return the output
parameter value if there is a single one, or a tuple containing the
output parameter values when there are more than one, so the
GetWindowRect function now returns a RECT instance, when called.

Output parameters can be combined with the \member{errcheck} protocol to do
further output processing and error checking.  The win32
\code{GetWindowRect} api function returns a \code{BOOL} to signal success or
failure, so this function could do the error checking, and raises an
exception when the api call failed:
\begin{verbatim}
>>> def errcheck(result, func, args):
...     if not result:
...         raise WinError()
...     return args
>>> GetWindowRect.errcheck = errcheck
>>>
\end{verbatim}

If the \member{errcheck} function returns the argument tuple it receives
unchanged, \code{ctypes} continues the normal processing it does on the
output parameters.  If you want to return a tuple of window
coordinates instead of a \code{RECT} instance, you can retrieve the
fields in the function and return them instead, the normal processing
will no longer take place:
\begin{verbatim}
>>> def errcheck(result, func, args):
...     if not result:
...         raise WinError()
...     rc = args[1]
...     return rc.left, rc.top, rc.bottom, rc.right
>>>
>>> GetWindowRect.errcheck = errcheck
>>>
\end{verbatim}


\subsubsection{Utility functions\label{ctypes-utility-functions}}

\begin{funcdesc}{addressof}{obj}
Returns the address of the memory buffer as integer.  \code{obj} must
be an instance of a ctypes type.
\end{funcdesc}

\begin{funcdesc}{alignment}{obj_or_type}
Returns the alignment requirements of a ctypes type.
\code{obj{\_}or{\_}type} must be a ctypes type or instance.
\end{funcdesc}

\begin{funcdesc}{byref}{obj}
Returns a light-weight pointer to \code{obj}, which must be an
instance of a ctypes type. The returned object can only be used as
a foreign function call parameter. It behaves similar to
\code{pointer(obj)}, but the construction is a lot faster.
\end{funcdesc}

\begin{funcdesc}{cast}{obj, type}
This function is similar to the cast operator in C. It returns a
new instance of \code{type} which points to the same memory block as
\code{obj}. \code{type} must be a pointer type, and \code{obj} must be an
object that can be interpreted as a pointer.
\end{funcdesc}

\begin{funcdesc}{create_string_buffer}{init_or_size\optional{, size}}
This function creates a mutable character buffer. The returned
object is a ctypes array of \class{c{\_}char}.

\code{init{\_}or{\_}size} must be an integer which specifies the size of
the array, or a string which will be used to initialize the array
items.

If a string is specified as first argument, the buffer is made one
item larger than the length of the string so that the last element
in the array is a NUL termination character. An integer can be
passed as second argument which allows to specify the size of the
array if the length of the string should not be used.

If the first parameter is a unicode string, it is converted into
an 8-bit string according to ctypes conversion rules.
\end{funcdesc}

\begin{funcdesc}{create_unicode_buffer}{init_or_size\optional{, size}}
This function creates a mutable unicode character buffer. The
returned object is a ctypes array of \class{c{\_}wchar}.

\code{init{\_}or{\_}size} must be an integer which specifies the size of
the array, or a unicode string which will be used to initialize
the array items.

If a unicode string is specified as first argument, the buffer is
made one item larger than the length of the string so that the
last element in the array is a NUL termination character. An
integer can be passed as second argument which allows to specify
the size of the array if the length of the string should not be
used.

If the first parameter is a 8-bit string, it is converted into an
unicode string according to ctypes conversion rules.
\end{funcdesc}

\begin{funcdesc}{DllCanUnloadNow}{}
Windows only: This function is a hook which allows to implement
inprocess COM servers with ctypes. It is called from the
DllCanUnloadNow function that the {\_}ctypes extension dll exports.
\end{funcdesc}

\begin{funcdesc}{DllGetClassObject}{}
Windows only: This function is a hook which allows to implement
inprocess COM servers with ctypes. It is called from the
DllGetClassObject function that the \code{{\_}ctypes} extension dll exports.
\end{funcdesc}

\begin{funcdesc}{FormatError}{\optional{code}}
Windows only: Returns a textual description of the error code. If
no error code is specified, the last error code is used by calling
the Windows api function GetLastError.
\end{funcdesc}

\begin{funcdesc}{GetLastError}{}
Windows only: Returns the last error code set by Windows in the
calling thread.
\end{funcdesc}

\begin{funcdesc}{memmove}{dst, src, count}
Same as the standard C memmove library function: copies \var{count}
bytes from \code{src} to \var{dst}. \var{dst} and \code{src} must be
integers or ctypes instances that can be converted to pointers.
\end{funcdesc}

\begin{funcdesc}{memset}{dst, c, count}
Same as the standard C memset library function: fills the memory
block at address \var{dst} with \var{count} bytes of value
\var{c}. \var{dst} must be an integer specifying an address, or a
ctypes instance.
\end{funcdesc}

\begin{funcdesc}{POINTER}{type}
This factory function creates and returns a new ctypes pointer
type. Pointer types are cached an reused internally, so calling
this function repeatedly is cheap. type must be a ctypes type.
\end{funcdesc}

\begin{funcdesc}{pointer}{obj}
This function creates a new pointer instance, pointing to
\code{obj}. The returned object is of the type POINTER(type(obj)).

Note: If you just want to pass a pointer to an object to a foreign
function call, you should use \code{byref(obj)} which is much faster.
\end{funcdesc}

\begin{funcdesc}{resize}{obj, size}
This function resizes the internal memory buffer of obj, which
must be an instance of a ctypes type. It is not possible to make
the buffer smaller than the native size of the objects type, as
given by sizeof(type(obj)), but it is possible to enlarge the
buffer.
\end{funcdesc}

\begin{funcdesc}{set_conversion_mode}{encoding, errors}
This function sets the rules that ctypes objects use when
converting between 8-bit strings and unicode strings. encoding
must be a string specifying an encoding, like \code{'utf-8'} or
\code{'mbcs'}, errors must be a string specifying the error handling
on encoding/decoding errors. Examples of possible values are
\code{"strict"}, \code{"replace"}, or \code{"ignore"}.

\code{set{\_}conversion{\_}mode} returns a 2-tuple containing the previous
conversion rules. On windows, the initial conversion rules are
\code{('mbcs', 'ignore')}, on other systems \code{('ascii', 'strict')}.
\end{funcdesc}

\begin{funcdesc}{sizeof}{obj_or_type}
Returns the size in bytes of a ctypes type or instance memory
buffer. Does the same as the C \code{sizeof()} function.
\end{funcdesc}

\begin{funcdesc}{string_at}{address\optional{, size}}
This function returns the string starting at memory address
address. If size is specified, it is used as size, otherwise the
string is assumed to be zero-terminated.
\end{funcdesc}

\begin{funcdesc}{WinError}{code=None, descr=None}
Windows only: this function is probably the worst-named thing in
ctypes. It creates an instance of WindowsError. If \var{code} is not
specified, \code{GetLastError} is called to determine the error
code. If \code{descr} is not spcified, \function{FormatError} is called to
get a textual description of the error.
\end{funcdesc}

\begin{funcdesc}{wstring_at}{address}
This function returns the wide character string starting at memory
address \code{address} as unicode string. If \code{size} is specified,
it is used as the number of characters of the string, otherwise
the string is assumed to be zero-terminated.
\end{funcdesc}


\subsubsection{Data types\label{ctypes-data-types}}

\begin{classdesc*}{_CData}
This non-public class is the common base class of all ctypes data
types.  Among other things, all ctypes type instances contain a
memory block that hold C compatible data; the address of the
memory block is returned by the \code{addressof()} helper function.
Another instance variable is exposed as \member{{\_}objects}; this
contains other Python objects that need to be kept alive in case
the memory block contains pointers.
\end{classdesc*}

Common methods of ctypes data types, these are all class methods (to
be exact, they are methods of the metaclass):

\begin{methoddesc}{from_address}{address}
This method returns a ctypes type instance using the memory
specified by address which must be an integer.
\end{methoddesc}

\begin{methoddesc}{from_param}{obj}
This method adapts obj to a ctypes type.  It is called with the
actual object used in a foreign function call, when the type is
present in the foreign functions \member{argtypes} tuple; it must
return an object that can be used as function call parameter.

All ctypes data types have a default implementation of this
classmethod, normally it returns \code{obj} if that is an instance of
the type.  Some types accept other objects as well.
\end{methoddesc}

\begin{methoddesc}{in_dll}{name, library}
This method returns a ctypes type instance exported by a shared
library. \var{name} is the name of the symbol that exports the data,
\code{library} is the loaded shared library.
\end{methoddesc}

Common instance variables of ctypes data types:

\begin{memberdesc}{_b_base_}
Sometimes ctypes data instances do not own the memory block they
contain, instead they share part of the memory block of a base
object.  The \member{{\_}b{\_}base{\_}} readonly member is the root ctypes
object that owns the memory block.
\end{memberdesc}

\begin{memberdesc}{_b_needsfree_}
This readonly variable is true when the ctypes data instance has
allocated the memory block itself, false otherwise.
\end{memberdesc}

\begin{memberdesc}{_objects}
This member is either \code{None} or a dictionary containing Python
objects that need to be kept alive so that the memory block
contents is kept valid.  This object is only exposed for
debugging; never modify the contents of this dictionary.
\end{memberdesc}


\subsubsection{Fundamental data types\label{ctypes-fundamental-data-types}}

\begin{classdesc*}{_SimpleCData}
This non-public class is the base class of all fundamental ctypes
data types. It is mentioned here because it contains the common
attributes of the fundamental ctypes data types.  \code{{\_}SimpleCData}
is a subclass of \code{{\_}CData}, so it inherits their methods and
attributes.
\end{classdesc*}

Instances have a single attribute:

\begin{memberdesc}{value}
This attribute contains the actual value of the instance. For
integer and pointer types, it is an integer, for character types,
it is a single character string, for character pointer types it
is a Python string or unicode string.

When the \code{value} attribute is retrieved from a ctypes instance,
usually a new object is returned each time.  \code{ctypes} does \emph{not}
implement original object return, always a new object is
constructed.  The same is true for all other ctypes object
instances.
\end{memberdesc}

Fundamental data types, when returned as foreign function call
results, or, for example, by retrieving structure field members or
array items, are transparently converted to native Python types.  In
other words, if a foreign function has a \member{restype} of \class{c{\_}char{\_}p},
you will always receive a Python string, \emph{not} a \class{c{\_}char{\_}p}
instance.

Subclasses of fundamental data types do \emph{not} inherit this behaviour.
So, if a foreign functions \member{restype} is a subclass of \class{c{\_}void{\_}p},
you will receive an instance of this subclass from the function call.
Of course, you can get the value of the pointer by accessing the
\code{value} attribute.

These are the fundamental ctypes data types:

\begin{classdesc*}{c_byte}
Represents the C signed char datatype, and interprets the value as
small integer. The constructor accepts an optional integer
initializer; no overflow checking is done.
\end{classdesc*}

\begin{classdesc*}{c_char}
Represents the C char datatype, and interprets the value as a single
character. The constructor accepts an optional string initializer,
the length of the string must be exactly one character.
\end{classdesc*}

\begin{classdesc*}{c_char_p}
Represents the C char * datatype, which must be a pointer to a
zero-terminated string. The constructor accepts an integer
address, or a string.
\end{classdesc*}

\begin{classdesc*}{c_double}
Represents the C double datatype. The constructor accepts an
optional float initializer.
\end{classdesc*}

\begin{classdesc*}{c_float}
Represents the C double datatype. The constructor accepts an
optional float initializer.
\end{classdesc*}

\begin{classdesc*}{c_int}
Represents the C signed int datatype. The constructor accepts an
optional integer initializer; no overflow checking is done. On
platforms where \code{sizeof(int) == sizeof(long)} it is an alias to
\class{c{\_}long}.
\end{classdesc*}

\begin{classdesc*}{c_int8}
Represents the C 8-bit \code{signed int} datatype. Usually an alias for
\class{c{\_}byte}.
\end{classdesc*}

\begin{classdesc*}{c_int16}
Represents the C 16-bit signed int datatype. Usually an alias for
\class{c{\_}short}.
\end{classdesc*}

\begin{classdesc*}{c_int32}
Represents the C 32-bit signed int datatype. Usually an alias for
\class{c{\_}int}.
\end{classdesc*}

\begin{classdesc*}{c_int64}
Represents the C 64-bit \code{signed int} datatype. Usually an alias
for \class{c{\_}longlong}.
\end{classdesc*}

\begin{classdesc*}{c_long}
Represents the C \code{signed long} datatype. The constructor accepts an
optional integer initializer; no overflow checking is done.
\end{classdesc*}

\begin{classdesc*}{c_longlong}
Represents the C \code{signed long long} datatype. The constructor accepts
an optional integer initializer; no overflow checking is done.
\end{classdesc*}

\begin{classdesc*}{c_short}
Represents the C \code{signed short} datatype. The constructor accepts an
optional integer initializer; no overflow checking is done.
\end{classdesc*}

\begin{classdesc*}{c_size_t}
Represents the C \code{size{\_}t} datatype.
\end{classdesc*}

\begin{classdesc*}{c_ubyte}
Represents the C \code{unsigned char} datatype, it interprets the
value as small integer. The constructor accepts an optional
integer initializer; no overflow checking is done.
\end{classdesc*}

\begin{classdesc*}{c_uint}
Represents the C \code{unsigned int} datatype. The constructor accepts an
optional integer initializer; no overflow checking is done. On
platforms where \code{sizeof(int) == sizeof(long)} it is an alias for
\class{c{\_}ulong}.
\end{classdesc*}

\begin{classdesc*}{c_uint8}
Represents the C 8-bit unsigned int datatype. Usually an alias for
\class{c{\_}ubyte}.
\end{classdesc*}

\begin{classdesc*}{c_uint16}
Represents the C 16-bit unsigned int datatype. Usually an alias for
\class{c{\_}ushort}.
\end{classdesc*}

\begin{classdesc*}{c_uint32}
Represents the C 32-bit unsigned int datatype. Usually an alias for
\class{c{\_}uint}.
\end{classdesc*}

\begin{classdesc*}{c_uint64}
Represents the C 64-bit unsigned int datatype. Usually an alias for
\class{c{\_}ulonglong}.
\end{classdesc*}

\begin{classdesc*}{c_ulong}
Represents the C \code{unsigned long} datatype. The constructor accepts an
optional integer initializer; no overflow checking is done.
\end{classdesc*}

\begin{classdesc*}{c_ulonglong}
Represents the C \code{unsigned long long} datatype. The constructor
accepts an optional integer initializer; no overflow checking is
done.
\end{classdesc*}

\begin{classdesc*}{c_ushort}
Represents the C \code{unsigned short} datatype. The constructor accepts an
optional integer initializer; no overflow checking is done.
\end{classdesc*}

\begin{classdesc*}{c_void_p}
Represents the C \code{void *} type. The value is represented as
integer. The constructor accepts an optional integer initializer.
\end{classdesc*}

\begin{classdesc*}{c_wchar}
Represents the C \code{wchar{\_}t} datatype, and interprets the value as a
single character unicode string. The constructor accepts an
optional string initializer, the length of the string must be
exactly one character.
\end{classdesc*}

\begin{classdesc*}{c_wchar_p}
Represents the C \code{wchar{\_}t *} datatype, which must be a pointer to
a zero-terminated wide character string. The constructor accepts
an integer address, or a string.
\end{classdesc*}

\begin{classdesc*}{HRESULT}
Windows only: Represents a \class{HRESULT} value, which contains success
or error information for a function or method call.
\end{classdesc*}

\begin{classdesc*}{py_object}
Represents the C \code{PyObject *} datatype.
\end{classdesc*}

The \code{ctypes.wintypes} module provides quite some other Windows
specific data types, for example \code{HWND}, \code{WPARAM}, or \code{DWORD}.
Some useful structures like \code{MSG} or \code{RECT} are also defined.


\subsubsection{Structured data types\label{ctypes-structured-data-types}}

\begin{classdesc}{Union}{*args, **kw}
Abstract base class for unions in native byte order.
\end{classdesc}

\begin{classdesc}{BigEndianStructure}{*args, **kw}
Abstract base class for structures in \emph{big endian} byte order.
\end{classdesc}

\begin{classdesc}{LittleEndianStructure}{*args, **kw}
Abstract base class for structures in \emph{little endian} byte order.
\end{classdesc}

Structures with non-native byte order cannot contain pointer type
fields, or any other data types containing pointer type fields.

\begin{classdesc}{Structure}{*args, **kw}
Abstract base class for structures in \emph{native} byte order.
\end{classdesc}

Concrete structure and union types must be created by subclassing one
of these types, and at least define a \member{{\_}fields{\_}} class variable.
\code{ctypes} will create descriptors which allow reading and writing the
fields by direct attribute accesses.  These are the

\begin{memberdesc}{_fields_}
A sequence defining the structure fields.  The items must be
2-tuples or 3-tuples.  The first item is the name of the field,
the second item specifies the type of the field; it can be any
ctypes data type.

For integer type fields, a third optional item can be given.  It
must be a small positive integer defining the bit width of the
field.

Field names must be unique within one structure or union.  This is
not checked, only one field can be accessed when names are
repeated.

It is possible to define the \member{{\_}fields{\_}} class variable \emph{after}
the class statement that defines the Structure subclass, this
allows to create data types that directly or indirectly reference
themselves:
\begin{verbatim}
class List(Structure):
    pass
List._fields_ = [("pnext", POINTER(List)),
                 ...
                ]
\end{verbatim}

The \member{{\_}fields{\_}} class variable must, however, be defined before
the type is first used (an instance is created, \code{sizeof()} is
called on it, and so on).  Later assignments to the \member{{\_}fields{\_}}
class variable will raise an AttributeError.

Structure and union subclass constructors accept both positional
and named arguments.  Positional arguments are used to initialize
the fields in the same order as they appear in the \member{{\_}fields{\_}}
definition, named arguments are used to initialize the fields with
the corresponding name.

It is possible to defined sub-subclasses of structure types, they
inherit the fields of the base class plus the \member{{\_}fields{\_}} defined
in the sub-subclass, if any.
\end{memberdesc}

\begin{memberdesc}{_pack_}
An optional small integer that allows to override the alignment of
structure fields in the instance.  \member{{\_}pack{\_}} must already be
defined when \member{{\_}fields{\_}} is assigned, otherwise it will have no
effect.
\end{memberdesc}

\begin{memberdesc}{_anonymous_}
An optional sequence that lists the names of unnamed (anonymous)
fields.  \code{{\_}anonymous{\_}} must be already defined when \member{{\_}fields{\_}}
is assigned, otherwise it will have no effect.

The fields listed in this variable must be structure or union type
fields.  \code{ctypes} will create descriptors in the structure type
that allows to access the nested fields directly, without the need
to create the structure or union field.

Here is an example type (Windows):
\begin{verbatim}
class _U(Union):
    _fields_ = [("lptdesc", POINTER(TYPEDESC)),
                ("lpadesc", POINTER(ARRAYDESC)),
                ("hreftype", HREFTYPE)]

class TYPEDESC(Structure):
    _fields_ = [("u", _U),
                ("vt", VARTYPE)]

    _anonymous_ = ("u",)
\end{verbatim}

The \code{TYPEDESC} structure describes a COM data type, the \code{vt}
field specifies which one of the union fields is valid.  Since the
\code{u} field is defined as anonymous field, it is now possible to
access the members directly off the TYPEDESC instance.
\code{td.lptdesc} and \code{td.u.lptdesc} are equivalent, but the former
is faster since it does not need to create a temporary \code{{\_}U}
instance:
\begin{verbatim}
td = TYPEDESC()
td.vt = VT_PTR
td.lptdesc = POINTER(some_type)
td.u.lptdesc = POINTER(some_type)
\end{verbatim}
\end{memberdesc}

It is possible to defined sub-subclasses of structures, they inherit
the fields of the base class.  If the subclass definition has a
separate``{\_}fields{\_}`` variable, the fields specified in this are
appended to the fields of the base class.


\subsubsection{Arrays and pointers\label{ctypes-arrays-pointers}}

XXX

