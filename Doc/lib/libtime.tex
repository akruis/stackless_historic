\section{Built-in Module \sectcode{time}}

\bimodindex{time}
This module provides various time-related functions.
It is always available.

An explanation of some terminology and conventions is in order.

\begin{itemize}

\item
The ``epoch'' is the point where the time starts.  On January 1st of that
year, at 0 hours, the ``time since the epoch'' is zero.  For UNIX, the
epoch is 1970.  To find out what the epoch is, look at \code{gmtime(0)}.

\item
UTC is Coordinated Universal Time (formerly known as Greenwich Mean
Time).  The acronym UTC is not a mistake but a compromise between
English and French.

\item
DST is Daylight Saving Time, an adjustment of the timezone by
(usually) one hour during part of the year.  DST rules are magic
(determined by local law) and can change from year to year.  The C
library has a table containing the local rules (often it is read from
a system file for flexibility) and is the only source of True Wisdom
in this respect.

\item
The precision of the various real-time functions may be less than
suggested by the units in which their value or argument is expressed.
E.g.\ on most UNIX systems, the clock ``ticks'' only 50 or 100 times a
second, and on the Mac, times are only accurate to whole seconds.

\item
The time tuple as returned by \code{gmtime()} and \code{localtime()},
or as accpted by \code{mktime()} is a tuple of 9
integers: year (e.g.\ 1993), month (1--12), day (1--31), hour
(0--23), minute (0--59), second (0--59), weekday (0--6, monday is 0),
Julian day (1--366) and daylight savings flag (-1, 0  or 1).
Note that unlike the C structure, the month value is a range of 1-12, not
0-11.  A year value of $<$ 100 will typically be silently converted to
1900 $+$ year value.  A -1 argument as daylight savings flag, passed to
\code{mktime()} will usually result in the correct daylight savings
state to be filled in.


\end{itemize}

The module defines the following functions and data items:

\renewcommand{\indexsubitem}{(in module time)}

\begin{datadesc}{altzone}
The offset of the local DST timezone, in seconds west of the 0th
meridian, if one is defined.  Negative if the local DST timezone is
east of the 0th meridian (as in Western Europe, including the UK).
Only use this if \code{daylight} is nonzero.
\end{datadesc}

\begin{funcdesc}{asctime}{tuple}
Convert a tuple representing a time as returned by \code{gmtime()} or
\code{localtime()} to a 24-character string of the following form:
\code{'Sun Jun 20 23:21:05 1993'}.  Note: unlike the C function of
the same name, there is no trailing newline.
\end{funcdesc}

\begin{funcdesc}{clock}{}
Return the current CPU time as a floating point number expressed in
seconds.  The precision, and in fact the very definiton of the meaning
of ``CPU time'', depends on that of the C function of the same name.
\end{funcdesc}

\begin{funcdesc}{ctime}{secs}
Convert a time expressed in seconds since the epoch to a string
representing local time.  \code{ctime(t)} is equivalent to
\code{asctime(localtime(t))}.
\end{funcdesc}

\begin{datadesc}{daylight}
Nonzero if a DST timezone is defined.
\end{datadesc}

\begin{funcdesc}{gmtime}{secs}
Convert a time expressed in seconds since the epoch to a time tuple
in UTC in which the dst flag is always zero.  Fractions of a second are
ignored.
\end{funcdesc}

\begin{funcdesc}{localtime}{secs}
Like \code{gmtime} but converts to local time.  The dst flag is set
to 1 when DST applies to the given time.
\end{funcdesc}

\begin{funcdesc}{mktime}{tuple}
This is the inverse function of \code{localtime}.  Its argument is the
full 9-tuple (since the dst flag is needed --- pass -1 as the dst flag if
it is unknown) which expresses the time
in \em{local} time, not UTC.  It returns a floating
point number, for compatibility with \code{time.time()}.  If the input
value can't be represented as a valid time, OverflowError is raised.
\end{funcdesc}

\begin{funcdesc}{sleep}{secs}
Suspend execution for the given number of seconds.  The argument may
be a floating point number to indicate a more precise sleep time.
\end{funcdesc}

\begin{funcdesc}{strftime}{format, tuple}
Convert a tuple representing a time as returned by \code{gmtime()} or
\code{localtime()} to a string as specified by the format argument.

      The following directives, shown without the optional field width and
      precision specification, are replaced by the indicated characters:

\begin{tabular}{lp{25em}}
           \%a  &      Locale's abbreviated weekday name. \\
           \%A  &      Locale's full weekday name. \\
           \%b  &      Locale's abbreviated month name. \\
           \%B  &      Locale's full month name. \\
           \%c  &      Locale's appropriate date and time representation. \\
           \%d  &      Day of the month as a decimal number [01,31]. \\
           \%E  &      Locale's combined Emperor/Era name and year. \\
           \%H  &      Hour (24-hour clock) as a decimal number [00,23]. \\
           \%I  &      Hour (12-hour clock) as a decimal number [01,12]. \\
           \%j  &      Day of the year as a decimal number [001,366]. \\
           \%m  &      Month as a decimal number [01,12]. \\
           \%M  &      Minute as a decimal number [00,59]. \\
           \%n  &      New-line character. \\
           \%N  &      Locale's Emperor/Era name. \\
           \%o  &      Locale's Emperor/Era year. \\
           \%p  &      Locale's equivalent of either AM or PM. \\
           \%S  &      Second as a decimal number [00,61]. \\
           \%t  &      Tab character. \\
           \%U  &      Week number of the year (Sunday as the first day of the
                     week) as a decimal number [00,53].  All days in a new
                     year preceding the first Sunday are considered to be in
                     week 0. \\
           \%w  &      Weekday as a decimal number [0(Sunday),6]. \\
           \%W  &      Week number of the year (Monday as the first day of the
                     week) as a decimal number [00,53].  All days in a new
                     year preceding the first Sunday are considered to be in
                     week 0. \\
           \%x  &      Locale's appropriate date representation. \\
           \%X  &      Locale's appropriate time representation. \\
           \%y  &      Year without century as a decimal number [00,99]. \\
           \%Y  &      Year with century as a decimal number. \\
           \%Z  &      Time zone name (or by no characters if no time zone
                     exists). \\
           \%\%  &     \% \\
\end{tabular}

      An optional field width and precision specification can immediately
      follow the initial \% of a directive in the following order: \\

\begin{tabular}{lp{25em}}
      [-|0]w  &       the decimal digit string w specifies a minimum field
                     width in which the result of the conversion is right-
                     or left-justified.  It is right-justified (with space
                     padding) by default.  If the optional flag `-' is
                     specified, it is left-justified with space padding on
                     the right.  If the optional flag `0' is specified, it
                     is right-justified and padded with zeros on the left. \\
      .p      &       the decimal digit string p specifies the minimum number
                     of digits to appear for the d, H, I, j, m, M, o, S, U,
                     w, W, y and Y directives, and the maximum number of
                     characters to be used from the a, A, b, B, c, D, E, F,
                     h, n, N, p, r, t, T, x, X, z, Z, and % directives.  In
                     the first case, if a directive supplies fewer digits
                     than specified by the precision, it will be expanded
                     with leading zeros.  In the second case, if a directive
                     supplies more characters than specified by the
                     precision, excess characters will truncated on the
                     right.
\end{tabular}

      If no field width or precision is specified for a d, H, I, m, M, S, U,
      W, y, or j directive, a default of .2 is used for all but j for which
      .3 is used.

\end{funcdesc}

\begin{funcdesc}{time}{}
Return the time as a floating point number expressed in seconds since
the epoch, in UTC.  Note that even though the time is always returned
as a floating point number, not all systems provide time with a better
precision than 1 second.
\end{funcdesc}

\begin{datadesc}{timezone}
The offset of the local (non-DST) timezone, in seconds west of the 0th
meridian (i.e. negative in most of Western Europe, positive in the US,
zero in the UK).
\end{datadesc}

\begin{datadesc}{tzname}
A tuple of two strings: the first is the name of the local non-DST
timezone, the second is the name of the local DST timezone.  If no DST
timezone is defined, the second string should not be used.
\end{datadesc}

