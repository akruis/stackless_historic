\section{\module{time} ---
         Time access and conversions.}
\declaremodule{builtin}{time}

\modulesynopsis{Time access and conversions.}


This module provides various time-related functions.
It is always available.

An explanation of some terminology and conventions is in order.

\begin{itemize}

\item
The \dfn{epoch}\index{epoch} is the point where the time starts.  On
January 1st of that year, at 0 hours, the ``time since the epoch'' is
zero.  For \UNIX{}, the epoch is 1970.  To find out what the epoch is,
look at \code{gmtime(0)}.%
\index{epoch}

\item
The functions in this module don't handle dates and times before the
epoch or far in the future.  The cut-off point in the future is
determined by the C library; for \UNIX{}, it is typically in 2038.%
\index{Year 2038}

\item
\strong{Year 2000 (Y2K) issues}: Python depends on the platform's C library,
which generally doesn't have year 2000 issues, since all dates and
times are represented internally as seconds since the epoch.
Functions accepting a time tuple (see below) generally require a
4-digit year.  For backward compatibility, 2-digit years are supported
if the module variable \code{accept2dyear} is a non-zero integer; this
variable is initialized to \code{1} unless the environment variable
\code{PYTHONY2K} is set to a non-empty string, in which case it is
initialized to \code{0}.  Thus, you can set \code{PYTHONY2K} in the
environment to \code{x} to require 4-digit years for all year input.
When 2-digit years are accepted, they are converted according to the
POSIX or X/Open standard: values 69-99 are mapped to 1969-1999, and
values 0--68 are mapped to 2000--2068.  Values 100--1899 are always
illegal.  Note that this is new as of Python 1.5.2(a2); earlier
versions, up to Python 1.5.1 and 1.5.2a1, would add 1900 to year
values below 1900.%
\index{Year 2000}%
\index{Y2K}

\item
UTC is Coordinated Universal Time (formerly known as Greenwich Mean
Time).  The acronym UTC is not a mistake but a compromise between
English and French.%
\index{UTC}%
\index{Coordinated Universal Time}%
\index{Greenwich Mean Time}

\item
DST is Daylight Saving Time, an adjustment of the timezone by
(usually) one hour during part of the year.  DST rules are magic
(determined by local law) and can change from year to year.  The \C{}
library has a table containing the local rules (often it is read from
a system file for flexibility) and is the only source of True Wisdom
in this respect.%
\index{Daylight Saving Time}

\item
The precision of the various real-time functions may be less than
suggested by the units in which their value or argument is expressed.
E.g.\ on most \UNIX{} systems, the clock ``ticks'' only 50 or 100 times a
second, and on the Mac, times are only accurate to whole seconds.

\item
On the other hand, the precision of \function{time()} and
\function{sleep()} is better than their \UNIX{} equivalents: times are
expressed as floating point numbers, \function{time()} returns the
most accurate time available (using \UNIX{} \cfunction{gettimeofday()}
where available), and \function{sleep()} will accept a time with a
nonzero fraction (\UNIX{} \cfunction{select()} is used to implement
this, where available).

\item
The time tuple as returned by \function{gmtime()},
\function{localtime()}, and \function{strptime()}, and accepted by
\function{asctime()}, \function{mktime()} and \function{strftime()}, is a
tuple of 9 integers: year (e.g.\ 1993), month (1--12), day (1--31),
hour (0--23), minute (0--59), second (0--59), weekday (0--6, monday is
0), Julian day (1--366) and daylight savings flag (-1, 0  or 1).
Note that unlike the \C{} structure, the month value is a range of 1-12, not
0-11.  A year value less than 100 will typically be silently converted to
1900 plus the year value.  A \code{-1} argument as daylight savings
flag, passed to \function{mktime()} will usually result in the correct
daylight savings state to be filled in.

\end{itemize}

The module defines the following functions and data items:


\begin{datadesc}{altzone}
The offset of the local DST timezone, in seconds west of the 0th
meridian, if one is defined.  Negative if the local DST timezone is
east of the 0th meridian (as in Western Europe, including the UK).
Only use this if \code{daylight} is nonzero.
\end{datadesc}

\begin{funcdesc}{asctime}{tuple}
Convert a tuple representing a time as returned by \code{gmtime()} or
\code{localtime()} to a 24-character string of the following form:
\code{'Sun Jun 20 23:21:05 1993'}.  Note: unlike the C function of
the same name, there is no trailing newline.
\end{funcdesc}

\begin{funcdesc}{clock}{}
Return the current CPU time as a floating point number expressed in
seconds.  The precision, and in fact the very definiton of the meaning
of ``CPU time''\index{CPU time}, depends on that of the \C{} function
of the same name, but in any case, this is the function to use for
benchmarking\index{benchmarking} Python or timing algorithms.
\end{funcdesc}

\begin{funcdesc}{ctime}{secs}
Convert a time expressed in seconds since the epoch to a string
representing local time.  \code{ctime(\var{secs})} is equivalent to
\code{asctime(localtime(\var{secs}))}.
\end{funcdesc}

\begin{datadesc}{daylight}
Nonzero if a DST timezone is defined.
\end{datadesc}

\begin{funcdesc}{gmtime}{secs}
Convert a time expressed in seconds since the epoch to a time tuple
in UTC in which the dst flag is always zero.  Fractions of a second are
ignored.  See above for a description of the tuple lay-out.
\end{funcdesc}

\begin{funcdesc}{localtime}{secs}
Like \function{gmtime()} but converts to local time.  The dst flag is
set to \code{1} when DST applies to the given time.
\end{funcdesc}

\begin{funcdesc}{mktime}{tuple}
This is the inverse function of \code{localtime}.  Its argument is the
full 9-tuple (since the dst flag is needed --- pass \code{-1} as the
dst flag if it is unknown) which expresses the time
in \emph{local} time, not UTC.  It returns a floating
point number, for compatibility with \function{time()}.  If the input
value cannot be represented as a valid time, \exception{OverflowError}
is raised.
\end{funcdesc}

\begin{funcdesc}{sleep}{secs}
Suspend execution for the given number of seconds.  The argument may
be a floating point number to indicate a more precise sleep time.
\end{funcdesc}

\begin{funcdesc}{strftime}{format, tuple}
Convert a tuple representing a time as returned by \code{gmtime()} or
\code{localtime()} to a string as specified by the format argument.

The following directives, shown without the optional field width and
precision specification, are replaced by the indicated characters:

\begin{tableii}{c|p{24em}}{code}{Directive}{Meaning}
  \lineii{\%a}{Locale's abbreviated weekday name.}
  \lineii{\%A}{Locale's full weekday name.}
  \lineii{\%b}{Locale's abbreviated month name.}
  \lineii{\%B}{Locale's full month name.}
  \lineii{\%c}{Locale's appropriate date and time representation.}
  \lineii{\%d}{Day of the month as a decimal number [01,31].}
  \lineii{\%H}{Hour (24-hour clock) as a decimal number [00,23].}
  \lineii{\%I}{Hour (12-hour clock) as a decimal number [01,12].}
  \lineii{\%j}{Day of the year as a decimal number [001,366].}
  \lineii{\%m}{Month as a decimal number [01,12].}
  \lineii{\%M}{Minute as a decimal number [00,59].}
  \lineii{\%p}{Locale's equivalent of either AM or PM.}
  \lineii{\%S}{Second as a decimal number [00,61].}
  \lineii{\%U}{Week number of the year (Sunday as the first day of the
               week) as a decimal number [00,53].  All days in a new year
               preceding the first Sunday are considered to be in week 0.}
  \lineii{\%w}{Weekday as a decimal number [0(Sunday),6].}
  \lineii{\%W}{Week number of the year (Monday as the first day of the
               week) as a decimal number [00,53].  All days in a new year
               preceding the first Sunday are considered to be in week 0.}
  \lineii{\%x}{Locale's appropriate date representation.}
  \lineii{\%X}{Locale's appropriate time representation.}
  \lineii{\%y}{Year without century as a decimal number [00,99].}
  \lineii{\%Y}{Year with century as a decimal number.}
  \lineii{\%Z}{Time zone name (or by no characters if no time zone exists).}
  \lineii{\%\%}{\%}
\end{tableii}

Additional directives may be supported on certain platforms, but
only the ones listed here have a meaning standardized by ANSI C.

On some platforms, an optional field width and precision
specification can immediately follow the initial \code{\%} of a
directive in the following order; this is also not portable.
The field width is normally 2 except for \code{\%j} where it is 3.

\end{funcdesc}

\begin{funcdesc}{strptime}{string\optional{, format}}
Parse a string representing a time according to a format.  The return 
value is a tuple as returned by \code{gmtime()} or \code{localtime()}.
The format uses the same directives as those used by
\code{strftime()}; it defaults to \code{"\%a \%b \%d \%H:\%M:\%S \%Y"}
which matches the formatting returned by \code{ctime()}.  The same
platform caveats apply; see the local Unix documentation for
restrictions or additional supported directives.  This function may
not be defined on all platforms.

\end{funcdesc}

\begin{funcdesc}{time}{}
Return the time as a floating point number expressed in seconds since
the epoch, in UTC.  Note that even though the time is always returned
as a floating point number, not all systems provide time with a better
precision than 1 second.
\end{funcdesc}

\begin{datadesc}{timezone}
The offset of the local (non-DST) timezone, in seconds west of the 0th
meridian (i.e. negative in most of Western Europe, positive in the US,
zero in the UK).
\end{datadesc}

\begin{datadesc}{tzname}
A tuple of two strings: the first is the name of the local non-DST
timezone, the second is the name of the local DST timezone.  If no DST
timezone is defined, the second string should not be used.
\end{datadesc}

