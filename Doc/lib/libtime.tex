\section{\module{time} ---
         Time access and conversions}

\declaremodule{builtin}{time}
\modulesynopsis{Time access and conversions.}


This module provides various time-related functions.
It is always available, but not all functions are available
on all platforms.

An explanation of some terminology and conventions is in order.

\begin{itemize}

\item
The \dfn{epoch}\index{epoch} is the point where the time starts.  On
January 1st of that year, at 0 hours, the ``time since the epoch'' is
zero.  For \UNIX{}, the epoch is 1970.  To find out what the epoch is,
look at \code{gmtime(0)}.

\item
The functions in this module do not handle dates and times before the
epoch or far in the future.  The cut-off point in the future is
determined by the C library; for \UNIX{}, it is typically in
2038\index{Year 2038}.

\item
\strong{Year 2000 (Y2K) issues}:\index{Year 2000}\index{Y2K}  Python
depends on the platform's C library, which generally doesn't have year
2000 issues, since all dates and times are represented internally as
seconds since the epoch.  Functions accepting a time tuple (see below)
generally require a 4-digit year.  For backward compatibility, 2-digit
years are supported if the module variable \code{accept2dyear} is a
non-zero integer; this variable is initialized to \code{1} unless the
environment variable \envvar{PYTHONY2K} is set to a non-empty string,
in which case it is initialized to \code{0}.  Thus, you can set
\envvar{PYTHONY2K} to a non-empty string in the environment to require 4-digit
years for all year input.  When 2-digit years are accepted, they are
converted according to the \POSIX{} or X/Open standard: values 69-99
are mapped to 1969-1999, and values 0--68 are mapped to 2000--2068.
Values 100--1899 are always illegal.  Note that this is new as of
Python 1.5.2(a2); earlier versions, up to Python 1.5.1 and 1.5.2a1,
would add 1900 to year values below 1900.

\item
UTC\index{UTC} is Coordinated Universal Time\index{Coordinated
Universal Time} (formerly known as Greenwich Mean
Time,\index{Greenwich Mean Time} or GMT).  The acronym UTC is not a
mistake but a compromise between English and French.

\item
DST is Daylight Saving Time,\index{Daylight Saving Time} an adjustment
of the timezone by (usually) one hour during part of the year.  DST
rules are magic (determined by local law) and can change from year to
year.  The C library has a table containing the local rules (often it
is read from a system file for flexibility) and is the only source of
True Wisdom in this respect.

\item
The precision of the various real-time functions may be less than
suggested by the units in which their value or argument is expressed.
E.g.\ on most \UNIX{} systems, the clock ``ticks'' only 50 or 100 times a
second, and on the Mac, times are only accurate to whole seconds.

\item
On the other hand, the precision of \function{time()} and
\function{sleep()} is better than their \UNIX{} equivalents: times are
expressed as floating point numbers, \function{time()} returns the
most accurate time available (using \UNIX{} \cfunction{gettimeofday()}
where available), and \function{sleep()} will accept a time with a
nonzero fraction (\UNIX{} \cfunction{select()} is used to implement
this, where available).

\item

The time tuple as returned by \function{gmtime()},
\function{localtime()}, and \function{strptime()}, and accepted by
\function{asctime()}, \function{mktime()} and \function{strftime()},
is a tuple of 9 integers:

\begin{tableiii}{r|l|l}{textrm}{Index}{Field}{Values}
  \lineiii{0}{year}{(e.g.\ 1993)}
  \lineiii{1}{month}{range [1,12]}
  \lineiii{2}{day}{range [1,31]}
  \lineiii{3}{hour}{range [0,23]}
  \lineiii{4}{minute}{range [0,59]}
  \lineiii{5}{second}{range [0,61]; see \strong{(1)} in \function{strftime()} description}
  \lineiii{6}{weekday}{range [0,6], monday is 0}
  \lineiii{7}{Julian day}{range [1,366]}
  \lineiii{8}{daylight savings flag}{0, 1 or -1; see below}
\end{tableiii}

Note that unlike the C structure, the month value is a
range of 1-12, not 0-11.  A year value will be handled as described
under ``Year 2000 (Y2K) issues'' above.  A \code{-1} argument as
daylight savings flag, passed to \function{mktime()} will usually
result in the correct daylight savings state to be filled in.

\end{itemize}

The module defines the following functions and data items:


\begin{datadesc}{accept2dyear}
Boolean value indicating whether two-digit year values will be
accepted.  This is true by default, but will be set to false if the
environment variable \envvar{PYTHONY2K} has been set to a non-empty
string.  It may also be modified at run time.
\end{datadesc}

\begin{datadesc}{altzone}
The offset of the local DST timezone, in seconds west of UTC, if one
is defined.  Negative if the local DST timezone is east of the 0th
meridian (as in Western Europe, including the UK).  Only use this if
\code{daylight} is nonzero.
\end{datadesc}

\begin{funcdesc}{asctime}{tuple}
Convert a tuple representing a time as returned by \function{gmtime()}
or \function{localtime()} to a 24-character string of the following form:
\code{'Sun Jun 20 23:21:05 1993'}.  Note: unlike the C function of
the same name, there is no trailing newline.
\end{funcdesc}

\begin{funcdesc}{clock}{}
Return the current CPU time as a floating point number expressed in
seconds.  The precision, and in fact the very definiton of the meaning
of ``CPU time''\index{CPU time}, depends on that of the C function
of the same name, but in any case, this is the function to use for
benchmarking\index{benchmarking} Python or timing algorithms.
\end{funcdesc}

\begin{funcdesc}{ctime}{secs}
Convert a time expressed in seconds since the epoch to a string
representing local time.  \code{ctime(\var{secs})} is equivalent to
\code{asctime(localtime(\var{secs}))}.
\end{funcdesc}

\begin{datadesc}{daylight}
Nonzero if a DST timezone is defined.
\end{datadesc}

\begin{funcdesc}{gmtime}{secs}
Convert a time expressed in seconds since the epoch to a time tuple
in UTC in which the dst flag is always zero.  Fractions of a second are
ignored.  See above for a description of the tuple lay-out.
\end{funcdesc}

\begin{funcdesc}{localtime}{secs}
Like \function{gmtime()} but converts to local time.  The dst flag is
set to \code{1} when DST applies to the given time.
\end{funcdesc}

\begin{funcdesc}{mktime}{tuple}
This is the inverse function of \function{localtime()}.  Its argument
is the full 9-tuple (since the dst flag is needed --- pass \code{-1}
as the dst flag if it is unknown) which expresses the time in
\emph{local} time, not UTC.  It returns a floating point number, for
compatibility with \function{time()}.  If the input value cannot be
represented as a valid time, \exception{OverflowError} is raised.
\end{funcdesc}

\begin{funcdesc}{sleep}{secs}
Suspend execution for the given number of seconds.  The argument may
be a floating point number to indicate a more precise sleep time.
The actual suspension time may be less than that requested because any
caught signal will terminate the \function{sleep()} following
execution of that signal's catching routine.  Also, the suspension
time may be longer than requested by an arbitrary amount because of
the scheduling of other activity in the system.
\end{funcdesc}

\begin{funcdesc}{strftime}{format, tuple}
Convert a tuple representing a time as returned by \function{gmtime()}
or \function{localtime()} to a string as specified by the \var{format}
argument.  \var{format} must be a string.

The following directives can be embedded in the \var{format} string.
They are shown without the optional field width and precision
specification, and are replaced by the indicated characters in the
\function{strftime()} result:

\begin{tableiii}{c|p{24em}|c}{code}{Directive}{Meaning}{Notes}
  \lineiii{\%a}{Locale's abbreviated weekday name.}{}
  \lineiii{\%A}{Locale's full weekday name.}{}
  \lineiii{\%b}{Locale's abbreviated month name.}{}
  \lineiii{\%B}{Locale's full month name.}{}
  \lineiii{\%c}{Locale's appropriate date and time representation.}{}
  \lineiii{\%d}{Day of the month as a decimal number [01,31].}{}
  \lineiii{\%H}{Hour (24-hour clock) as a decimal number [00,23].}{}
  \lineiii{\%I}{Hour (12-hour clock) as a decimal number [01,12].}{}
  \lineiii{\%j}{Day of the year as a decimal number [001,366].}{}
  \lineiii{\%m}{Month as a decimal number [01,12].}{}
  \lineiii{\%M}{Minute as a decimal number [00,59].}{}
  \lineiii{\%p}{Locale's equivalent of either AM or PM.}{}
  \lineiii{\%S}{Second as a decimal number [00,61].}{(1)}
  \lineiii{\%U}{Week number of the year (Sunday as the first day of the
                week) as a decimal number [00,53].  All days in a new year
                preceding the first Sunday are considered to be in week 0.}{}
  \lineiii{\%w}{Weekday as a decimal number [0(Sunday),6].}{}
  \lineiii{\%W}{Week number of the year (Monday as the first day of the
                week) as a decimal number [00,53].  All days in a new year
                preceding the first Sunday are considered to be in week 0.}{}
  \lineiii{\%x}{Locale's appropriate date representation.}{}
  \lineiii{\%X}{Locale's appropriate time representation.}{}
  \lineiii{\%y}{Year without century as a decimal number [00,99].}{}
  \lineiii{\%Y}{Year with century as a decimal number.}{}
  \lineiii{\%Z}{Time zone name (or by no characters if no time zone exists).}{}
  \lineiii{\%\%}{A literal \character{\%} character.}{}
\end{tableiii}

\noindent
Notes:

\begin{description}
  \item[(1)]
    The range really is \code{0} to \code{61}; this accounts for leap
    seconds and the (very rare) double leap seconds.
\end{description}

Additional directives may be supported on certain platforms, but
only the ones listed here have a meaning standardized by ANSI C.

On some platforms, an optional field width and precision
specification can immediately follow the initial \character{\%} of a
directive in the following order; this is also not portable.
The field width is normally 2 except for \code{\%j} where it is 3.
\end{funcdesc}

\begin{funcdesc}{strptime}{string\optional{, format}}
Parse a string representing a time according to a format.  The return 
value is a tuple as returned by \function{gmtime()} or
\function{localtime()}.  The \var{format} parameter uses the same
directives as those used by \function{strftime()}; it defaults to
\code{"\%a \%b \%d \%H:\%M:\%S \%Y"} which matches the formatting
returned by \function{ctime()}.  The same platform caveats apply; see
the local \UNIX{} documentation for restrictions or additional
supported directives.  If \var{string} cannot be parsed according to
\var{format}, \exception{ValueError} is raised.

Availability: Most modern \UNIX{} systems.
\end{funcdesc}

\begin{funcdesc}{time}{}
Return the time as a floating point number expressed in seconds since
the epoch, in UTC.  Note that even though the time is always returned
as a floating point number, not all systems provide time with a better
precision than 1 second.
\end{funcdesc}

\begin{datadesc}{timezone}
The offset of the local (non-DST) timezone, in seconds west of UTC
(i.e. negative in most of Western Europe, positive in the US, zero in
the UK).
\end{datadesc}

\begin{datadesc}{tzname}
A tuple of two strings: the first is the name of the local non-DST
timezone, the second is the name of the local DST timezone.  If no DST
timezone is defined, the second string should not be used.
\end{datadesc}


\begin{seealso}
  \seemodule{locale}{Internationalization services.  The locale
                     settings can affect the return values for some of 
                     the functions in the \module{time} module.}
\end{seealso}
