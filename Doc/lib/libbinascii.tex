\section{Built-in Module \sectcode{binascii}}
\label{module-binascii}
\bimodindex{binascii}

The binascii module contains a number of methods to convert between
binary and various ascii-encoded binary representations. Normally, you
will not use these modules directly but use wrapper modules like
\var{uu} or \var{hexbin} in stead, this module solely exists because
bit-manipuation of large amounts of data is slow in python.

The \code{binascii} module defines the following functions:

\setindexsubitem{(in module binascii)}

\begin{funcdesc}{a2b_uu}{string}
Convert a single line of uuencoded data back to binary and return the
binary data. Lines normally contain 45 (binary) bytes, except for the
last line. Line data may be followed by whitespace.
\end{funcdesc}

\begin{funcdesc}{b2a_uu}{data}
Convert binary data to a line of ascii characters, the return value is
the converted line, including a newline char. The length of \var{data}
should be at most 45.
\end{funcdesc}

\begin{funcdesc}{a2b_base64}{string}
Convert a block of base64 data back to binary and return the
binary data. More than one line may be passed at a time.
\end{funcdesc}

\begin{funcdesc}{b2a_base64}{data}
Convert binary data to a line of ascii characters in base64 coding.
The return value is the converted line, including a newline char.
The length of \var{data} should be at most 57 to adhere to the base64
standard.
\end{funcdesc}

\begin{funcdesc}{a2b_hqx}{string}
Convert binhex4 formatted ascii data to binary, without doing
rle-decompression. The string should contain a complete number of
binary bytes, or (in case of the last portion of the binhex4 data)
have the remaining bits zero.
\end{funcdesc}

\begin{funcdesc}{rledecode_hqx}{data}
Perform RLE-decompression on the data, as per the binhex4
standard. The algorithm uses \code{0x90} after a byte as a repeat
indicator, followed by a count. A count of \code{0} specifies a byte
value of \code{0x90}. The routine returns the decompressed data,
unless data input data ends in an orphaned repeat indicator, in which
case the \var{Incomplete} exception is raised.
\end{funcdesc}

\begin{funcdesc}{rlecode_hqx}{data}
Perform binhex4 style RLE-compression on \var{data} and return the
result.
\end{funcdesc}

\begin{funcdesc}{b2a_hqx}{data}
Perform hexbin4 binary-to-ascii translation and return the resulting
string. The argument should already be rle-coded, and have a length
divisible by 3 (except possibly the last fragment).
\end{funcdesc}

\begin{funcdesc}{crc_hqx}{data, crc}
Compute the binhex4 crc value of \var{data}, starting with an initial
\var{crc} and returning the result.
\end{funcdesc}
 
\begin{excdesc}{Error}
Exception raised on errors. These are usually programming errors.
\end{excdesc}

\begin{excdesc}{Incomplete}
Exception raised on incomplete data. These are usually not programming
errors, but handled by reading a little more data and trying again.
\end{excdesc}
