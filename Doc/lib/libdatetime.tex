\section{\module{datetime} ---
         Basic date and time types}

\declaremodule{builtin}{datetime}
\modulesynopsis{Basic date and time types.}
\moduleauthor{Tim Peters}{tim@zope.com}
\sectionauthor{Tim Peters}{tim@zope.com}
\sectionauthor{A.M. Kuchling}{amk@amk.ca}

\newcommand{\Naive}{Na\"ive}
\newcommand{\naive}{na\"ive}

The \module{datetime} module supplies classes for manipulating dates
and times in both simple and complex ways.  While date and time
arithmetic is supported, the focus of the implementation is on
efficient field extraction, for output formatting and manipulation.

There are two kinds of date and time objects: ``\naive'' and ``aware''.
This distinction refers to whether the object has any notion of time
zone, daylight savings time, or other kind of algorithmic or political
time adjustment.  Whether a \naive\ \class{datetime} object represents
Coordinated Universal Time (UTC), local time, or time in some other
timezone is purely up to the program, just like it's up to the program
whether a particular number represents meters, miles, or mass.  \Naive\
\class{datetime} objects are easy to understand and to work with, at
the cost of ignoring some aspects of reality.

For applications requiring more, ``aware'' \class{datetime} subclasses add an
optional time zone information object to the basic \naive\ classes.
These \class{tzinfo} objects capture information about the offset from
UTC time, the time zone name, and whether Daylight Savings Time is in
effect.  Note that no concrete \class{tzinfo} classes are supplied by
the \module{datetime} module.  Instead, they provide a framework for
incorporating the level of detail an app may require.  The rules for
time adjustment across the world are more political than rational, and
there is no standard suitable for every app.

The \module{datetime} module exports the following constants:

\begin{datadesc}{MINYEAR}
  The smallest year number allowed in a \class{date},
  \class{datetime}, or \class{datetimetz} object.  \constant{MINYEAR}
  is \code{1}.
\end{datadesc}

\begin{datadesc}{MAXYEAR}
  The largest year number allowed in a \class{date}, \class{datetime},
  or \class{datetimetz} object.  \constant{MAXYEAR} is \code{9999}.
\end{datadesc}


\subsection{Available Types}

\begin{classdesc*}{date}
  An idealized \naive\ date, assuming the current Gregorian calendar
  always was, and always will be, in effect.
  Attributes: \member{year}, \member{month}, and \member{day}.
\end{classdesc*}

\begin{classdesc*}{time}
  An idealized \naive\ time, independent of any particular day, assuming
  that every day has exactly 24*60*60 seconds (there is no notion
  of "leap seconds" here).
  Attributes: \member{hour}, \member{minute}, \member{second}, and
              \member{microsecond}
\end{classdesc*}

\begin{classdesc*}{datetime}
  A combination of a \naive\ date and a \naive\ time.
  Attributes: \member{year}, \member{month}, \member{day},
              \member{hour}, \member{minute}, \member{second},
              and \member{microsecond}.
\end{classdesc*}

\begin{classdesc*}{timedelta}
  A duration, expressing the difference between two \class{date},
  \class{time}, or \class{datetime} instances, to microsecond
  resolution.
\end{classdesc*}

\begin{classdesc*}{tzinfo}
  An abstract base class for time zone information objects.  These
  are used by the  \class{datetimetz}  and \class{timetz} classes to
  provided a customizable notion of time adjustment (for example, to
  account for time zone and/or daylight savings time).
\end{classdesc*}

\begin{classdesc*}{timetz}
  An aware subclass of \class{time}, supporting a customizable notion of
  time adjustment.
\end{classdesc*}

\begin{classdesc*}{datetimetz}
  An aware subclass of \class{datetime}, supporting a customizable notion of
  time adjustment.
\end{classdesc*}

Objects of these types are immutable.

Objects of the \class{date}, \class{datetime}, and \class{time} types
are always \naive.

An object \code{D} of type \class{timetz} or \class{datetimetz} may be
\naive\ or aware.  \code{D} is aware if \code{D.tzinfo} is not
\code{None}, and \code{D.tzinfo.utcoffset(D)} does not return
\code{None}.  If \code{D.tzinfo} is \code{None}, or if \code{D.tzinfo}
is not \code{None} but \code{D.tzinfo.utcoffset(D)} returns
\code{None}, \code{D} is \naive.

The distinction between \naive\ and aware doesn't apply to
\code{timedelta} objects.

Subclass relationships:

\begin{verbatim}
object
    timedelta
    tzinfo
    time
        timetz
    date
        datetime
            datetimetz
\end{verbatim}

\subsection{\class{timedelta} \label{datetime-timedelta}}

A \class{timedelta} object represents a duration, the difference
between two dates or times.

Constructor:

    timedelta(days=0, seconds=0, microseconds=0,
              \# The following should only be used as keyword args:
              milliseconds=0, minutes=0, hours=0, weeks=0)

    All arguments are optional.  Arguments may be ints, longs, or floats,
    and may be positive or negative.

    Only days, seconds and microseconds are stored internally.  Arguments
    are converted to those units:

        A millisecond is converted 1000 microseconds.
        A minute is converted to 60 seconds.
        An hour is converted to 3600 seconds.
        A week is converted to 7 days.

    and days, seconds and microseconds are then normalized so that the
    representation is unique, with

        0 <= microseconds < 1000000
        0 <= seconds < 3600*24 (the number of seconds in one day)
        -999999999 <= days <= 999999999

    If any argument is a float, and there are fractional microseconds,
    the fractional microseconds left over from all arguments are combined
    and their sum is rounded to the nearest microsecond.  If no
    argument is a flost, the conversion and normalization processes
    are exact (no information is lost).

    If the normalized value of days lies outside the indicated range,
    \exception{OverflowError} is raised.

    Note that normalization of negative values may be surprising at first.
    For example,

\begin{verbatim}
>>> d = timedelta(microseconds=-1)
>>> (d.days, d.seconds, d.microseconds)
(-1, 86399, 999999)
\end{verbatim}


Class attributes:

    .min
        The most negative timedelta object, timedelta(-999999999).

    .max
        The most positive timedelta object,
        timedelta(days=999999999, hours=23, minutes=59, seconds=59,
                  microseconds=999999)

    .resolution
        The smallest possible difference between non-equal timedelta
        objects, \code{timedelta(microseconds=1)}.

    Note that, because of normalization, timedelta.max > -timedelta.min.
    -timedelta.max is not representable as a timedelta object.

Instance attributes (read-only):

    .days           between -999999999 and 999999999 inclusive
    .seconds        between 0 and 86399 inclusive
    .microseconds   between 0 and 999999 inclusive

Supported operations:

\begin{itemize}
  \item
    timedelta + timedelta -> timedelta
    This is exact, but may overflow.  After
        t1 = t2 + t3
    t1-t2 == t3 and t1-t3 == t2 are true.

  \item
    timedelta - timedelta -> timedelta
    This is exact, but may overflow.  After
        t1 = t2 - t3
     t2 == t1 + t3 is true.

  \item
    timedelta * (int or long) -> timedelta
    (int or long) * timedelta -> timedelta
    This is exact, but may overflow.  After
        t1 = t2 * i
    t1 // i == t2 is true, provided i != 0.  In general,
        t * i == t * (i-1) + t
    is true.

  \item
    timedelta // (int or long) -> timedelta
    The floor is computed and the remainder (if any) is thrown away.
    Division by 0 raises \exception{ZeroDivisionError}.

  \item
    certain additions and subtractions with date, datetime, and datimetz
    objects (see below)

  \item
    +timedelta -> timedelta
    Returns a timedelta object with the same value.

  \item
    -timedelta -> timedelta
    -t is equivalent to timedelta(-t.days, -t.seconds, -t.microseconds),
    and to t*-1.  This is exact, but may overflow (for example,
    -timedelta.max is not representable as a timedelta object).

  \item
    abs(timedelta) -> timedelta
    abs(t) is equivalent to +t when t.days >= 0, and to -t when
    t.days < 0.  This is exact, and cannot overflow.

  \item
    comparison of timedelta to timedelta; the timedelta representing
    the smaller duration is considered to be the smaller timedelta

  \item
    hash, use as dict key

  \item
    efficient pickling

  \item
    in Boolean contexts, a timedelta object is considred to be true
    if and only if it isn't equal to \code{timedelta(0)}
\end{itemize}


\subsection{\class{date} \label{datetime-date}}

A date object represents a date (year, month and day) in an idealized
calendar, the current Gregorian calendar indefinitely extended in both
directions.  January 1 of year 1 is called day number 1, January 2 of year
1 is called day number 2, and so on.  This matches the definition of the
"proleptic Gregorian" calendar in Dershowitz and Reingold's book
"Calendrical Calculations", where it's the base calendar for all
computations.  See the book for algorithms for converting between
proleptic Gregorian ordinals and many other calendar systems.

Constructor:

    date(year, month, day)

    All arguments are required.  Arguments may be ints or longs, in the
    following ranges:

        MINYEAR <= year <= MAXYEAR
        1 <= month <= 12
        1 <= day <= number of days in the given month and year

    If an argument outside those ranges is given,
    \exception{ValueError} is raised.

Other constructors (class methods):

  - today()
    Return the current local date.  This is equivalent to
    date.fromtimestamp(time.time()).

  - fromtimestamp(timestamp)
    Return the local date corresponding to the POSIX timestamp, such
    as is returned by \function{time.time()}.  This may raise
    \exception{ValueError}, if the timestamp is out of the range of
    values supported by the platform C \cfunction{localtime()}
    function.  It's common for this to be restricted to years in 1970
    through 2038.

  - fromordinal(ordinal)
    Return the date corresponding to the proleptic Gregorian ordinal,
    where January 1 of year 1 has ordinal 1.  \exception{ValueError}
    is raised unless 1 <= ordinal <= date.max.toordinal().  For any
    date d, date.fromordinal(d.toordinal()) == d.

Class attributes:

    .min
        The earliest representable date, \code{date(MINYEAR, 1, 1)}.

    .max
        The latest representable date, \code{date(MAXYEAR, 12, 31)}.

    .resolution
        The smallest possible difference between non-equal date
        objects, \code{timedelta(days=1)}.

Instance attributes (read-only):

    .year           between \constant{MINYEAR} and \constant{MAXYEAR} inclusive
    .month          between 1 and 12 inclusive
    .day            between 1 and the number of days in the given month
                    of the given year

Supported operations:

\begin{itemize}
  \item
    date1 + timedelta -> date2
    timedelta + date1 -> date2
    date2 is timedelta.days days removed from the date1, moving forward
    in time if timedelta.days > 0, or backward if timedetla.days < 0.
    date2 - date1 == timedelta.days after.  timedelta.seconds and
    timedelta.microseconds are ignored.  \exception{OverflowError} is
    raised if date2.year would be smaller than \constant{MINYEAR} or
    larger than \constant{MAXYEAR}.

  \item
    date1 - timedelta -> date2
    Computes the date2 such that date2 + timedelta == date1.  This
    isn't quite equivalent to date1 + (-timedelta), because -timedelta
    in isolation can overflow in cases where date1 - timedelta does
    not.  timedelta.seconds and timedelta.microseconds are ignored.

  \item
    date1 - date2 -> timedelta
    This is exact, and cannot overflow.  timedelta.seconds and
    timedelta.microseconds are 0, and date2 + timedelta == date1
    after.

  \item
    comparison of date to date, where date1 is considered less than
    date2 when date1 precedes date2 in time.  In other words,
    date1 < date2 if and only if date1.toordinal() < date2.toordinal().

  \item
    hash, use as dict key

  \item
    efficient pickling

  \item
    in Boolean contexts, all date objects are considered to be true
\end{itemize}

Instance methods:

  - replace(year=, month=, day=)
    Return a date with the same value, except for those fields given
    new values by whichever keyword arguments are specified.  For
    example, if \code{d == date(2002, 12, 31)}, then
    \code{d.replace(day=26) == date(2000, 12, 26)}.

  - timetuple()
    Return a 9-element tuple of the form returned by
    \function{time.localtime()}.  The hours, minutes and seconds are
    0, and the DST flag is -1.
    d.timetuple() is equivalent to
        (d.year, d.month, d.day,
         0, 0, 0,  \# h, m, s
         d.weekday(),  \# 0 is Monday
         d.toordinal() - date(d.year, 1, 1).toordinal() + 1, \# day of year
         -1)

  - toordinal()
    Return the proleptic Gregorian ordinal of the date, where January 1
    of year 1 has ordinal 1.  For any date object \var{d},
    \code{date.fromordinal(\var{d}.toordinal()) == \var{d}}.

  - weekday()
    Return the day of the week as an integer, where Monday is 0 and
    Sunday is 6.  For example, date(2002, 12, 4).weekday() == 2, a
    Wednesday.
    See also \method{isoweekday()}.

  - isoweekday()
    Return the day of the week as an integer, where Monday is 1 and
    Sunday is 7.  For example, date(2002, 12, 4).isoweekday() == 3, a
    Wednesday.
    See also \method{weekday()}, \method{isocalendar()}.

  - isocalendar()
    Return a 3-tuple, (ISO year, ISO week number, ISO weekday).

    The ISO calendar is a widely used variant of the Gregorian calendar.
    See \url{http://www.phys.uu.nl/~vgent/calendar/isocalendar.htm}
    for a good explanation.

    The ISO year consists of 52 or 53 full weeks, and where a week starts
    on a Monday and ends on a Sunday.  The first week of an ISO year is
    the first (Gregorian) calendar week of a year containing a Thursday.
    This is called week number 1, and the ISO year of that Thursday is
    the same as its Gregorian year.

    For example, 2004 begins on a Thursday, so the first week of ISO
    year 2004 begins on Monday, 29 Dec 2003 and ends on Sunday, 4 Jan
    2004, so that

    date(2003, 12, 29).isocalendar() == (2004, 1, 1)
    date(2004, 1, 4).isocalendar() == (2004, 1, 7)

  - isoformat()
    Return a string representing the date in ISO 8601 format,
    'YYYY-MM-DD'.  For example,
    date(2002, 12, 4).isoformat() == '2002-12-04'.

  - __str__()
    For a date \var{d}, \code{str(\var{d})} is equivalent to
    \code{\var{d}.isoformat()}.

  - ctime()
    Return a string representing the date, for example
    date(2002, 12, 4).ctime() == 'Wed Dec  4 00:00:00 2002'.
    d.ctime() is equivalent to time.ctime(time.mktime(d.timetuple()))
    on platforms where the native C \cfunction{ctime()} function
    (which \function{time.ctime()} invokes, but which
    \method{date.ctime()} does not invoke) conforms to the C standard.

  - strftime(format)
    Return a string representing the date, controlled by an explicit
    format string.  Format codes referring to hours, minutes or seconds
    will see 0 values.
    See the section on \method{strftime()} behavior.


\subsection{\class{datetime} \label{datetime-datetime}}

A \class{datetime} object is a single object containing all the
information from a date object and a time object.  Like a date object,
\class{datetime} assumes the current Gregorian calendar extended in
both directions; like a time object, \class{datetime} assumes there
are exactly 3600*24 seconds in every day.

Constructor:

    datetime(year, month, day,
             hour=0, minute=0, second=0, microsecond=0)

    The year, month and day arguments are required.  Arguments may be ints
    or longs, in the following ranges:

        MINYEAR <= year <= MAXYEAR
        1 <= month <= 12
        1 <= day <= number of days in the given month and year
        0 <= hour < 24
        0 <= minute < 60
        0 <= second < 60
        0 <= microsecond < 1000000

    If an argument outside those ranges is given,
    \exception{ValueError} is raised.

Other constructors (class methods):

  - today()
    Return the current local datetime.  This is equivalent to
    \code{datetime.fromtimestamp(time.time())}.
    See also \method{now()}, \method{fromtimestamp()}.

  - now()
    Return the current local datetime.  This is like \method{today()},
    but, if possible, supplies more precision than can be gotten from
    going through a \function{time.time()} timestamp (for example,
    this may be possible on platforms that supply the C
    \cfunction{gettimeofday()} function).
    See also \method{today()}, \method{utcnow()}.

  - utcnow()
    Return the current UTC datetime.  This is like \method{now()}, but
    returns the current UTC date and time.
    See also \method{now()}.

  - fromtimestamp(timestamp)
    Return the local \class{datetime} corresponding to the \POSIX{}
    timestamp, such as is returned by \function{time.time()}.  This
    may raise \exception{ValueError}, if the timestamp is out of the
    range of values supported by the platform C
    \cfunction{localtime()} function.  It's common for this to be
    restricted to years in 1970 through 2038.
    See also \method{utcfromtimestamp()}.

  - utcfromtimestamp(timestamp)
    Return the UTC \class{datetime} corresponding to the \POSIX{}
    timestamp.  This may raise \exception{ValueError}, if the
    timestamp is out of the range of values supported by the platform
    C \cfunction{gmtime()} function.  It's common for this to be
    restricted to years in 1970 through 2038.
    See also \method{fromtimestamp()}.

  - fromordinal(ordinal)
    Return the \class{datetime} corresponding to the proleptic
    Gregorian ordinal, where January 1 of year 1 has ordinal 1.
    \exception{ValueError} is raised unless 1 <= ordinal <=
    datetime.max.toordinal().  The hour, minute, second and
    microsecond of the result are all 0.

  - combine(date, time)
    Return a new \class{datetime} object whose date components are
    equal to the given date object's, and whose time components are
    equal to the given time object's.  For any \class{datetime} object
    d, d == datetime.combine(d.date(), d.time()).
    If date is a \class{datetime} or \class{datetimetz} object, its
    time components are ignored.  If date is \class{datetimetz}
    object, its \member{tzinfo} component is also ignored.  If time is
    a \class{timetz} object, its \member{tzinfo} component is ignored.

Class attributes:

    .min
        The earliest representable datetime,
        datetime(MINYEAR, 1, 1).

    .max
        The latest representable datetime,
        datetime(MAXYEAR, 12, 31, 23, 59, 59, 999999).

    .resolution
        The smallest possible difference between non-equal datetime
        objects, timedelta(microseconds=1).

Instance attributes (read-only):

    .year           between \constant{MINYEAR} and \constant{MAXYEAR} inclusive
    .month          between 1 and 12 inclusive
    .day            between 1 and the number of days in the given month
                    of the given year
    .hour           in range(24)
    .minute         in range(60)
    .second         in range(60)
    .microsecond    in range(1000000)

Supported operations:

\begin{itemize}
  \item
    datetime1 + timedelta -> datetime2
    timedelta + datetime1 -> datetime2
    datetime2 is a duration of timedelta removed from datetime1, moving
    forward in time if timedelta.days > 0, or backward if
    timedelta.days < 0.  datetime2 - datetime1 == timedelta after.
    \exception{OverflowError} is raised if datetime2.year would be
    smaller than \constant{MINYEAR} or larger than \constant{MAXYEAR}.

  \item
    datetime1 - timedelta -> datetime2
    Computes the datetime2 such that datetime2 + timedelta == datetime1.
    This isn't quite equivalent to datetime1 + (-timedelta), because
    -timedelta in isolation can overflow in cases where
    datetime1 - timedelta does not.

  \item
    datetime1 - datetime2 -> timedelta
    This is exact, and cannot overflow.
    datetime2 + timedelta == datetime1 after.

  \item
    comparison of \class{datetime} to datetime, where datetime1 is
    considered less than datetime2 when datetime1 precedes datetime2
    in time.

  \item
    hash, use as dict key

  \item
    efficient pickling

  \item
    in Boolean contexts, all \class{datetime} objects are considered
    to be true
\end{itemize}

Instance methods:

  - date()
    Return date object with same year, month and day.

  - time()
    Return time object with same hour, minute, second and microsecond.

  - replace(year=, month=, day=, hour=, minute=, second=, microsecond=)
    Return a datetime with the same value, except for those fields given
    new values by whichever keyword arguments are specified.

  - astimezone(tz)
    Return a \class{datetimetz} with the same date and time fields, and
    with \member{tzinfo} member \var{tz}.  \var{tz} must be an instance
    of a \class{tzinfo} subclass.

  - timetuple()
    Return a 9-element tuple of the form returned by
    \function{time.localtime()}.
    The DST flag is -1.   \code{d.timetuple()} is equivalent to
        (d.year, d.month, d.day,
         d.hour, d.minute, d.second,
         d.weekday(),  \# 0 is Monday
         d.toordinal() - date(d.year, 1, 1).toordinal() + 1, \# day of year
         -1)

  - toordinal()
    Return the proleptic Gregorian ordinal of the date.  The same as
    \method{date.toordinal()}.

  - weekday()
    Return the day of the week as an integer, where Monday is 0 and
    Sunday is 6.  The same as \method{date.weekday()}.
    See also \method{isoweekday()}.

  - isoweekday()
    Return the day of the week as an integer, where Monday is 1 and
    Sunday is 7.  The same as \method{date.isoweekday()}.
    See also \method{weekday()}, \method{isocalendar()}.

  - isocalendar()
    Return a 3-tuple, (ISO year, ISO week number, ISO weekday).  The
    same as \method{date.isocalendar()}.

  - isoformat(sep='T')
    Return a string representing the date and time in ISO 8601 format,
        YYYY-MM-DDTHH:MM:SS.mmmmmm
    or, if self.microsecond is 0,
        YYYY-MM-DDTHH:MM:SS
    The optional argument \var{sep} (default \code{'T'}) is a
    one-character separator, placed between the date and time portions
    of the result.  For example,
        datetime(2002, 12, 4, 1, 2, 3, 4).isoformat(' ') ==
        '2002-12-04 01:02:03.000004'

  - __str__()
    For a \class{datetime} instance \var{d}, \code{str(\var{d})} is
    equivalent to \code{\var{d}.isoformat(' ')}.

  - ctime()
    Return a string representing the date, for example
    datetime(2002, 12, 4, 20, 30, 40).ctime() == 'Wed Dec  4 20:30:40 2002'.
    \code{d.ctime()} is equivalent to
    \code{time.ctime(time.mktime(d.timetuple()))} on platforms where
    the native C \cfunction{ctime()} function (which
    \function{time.ctime()} invokes, but which
    \method{datetime.ctime()} does not invoke) conforms to the C
    standard.

  - strftime(format)
    Return a string representing the date and time, controlled by an
    explicit format string.  See the section on \method{strftime()}
    behavior.


\subsection{\class{time} \label{datetime-time}}

A time object represents an idealized time of day, independent of day
and timezone.

Constructor:

    time(hour=0, minute=0, second=0, microsecond=0)

    All arguments are optional.  They may be ints or longs, in the
    following ranges:

        0 <= hour < 24
        0 <= minute < 60
        0 <= second < 60
        0 <= microsecond < 1000000

    If an argument outside those ranges is given,
    \exception{ValueError} is raised.

Class attributes:

    .min
        The earliest representable time, time(0, 0, 0, 0).

    .max
        The latest representable time, time(23, 59, 59, 999999).

    .resolution
        The smallest possible difference between non-equal time
        objects, timedelta(microseconds=1), although note that
        arithmetic on time objects is not supported.

Instance attributes (read-only):

    .hour           in range(24)
    .minute         in range(60)
    .second         in range(60)
    .microsecond    in range(1000000)

Supported operations:

\begin{itemize}
  \item
    comparison of time to time, where time1 is considered
    less than time2 when time1 precedes time2 in time.

  \item
    hash, use as dict key

  \item
    efficient pickling

  \item
    in Boolean contexts, a time object is considered to be true
    if and only if it isn't equal to time(0)
\end{itemize}

Instance methods:

  - replace(hour=, minute=, second=, microsecond=)
    Return a time with the same value, except for those fields given
    new values by whichever keyword arguments are specified.

  - isoformat()
    Return a string representing the time in ISO 8601 format,
        HH:MM:SS.mmmmmm
    or, if self.microsecond is 0
        HH:MM:SS

  - __str__()
    For a time \var{t}, \code{str(\var{t})} is equivalent to
    \code{\var{t}.isoformat()}.

  - strftime(format)
    Return a string representing the time, controlled by an explicit
    format string.  See the section on \method{strftime()} behavior.


\subsection{\class{tzinfo} \label{datetime-tzinfo}}

\class{tzinfo} is an abstract base clase, meaning that this class
should not be instantiated directly.  You need to derive a concrete
subclass, and (at least) supply implementations of the standard
\class{tzinfo} methods needed by the \class{datetime} methods you
use. The \module{datetime} module does not supply any concrete
subclasses of \class{tzinfo}.

An instance of (a concrete subclass of) \class{tzinfo} can be passed
to the constructors for \class{datetimetz} and \class{timetz} objects.
The latter objects view their fields as being in local time, and the
\class{tzinfo} object supports methods revealing offset of local time
from UTC, the name of the time zone, and DST offset, all relative to a
date or time object passed to them.

Special requirement for pickling:  A tzinfo subclass must have an
\method{__init__} method that can be called with no arguments, else it
can be pickled but possibly not unpickled again.  This is a technical
requirement that may be relaxed in the future.

A concrete subclass of \class{tzinfo} may need to implement the
following methods.  Exactly which methods are needed depends on the
uses made of aware \module{datetime} objects; if in doubt, simply
implement all of them.  The methods are called by a \class{datetimetz}
or \class{timetz} object, passing itself as the argument.  A
\class{tzinfo} subclass's methods should be prepared to accept a dt
argument of \code{None} or of type \class{timetz} or
\class{datetimetz}.

  - utcoffset(dt)
    Return offset of local time from UTC, in minutes east of UTC.  If
    local time is west of UTC, this should be negative.  Note that this
    is intended to be the total offset from UTC; for example, if a
    \class{tzinfo} object represents both time zone and DST adjustments,
    \method{utcoffset()} should return their sum.  If the UTC offset
    isn't known, return \code{None}.  Else the value returned must be
    an integer, in the range -1439 to 1439 inclusive (1440 = 24*60;
    the magnitude of the offset must be less than one day), or a
    \class{timedelta} object representing a whole number of minutes
    in the same range.

  - tzname(dt)
    Return the timezone name corresponding to the \class{datetime} represented
    by dt, as a string.  Nothing about string names is defined by the
    \module{datetime} module, and there's no requirement that it mean anything
    in particular.  For example, "GMT", "UTC", "-500", "-5:00", "EDT",
    "US/Eastern", "America/New York" are all valid replies.  Return
    \code{None} if a string name isn't known.  Note that this is a method
    rather than a fixed string primarily because some \class{tzinfo} objects
    will wish to return different names depending on the specific value
    of dt passed, especially if the \class{tzinfo} class is accounting for DST.

  - dst(dt)
    Return the DST offset, in minutes east of UTC, or \code{None} if
    DST information isn't known.  Return 0 if DST is not in effect.
    If DST is in effect, return the offset as an integer or
    \class{timedelta} object (see \method{utcoffset()} for details).
    Note that DST offset, if applicable, has
    already been added to the UTC offset returned by
    \method{utcoffset()}, so there's no need to consult \method{dst()}
    unless you're interested in displaying DST info separately.  For
    example, \method{datetimetz.timetuple()} calls its \member{tzinfo}
    member's \method{dst()} method to determine how the
    \member{tm_isdst} flag should be set.

Example \class{tzinfo} classes:

\verbatiminput{tzinfo-examples.py}


\subsection{\class{timetz} \label{datetime-timetz}}

A time object represents a (local) time of day, independent of any
particular day, and subject to adjustment via a \class{tzinfo} object.

Constructor:

    time(hour=0, minute=0, second=0, microsecond=0, tzinfo=None)

    All arguments are optional.  \var{tzinfo} may be \code{None}, or
    an instance of a \class{tzinfo} subclass.  The remaining arguments
    may be ints or longs, in the following ranges:

        0 <= hour < 24
        0 <= minute < 60
        0 <= second < 60
        0 <= microsecond < 1000000

    If an argument outside those ranges is given,
    \exception{ValueError} is raised.

Class attributes:

    .min
        The earliest representable time, timetz(0, 0, 0, 0).

    .max
        The latest representable time, timetz(23, 59, 59, 999999).

    .resolution
        The smallest possible difference between non-equal timetz
        objects, timedelta(microseconds=1), although note that
        arithmetic on \class{timetz} objects is not supported.

Instance attributes (read-only):

    .hour           in range(24)
    .minute         in range(60)
    .second         in range(60)
    .microsecond    in range(1000000)
    .tzinfo         the object passed as the tzinfo argument to the
                    \class{timetz} constructor, or \code{None} if none
                    was passed.

Supported operations:

\begin{itemize}
  \item
    comparison of \class{timetz} to \class{time} or \class{timetz},
    where \var{a} is considered less than \var{b} when \var{a} precedes
    \var{b} in time.  If one comparand is naive and the other is aware,
    \exception{TypeError} is raised.  If both comparands are aware, and
    have the same \member{tzinfo} member, the common \member{tzinfo}
    member is ignored and the base times are compared.  If both
    comparands are aware and have different \member{tzinfo} members,
    the comparands are first adjusted by subtracting their UTC offsets
    (obtained from \code{self.utcoffset()}).

  \item
    hash, use as dict key

  \item
    pickling

  \item
    in Boolean contexts, a \class{timetz} object is considered to be
    true if and only if, after converting it to minutes and
    subtracting \method{utcoffset()} (or \code{0} if that's
    \code{None}), the result is non-zero.
\end{itemize}

Instance methods:

  - replace(hour=, minute=, second=, microsecond=, tzinfo=)
    Return a timetz with the same value, except for those fields given
    new values by whichever keyword arguments are specified.  Note that
    \code{tzinfo=None} can be specified to create a naive timetz from an
    aware timetz.

  - isoformat()
    Return a string representing the time in ISO 8601 format,
        HH:MM:SS.mmmmmm
    or, if self.microsecond is 0
        HH:MM:SS
    If \method{utcoffset()} does not return \code{None}, a 6-character
    string is appended, giving the UTC offset in (signed) hours and
    minutes:
        HH:MM:SS.mmmmmm+HH:MM
    or, if self.microsecond is 0
        HH:MM:SS+HH:MM

  - __str__()
    For a \class{timetz} \var{t}, \code{str(\var{t})} is equivalent to
    \code{\var{t}.isoformat()}.

  - strftime(format)
    Return a string representing the time, controlled by an explicit
    format string.  See the section on \method{strftime()} behavior.

  - utcoffset()
    If \member{tzinfo} is \code{None}, returns \code{None}, else
    \code{tzinfo.utcoffset(self)} converted to a \class{timedelta}
    object.

  - tzname():
    If \member{tzinfo} is \code{None}, returns \code{None}, else
    \code{tzinfo.tzname(self)}.

  - dst()
    If \member{tzinfo} is \code{None}, returns \code{None}, else
    \code{tzinfo.dst(self)} converted to a \class{timedelta} object.



\subsection{ \class{datetimetz}  \label{datetime-datetimetz}}

\begin{notice}[warning]
  I think this is \emph{still} missing some methods from the
  Python implementation.
\end{notice}

A \class{datetimetz} object is a single object containing all the information
from a date object and a \class{timetz} object.

Constructor:

    datetimetz(year, month, day,
               hour=0, minute=0, second=0, microsecond=0, tzinfo=None)

    The year, month and day arguments are required.  \var{tzinfo} may
    be \code{None}, or an instance of a \class{tzinfo} subclass.  The
    remaining arguments may be ints or longs, in the following ranges:

        MINYEAR <= year <= MAXYEAR
        1 <= month <= 12
        1 <= day <= number of days in the given month and year
        0 <= hour < 24
        0 <= minute < 60
        0 <= second < 60
        0 <= microsecond < 1000000

    If an argument outside those ranges is given,
    \exception{ValueError} is raised.

Other constructors (class methods):

  - today()
    utcnow()
    utcfromtimestamp(timestamp)
    fromordinal(ordinal)

    These are the same as the \class{datetime} class methods of the
    same names, except that they construct a \class{datetimetz}
    object, with tzinfo \code{None}.

  - now([tzinfo=None])
    fromtimestamp(timestamp[, tzinfo=None])

    These are the same as the \class{datetime} class methods of the same names,
    except that they accept an additional, optional tzinfo argument, and
    construct a \class{datetimetz} object with that \class{tzinfo} object attached.

  - combine(date, time)
    This is the same as \method{datetime.combine()}, except that it constructs
    a \class{datetimetz} object, and, if the time object is of type timetz,
    the \class{datetimetz} object has the same \class{tzinfo} object as the time object.

Class attributes:

    .min
        The earliest representable datetimetz,
        datetimetz(MINYEAR, 1, 1).

    .max
        The latest representable datetime,
        datetimetz(MAXYEAR, 12, 31, 23, 59, 59, 999999).

    .resolution
        The smallest possible difference between non-equal datetimetz
        objects, timedelta(microseconds=1).

Instance attributes (read-only):

    .year           between MINYEAR and MAXYEAR inclusive
    .month          between 1 and 12 inclusive
    .day            between 1 and the number of days in the given month
                    of the given year
    .hour           in range(24)
    .minute         in range(60)
    .second         in range(60)
    .microsecond    in range(1000000)
    .tzinfo         the object passed as the \var{tzinfo} argument to
                    the \class{datetimetz} constructor, or \code{None}
                    if none was passed.

Supported operations:

\begin{itemize}
  \item
    datetimetz1 + timedelta -> datetimetz2
    timedelta + datetimetz1 -> datetimetz2

    The same as addition of \class{datetime} objects, except that
    datetimetz2.tzinfo is set to datetimetz1.tzinfo.

  \item
    datetimetz1 - timedelta -> datetimetz2

    The same as addition of \class{datetime} objects, except that
    datetimetz2.tzinfo is set to datetimetz1.tzinfo.

  \item
    aware_datetimetz1 - aware_datetimetz2 -> timedelta
    \naive\_datetimetz1 - \naive\_datetimetz2 -> timedelta
    \naive\_datetimetz1 - datetime2 -> timedelta
    datetime1 - \naive\_datetimetz2 -> timedelta

    Subtraction of a \class{datetime} or \class{datetimetz}, from a
    \class{datetime} or \class{datetimetz}, is defined only if both
    operands are \naive, or if both are aware.  If one is aware and the
    other is \naive, \exception{TypeError} is raised.

    If both are \naive, or both are aware and have the same \member{tzinfo}
    member, subtraction acts as for \class{datetime} subtraction.

    If both are aware and have different \member{tzinfo} members,
    \code{a-b} acts as if \var{a} and \var{b} were first converted to UTC
    datetimes (by subtracting \code{a.utcoffset()} minutes from \var{a},
    and \code{b.utcoffset()} minutes from \var{b}), and then doing
    \class{datetime} subtraction, except that the implementation never
    overflows.

  \item
    comparison of \class{datetimetz} to \class{datetime} or
    \class{datetimetz}, where \var{a} is considered less than \var{b}
    when \var{a} precedes \var{b} in time.  If one comparand is naive and
    the other is aware, \exception{TypeError} is raised.  If both
    comparands are aware, and have the same \member{tzinfo} member,
    the common \member{tzinfo} member is ignored and the base datetimes
    are compared.  If both comparands are aware and have different
    \member{tzinfo} members, the comparands are first adjusted by
    subtracting their UTC offsets (obtained from \code{self.utcoffset()}).

  \item
    hash, use as dict key

  \item
    efficient pickling

  \item
    in Boolean contexts, all \class{datetimetz} objects are considered to be
    true
\end{itemize}

Instance methods:

  - date()
    time()
    toordinal()
    weekday()
    isoweekday()
    isocalendar()
    ctime()
    __str__()
    strftime(format)

    These are the same as the \class{datetime} methods of the same names.

  - timetz()
    Return \class{timetz} object with same hour, minute, second, microsecond,
    and tzinfo.

  - replace(year=, month=, day=, hour=, minute=, second=, microsecond=,
            tzinfo=)
    Return a datetimetz with the same value, except for those fields given
    new values by whichever keyword arguments are specified.  Note that
    \code{tzinfo=None} can be specified to create a naive datetimetz from
    an aware datetimetz.

  - astimezone(tz)
    Return a \class{datetimetz} with new tzinfo member \var{tz}.  \var{tz}
    must be an instance of a \class{tzinfo} subclass.  If self is naive, or
    if \code(tz.utcoffset(self)} returns \code{None},
    \code{self.astimezone(tz)} is equivalent to
    \code{self.replace(tzinfo=tz)}:  a new timezone object is attached
    without any conversion of date or time fields.  If self is aware and
    \code{tz.utcoffset(self)} does not return \code{None}, the date and
    time fields are adjusted so that the result is local time in timezone
    tz, representing the same UTC time as self.  \code{self.astimezone(tz)}
    is then equivalent to
    \begin{verbatim}
        (self - (self.utcoffset() - tz.utcoffset(self)).replace(tzinfo=tz)
    \end{verbatim}
    where the result of \code{tz.uctcoffset(self)} is converted to a
    \class{timedelta} if it's an integer.

  - utcoffset()
    If \member{tzinfo} is \code{None}, returns \code{None}, else
    \code{tzinfo.utcoffset(self)} converted to a \class{timedelta}
    object.

  - tzname()
    If \member{tzinfo} is \code{None}, returns \code{None}, else
    \code{tzinfo.tzname(self)}.

  - dst()
    If \member{tzinfo} is \code{None}, returns \code{None}, else
    \code{tzinfo.dst(self)} converted to a \class{timedelta}
    object.

  - timetuple()
    Like \function{datetime.timetuple()}, but sets the
    \member{tm_isdst} flag according to the \method{dst()} method:  if
    \method{dst()} returns \code{None}, \member{tm_isdst} is set to
    \code{-1}; else if \method{dst()} returns a non-zero value,
    \member{tm_isdst} is set to \code{1}; else \code{tm_isdst} is set
    to \code{0}.

  - utctimetuple()
    If \class{datetimetz} instance \var{d} is \naive, this is the same as
    \code{\var{d}.timetuple()} except that \member{tm_isdst} is forced to 0
    regardless of what \code{d.dst()} returns.  DST is never in effect
    for a UTC time.

    If \var{d} is aware, \var{d} is normalized to UTC time, by subtracting
    \code{\var{d}.utcoffset()} minutes, and a timetuple for the
    normalized time is returned.  \member{tm_isdst} is forced to 0.
    Note that the result's \member{tm_year} field may be
    \constant{MINYEAR}-1 or \constant{MAXYEAR}+1, if \var{d}.year was
    \code{MINYEAR} or \code{MAXYEAR} and UTC adjustment spills over a
    year boundary.

  - isoformat(sep='T')
    Return a string representing the date and time in ISO 8601 format,
        YYYY-MM-DDTHH:MM:SS.mmmmmm
    or, if \member{microsecond} is 0,
        YYYY-MM-DDTHH:MM:SS

    If \method{utcoffset()} does not return \code{None}, a 6-character
    string is appended, giving the UTC offset in (signed) hours and
    minutes:
        YYYY-MM-DDTHH:MM:SS.mmmmmm+HH:MM
    or, if \member{microsecond} is 0
        YYYY-MM-DDTHH:MM:SS+HH:MM

    The optional argument \var{sep} (default \code{'T'}) is a
    one-character separator, placed between the date and time portions
    of the result.  For example,

\begin{verbatim}
>>> from datetime import *
>>> class TZ(tzinfo):
...     def utcoffset(self, dt): return -399
...
>>> datetimetz(2002, 12, 25, tzinfo=TZ()).isoformat(' ')
'2002-12-25 00:00:00-06:39'
\end{verbatim}

\code{str(\var{d})} is equivalent to \code{\var{d}.isoformat(' ')}.


\subsection{\method{strftime()} Behavior}

\class{date}, \class{datetime}, \class{datetimetz}, \class{time},
and \class{timetz} objects all support a \code{strftime(\var{format})}
method, to create a string representing the time under the control of
an explicit format string.  Broadly speaking,
\begin{verbatim}
d.strftime(fmt)
\end{verbatim}
acts like the \refmodule{time} module's
\begin{verbatim}
time.strftime(fmt, d.timetuple())
\end{verbatim}
although not all objects support a \method{timetuple()} method.

For \class{time} and \class{timetz} objects, format codes for year,
month, and day should not be used, as time objects have no such values.
\code{1900} is used for the year, and \code{0} for the month and day.

For \class{date} objects, format codes for hours, minutes, and seconds
should not be used, as date objects have no such values.  \code{0} is
used instead.

For a \naive\ object, the \code{\%z} and \code{\%Z} format codes are
replaced by empty strings.

For an aware object:

\begin{itemize}
  \item[\code{\%z}]
    \method{utcoffset()} is transformed into a 5-character string of
    the form +HHMM or -HHMM, where HH is a 2-digit string giving the
    number of UTC offset hours, and MM is a 2-digit string giving the
    number of UTC offset minutes.  For example, if
    \method{utcoffset()} returns \code{timedelta(hours=-3, minutes=-30}},
    \code{\%z} is replaced with the string \code{'-0330'}.

  \item[\code{\%Z}]
    If \method{tzname()} returns \code{None}, \code{\%Z} is replaced
    by an empty string.  Else \code{\%Z} is replaced by the returned
    value, which must be a string.
\end{itemize}

The full set of format codes supported varies across platforms,
because Python calls the platform C library's \function{strftime()}
function, and platform variations are common.  The documentation for
Python's \refmodule{time} module lists the format codes that the C
standard (1989 version) requires, and those work on all platforms
with a standard C implementation.  Note that the 1999 version of the
C standard added additional format codes.

The exact range of years for which \method{strftime()} works also
varies across platforms.  Regardless of platform, years before 1900
cannot be used.


\subsection{C API}

Struct typedefs:

    PyDateTime_Date
    PyDateTime_DateTime
    PyDateTime_DateTimeTZ
    PyDateTime_Time
    PyDateTime_TimeTZ
    PyDateTime_Delta
    PyDateTime_TZInfo

Type-check macros:

    PyDate_Check(op)
    PyDate_CheckExact(op)

    PyDateTime_Check(op)
    PyDateTime_CheckExact(op)

    PyDateTimeTZ_Check(op)
    PyDateTimeTZ_CheckExact(op)

    PyTime_Check(op)
    PyTime_CheckExact(op)

    PyTimeTZ_Check(op)
    PyTimeTZ_CheckExact(op)

    PyDelta_Check(op)
    PyDelta_CheckExact(op)

    PyTZInfo_Check(op)
    PyTZInfo_CheckExact(op

Accessor macros:

All objects are immutable, so accessors are read-only.  All macros
return ints:

    For \class{date}, \class{datetime}, and \class{datetimetz} instances:
        PyDateTime_GET_YEAR(o)
        PyDateTime_GET_MONTH(o)
        PyDateTime_GET_DAY(o)

    For \class{datetime} and \class{datetimetz} instances:
        PyDateTime_DATE_GET_HOUR(o)
        PyDateTime_DATE_GET_MINUTE(o)
        PyDateTime_DATE_GET_SECOND(o)
        PyDateTime_DATE_GET_MICROSECOND(o)

    For \class{time} and \class{timetz} instances:
        PyDateTime_TIME_GET_HOUR(o)
        PyDateTime_TIME_GET_MINUTE(o)
        PyDateTime_TIME_GET_SECOND(o)
        PyDateTime_TIME_GET_MICROSECOND(o)
