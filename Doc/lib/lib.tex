\documentclass{manual}

% NOTE: this file controls which chapters/sections of the library
% manual are actually printed.  It is easy to customize your manual
% by commenting out sections that you're not interested in.

\title{Python Library Reference}

\author{Guido van Rossum\\
	Fred L. Drake, Jr., editor}
\authoraddress{
	PythonLabs\\
	E-mail: \email{python-docs@python.org}
}

\date{\today}		% XXX update before release!
\release{2.1 alpha}		% software release, not documentation
\setshortversion{2.1}		% major.minor only for software


\makeindex			% tell \index to actually write the
				% .idx file
\makemodindex			% ... and the module index as well.


\begin{document}

\maketitle

\ifhtml
\chapter*{Front Matter\label{front}}
\fi

Copyright \copyright{} 2001-2006 Python Software Foundation.
All rights reserved.

Copyright \copyright{} 2000 BeOpen.com.
All rights reserved.

Copyright \copyright{} 1995-2000 Corporation for National Research Initiatives.
All rights reserved.

Copyright \copyright{} 1991-1995 Stichting Mathematisch Centrum.
All rights reserved.

See the end of this document for complete license and permissions
information.


\begin{abstract}

\noindent
Python is an extensible, interpreted, object-oriented programming
language.  It supports a wide range of applications, from simple text
processing scripts to interactive WWW browsers.

While the \emph{Python Reference Manual} describes the exact syntax and
semantics of the language, it does not describe the standard library
that is distributed with the language, and which greatly enhances its
immediate usability.  This library contains built-in modules (written
in C) that provide access to system functionality such as file I/O
that would otherwise be inaccessible to Python programmers, as well as
modules written in Python that provide standardized solutions for many
problems that occur in everyday programming.  Some of these modules
are explicitly designed to encourage and enhance the portability of
Python programs.

This library reference manual documents Python's standard library, as
well as many optional library modules (which may or may not be
available, depending on whether the underlying platform supports them
and on the configuration choices made at compile time).  It also
documents the standard types of the language and its built-in
functions and exceptions, many of which are not or incompletely
documented in the Reference Manual.

This manual assumes basic knowledge about the Python language.  For an
informal introduction to Python, see the \emph{Python Tutorial}; the
\emph{Python Reference Manual} remains the highest authority on
syntactic and semantic questions.  Finally, the manual entitled
\emph{Extending and Embedding the Python Interpreter} describes how to
add new extensions to Python and how to embed it in other applications.

\end{abstract}

\tableofcontents

				% Chapter title:

\chapter{Introduction}
\label{intro}

The ``Python library'' contains several different kinds of components.

It contains data types that would normally be considered part of the
``core'' of a language, such as numbers and lists.  For these types,
the Python language core defines the form of literals and places some
constraints on their semantics, but does not fully define the
semantics.  (On the other hand, the language core does define
syntactic properties like the spelling and priorities of operators.)

The library also contains built-in functions and exceptions ---
objects that can be used by all Python code without the need of an
\keyword{import} statement.  Some of these are defined by the core
language, but many are not essential for the core semantics and are
only described here.

The bulk of the library, however, consists of a collection of modules.
There are many ways to dissect this collection.  Some modules are
written in C and built in to the Python interpreter; others are
written in Python and imported in source form.  Some modules provide
interfaces that are highly specific to Python, like printing a stack
trace; some provide interfaces that are specific to particular
operating systems, like socket I/O; others provide interfaces that are
specific to a particular application domain, like the World-Wide Web.
Some modules are avaiable in all versions and ports of Python; others
are only available when the underlying system supports or requires
them; yet others are available only when a particular configuration
option was chosen at the time when Python was compiled and installed.

This manual is organized ``from the inside out'': it first describes
the built-in data types, then the built-in functions and exceptions,
and finally the modules, grouped in chapters of related modules.  The
ordering of the chapters as well as the ordering of the modules within
each chapter is roughly from most relevant to least important.

This means that if you start reading this manual from the start, and
skip to the next chapter when you get bored, you will get a reasonable
overview of the available modules and application areas that are
supported by the Python library.  Of course, you don't \emph{have} to
read it like a novel --- you can also browse the table of contents (in
front of the manual), or look for a specific function, module or term
in the index (in the back).  And finally, if you enjoy learning about
random subjects, you choose a random page number (see module
\module{rand}) and read a section or two.

Let the show begin!
		% Introduction

\chapter{Built-in Types, Exceptions and Functions}
\nodename{Built-in Objects}
\label{builtin}

Names for built-in exceptions and functions are found in a separate
symbol table.  This table is searched last when the interpreter looks
up the meaning of a name, so local and global
user-defined names can override built-in names.  Built-in types are
described together here for easy reference.\footnote{
	Most descriptions sorely lack explanations of the exceptions
	that may be raised --- this will be fixed in a future version of
	this manual.}
\indexii{built-in}{types}
\indexii{built-in}{exceptions}
\indexii{built-in}{functions}
\index{symbol table}

The tables in this chapter document the priorities of operators by
listing them in order of ascending priority (within a table) and
grouping operators that have the same priority in the same box.
Binary operators of the same priority group from left to right.
(Unary operators group from right to left, but there you have no real
choice.)  See Chapter 5 of the \emph{Python Reference Manual} for the
complete picture on operator priorities.
			% Built-in Types, Exceptions and Functions
\section{Built-in Types \label{types}}

The following sections describe the standard types that are built into
the interpreter.  These are the numeric types, sequence types, and
several others, including types themselves.  There is no explicit
Boolean type; use integers instead.
\indexii{built-in}{types}
\indexii{Boolean}{type}

Some operations are supported by several object types; in particular,
all objects can be compared, tested for truth value, and converted to
a string (with the \code{`\textrm{\ldots}`} notation).  The latter
conversion is implicitly used when an object is written by the
\keyword{print}\stindex{print} statement.


\subsection{Truth Value Testing \label{truth}}

Any object can be tested for truth value, for use in an \keyword{if} or
\keyword{while} condition or as operand of the Boolean operations below.
The following values are considered false:
\stindex{if}
\stindex{while}
\indexii{truth}{value}
\indexii{Boolean}{operations}
\index{false}

\begin{itemize}

\item	\code{None}
	\withsubitem{(Built-in object)}{\ttindex{None}}

\item	zero of any numeric type, for example, \code{0}, \code{0L},
        \code{0.0}, \code{0j}.

\item	any empty sequence, for example, \code{''}, \code{()}, \code{[]}.

\item	any empty mapping, for example, \code{\{\}}.

\item	instances of user-defined classes, if the class defines a
	\method{__nonzero__()} or \method{__len__()} method, when that
	method returns zero.\footnote{Additional information on these
special methods may be found in the \citetitle[../ref/ref.html]{Python
Reference Manual}.}

\end{itemize}

All other values are considered true --- so objects of many types are
always true.
\index{true}

Operations and built-in functions that have a Boolean result always
return \code{0} for false and \code{1} for true, unless otherwise
stated.  (Important exception: the Boolean operations
\samp{or}\opindex{or} and \samp{and}\opindex{and} always return one of
their operands.)


\subsection{Boolean Operations \label{boolean}}

These are the Boolean operations, ordered by ascending priority:
\indexii{Boolean}{operations}

\begin{tableiii}{c|l|c}{code}{Operation}{Result}{Notes}
  \lineiii{\var{x} or \var{y}}
          {if \var{x} is false, then \var{y}, else \var{x}}{(1)}
  \lineiii{\var{x} and \var{y}}
          {if \var{x} is false, then \var{x}, else \var{y}}{(1)}
  \hline
  \lineiii{not \var{x}}
          {if \var{x} is false, then \code{1}, else \code{0}}{(2)}
\end{tableiii}
\opindex{and}
\opindex{or}
\opindex{not}

\noindent
Notes:

\begin{description}

\item[(1)]
These only evaluate their second argument if needed for their outcome.

\item[(2)]
\samp{not} has a lower priority than non-Boolean operators, so
\code{not \var{a} == \var{b}} is interpreted as \code{not (\var{a} ==
\var{b})}, and \code{\var{a} == not \var{b}} is a syntax error.

\end{description}


\subsection{Comparisons \label{comparisons}}

Comparison operations are supported by all objects.  They all have the
same priority (which is higher than that of the Boolean operations).
Comparisons can be chained arbitrarily; for example, \code{\var{x} <
\var{y} <= \var{z}} is equivalent to \code{\var{x} < \var{y} and
\var{y} <= \var{z}}, except that \var{y} is evaluated only once (but
in both cases \var{z} is not evaluated at all when \code{\var{x} <
\var{y}} is found to be false).
\indexii{chaining}{comparisons}

This table summarizes the comparison operations:

\begin{tableiii}{c|l|c}{code}{Operation}{Meaning}{Notes}
  \lineiii{<}{strictly less than}{}
  \lineiii{<=}{less than or equal}{}
  \lineiii{>}{strictly greater than}{}
  \lineiii{>=}{greater than or equal}{}
  \lineiii{==}{equal}{}
  \lineiii{!=}{not equal}{(1)}
  \lineiii{<>}{not equal}{(1)}
  \lineiii{is}{object identity}{}
  \lineiii{is not}{negated object identity}{}
\end{tableiii}
\indexii{operator}{comparison}
\opindex{==} % XXX *All* others have funny characters < ! >
\opindex{is}
\opindex{is not}

\noindent
Notes:

\begin{description}

\item[(1)]
\code{<>} and \code{!=} are alternate spellings for the same operator.
(I couldn't choose between \ABC{} and C! :-)
\index{ABC language@\ABC{} language}
\index{language!ABC@\ABC{}}
\indexii{C}{language}
\code{!=} is the preferred spelling; \code{<>} is obsolescent.

\end{description}

Objects of different types, except different numeric types, never
compare equal; such objects are ordered consistently but arbitrarily
(so that sorting a heterogeneous array yields a consistent result).
Furthermore, some types (for example, file objects) support only a
degenerate notion of comparison where any two objects of that type are
unequal.  Again, such objects are ordered arbitrarily but
consistently.
\indexii{object}{numeric}
\indexii{objects}{comparing}

Instances of a class normally compare as non-equal unless the class
\withsubitem{(instance method)}{\ttindex{__cmp__()}}
defines the \method{__cmp__()} method.  Refer to the
\citetitle[../ref/customization.html]{Python Reference Manual} for
information on the use of this method to effect object comparisons.

\strong{Implementation note:} Objects of different types except
numbers are ordered by their type names; objects of the same types
that don't support proper comparison are ordered by their address.

Two more operations with the same syntactic priority,
\samp{in}\opindex{in} and \samp{not in}\opindex{not in}, are supported
only by sequence types (below).


\subsection{Numeric Types \label{typesnumeric}}

There are four numeric types: \dfn{plain integers}, \dfn{long integers}, 
\dfn{floating point numbers}, and \dfn{complex numbers}.
Plain integers (also just called \dfn{integers})
are implemented using \ctype{long} in C, which gives them at least 32
bits of precision.  Long integers have unlimited precision.  Floating
point numbers are implemented using \ctype{double} in C.  All bets on
their precision are off unless you happen to know the machine you are
working with.
\obindex{numeric}
\obindex{integer}
\obindex{long integer}
\obindex{floating point}
\obindex{complex number}
\indexii{C}{language}

Complex numbers have a real and imaginary part, which are both
implemented using \ctype{double} in C.  To extract these parts from
a complex number \var{z}, use \code{\var{z}.real} and \code{\var{z}.imag}.  

Numbers are created by numeric literals or as the result of built-in
functions and operators.  Unadorned integer literals (including hex
and octal numbers) yield plain integers.  Integer literals with an
\character{L} or \character{l} suffix yield long integers
(\character{L} is preferred because \samp{1l} looks too much like
eleven!).  Numeric literals containing a decimal point or an exponent
sign yield floating point numbers.  Appending \character{j} or
\character{J} to a numeric literal yields a complex number.
\indexii{numeric}{literals}
\indexii{integer}{literals}
\indexiii{long}{integer}{literals}
\indexii{floating point}{literals}
\indexii{complex number}{literals}
\indexii{hexadecimal}{literals}
\indexii{octal}{literals}

Python fully supports mixed arithmetic: when a binary arithmetic
operator has operands of different numeric types, the operand with the
``smaller'' type is converted to that of the other, where plain
integer is smaller than long integer is smaller than floating point is
smaller than complex.
Comparisons between numbers of mixed type use the same rule.\footnote{
	As a consequence, the list \code{[1, 2]} is considered equal
        to \code{[1.0, 2.0]}, and similar for tuples.
} The functions \function{int()}, \function{long()}, \function{float()},
and \function{complex()} can be used
to coerce numbers to a specific type.
\index{arithmetic}
\bifuncindex{int}
\bifuncindex{long}
\bifuncindex{float}
\bifuncindex{complex}

All numeric types support the following operations, sorted by
ascending priority (operations in the same box have the same
priority; all numeric operations have a higher priority than
comparison operations):

\begin{tableiii}{c|l|c}{code}{Operation}{Result}{Notes}
  \lineiii{\var{x} + \var{y}}{sum of \var{x} and \var{y}}{}
  \lineiii{\var{x} - \var{y}}{difference of \var{x} and \var{y}}{}
  \hline
  \lineiii{\var{x} * \var{y}}{product of \var{x} and \var{y}}{}
  \lineiii{\var{x} / \var{y}}{quotient of \var{x} and \var{y}}{(1)}
  \lineiii{\var{x} \%{} \var{y}}{remainder of \code{\var{x} / \var{y}}}{}
  \hline
  \lineiii{-\var{x}}{\var{x} negated}{}
  \lineiii{+\var{x}}{\var{x} unchanged}{}
  \hline
  \lineiii{abs(\var{x})}{absolute value or magnitude of \var{x}}{}
  \lineiii{int(\var{x})}{\var{x} converted to integer}{(2)}
  \lineiii{long(\var{x})}{\var{x} converted to long integer}{(2)}
  \lineiii{float(\var{x})}{\var{x} converted to floating point}{}
  \lineiii{complex(\var{re},\var{im})}{a complex number with real part \var{re}, imaginary part \var{im}.  \var{im} defaults to zero.}{}
  \lineiii{\var{c}.conjugate()}{conjugate of the complex number \var{c}}{}
  \lineiii{divmod(\var{x}, \var{y})}{the pair \code{(\var{x} / \var{y}, \var{x} \%{} \var{y})}}{(3)}
  \lineiii{pow(\var{x}, \var{y})}{\var{x} to the power \var{y}}{}
  \lineiii{\var{x} ** \var{y}}{\var{x} to the power \var{y}}{}
\end{tableiii}
\indexiii{operations on}{numeric}{types}
\withsubitem{(complex number method)}{\ttindex{conjugate()}}

\noindent
Notes:
\begin{description}

\item[(1)]
For (plain or long) integer division, the result is an integer.
The result is always rounded towards minus infinity: 1/2 is 0, 
(-1)/2 is -1, 1/(-2) is -1, and (-1)/(-2) is 0.  Note that the result
is a long integer if either operand is a long integer, regardless of
the numeric value.
\indexii{integer}{division}
\indexiii{long}{integer}{division}

\item[(2)]
Conversion from floating point to (long or plain) integer may round or
truncate as in C; see functions \function{floor()} and
\function{ceil()} in the \refmodule{math}\refbimodindex{math} module
for well-defined conversions.
\withsubitem{(in module math)}{\ttindex{floor()}\ttindex{ceil()}}
\indexii{numeric}{conversions}
\indexii{C}{language}

\item[(3)]
See section \ref{built-in-funcs}, ``Built-in Functions,'' for a full
description.

\end{description}
% XXXJH exceptions: overflow (when? what operations?) zerodivision

\subsubsection{Bit-string Operations on Integer Types \label{bitstring-ops}}
\nodename{Bit-string Operations}

Plain and long integer types support additional operations that make
sense only for bit-strings.  Negative numbers are treated as their 2's
complement value (for long integers, this assumes a sufficiently large
number of bits that no overflow occurs during the operation).

The priorities of the binary bit-wise operations are all lower than
the numeric operations and higher than the comparisons; the unary
operation \samp{\~} has the same priority as the other unary numeric
operations (\samp{+} and \samp{-}).

This table lists the bit-string operations sorted in ascending
priority (operations in the same box have the same priority):

\begin{tableiii}{c|l|c}{code}{Operation}{Result}{Notes}
  \lineiii{\var{x} | \var{y}}{bitwise \dfn{or} of \var{x} and \var{y}}{}
  \lineiii{\var{x} \^{} \var{y}}{bitwise \dfn{exclusive or} of \var{x} and \var{y}}{}
  \lineiii{\var{x} \&{} \var{y}}{bitwise \dfn{and} of \var{x} and \var{y}}{}
  \lineiii{\var{x} << \var{n}}{\var{x} shifted left by \var{n} bits}{(1), (2)}
  \lineiii{\var{x} >> \var{n}}{\var{x} shifted right by \var{n} bits}{(1), (3)}
  \hline
  \lineiii{\~\var{x}}{the bits of \var{x} inverted}{}
\end{tableiii}
\indexiii{operations on}{integer}{types}
\indexii{bit-string}{operations}
\indexii{shifting}{operations}
\indexii{masking}{operations}

\noindent
Notes:
\begin{description}
\item[(1)] Negative shift counts are illegal and cause a
\exception{ValueError} to be raised.
\item[(2)] A left shift by \var{n} bits is equivalent to
multiplication by \code{pow(2, \var{n})} without overflow check.
\item[(3)] A right shift by \var{n} bits is equivalent to
division by \code{pow(2, \var{n})} without overflow check.
\end{description}


\subsection{Sequence Types \label{typesseq}}

There are six sequence types: strings, Unicode strings, lists,
tuples, buffers, and xrange objects.

Strings literals are written in single or double quotes:
\code{'xyzzy'}, \code{"frobozz"}.  See chapter 2 of the
\citetitle[../ref/strings.html]{Python Reference Manual} for more about
string literals.  Unicode strings are much like strings, but are
specified in the syntax using a preceeding \character{u} character:
\code{u'abc'}, \code{u"def"}.  Lists are constructed with square brackets,
separating items with commas: \code{[a, b, c]}.  Tuples are
constructed by the comma operator (not within square brackets), with
or without enclosing parentheses, but an empty tuple must have the
enclosing parentheses, e.g., \code{a, b, c} or \code{()}.  A single
item tuple must have a trailing comma, e.g., \code{(d,)}.  Buffers are
not directly supported by Python syntax, but can be created by calling the
builtin function \function{buffer()}.\bifuncindex{buffer}  XRanges
objects are similar to buffers in that there is no specific syntax to
create them, but they are created using the \function{xrange()}
function.\bifuncindex{xrange}
\obindex{sequence}
\obindex{string}
\obindex{Unicode}
\obindex{buffer}
\obindex{tuple}
\obindex{list}
\obindex{xrange}

Sequence types support the following operations.  The \samp{in} and
\samp{not in} operations have the same priorities as the comparison
operations.  The \samp{+} and \samp{*} operations have the same
priority as the corresponding numeric operations.\footnote{They must
have since the parser can't tell the type of the operands.}

This table lists the sequence operations sorted in ascending priority
(operations in the same box have the same priority).  In the table,
\var{s} and \var{t} are sequences of the same type; \var{n}, \var{i}
and \var{j} are integers:

\begin{tableiii}{c|l|c}{code}{Operation}{Result}{Notes}
  \lineiii{\var{x} in \var{s}}{\code{1} if an item of \var{s} is equal to \var{x}, else \code{0}}{}
  \lineiii{\var{x} not in \var{s}}{\code{0} if an item of \var{s} is
equal to \var{x}, else \code{1}}{}
  \hline
  \lineiii{\var{s} + \var{t}}{the concatenation of \var{s} and \var{t}}{}
  \lineiii{\var{s} * \var{n}\textrm{,} \var{n} * \var{s}}{\var{n} copies of \var{s} concatenated}{(1)}
  \hline
  \lineiii{\var{s}[\var{i}]}{\var{i}'th item of \var{s}, origin 0}{(2)}
  \lineiii{\var{s}[\var{i}:\var{j}]}{slice of \var{s} from \var{i} to \var{j}}{(2), (3)}
  \hline
  \lineiii{len(\var{s})}{length of \var{s}}{}
  \lineiii{min(\var{s})}{smallest item of \var{s}}{}
  \lineiii{max(\var{s})}{largest item of \var{s}}{}
\end{tableiii}
\indexiii{operations on}{sequence}{types}
\bifuncindex{len}
\bifuncindex{min}
\bifuncindex{max}
\indexii{concatenation}{operation}
\indexii{repetition}{operation}
\indexii{subscript}{operation}
\indexii{slice}{operation}
\opindex{in}
\opindex{not in}

\noindent
Notes:

\begin{description}
\item[(1)] Values of \var{n} less than \code{0} are treated as
  \code{0} (which yields an empty sequence of the same type as
  \var{s}).

\item[(2)] If \var{i} or \var{j} is negative, the index is relative to
  the end of the string, i.e., \code{len(\var{s}) + \var{i}} or
  \code{len(\var{s}) + \var{j}} is substituted.  But note that \code{-0} is
  still \code{0}.
  
\item[(3)] The slice of \var{s} from \var{i} to \var{j} is defined as
  the sequence of items with index \var{k} such that \code{\var{i} <=
  \var{k} < \var{j}}.  If \var{i} or \var{j} is greater than
  \code{len(\var{s})}, use \code{len(\var{s})}.  If \var{i} is omitted,
  use \code{0}.  If \var{j} is omitted, use \code{len(\var{s})}.  If
  \var{i} is greater than or equal to \var{j}, the slice is empty.
\end{description}


\subsubsection{String Methods \label{string-methods}}

These are the string methods which both 8-bit strings and Unicode
objects support:

\begin{methoddesc}[string]{capitalize}{}
Return a copy of the string with only its first character capitalized.
\end{methoddesc}

\begin{methoddesc}[string]{center}{width}
Return centered in a string of length \var{width}. Padding is done
using spaces.
\end{methoddesc}

\begin{methoddesc}[string]{count}{sub\optional{, start\optional{, end}}}
Return the number of occurrences of substring \var{sub} in string
S\code{[\var{start}:\var{end}]}.  Optional arguments \var{start} and
\var{end} are interpreted as in slice notation.
\end{methoddesc}

\begin{methoddesc}[string]{encode}{\optional{encoding\optional{,errors}}}
Return an encoded version of the string.  Default encoding is the current
default string encoding.  \var{errors} may be given to set a different
error handling scheme.  The default for \var{errors} is
\code{'strict'}, meaning that encoding errors raise a
\exception{ValueError}.  Other possible values are \code{'ignore'} and
\code{'replace'}.
\versionadded{2.0}
\end{methoddesc}

\begin{methoddesc}[string]{endswith}{suffix\optional{, start\optional{, end}}}
Return true if the string ends with the specified \var{suffix},
otherwise return false.  With optional \var{start}, test beginning at
that position.  With optional \var{end}, stop comparing at that position.
\end{methoddesc}

\begin{methoddesc}[string]{expandtabs}{\optional{tabsize}}
Return a copy of the string where all tab characters are expanded
using spaces.  If \var{tabsize} is not given, a tab size of \code{8}
characters is assumed.
\end{methoddesc}

\begin{methoddesc}[string]{find}{sub\optional{, start\optional{, end}}}
Return the lowest index in the string where substring \var{sub} is
found, such that \var{sub} is contained in the range [\var{start},
\var{end}).  Optional arguments \var{start} and \var{end} are
interpreted as in slice notation.  Return \code{-1} if \var{sub} is
not found.
\end{methoddesc}

\begin{methoddesc}[string]{index}{sub\optional{, start\optional{, end}}}
Like \method{find()}, but raise \exception{ValueError} when the
substring is not found.
\end{methoddesc}

\begin{methoddesc}[string]{isalnum}{}
Return true if all characters in the string are alphanumeric and there
is at least one character, false otherwise.
\end{methoddesc}

\begin{methoddesc}[string]{isalpha}{}
Return true if all characters in the string are alphabetic and there
is at least one character, false otherwise.
\end{methoddesc}

\begin{methoddesc}[string]{isdigit}{}
Return true if there are only digit characters, false otherwise.
\end{methoddesc}

\begin{methoddesc}[string]{islower}{}
Return true if all cased characters in the string are lowercase and
there is at least one cased character, false otherwise.
\end{methoddesc}

\begin{methoddesc}[string]{isspace}{}
Return true if there are only whitespace characters in the string and
the string is not empty, false otherwise.
\end{methoddesc}

\begin{methoddesc}[string]{istitle}{}
Return true if the string is a titlecased string, i.e.\ uppercase
characters may only follow uncased characters and lowercase characters
only cased ones.  Return false otherwise.
\end{methoddesc}

\begin{methoddesc}[string]{isupper}{}
Return true if all cased characters in the string are uppercase and
there is at least one cased character, false otherwise.
\end{methoddesc}

\begin{methoddesc}[string]{join}{seq}
Return a string which is the concatenation of the strings in the
sequence \var{seq}.  The separator between elements is the string
providing this method.
\end{methoddesc}

\begin{methoddesc}[string]{ljust}{width}
Return the string left justified in a string of length \var{width}.
Padding is done using spaces.  The original string is returned if
\var{width} is less than \code{len(\var{s})}.
\end{methoddesc}

\begin{methoddesc}[string]{lower}{}
Return a copy of the string converted to lowercase.
\end{methoddesc}

\begin{methoddesc}[string]{lstrip}{}
Return a copy of the string with leading whitespace removed.
\end{methoddesc}

\begin{methoddesc}[string]{replace}{old, new\optional{, maxsplit}}
Return a copy of the string with all occurrences of substring
\var{old} replaced by \var{new}.  If the optional argument
\var{maxsplit} is given, only the first \var{maxsplit} occurrences are
replaced.
\end{methoddesc}

\begin{methoddesc}[string]{rfind}{sub \optional{,start \optional{,end}}}
Return the highest index in the string where substring \var{sub} is
found, such that \var{sub} is contained within s[start,end].  Optional
arguments \var{start} and \var{end} are interpreted as in slice
notation.  Return \code{-1} on failure.
\end{methoddesc}

\begin{methoddesc}[string]{rindex}{sub\optional{, start\optional{, end}}}
Like \method{rfind()} but raises \exception{ValueError} when the
substring \var{sub} is not found.
\end{methoddesc}

\begin{methoddesc}[string]{rjust}{width}
Return the string right justified in a string of length \var{width}.
Padding is done using spaces.  The original string is returned if
\var{width} is less than \code{len(\var{s})}.
\end{methoddesc}

\begin{methoddesc}[string]{rstrip}{}
Return a copy of the string with trailing whitespace removed.
\end{methoddesc}

\begin{methoddesc}[string]{split}{\optional{sep \optional{,maxsplit}}}
Return a list of the words in the string, using \var{sep} as the
delimiter string.  If \var{maxsplit} is given, at most \var{maxsplit}
splits are done.  If \var{sep} is not specified or \code{None}, any
whitespace string is a separator.
\end{methoddesc}

\begin{methoddesc}[string]{splitlines}{\optional{keepends}}
Return a list of the lines in the string, breaking at line
boundaries.  Line breaks are not included in the resulting list unless
\var{keepends} is given and true.
\end{methoddesc}

\begin{methoddesc}[string]{startswith}{prefix\optional{, start\optional{, end}}}
Return true if string starts with the \var{prefix}, otherwise
return false.  With optional \var{start}, test string beginning at
that position.  With optional \var{end}, stop comparing string at that
position.
\end{methoddesc}

\begin{methoddesc}[string]{strip}{}
Return a copy of the string with leading and trailing whitespace
removed.
\end{methoddesc}

\begin{methoddesc}[string]{swapcase}{}
Return a copy of the string with uppercase characters converted to
lowercase and vice versa.
\end{methoddesc}

\begin{methoddesc}[string]{title}{}
Return a titlecased version of, i.e.\ words start with uppercase
characters, all remaining cased characters are lowercase.
\end{methoddesc}

\begin{methoddesc}[string]{translate}{table\optional{, deletechars}}
Return a copy of the string where all characters occurring in the
optional argument \var{deletechars} are removed, and the remaining
characters have been mapped through the given translation table, which
must be a string of length 256.
\end{methoddesc}

\begin{methoddesc}[string]{upper}{}
Return a copy of the string converted to uppercase.
\end{methoddesc}


\subsubsection{String Formatting Operations \label{typesseq-strings}}

\index{formatting, string}
\index{string!formatting}
\index{printf-style formatting}
\index{sprintf-style formatting}

String and Unicode objects have one unique built-in operation: the
\code{\%} operator (modulo).  Given \code{\var{format} \%
\var{values}} (where \var{format} is a string or Unicode object),
\code{\%} conversion specifications in \var{format} are replaced with
zero or more elements of \var{values}.  The effect is similar to the
using \cfunction{sprintf()} in the C language.  If \var{format} is a
Unicode object, or if any of the objects being converted using the
\code{\%s} conversion are Unicode objects, the result will be a
Unicode object as well.

If \var{format} requires a single argument, \var{values} may be a
single non-tuple object. \footnote{A tuple object in this case should
  be a singleton.}  Otherwise, \var{values} must be a tuple with
exactly the number of items specified by the format string, or a
single mapping object (for example, a dictionary).

A conversion specifier contains two or more characters and has the
following components, which must occur in this order:

\begin{enumerate}
  \item  The \character{\%} character, which marks the start of the
         specifier.
  \item  Mapping key value (optional), consisting of an identifier in
         parentheses (for example, \code{(somename)}).
  \item  Conversion flags (optional), which affect the result of some
         conversion types.
  \item  Minimum field width (optional).  If specified as an
         \character{*} (asterisk), the actual width is read from the
         next element of the tuple in \var{values}, and the object to
         convert comes after the minimum field width and optional
         precision.
  \item  Precision (optional), given as a \character{.} (dot) followed
         by the precision.  If specified as \character{*} (an
         asterisk), the actual width is read from the next element of
         the tuple in \var{values}, and the value to convert comes after
         the precision.
  \item  Length modifier (optional).
  \item  Conversion type.
\end{enumerate}

If the right argument is a dictionary (or any kind of mapping), then
the formats in the string \emph{must} have a parenthesized key into
that dictionary inserted immediately after the \character{\%}
character, and each format formats the corresponding entry from the
mapping.  For example:

\begin{verbatim}
>>> count = 2
>>> language = 'Python'
>>> print '%(language)s has %(count)03d quote types.' % vars()
Python has 002 quote types.
\end{verbatim}

In this case no \code{*} specifiers may occur in a format (since they
require a sequential parameter list).

The conversion flag characters are:

\begin{tableii}{c|l}{character}{Flag}{Meaning}
  \lineii{\#}{The value conversion will use the ``alternate form''
              (where defined below).}
  \lineii{0}{The conversion will be zero padded.}
  \lineii{-}{The converted value is left adjusted (overrides
             \character{-}).}
  \lineii{{~}}{(a space) A blank should be left before a positive number
             (or empty string) produced by a signed conversion.}
  \lineii{+}{A sign character (\character{+} or \character{-}) will
             precede the conversion (overrides a "space" flag).}
\end{tableii}

The length modifier may be \code{h}, \code{l}, and \code{L} may be
present, but are ignored as they are not necessary for Python.

The conversion types are:

\begin{tableii}{c|l}{character}{Conversion}{Meaning}
  \lineii{d}{Signed integer decimal.}
  \lineii{i}{Signed integer decimal.}
  \lineii{o}{Unsigned octal.}
  \lineii{u}{Unsigned decimal.}
  \lineii{x}{Unsigned hexidecimal (lowercase).}
  \lineii{X}{Unsigned hexidecimal (uppercase).}
  \lineii{e}{Floating point exponential format (lowercase).}
  \lineii{E}{Floating point exponential format (uppercase).}
  \lineii{f}{Floating point decimal format.}
  \lineii{F}{Floating point decimal format.}
  \lineii{g}{Same as \character{e} if exponent is greater than -4 or
             less than precision, \character{f} otherwise.}
  \lineii{G}{Same as \character{E} if exponent is greater than -4 or
             less than precision, \character{F} otherwise.}
  \lineii{c}{Single character (accepts integer or single character
             string).}
  \lineii{r}{String (converts any python object using
             \function{repr()}).}
  \lineii{s}{String (converts any python object using
             \function{str()}).}
  \lineii{\%}{No argument is converted, results in a \character{\%}
              character in the result.  (The complete specification is
              \code{\%\%}.)}
\end{tableii}

% XXX Examples?


Since Python strings have an explicit length, \code{\%s} conversions
do not assume that \code{'\e0'} is the end of the string.

For safety reasons, floating point precisions are clipped to 50;
\code{\%f} conversions for numbers whose absolute value is over 1e25
are replaced by \code{\%g} conversions.\footnote{
  These numbers are fairly arbitrary.  They are intended to
  avoid printing endless strings of meaningless digits without hampering
  correct use and without having to know the exact precision of floating
  point values on a particular machine.
}  All other errors raise exceptions.

Additional string operations are defined in standard module
\refmodule{string} and in built-in module \refmodule{re}.
\refstmodindex{string}
\refstmodindex{re}


\subsubsection{XRange Type \label{typesseq-xrange}}

The xrange\obindex{xrange} type is an immutable sequence which is
commonly used for looping.  The advantage of the xrange type is that an
xrange object will always take the same amount of memory, no matter the
size of the range it represents.  There are no consistent performance
advantages.

XRange objects behave like tuples, and offer a single method:

\begin{methoddesc}[xrange]{tolist}{}
  Return a list object which represents the same values as the xrange
  object.
\end{methoddesc}


\subsubsection{Mutable Sequence Types \label{typesseq-mutable}}

List objects support additional operations that allow in-place
modification of the object.
These operations would be supported by other mutable sequence types
(when added to the language) as well.
Strings and tuples are immutable sequence types and such objects cannot
be modified once created.
The following operations are defined on mutable sequence types (where
\var{x} is an arbitrary object):
\indexiii{mutable}{sequence}{types}
\obindex{list}

\begin{tableiii}{c|l|c}{code}{Operation}{Result}{Notes}
  \lineiii{\var{s}[\var{i}] = \var{x}}
	{item \var{i} of \var{s} is replaced by \var{x}}{}
  \lineiii{\var{s}[\var{i}:\var{j}] = \var{t}}
  	{slice of \var{s} from \var{i} to \var{j} is replaced by \var{t}}{}
  \lineiii{del \var{s}[\var{i}:\var{j}]}
	{same as \code{\var{s}[\var{i}:\var{j}] = []}}{}
  \lineiii{\var{s}.append(\var{x})}
	{same as \code{\var{s}[len(\var{s}):len(\var{s})] = [\var{x}]}}{(1)}
  \lineiii{\var{s}.extend(\var{x})}
        {same as \code{\var{s}[len(\var{s}):len(\var{s})] = \var{x}}}{(2)}
  \lineiii{\var{s}.count(\var{x})}
    {return number of \var{i}'s for which \code{\var{s}[\var{i}] == \var{x}}}{}
  \lineiii{\var{s}.index(\var{x})}
    {return smallest \var{i} such that \code{\var{s}[\var{i}] == \var{x}}}{(3)}
  \lineiii{\var{s}.insert(\var{i}, \var{x})}
	{same as \code{\var{s}[\var{i}:\var{i}] = [\var{x}]}
	  if \code{\var{i} >= 0}}{}
  \lineiii{\var{s}.pop(\optional{\var{i}})}
    {same as \code{\var{x} = \var{s}[\var{i}]; del \var{s}[\var{i}]; return \var{x}}}{(4)}
  \lineiii{\var{s}.remove(\var{x})}
	{same as \code{del \var{s}[\var{s}.index(\var{x})]}}{(3)}
  \lineiii{\var{s}.reverse()}
	{reverses the items of \var{s} in place}{(5)}
  \lineiii{\var{s}.sort(\optional{\var{cmpfunc}})}
	{sort the items of \var{s} in place}{(5), (6)}
\end{tableiii}
\indexiv{operations on}{mutable}{sequence}{types}
\indexiii{operations on}{sequence}{types}
\indexiii{operations on}{list}{type}
\indexii{subscript}{assignment}
\indexii{slice}{assignment}
\stindex{del}
\withsubitem{(list method)}{
  \ttindex{append()}\ttindex{extend()}\ttindex{count()}\ttindex{index()}
  \ttindex{insert()}\ttindex{pop()}\ttindex{remove()}\ttindex{reverse()}
  \ttindex{sort()}}
\noindent
Notes:
\begin{description}
\item[(1)] The C implementation of Python has historically accepted
  multiple parameters and implicitly joined them into a tuple; this
  no longer works in Python 2.0.  Use of this misfeature has been
  deprecated since Python 1.4.

\item[(2)] Raises an exception when \var{x} is not a list object.  The 
  \method{extend()} method is experimental and not supported by
  mutable sequence types other than lists.

\item[(3)] Raises \exception{ValueError} when \var{x} is not found in
  \var{s}.

\item[(4)] The \method{pop()} method is only supported by the list and
  array types.  The optional argument \var{i} defaults to \code{-1},
  so that by default the last item is removed and returned.

\item[(5)] The \method{sort()} and \method{reverse()} methods modify the
  list in place for economy of space when sorting or reversing a large
  list.  They don't return the sorted or reversed list to remind you
  of this side effect.

\item[(6)] The \method{sort()} method takes an optional argument
  specifying a comparison function of two arguments (list items) which
  should return \code{-1}, \code{0} or \code{1} depending on whether
  the first argument is considered smaller than, equal to, or larger
  than the second argument.  Note that this slows the sorting process
  down considerably; e.g. to sort a list in reverse order it is much
  faster to use calls to the methods \method{sort()} and
  \method{reverse()} than to use the built-in function
  \function{sort()} with a comparison function that reverses the
  ordering of the elements.
\end{description}


\subsection{Mapping Types \label{typesmapping}}
\obindex{mapping}
\obindex{dictionary}

A \dfn{mapping} object maps values of one type (the key type) to
arbitrary objects.  Mappings are mutable objects.  There is currently
only one standard mapping type, the \dfn{dictionary}.  A dictionary's keys are
almost arbitrary values.  The only types of values not acceptable as
keys are values containing lists or dictionaries or other mutable
types that are compared by value rather than by object identity.
Numeric types used for keys obey the normal rules for numeric
comparison: if two numbers compare equal (e.g. \code{1} and
\code{1.0}) then they can be used interchangeably to index the same
dictionary entry.

Dictionaries are created by placing a comma-separated list of
\code{\var{key}: \var{value}} pairs within braces, for example:
\code{\{'jack': 4098, 'sjoerd': 4127\}} or
\code{\{4098: 'jack', 4127: 'sjoerd'\}}.

The following operations are defined on mappings (where \var{a} and
\var{b} are mappings, \var{k} is a key, and \var{v} and \var{x} are
arbitrary objects):
\indexiii{operations on}{mapping}{types}
\indexiii{operations on}{dictionary}{type}
\stindex{del}
\bifuncindex{len}
\withsubitem{(dictionary method)}{
  \ttindex{clear()}
  \ttindex{copy()}
  \ttindex{has_key()}
  \ttindex{items()}
  \ttindex{keys()}
  \ttindex{update()}
  \ttindex{values()}
  \ttindex{get()}}

\begin{tableiii}{c|l|c}{code}{Operation}{Result}{Notes}
  \lineiii{len(\var{a})}{the number of items in \var{a}}{}
  \lineiii{\var{a}[\var{k}]}{the item of \var{a} with key \var{k}}{(1)}
  \lineiii{\var{a}[\var{k}] = \var{v}}
          {set \code{\var{a}[\var{k}]} to \var{v}}
          {}
  \lineiii{del \var{a}[\var{k}]}
          {remove \code{\var{a}[\var{k}]} from \var{a}}
          {(1)}
  \lineiii{\var{a}.clear()}{remove all items from \code{a}}{}
  \lineiii{\var{a}.copy()}{a (shallow) copy of \code{a}}{}
  \lineiii{\var{k} \code{in} \var{a}}
          {\code{1} if \var{a} has a key \var{k}, else \code{0}}
          {}
  \lineiii{\var{k} not in \var{a}}
          {\code{0} if \var{a} has a key \var{k}, else \code{1}}
          {}
  \lineiii{\var{a}.has_key(\var{k})}
          {Equivalent to \var{k} \code{in} \var{a}}
          {}
  \lineiii{\var{a}.items()}
          {a copy of \var{a}'s list of (\var{key}, \var{value}) pairs}
          {(2)}
  \lineiii{\var{a}.keys()}{a copy of \var{a}'s list of keys}{(2)}
  \lineiii{\var{a}.update(\var{b})}
          {\code{for k in \var{b}.keys(): \var{a}[k] = \var{b}[k]}}
          {(3)}
  \lineiii{\var{a}.values()}{a copy of \var{a}'s list of values}{(2)}
  \lineiii{\var{a}.get(\var{k}\optional{, \var{x}})}
          {\code{\var{a}[\var{k}]} if \code{\var{k} in \var{a}}},
           else \var{x}}
          {(4)}
  \lineiii{\var{a}.setdefault(\var{k}\optional{, \var{x}})}
          {\code{\var{a}[\var{k}]} if \code{\var{k} in \var{a}}},
           else \var{x} (also setting it)}
          {(5)}
  \lineiii{\var{a}.popitem()}
          {remove and return an arbitrary (\var{key}, \var{value}) pair}
          {(6)}
\end{tableiii}

\noindent
Notes:
\begin{description}
\item[(1)] Raises a \exception{KeyError} exception if \var{k} is not
in the map.

\item[(2)] Keys and values are listed in random order.  If
\method{keys()} and \method{values()} are called with no intervening
modifications to the dictionary, the two lists will directly
correspond.  This allows the creation of \code{(\var{value},
\var{key})} pairs using \function{map()}: \samp{pairs = map(None,
\var{a}.values(), \var{a}.keys())}.

\item[(3)] \var{b} must be of the same type as \var{a}.

\item[(4)] Never raises an exception if \var{k} is not in the map,
instead it returns \var{x}.  \var{x} is optional; when \var{x} is not
provided and \var{k} is not in the map, \code{None} is returned.

\item[(5)] \function{setdefault()} is like \function{get()}, except
that if \var{k} is missing, \var{x} is both returned and inserted into
the dictionary as the value of \var{k}.

\item[(6)] \function{popitem()} is useful to destructively iterate
over a dictionary, as often used in set algorithms.
\end{description}


\subsection{Other Built-in Types \label{typesother}}

The interpreter supports several other kinds of objects.
Most of these support only one or two operations.


\subsubsection{Modules \label{typesmodules}}

The only special operation on a module is attribute access:
\code{\var{m}.\var{name}}, where \var{m} is a module and \var{name}
accesses a name defined in \var{m}'s symbol table.  Module attributes
can be assigned to.  (Note that the \keyword{import} statement is not,
strictly speaking, an operation on a module object; \code{import
\var{foo}} does not require a module object named \var{foo} to exist,
rather it requires an (external) \emph{definition} for a module named
\var{foo} somewhere.)

A special member of every module is \member{__dict__}.
This is the dictionary containing the module's symbol table.
Modifying this dictionary will actually change the module's symbol
table, but direct assignment to the \member{__dict__} attribute is not
possible (i.e., you can write \code{\var{m}.__dict__['a'] = 1}, which
defines \code{\var{m}.a} to be \code{1}, but you can't write
\code{\var{m}.__dict__ = \{\}}.

Modules built into the interpreter are written like this:
\code{<module 'sys' (built-in)>}.  If loaded from a file, they are
written as \code{<module 'os' from
'/usr/local/lib/python\shortversion/os.pyc'>}.


\subsubsection{Classes and Class Instances \label{typesobjects}}
\nodename{Classes and Instances}

See chapters 3 and 7 of the \citetitle[../ref/ref.html]{Python
Reference Manual} for these.


\subsubsection{Functions \label{typesfunctions}}

Function objects are created by function definitions.  The only
operation on a function object is to call it:
\code{\var{func}(\var{argument-list})}.

There are really two flavors of function objects: built-in functions
and user-defined functions.  Both support the same operation (to call
the function), but the implementation is different, hence the
different object types.

The implementation adds two special read-only attributes:
\code{\var{f}.func_code} is a function's \dfn{code
object}\obindex{code} (see below) and \code{\var{f}.func_globals} is
the dictionary used as the function's global namespace (this is the
same as \code{\var{m}.__dict__} where \var{m} is the module in which
the function \var{f} was defined).

Function objects also support getting and setting arbitrary
attributes, which can be used to, e.g. attach metadata to functions.
Regular attribute dot-notation is used to get and set such
attributes. \emph{Note that the current implementation only supports
function attributes on functions written in Python.  Function
attributes on built-ins may be supported in the future.}

Functions have another special attribute \code{\var{f}.__dict__}
(a.k.a. \code{\var{f}.func_dict}) which contains the namespace used to
support function attributes.  \code{__dict__} can be accessed
directly, set to a dictionary object, or \code{None}.  It can also be
deleted (but the following two lines are equivalent):

\begin{verbatim}
del func.__dict__
func.__dict__ = None
\end{verbatim}

\subsubsection{Methods \label{typesmethods}}
\obindex{method}

Methods are functions that are called using the attribute notation.
There are two flavors: built-in methods (such as \method{append()} on
lists) and class instance methods.  Built-in methods are described
with the types that support them.

The implementation adds two special read-only attributes to class
instance methods: \code{\var{m}.im_self} is the object on which the
method operates, and \code{\var{m}.im_func} is the function
implementing the method.  Calling \code{\var{m}(\var{arg-1},
\var{arg-2}, \textrm{\ldots}, \var{arg-n})} is completely equivalent to
calling \code{\var{m}.im_func(\var{m}.im_self, \var{arg-1},
\var{arg-2}, \textrm{\ldots}, \var{arg-n})}.

Class instance methods are either \emph{bound} or \emph{unbound},
referring to whether the method was accessed through an instance or a
class, respectively.  When a method is unbound, its \code{im_self}
attribute will be \code{None} and if called, an explicit \code{self}
object must be passed as the first argument.  In this case,
\code{self} must be an instance of the unbound method's class (or a
subclass of that class), otherwise a \code{TypeError} is raised.

Like function objects, methods objects support getting
arbitrary attributes.  However, since method attributes are actually
stored on the underlying function object (i.e. \code{meth.im_func}),
setting method attributes on either bound or unbound methods is
disallowed.  Attempting to set a method attribute results in a
\code{TypeError} being raised.  In order to set a method attribute,
you need to explicitly set it on the underlying function object:

\begin{verbatim}
class C:
    def method(self):
        pass

c = C()
c.method.im_func.whoami = 'my name is c'
\end{verbatim}

See the \citetitle[../ref/ref.html]{Python Reference Manual} for more
information.


\subsubsection{Code Objects \label{bltin-code-objects}}
\obindex{code}

Code objects are used by the implementation to represent
``pseudo-compiled'' executable Python code such as a function body.
They differ from function objects because they don't contain a
reference to their global execution environment.  Code objects are
returned by the built-in \function{compile()} function and can be
extracted from function objects through their \member{func_code}
attribute.
\bifuncindex{compile}
\withsubitem{(function object attribute)}{\ttindex{func_code}}

A code object can be executed or evaluated by passing it (instead of a
source string) to the \keyword{exec} statement or the built-in
\function{eval()} function.
\stindex{exec}
\bifuncindex{eval}

See the \citetitle[../ref/ref.html]{Python Reference Manual} for more
information.


\subsubsection{Type Objects \label{bltin-type-objects}}

Type objects represent the various object types.  An object's type is
accessed by the built-in function \function{type()}.  There are no special
operations on types.  The standard module \module{types} defines names
for all standard built-in types.
\bifuncindex{type}
\refstmodindex{types}

Types are written like this: \code{<type 'int'>}.


\subsubsection{The Null Object \label{bltin-null-object}}

This object is returned by functions that don't explicitly return a
value.  It supports no special operations.  There is exactly one null
object, named \code{None} (a built-in name).

It is written as \code{None}.


\subsubsection{The Ellipsis Object \label{bltin-ellipsis-object}}

This object is used by extended slice notation (see the
\citetitle[../ref/ref.html]{Python Reference Manual}).  It supports no
special operations.  There is exactly one ellipsis object, named
\constant{Ellipsis} (a built-in name).

It is written as \code{Ellipsis}.


\subsubsection{File Objects\obindex{file}
               \label{bltin-file-objects}}

File objects are implemented using C's \code{stdio} package and can be
created with the built-in function
\function{open()}\bifuncindex{open} described in section 
\ref{built-in-funcs}, ``Built-in Functions.''  They are also returned
by some other built-in functions and methods, e.g.,
\function{os.popen()} and \function{os.fdopen()} and the
\method{makefile()} method of socket objects.
\refstmodindex{os}
\refbimodindex{socket}

When a file operation fails for an I/O-related reason, the exception
\exception{IOError} is raised.  This includes situations where the
operation is not defined for some reason, like \method{seek()} on a tty
device or writing a file opened for reading.

Files have the following methods:


\begin{methoddesc}[file]{close}{}
  Close the file.  A closed file cannot be read or written anymore.
  Any operation which requires that the file be open will raise a
  \exception{ValueError} after the file has been closed.  Calling
  \method{close()} more than once is allowed.
\end{methoddesc}

\begin{methoddesc}[file]{flush}{}
  Flush the internal buffer, like \code{stdio}'s
  \cfunction{fflush()}.  This may be a no-op on some file-like
  objects.
\end{methoddesc}

\begin{methoddesc}[file]{isatty}{}
  Return true if the file is connected to a tty(-like) device, else
  false.  \strong{Note:} If a file-like object is not associated
  with a real file, this method should \emph{not} be implemented.
\end{methoddesc}

\begin{methoddesc}[file]{fileno}{}
  \index{file descriptor}
  \index{descriptor, file}
  Return the integer ``file descriptor'' that is used by the
  underlying implementation to request I/O operations from the
  operating system.  This can be useful for other, lower level
  interfaces that use file descriptors, e.g.\ module
  \refmodule{fcntl}\refbimodindex{fcntl} or \function{os.read()} and
  friends.  \strong{Note:} File-like objects which do not have a real
  file descriptor should \emph{not} provide this method!
\end{methoddesc}

\begin{methoddesc}[file]{read}{\optional{size}}
  Read at most \var{size} bytes from the file (less if the read hits
  \EOF{} before obtaining \var{size} bytes).  If the \var{size}
  argument is negative or omitted, read all data until \EOF{} is
  reached.  The bytes are returned as a string object.  An empty
  string is returned when \EOF{} is encountered immediately.  (For
  certain files, like ttys, it makes sense to continue reading after
  an \EOF{} is hit.)  Note that this method may call the underlying
  C function \cfunction{fread()} more than once in an effort to
  acquire as close to \var{size} bytes as possible.
\end{methoddesc}

\begin{methoddesc}[file]{readline}{\optional{size}}
  Read one entire line from the file.  A trailing newline character is
  kept in the string\footnote{
	The advantage of leaving the newline on is that an empty string 
	can be returned to mean \EOF{} without being ambiguous.  Another 
	advantage is that (in cases where it might matter, e.g. if you 
	want to make an exact copy of a file while scanning its lines) 
	you can tell whether the last line of a file ended in a newline
	or not (yes this happens!).
  } (but may be absent when a file ends with an
  incomplete line).  If the \var{size} argument is present and
  non-negative, it is a maximum byte count (including the trailing
  newline) and an incomplete line may be returned.
  An empty string is returned when \EOF{} is hit
  immediately.  Note: Unlike \code{stdio}'s \cfunction{fgets()}, the
  returned string contains null characters (\code{'\e 0'}) if they
  occurred in the input.
\end{methoddesc}

\begin{methoddesc}[file]{readlines}{\optional{sizehint}}
  Read until \EOF{} using \method{readline()} and return a list containing
  the lines thus read.  If the optional \var{sizehint} argument is
  present, instead of reading up to \EOF{}, whole lines totalling
  approximately \var{sizehint} bytes (possibly after rounding up to an
  internal buffer size) are read.  Objects implementing a file-like
  interface may choose to ignore \var{sizehint} if it cannot be
  implemented, or cannot be implemented efficiently.
\end{methoddesc}

\begin{methoddesc}[file]{xreadlines}{}
  Equivalent to \function{xreadlines.xreadlines(file)}.\refstmodindex{xreadlines}
\end{methoddesc}

\begin{methoddesc}[file]{seek}{offset\optional{, whence}}
  Set the file's current position, like \code{stdio}'s \cfunction{fseek()}.
  The \var{whence} argument is optional and defaults to \code{0}
  (absolute file positioning); other values are \code{1} (seek
  relative to the current position) and \code{2} (seek relative to the
  file's end).  There is no return value.  Note that if the file is
  opened for appending (mode \code{'a'} or \code{'a+'}), any
  \method{seek()} operations will be undone at the next write.  If the
  file is only opened for writing in append mode (mode \code{'a'}),
  this method is essentially a no-op, but it remains useful for files
  opened in append mode with reading enabled (mode \code{'a+'}).
\end{methoddesc}

\begin{methoddesc}[file]{tell}{}
  Return the file's current position, like \code{stdio}'s
  \cfunction{ftell()}.
\end{methoddesc}

\begin{methoddesc}[file]{truncate}{\optional{size}}
  Truncate the file's size.  If the optional \var{size} argument
  present, the file is truncated to (at most) that size.  The size
  defaults to the current position.  Availability of this function
  depends on the operating system version (for example, not all
  \UNIX{} versions support this operation).
\end{methoddesc}

\begin{methoddesc}[file]{write}{str}
  Write a string to the file.  There is no return value.  Note: Due to
  buffering, the string may not actually show up in the file until
  the \method{flush()} or \method{close()} method is called.
\end{methoddesc}

\begin{methoddesc}[file]{writelines}{list}
  Write a list of strings to the file.  There is no return value.
  (The name is intended to match \method{readlines()};
  \method{writelines()} does not add line separators.)
\end{methoddesc}

\begin{methoddesc}[file]{xreadlines}{}
  Equivalent to
  \function{xreadlines.xreadlines(\var{file})}.\refstmodindex{xreadlines}
  (See the \refmodule{xreadlines} module for more information.)
\end{methoddesc}


File objects also offer a number of other interesting attributes.
These are not required for file-like objects, but should be
implemented if they make sense for the particular object.

\begin{memberdesc}[file]{closed}
Boolean indicating the current state of the file object.  This is a
read-only attribute; the \method{close()} method changes the value.
It may not be available on all file-like objects.
\end{memberdesc}

\begin{memberdesc}[file]{mode}
The I/O mode for the file.  If the file was created using the
\function{open()} built-in function, this will be the value of the
\var{mode} parameter.  This is a read-only attribute and may not be
present on all file-like objects.
\end{memberdesc}

\begin{memberdesc}[file]{name}
If the file object was created using \function{open()}, the name of
the file.  Otherwise, some string that indicates the source of the
file object, of the form \samp{<\mbox{\ldots}>}.  This is a read-only
attribute and may not be present on all file-like objects.
\end{memberdesc}

\begin{memberdesc}[file]{softspace}
Boolean that indicates whether a space character needs to be printed
before another value when using the \keyword{print} statement.
Classes that are trying to simulate a file object should also have a
writable \member{softspace} attribute, which should be initialized to
zero.  This will be automatic for most classes implemented in Python
(care may be needed for objects that override attribute access); types
implemented in C will have to provide a writable
\member{softspace} attribute.
\strong{Note:}  This attribute is not used to control the
\keyword{print} statement, but to allow the implementation of
\keyword{print} to keep track of its internal state.
\end{memberdesc}


\subsubsection{Internal Objects \label{typesinternal}}

See the \citetitle[../ref/ref.html]{Python Reference Manual} for this
information.  It describes stack frame objects, traceback objects, and
slice objects.


\subsection{Special Attributes \label{specialattrs}}

The implementation adds a few special read-only attributes to several
object types, where they are relevant:

\begin{memberdesc}[object]{__dict__}
A dictionary or other mapping object used to store an
object's (writable) attributes.
\end{memberdesc}

\begin{memberdesc}[object]{__methods__}
List of the methods of many built-in object types,
e.g., \code{[].__methods__} yields
\code{['append', 'count', 'index', 'insert', 'pop', 'remove',
'reverse', 'sort']}.  This usually does not need to be explicitly
provided by the object.
\end{memberdesc}

\begin{memberdesc}[object]{__members__}
Similar to \member{__methods__}, but lists data attributes.  This
usually does not need to be explicitly provided by the object.
\end{memberdesc}

\begin{memberdesc}[instance]{__class__}
The class to which a class instance belongs.
\end{memberdesc}

\begin{memberdesc}[class]{__bases__}
The tuple of base classes of a class object.
\end{memberdesc}

\section{Built-in Exceptions}

\declaremodule{standard}{exceptions}
\modulesynopsis{Standard exceptions classes.}


Exceptions can be class objects or string objects.  Though most
exceptions have been string objects in past versions of Python, in
Python 1.5 and newer versions, all standard exceptions have been
converted to class objects, and users are encouraged to do the same.
The exceptions are defined in the module \module{exceptions}.  This
module never needs to be imported explicitly: the exceptions are
provided in the built-in namespace.

Two distinct string objects with the same value are considered different
exceptions.  This is done to force programmers to use exception names
rather than their string value when specifying exception handlers.
The string value of all built-in exceptions is their name, but this is
not a requirement for user-defined exceptions or exceptions defined by
library modules.

For class exceptions, in a \keyword{try}\stindex{try} statement with
an \keyword{except}\stindex{except} clause that mentions a particular
class, that clause also handles any exception classes derived from
that class (but not exception classes from which \emph{it} is
derived).  Two exception classes that are not related via subclassing
are never equivalent, even if they have the same name.

The built-in exceptions listed below can be generated by the
interpreter or built-in functions.  Except where mentioned, they have
an ``associated value'' indicating the detailed cause of the error.
This may be a string or a tuple containing several items of
information (e.g., an error code and a string explaining the code).
The associated value is the second argument to the
\keyword{raise}\stindex{raise} statement.  For string exceptions, the
associated value itself will be stored in the variable named as the
second argument of the \keyword{except} clause (if any).  For class
exceptions, that variable receives the exception instance.  If the
exception class is derived from the standard root class
\exception{Exception}, the associated value is present as the
exception instance's \member{args} attribute, and possibly on other
attributes as well.

User code can raise built-in exceptions.  This can be used to test an
exception handler or to report an error condition ``just like'' the
situation in which the interpreter raises the same exception; but
beware that there is nothing to prevent user code from raising an
inappropriate error.

\setindexsubitem{(built-in exception base class)}

The following exceptions are only used as base classes for other
exceptions.

\begin{excdesc}{Exception}
The root class for exceptions.  All built-in exceptions are derived
from this class.  All user-defined exceptions should also be derived
from this class, but this is not (yet) enforced.  The \function{str()}
function, when applied to an instance of this class (or most derived
classes) returns the string value of the argument or arguments, or an
empty string if no arguments were given to the constructor.  When used
as a sequence, this accesses the arguments given to the constructor
(handy for backward compatibility with old code).  The arguments are
also available on the instance's \member{args} attribute, as a tuple.
\end{excdesc}

\begin{excdesc}{StandardError}
The base class for all built-in exceptions except
\exception{StopIteration} and \exception{SystemExit}.
\exception{StandardError} itself is derived from the root class
\exception{Exception}.
\end{excdesc}

\begin{excdesc}{ArithmeticError}
The base class for those built-in exceptions that are raised for
various arithmetic errors: \exception{OverflowError},
\exception{ZeroDivisionError}, \exception{FloatingPointError}.
\end{excdesc}

\begin{excdesc}{LookupError}
The base class for the exceptions that are raised when a key or
index used on a mapping or sequence is invalid: \exception{IndexError},
\exception{KeyError}.  This can be raised directly by
\function{sys.setdefaultencoding()}.
\end{excdesc}

\begin{excdesc}{EnvironmentError}
The base class for exceptions that
can occur outside the Python system: \exception{IOError},
\exception{OSError}.  When exceptions of this type are created with a
2-tuple, the first item is available on the instance's \member{errno}
attribute (it is assumed to be an error number), and the second item
is available on the \member{strerror} attribute (it is usually the
associated error message).  The tuple itself is also available on the
\member{args} attribute.
\versionadded{1.5.2}

When an \exception{EnvironmentError} exception is instantiated with a
3-tuple, the first two items are available as above, while the third
item is available on the \member{filename} attribute.  However, for
backwards compatibility, the \member{args} attribute contains only a
2-tuple of the first two constructor arguments.

The \member{filename} attribute is \code{None} when this exception is
created with other than 3 arguments.  The \member{errno} and
\member{strerror} attributes are also \code{None} when the instance was
created with other than 2 or 3 arguments.  In this last case,
\member{args} contains the verbatim constructor arguments as a tuple.
\end{excdesc}


\setindexsubitem{(built-in exception)}

The following exceptions are the exceptions that are actually raised.

\begin{excdesc}{AssertionError}
\stindex{assert}
Raised when an \keyword{assert} statement fails.
\end{excdesc}

\begin{excdesc}{AttributeError}
% xref to attribute reference?
  Raised when an attribute reference or assignment fails.  (When an
  object does not support attribute references or attribute assignments
  at all, \exception{TypeError} is raised.)
\end{excdesc}

\begin{excdesc}{EOFError}
% XXXJH xrefs here
  Raised when one of the built-in functions (\function{input()} or
  \function{raw_input()}) hits an end-of-file condition (\EOF{}) without
  reading any data.
% XXXJH xrefs here
  (N.B.: the \method{read()} and \method{readline()} methods of file
  objects return an empty string when they hit \EOF{}.)
\end{excdesc}

\begin{excdesc}{FloatingPointError}
  Raised when a floating point operation fails.  This exception is
  always defined, but can only be raised when Python is configured
  with the \longprogramopt{with-fpectl} option, or the
  \constant{WANT_SIGFPE_HANDLER} symbol is defined in the
  \file{config.h} file.
\end{excdesc}

\begin{excdesc}{IOError}
% XXXJH xrefs here
  Raised when an I/O operation (such as a \keyword{print} statement,
  the built-in \function{open()} function or a method of a file
  object) fails for an I/O-related reason, e.g., ``file not found'' or
  ``disk full''.

  This class is derived from \exception{EnvironmentError}.  See the
  discussion above for more information on exception instance
  attributes.
\end{excdesc}

\begin{excdesc}{ImportError}
% XXXJH xref to import statement?
  Raised when an \keyword{import} statement fails to find the module
  definition or when a \code{from \textrm{\ldots} import} fails to find a
  name that is to be imported.
\end{excdesc}

\begin{excdesc}{IndexError}
% XXXJH xref to sequences
  Raised when a sequence subscript is out of range.  (Slice indices are
  silently truncated to fall in the allowed range; if an index is not a
  plain integer, \exception{TypeError} is raised.)
\end{excdesc}

\begin{excdesc}{KeyError}
% XXXJH xref to mapping objects?
  Raised when a mapping (dictionary) key is not found in the set of
  existing keys.
\end{excdesc}

\begin{excdesc}{KeyboardInterrupt}
  Raised when the user hits the interrupt key (normally
  \kbd{Control-C} or \kbd{Delete}).  During execution, a check for
  interrupts is made regularly.
% XXXJH xrefs here
  Interrupts typed when a built-in function \function{input()} or
  \function{raw_input()}) is waiting for input also raise this
  exception.
\end{excdesc}

\begin{excdesc}{MemoryError}
  Raised when an operation runs out of memory but the situation may
  still be rescued (by deleting some objects).  The associated value is
  a string indicating what kind of (internal) operation ran out of memory.
  Note that because of the underlying memory management architecture
  (C's \cfunction{malloc()} function), the interpreter may not
  always be able to completely recover from this situation; it
  nevertheless raises an exception so that a stack traceback can be
  printed, in case a run-away program was the cause.
\end{excdesc}

\begin{excdesc}{NameError}
  Raised when a local or global name is not found.  This applies only
  to unqualified names.  The associated value is the name that could
  not be found.
\end{excdesc}

\begin{excdesc}{NotImplementedError}
  This exception is derived from \exception{RuntimeError}.  In user
  defined base classes, abstract methods should raise this exception
  when they require derived classes to override the method.
  \versionadded{1.5.2}
\end{excdesc}

\begin{excdesc}{OSError}
  %xref for os module
  This class is derived from \exception{EnvironmentError} and is used
  primarily as the \refmodule{os} module's \code{os.error} exception.
  See \exception{EnvironmentError} above for a description of the
  possible associated values.
  \versionadded{1.5.2}
\end{excdesc}

\begin{excdesc}{OverflowError}
% XXXJH reference to long's and/or int's?
  Raised when the result of an arithmetic operation is too large to be
  represented.  This cannot occur for long integers (which would rather
  raise \exception{MemoryError} than give up).  Because of the lack of
  standardization of floating point exception handling in C, most
  floating point operations also aren't checked.  For plain integers,
  all operations that can overflow are checked except left shift, where
  typical applications prefer to drop bits than raise an exception.
\end{excdesc}

\begin{excdesc}{RuntimeError}
  Raised when an error is detected that doesn't fall in any of the
  other categories.  The associated value is a string indicating what
  precisely went wrong.  (This exception is mostly a relic from a
  previous version of the interpreter; it is not used very much any
  more.)
\end{excdesc}

\begin{excdesc}{StopIteration}
  Raised by an iterator's \method{next()} method to signal that there
  are no further values.
  This is derived from \exception{Exception} rather than
  \exception{StandardError}, since this is not considered an error in
  its normal application.
  \versionadded{2.2}
\end{excdesc}

\begin{excdesc}{SyntaxError}
% XXXJH xref to these functions?
  Raised when the parser encounters a syntax error.  This may occur in
  an \keyword{import} statement, in an \keyword{exec} statement, in a call
  to the built-in function \function{eval()} or \function{input()}, or
  when reading the initial script or standard input (also
  interactively).

When class exceptions are used, instances of this class have
atttributes \member{filename}, \member{lineno}, \member{offset} and
\member{text} for easier access to the details; for string exceptions,
the associated value is usually a tuple of the form
\code{(message, (filename, lineno, offset, text))}.
For class exceptions, \function{str()} returns only the message.
\end{excdesc}

\begin{excdesc}{SystemError}
  Raised when the interpreter finds an internal error, but the
  situation does not look so serious to cause it to abandon all hope.
  The associated value is a string indicating what went wrong (in
  low-level terms).
  
  You should report this to the author or maintainer of your Python
  interpreter.  Be sure to report the version string of the Python
  interpreter (\code{sys.version}; it is also printed at the start of an
  interactive Python session), the exact error message (the exception's
  associated value) and if possible the source of the program that
  triggered the error.
\end{excdesc}

\begin{excdesc}{SystemExit}
% XXXJH xref to module sys?
  This exception is raised by the \function{sys.exit()} function.  When it
  is not handled, the Python interpreter exits; no stack traceback is
  printed.  If the associated value is a plain integer, it specifies the
  system exit status (passed to C's \cfunction{exit()} function); if it is
  \code{None}, the exit status is zero; if it has another type (such as
  a string), the object's value is printed and the exit status is one.

  Instances have an attribute \member{code} which is set to the
  proposed exit status or error message (defaulting to \code{None}).
  Also, this exception derives directly from \exception{Exception} and
  not \exception{StandardError}, since it is not technically an error.

  A call to \function{sys.exit()} is translated into an exception so that
  clean-up handlers (\keyword{finally} clauses of \keyword{try} statements)
  can be executed, and so that a debugger can execute a script without
  running the risk of losing control.  The \function{os._exit()} function
  can be used if it is absolutely positively necessary to exit
  immediately (for example, in the child process after a call to
  \function{fork()}).
\end{excdesc}

\begin{excdesc}{TypeError}
  Raised when a built-in operation or function is applied to an object
  of inappropriate type.  The associated value is a string giving
  details about the type mismatch.
\end{excdesc}

\begin{excdesc}{UnboundLocalError}
  Raised when a reference is made to a local variable in a function or
  method, but no value has been bound to that variable.  This is a
  subclass of \exception{NameError}.
\versionadded{2.0}
\end{excdesc}

\begin{excdesc}{UnicodeError}
  Raised when a Unicode-related encoding or decoding error occurs.  It
  is a subclass of \exception{ValueError}.
\versionadded{2.0}
\end{excdesc}

\begin{excdesc}{ValueError}
  Raised when a built-in operation or function receives an argument
  that has the right type but an inappropriate value, and the
  situation is not described by a more precise exception such as
  \exception{IndexError}.
\end{excdesc}

\begin{excdesc}{WindowsError}
  Raised when a Windows-specific error occurs or when the error number
  does not correspond to an \cdata{errno} value.  The
  \member{errno} and \member{strerror} values are created from the
  return values of the \cfunction{GetLastError()} and
  \cfunction{FormatMessage()} functions from the Windows Platform API.
  This is a subclass of \exception{OSError}.
\versionadded{2.0}
\end{excdesc}

\begin{excdesc}{ZeroDivisionError}
  Raised when the second argument of a division or modulo operation is
  zero.  The associated value is a string indicating the type of the
  operands and the operation.
\end{excdesc}


\setindexsubitem{(built-in warning category)}

The following exceptions are used as warning categories; see the
\module{warnings} module for more information.

\begin{excdesc}{Warning}
Base class for warning categories.
\end{excdesc}

\begin{excdesc}{UserWarning}
Base class for warnings generated by user code.
\end{excdesc}

\begin{excdesc}{DeprecationWarning}
Base class for warnings about deprecated features.
\end{excdesc}

\begin{excdesc}{SyntaxWarning}
Base class for warnings about dubious syntax
\end{excdesc}

\begin{excdesc}{RuntimeWarning}
Base class for warnings about dubious runtime behavior.
\end{excdesc}

\section{Built-in Functions \label{built-in-funcs}}

The Python interpreter has a number of functions built into it that
are always available.  They are listed here in alphabetical order.


\setindexsubitem{(built-in function)}

\begin{funcdesc}{__import__}{name\optional{, globals\optional{, locals\optional{, fromlist}}}}
  This function is invoked by the \keyword{import}\stindex{import}
  statement.  It mainly exists so that you can replace it with another
  function that has a compatible interface, in order to change the
  semantics of the \keyword{import} statement.  For examples of why
  and how you would do this, see the standard library modules
  \module{ihooks}\refstmodindex{ihooks} and
  \refmodule{rexec}\refstmodindex{rexec}.  See also the built-in
  module \refmodule{imp}\refbimodindex{imp}, which defines some useful
  operations out of which you can build your own
  \function{__import__()} function.

  For example, the statement \samp{import spam} results in the
  following call: \code{__import__('spam',} \code{globals(),}
  \code{locals(), [])}; the statement \samp{from spam.ham import eggs}
  results in \samp{__import__('spam.ham', globals(), locals(),
  ['eggs'])}.  Note that even though \code{locals()} and
  \code{['eggs']} are passed in as arguments, the
  \function{__import__()} function does not set the local variable
  named \code{eggs}; this is done by subsequent code that is generated
  for the import statement.  (In fact, the standard implementation
  does not use its \var{locals} argument at all, and uses its
  \var{globals} only to determine the package context of the
  \keyword{import} statement.)

  When the \var{name} variable is of the form \code{package.module},
  normally, the top-level package (the name up till the first dot) is
  returned, \emph{not} the module named by \var{name}.  However, when
  a non-empty \var{fromlist} argument is given, the module named by
  \var{name} is returned.  This is done for compatibility with the
  bytecode generated for the different kinds of import statement; when
  using \samp{import spam.ham.eggs}, the top-level package \module{spam}
  must be placed in the importing namespace, but when using \samp{from
  spam.ham import eggs}, the \code{spam.ham} subpackage must be used
  to find the \code{eggs} variable.  As a workaround for this
  behavior, use \function{getattr()} to extract the desired
  components.  For example, you could define the following helper:

\begin{verbatim}
def my_import(name):
    mod = __import__(name)
    components = name.split('.')
    for comp in components[1:]:
        mod = getattr(mod, comp)
    return mod
\end{verbatim}
\end{funcdesc}

\begin{funcdesc}{abs}{x}
  Return the absolute value of a number.  The argument may be a plain
  or long integer or a floating point number.  If the argument is a
  complex number, its magnitude is returned.
\end{funcdesc}

\begin{funcdesc}{basestring}{}
  This abstract type is the superclass for \class{str} and \class{unicode}.
  It cannot be called or instantiated, but it can be used to test whether
  an object is an instance of \class{str} or \class{unicode}.
  \code{isinstance(obj, basestring)} is equivalent to
  \code{isinstance(obj, (str, unicode))}.
  \versionadded{2.3}
\end{funcdesc}

\begin{funcdesc}{bool}{\optional{x}}
  Convert a value to a Boolean, using the standard truth testing
  procedure.  If \var{x} is false or omitted, this returns
  \constant{False}; otherwise it returns \constant{True}.
  \class{bool} is also a class, which is a subclass of \class{int}.
  Class \class{bool} cannot be subclassed further.  Its only instances
  are \constant{False} and \constant{True}.

  \indexii{Boolean}{type}
  \versionadded{2.2.1}
  \versionchanged[If no argument is given, this function returns 
                  \constant{False}]{2.3}
\end{funcdesc}

\begin{funcdesc}{callable}{object}
  Return true if the \var{object} argument appears callable, false if
  not.  If this returns true, it is still possible that a call fails,
  but if it is false, calling \var{object} will never succeed.  Note
  that classes are callable (calling a class returns a new instance);
  class instances are callable if they have a \method{__call__()}
  method.
\end{funcdesc}

\begin{funcdesc}{chr}{i}
  Return a string of one character whose \ASCII{} code is the integer
  \var{i}.  For example, \code{chr(97)} returns the string \code{'a'}.
  This is the inverse of \function{ord()}.  The argument must be in
  the range [0..255], inclusive; \exception{ValueError} will be raised
  if \var{i} is outside that range.
\end{funcdesc}

\begin{funcdesc}{classmethod}{function}
  Return a class method for \var{function}.

  A class method receives the class as implicit first argument,
  just like an instance method receives the instance.
  To declare a class method, use this idiom:

\begin{verbatim}
class C:
    def f(cls, arg1, arg2, ...): ...
    f = classmethod(f)
\end{verbatim}

  It can be called either on the class (such as \code{C.f()}) or on an
  instance (such as \code{C().f()}).  The instance is ignored except for
  its class.
  If a class method is called for a derived class, the derived class
  object is passed as the implied first argument.

  Class methods are different than \Cpp{} or Java static methods.
  If you want those, see \function{staticmethod()} in this section.
  \versionadded{2.2}
\end{funcdesc}

\begin{funcdesc}{cmp}{x, y}
  Compare the two objects \var{x} and \var{y} and return an integer
  according to the outcome.  The return value is negative if \code{\var{x}
  < \var{y}}, zero if \code{\var{x} == \var{y}} and strictly positive if
  \code{\var{x} > \var{y}}.
\end{funcdesc}

\begin{funcdesc}{compile}{string, filename, kind\optional{,
                          flags\optional{, dont_inherit}}}
  Compile the \var{string} into a code object.  Code objects can be
  executed by an \keyword{exec} statement or evaluated by a call to
  \function{eval()}.  The \var{filename} argument should
  give the file from which the code was read; pass some recognizable value
  if it wasn't read from a file (\code{'<string>'} is commonly used).
  The \var{kind} argument specifies what kind of code must be
  compiled; it can be \code{'exec'} if \var{string} consists of a
  sequence of statements, \code{'eval'} if it consists of a single
  expression, or \code{'single'} if it consists of a single
  interactive statement (in the latter case, expression statements
  that evaluate to something else than \code{None} will printed).

  When compiling multi-line statements, two caveats apply: line
  endings must be represented by a single newline character
  (\code{'\e n'}), and the input must be terminated by at least one
  newline character.  If line endings are represented by
  \code{'\e r\e n'}, use the string \method{replace()} method to
  change them into \code{'\e n'}.

  The optional arguments \var{flags} and \var{dont_inherit}
  (which are new in Python 2.2) control which future statements (see
  \pep{236}) affect the compilation of \var{string}.  If neither is
  present (or both are zero) the code is compiled with those future
  statements that are in effect in the code that is calling compile.
  If the \var{flags} argument is given and \var{dont_inherit} is not
  (or is zero) then the future statements specified by the \var{flags}
  argument are used in addition to those that would be used anyway.
  If \var{dont_inherit} is a non-zero integer then the \var{flags}
  argument is it -- the future statements in effect around the call to
  compile are ignored.

  Future statemants are specified by bits which can be bitwise or-ed
  together to specify multiple statements.  The bitfield required to
  specify a given feature can be found as the \member{compiler_flag}
  attribute on the \class{_Feature} instance in the
  \module{__future__} module.
\end{funcdesc}

\begin{funcdesc}{complex}{\optional{real\optional{, imag}}}
  Create a complex number with the value \var{real} + \var{imag}*j or
  convert a string or number to a complex number.  If the first
  parameter is a string, it will be interpreted as a complex number
  and the function must be called without a second parameter.  The
  second parameter can never be a string.
  Each argument may be any numeric type (including complex).
  If \var{imag} is omitted, it defaults to zero and the function
  serves as a numeric conversion function like \function{int()},
  \function{long()} and \function{float()}.  If both arguments
  are omitted, returns \code{0j}.
\end{funcdesc}

\begin{funcdesc}{delattr}{object, name}
  This is a relative of \function{setattr()}.  The arguments are an
  object and a string.  The string must be the name
  of one of the object's attributes.  The function deletes
  the named attribute, provided the object allows it.  For example,
  \code{delattr(\var{x}, '\var{foobar}')} is equivalent to
  \code{del \var{x}.\var{foobar}}.
\end{funcdesc}

\begin{funcdesc}{dict}{\optional{mapping-or-sequence}}
  Return a new dictionary initialized from an optional positional
  argument or from a set of keyword arguments.
  If no arguments are given, return a new empty dictionary.
  If the positional argument is a mapping object, return a dictionary
  mapping the same keys to the same values as does the mapping object.
  Otherwise the positional argument must be a sequence, a container that
  supports iteration, or an iterator object.  The elements of the argument
  must each also be of one of those kinds, and each must in turn contain
  exactly two objects.  The first is used as a key in the new dictionary,
  and the second as the key's value.  If a given key is seen more than
  once, the last value associated with it is retained in the new
  dictionary.

  If keyword arguments are given, the keywords themselves with their
  associated values are added as items to the dictionary. If a key
  is specified both in the positional argument and as a keyword argument,
  the value associated with the keyword is retained in the dictionary.
  For example, these all return a dictionary equal to
  \code{\{"one": 2, "two": 3\}}:

  \begin{itemize}
    \item \code{dict(\{'one': 2, 'two': 3\})}
    \item \code{dict(\{'one': 2, 'two': 3\}.items())}
    \item \code{dict(\{'one': 2, 'two': 3\}.iteritems())}
    \item \code{dict(zip(('one', 'two'), (2, 3)))}
    \item \code{dict([['two', 3], ['one', 2]])}
    \item \code{dict(one=2, two=3)}
    \item \code{dict([(['one', 'two'][i-2], i) for i in (2, 3)])}
  \end{itemize}

  \versionadded{2.2}
  \versionchanged[Support for building a dictionary from keyword
                  arguments added]{2.3}
\end{funcdesc}

\begin{funcdesc}{dir}{\optional{object}}
  Without arguments, return the list of names in the current local
  symbol table.  With an argument, attempts to return a list of valid
  attributes for that object.  This information is gleaned from the
  object's \member{__dict__} attribute, if defined, and from the class
  or type object.  The list is not necessarily complete.
  If the object is a module object, the list contains the names of the
  module's attributes.
  If the object is a type or class object,
  the list contains the names of its attributes,
  and recursively of the attributes of its bases.
  Otherwise, the list contains the object's attributes' names,
  the names of its class's attributes,
  and recursively of the attributes of its class's base classes.
  The resulting list is sorted alphabetically.
  For example:

\begin{verbatim}
>>> import struct
>>> dir()
['__builtins__', '__doc__', '__name__', 'struct']
>>> dir(struct)
['__doc__', '__name__', 'calcsize', 'error', 'pack', 'unpack']
\end{verbatim}

  \note{Because \function{dir()} is supplied primarily as a convenience
  for use at an interactive prompt,
  it tries to supply an interesting set of names more than it tries to
  supply a rigorously or consistently defined set of names,
  and its detailed behavior may change across releases.}
\end{funcdesc}

\begin{funcdesc}{divmod}{a, b}
  Take two (non complex) numbers as arguments and return a pair of numbers
  consisting of their quotient and remainder when using long division.  With
  mixed operand types, the rules for binary arithmetic operators apply.  For
  plain and long integers, the result is the same as
  \code{(\var{a} / \var{b}, \var{a} \%{} \var{b})}.
  For floating point numbers the result is \code{(\var{q}, \var{a} \%{}
  \var{b})}, where \var{q} is usually \code{math.floor(\var{a} /
  \var{b})} but may be 1 less than that.  In any case \code{\var{q} *
  \var{b} + \var{a} \%{} \var{b}} is very close to \var{a}, if
  \code{\var{a} \%{} \var{b}} is non-zero it has the same sign as
  \var{b}, and \code{0 <= abs(\var{a} \%{} \var{b}) < abs(\var{b})}.

  \versionchanged[Using \function{divmod()} with complex numbers is
                  deprecated]{2.3}
\end{funcdesc}

\begin{funcdesc}{enumerate}{iterable}
  Return an enumerate object. \var{iterable} must be a sequence, an
  iterator, or some other object which supports iteration.  The
  \method{next()} method of the iterator returned by
  \function{enumerate()} returns a tuple containing a count (from
  zero) and the corresponding value obtained from iterating over
  \var{iterable}.  \function{enumerate()} is useful for obtaining an
  indexed series: \code{(0, seq[0])}, \code{(1, seq[1])}, \code{(2,
  seq[2])}, \ldots.
  \versionadded{2.3}
\end{funcdesc}

\begin{funcdesc}{eval}{expression\optional{, globals\optional{, locals}}}
  The arguments are a string and two optional dictionaries.  The
  \var{expression} argument is parsed and evaluated as a Python
  expression (technically speaking, a condition list) using the
  \var{globals} and \var{locals} dictionaries as global and local name
  space.  If the \var{globals} dictionary is present and lacks
  '__builtins__', the current globals are copied into \var{globals} before
  \var{expression} is parsed.  This means that \var{expression}
  normally has full access to the standard
  \refmodule[builtin]{__builtin__} module and restricted environments
  are propagated.  If the \var{locals} dictionary is omitted it defaults to
  the \var{globals} dictionary.  If both dictionaries are omitted, the
  expression is executed in the environment where \keyword{eval} is
  called.  The return value is the result of the evaluated expression.
  Syntax errors are reported as exceptions.  Example:

\begin{verbatim}
>>> x = 1
>>> print eval('x+1')
2
\end{verbatim}

  This function can also be used to execute arbitrary code objects
  (such as those created by \function{compile()}).  In this case pass
  a code object instead of a string.  The code object must have been
  compiled passing \code{'eval'} as the \var{kind} argument.

  Hints: dynamic execution of statements is supported by the
  \keyword{exec} statement.  Execution of statements from a file is
  supported by the \function{execfile()} function.  The
  \function{globals()} and \function{locals()} functions returns the
  current global and local dictionary, respectively, which may be
  useful to pass around for use by \function{eval()} or
  \function{execfile()}.
\end{funcdesc}

\begin{funcdesc}{execfile}{filename\optional{, globals\optional{, locals}}}
  This function is similar to the
  \keyword{exec} statement, but parses a file instead of a string.  It
  is different from the \keyword{import} statement in that it does not
  use the module administration --- it reads the file unconditionally
  and does not create a new module.\footnote{It is used relatively
  rarely so does not warrant being made into a statement.}

  The arguments are a file name and two optional dictionaries.  The
  file is parsed and evaluated as a sequence of Python statements
  (similarly to a module) using the \var{globals} and \var{locals}
  dictionaries as global and local namespace.  If the \var{locals}
  dictionary is omitted it defaults to the \var{globals} dictionary.
  If both dictionaries are omitted, the expression is executed in the
  environment where \function{execfile()} is called.  The return value is
  \code{None}.

  \warning{The default \var{locals} act as described for function
  \function{locals()} below:  modifications to the default \var{locals}
  dictionary should not be attempted.  Pass an explicit \var{locals}
  dictionary if you need to see effects of the code on \var{locals} after
  function \function{execfile()} returns.  \function{execfile()} cannot
  be used reliably to modify a function's locals.}
\end{funcdesc}

\begin{funcdesc}{file}{filename\optional{, mode\optional{, bufsize}}}
  Return a new file object (described in
  section~\ref{bltin-file-objects}, ``\ulink{File
  Objects}{bltin-file-objects.html}'').
  The first two arguments are the same as for \code{stdio}'s
  \cfunction{fopen()}: \var{filename} is the file name to be opened,
  \var{mode} indicates how the file is to be opened: \code{'r'} for
  reading, \code{'w'} for writing (truncating an existing file), and
  \code{'a'} opens it for appending (which on \emph{some} \UNIX{}
  systems means that \emph{all} writes append to the end of the file,
  regardless of the current seek position).

  Modes \code{'r+'}, \code{'w+'} and \code{'a+'} open the file for
  updating (note that \code{'w+'} truncates the file).  Append
  \code{'b'} to the mode to open the file in binary mode, on systems
  that differentiate between binary and text files (else it is
  ignored).  If the file cannot be opened, \exception{IOError} is
  raised.
  
  In addition to the standard \cfunction{fopen()} values \var{mode}
  may be \code{'U'} or \code{'rU'}. If Python is built with universal
  newline support (the default) the file is opened as a text file, but
  lines may be terminated by any of \code{'\e n'}, the Unix end-of-line
  convention,
  \code{'\e r'}, the Macintosh convention or \code{'\e r\e n'}, the Windows
  convention. All of these external representations are seen as
  \code{'\e n'}
  by the Python program. If Python is built without universal newline support
  \var{mode} \code{'U'} is the same as normal text mode.  Note that
  file objects so opened also have an attribute called
  \member{newlines} which has a value of \code{None} (if no newlines
  have yet been seen), \code{'\e n'}, \code{'\e r'}, \code{'\e r\e n'}, 
  or a tuple containing all the newline types seen.

  If \var{mode} is omitted, it defaults to \code{'r'}.  When opening a
  binary file, you should append \code{'b'} to the \var{mode} value
  for improved portability.  (It's useful even on systems which don't
  treat binary and text files differently, where it serves as
  documentation.)
  \index{line-buffered I/O}\index{unbuffered I/O}\index{buffer size, I/O}
  \index{I/O control!buffering}
  The optional \var{bufsize} argument specifies the
  file's desired buffer size: 0 means unbuffered, 1 means line
  buffered, any other positive value means use a buffer of
  (approximately) that size.  A negative \var{bufsize} means to use
  the system default, which is usually line buffered for tty
  devices and fully buffered for other files.  If omitted, the system
  default is used.\footnote{
    Specifying a buffer size currently has no effect on systems that
    don't have \cfunction{setvbuf()}.  The interface to specify the
    buffer size is not done using a method that calls
    \cfunction{setvbuf()}, because that may dump core when called
    after any I/O has been performed, and there's no reliable way to
    determine whether this is the case.}

  The \function{file()} constructor is new in Python 2.2.  The previous
  spelling, \function{open()}, is retained for compatibility, and is an
  alias for \function{file()}.
\end{funcdesc}

\begin{funcdesc}{filter}{function, list}
  Construct a list from those elements of \var{list} for which
  \var{function} returns true.  \var{list} may be either a sequence, a
  container which supports iteration, or an iterator,  If \var{list}
  is a string or a tuple, the result also has that type; otherwise it
  is always a list.  If \var{function} is \code{None}, the identity
  function is assumed, that is, all elements of \var{list} that are false
  (zero or empty) are removed.

  Note that \code{filter(function, \var{list})} is equivalent to
  \code{[item for item in \var{list} if function(item)]} if function is
  not \code{None} and \code{[item for item in \var{list} if item]} if
  function is \code{None}.
\end{funcdesc}

\begin{funcdesc}{float}{\optional{x}}
  Convert a string or a number to floating point.  If the argument is a
  string, it must contain a possibly signed decimal or floating point
  number, possibly embedded in whitespace. Otherwise, the argument may be a plain
  or long integer or a floating point number, and a floating point
  number with the same value (within Python's floating point
  precision) is returned.  If no argument is given, returns \code{0.0}.

  \note{When passing in a string, values for NaN\index{NaN}
  and Infinity\index{Infinity} may be returned, depending on the
  underlying C library.  The specific set of strings accepted which
  cause these values to be returned depends entirely on the C library
  and is known to vary.}
\end{funcdesc}

\begin{funcdesc}{frozenset}{\optional{iterable}}
  Return a frozenset object whose elements are taken from \var{iterable}.
  Frozensets are sets that have no update methods but can be hashed and
  used as members of other sets or as dictionary keys.  The elements of
  a frozenset must be immutable themselves.  To represent sets of sets,
  the inner sets should also be \class{frozenset} objects.  If
  \var{iterable} is not specified, returns a new empty set,
  \code{frozenset([])}.
  \versionadded{2.4}  
\end{funcdesc}

\begin{funcdesc}{getattr}{object, name\optional{, default}}
  Return the value of the named attributed of \var{object}.  \var{name}
  must be a string.  If the string is the name of one of the object's
  attributes, the result is the value of that attribute.  For example,
  \code{getattr(x, 'foobar')} is equivalent to \code{x.foobar}.  If the
  named attribute does not exist, \var{default} is returned if provided,
  otherwise \exception{AttributeError} is raised.
\end{funcdesc}

\begin{funcdesc}{globals}{}
  Return a dictionary representing the current global symbol table.
  This is always the dictionary of the current module (inside a
  function or method, this is the module where it is defined, not the
  module from which it is called).
\end{funcdesc}

\begin{funcdesc}{hasattr}{object, name}
  The arguments are an object and a string.  The result is \code{True} if the
  string is the name of one of the object's attributes, \code{False} if not.
  (This is implemented by calling \code{getattr(\var{object},
  \var{name})} and seeing whether it raises an exception or not.)
\end{funcdesc}

\begin{funcdesc}{hash}{object}
  Return the hash value of the object (if it has one).  Hash values
  are integers.  They are used to quickly compare dictionary
  keys during a dictionary lookup.  Numeric values that compare equal
  have the same hash value (even if they are of different types, as is
  the case for 1 and 1.0).
\end{funcdesc}

\begin{funcdesc}{help}{\optional{object}}
  Invoke the built-in help system.  (This function is intended for
  interactive use.)  If no argument is given, the interactive help
  system starts on the interpreter console.  If the argument is a
  string, then the string is looked up as the name of a module,
  function, class, method, keyword, or documentation topic, and a
  help page is printed on the console.  If the argument is any other
  kind of object, a help page on the object is generated.
  \versionadded{2.2}
\end{funcdesc}

\begin{funcdesc}{hex}{x}
  Convert an integer number (of any size) to a hexadecimal string.
  The result is a valid Python expression.  Note: this always yields
  an unsigned literal.  For example, on a 32-bit machine,
  \code{hex(-1)} yields \code{'0xffffffff'}.  When evaluated on a
  machine with the same word size, this literal is evaluated as -1; at
  a different word size, it may turn up as a large positive number or
  raise an \exception{OverflowError} exception.
\end{funcdesc}

\begin{funcdesc}{id}{object}
  Return the `identity' of an object.  This is an integer (or long
  integer) which is guaranteed to be unique and constant for this
  object during its lifetime.  Two objects whose lifetimes are
  disjunct may have the same \function{id()} value.  (Implementation
  note: this is the address of the object.)
\end{funcdesc}

\begin{funcdesc}{input}{\optional{prompt}}
  Equivalent to \code{eval(raw_input(\var{prompt}))}.
  \warning{This function is not safe from user errors!  It
  expects a valid Python expression as input; if the input is not
  syntactically valid, a \exception{SyntaxError} will be raised.
  Other exceptions may be raised if there is an error during
  evaluation.  (On the other hand, sometimes this is exactly what you
  need when writing a quick script for expert use.)}

  If the \refmodule{readline} module was loaded, then
  \function{input()} will use it to provide elaborate line editing and
  history features.

  Consider using the \function{raw_input()} function for general input
  from users.
\end{funcdesc}

\begin{funcdesc}{int}{\optional{x\optional{, radix}}}
  Convert a string or number to a plain integer.  If the argument is a
  string, it must contain a possibly signed decimal number
  representable as a Python integer, possibly embedded in whitespace.
  The \var{radix} parameter gives the base for the
  conversion and may be any integer in the range [2, 36], or zero.  If
  \var{radix} is zero, the proper radix is guessed based on the
  contents of string; the interpretation is the same as for integer
  literals.  If \var{radix} is specified and \var{x} is not a string,
  \exception{TypeError} is raised.
  Otherwise, the argument may be a plain or
  long integer or a floating point number.  Conversion of floating
  point numbers to integers truncates (towards zero).
  If the argument is outside the integer range a long object will
  be returned instead.  If no arguments are given, returns \code{0}.
\end{funcdesc}

\begin{funcdesc}{isinstance}{object, classinfo}
  Return true if the \var{object} argument is an instance of the
  \var{classinfo} argument, or of a (direct or indirect) subclass
  thereof.  Also return true if \var{classinfo} is a type object and
  \var{object} is an object of that type.  If \var{object} is not a
  class instance or an object of the given type, the function always
  returns false.  If \var{classinfo} is neither a class object nor a
  type object, it may be a tuple of class or type objects, or may
  recursively contain other such tuples (other sequence types are not
  accepted).  If \var{classinfo} is not a class, type, or tuple of
  classes, types, and such tuples, a \exception{TypeError} exception
  is raised.
  \versionchanged[Support for a tuple of type information was added]{2.2}
\end{funcdesc}

\begin{funcdesc}{issubclass}{class, classinfo}
  Return true if \var{class} is a subclass (direct or indirect) of
  \var{classinfo}.  A class is considered a subclass of itself.
  \var{classinfo} may be a tuple of class objects, in which case every
  entry in \var{classinfo} will be checked. In any other case, a
  \exception{TypeError} exception is raised.
  \versionchanged[Support for a tuple of type information was added]{2.3}
\end{funcdesc}

\begin{funcdesc}{iter}{o\optional{, sentinel}}
  Return an iterator object.  The first argument is interpreted very
  differently depending on the presence of the second argument.
  Without a second argument, \var{o} must be a collection object which
  supports the iteration protocol (the \method{__iter__()} method), or
  it must support the sequence protocol (the \method{__getitem__()}
  method with integer arguments starting at \code{0}).  If it does not
  support either of those protocols, \exception{TypeError} is raised.
  If the second argument, \var{sentinel}, is given, then \var{o} must
  be a callable object.  The iterator created in this case will call
  \var{o} with no arguments for each call to its \method{next()}
  method; if the value returned is equal to \var{sentinel},
  \exception{StopIteration} will be raised, otherwise the value will
  be returned.
  \versionadded{2.2}
\end{funcdesc}

\begin{funcdesc}{len}{s}
  Return the length (the number of items) of an object.  The argument
  may be a sequence (string, tuple or list) or a mapping (dictionary).
\end{funcdesc}

\begin{funcdesc}{list}{\optional{sequence}}
  Return a list whose items are the same and in the same order as
  \var{sequence}'s items.  \var{sequence} may be either a sequence, a
  container that supports iteration, or an iterator object.  If
  \var{sequence} is already a list, a copy is made and returned,
  similar to \code{\var{sequence}[:]}.  For instance,
  \code{list('abc')} returns \code{['a', 'b', 'c']} and \code{list(
  (1, 2, 3) )} returns \code{[1, 2, 3]}.  If no argument is given,
  returns a new empty list, \code{[]}.
\end{funcdesc}

\begin{funcdesc}{locals}{}
  Update and return a dictionary representing the current local symbol table.
  \warning{The contents of this dictionary should not be modified;
  changes may not affect the values of local variables used by the
  interpreter.}
\end{funcdesc}

\begin{funcdesc}{long}{\optional{x\optional{, radix}}}
  Convert a string or number to a long integer.  If the argument is a
  string, it must contain a possibly signed number of
  arbitrary size, possibly embedded in whitespace. The
  \var{radix} argument is interpreted in the same way as for
  \function{int()}, and may only be given when \var{x} is a string.
  Otherwise, the argument may be a plain or
  long integer or a floating point number, and a long integer with
  the same value is returned.    Conversion of floating
  point numbers to integers truncates (towards zero).  If no arguments
  are given, returns \code{0L}.
\end{funcdesc}

\begin{funcdesc}{map}{function, list, ...}
  Apply \var{function} to every item of \var{list} and return a list
  of the results.  If additional \var{list} arguments are passed,
  \var{function} must take that many arguments and is applied to the
  items of all lists in parallel; if a list is shorter than another it
  is assumed to be extended with \code{None} items.  If \var{function}
  is \code{None}, the identity function is assumed; if there are
  multiple list arguments, \function{map()} returns a list consisting
  of tuples containing the corresponding items from all lists (a kind
  of transpose operation).  The \var{list} arguments may be any kind
  of sequence; the result is always a list.
\end{funcdesc}

\begin{funcdesc}{max}{s\optional{, args...}}
  With a single argument \var{s}, return the largest item of a
  non-empty sequence (such as a string, tuple or list).  With more
  than one argument, return the largest of the arguments.
\end{funcdesc}

\begin{funcdesc}{min}{s\optional{, args...}}
  With a single argument \var{s}, return the smallest item of a
  non-empty sequence (such as a string, tuple or list).  With more
  than one argument, return the smallest of the arguments.
\end{funcdesc}

\begin{funcdesc}{object}{}
  Return a new featureless object.  \function{object()} is a base 
  for all new style classes.  It has the methods that are common
  to all instances of new style classes.
  \versionadded{2.2}

  \versionchanged[This function does not accept any arguments.
  Formerly, it accepted arguments but ignored them]{2.3}
\end{funcdesc}

\begin{funcdesc}{oct}{x}
  Convert an integer number (of any size) to an octal string.  The
  result is a valid Python expression.  Note: this always yields an
  unsigned literal.  For example, on a 32-bit machine, \code{oct(-1)}
  yields \code{'037777777777'}.  When evaluated on a machine with the
  same word size, this literal is evaluated as -1; at a different word
  size, it may turn up as a large positive number or raise an
  \exception{OverflowError} exception.
\end{funcdesc}

\begin{funcdesc}{open}{filename\optional{, mode\optional{, bufsize}}}
  An alias for the \function{file()} function above.
\end{funcdesc}

\begin{funcdesc}{ord}{c}
  Return the \ASCII{} value of a string of one character or a Unicode
  character.  E.g., \code{ord('a')} returns the integer \code{97},
  \code{ord(u'\e u2020')} returns \code{8224}.  This is the inverse of
  \function{chr()} for strings and of \function{unichr()} for Unicode
  characters.
\end{funcdesc}

\begin{funcdesc}{pow}{x, y\optional{, z}}
  Return \var{x} to the power \var{y}; if \var{z} is present, return
  \var{x} to the power \var{y}, modulo \var{z} (computed more
  efficiently than \code{pow(\var{x}, \var{y}) \%\ \var{z}}).  The
  arguments must have numeric types.  With mixed operand types, the
  coercion rules for binary arithmetic operators apply.  For int and
  long int operands, the result has the same type as the operands
  (after coercion) unless the second argument is negative; in that
  case, all arguments are converted to float and a float result is
  delivered.  For example, \code{10**2} returns \code{100}, but
  \code{10**-2} returns \code{0.01}.  (This last feature was added in
  Python 2.2.  In Python 2.1 and before, if both arguments were of integer
  types and the second argument was negative, an exception was raised.)
  If the second argument is negative, the third argument must be omitted.
  If \var{z} is present, \var{x} and \var{y} must be of integer types,
  and \var{y} must be non-negative.  (This restriction was added in
  Python 2.2.  In Python 2.1 and before, floating 3-argument \code{pow()}
  returned platform-dependent results depending on floating-point
  rounding accidents.)
\end{funcdesc}

\begin{funcdesc}{property}{\optional{fget\optional{, fset\optional{,
                           fdel\optional{, doc}}}}}
  Return a property attribute for new-style classes (classes that
  derive from \class{object}).

  \var{fget} is a function for getting an attribute value, likewise
  \var{fset} is a function for setting, and \var{fdel} a function
  for del'ing, an attribute.  Typical use is to define a managed attribute x:

\begin{verbatim}
class C(object):
    def getx(self): return self.__x
    def setx(self, value): self.__x = value
    def delx(self): del self.__x
    x = property(getx, setx, delx, "I'm the 'x' property.")
\end{verbatim}

  \versionadded{2.2}
\end{funcdesc}

\begin{funcdesc}{range}{\optional{start,} stop\optional{, step}}
  This is a versatile function to create lists containing arithmetic
  progressions.  It is most often used in \keyword{for} loops.  The
  arguments must be plain integers.  If the \var{step} argument is
  omitted, it defaults to \code{1}.  If the \var{start} argument is
  omitted, it defaults to \code{0}.  The full form returns a list of
  plain integers \code{[\var{start}, \var{start} + \var{step},
  \var{start} + 2 * \var{step}, \ldots]}.  If \var{step} is positive,
  the last element is the largest \code{\var{start} + \var{i} *
  \var{step}} less than \var{stop}; if \var{step} is negative, the last
  element is the largest \code{\var{start} + \var{i} * \var{step}}
  greater than \var{stop}.  \var{step} must not be zero (or else
  \exception{ValueError} is raised).  Example:

\begin{verbatim}
>>> range(10)
[0, 1, 2, 3, 4, 5, 6, 7, 8, 9]
>>> range(1, 11)
[1, 2, 3, 4, 5, 6, 7, 8, 9, 10]
>>> range(0, 30, 5)
[0, 5, 10, 15, 20, 25]
>>> range(0, 10, 3)
[0, 3, 6, 9]
>>> range(0, -10, -1)
[0, -1, -2, -3, -4, -5, -6, -7, -8, -9]
>>> range(0)
[]
>>> range(1, 0)
[]
\end{verbatim}
\end{funcdesc}

\begin{funcdesc}{raw_input}{\optional{prompt}}
  If the \var{prompt} argument is present, it is written to standard output
  without a trailing newline.  The function then reads a line from input,
  converts it to a string (stripping a trailing newline), and returns that.
  When \EOF{} is read, \exception{EOFError} is raised. Example:

\begin{verbatim}
>>> s = raw_input('--> ')
--> Monty Python's Flying Circus
>>> s
"Monty Python's Flying Circus"
\end{verbatim}

  If the \refmodule{readline} module was loaded, then
  \function{raw_input()} will use it to provide elaborate
  line editing and history features.
\end{funcdesc}

\begin{funcdesc}{reduce}{function, sequence\optional{, initializer}}
  Apply \var{function} of two arguments cumulatively to the items of
  \var{sequence}, from left to right, so as to reduce the sequence to
  a single value.  For example, \code{reduce(lambda x, y: x+y, [1, 2,
  3, 4, 5])} calculates \code{((((1+2)+3)+4)+5)}.  The left argument,
  \var{x}, is the accumulated value and the right argument, \var{y},
  is the update value from the \var{sequence}.  If the optional
  \var{initializer} is present, it is placed before the items of the
  sequence in the calculation, and serves as a default when the
  sequence is empty.  If \var{initializer} is not given and
  \var{sequence} contains only one item, the first item is returned.
\end{funcdesc}

\begin{funcdesc}{reload}{module}
  Reload a previously imported \var{module}.  The
  argument must be a module object, so it must have been successfully
  imported before.  This is useful if you have edited the module
  source file using an external editor and want to try out the new
  version without leaving the Python interpreter.  The return value is
  the module object (the same as the \var{module} argument).

  When \code{reload(module)} is executed:

\begin{itemize}

    \item{Python modules' code is recompiled and the module-level code
    reexecuted, defining a new set of objects which are bound to names in
    the module's dictionary.  The \code{init} function of extension
    modules is not called a second time.}

    \item{As with all other objects in Python the old objects are only
    reclaimed after their reference counts drop to zero.}

    \item{The names in the module namespace are updated to point to
    any new or changed objects.}

    \item{Other references to the old objects (such as names external
    to the module) are not rebound to refer to the new objects and
    must be updated in each namespace where they occur if that is
    desired.}

\end{itemize}

  There are a number of other caveats:

  If a module is syntactically correct but its initialization fails,
  the first \keyword{import} statement for it does not bind its name
  locally, but does store a (partially initialized) module object in
  \code{sys.modules}.  To reload the module you must first
  \keyword{import} it again (this will bind the name to the partially
  initialized module object) before you can \function{reload()} it.

  When a module is reloaded, its dictionary (containing the module's
  global variables) is retained.  Redefinitions of names will override
  the old definitions, so this is generally not a problem.  If the new
  version of a module does not define a name that was defined by the
  old version, the old definition remains.  This feature can be used
  to the module's advantage if it maintains a global table or cache of
  objects --- with a \keyword{try} statement it can test for the
  table's presence and skip its initialization if desired.

  It is legal though generally not very useful to reload built-in or
  dynamically loaded modules, except for \refmodule{sys},
  \refmodule[main]{__main__} and \refmodule[builtin]{__builtin__}.  In
  many cases, however, extension modules are not designed to be
  initialized more than once, and may fail in arbitrary ways when
  reloaded.

  If a module imports objects from another module using \keyword{from}
  \ldots{} \keyword{import} \ldots{}, calling \function{reload()} for
  the other module does not redefine the objects imported from it ---
  one way around this is to re-execute the \keyword{from} statement,
  another is to use \keyword{import} and qualified names
  (\var{module}.\var{name}) instead.

  If a module instantiates instances of a class, reloading the module
  that defines the class does not affect the method definitions of the
  instances --- they continue to use the old class definition.  The
  same is true for derived classes.
\end{funcdesc}

\begin{funcdesc}{repr}{object}
  Return a string containing a printable representation of an object.
  This is the same value yielded by conversions (reverse quotes).
  It is sometimes useful to be able to access this operation as an
  ordinary function.  For many types, this function makes an attempt
  to return a string that would yield an object with the same value
  when passed to \function{eval()}.
\end{funcdesc}

\begin{funcdesc}{reversed}{seq}
  Return a reverse iterator.  \var{seq} must be an object which
  supports the sequence protocol (the __len__() method and the
  \method{__getitem__()} method with integer arguments starting at
  \code{0}).
  \versionadded{2.4}
\end{funcdesc}

\begin{funcdesc}{round}{x\optional{, n}}
  Return the floating point value \var{x} rounded to \var{n} digits
  after the decimal point.  If \var{n} is omitted, it defaults to zero.
  The result is a floating point number.  Values are rounded to the
  closest multiple of 10 to the power minus \var{n}; if two multiples
  are equally close, rounding is done away from 0 (so. for example,
  \code{round(0.5)} is \code{1.0} and \code{round(-0.5)} is \code{-1.0}).
\end{funcdesc}

\begin{funcdesc}{set}{\optional{iterable}}
  Return a set whose elements are taken from \var{iterable}.  The elements
  must be immutable.  To represent sets of sets, the inner sets should
  be \class{frozenset} objects.  If \var{iterable} is not specified,
  returns a new empty set, \code{set([])}.
  \versionadded{2.4}  
\end{funcdesc}

\begin{funcdesc}{setattr}{object, name, value}
  This is the counterpart of \function{getattr()}.  The arguments are an
  object, a string and an arbitrary value.  The string may name an
  existing attribute or a new attribute.  The function assigns the
  value to the attribute, provided the object allows it.  For example,
  \code{setattr(\var{x}, '\var{foobar}', 123)} is equivalent to
  \code{\var{x}.\var{foobar} = 123}.
\end{funcdesc}

\begin{funcdesc}{slice}{\optional{start,} stop\optional{, step}}
  Return a slice object representing the set of indices specified by
  \code{range(\var{start}, \var{stop}, \var{step})}.  The \var{start}
  and \var{step} arguments default to \code{None}.  Slice objects have
  read-only data attributes \member{start}, \member{stop} and
  \member{step} which merely return the argument values (or their
  default).  They have no other explicit functionality; however they
  are used by Numerical Python\index{Numerical Python} and other third
  party extensions.  Slice objects are also generated when extended
  indexing syntax is used.  For example: \samp{a[start:stop:step]} or
  \samp{a[start:stop, i]}.
\end{funcdesc}

\begin{funcdesc}{sorted}{iterable\optional{, cmp\optional{,
                         key\optional{, reverse}}}}
  Return a new sorted list from the items in \var{iterable}.
  The optional arguments \var{cmp}, \var{key}, and \var{reverse}
  have the same meaning as those for the \method{list.sort()} method.
  \versionadded{2.4}    
\end{funcdesc}

\begin{funcdesc}{staticmethod}{function}
  Return a static method for \var{function}.

  A static method does not receive an implicit first argument.
  To declare a static method, use this idiom:

\begin{verbatim}
class C:
    def f(arg1, arg2, ...): ...
    f = staticmethod(f)
\end{verbatim}

  It can be called either on the class (such as \code{C.f()}) or on an
  instance (such as \code{C().f()}).  The instance is ignored except
  for its class.

  Static methods in Python are similar to those found in Java or \Cpp.
  For a more advanced concept, see \function{classmethod()} in this
  section.
  \versionadded{2.2}
\end{funcdesc}

\begin{funcdesc}{str}{\optional{object}}
  Return a string containing a nicely printable representation of an
  object.  For strings, this returns the string itself.  The
  difference with \code{repr(\var{object})} is that
  \code{str(\var{object})} does not always attempt to return a string
  that is acceptable to \function{eval()}; its goal is to return a
  printable string.  If no argument is given, returns the empty
  string, \code{''}.
\end{funcdesc}

\begin{funcdesc}{sum}{sequence\optional{, start}}
  Sums \var{start} and the items of a \var{sequence}, from left to
  right, and returns the total.  \var{start} defaults to \code{0}.
  The \var{sequence}'s items are normally numbers, and are not allowed
  to be strings.  The fast, correct way to concatenate sequence of
  strings is by calling \code{''.join(\var{sequence})}.
  Note that \code{sum(range(\var{n}), \var{m})} is equivalent to
  \code{reduce(operator.add, range(\var{n}), \var{m})}
  \versionadded{2.3}
\end{funcdesc}

\begin{funcdesc}{super}{type\optional{, object-or-type}}
  Return the superclass of \var{type}.  If the second argument is omitted
  the super object returned is unbound.  If the second argument is an
  object, \code{isinstance(\var{obj}, \var{type})} must be true.  If
  the second argument is a type, \code{issubclass(\var{type2},
  \var{type})} must be true.
  \function{super()} only works for new-style classes.

  A typical use for calling a cooperative superclass method is:
\begin{verbatim}
class C(B):
    def meth(self, arg):
        super(C, self).meth(arg)
\end{verbatim}
\versionadded{2.2}
\end{funcdesc}

\begin{funcdesc}{tuple}{\optional{sequence}}
  Return a tuple whose items are the same and in the same order as
  \var{sequence}'s items.  \var{sequence} may be a sequence, a
  container that supports iteration, or an iterator object.
  If \var{sequence} is already a tuple, it
  is returned unchanged.  For instance, \code{tuple('abc')} returns
  \code{('a', 'b', 'c')} and \code{tuple([1, 2, 3])} returns
  \code{(1, 2, 3)}.  If no argument is given, returns a new empty
  tuple, \code{()}.
\end{funcdesc}

\begin{funcdesc}{type}{object}
  Return the type of an \var{object}.  The return value is a
  type\obindex{type} object.  The standard module
  \module{types}\refstmodindex{types} defines names for all built-in
  types that don't already have built-in names.
  For instance:

\begin{verbatim}
>>> import types
>>> x = 'abc'
>>> if type(x) is str: print "It's a string"
...
It's a string
>>> def f(): pass
...
>>> if type(f) is types.FunctionType: print "It's a function"
...
It's a function
\end{verbatim}

  The \function{isinstance()} built-in function is recommended for
  testing the type of an object.
\end{funcdesc}

\begin{funcdesc}{unichr}{i}
  Return the Unicode string of one character whose Unicode code is the
  integer \var{i}.  For example, \code{unichr(97)} returns the string
  \code{u'a'}.  This is the inverse of \function{ord()} for Unicode
  strings.  The argument must be in the range [0..65535], inclusive.
  \exception{ValueError} is raised otherwise.
  \versionadded{2.0}
\end{funcdesc}

\begin{funcdesc}{unicode}{\optional{object\optional{, encoding
				    \optional{, errors}}}}
  Return the Unicode string version of \var{object} using one of the
  following modes:

  If \var{encoding} and/or \var{errors} are given, \code{unicode()}
  will decode the object which can either be an 8-bit string or a
  character buffer using the codec for \var{encoding}. The
  \var{encoding} parameter is a string giving the name of an encoding;
  if the encoding is not known, \exception{LookupError} is raised.
  Error handling is done according to \var{errors}; this specifies the
  treatment of characters which are invalid in the input encoding.  If
  \var{errors} is \code{'strict'} (the default), a
  \exception{ValueError} is raised on errors, while a value of
  \code{'ignore'} causes errors to be silently ignored, and a value of
  \code{'replace'} causes the official Unicode replacement character,
  \code{U+FFFD}, to be used to replace input characters which cannot
  be decoded.  See also the \refmodule{codecs} module.

  If no optional parameters are given, \code{unicode()} will mimic the
  behaviour of \code{str()} except that it returns Unicode strings
  instead of 8-bit strings. More precisely, if \var{object} is a
  Unicode string or subclass it will return that Unicode string without
  any additional decoding applied.

  For objects which provide a \method{__unicode__()} method, it will
  call this method without arguments to create a Unicode string. For
  all other objects, the 8-bit string version or representation is
  requested and then converted to a Unicode string using the codec for
  the default encoding in \code{'strict'} mode.

  \versionadded{2.0}
  \versionchanged[Support for \method{__unicode__()} added]{2.2}
\end{funcdesc}

\begin{funcdesc}{vars}{\optional{object}}
  Without arguments, return a dictionary corresponding to the current
  local symbol table.  With a module, class or class instance object
  as argument (or anything else that has a \member{__dict__}
  attribute), returns a dictionary corresponding to the object's
  symbol table.  The returned dictionary should not be modified: the
  effects on the corresponding symbol table are undefined.\footnote{
    In the current implementation, local variable bindings cannot
    normally be affected this way, but variables retrieved from
    other scopes (such as modules) can be.  This may change.}
\end{funcdesc}

\begin{funcdesc}{xrange}{\optional{start,} stop\optional{, step}}
  This function is very similar to \function{range()}, but returns an
  ``xrange object'' instead of a list.  This is an opaque sequence
  type which yields the same values as the corresponding list, without
  actually storing them all simultaneously.  The advantage of
  \function{xrange()} over \function{range()} is minimal (since
  \function{xrange()} still has to create the values when asked for
  them) except when a very large range is used on a memory-starved
  machine or when all of the range's elements are never used (such as
  when the loop is usually terminated with \keyword{break}).
\end{funcdesc}

\begin{funcdesc}{zip}{\optional{seq1, \moreargs}}
  This function returns a list of tuples, where the \var{i}-th tuple contains
  the \var{i}-th element from each of the argument sequences.
  The returned list is truncated in length to the length of
  the shortest argument sequence.  When there are multiple argument
  sequences which are all of the same length, \function{zip()} is
  similar to \function{map()} with an initial argument of \code{None}.
  With a single sequence argument, it returns a list of 1-tuples.
  With no arguments, it returns an empty list.
  \versionadded{2.0}

  \versionchanged[Formerly, \function{zip()} required at least one argument
  and \code{zip()} raised a \exception{TypeError} instead of returning
  an empty list.]{2.4} 
\end{funcdesc}


% ---------------------------------------------------------------------------	


\section{Non-essential Built-in Functions \label{non-essential-built-in-funcs}}

There are several built-in functions that are no longer essential to learn,
know or use in modern Python programming.  They have been kept here to
maintain backwards compatability with programs written for older versions
of Python.

Python programmers, trainers, students and bookwriters should feel free to
bypass these functions without concerns about missing something important.


\setindexsubitem{(non-essential built-in functions)}

\begin{funcdesc}{apply}{function, args\optional{, keywords}}
  The \var{function} argument must be a callable object (a
  user-defined or built-in function or method, or a class object) and
  the \var{args} argument must be a sequence.  The \var{function} is
  called with \var{args} as the argument list; the number of arguments
  is the length of the tuple.
  If the optional \var{keywords} argument is present, it must be a
  dictionary whose keys are strings.  It specifies keyword arguments
  to be added to the end of the argument list.
  Calling \function{apply()} is different from just calling
  \code{\var{function}(\var{args})}, since in that case there is always
  exactly one argument.  The use of \function{apply()} is equivalent
  to \code{\var{function}(*\var{args}, **\var{keywords})}.
  Use of \function{apply()} is not necessary since the ``extended call
  syntax,'' as used in the last example, is completely equivalent.

  \deprecated{2.3}{Use the extended call syntax instead, as described
                   above.}
\end{funcdesc}

\begin{funcdesc}{buffer}{object\optional{, offset\optional{, size}}}
  The \var{object} argument must be an object that supports the buffer
  call interface (such as strings, arrays, and buffers).  A new buffer
  object will be created which references the \var{object} argument.
  The buffer object will be a slice from the beginning of \var{object}
  (or from the specified \var{offset}). The slice will extend to the
  end of \var{object} (or will have a length given by the \var{size}
  argument).
\end{funcdesc}

\begin{funcdesc}{coerce}{x, y}
  Return a tuple consisting of the two numeric arguments converted to
  a common type, using the same rules as used by arithmetic
  operations.
\end{funcdesc}

\begin{funcdesc}{intern}{string}
  Enter \var{string} in the table of ``interned'' strings and return
  the interned string -- which is \var{string} itself or a copy.
  Interning strings is useful to gain a little performance on
  dictionary lookup -- if the keys in a dictionary are interned, and
  the lookup key is interned, the key comparisons (after hashing) can
  be done by a pointer compare instead of a string compare.  Normally,
  the names used in Python programs are automatically interned, and
  the dictionaries used to hold module, class or instance attributes
  have interned keys.  \versionchanged[Interned strings are not
  immortal (like they used to be in Python 2.2 and before);
  you must keep a reference to the return value of \function{intern()}
  around to benefit from it]{2.3}
\end{funcdesc}


\chapter{Python Services}

The modules described in this chapter provide a wide range of services
related to the Python interpreter and its interaction with its
environment.  Here's an overview:

\begin{description}

\item[sys]
--- Access system specific parameters and functions.

\item[types]
--- Names for all built-in types.

\item[traceback]
--- Print or retrieve a stack traceback.

\item[pickle]
--- Convert Python objects to streams of bytes and back.

\item[shelve]
--- Python object persistency.

\item[copy]
--- Shallow and deep copy operations.

\item[marshal]
--- Convert Python objects to streams of bytes and back (with
different constraints).

\item[imp]
--- Access the implementation of the \code{import} statement.

\item[parser]
--- Retrieve and submit parse trees from and to the runtime support
environment.

\item[__builtin__]
--- The set of built-in functions.

\item[__main__]
--- The environment where the top-level script is run.

\end{description}
		% Python Services
\section{\module{sys} ---
         System-specific parameters and functions}

\declaremodule{builtin}{sys}
\modulesynopsis{Access system-specific parameters and functions.}

This module provides access to some variables used or maintained by the
interpreter and to functions that interact strongly with the interpreter.
It is always available.


\begin{datadesc}{argv}
  The list of command line arguments passed to a Python script.
  \code{argv[0]} is the script name (it is operating system dependent
  whether this is a full pathname or not).  If the command was
  executed using the \programopt{-c} command line option to the
  interpreter, \code{argv[0]} is set to the string \code{'-c'}.  If no
  script name was passed to the Python interpreter, \code{argv} has
  zero length.
\end{datadesc}

\begin{datadesc}{byteorder}
  An indicator of the native byte order.  This will have the value
  \code{'big'} on big-endian (most-signigicant byte first) platforms,
  and \code{'little'} on little-endian (least-significant byte first)
  platforms.
  \versionadded{2.0}
\end{datadesc}

\begin{datadesc}{builtin_module_names}
  A tuple of strings giving the names of all modules that are compiled
  into this Python interpreter.  (This information is not available in
  any other way --- \code{modules.keys()} only lists the imported
  modules.)
\end{datadesc}

\begin{datadesc}{copyright}
  A string containing the copyright pertaining to the Python
  interpreter.
\end{datadesc}

\begin{datadesc}{dllhandle}
  Integer specifying the handle of the Python DLL.
  Availability: Windows.
\end{datadesc}

\begin{funcdesc}{displayhook}{\var{value}}
  If \var{value} is not \code{None}, this function prints it to
  \code{sys.stdout}, and saves it in \code{__builtin__._}.

  \code{sys.displayhook} is called on the result of evaluating an
  expression entered in an interactive Python session.  The display of
  these values can be customized by assigning another one-argument
  function to \code{sys.displayhook}.
\end{funcdesc}

\begin{funcdesc}{excepthook}{\var{type}, \var{value}, \var{traceback}}
  This function prints out a given traceback and exception to
  \code{sys.stderr}.

  When an exception is raised and uncaught, the interpreter calls
  \code{sys.excepthook} with three arguments, the exception class,
  exception instance, and a traceback object.  In an interactive
  session this happens just before control is returned to the prompt;
  in a Python program this happens just before the program exits.  The
  handling of such top-level exceptions can be customized by assigning
  another three-argument function to \code{sys.excepthook}.
\end{funcdesc}

\begin{datadesc}{__displayhook__}
\dataline{__excepthook__}
  These objects contain the original values of \code{displayhook} and
  \code{excepthook} at the start of the program.  They are saved so
  that \code{displayhook} and \code{excepthook} can be restored in
  case they happen to get replaced with broken objects.
\end{datadesc}

\begin{funcdesc}{exc_info}{}
  This function returns a tuple of three values that give information
  about the exception that is currently being handled.  The
  information returned is specific both to the current thread and to
  the current stack frame.  If the current stack frame is not handling
  an exception, the information is taken from the calling stack frame,
  or its caller, and so on until a stack frame is found that is
  handling an exception.  Here, ``handling an exception'' is defined
  as ``executing or having executed an except clause.''  For any stack
  frame, only information about the most recently handled exception is
  accessible.

  If no exception is being handled anywhere on the stack, a tuple
  containing three \code{None} values is returned.  Otherwise, the
  values returned are \code{(\var{type}, \var{value},
  \var{traceback})}.  Their meaning is: \var{type} gets the exception
  type of the exception being handled (a class object);
  \var{value} gets the exception parameter (its \dfn{associated value}
  or the second argument to \keyword{raise}, which is always a class
  instance if the exception type is a class object); \var{traceback}
  gets a traceback object (see the Reference Manual) which
  encapsulates the call stack at the point where the exception
  originally occurred.  \obindex{traceback}

  If \function{exc_clear()} is called, this function will return three
  \code{None} values until either another exception is raised in the
  current thread or the execution stack returns to a frame where
  another exception is being handled.

  \warning{Assigning the \var{traceback} return value to a
  local variable in a function that is handling an exception will
  cause a circular reference.  This will prevent anything referenced
  by a local variable in the same function or by the traceback from
  being garbage collected.  Since most functions don't need access to
  the traceback, the best solution is to use something like
  \code{exctype, value = sys.exc_info()[:2]} to extract only the
  exception type and value.  If you do need the traceback, make sure
  to delete it after use (best done with a \keyword{try}
  ... \keyword{finally} statement) or to call \function{exc_info()} in
  a function that does not itself handle an exception.} \note{Beginning
  with Python 2.2, such cycles are automatically reclaimed when garbage
  collection is enabled and they become unreachable, but it remains more
  efficient to avoid creating cycles.}
\end{funcdesc}

\begin{funcdesc}{exc_clear}{}
  This function clears all information relating to the current or last
  exception that occured in the current thread.  After calling this
  function, \function{exc_info()} will return three \code{None} values until
  another exception is raised in the current thread or the execution stack
  returns to a frame where another exception is being handled.
  
  This function is only needed in only a few obscure situations.  These
  include logging and error handling systems that report information on the
  last or current exception.  This function can also be used to try to free
  resources and trigger object finalization, though no guarantee is made as
  to what objects will be freed, if any.
\versionadded{2.3}
\end{funcdesc}

\begin{datadesc}{exc_type}
\dataline{exc_value}
\dataline{exc_traceback}
\deprecated {1.5}
            {Use \function{exc_info()} instead.}
  Since they are global variables, they are not specific to the
  current thread, so their use is not safe in a multi-threaded
  program.  When no exception is being handled, \code{exc_type} is set
  to \code{None} and the other two are undefined.
\end{datadesc}

\begin{datadesc}{exec_prefix}
  A string giving the site-specific directory prefix where the
  platform-dependent Python files are installed; by default, this is
  also \code{'/usr/local'}.  This can be set at build time with the
  \longprogramopt{exec-prefix} argument to the \program{configure}
  script.  Specifically, all configuration files (e.g. the
  \file{pyconfig.h} header file) are installed in the directory
  \code{exec_prefix + '/lib/python\var{version}/config'}, and shared
  library modules are installed in \code{exec_prefix +
  '/lib/python\var{version}/lib-dynload'}, where \var{version} is
  equal to \code{version[:3]}.
\end{datadesc}

\begin{datadesc}{executable}
  A string giving the name of the executable binary for the Python
  interpreter, on systems where this makes sense.
\end{datadesc}

\begin{funcdesc}{exit}{\optional{arg}}
  Exit from Python.  This is implemented by raising the
  \exception{SystemExit} exception, so cleanup actions specified by
  finally clauses of \keyword{try} statements are honored, and it is
  possible to intercept the exit attempt at an outer level.  The
  optional argument \var{arg} can be an integer giving the exit status
  (defaulting to zero), or another type of object.  If it is an
  integer, zero is considered ``successful termination'' and any
  nonzero value is considered ``abnormal termination'' by shells and
  the like.  Most systems require it to be in the range 0-127, and
  produce undefined results otherwise.  Some systems have a convention
  for assigning specific meanings to specific exit codes, but these
  are generally underdeveloped; \UNIX{} programs generally use 2 for
  command line syntax errors and 1 for all other kind of errors.  If
  another type of object is passed, \code{None} is equivalent to
  passing zero, and any other object is printed to \code{sys.stderr}
  and results in an exit code of 1.  In particular,
  \code{sys.exit("some error message")} is a quick way to exit a
  program when an error occurs.
\end{funcdesc}

\begin{datadesc}{exitfunc}
  This value is not actually defined by the module, but can be set by
  the user (or by a program) to specify a clean-up action at program
  exit.  When set, it should be a parameterless function.  This
  function will be called when the interpreter exits.  Only one
  function may be installed in this way; to allow multiple functions
  which will be called at termination, use the \refmodule{atexit}
  module.  \note{The exit function is not called when the program is
  killed by a signal, when a Python fatal internal error is detected,
  or when \code{os._exit()} is called.}
  \deprecated{2.4}{Use \refmodule{atexit} instead.}
\end{datadesc}

\begin{funcdesc}{getcheckinterval}{}
  Return the interpreter's ``check interval'';
  see \function{setcheckinterval()}.
  \versionadded{2.3}
\end{funcdesc}

\begin{funcdesc}{getdefaultencoding}{}
  Return the name of the current default string encoding used by the
  Unicode implementation.
  \versionadded{2.0}
\end{funcdesc}

\begin{funcdesc}{getdlopenflags}{}
  Return the current value of the flags that are used for
  \cfunction{dlopen()} calls. The flag constants are defined in the
  \refmodule{dl} and \module{DLFCN} modules.
  Availability: \UNIX.
  \versionadded{2.2}
\end{funcdesc}

\begin{funcdesc}{getfilesystemencoding}{}
  Return the name of the encoding used to convert Unicode filenames
  into system file names, or \code{None} if the system default encoding
  is used. The result value depends on the operating system:
\begin{itemize}
\item On Windows 9x, the encoding is ``mbcs''.
\item On Mac OS X, the encoding is ``utf-8''.
\item On Unix, the encoding is the user's preference 
      according to the result of nl_langinfo(CODESET), or None if
      the nl_langinfo(CODESET) failed.
\item On Windows NT+, file names are Unicode natively, so no conversion
      is performed. \code{getfilesystemencoding} still returns ``mbcs'',
      as this is the encoding that applications should use when they
      explicitly want to convert Unicode strings to byte strings that
      are equivalent when used as file names.
\end{itemize}
  \versionadded{2.3}
\end{funcdesc}

\begin{funcdesc}{getrefcount}{object}
  Return the reference count of the \var{object}.  The count returned
  is generally one higher than you might expect, because it includes
  the (temporary) reference as an argument to
  \function{getrefcount()}.
\end{funcdesc}

\begin{funcdesc}{getrecursionlimit}{}
  Return the current value of the recursion limit, the maximum depth
  of the Python interpreter stack.  This limit prevents infinite
  recursion from causing an overflow of the C stack and crashing
  Python.  It can be set by \function{setrecursionlimit()}.
\end{funcdesc}

\begin{funcdesc}{_getframe}{\optional{depth}}
  Return a frame object from the call stack.  If optional integer
  \var{depth} is given, return the frame object that many calls below
  the top of the stack.  If that is deeper than the call stack,
  \exception{ValueError} is raised.  The default for \var{depth} is
  zero, returning the frame at the top of the call stack.

  This function should be used for internal and specialized purposes
  only.
\end{funcdesc}

\begin{funcdesc}{getwindowsversion}{}
  Return a tuple containing five components, describing the Windows 
  version currently running.  The elements are \var{major}, \var{minor}, 
  \var{build}, \var{platform}, and \var{text}.  \var{text} contains
  a string while all other values are integers.

  \var{platform} may be one of the following values:
  \begin{list}{}{\leftmargin 0.7in \labelwidth 0.65in}
    \item[0 (\constant{VER_PLATFORM_WIN32s})]
      Win32s on Windows 3.1.
    \item[1 (\constant{VER_PLATFORM_WIN32_WINDOWS})] 
      Windows 95/98/ME
    \item[2 (\constant{VER_PLATFORM_WIN32_NT})] 
      Windows NT/2000/XP
    \item[3 (\constant{VER_PLATFORM_WIN32_CE})] 
      Windows CE.
  \end{list}
  
  This function wraps the Win32 \function{GetVersionEx()} function;
  see the Microsoft Documentation for more information about these
  fields.

  Availability: Windows.
  \versionadded{2.3}
\end{funcdesc}

\begin{datadesc}{hexversion}
  The version number encoded as a single integer.  This is guaranteed
  to increase with each version, including proper support for
  non-production releases.  For example, to test that the Python
  interpreter is at least version 1.5.2, use:

\begin{verbatim}
if sys.hexversion >= 0x010502F0:
    # use some advanced feature
    ...
else:
    # use an alternative implementation or warn the user
    ...
\end{verbatim}

  This is called \samp{hexversion} since it only really looks
  meaningful when viewed as the result of passing it to the built-in
  \function{hex()} function.  The \code{version_info} value may be
  used for a more human-friendly encoding of the same information.
  \versionadded{1.5.2}
\end{datadesc}

\begin{datadesc}{last_type}
\dataline{last_value}
\dataline{last_traceback}
  These three variables are not always defined; they are set when an
  exception is not handled and the interpreter prints an error message
  and a stack traceback.  Their intended use is to allow an
  interactive user to import a debugger module and engage in
  post-mortem debugging without having to re-execute the command that
  caused the error.  (Typical use is \samp{import pdb; pdb.pm()} to
  enter the post-mortem debugger; see chapter \ref{debugger}, ``The
  Python Debugger,'' for more information.)

  The meaning of the variables is the same as that of the return
  values from \function{exc_info()} above.  (Since there is only one
  interactive thread, thread-safety is not a concern for these
  variables, unlike for \code{exc_type} etc.)
\end{datadesc}

\begin{datadesc}{maxint}
  The largest positive integer supported by Python's regular integer
  type.  This is at least 2**31-1.  The largest negative integer is
  \code{-maxint-1} --- the asymmetry results from the use of 2's
  complement binary arithmetic.
\end{datadesc}

\begin{datadesc}{maxunicode}
  An integer giving the largest supported code point for a Unicode
  character.  The value of this depends on the configuration option
  that specifies whether Unicode characters are stored as UCS-2 or
  UCS-4.
\end{datadesc}

\begin{datadesc}{modules}
  This is a dictionary that maps module names to modules which have
  already been loaded.  This can be manipulated to force reloading of
  modules and other tricks.  Note that removing a module from this
  dictionary is \emph{not} the same as calling
  \function{reload()}\bifuncindex{reload} on the corresponding module
  object.
\end{datadesc}

\begin{datadesc}{path}
\indexiii{module}{search}{path}
  A list of strings that specifies the search path for modules.
  Initialized from the environment variable \envvar{PYTHONPATH}, plus an
  installation-dependent default.

  As initialized upon program startup,
  the first item of this list, \code{path[0]}, is the directory
  containing the script that was used to invoke the Python
  interpreter.  If the script directory is not available (e.g.  if the
  interpreter is invoked interactively or if the script is read from
  standard input), \code{path[0]} is the empty string, which directs
  Python to search modules in the current directory first.  Notice
  that the script directory is inserted \emph{before} the entries
  inserted as a result of \envvar{PYTHONPATH}.

  A program is free to modify this list for its own purposes.

  \versionchanged[Unicode strings are no longer ignored]{2.3}
\end{datadesc}

\begin{datadesc}{platform}
  This string contains a platform identifier, e.g. \code{'sunos5'} or
  \code{'linux1'}.  This can be used to append platform-specific
  components to \code{path}, for instance.
\end{datadesc}

\begin{datadesc}{prefix}
  A string giving the site-specific directory prefix where the
  platform independent Python files are installed; by default, this is
  the string \code{'/usr/local'}.  This can be set at build time with
  the \longprogramopt{prefix} argument to the \program{configure}
  script.  The main collection of Python library modules is installed
  in the directory \code{prefix + '/lib/python\var{version}'} while
  the platform independent header files (all except \file{pyconfig.h})
  are stored in \code{prefix + '/include/python\var{version}'}, where
  \var{version} is equal to \code{version[:3]}.
\end{datadesc}

\begin{datadesc}{ps1}
\dataline{ps2}
\index{interpreter prompts}
\index{prompts, interpreter}
  Strings specifying the primary and secondary prompt of the
  interpreter.  These are only defined if the interpreter is in
  interactive mode.  Their initial values in this case are
  \code{'>\code{>}> '} and \code{'... '}.  If a non-string object is
  assigned to either variable, its \function{str()} is re-evaluated
  each time the interpreter prepares to read a new interactive
  command; this can be used to implement a dynamic prompt.
\end{datadesc}

\begin{funcdesc}{setcheckinterval}{interval}
  Set the interpreter's ``check interval''.  This integer value
  determines how often the interpreter checks for periodic things such
  as thread switches and signal handlers.  The default is \code{100},
  meaning the check is performed every 100 Python virtual instructions.
  Setting it to a larger value may increase performance for programs
  using threads.  Setting it to a value \code{<=} 0 checks every
  virtual instruction, maximizing responsiveness as well as overhead.
\end{funcdesc}

\begin{funcdesc}{setdefaultencoding}{name}
  Set the current default string encoding used by the Unicode
  implementation.  If \var{name} does not match any available
  encoding, \exception{LookupError} is raised.  This function is only
  intended to be used by the \refmodule{site} module implementation
  and, where needed, by \module{sitecustomize}.  Once used by the
  \refmodule{site} module, it is removed from the \module{sys}
  module's namespace.
%  Note that \refmodule{site} is not imported if
%  the \programopt{-S} option is passed to the interpreter, in which
%  case this function will remain available.
  \versionadded{2.0}
\end{funcdesc}

\begin{funcdesc}{setdlopenflags}{n}
  Set the flags used by the interpreter for \cfunction{dlopen()}
  calls, such as when the interpreter loads extension modules.  Among
  other things, this will enable a lazy resolving of symbols when
  importing a module, if called as \code{sys.setdlopenflags(0)}.  To
  share symbols across extension modules, call as
  \code{sys.setdlopenflags(dl.RTLD_NOW | dl.RTLD_GLOBAL)}.  Symbolic
  names for the flag modules can be either found in the \refmodule{dl}
  module, or in the \module{DLFCN} module. If \module{DLFCN} is not
  available, it can be generated from \file{/usr/include/dlfcn.h}
  using the \program{h2py} script.
  Availability: \UNIX.
  \versionadded{2.2}
\end{funcdesc}

\begin{funcdesc}{setprofile}{profilefunc}
  Set the system's profile function,\index{profile function} which
  allows you to implement a Python source code profiler in
  Python.\index{profiler}  See chapter \ref{profile} for more
  information on the Python profiler.  The system's profile function
  is called similarly to the system's trace function (see
  \function{settrace()}), but it isn't called for each executed line
  of code (only on call and return, but the return event is reported
  even when an exception has been set).  The function is
  thread-specific, but there is no way for the profiler to know about
  context switches between threads, so it does not make sense to use
  this in the presence of multiple threads.
  Also, its return value is not used, so it can simply return
  \code{None}.
\end{funcdesc}

\begin{funcdesc}{setrecursionlimit}{limit}
  Set the maximum depth of the Python interpreter stack to
  \var{limit}.  This limit prevents infinite recursion from causing an
  overflow of the C stack and crashing Python.

  The highest possible limit is platform-dependent.  A user may need
  to set the limit higher when she has a program that requires deep
  recursion and a platform that supports a higher limit.  This should
  be done with care, because a too-high limit can lead to a crash.
\end{funcdesc}

\begin{funcdesc}{settrace}{tracefunc}
  Set the system's trace function,\index{trace function} which allows
  you to implement a Python source code debugger in Python.  See
  section \ref{debugger-hooks}, ``How It Works,'' in the chapter on
  the Python debugger.\index{debugger}  The function is
  thread-specific; for a debugger to support multiple threads, it must
  be registered using \function{settrace()} for each thread being
  debugged.  \note{The \function{settrace()} function is intended only
  for implementing debuggers, profilers, coverage tools and the like.
  Its behavior is part of the implementation platform, rather than 
  part of the language definition, and thus may not be available in
  all Python implementations.}
\end{funcdesc}

\begin{funcdesc}{settscdump}{on_flag}
  Activate dumping of VM measurements using the Pentium timestamp
  counter, if \var{on_flag} is true. Deactivate these dumps if
  \var{on_flag} is off. The function is available only if Python
  was compiled with \longprogramopt{with-tsc}. To understand the
  output of this dump, read \file{Python/ceval.c} in the Python
  sources.
  \versionadded{2.4}
\end{funcdesc}

\begin{datadesc}{stdin}
\dataline{stdout}
\dataline{stderr}
  File objects corresponding to the interpreter's standard input,
  output and error streams.  \code{stdin} is used for all interpreter
  input except for scripts but including calls to
  \function{input()}\bifuncindex{input} and
  \function{raw_input()}\bifuncindex{raw_input}.  \code{stdout} is
  used for the output of \keyword{print} and expression statements and
  for the prompts of \function{input()} and \function{raw_input()}.
  The interpreter's own prompts and (almost all of) its error messages
  go to \code{stderr}.  \code{stdout} and \code{stderr} needn't be
  built-in file objects: any object is acceptable as long as it has a
  \method{write()} method that takes a string argument.  (Changing
  these objects doesn't affect the standard I/O streams of processes
  executed by \function{os.popen()}, \function{os.system()} or the
  \function{exec*()} family of functions in the \refmodule{os}
  module.)
\end{datadesc}

\begin{datadesc}{__stdin__}
\dataline{__stdout__}
\dataline{__stderr__}
  These objects contain the original values of \code{stdin},
  \code{stderr} and \code{stdout} at the start of the program.  They
  are used during finalization, and could be useful to restore the
  actual files to known working file objects in case they have been
  overwritten with a broken object.
\end{datadesc}

\begin{datadesc}{tracebacklimit}
  When this variable is set to an integer value, it determines the
  maximum number of levels of traceback information printed when an
  unhandled exception occurs.  The default is \code{1000}.  When set
  to \code{0} or less, all traceback information is suppressed and
  only the exception type and value are printed.
\end{datadesc}

\begin{datadesc}{version}
  A string containing the version number of the Python interpreter
  plus additional information on the build number and compiler used.
  It has a value of the form \code{'\var{version}
  (\#\var{build_number}, \var{build_date}, \var{build_time})
  [\var{compiler}]'}.  The first three characters are used to identify
  the version in the installation directories (where appropriate on
  each platform).  An example:

\begin{verbatim}
>>> import sys
>>> sys.version
'1.5.2 (#0 Apr 13 1999, 10:51:12) [MSC 32 bit (Intel)]'
\end{verbatim}
\end{datadesc}

\begin{datadesc}{api_version}
  The C API version for this interpreter.  Programmers may find this useful
  when debugging version conflicts between Python and extension
  modules. \versionadded{2.3}
\end{datadesc}

\begin{datadesc}{version_info}
  A tuple containing the five components of the version number:
  \var{major}, \var{minor}, \var{micro}, \var{releaselevel}, and
  \var{serial}.  All values except \var{releaselevel} are integers;
  the release level is \code{'alpha'}, \code{'beta'},
  \code{'candidate'}, or \code{'final'}.  The \code{version_info}
  value corresponding to the Python version 2.0 is \code{(2, 0, 0,
  'final', 0)}.
  \versionadded{2.0}
\end{datadesc}

\begin{datadesc}{warnoptions}
  This is an implementation detail of the warnings framework; do not
  modify this value.  Refer to the \refmodule{warnings} module for
  more information on the warnings framework.
\end{datadesc}

\begin{datadesc}{winver}
  The version number used to form registry keys on Windows platforms.
  This is stored as string resource 1000 in the Python DLL.  The value
  is normally the first three characters of \constant{version}.  It is
  provided in the \module{sys} module for informational purposes;
  modifying this value has no effect on the registry keys used by
  Python.
  Availability: Windows.
\end{datadesc}


\begin{seealso}
  \seemodule{site}
    {This describes how to use .pth files to extend \code{sys.path}.}
\end{seealso}

\section{Built-in Types}
\label{types}

The following sections describe the standard types that are built into
the interpreter.  These are the numeric types, sequence types, and
several others, including types themselves.  There is no explicit
Boolean type; use integers instead.
\indexii{built-in}{types}
\indexii{Boolean}{type}

Some operations are supported by several object types; in particular,
all objects can be compared, tested for truth value, and converted to
a string (with the \code{`{\rm \ldots}`} notation).  The latter conversion is
implicitly used when an object is written by the \code{print} statement.
\stindex{print}


\subsection{Truth Value Testing}
\label{truth}

Any object can be tested for truth value, for use in an \code{if} or
\code{while} condition or as operand of the Boolean operations below.
The following values are considered false:
\stindex{if}
\stindex{while}
\indexii{truth}{value}
\indexii{Boolean}{operations}
\index{false}

\setindexsubitem{(Built-in object)}
\begin{itemize}

\item	\code{None}
	\ttindex{None}

\item	zero of any numeric type, e.g., \code{0}, \code{0L}, \code{0.0}.

\item	any empty sequence, e.g., \code{''}, \code{()}, \code{[]}.

\item	any empty mapping, e.g., \code{\{\}}.

\item	instances of user-defined classes, if the class defines a
	\code{__nonzero__()} or \code{__len__()} method, when that
	method returns zero.

\end{itemize}

All other values are considered true --- so objects of many types are
always true.
\index{true}

Operations and built-in functions that have a Boolean result always
return \code{0} for false and \code{1} for true, unless otherwise
stated.  (Important exception: the Boolean operations
\samp{or}\opindex{or} and \samp{and}\opindex{and} always return one of
their operands.)


\subsection{Boolean Operations}
\label{boolean}

These are the Boolean operations, ordered by ascending priority:
\indexii{Boolean}{operations}

\begin{tableiii}{|c|l|c|}{code}{Operation}{Result}{Notes}
  \lineiii{\var{x} or \var{y}}{if \var{x} is false, then \var{y}, else \var{x}}{(1)}
  \hline
  \lineiii{\var{x} and \var{y}}{if \var{x} is false, then \var{x}, else \var{y}}{(1)}
  \hline
  \lineiii{not \var{x}}{if \var{x} is false, then \code{1}, else \code{0}}{(2)}
\end{tableiii}
\opindex{and}
\opindex{or}
\opindex{not}

\noindent
Notes:

\begin{description}

\item[(1)]
These only evaluate their second argument if needed for their outcome.

\item[(2)]
\samp{not} has a lower priority than non-Boolean operators, so e.g.
\code{not a == b} is interpreted as \code{not(a == b)}, and
\code{a == not b} is a syntax error.

\end{description}


\subsection{Comparisons}
\label{comparisons}

Comparison operations are supported by all objects.  They all have the
same priority (which is higher than that of the Boolean operations).
Comparisons can be chained arbitrarily, e.g. \code{x < y <= z} is
equivalent to \code{x < y and y <= z}, except that \code{y} is
evaluated only once (but in both cases \code{z} is not evaluated at
all when \code{x < y} is found to be false).
\indexii{chaining}{comparisons}

This table summarizes the comparison operations:

\begin{tableiii}{|c|l|c|}{code}{Operation}{Meaning}{Notes}
  \lineiii{<}{strictly less than}{}
  \lineiii{<=}{less than or equal}{}
  \lineiii{>}{strictly greater than}{}
  \lineiii{>=}{greater than or equal}{}
  \lineiii{==}{equal}{}
  \lineiii{<>}{not equal}{(1)}
  \lineiii{!=}{not equal}{(1)}
  \lineiii{is}{object identity}{}
  \lineiii{is not}{negated object identity}{}
\end{tableiii}
\indexii{operator}{comparison}
\opindex{==} % XXX *All* others have funny characters < ! >
\opindex{is}
\opindex{is not}

\noindent
Notes:

\begin{description}

\item[(1)]
\code{<>} and \code{!=} are alternate spellings for the same operator.
(I couldn't choose between \ABC{} and \C{}! :-)
\indexii{ABC@\ABC{}}{language}
\indexii{\C{}}{language}

\end{description}

Objects of different types, except different numeric types, never
compare equal; such objects are ordered consistently but arbitrarily
(so that sorting a heterogeneous array yields a consistent result).
Furthermore, some types (e.g., windows) support only a degenerate
notion of comparison where any two objects of that type are unequal.
Again, such objects are ordered arbitrarily but consistently.
\indexii{types}{numeric}
\indexii{objects}{comparing}

(Implementation note: objects of different types except numbers are
ordered by their type names; objects of the same types that don't
support proper comparison are ordered by their address.)

Two more operations with the same syntactic priority, \code{in} and
\code{not in}, are supported only by sequence types (below).
\opindex{in}
\opindex{not in}


\subsection{Numeric Types}
\label{typesnumeric}

There are four numeric types: \dfn{plain integers}, \dfn{long integers}, 
\dfn{floating point numbers}, and \dfn{complex numbers}.
Plain integers (also just called \dfn{integers})
are implemented using \code{long} in \C{}, which gives them at least 32
bits of precision.  Long integers have unlimited precision.  Floating
point numbers are implemented using \code{double} in \C{}.  All bets on
their precision are off unless you happen to know the machine you are
working with.
\indexii{numeric}{types}
\indexii{integer}{types}
\indexii{integer}{type}
\indexiii{long}{integer}{type}
\indexii{floating point}{type}
\indexii{complex number}{type}
\indexii{\C{}}{language}

Complex numbers have a real and imaginary part, which are both
implemented using \code{double} in \C{}.  To extract these parts from
a complex number \code{z}, use \code{z.real} and \code{z.imag}.  

Numbers are created by numeric literals or as the result of built-in
functions and operators.  Unadorned integer literals (including hex
and octal numbers) yield plain integers.  Integer literals with an \samp{L}
or \samp{l} suffix yield long integers
(\samp{L} is preferred because \samp{1l} looks too much like eleven!).
Numeric literals containing a decimal point or an exponent sign yield
floating point numbers.  Appending \samp{j} or \samp{J} to a numeric
literal yields a complex number.
\indexii{numeric}{literals}
\indexii{integer}{literals}
\indexiii{long}{integer}{literals}
\indexii{floating point}{literals}
\indexii{complex number}{literals}
\indexii{hexadecimal}{literals}
\indexii{octal}{literals}

Python fully supports mixed arithmetic: when a binary arithmetic
operator has operands of different numeric types, the operand with the
``smaller'' type is converted to that of the other, where plain
integer is smaller than long integer is smaller than floating point is
smaller than complex.
Comparisons between numbers of mixed type use the same rule.%
\footnote{As a consequence, the list \code{[1, 2]} is considered equal
	to \code{[1.0, 2.0]}, and similar for tuples.}
The functions \code{int()}, \code{long()}, \code{float()},
and \code{complex()} can be used
to coerce numbers to a specific type.
\index{arithmetic}
\bifuncindex{int}
\bifuncindex{long}
\bifuncindex{float}
\bifuncindex{complex}

All numeric types support the following operations, sorted by
ascending priority (operations in the same box have the same
priority; all numeric operations have a higher priority than
comparison operations):

\begin{tableiii}{|c|l|c|}{code}{Operation}{Result}{Notes}
  \lineiii{\var{x} + \var{y}}{sum of \var{x} and \var{y}}{}
  \lineiii{\var{x} - \var{y}}{difference of \var{x} and \var{y}}{}
  \hline
  \lineiii{\var{x} * \var{y}}{product of \var{x} and \var{y}}{}
  \lineiii{\var{x} / \var{y}}{quotient of \var{x} and \var{y}}{(1)}
  \lineiii{\var{x} \%{} \var{y}}{remainder of \code{\var{x} / \var{y}}}{}
  \hline
  \lineiii{-\var{x}}{\var{x} negated}{}
  \lineiii{+\var{x}}{\var{x} unchanged}{}
  \hline
  \lineiii{abs(\var{x})}{absolute value or magnitude of \var{x}}{}
  \lineiii{int(\var{x})}{\var{x} converted to integer}{(2)}
  \lineiii{long(\var{x})}{\var{x} converted to long integer}{(2)}
  \lineiii{float(\var{x})}{\var{x} converted to floating point}{}
  \lineiii{complex(\var{re},\var{im})}{a complex number with real part \var{re}, imaginary part \var{im}.  \var{im} defaults to zero.}{}
  \lineiii{divmod(\var{x}, \var{y})}{the pair \code{(\var{x} / \var{y}, \var{x} \%{} \var{y})}}{(3)}
  \lineiii{pow(\var{x}, \var{y})}{\var{x} to the power \var{y}}{}
  \lineiii{\var{x}**\var{y}}{\var{x} to the power \var{y}}{}
\end{tableiii}
\indexiii{operations on}{numeric}{types}

\noindent
Notes:
\begin{description}

\item[(1)]
For (plain or long) integer division, the result is an integer.
The result is always rounded towards minus infinity: 1/2 is 0, 
(-1)/2 is -1, 1/(-2) is -1, and (-1)/(-2) is 0.
\indexii{integer}{division}
\indexiii{long}{integer}{division}

\item[(2)]
Conversion from floating point to (long or plain) integer may round or
truncate as in \C{}; see functions \code{floor()} and \code{ceil()} in
module \code{math} for well-defined conversions.
\bifuncindex{floor}
\bifuncindex{ceil}
\indexii{numeric}{conversions}
\refbimodindex{math}
\indexii{\C{}}{language}

\item[(3)]
See the section on built-in functions for an exact definition.

\end{description}
% XXXJH exceptions: overflow (when? what operations?) zerodivision

\subsubsection{Bit-string Operations on Integer Types}
\nodename{Bit-string Operations}

Plain and long integer types support additional operations that make
sense only for bit-strings.  Negative numbers are treated as their 2's
complement value (for long integers, this assumes a sufficiently large
number of bits that no overflow occurs during the operation).

The priorities of the binary bit-wise operations are all lower than
the numeric operations and higher than the comparisons; the unary
operation \samp{\~} has the same priority as the other unary numeric
operations (\samp{+} and \samp{-}).

This table lists the bit-string operations sorted in ascending
priority (operations in the same box have the same priority):

\begin{tableiii}{|c|l|c|}{code}{Operation}{Result}{Notes}
  \lineiii{\var{x} | \var{y}}{bitwise \dfn{or} of \var{x} and \var{y}}{}
  \hline
  \lineiii{\var{x} \^{} \var{y}}{bitwise \dfn{exclusive or} of \var{x} and \var{y}}{}
  \hline
  \lineiii{\var{x} \&{} \var{y}}{bitwise \dfn{and} of \var{x} and \var{y}}{}
  \hline
  \lineiii{\var{x} << \var{n}}{\var{x} shifted left by \var{n} bits}{(1), (2)}
  \lineiii{\var{x} >> \var{n}}{\var{x} shifted right by \var{n} bits}{(1), (3)}
  \hline
  \hline
  \lineiii{\~\var{x}}{the bits of \var{x} inverted}{}
\end{tableiii}
\indexiii{operations on}{integer}{types}
\indexii{bit-string}{operations}
\indexii{shifting}{operations}
\indexii{masking}{operations}

\noindent
Notes:
\begin{description}
\item[(1)] Negative shift counts are illegal and cause a
\exception{ValueError} to be raised.
\item[(2)] A left shift by \var{n} bits is equivalent to
multiplication by \code{pow(2, \var{n})} without overflow check.
\item[(3)] A right shift by \var{n} bits is equivalent to
division by \code{pow(2, \var{n})} without overflow check.
\end{description}


\subsection{Sequence Types}
\label{typesseq}

There are three sequence types: strings, lists and tuples.

Strings literals are written in single or double quotes:
\code{'xyzzy'}, \code{"frobozz"}.  See Chapter 2 of the \emph{Python
Reference Manual} for more about string literals.  Lists are
constructed with square brackets, separating items with commas:
\code{[a, b, c]}.  Tuples are constructed by the comma operator (not
within square brackets), with or without enclosing parentheses, but an
empty tuple must have the enclosing parentheses, e.g.,
\code{a, b, c} or \code{()}.  A single item tuple must have a trailing
comma, e.g., \code{(d,)}.
\indexii{sequence}{types}
\indexii{string}{type}
\indexii{tuple}{type}
\indexii{list}{type}

Sequence types support the following operations.  The \samp{in} and
\samp{not in} operations have the same priorities as the comparison
operations.  The \samp{+} and \samp{*} operations have the same
priority as the corresponding numeric operations.\footnote{They must
have since the parser can't tell the type of the operands.}

This table lists the sequence operations sorted in ascending priority
(operations in the same box have the same priority).  In the table,
\var{s} and \var{t} are sequences of the same type; \var{n}, \var{i}
and \var{j} are integers:

\begin{tableiii}{|c|l|c|}{code}{Operation}{Result}{Notes}
  \lineiii{\var{x} in \var{s}}{\code{1} if an item of \var{s} is equal to \var{x}, else \code{0}}{}
  \lineiii{\var{x} not in \var{s}}{\code{0} if an item of \var{s} is
equal to \var{x}, else \code{1}}{}
  \hline
  \lineiii{\var{s} + \var{t}}{the concatenation of \var{s} and \var{t}}{}
  \hline
  \lineiii{\var{s} * \var{n}{\rm ,} \var{n} * \var{s}}{\var{n} copies of \var{s} concatenated}{(3)}
  \hline
  \lineiii{\var{s}[\var{i}]}{\var{i}'th item of \var{s}, origin 0}{(1)}
  \lineiii{\var{s}[\var{i}:\var{j}]}{slice of \var{s} from \var{i} to \var{j}}{(1), (2)}
  \hline
  \lineiii{len(\var{s})}{length of \var{s}}{}
  \lineiii{min(\var{s})}{smallest item of \var{s}}{}
  \lineiii{max(\var{s})}{largest item of \var{s}}{}
\end{tableiii}
\indexiii{operations on}{sequence}{types}
\bifuncindex{len}
\bifuncindex{min}
\bifuncindex{max}
\indexii{concatenation}{operation}
\indexii{repetition}{operation}
\indexii{subscript}{operation}
\indexii{slice}{operation}
\opindex{in}
\opindex{not in}

\noindent
Notes:

\begin{description}
  
\item[(1)] If \var{i} or \var{j} is negative, the index is relative to
  the end of the string, i.e., \code{len(\var{s}) + \var{i}} or
  \code{len(\var{s}) + \var{j}} is substituted.  But note that \code{-0} is
  still \code{0}.
  
\item[(2)] The slice of \var{s} from \var{i} to \var{j} is defined as
  the sequence of items with index \var{k} such that \code{\var{i} <=
  \var{k} < \var{j}}.  If \var{i} or \var{j} is greater than
  \code{len(\var{s})}, use \code{len(\var{s})}.  If \var{i} is omitted,
  use \code{0}.  If \var{j} is omitted, use \code{len(\var{s})}.  If
  \var{i} is greater than or equal to \var{j}, the slice is empty.

\item[(3)] Values of \var{n} less than \code{0} are treated as
  \code{0} (which yields an empty sequence of the same type as
  \var{s}).

\end{description}

\subsubsection{More String Operations}

String objects have one unique built-in operation: the \code{\%}
operator (modulo) with a string left argument interprets this string
as a \C{} \cfunction{sprintf()} format string to be applied to the
right argument, and returns the string resulting from this formatting
operation.

The right argument should be a tuple with one item for each argument
required by the format string; if the string requires a single
argument, the right argument may also be a single non-tuple object.%
\footnote{A tuple object in this case should be a singleton.}
The following format characters are understood:
\%, c, s, i, d, u, o, x, X, e, E, f, g, G.
Width and precision may be a * to specify that an integer argument
specifies the actual width or precision.  The flag characters -, +,
blank, \# and 0 are understood.  The size specifiers h, l or L may be
present but are ignored.  The \code{\%s} conversion takes any Python
object and converts it to a string using \code{str()} before
formatting it.  The ANSI features \code{\%p} and \code{\%n}
are not supported.  Since Python strings have an explicit length,
\code{\%s} conversions don't assume that \code{'\e0'} is the end of
the string.

For safety reasons, floating point precisions are clipped to 50;
\code{\%f} conversions for numbers whose absolute value is over 1e25
are replaced by \code{\%g} conversions.%
\footnote{These numbers are fairly arbitrary.  They are intended to
avoid printing endless strings of meaningless digits without hampering
correct use and without having to know the exact precision of floating
point values on a particular machine.}
All other errors raise exceptions.

If the right argument is a dictionary (or any kind of mapping), then
the formats in the string must have a parenthesized key into that
dictionary inserted immediately after the \code{\%} character, and
each format formats the corresponding entry from the mapping.  E.g.
\begin{verbatim}
>>> count = 2
>>> language = 'Python'
>>> print '%(language)s has %(count)03d quote types.' % vars()
Python has 002 quote types.
>>> 
\end{verbatim}
In this case no * specifiers may occur in a format (since they
require a sequential parameter list).

Additional string operations are defined in standard module
\code{string} and in built-in module \code{re}.
\refstmodindex{string}
\refbimodindex{re}

\subsubsection{Mutable Sequence Types}

List objects support additional operations that allow in-place
modification of the object.
These operations would be supported by other mutable sequence types
(when added to the language) as well.
Strings and tuples are immutable sequence types and such objects cannot
be modified once created.
The following operations are defined on mutable sequence types (where
\var{x} is an arbitrary object):
\indexiii{mutable}{sequence}{types}
\indexii{list}{type}

\begin{tableiii}{|c|l|c|}{code}{Operation}{Result}{Notes}
  \lineiii{\var{s}[\var{i}] = \var{x}}
	{item \var{i} of \var{s} is replaced by \var{x}}{}
  \lineiii{\var{s}[\var{i}:\var{j}] = \var{t}}
  	{slice of \var{s} from \var{i} to \var{j} is replaced by \var{t}}{}
  \lineiii{del \var{s}[\var{i}:\var{j}]}
	{same as \code{\var{s}[\var{i}:\var{j}] = []}}{}
  \lineiii{\var{s}.append(\var{x})}
	{same as \code{\var{s}[len(\var{s}):len(\var{s})] = [\var{x}]}}{}
  \lineiii{\var{s}.count(\var{x})}
	{return number of \var{i}'s for which \code{\var{s}[\var{i}] == \var{x}}}{}
  \lineiii{\var{s}.index(\var{x})}
	{return smallest \var{i} such that \code{\var{s}[\var{i}] == \var{x}}}{(1)}
  \lineiii{\var{s}.insert(\var{i}, \var{x})}
	{same as \code{\var{s}[\var{i}:\var{i}] = [\var{x}]}
	  if \code{\var{i} >= 0}}{}
  \lineiii{\var{s}.remove(\var{x})}
	{same as \code{del \var{s}[\var{s}.index(\var{x})]}}{(1)}
  \lineiii{\var{s}.reverse()}
	{reverses the items of \var{s} in place}{(3)}
  \lineiii{\var{s}.sort()}
	{sort the items of \var{s} in place}{(2), (3)}
\end{tableiii}
\indexiv{operations on}{mutable}{sequence}{types}
\indexiii{operations on}{sequence}{types}
\indexiii{operations on}{list}{type}
\indexii{subscript}{assignment}
\indexii{slice}{assignment}
\stindex{del}
\setindexsubitem{(list method)}
\ttindex{append}
\ttindex{count}
\ttindex{index}
\ttindex{insert}
\ttindex{remove}
\ttindex{reverse}
\ttindex{sort}

\noindent
Notes:
\begin{description}
\item[(1)] Raises an exception when \var{x} is not found in \var{s}.
  
\item[(2)] The \code{sort()} method takes an optional argument
  specifying a comparison function of two arguments (list items) which
  should return \code{-1}, \code{0} or \code{1} depending on whether the
  first argument is considered smaller than, equal to, or larger than the
  second argument.  Note that this slows the sorting process down
  considerably; e.g. to sort a list in reverse order it is much faster
  to use calls to \code{sort()} and \code{reverse()} than to use
  \code{sort()} with a comparison function that reverses the ordering of
  the elements.

\item[(3)] The \code{sort()} and \code{reverse()} methods modify the
list in place for economy of space when sorting or reversing a large
list.  They don't return the sorted or reversed list to remind you of
this side effect.

\end{description}


\subsection{Mapping Types}
\label{typesmapping}

A \dfn{mapping} object maps values of one type (the key type) to
arbitrary objects.  Mappings are mutable objects.  There is currently
only one standard mapping type, the \dfn{dictionary}.  A dictionary's keys are
almost arbitrary values.  The only types of values not acceptable as
keys are values containing lists or dictionaries or other mutable
types that are compared by value rather than by object identity.
Numeric types used for keys obey the normal rules for numeric
comparison: if two numbers compare equal (e.g. \code{1} and
\code{1.0}) then they can be used interchangeably to index the same
dictionary entry.

\indexii{mapping}{types}
\indexii{dictionary}{type}

Dictionaries are created by placing a comma-separated list of
\code{\var{key}:\,\var{value}} pairs within braces, for example:
\code{\{'jack':\,4098, 'sjoerd':\,4127\}} or
\code{\{4098:\,'jack', 4127:\,'sjoerd'\}}.

The following operations are defined on mappings (where \var{a} is a
mapping, \var{k} is a key and \var{x} is an arbitrary object):

\begin{tableiii}{|c|l|c|}{code}{Operation}{Result}{Notes}
  \lineiii{len(\var{a})}{the number of items in \var{a}}{}
  \lineiii{\var{a}[\var{k}]}{the item of \var{a} with key \var{k}}{(1)}
  \lineiii{\var{a}[\var{k}] = \var{x}}{set \code{\var{a}[\var{k}]} to \var{x}}{}
  \lineiii{del \var{a}[\var{k}]}{remove \code{\var{a}[\var{k}]} from \var{a}}{(1)}
  \lineiii{\var{a}.clear()}{remove all items from \code{a}}{}
  \lineiii{\var{a}.copy()}{a (shallow) copy of \code{a}}{}
  \lineiii{\var{a}.has_key(\var{k})}{\code{1} if \var{a} has a key \var{k}, else \code{0}}{}
  \lineiii{\var{a}.items()}{a copy of \var{a}'s list of (key, item) pairs}{(2)}
  \lineiii{\var{a}.keys()}{a copy of \var{a}'s list of keys}{(2)}
  \lineiii{\var{a}.update(\var{b})}{\code{for k, v in \var{b}.items(): \var{a}[k] = v}}{(3)}
  \lineiii{\var{a}.values()}{a copy of \var{a}'s list of values}{(2)}
  \lineiii{\var{a}.get(\var{k}, \var{f})}{the item of \var{a} with key \var{k}}{(4)}
\end{tableiii}
\indexiii{operations on}{mapping}{types}
\indexiii{operations on}{dictionary}{type}
\stindex{del}
\bifuncindex{len}
\setindexsubitem{(dictionary method)}
\ttindex{keys}
\ttindex{has_key}

\noindent
Notes:
\begin{description}
\item[(1)] Raises an exception if \var{k} is not in the map.

\item[(2)] Keys and values are listed in random order.

\item[(3)] \var{b} must be of the same type as \var{a}.

\item[(4)] Never raises an exception if \var{k} is not in the map,
instead it returns \var{f}.  \var{f} is optional, when not provided
and \var{k} is not in the map, \code{None} is returned.
\end{description}


\subsection{Other Built-in Types}
\label{typesother}

The interpreter supports several other kinds of objects.
Most of these support only one or two operations.

\subsubsection{Modules}

The only special operation on a module is attribute access:
\code{\var{m}.\var{name}}, where \var{m} is a module and \var{name} accesses
a name defined in \var{m}'s symbol table.  Module attributes can be
assigned to.  (Note that the \code{import} statement is not, strictly
spoken, an operation on a module object; \code{import \var{foo}} does not
require a module object named \var{foo} to exist, rather it requires
an (external) \emph{definition} for a module named \var{foo}
somewhere.)

A special member of every module is \code{__dict__}.
This is the dictionary containing the module's symbol table.
Modifying this dictionary will actually change the module's symbol
table, but direct assignment to the \code{__dict__} attribute is not
possible (i.e., you can write \code{\var{m}.__dict__['a'] = 1}, which
defines \code{\var{m}.a} to be \code{1}, but you can't write \code{\var{m}.__dict__ = \{\}}.

Modules are written like this: \code{<module 'sys'>}.

\subsubsection{Classes and Class Instances}
\nodename{Classes and Instances}

See Chapters 3 and 7 of the \emph{Python Reference Manual} for these.

\subsubsection{Functions}

Function objects are created by function definitions.  The only
operation on a function object is to call it:
\code{\var{func}(\var{argument-list})}.

There are really two flavors of function objects: built-in functions
and user-defined functions.  Both support the same operation (to call
the function), but the implementation is different, hence the
different object types.

The implementation adds two special read-only attributes:
\code{\var{f}.func_code} is a function's \dfn{code object} (see below) and
\code{\var{f}.func_globals} is the dictionary used as the function's
global name space (this is the same as \code{\var{m}.__dict__} where
\var{m} is the module in which the function \var{f} was defined).

\subsubsection{Methods}
\obindex{method}

Methods are functions that are called using the attribute notation.
There are two flavors: built-in methods (such as \code{append()} on
lists) and class instance methods.  Built-in methods are described
with the types that support them.

The implementation adds two special read-only attributes to class
instance methods: \code{\var{m}.im_self} is the object whose method this
is, and \code{\var{m}.im_func} is the function implementing the method.
Calling \code{\var{m}(\var{arg-1}, \var{arg-2}, {\rm \ldots},
\var{arg-n})} is completely equivalent to calling
\code{\var{m}.im_func(\var{m}.im_self, \var{arg-1}, \var{arg-2}, {\rm
\ldots}, \var{arg-n})}.

See the \emph{Python Reference Manual} for more information.

\subsubsection{Code Objects}
\obindex{code}

Code objects are used by the implementation to represent
``pseudo-compiled'' executable Python code such as a function body.
They differ from function objects because they don't contain a
reference to their global execution environment.  Code objects are
returned by the built-in \code{compile()} function and can be
extracted from function objects through their \code{func_code}
attribute.
\bifuncindex{compile}
\ttindex{func_code}

A code object can be executed or evaluated by passing it (instead of a
source string) to the \code{exec} statement or the built-in
\code{eval()} function.
\stindex{exec}
\bifuncindex{eval}

See the \emph{Python Reference Manual} for more information.

\subsubsection{Type Objects}

Type objects represent the various object types.  An object's type is
accessed by the built-in function \code{type()}.  There are no special
operations on types.  The standard module \code{types} defines names
for all standard built-in types.
\bifuncindex{type}
\refstmodindex{types}

Types are written like this: \code{<type 'int'>}.

\subsubsection{The Null Object}

This object is returned by functions that don't explicitly return a
value.  It supports no special operations.  There is exactly one null
object, named \code{None} (a built-in name).

It is written as \code{None}.

\subsubsection{File Objects}

File objects are implemented using \C{}'s \code{stdio} package and can be
created with the built-in function \code{open()} described under
Built-in Functions below.  They are also returned by some other
built-in functions and methods, e.g.\ \code{posix.popen()} and
\code{posix.fdopen()} and the \code{makefile()} method of socket
objects.
\bifuncindex{open}
\refbimodindex{posix}
\refbimodindex{socket}

When a file operation fails for an I/O-related reason, the exception
\code{IOError} is raised.  This includes situations where the
operation is not defined for some reason, like \code{seek()} on a tty
device or writing a file opened for reading.

Files have the following methods:


\setindexsubitem{(file method)}

\begin{funcdesc}{close}{}
  Close the file.  A closed file cannot be read or written anymore.
\end{funcdesc}

\begin{funcdesc}{flush}{}
  Flush the internal buffer, like \code{stdio}'s \code{fflush()}.
\end{funcdesc}

\begin{funcdesc}{isatty}{}
  Return \code{1} if the file is connected to a tty(-like) device, else
  \code{0}.
\end{funcdesc}

\begin{funcdesc}{fileno}{}
Return the integer ``file descriptor'' that is used by the underlying
implementation to request I/O operations from the operating system.
This can be useful for other, lower level interfaces that use file
descriptors, e.g. module \code{fcntl} or \code{os.read()} and friends.
\refbimodindex{fcntl}
\end{funcdesc}

\begin{funcdesc}{read}{\optional{size}}
  Read at most \var{size} bytes from the file (less if the read hits
  \EOF{} or no more data is immediately available on a pipe, tty or
  similar device).  If the \var{size} argument is negative or omitted,
  read all data until \EOF{} is reached.  The bytes are returned as a string
  object.  An empty string is returned when \EOF{} is encountered
  immediately.  (For certain files, like ttys, it makes sense to
  continue reading after an \EOF{} is hit.)
\end{funcdesc}

\begin{funcdesc}{readline}{\optional{size}}
  Read one entire line from the file.  A trailing newline character is
  kept in the string%
\footnote{The advantage of leaving the newline on is that an empty string 
	can be returned to mean \EOF{} without being ambiguous.  Another 
	advantage is that (in cases where it might matter, e.g. if you 
	want to make an exact copy of a file while scanning its lines) 
	you can tell whether the last line of a file ended in a newline
	or not (yes this happens!).}
  (but may be absent when a file ends with an
  incomplete line).  If the \var{size} argument is present and
  non-negative, it is a maximum byte count (including the trailing
  newline) and an incomplete line may be returned.
  An empty string is returned when \EOF{} is hit
  immediately.  Note: unlike \code{stdio}'s \code{fgets()}, the returned
  string contains null characters (\code{'\e 0'}) if they occurred in the
  input.
\end{funcdesc}

\begin{funcdesc}{readlines}{\optional{sizehint}}
  Read until \EOF{} using \code{readline()} and return a list containing
  the lines thus read.  If the optional \var{sizehint} argument is
  present, instead of reading up to \EOF{}, whole lines totalling
  approximately \var{sizehint} bytes (possibly after rounding up to an
  internal buffer size) are read.
\end{funcdesc}

\begin{funcdesc}{seek}{offset\, whence}
  Set the file's current position, like \code{stdio}'s \code{fseek()}.
  The \var{whence} argument is optional and defaults to \code{0}
  (absolute file positioning); other values are \code{1} (seek
  relative to the current position) and \code{2} (seek relative to the
  file's end).  There is no return value.
\end{funcdesc}

\begin{funcdesc}{tell}{}
  Return the file's current position, like \code{stdio}'s \code{ftell()}.
\end{funcdesc}

\begin{funcdesc}{truncate}{\optional{size}}
Truncate the file's size.  If the optional size argument present, the
file is truncated to (at most) that size.  The size defaults to the
current position.  Availability of this function depends on the
operating system version (e.g., not all \UNIX{} versions support this
operation).
\end{funcdesc}

\begin{funcdesc}{write}{str}
Write a string to the file.  There is no return value.  Note: due to
buffering, the string may not actually show up in the file until
the \code{flush()} or \code{close()} method is called.
\end{funcdesc}

\begin{funcdesc}{writelines}{list}
Write a list of strings to the file.  There is no return value.
(The name is intended to match \code{readlines}; \code{writelines}
does not add line separators.)
\end{funcdesc}

File objects also offer the following attributes:

\setindexsubitem{(file attribute)}

\begin{datadesc}{closed}
Boolean indicating the current state of the file object.  This is a
read-only attribute; the \method{close()} method changes the value.
\end{datadesc}

\begin{datadesc}{mode}
The I/O mode for the file.  If the file was created using the
\function{open()} built-in function, this will be the value of the
\var{mode} parameter.  This is a read-only attribute.
\end{datadesc}

\begin{datadesc}{name}
If the file object was created using \function{open()}, the name of
the file.  Otherwise, some string that indicates the source of the
file object, of the form \samp{<\mbox{\ldots}>}.  This is a read-only
attribute.
\end{datadesc}

\begin{datadesc}{softspace}
Boolean that indicates whether a space character needs to be printed
before another value when using the \keyword{print} statement.
Classes that are trying to simulate a file object should also have a
writable \code{softspace} attribute, which should be initialized to
zero.  This will be automatic for classes implemented in Python; types
implemented in \C{} will have to provide a writable \code{softspace}
attribute.
\end{datadesc}

\subsubsection{Internal Objects}

See the \emph{Python Reference Manual} for this information.  It
describes code objects, stack frame objects, traceback objects, and
slice objects.


\subsection{Special Attributes}
\label{specialattrs}

The implementation adds a few special read-only attributes to several
object types, where they are relevant:

\begin{itemize}

\item
\code{\var{x}.__dict__} is a dictionary of some sort used to store an
object's (writable) attributes;

\item
\code{\var{x}.__methods__} lists the methods of many built-in object types,
e.g., \code{[].__methods__} yields
\code{['append', 'count', 'index', 'insert', 'remove', 'reverse', 'sort']};

\item
\code{\var{x}.__members__} lists data attributes;

\item
\code{\var{x}.__class__} is the class to which a class instance belongs;

\item
\code{\var{x}.__bases__} is the tuple of base classes of a class object.

\end{itemize}

\section{\module{UserDict} ---
         Class wrapper for dictionary objects}

\declaremodule{standard}{UserDict}
\modulesynopsis{Class wrapper for dictionary objects.}

This module defines a class that acts as a wrapper around
dictionary objects.  It is a useful base class for
your own dictionary-like classes, which can inherit from
them and override existing methods or add new ones.  In this way one
can add new behaviors to dictionaries.

The \module{UserDict} module defines the \class{UserDict} class:

\begin{classdesc}{UserDict}{\optional{initialdata}}
Return a class instance that simulates a dictionary.  The instance's
contents are kept in a regular dictionary, which is accessible via the
\member{data} attribute of \class{UserDict} instances.  If
\var{initialdata} is provided, \member{data} is initialized with its
contents; note that a reference to \var{initialdata} will not be kept, 
allowing it be used used for other purposes.
\end{classdesc}

In addition to supporting the methods and operations of mappings (see
section \ref{typesmapping}), \class{UserDict} instances provide the
following attribute:

\begin{memberdesc}{data}
A real dictionary used to store the contents of the \class{UserDict}
class.
\end{memberdesc}


\section{\module{UserList} ---
         Class wrapper for list objects}

\declaremodule{standard}{UserList}
\modulesynopsis{Class wrapper for list objects.}


This module defines a class that acts as a wrapper around
list objects.  It is a useful base class for
your own list-like classes, which can inherit from
them and override existing methods or add new ones.  In this way one
can add new behaviors to lists.

The \module{UserList} module defines the \class{UserList} class:

\begin{classdesc}{UserList}{\optional{list}}
Return a class instance that simulates a list.  The instance's
contents are kept in a regular list, which is accessible via the
\member{data} attribute of \class{UserList} instances.  The instance's
contents are initially set to a copy of \var{list}, defaulting to the
empty list \code{[]}.  \var{list} can be either a regular Python list,
or an instance of \class{UserList} (or a subclass).
\end{classdesc}

In addition to supporting the methods and operations of mutable
sequences (see section \ref{typesseq}), \class{UserList} instances
provide the following attribute:

\begin{memberdesc}{data}
A real Python list object used to store the contents of the
\class{UserList} class.
\end{memberdesc}

\strong{Subclassing requirements:}
Subclasses of \class{UserList} are expect to offer a constructor which
can be called with either no arguments or one argument.  List
operations which return a new sequence attempt to create an instance
of the actual implementation class.  To do so, it assumes that the
constructor can be called with a single parameter, which is a sequence
object used as a data source.

If a derived class does not wish to comply with this requirement, all
of the special methods supported by this class will need to be
overridden; please consult the sources for information about the
methods which need to be provided in that case.

\versionchanged[Python versions 1.5.2 and 1.6 also required that the
                constructor be callable with no parameters, and offer
                a mutable \member{data} attribute.  Earlier versions
                of Python did not attempt to create instances of the
                derived class]{2.0}


\section{\module{UserString} ---
         Class wrapper for string objects}

\declaremodule{standard}{UserString}
\modulesynopsis{Class wrapper for string objects.}
\moduleauthor{Peter Funk}{pf@artcom-gmbh.de}
\sectionauthor{Peter Funk}{pf@artcom-gmbh.de}

This module defines a class that acts as a wrapper around string
objects.  It is a useful base class for your own string-like classes,
which can inherit from them and override existing methods or add new
ones.  In this way one can add new behaviors to strings.

It should be noted that these classes are highly inefficient compared
to real string or Unicode objects; this is especially the case for
\class{MutableString}.

The \module{UserString} module defines the following classes:

\begin{classdesc}{UserString}{\optional{sequence}}
Return a class instance that simulates a string or a Unicode string
object.  The instance's content is kept in a regular string or Unicode
string object, which is accessible via the \member{data} attribute of
\class{UserString} instances.  The instance's contents are initially
set to a copy of \var{sequence}.  \var{sequence} can be either a
regular Python string or Unicode string, an instance of
\class{UserString} (or a subclass) or an arbitrary sequence which can
be converted into a string using the built-in \function{str()} function.
\end{classdesc}

\begin{classdesc}{MutableString}{\optional{sequence}}
This class is derived from the \class{UserString} above and redefines
strings to be \emph{mutable}.  Mutable strings can't be used as
dictionary keys, because dictionaries require \emph{immutable} objects as
keys.  The main intention of this class is to serve as an educational
example for inheritance and necessity to remove (override) the
\method{__hash__()} method in order to trap attempts to use a
mutable object as dictionary key, which would be otherwise very
error prone and hard to track down.
\end{classdesc}

In addition to supporting the methods and operations of string and
Unicode objects (see section \ref{string-methods}, ``String
Methods''), \class{UserString} instances provide the following
attribute:

\begin{memberdesc}{data}
A real Python string or Unicode object used to store the content of the
\class{UserString} class.
\end{memberdesc}

\section{\module{operator} ---
         Standard operators as functions.}
\declaremodule{builtin}{operator}
\sectionauthor{Skip Montanaro}{skip@automatrix.com}

\modulesynopsis{All Python's standard operators as built-in functions.}


The \module{operator} module exports a set of functions implemented in C
corresponding to the intrinsic operators of Python.  For example,
\code{operator.add(x, y)} is equivalent to the expression \code{x+y}.  The
function names are those used for special class methods; variants without
leading and trailing \samp{__} are also provided for convenience.

The functions fall into categories that perform object comparisons,
logical operations, mathematical operations, sequence operations, and
abstract type tests.

The object comparison functions are useful for all objects, and are
named after the rich comparison operators they support:

\begin{funcdesc}{lt}{a, b}
\funcline{le}{a, b}
\funcline{eq}{a, b}
\funcline{ne}{a, b}
\funcline{ge}{a, b}
\funcline{gt}{a, b}
\funcline{__lt__}{a, b}
\funcline{__le__}{a, b}
\funcline{__eq__}{a, b}
\funcline{__ne__}{a, b}
\funcline{__ge__}{a, b}
\funcline{__gt__}{a, b}
Perform ``rich comparisons'' between \var{a} and \var{b}. Specifically,
\code{lt(\var{a}, \var{b})} is equivalent to \code{\var{a} < \var{b}},
\code{le(\var{a}, \var{b})} is equivalent to \code{\var{a} <= \var{b}},
\code{eq(\var{a}, \var{b})} is equivalent to \code{\var{a} == \var{b}},
\code{ne(\var{a}, \var{b})} is equivalent to \code{\var{a} != \var{b}},
\code{gt(\var{a}, \var{b})} is equivalent to \code{\var{a} > \var{b}}
and
\code{ge(\var{a}, \var{b})} is equivalent to \code{\var{a} >= \var{b}}.
Note that unlike the built-in \function{cmp()}, these functions can
return any value, which may or may not be interpretable as a Boolean
value.  See the \citetitle[../ref/ref.html]{Python Reference Manual}
for more information about rich comparisons.
\versionadded{2.2}
\end{funcdesc}


The logical operations are also generally applicable to all objects,
and support truth tests, identity tests, and boolean operations:

\begin{funcdesc}{not_}{o}
\funcline{__not__}{o}
Return the outcome of \keyword{not} \var{o}.  (Note that there is no
\method{__not__()} method for object instances; only the interpreter
core defines this operation.  The result is affected by the
\method{__nonzero__()} and \method{__len__()} methods.)
\end{funcdesc}

\begin{funcdesc}{truth}{o}
Return \constant{True} if \var{o} is true, and \constant{False}
otherwise.  This is equivalent to using the \class{bool}
constructor.
\end{funcdesc}

\begin{funcdesc}{is_}{a, b}
Return \code{\var{a} is \var{b}}.  Tests object identity.
\versionadded{2.3}
\end{funcdesc}

\begin{funcdesc}{is_not}{a, b}
Return \code{\var{a} is not \var{b}}.  Tests object identity.
\versionadded{2.3}
\end{funcdesc}


The mathematical and bitwise operations are the most numerous:

\begin{funcdesc}{abs}{o}
\funcline{__abs__}{o}
Return the absolute value of \var{o}.
\end{funcdesc}

\begin{funcdesc}{add}{a, b}
\funcline{__add__}{a, b}
Return \var{a} \code{+} \var{b}, for \var{a} and \var{b} numbers.
\end{funcdesc}

\begin{funcdesc}{and_}{a, b}
\funcline{__and__}{a, b}
Return the bitwise and of \var{a} and \var{b}.
\end{funcdesc}

\begin{funcdesc}{div}{a, b}
\funcline{__div__}{a, b}
Return \var{a} \code{/} \var{b} when \code{__future__.division} is not
in effect.  This is also known as ``classic'' division.
\end{funcdesc}

\begin{funcdesc}{floordiv}{a, b}
\funcline{__floordiv__}{a, b}
Return \var{a} \code{//} \var{b}.
\versionadded{2.2}
\end{funcdesc}

\begin{funcdesc}{inv}{o}
\funcline{invert}{o}
\funcline{__inv__}{o}
\funcline{__invert__}{o}
Return the bitwise inverse of the number \var{o}.  This is equivalent
to \code{\textasciitilde}\var{o}.  The names \function{invert()} and
\function{__invert__()} were added in Python 2.0.
\end{funcdesc}

\begin{funcdesc}{lshift}{a, b}
\funcline{__lshift__}{a, b}
Return \var{a} shifted left by \var{b}.
\end{funcdesc}

\begin{funcdesc}{mod}{a, b}
\funcline{__mod__}{a, b}
Return \var{a} \code{\%} \var{b}.
\end{funcdesc}

\begin{funcdesc}{mul}{a, b}
\funcline{__mul__}{a, b}
Return \var{a} \code{*} \var{b}, for \var{a} and \var{b} numbers.
\end{funcdesc}

\begin{funcdesc}{neg}{o}
\funcline{__neg__}{o}
Return \var{o} negated.
\end{funcdesc}

\begin{funcdesc}{or_}{a, b}
\funcline{__or__}{a, b}
Return the bitwise or of \var{a} and \var{b}.
\end{funcdesc}

\begin{funcdesc}{pos}{o}
\funcline{__pos__}{o}
Return \var{o} positive.
\end{funcdesc}

\begin{funcdesc}{pow}{a, b}
\funcline{__pow__}{a, b}
Return \var{a} \code{**} \var{b}, for \var{a} and \var{b} numbers.
\versionadded{2.3}
\end{funcdesc}

\begin{funcdesc}{rshift}{a, b}
\funcline{__rshift__}{a, b}
Return \var{a} shifted right by \var{b}.
\end{funcdesc}

\begin{funcdesc}{sub}{a, b}
\funcline{__sub__}{a, b}
Return \var{a} \code{-} \var{b}.
\end{funcdesc}

\begin{funcdesc}{truediv}{a, b}
\funcline{__truediv__}{a, b}
Return \var{a} \code{/} \var{b} when \code{__future__.division} is in
effect.  This is also known as division.
\versionadded{2.2}
\end{funcdesc}

\begin{funcdesc}{xor}{a, b}
\funcline{__xor__}{a, b}
Return the bitwise exclusive or of \var{a} and \var{b}.
\end{funcdesc}


Operations which work with sequences include:

\begin{funcdesc}{concat}{a, b}
\funcline{__concat__}{a, b}
Return \var{a} \code{+} \var{b} for \var{a} and \var{b} sequences.
\end{funcdesc}

\begin{funcdesc}{contains}{a, b}
\funcline{__contains__}{a, b}
Return the outcome of the test \var{b} \code{in} \var{a}.
Note the reversed operands.  The name \function{__contains__()} was
added in Python 2.0.
\end{funcdesc}

\begin{funcdesc}{countOf}{a, b}
Return the number of occurrences of \var{b} in \var{a}.
\end{funcdesc}

\begin{funcdesc}{delitem}{a, b}
\funcline{__delitem__}{a, b}
Remove the value of \var{a} at index \var{b}.
\end{funcdesc}

\begin{funcdesc}{delslice}{a, b, c}
\funcline{__delslice__}{a, b, c}
Delete the slice of \var{a} from index \var{b} to index \var{c}\code{-1}.
\end{funcdesc}

\begin{funcdesc}{getitem}{a, b}
\funcline{__getitem__}{a, b}
Return the value of \var{a} at index \var{b}.
\end{funcdesc}

\begin{funcdesc}{getslice}{a, b, c}
\funcline{__getslice__}{a, b, c}
Return the slice of \var{a} from index \var{b} to index \var{c}\code{-1}.
\end{funcdesc}

\begin{funcdesc}{indexOf}{a, b}
Return the index of the first of occurrence of \var{b} in \var{a}.
\end{funcdesc}

\begin{funcdesc}{repeat}{a, b}
\funcline{__repeat__}{a, b}
Return \var{a} \code{*} \var{b} where \var{a} is a sequence and
\var{b} is an integer.
\end{funcdesc}

\begin{funcdesc}{sequenceIncludes}{\unspecified}
\deprecated{2.0}{Use \function{contains()} instead.}
Alias for \function{contains()}.
\end{funcdesc}

\begin{funcdesc}{setitem}{a, b, c}
\funcline{__setitem__}{a, b, c}
Set the value of \var{a} at index \var{b} to \var{c}.
\end{funcdesc}

\begin{funcdesc}{setslice}{a, b, c, v}
\funcline{__setslice__}{a, b, c, v}
Set the slice of \var{a} from index \var{b} to index \var{c}\code{-1} to the
sequence \var{v}.
\end{funcdesc}


The \module{operator} module also defines a few predicates to test the
type of objects.  \note{Be careful not to misinterpret the
results of these functions; only \function{isCallable()} has any
measure of reliability with instance objects.  For example:}

\begin{verbatim}
>>> class C:
...     pass
... 
>>> import operator
>>> o = C()
>>> operator.isMappingType(o)
True
\end{verbatim}

\begin{funcdesc}{isCallable}{o}
\deprecated{2.0}{Use the \function{callable()} built-in function instead.}
Returns true if the object \var{o} can be called like a function,
otherwise it returns false.  True is returned for functions, bound and
unbound methods, class objects, and instance objects which support the
\method{__call__()} method.
\end{funcdesc}

\begin{funcdesc}{isMappingType}{o}
Returns true if the object \var{o} supports the mapping interface.
This is true for dictionaries and all instance objects defining
\method{__getitem__}.
\warning{There is no reliable way to test if an instance
supports the complete mapping protocol since the interface itself is
ill-defined.  This makes this test less useful than it otherwise might
be.}
\end{funcdesc}

\begin{funcdesc}{isNumberType}{o}
Returns true if the object \var{o} represents a number.  This is true
for all numeric types implemented in C.
\warning{There is no reliable way to test if an instance
supports the complete numeric interface since the interface itself is
ill-defined.  This makes this test less useful than it otherwise might
be.}
\end{funcdesc}

\begin{funcdesc}{isSequenceType}{o}
Returns true if the object \var{o} supports the sequence protocol.
This returns true for all objects which define sequence methods in C,
and for all instance objects defining \method{__getitem__}.
\warning{There is no reliable
way to test if an instance supports the complete sequence interface
since the interface itself is ill-defined.  This makes this test less
useful than it otherwise might be.}
\end{funcdesc}


Example: Build a dictionary that maps the ordinals from \code{0} to
\code{255} to their character equivalents.

\begin{verbatim}
>>> import operator
>>> d = {}
>>> keys = range(256)
>>> vals = map(chr, keys)
>>> map(operator.setitem, [d]*len(keys), keys, vals)
\end{verbatim}


The \module{operator} module also defines tools for generalized attribute
and item lookups.  These are useful for making fast field extractors
as arguments for \function{map()}, \function{sorted()},
\method{itertools.groupby()}, or other functions that expect a
function argument.

\begin{funcdesc}{attrgetter}{attr\optional{, args...}}
Return a callable object that fetches \var{attr} from its operand.
If more than one attribute is requested, returns a tuple of attributes.
After, \samp{f=attrgetter('name')}, the call \samp{f(b)} returns
\samp{b.name}.  After, \samp{f=attrgetter('name', 'date')}, the call
\samp{f(b)} returns \samp{(b.name, b.date)}. 
\versionadded{2.4}
\versionchanged[Added support for multiple attributes]{2.5}
\end{funcdesc}
    
\begin{funcdesc}{itemgetter}{item\optional{, args...}}
Return a callable object that fetches \var{item} from its operand.
If more than one item is requested, returns a tuple of items.
After, \samp{f=itemgetter(2)}, the call \samp{f(b)} returns
\samp{b[2]}.
After, \samp{f=itemgetter(2,5,3)}, the call \samp{f(b)} returns
\samp{(b[2], b[5], b[3])}.		
\versionadded{2.4}
\versionchanged[Added support for multiple item extraction]{2.5}		
\end{funcdesc}

Examples:
                
\begin{verbatim}
>>> from operator import itemgetter
>>> inventory = [('apple', 3), ('banana', 2), ('pear', 5), ('orange', 1)]
>>> getcount = itemgetter(1)
>>> map(getcount, inventory)
[3, 2, 5, 1]
>>> sorted(inventory, key=getcount)
[('orange', 1), ('banana', 2), ('apple', 3), ('pear', 5)]
\end{verbatim}
                

\subsection{Mapping Operators to Functions \label{operator-map}}

This table shows how abstract operations correspond to operator
symbols in the Python syntax and the functions in the
\refmodule{operator} module.


\begin{tableiii}{l|c|l}{textrm}{Operation}{Syntax}{Function}
  \lineiii{Addition}{\code{\var{a} + \var{b}}}
          {\code{add(\var{a}, \var{b})}}
  \lineiii{Concatenation}{\code{\var{seq1} + \var{seq2}}}
          {\code{concat(\var{seq1}, \var{seq2})}}
  \lineiii{Containment Test}{\code{\var{o} in \var{seq}}}
          {\code{contains(\var{seq}, \var{o})}}
  \lineiii{Division}{\code{\var{a} / \var{b}}}
          {\code{div(\var{a}, \var{b}) \#} without \code{__future__.division}}
  \lineiii{Division}{\code{\var{a} / \var{b}}}
          {\code{truediv(\var{a}, \var{b}) \#} with \code{__future__.division}}
  \lineiii{Division}{\code{\var{a} // \var{b}}}
          {\code{floordiv(\var{a}, \var{b})}}
  \lineiii{Bitwise And}{\code{\var{a} \&\ \var{b}}}
          {\code{and_(\var{a}, \var{b})}}
  \lineiii{Bitwise Exclusive Or}{\code{\var{a} \^\ \var{b}}}
          {\code{xor(\var{a}, \var{b})}}
  \lineiii{Bitwise Inversion}{\code{\~{} \var{a}}}
          {\code{invert(\var{a})}}
  \lineiii{Bitwise Or}{\code{\var{a} | \var{b}}}
          {\code{or_(\var{a}, \var{b})}}
  \lineiii{Exponentiation}{\code{\var{a} ** \var{b}}}
          {\code{pow(\var{a}, \var{b})}}
  \lineiii{Identity}{\code{\var{a} is \var{b}}}
          {\code{is_(\var{a}, \var{b})}}
  \lineiii{Identity}{\code{\var{a} is not \var{b}}}
          {\code{is_not(\var{a}, \var{b})}}
  \lineiii{Indexed Assignment}{\code{\var{o}[\var{k}] = \var{v}}}
          {\code{setitem(\var{o}, \var{k}, \var{v})}}
  \lineiii{Indexed Deletion}{\code{del \var{o}[\var{k}]}}
          {\code{delitem(\var{o}, \var{k})}}
  \lineiii{Indexing}{\code{\var{o}[\var{k}]}}
          {\code{getitem(\var{o}, \var{k})}}
  \lineiii{Left Shift}{\code{\var{a} <\code{<} \var{b}}}
          {\code{lshift(\var{a}, \var{b})}}
  \lineiii{Modulo}{\code{\var{a} \%\ \var{b}}}
          {\code{mod(\var{a}, \var{b})}}
  \lineiii{Multiplication}{\code{\var{a} * \var{b}}}
          {\code{mul(\var{a}, \var{b})}}
  \lineiii{Negation (Arithmetic)}{\code{- \var{a}}}
          {\code{neg(\var{a})}}
  \lineiii{Negation (Logical)}{\code{not \var{a}}}
          {\code{not_(\var{a})}}
  \lineiii{Right Shift}{\code{\var{a} >\code{>} \var{b}}}
          {\code{rshift(\var{a}, \var{b})}}
  \lineiii{Sequence Repitition}{\code{\var{seq} * \var{i}}}
          {\code{repeat(\var{seq}, \var{i})}}
  \lineiii{Slice Assignment}{\code{\var{seq}[\var{i}:\var{j}]} = \var{values}}
          {\code{setslice(\var{seq}, \var{i}, \var{j}, \var{values})}}
  \lineiii{Slice Deletion}{\code{del \var{seq}[\var{i}:\var{j}]}}
          {\code{delslice(\var{seq}, \var{i}, \var{j})}}
  \lineiii{Slicing}{\code{\var{seq}[\var{i}:\var{j}]}}
          {\code{getslice(\var{seq}, \var{i}, \var{j})}}
  \lineiii{String Formatting}{\code{\var{s} \%\ \var{o}}}
          {\code{mod(\var{s}, \var{o})}}
  \lineiii{Subtraction}{\code{\var{a} - \var{b}}}
          {\code{sub(\var{a}, \var{b})}}
  \lineiii{Truth Test}{\code{\var{o}}}
          {\code{truth(\var{o})}}
  \lineiii{Ordering}{\code{\var{a} < \var{b}}}
          {\code{lt(\var{a}, \var{b})}}
  \lineiii{Ordering}{\code{\var{a} <= \var{b}}}
          {\code{le(\var{a}, \var{b})}}
  \lineiii{Equality}{\code{\var{a} == \var{b}}}
          {\code{eq(\var{a}, \var{b})}}
  \lineiii{Difference}{\code{\var{a} != \var{b}}}
          {\code{ne(\var{a}, \var{b})}}
  \lineiii{Ordering}{\code{\var{a} >= \var{b}}}
          {\code{ge(\var{a}, \var{b})}}
  \lineiii{Ordering}{\code{\var{a} > \var{b}}}
          {\code{gt(\var{a}, \var{b})}}
\end{tableiii}

\section{Standard Module \module{traceback}}
\declaremodule{standard}{traceback}

\modulesynopsis{Print or retrieve a stack traceback.}



This module provides a standard interface to extract, format and print
stack traces of Python programs.  It exactly mimics the behavior of
the Python interpreter when it prints a stack trace.  This is useful
when you want to print stack traces under program control, e.g. in a
``wrapper'' around the interpreter.

The module uses traceback objects --- this is the object type
that is stored in the variables \code{sys.exc_traceback} and
\code{sys.last_traceback} and returned as the third item from
\function{sys.exc_info()}.
\obindex{traceback}

The module defines the following functions:

\begin{funcdesc}{print_tb}{traceback\optional{, limit\optional{, file}}}
Print up to \var{limit} stack trace entries from \var{traceback}.  If
\var{limit} is omitted or \code{None}, all entries are printed.
If \var{file} is omitted or \code{None}, the output goes to
\code{sys.stderr}; otherwise it should be an open file or file-like
object to receive the output.
\end{funcdesc}

\begin{funcdesc}{extract_tb}{traceback\optional{, limit}}
Return a list of up to \var{limit} ``pre-processed'' stack trace
entries extracted from \var{traceback}.  It is useful for alternate
formatting of stack traces.  If \var{limit} is omitted or \code{None},
all entries are extracted.  A ``pre-processed'' stack trace entry is a
quadruple (\var{filename}, \var{line number}, \var{function name},
\var{line text}) representing the information that is usually printed
for a stack trace.  The \var{line text} is a string with leading and
trailing whitespace stripped; if the source is not available it is
\code{None}.
\end{funcdesc}

\begin{funcdesc}{print_exception}{type, value,
traceback\optional{, limit\optional{, file}}}
Print exception information and up to \var{limit} stack trace entries
from \var{traceback} to \var{file}.
This differs from \function{print_tb()} in the
following ways: (1) if \var{traceback} is not \code{None}, it prints a
header \samp{Traceback (innermost last):}; (2) it prints the
exception \var{type} and \var{value} after the stack trace; (3) if
\var{type} is \exception{SyntaxError} and \var{value} has the appropriate
format, it prints the line where the syntax error occurred with a
caret indicating the approximate position of the error.
\end{funcdesc}

\begin{funcdesc}{print_exc}{\optional{limit\optional{, file}}}
This is a shorthand for `\code{print_exception(sys.exc_type,}
\code{sys.exc_value,} \code{sys.exc_traceback,} \var{limit}\code{,}
\var{file}\code{)}'.  (In fact, it uses \code{sys.exc_info()} to
retrieve the same information in a thread-safe way.)
\end{funcdesc}

\begin{funcdesc}{print_last}{\optional{limit\optional{, file}}}
This is a shorthand for `\code{print_exception(sys.last_type,}
\code{sys.last_value,} \code{sys.last_traceback,} \var{limit}\code{,}
\var{file}\code{)}'.
\end{funcdesc}

\begin{funcdesc}{print_stack}{\optional{f\optional{, limit\optional{, file}}}}
This function prints a stack trace from its invocation point.  The
optional \var{f} argument can be used to specify an alternate stack
frame to start.  The optional \var{limit} and \var{file} arguments have the
same meaning as for \function{print_exception()}.
\end{funcdesc}

\begin{funcdesc}{extract_tb}{tb\optional{, limit}}
Return a list containing the raw (unformatted) traceback information
extracted from the traceback object \var{tb}.  The optional
\var{limit} argument has the same meaning as for
\function{print_exception()}.  The items in the returned list are
4-tuples containing the following values: filename, line number,
function name, and source text line.  The source text line is stripped 
of leading and trailing whitespace; it is \code{None} when the source
text file is unavailable.
\end{funcdesc}

\begin{funcdesc}{extract_stack}{\optional{f\optional{, limit}}}
Extract the raw traceback from the current stack frame.  The return
value has the same format as for \function{extract_tb()}.  The
optional \var{f} and \var{limit} arguments have the same meaning as
for \function{print_stack()}.
\end{funcdesc}

\begin{funcdesc}{format_list}{list}
Given a list of tuples as returned by \function{extract_tb()} or
\function{extract_stack()}, return a list of strings ready for
printing.  Each string in the resulting list corresponds to the item
with the same index in the argument list.  Each string ends in a
newline; the strings may contain internal newlines as well, for those
items whose source text line is not \code{None}.
\end{funcdesc}

\begin{funcdesc}{format_exception_only}{type, value}
Format the exception part of a traceback.  The arguments are the
exception type and value such as given by \code{sys.last_type} and
\code{sys.last_value}.  The return value is a list of strings, each
ending in a newline.  Normally, the list contains a single string;
however, for \code{SyntaxError} exceptions, it contains several lines
that (when printed) display detailed information about where the
syntax error occurred.  The message indicating which exception
occurred is the always last string in the list.
\end{funcdesc}

\begin{funcdesc}{format_exception}{type, value, tb\optional{, limit}}
Format a stack trace and the exception information.  The arguments 
have the same meaning as the corresponding arguments to
\function{print_exception()}.  The return value is a list of strings,
each ending in a newline and some containing internal newlines.  When
these lines are contatenated and printed, exactly the same text is
printed as does \function{print_exception()}.
\end{funcdesc}

\begin{funcdesc}{format_tb}{tb\optional{, limit}}
A shorthand for \code{format_list(extract_tb(\var{tb}, \var{limit}))}.
\end{funcdesc}

\begin{funcdesc}{format_stack}{\optional{f\optional{, limit}}}
A shorthand for \code{format_list(extract_stack(\var{f}, \var{limit}))}.
\end{funcdesc}

\begin{funcdesc}{tb_lineno}{tb}
This function returns the current line number set in the traceback
object.  This is normally the same as the \code{\var{tb}.tb_lineno}
field of the object, but when optimization is used (the -O flag) this
field is not updated correctly; this function calculates the correct
value.
\end{funcdesc}

A simple example follows:

\begin{verbatim}
import sys, traceback

def run_user_code(envdir):
    source = raw_input(">>> ")
    try:
        exec source in envdir
    except:
        print "Exception in user code:"
        print '-'*60
        traceback.print_exc(file=sys.stdout)
        print '-'*60

envdir = {}
while 1:
    run_user_code(envdir)
\end{verbatim}

\section{\module{linecache} ---
         Random access to text lines}

\declaremodule{standard}{linecache}
\sectionauthor{Moshe Zadka}{moshez@zadka.site.co.il}
\modulesynopsis{This module provides random access to individual lines
                from text files.}


The \module{linecache} module allows one to get any line from any file,
while attempting to optimize internally, using a cache, the common case
where many lines are read from a single file.  This is used by the
\refmodule{traceback} module to retrieve source lines for inclusion in 
the formatted traceback.

The \module{linecache} module defines the following functions:

\begin{funcdesc}{getline}{filename, lineno}
Get line \var{lineno} from file named \var{filename}. This function
will never throw an exception --- it will return \code{''} on errors
(the terminating newline character will be included for lines that are
found).

If a file named \var{filename} is not found, the function will look
for it in the module\indexiii{module}{search}{path} search path,
\code{sys.path}.
\end{funcdesc}

\begin{funcdesc}{clearcache}{}
Clear the cache.  Use this function if you no longer need lines from
files previously read using \function{getline()}.
\end{funcdesc}

\begin{funcdesc}{checkcache}{\optional{filename}}
Check the cache for validity.  Use this function if files in the cache 
may have changed on disk, and you require the updated version.  If
\var{filename} is omitted, it will check all the entries in the cache.
\end{funcdesc}

Example:

\begin{verbatim}
>>> import linecache
>>> linecache.getline('/etc/passwd', 4)
'sys:x:3:3:sys:/dev:/bin/sh\n'
\end{verbatim}

\section{\module{pickle} ---
         Python object serialization}

\declaremodule{standard}{pickle}
\modulesynopsis{Convert Python objects to streams of bytes and back.}

\index{persistency}
\indexii{persistent}{objects}
\indexii{serializing}{objects}
\indexii{marshalling}{objects}
\indexii{flattening}{objects}
\indexii{pickling}{objects}


The \module{pickle} module implements a basic but powerful algorithm
for ``pickling'' (a.k.a.\ serializing, marshalling or flattening)
nearly arbitrary Python objects.  This is the act of converting
objects to a stream of bytes (and back: ``unpickling'').  This is a
more primitive notion than persistency --- although \module{pickle}
reads and writes file objects, it does not handle the issue of naming
persistent objects, nor the (even more complicated) area of concurrent
access to persistent objects.  The \module{pickle} module can
transform a complex object into a byte stream and it can transform the
byte stream into an object with the same internal structure.  The most
obvious thing to do with these byte streams is to write them onto a
file, but it is also conceivable to send them across a network or
store them in a database.  The module
\refmodule{shelve}\refstmodindex{shelve} provides a simple interface
to pickle and unpickle objects on DBM-style database files.


\strong{Note:} The \module{pickle} module is rather slow.  A
reimplementation of the same algorithm in C, which is up to 1000 times
faster, is available as the
\refmodule{cPickle}\refbimodindex{cPickle} module.  This has the same
interface except that \class{Pickler} and \class{Unpickler} are
factory functions, not classes (so they cannot be used as base classes
for inheritance).

Unlike the built-in module \refmodule{marshal}\refbimodindex{marshal},
\module{pickle} handles the following correctly:


\begin{itemize}

\item recursive objects (objects containing references to themselves)

\item object sharing (references to the same object in different places)

\item user-defined classes and their instances

\end{itemize}

The data format used by \module{pickle} is Python-specific.  This has
the advantage that there are no restrictions imposed by external
standards such as
XDR\index{XDR}\index{External Data Representation} (which can't
represent pointer sharing); however it means that non-Python programs
may not be able to reconstruct pickled Python objects.

By default, the \module{pickle} data format uses a printable \ASCII{}
representation.  This is slightly more voluminous than a binary
representation.  The big advantage of using printable \ASCII{} (and of
some other characteristics of \module{pickle}'s representation) is that
for debugging or recovery purposes it is possible for a human to read
the pickled file with a standard text editor.

A binary format, which is slightly more efficient, can be chosen by
specifying a nonzero (true) value for the \var{bin} argument to the
\class{Pickler} constructor or the \function{dump()} and \function{dumps()}
functions.  The binary format is not the default because of backwards
compatibility with the Python 1.4 pickle module.  In a future version,
the default may change to binary.

The \module{pickle} module doesn't handle code objects, which the
\refmodule{marshal}\refbimodindex{marshal} module does.  I suppose
\module{pickle} could, and maybe it should, but there's probably no
great need for it right now (as long as \refmodule{marshal} continues
to be used for reading and writing code objects), and at least this
avoids the possibility of smuggling Trojan horses into a program.

For the benefit of persistency modules written using \module{pickle}, it
supports the notion of a reference to an object outside the pickled
data stream.  Such objects are referenced by a name, which is an
arbitrary string of printable \ASCII{} characters.  The resolution of
such names is not defined by the \module{pickle} module --- the
persistent object module will have to implement a method
\method{persistent_load()}.  To write references to persistent objects,
the persistent module must define a method \method{persistent_id()} which
returns either \code{None} or the persistent ID of the object.

There are some restrictions on the pickling of class instances.

First of all, the class must be defined at the top level in a module.
Furthermore, all its instance variables must be picklable.

\setindexsubitem{(pickle protocol)}

When a pickled class instance is unpickled, its \method{__init__()} method
is normally \emph{not} invoked.  \strong{Note:} This is a deviation
from previous versions of this module; the change was introduced in
Python 1.5b2.  The reason for the change is that in many cases it is
desirable to have a constructor that requires arguments; it is a
(minor) nuisance to have to provide a \method{__getinitargs__()} method.

If it is desirable that the \method{__init__()} method be called on
unpickling, a class can define a method \method{__getinitargs__()},
which should return a \emph{tuple} containing the arguments to be
passed to the class constructor (\method{__init__()}).  This method is
called at pickle time; the tuple it returns is incorporated in the
pickle for the instance.
\withsubitem{(copy protocol)}{\ttindex{__getinitargs__()}}
\withsubitem{(instance constructor)}{\ttindex{__init__()}}

Classes can further influence how their instances are pickled --- if
the class
\withsubitem{(copy protocol)}{
  \ttindex{__getstate__()}\ttindex{__setstate__()}}
\withsubitem{(instance attribute)}{
  \ttindex{__dict__}}
defines the method \method{__getstate__()}, it is called and the return
state is pickled as the contents for the instance, and if the class
defines the method \method{__setstate__()}, it is called with the
unpickled state.  (Note that these methods can also be used to
implement copying class instances.)  If there is no
\method{__getstate__()} method, the instance's \member{__dict__} is
pickled.  If there is no \method{__setstate__()} method, the pickled
object must be a dictionary and its items are assigned to the new
instance's dictionary.  (If a class defines both \method{__getstate__()}
and \method{__setstate__()}, the state object needn't be a dictionary
--- these methods can do what they want.)  This protocol is also used
by the shallow and deep copying operations defined in the
\refmodule{copy}\refstmodindex{copy} module.

Note that when class instances are pickled, their class's code and
data are not pickled along with them.  Only the instance data are
pickled.  This is done on purpose, so you can fix bugs in a class or
add methods and still load objects that were created with an earlier
version of the class.  If you plan to have long-lived objects that
will see many versions of a class, it may be worthwhile to put a version
number in the objects so that suitable conversions can be made by the
class's \method{__setstate__()} method.

When a class itself is pickled, only its name is pickled --- the class
definition is not pickled, but re-imported by the unpickling process.
Therefore, the restriction that the class must be defined at the top
level in a module applies to pickled classes as well.

\setindexsubitem{(in module pickle)}

The interface can be summarized as follows.

To pickle an object \code{x} onto a file \code{f}, open for writing:

\begin{verbatim}
p = pickle.Pickler(f)
p.dump(x)
\end{verbatim}

A shorthand for this is:

\begin{verbatim}
pickle.dump(x, f)
\end{verbatim}

To unpickle an object \code{x} from a file \code{f}, open for reading:

\begin{verbatim}
u = pickle.Unpickler(f)
x = u.load()
\end{verbatim}

A shorthand is:

\begin{verbatim}
x = pickle.load(f)
\end{verbatim}

The \class{Pickler} class only calls the method \code{f.write()} with a
\withsubitem{(class in pickle)}{
  \ttindex{Unpickler}\ttindex{Pickler}}
string argument.  The \class{Unpickler} calls the methods \code{f.read()}
(with an integer argument) and \code{f.readline()} (without argument),
both returning a string.  It is explicitly allowed to pass non-file
objects here, as long as they have the right methods.

The constructor for the \class{Pickler} class has an optional second
argument, \var{bin}.  If this is present and nonzero, the binary
pickle format is used; if it is zero or absent, the (less efficient,
but backwards compatible) text pickle format is used.  The
\class{Unpickler} class does not have an argument to distinguish
between binary and text pickle formats; it accepts either format.

The following types can be pickled:

\begin{itemize}

\item \code{None}

\item integers, long integers, floating point numbers

\item strings

\item tuples, lists and dictionaries containing only picklable objects

\item classes that are defined at the top level in a module

\item instances of such classes whose \member{__dict__} or
\method{__setstate__()} is picklable

\end{itemize}

Attempts to pickle unpicklable objects will raise the
\exception{PicklingError} exception; when this happens, an unspecified
number of bytes may have been written to the file.

It is possible to make multiple calls to the \method{dump()} method of
the same \class{Pickler} instance.  These must then be matched to the
same number of calls to the \method{load()} method of the
corresponding \class{Unpickler} instance.  If the same object is
pickled by multiple \method{dump()} calls, the \method{load()} will all
yield references to the same object.  \emph{Warning}: this is intended
for pickling multiple objects without intervening modifications to the
objects or their parts.  If you modify an object and then pickle it
again using the same \class{Pickler} instance, the object is not
pickled again --- a reference to it is pickled and the
\class{Unpickler} will return the old value, not the modified one.
(There are two problems here: (a) detecting changes, and (b)
marshalling a minimal set of changes.  I have no answers.  Garbage
Collection may also become a problem here.)

Apart from the \class{Pickler} and \class{Unpickler} classes, the
module defines the following functions, and an exception:

\begin{funcdesc}{dump}{object, file\optional{, bin}}
Write a pickled representation of \var{obect} to the open file object
\var{file}.  This is equivalent to
\samp{Pickler(\var{file}, \var{bin}).dump(\var{object})}.
If the optional \var{bin} argument is present and nonzero, the binary
pickle format is used; if it is zero or absent, the (less efficient)
text pickle format is used.
\end{funcdesc}

\begin{funcdesc}{load}{file}
Read a pickled object from the open file object \var{file}.  This is
equivalent to \samp{Unpickler(\var{file}).load()}.
\end{funcdesc}

\begin{funcdesc}{dumps}{object\optional{, bin}}
Return the pickled representation of the object as a string, instead
of writing it to a file.  If the optional \var{bin} argument is
present and nonzero, the binary pickle format is used; if it is zero
or absent, the (less efficient) text pickle format is used.
\end{funcdesc}

\begin{funcdesc}{loads}{string}
Read a pickled object from a string instead of a file.  Characters in
the string past the pickled object's representation are ignored.
\end{funcdesc}

\begin{excdesc}{PicklingError}
This exception is raised when an unpicklable object is passed to
\method{Pickler.dump()}.
\end{excdesc}


\begin{seealso}
  \seemodule[copyreg]{copy_reg}{pickle interface constructor
                                registration}

  \seemodule{shelve}{indexed databases of objects; uses \module{pickle}}

  \seemodule{copy}{shallow and deep object copying}

  \seemodule{marshal}{high-performance serialization of built-in types}
\end{seealso}


\section{\module{cPickle} ---
         Alternate implementation of \module{pickle}}

\declaremodule{builtin}{cPickle}
\modulesynopsis{Faster version of \module{pickle}, but not subclassable.}
\moduleauthor{Jim Fulton}{jfulton@digicool.com}
\sectionauthor{Fred L. Drake, Jr.}{fdrake@acm.org}


The \module{cPickle} module provides a similar interface and identical
functionality as the \refmodule{pickle}\refstmodindex{pickle} module,
but can be up to 1000 times faster since it is implemented in C.  The
only other important difference to note is that \function{Pickler()}
and \function{Unpickler()} are functions and not classes, and so
cannot be subclassed.  This should not be an issue in most cases.

The format of the pickle data is identical to that produced using the
\refmodule{pickle} module, so it is possible to use \refmodule{pickle} and
\module{cPickle} interchangably with existing pickles.

(Since the pickle data format is actually a tiny stack-oriented
programming language, and there are some freedoms in the encodings of
certain objects, it's possible that the two modules produce different
pickled data for the same input objects; however they will always be
able to read each others pickles back in.)

\section{\module{copy_reg} ---
         Register \module{pickle} support functions}

\declaremodule[copyreg]{standard}{copy_reg}
\modulesynopsis{Register \module{pickle} support functions.}


The \module{copy_reg} module provides support for the
\refmodule{pickle}\refstmodindex{pickle}\ and
\refmodule{cPickle}\refbimodindex{cPickle}\ modules.  The
\refmodule{copy}\refstmodindex{copy}\ module is likely to use this in the
future as well.  It provides configuration information about object
constructors which are not classes.  Such constructors may be factory
functions or class instances.


\begin{funcdesc}{constructor}{object}
  Declares \var{object} to be a valid constructor.  If \var{object} is
  not callable (and hence not valid as a constructor), raises
  \exception{TypeError}.
\end{funcdesc}

\begin{funcdesc}{pickle}{type, function\optional{, constructor}}
  Declares that \var{function} should be used as a ``reduction''
  function for objects of type \var{type}; \var{type} must not be a
  ``classic'' class object.  (Classic classes are handled differently;
  see the documentation for the \refmodule{pickle} module for
  details.)  \var{function} should return either a string or a tuple
  containing two or three elements.

  The optional \var{constructor} parameter, if provided, is a
  callable object which can be used to reconstruct the object when
  called with the tuple of arguments returned by \var{function} at
  pickling time.  \exception{TypeError} will be raised if
  \var{object} is a class or \var{constructor} is not callable.

  See the \refmodule{pickle} module for more
  details on the interface expected of \var{function} and
  \var{constructor}.
\end{funcdesc}
		% really copy_reg
\section{Standard Module \module{shelve}}
\label{module-shelve}
\stmodindex{shelve}

A ``shelf'' is a persistent, dictionary-like object.  The difference
with ``dbm'' databases is that the values (not the keys!) in a shelf
can be essentially arbitrary Python objects --- anything that the
\code{pickle} module can handle.  This includes most class instances,
recursive data types, and objects containing lots of shared
sub-objects.  The keys are ordinary strings.
\refstmodindex{pickle}

To summarize the interface (\code{key} is a string, \code{data} is an
arbitrary object):

\begin{verbatim}
import shelve

d = shelve.open(filename) # open, with (g)dbm filename -- no suffix

d[key] = data   # store data at key (overwrites old data if
                # using an existing key)
data = d[key]   # retrieve data at key (raise KeyError if no
                # such key)
del d[key]      # delete data stored at key (raises KeyError
                # if no such key)
flag = d.has_key(key)   # true if the key exists
list = d.keys() # a list of all existing keys (slow!)

d.close()       # close it
\end{verbatim}
%
Restrictions:

\begin{itemize}

\item
The choice of which database package will be used (e.g. \code{dbm} or
\code{gdbm})
depends on which interface is available.  Therefore it isn't safe to
open the database directly using \code{dbm}.  The database is also
(unfortunately) subject to the limitations of \code{dbm}, if it is used ---
this means that (the pickled representation of) the objects stored in
the database should be fairly small, and in rare cases key collisions
may cause the database to refuse updates.
\refbimodindex{dbm}
\refbimodindex{gdbm}

\item
Dependent on the implementation, closing a persistent dictionary may
or may not be necessary to flush changes to disk.

\item
The \code{shelve} module does not support \emph{concurrent} read/write
access to shelved objects.  (Multiple simultaneous read accesses are
safe.)  When a program has a shelf open for writing, no other program
should have it open for reading or writing.  \UNIX{} file locking can
be used to solve this, but this differs across \UNIX{} versions and
requires knowledge about the database implementation used.

\end{itemize}

\section{\module{copy} ---
         Shallow and deep copy operations}
\declaremodule{standard}{copy}

\modulesynopsis{Shallow and deep copy operations.}


This module provides generic (shallow and deep) copying operations.
\withsubitem{(in copy)}{\ttindex{copy()}\ttindex{deepcopy()}}

Interface summary:

\begin{verbatim}
import copy

x = copy.copy(y)        # make a shallow copy of y
x = copy.deepcopy(y)    # make a deep copy of y
\end{verbatim}
%
For module specific errors, \exception{copy.error} is raised.

The difference between shallow and deep copying is only relevant for
compound objects (objects that contain other objects, like lists or
class instances):

\begin{itemize}

\item
A \emph{shallow copy} constructs a new compound object and then (to the
extent possible) inserts \emph{references} into it to the objects found
in the original.

\item
A \emph{deep copy} constructs a new compound object and then,
recursively, inserts \emph{copies} into it of the objects found in the
original.

\end{itemize}

Two problems often exist with deep copy operations that don't exist
with shallow copy operations:

\begin{itemize}

\item
Recursive objects (compound objects that, directly or indirectly,
contain a reference to themselves) may cause a recursive loop.

\item
Because deep copy copies \emph{everything} it may copy too much,
e.g., administrative data structures that should be shared even
between copies.

\end{itemize}

The \function{deepcopy()} function avoids these problems by:

\begin{itemize}

\item
keeping a ``memo'' dictionary of objects already copied during the current
copying pass; and

\item
letting user-defined classes override the copying operation or the
set of components copied.

\end{itemize}

This version does not copy types like module, class, function, method,
nor stack trace, stack frame, nor file, socket, window, nor array, nor
any similar types.

Classes can use the same interfaces to control copying that they use
to control pickling: they can define methods called
\method{__getinitargs__()}, \method{__getstate__()} and
\method{__setstate__()}.  See the description of module
\refmodule{pickle}\refstmodindex{pickle} for information on these
methods.  The \module{copy} module does not use the \module{copy_reg}
registration module.
\withsubitem{(copy protocol)}{\ttindex{__getinitargs__()}
  \ttindex{__getstate__()}\ttindex{__setstate__()}}

In order for a class to define its own copy implementation, it can
define special methods \method{__copy__()} and
\method{__deepcopy__()}.  The former is called to implement the
shallow copy operation; no additional arguments are passed.  The
latter is called to implement the deep copy operation; it is passed
one argument, the memo dictionary.  If the \method{__deepcopy__()}
implementation needs to make a deep copy of a component, it should
call the \function{deepcopy()} function with the component as first
argument and the memo dictionary as second argument.
\withsubitem{(copy protocol)}{\ttindex{__copy__()}\ttindex{__deepcopy__()}}

\begin{seealso}
\seemodule{pickle}{Discussion of the special disciplines used to
support object state retrieval and restoration.}
\end{seealso}

\section{\module{marshal} ---
         Internal Python object serialization}

\declaremodule{builtin}{marshal}
\modulesynopsis{Convert Python objects to streams of bytes and back
                (with different constraints).}


This module contains functions that can read and write Python
values in a binary format.  The format is specific to Python, but
independent of machine architecture issues (e.g., you can write a
Python value to a file on a PC, transport the file to a Sun, and read
it back there).  Details of the format are undocumented on purpose;
it may change between Python versions (although it rarely
does).\footnote{The name of this module stems from a bit of
  terminology used by the designers of Modula-3 (amongst others), who
  use the term ``marshalling'' for shipping of data around in a
  self-contained form. Strictly speaking, ``to marshal'' means to
  convert some data from internal to external form (in an RPC buffer for
  instance) and ``unmarshalling'' for the reverse process.}

This is not a general ``persistence'' module.  For general persistence
and transfer of Python objects through RPC calls, see the modules
\refmodule{pickle} and \refmodule{shelve}.  The \module{marshal} module exists
mainly to support reading and writing the ``pseudo-compiled'' code for
Python modules of \file{.pyc} files.  Therefore, the Python
maintainers reserve the right to modify the marshal format in backward
incompatible ways should the need arise.  If you're serializing and
de-serializing Python objects, use the \module{pickle} module instead.  
\refstmodindex{pickle}
\refstmodindex{shelve}
\obindex{code}

\begin{notice}[warning]
The \module{marshal} module is not intended to be secure against
erroneous or maliciously constructed data.  Never unmarshal data
received from an untrusted or unauthenticated source.
\end{notice}

Not all Python object types are supported; in general, only objects
whose value is independent from a particular invocation of Python can
be written and read by this module.  The following types are supported:
\code{None}, integers, long integers, floating point numbers,
strings, Unicode objects, tuples, lists, dictionaries, and code
objects, where it should be understood that tuples, lists and
dictionaries are only supported as long as the values contained
therein are themselves supported; and recursive lists and dictionaries
should not be written (they will cause infinite loops).

\strong{Caveat:} On machines where C's \code{long int} type has more than
32 bits (such as the DEC Alpha), it is possible to create plain Python
integers that are longer than 32 bits.
If such an integer is marshaled and read back in on a machine where
C's \code{long int} type has only 32 bits, a Python long integer object
is returned instead.  While of a different type, the numeric value is
the same.  (This behavior is new in Python 2.2.  In earlier versions,
all but the least-significant 32 bits of the value were lost, and a
warning message was printed.)

There are functions that read/write files as well as functions
operating on strings.

The module defines these functions:

\begin{funcdesc}{dump}{value, file\optional{, version}}
  Write the value on the open file.  The value must be a supported
  type.  The file must be an open file object such as
  \code{sys.stdout} or returned by \function{open()} or
  \function{posix.popen()}.  It must be opened in binary mode
  (\code{'wb'} or \code{'w+b'}).

  If the value has (or contains an object that has) an unsupported type,
  a \exception{ValueError} exception is raised --- but garbage data
  will also be written to the file.  The object will not be properly
  read back by \function{load()}.

  \versionadded[The \var{version} argument indicates the data
  format that \code{dump} should use (see below)]{2.4}
\end{funcdesc}

\begin{funcdesc}{load}{file}
  Read one value from the open file and return it.  If no valid value
  is read, raise \exception{EOFError}, \exception{ValueError} or
  \exception{TypeError}.  The file must be an open file object opened
  in binary mode (\code{'rb'} or \code{'r+b'}).

  \warning{If an object containing an unsupported type was
  marshalled with \function{dump()}, \function{load()} will substitute
  \code{None} for the unmarshallable type.}
\end{funcdesc}

\begin{funcdesc}{dumps}{value\optional{, version}}
  Return the string that would be written to a file by
  \code{dump(\var{value}, \var{file})}.  The value must be a supported
  type.  Raise a \exception{ValueError} exception if value has (or
  contains an object that has) an unsupported type.

  \versionadded[The \var{version} argument indicates the data
  format that \code{dumps} should use (see below)]{2.4}
\end{funcdesc}

\begin{funcdesc}{loads}{string}
  Convert the string to a value.  If no valid value is found, raise
  \exception{EOFError}, \exception{ValueError} or
  \exception{TypeError}.  Extra characters in the string are ignored.
\end{funcdesc}

In addition, the following constants are defined:

\begin{datadesc}{version}
  Indicates the format that the module uses. Version 0 is the
  historical format, version 1 (added in Python 2.4) shares
  interned strings. The current version is 1.

  \versionadded{2.4}
\end{datadesc}
\section{\module{imp} ---
         Access the \keyword{import} internals}

\declaremodule{builtin}{imp}
\modulesynopsis{Access the implementation of the \keyword{import} statement.}


This\stindex{import} module provides an interface to the mechanisms
used to implement the \keyword{import} statement.  It defines the
following constants and functions:


\begin{funcdesc}{get_magic}{}
\indexii{file}{byte-code}
Return the magic string value used to recognize byte-compiled code
files (\file{.pyc} files).  (This value may be different for each
Python version.)
\end{funcdesc}

\begin{funcdesc}{get_suffixes}{}
Return a list of triples, each describing a particular type of module.
Each triple has the form \code{(\var{suffix}, \var{mode},
\var{type})}, where \var{suffix} is a string to be appended to the
module name to form the filename to search for, \var{mode} is the mode
string to pass to the built-in \function{open()} function to open the
file (this can be \code{'r'} for text files or \code{'rb'} for binary
files), and \var{type} is the file type, which has one of the values
\constant{PY_SOURCE}, \constant{PY_COMPILED}, or
\constant{C_EXTENSION}, described below.
\end{funcdesc}

\begin{funcdesc}{find_module}{name\optional{, path}}
Try to find the module \var{name} on the search path \var{path}.  If
\var{path} is a list of directory names, each directory is searched
for files with any of the suffixes returned by \function{get_suffixes()}
above.  Invalid names in the list are silently ignored (but all list
items must be strings).  If \var{path} is omitted or \code{None}, the
list of directory names given by \code{sys.path} is searched, but
first it searches a few special places: it tries to find a built-in
module with the given name (\constant{C_BUILTIN}), then a frozen module
(\constant{PY_FROZEN}), and on some systems some other places are looked
in as well (on the Mac, it looks for a resource (\constant{PY_RESOURCE});
on Windows, it looks in the registry which may point to a specific
file).

If search is successful, the return value is a triple
\code{(\var{file}, \var{pathname}, \var{description})} where
\var{file} is an open file object positioned at the beginning,
\var{pathname} is the pathname of the
file found, and \var{description} is a triple as contained in the list
returned by \function{get_suffixes()} describing the kind of module found.
If the module does not live in a file, the returned \var{file} is
\code{None}, \var{filename} is the empty string, and the
\var{description} tuple contains empty strings for its suffix and
mode; the module type is as indicate in parentheses above.  If the
search is unsuccessful, \exception{ImportError} is raised.  Other
exceptions indicate problems with the arguments or environment.

This function does not handle hierarchical module names (names
containing dots).  In order to find \var{P}.\var{M}, that is, submodule
\var{M} of package \var{P}, use \function{find_module()} and
\function{load_module()} to find and load package \var{P}, and then use
\function{find_module()} with the \var{path} argument set to
\code{\var{P}.__path__}.  When \var{P} itself has a dotted name, apply
this recipe recursively.
\end{funcdesc}

\begin{funcdesc}{load_module}{name, file, filename, description}
Load a module that was previously found by \function{find_module()} (or by
an otherwise conducted search yielding compatible results).  This
function does more than importing the module: if the module was
already imported, it is equivalent to a
\function{reload()}\bifuncindex{reload}!  The \var{name} argument
indicates the full module name (including the package name, if this is
a submodule of a package).  The \var{file} argument is an open file,
and \var{filename} is the corresponding file name; these can be
\code{None} and \code{''}, respectively, when the module is not being
loaded from a file.  The \var{description} argument is a tuple, as
would be returned by \function{get_suffixes()}, describing what kind
of module must be loaded.

If the load is successful, the return value is the module object;
otherwise, an exception (usually \exception{ImportError}) is raised.

\strong{Important:} the caller is responsible for closing the
\var{file} argument, if it was not \code{None}, even when an exception
is raised.  This is best done using a \keyword{try}
... \keyword{finally} statement.
\end{funcdesc}

\begin{funcdesc}{new_module}{name}
Return a new empty module object called \var{name}.  This object is
\emph{not} inserted in \code{sys.modules}.
\end{funcdesc}

\begin{funcdesc}{lock_held}{}
Return \code{True} if the import lock is currently held, else \code{False}.
On platforms without threads, always return \code{False}.

On platforms with threads, a thread executing an import holds an internal
lock until the import is complete.
This lock blocks other threads from doing an import until the original
import completes, which in turn prevents other threads from seeing
incomplete module objects constructed by the original thread while in
the process of completing its import (and the imports, if any,
triggered by that).
\end{funcdesc}

\begin{funcdesc}{set_frozenmodules}{seq_of_tuples}
Set the global list of frozen modules.  \var{seq_of_tuples} is a sequence
of tuples of length 3: (\var{modulename}, \var{codedata}, \var{ispkg})
\var{modulename} is the name of the frozen module (may contain dots).
\var{codedata} is a marshalled code object. \var{ispkg} is a boolean
indicating whether the module is a package.
\versionadded{2.3}
\end{funcdesc}

\begin{funcdesc}{get_frozenmodules}{}
Return the global list of frozen modules as a tuple of tuples. See
\function{set_frozenmodules()} for a description of its contents.
\versionadded{2.3}
\end{funcdesc}

The following constants with integer values, defined in this module,
are used to indicate the search result of \function{find_module()}.

\begin{datadesc}{PY_SOURCE}
The module was found as a source file.
\end{datadesc}

\begin{datadesc}{PY_COMPILED}
The module was found as a compiled code object file.
\end{datadesc}

\begin{datadesc}{C_EXTENSION}
The module was found as dynamically loadable shared library.
\end{datadesc}

\begin{datadesc}{PY_RESOURCE}
The module was found as a Macintosh resource.  This value can only be
returned on a Macintosh.
\end{datadesc}

\begin{datadesc}{PKG_DIRECTORY}
The module was found as a package directory.
\end{datadesc}

\begin{datadesc}{C_BUILTIN}
The module was found as a built-in module.
\end{datadesc}

\begin{datadesc}{PY_FROZEN}
The module was found as a frozen module (see \function{init_frozen()}).
\end{datadesc}

The following constant and functions are obsolete; their functionality
is available through \function{find_module()} or \function{load_module()}.
They are kept around for backward compatibility:

\begin{datadesc}{SEARCH_ERROR}
Unused.
\end{datadesc}

\begin{funcdesc}{init_builtin}{name}
Initialize the built-in module called \var{name} and return its module
object.  If the module was already initialized, it will be initialized
\emph{again}.  A few modules cannot be initialized twice --- attempting
to initialize these again will raise an \exception{ImportError}
exception.  If there is no
built-in module called \var{name}, \code{None} is returned.
\end{funcdesc}

\begin{funcdesc}{init_frozen}{name}
Initialize the frozen module called \var{name} and return its module
object.  If the module was already initialized, it will be initialized
\emph{again}.  If there is no frozen module called \var{name},
\code{None} is returned.  (Frozen modules are modules written in
Python whose compiled byte-code object is incorporated into a
custom-built Python interpreter by Python's \program{freeze} utility.
See \file{Tools/freeze/} for now.)
\end{funcdesc}

\begin{funcdesc}{is_builtin}{name}
Return \code{1} if there is a built-in module called \var{name} which
can be initialized again.  Return \code{-1} if there is a built-in
module called \var{name} which cannot be initialized again (see
\function{init_builtin()}).  Return \code{0} if there is no built-in
module called \var{name}.
\end{funcdesc}

\begin{funcdesc}{is_frozen}{name}
Return \code{True} if there is a frozen module (see
\function{init_frozen()}) called \var{name}, or \code{False} if there is
no such module.
\end{funcdesc}

\begin{funcdesc}{load_compiled}{name, pathname, file}
\indexii{file}{byte-code}
Load and initialize a module implemented as a byte-compiled code file
and return its module object.  If the module was already initialized,
it will be initialized \emph{again}.  The \var{name} argument is used
to create or access a module object.  The \var{pathname} argument
points to the byte-compiled code file.  The \var{file}
argument is the byte-compiled code file, open for reading in binary
mode, from the beginning.
It must currently be a real file object, not a
user-defined class emulating a file.
\end{funcdesc}

\begin{funcdesc}{load_dynamic}{name, pathname\optional{, file}}
Load and initialize a module implemented as a dynamically loadable
shared library and return its module object.  If the module was
already initialized, it will be initialized \emph{again}.  Some modules
don't like that and may raise an exception.  The \var{pathname}
argument must point to the shared library.  The \var{name} argument is
used to construct the name of the initialization function: an external
C function called \samp{init\var{name}()} in the shared library is
called.  The optional \var{file} argument is ignored.  (Note: using
shared libraries is highly system dependent, and not all systems
support it.)
\end{funcdesc}

\begin{funcdesc}{load_source}{name, pathname, file}
Load and initialize a module implemented as a Python source file and
return its module object.  If the module was already initialized, it
will be initialized \emph{again}.  The \var{name} argument is used to
create or access a module object.  The \var{pathname} argument points
to the source file.  The \var{file} argument is the source
file, open for reading as text, from the beginning.
It must currently be a real file
object, not a user-defined class emulating a file.  Note that if a
properly matching byte-compiled file (with suffix \file{.pyc} or
\file{.pyo}) exists, it will be used instead of parsing the given
source file.
\end{funcdesc}


\subsection{Examples}
\label{examples-imp}

The following function emulates what was the standard import statement
up to Python 1.4 (no hierarchical module names).  (This
\emph{implementation} wouldn't work in that version, since
\function{find_module()} has been extended and
\function{load_module()} has been added in 1.4.)

\begin{verbatim}
import imp
import sys

def __import__(name, globals=None, locals=None, fromlist=None):
    # Fast path: see if the module has already been imported.
    try:
        return sys.modules[name]
    except KeyError:
        pass

    # If any of the following calls raises an exception,
    # there's a problem we can't handle -- let the caller handle it.

    fp, pathname, description = imp.find_module(name)
    
    try:
        return imp.load_module(name, fp, pathname, description)
    finally:
        # Since we may exit via an exception, close fp explicitly.
        if fp:
            fp.close()
\end{verbatim}

A more complete example that implements hierarchical module names and
includes a \function{reload()}\bifuncindex{reload} function can be
found in the module \module{knee}\refmodindex{knee}.  The
\module{knee} module can be found in \file{Demo/imputil/} in the
Python source distribution.

%\section{Built-in Module \sectcode{ni}}
\label{module-ni}
\bimodindex{ni}

\strong{Warning: This module is obsolete.}  As of Python 1.5a4,
package support (with different semantics for \code{__init__} and no
support for \code{__domain__} or\code{__}) is built in the
interpreter.  The ni module is retained only for backward
compatibility.

The \code{ni} module defines a new importing scheme, which supports
packages containing several Python modules.  To enable package
support, execute \code{import ni} before importing any packages.  Importing
this module automatically installs the relevant import hooks.  There
are no publicly-usable functions or variables in the \code{ni} module.

To create a package named \code{spam} containing sub-modules \code{ham}, \code{bacon} and
\code{eggs}, create a directory \file{spam} somewhere on Python's module search
path, as given in \code{sys.path}.  Then, create files called \file{ham.py}, \file{bacon.py} and
\file{eggs.py} inside \file{spam}.

To import module \code{ham} from package \code{spam} and use function
\code{hamneggs()} from that module, you can use any of the following
possibilities:

\bcode\begin{verbatim}
import spam.ham		# *not* "import spam" !!!
spam.ham.hamneggs()
\end{verbatim}\ecode
%
\bcode\begin{verbatim}
from spam import ham
ham.hamneggs()
\end{verbatim}\ecode
%
\bcode\begin{verbatim}
from spam.ham import hamneggs
hamneggs()
\end{verbatim}\ecode
%
\code{import spam} creates an
empty package named \code{spam} if one does not already exist, but it does
\emph{not} automatically import \code{spam}'s submodules.  
The only submodule that is guaranteed to be imported is
\code{spam.__init__}, if it exists; it would be in a file named
\file{__init__.py} in the \file{spam} directory.  Note that
\code{spam.__init__} is a submodule of package spam.  It can refer to
spam's namespace as \code{__} (two underscores):

\bcode\begin{verbatim}
__.spam_inited = 1		# Set a package-level variable
\end{verbatim}\ecode
%
Additional initialization code (setting up variables, importing other
submodules) can be performed in \file{spam/__init__.py}.

% libparser.tex
%
% Copyright 1995 Virginia Polytechnic Institute and State University
% and Fred L. Drake, Jr.  This copyright notice must be distributed on
% all copies, but this document otherwise may be distributed as part
% of the Python distribution.  No fee may be charged for this document
% in any representation, either on paper or electronically.  This
% restriction does not affect other elements in a distributed package
% in any way.
%

\section{Built-in Module \module{parser}}
\label{module-parser}
\bimodindex{parser}
\index{parsing!Python source code}

The \module{parser} module provides an interface to Python's internal
parser and byte-code compiler.  The primary purpose for this interface
is to allow Python code to edit the parse tree of a Python expression
and create executable code from this.  This is better than trying
to parse and modify an arbitrary Python code fragment as a string
because parsing is performed in a manner identical to the code
forming the application.  It is also faster.

The \module{parser} module was written and documented by Fred
L. Drake, Jr. (\email{fdrake@acm.org}).%
\index{Drake, Fred L., Jr.}

There are a few things to note about this module which are important
to making use of the data structures created.  This is not a tutorial
on editing the parse trees for Python code, but some examples of using
the \module{parser} module are presented.

Most importantly, a good understanding of the Python grammar processed
by the internal parser is required.  For full information on the
language syntax, refer to the \emph{Python Language Reference}.  The
parser itself is created from a grammar specification defined in the file
\file{Grammar/Grammar} in the standard Python distribution.  The parse
trees stored in the AST objects created by this module are the
actual output from the internal parser when created by the
\function{expr()} or \function{suite()} functions, described below.  The AST
objects created by \function{sequence2ast()} faithfully simulate those
structures.  Be aware that the values of the sequences which are
considered ``correct'' will vary from one version of Python to another
as the formal grammar for the language is revised.  However,
transporting code from one Python version to another as source text
will always allow correct parse trees to be created in the target
version, with the only restriction being that migrating to an older
version of the interpreter will not support more recent language
constructs.  The parse trees are not typically compatible from one
version to another, whereas source code has always been
forward-compatible.

Each element of the sequences returned by \function{ast2list()} or
\function{ast2tuple()} has a simple form.  Sequences representing
non-terminal elements in the grammar always have a length greater than
one.  The first element is an integer which identifies a production in
the grammar.  These integers are given symbolic names in the C header
file \file{Include/graminit.h} and the Python module
\module{symbol}.  Each additional element of the sequence represents
a component of the production as recognized in the input string: these
are always sequences which have the same form as the parent.  An
important aspect of this structure which should be noted is that
keywords used to identify the parent node type, such as the keyword
\keyword{if} in an \constant{if_stmt}, are included in the node tree without
any special treatment.  For example, the \keyword{if} keyword is
represented by the tuple \code{(1, 'if')}, where \code{1} is the
numeric value associated with all \code{NAME} tokens, including
variable and function names defined by the user.  In an alternate form
returned when line number information is requested, the same token
might be represented as \code{(1, 'if', 12)}, where the \code{12}
represents the line number at which the terminal symbol was found.

Terminal elements are represented in much the same way, but without
any child elements and the addition of the source text which was
identified.  The example of the \keyword{if} keyword above is
representative.  The various types of terminal symbols are defined in
the C header file \file{Include/token.h} and the Python module
\module{token}.

The AST objects are not required to support the functionality of this
module, but are provided for three purposes: to allow an application
to amortize the cost of processing complex parse trees, to provide a
parse tree representation which conserves memory space when compared
to the Python list or tuple representation, and to ease the creation
of additional modules in C which manipulate parse trees.  A simple
``wrapper'' class may be created in Python to hide the use of AST
objects.

The \module{parser} module defines functions for a few distinct
purposes.  The most important purposes are to create AST objects and
to convert AST objects to other representations such as parse trees
and compiled code objects, but there are also functions which serve to
query the type of parse tree represented by an AST object.


\subsection{Creating AST Objects}
\label{Creating ASTs}

AST objects may be created from source code or from a parse tree.
When creating an AST object from source, different functions are used
to create the \code{'eval'} and \code{'exec'} forms.

\begin{funcdesc}{expr}{string}
The \function{expr()} function parses the parameter \code{\var{string}}
as if it were an input to \samp{compile(\var{string}, 'eval')}.  If
the parse succeeds, an AST object is created to hold the internal
parse tree representation, otherwise an appropriate exception is
thrown.
\end{funcdesc}

\begin{funcdesc}{suite}{string}
The \function{suite()} function parses the parameter \code{\var{string}}
as if it were an input to \samp{compile(\var{string}, 'exec')}.  If
the parse succeeds, an AST object is created to hold the internal
parse tree representation, otherwise an appropriate exception is
thrown.
\end{funcdesc}

\begin{funcdesc}{sequence2ast}{sequence}
This function accepts a parse tree represented as a sequence and
builds an internal representation if possible.  If it can validate
that the tree conforms to the Python grammar and all nodes are valid
node types in the host version of Python, an AST object is created
from the internal representation and returned to the called.  If there
is a problem creating the internal representation, or if the tree
cannot be validated, a \exception{ParserError} exception is thrown.  An AST
object created this way should not be assumed to compile correctly;
normal exceptions thrown by compilation may still be initiated when
the AST object is passed to \function{compileast()}.  This may indicate
problems not related to syntax (such as a \exception{MemoryError}
exception), but may also be due to constructs such as the result of
parsing \code{del f(0)}, which escapes the Python parser but is
checked by the bytecode compiler.

Sequences representing terminal tokens may be represented as either
two-element lists of the form \code{(1, 'name')} or as three-element
lists of the form \code{(1, 'name', 56)}.  If the third element is
present, it is assumed to be a valid line number.  The line number
may be specified for any subset of the terminal symbols in the input
tree.
\end{funcdesc}

\begin{funcdesc}{tuple2ast}{sequence}
This is the same function as \function{sequence2ast()}.  This entry point
is maintained for backward compatibility.
\end{funcdesc}


\subsection{Converting AST Objects}
\label{Converting ASTs}

AST objects, regardless of the input used to create them, may be
converted to parse trees represented as list- or tuple- trees, or may
be compiled into executable code objects.  Parse trees may be
extracted with or without line numbering information.

\begin{funcdesc}{ast2list}{ast\optional{, line_info}}
This function accepts an AST object from the caller in
\code{\var{ast}} and returns a Python list representing the
equivelent parse tree.  The resulting list representation can be used
for inspection or the creation of a new parse tree in list form.  This
function does not fail so long as memory is available to build the
list representation.  If the parse tree will only be used for
inspection, \function{ast2tuple()} should be used instead to reduce memory
consumption and fragmentation.  When the list representation is
required, this function is significantly faster than retrieving a
tuple representation and converting that to nested lists.

If \code{\var{line_info}} is true, line number information will be
included for all terminal tokens as a third element of the list
representing the token.  Note that the line number provided specifies
the line on which the token \emph{ends}.  This information is
omitted if the flag is false or omitted.
\end{funcdesc}

\begin{funcdesc}{ast2tuple}{ast\optional{, line_info}}
This function accepts an AST object from the caller in
\code{\var{ast}} and returns a Python tuple representing the
equivelent parse tree.  Other than returning a tuple instead of a
list, this function is identical to \function{ast2list()}.

If \code{\var{line_info}} is true, line number information will be
included for all terminal tokens as a third element of the list
representing the token.  This information is omitted if the flag is
false or omitted.
\end{funcdesc}

\begin{funcdesc}{compileast}{ast\optional{, filename\code{ = '<ast>'}}}
The Python byte compiler can be invoked on an AST object to produce
code objects which can be used as part of an \code{exec} statement or
a call to the built-in \function{eval()}\bifuncindex{eval} function.
This function provides the interface to the compiler, passing the
internal parse tree from \code{\var{ast}} to the parser, using the
source file name specified by the \code{\var{filename}} parameter.
The default value supplied for \code{\var{filename}} indicates that
the source was an AST object.

Compiling an AST object may result in exceptions related to
compilation; an example would be a \exception{SyntaxError} caused by the
parse tree for \code{del f(0)}: this statement is considered legal
within the formal grammar for Python but is not a legal language
construct.  The \exception{SyntaxError} raised for this condition is
actually generated by the Python byte-compiler normally, which is why
it can be raised at this point by the \module{parser} module.  Most
causes of compilation failure can be diagnosed programmatically by
inspection of the parse tree.
\end{funcdesc}


\subsection{Queries on AST Objects}
\label{Querying ASTs}

Two functions are provided which allow an application to determine if
an AST was created as an expression or a suite.  Neither of these
functions can be used to determine if an AST was created from source
code via \function{expr()} or \function{suite()} or from a parse tree
via \function{sequence2ast()}.

\begin{funcdesc}{isexpr}{ast}
When \code{\var{ast}} represents an \code{'eval'} form, this function
returns true, otherwise it returns false.  This is useful, since code
objects normally cannot be queried for this information using existing
built-in functions.  Note that the code objects created by
\function{compileast()} cannot be queried like this either, and are
identical to those created by the built-in
\function{compile()}\bifuncindex{compile} function.
\end{funcdesc}


\begin{funcdesc}{issuite}{ast}
This function mirrors \function{isexpr()} in that it reports whether an
AST object represents an \code{'exec'} form, commonly known as a
``suite.''  It is not safe to assume that this function is equivelent
to \samp{not isexpr(\var{ast})}, as additional syntactic fragments may
be supported in the future.
\end{funcdesc}


\subsection{Exceptions and Error Handling}
\label{AST Errors}

The parser module defines a single exception, but may also pass other
built-in exceptions from other portions of the Python runtime
environment.  See each function for information about the exceptions
it can raise.

\begin{excdesc}{ParserError}
Exception raised when a failure occurs within the parser module.  This
is generally produced for validation failures rather than the built in
\exception{SyntaxError} thrown during normal parsing.
The exception argument is either a string describing the reason of the
failure or a tuple containing a sequence causing the failure from a parse
tree passed to \function{sequence2ast()} and an explanatory string.  Calls to
\function{sequence2ast()} need to be able to handle either type of exception,
while calls to other functions in the module will only need to be
aware of the simple string values.
\end{excdesc}

Note that the functions \function{compileast()}, \function{expr()}, and
\function{suite()} may throw exceptions which are normally thrown by the
parsing and compilation process.  These include the built in
exceptions \exception{MemoryError}, \exception{OverflowError},
\exception{SyntaxError}, and \exception{SystemError}.  In these cases, these
exceptions carry all the meaning normally associated with them.  Refer
to the descriptions of each function for detailed information.


\subsection{AST Objects}
\label{AST Objects}

AST objects returned by \function{expr()}, \function{suite()} and
\function{sequence2ast()} have no methods of their own.
Some of the functions defined which accept an AST object as their
first argument may change to object methods in the future.

Ordered and equality comparisons are supported between AST objects.
Pickling of AST objects (using the \module{pickle} module) is also
supported.

\begin{datadesc}{ASTType}
The type of the objects returned by \function{expr()},
\function{suite()} and \function{sequence2ast()}.
\end{datadesc}


\subsection{Examples}
\nodename{AST Examples}

The parser modules allows operations to be performed on the parse tree
of Python source code before the bytecode is generated, and provides
for inspection of the parse tree for information gathering purposes.
Two examples are presented.  The simple example demonstrates emulation
of the \function{compile()}\bifuncindex{compile} built-in function and
the complex example shows the use of a parse tree for information
discovery.

\subsubsection{Emulation of \function{compile()}}

While many useful operations may take place between parsing and
bytecode generation, the simplest operation is to do nothing.  For
this purpose, using the \module{parser} module to produce an
intermediate data structure is equivelent to the code

\begin{verbatim}
>>> code = compile('a + 5', 'eval')
>>> a = 5
>>> eval(code)
10
\end{verbatim}

The equivelent operation using the \module{parser} module is somewhat
longer, and allows the intermediate internal parse tree to be retained
as an AST object:

\begin{verbatim}
>>> import parser
>>> ast = parser.expr('a + 5')
>>> code = parser.compileast(ast)
>>> a = 5
>>> eval(code)
10
\end{verbatim}

An application which needs both AST and code objects can package this
code into readily available functions:

\begin{verbatim}
import parser

def load_suite(source_string):
    ast = parser.suite(source_string)
    code = parser.compileast(ast)
    return ast, code

def load_expression(source_string):
    ast = parser.expr(source_string)
    code = parser.compileast(ast)
    return ast, code
\end{verbatim}

\subsubsection{Information Discovery}

Some applications benefit from direct access to the parse tree.  The
remainder of this section demonstrates how the parse tree provides
access to module documentation defined in docstrings without requiring
that the code being examined be loaded into a running interpreter via
\keyword{import}.  This can be very useful for performing analyses of
untrusted code.

Generally, the example will demonstrate how the parse tree may be
traversed to distill interesting information.  Two functions and a set
of classes are developed which provide programmatic access to high
level function and class definitions provided by a module.  The
classes extract information from the parse tree and provide access to
the information at a useful semantic level, one function provides a
simple low-level pattern matching capability, and the other function
defines a high-level interface to the classes by handling file
operations on behalf of the caller.  All source files mentioned here
which are not part of the Python installation are located in the
\file{Demo/parser/} directory of the distribution.

The dynamic nature of Python allows the programmer a great deal of
flexibility, but most modules need only a limited measure of this when
defining classes, functions, and methods.  In this example, the only
definitions that will be considered are those which are defined in the
top level of their context, e.g., a function defined by a \keyword{def}
statement at column zero of a module, but not a function defined
within a branch of an \code{if} ... \code{else} construct, though
there are some good reasons for doing so in some situations.  Nesting
of definitions will be handled by the code developed in the example.

To construct the upper-level extraction methods, we need to know what
the parse tree structure looks like and how much of it we actually
need to be concerned about.  Python uses a moderately deep parse tree
so there are a large number of intermediate nodes.  It is important to
read and understand the formal grammar used by Python.  This is
specified in the file \file{Grammar/Grammar} in the distribution.
Consider the simplest case of interest when searching for docstrings:
a module consisting of a docstring and nothing else.  (See file
\file{docstring.py}.)

\begin{verbatim}
"""Some documentation.
"""
\end{verbatim}

Using the interpreter to take a look at the parse tree, we find a
bewildering mass of numbers and parentheses, with the documentation
buried deep in nested tuples.

\begin{verbatim}
>>> import parser
>>> import pprint
>>> ast = parser.suite(open('docstring.py').read())
>>> tup = parser.ast2tuple(ast)
>>> pprint.pprint(tup)
(257,
 (264,
  (265,
   (266,
    (267,
     (307,
      (287,
       (288,
        (289,
         (290,
          (292,
           (293,
            (294,
             (295,
              (296,
               (297,
                (298,
                 (299,
                  (300, (3, '"""Some documentation.\012"""'))))))))))))))))),
   (4, ''))),
 (4, ''),
 (0, ''))
\end{verbatim}

The numbers at the first element of each node in the tree are the node
types; they map directly to terminal and non-terminal symbols in the
grammar.  Unfortunately, they are represented as integers in the
internal representation, and the Python structures generated do not
change that.  However, the \module{symbol} and \module{token} modules
provide symbolic names for the node types and dictionaries which map
from the integers to the symbolic names for the node types.

In the output presented above, the outermost tuple contains four
elements: the integer \code{257} and three additional tuples.  Node
type \code{257} has the symbolic name \constant{file_input}.  Each of
these inner tuples contains an integer as the first element; these
integers, \code{264}, \code{4}, and \code{0}, represent the node types
\constant{stmt}, \constant{NEWLINE}, and \constant{ENDMARKER},
respectively.
Note that these values may change depending on the version of Python
you are using; consult \file{symbol.py} and \file{token.py} for
details of the mapping.  It should be fairly clear that the outermost
node is related primarily to the input source rather than the contents
of the file, and may be disregarded for the moment.  The \constant{stmt}
node is much more interesting.  In particular, all docstrings are
found in subtrees which are formed exactly as this node is formed,
with the only difference being the string itself.  The association
between the docstring in a similar tree and the defined entity (class,
function, or module) which it describes is given by the position of
the docstring subtree within the tree defining the described
structure.

By replacing the actual docstring with something to signify a variable
component of the tree, we allow a simple pattern matching approach to
check any given subtree for equivelence to the general pattern for
docstrings.  Since the example demonstrates information extraction, we
can safely require that the tree be in tuple form rather than list
form, allowing a simple variable representation to be
\code{['variable_name']}.  A simple recursive function can implement
the pattern matching, returning a boolean and a dictionary of variable
name to value mappings.  (See file \file{example.py}.)

\begin{verbatim}
from types import ListType, TupleType

def match(pattern, data, vars=None):
    if vars is None:
        vars = {}
    if type(pattern) is ListType:
        vars[pattern[0]] = data
        return 1, vars
    if type(pattern) is not TupleType:
        return (pattern == data), vars
    if len(data) != len(pattern):
        return 0, vars
    for pattern, data in map(None, pattern, data):
        same, vars = match(pattern, data, vars)
        if not same:
            break
    return same, vars
\end{verbatim}

Using this simple representation for syntactic variables and the symbolic
node types, the pattern for the candidate docstring subtrees becomes
fairly readable.  (See file \file{example.py}.)

\begin{verbatim}
import symbol
import token

DOCSTRING_STMT_PATTERN = (
    symbol.stmt,
    (symbol.simple_stmt,
     (symbol.small_stmt,
      (symbol.expr_stmt,
       (symbol.testlist,
        (symbol.test,
         (symbol.and_test,
          (symbol.not_test,
           (symbol.comparison,
            (symbol.expr,
             (symbol.xor_expr,
              (symbol.and_expr,
               (symbol.shift_expr,
                (symbol.arith_expr,
                 (symbol.term,
                  (symbol.factor,
                   (symbol.power,
                    (symbol.atom,
                     (token.STRING, ['docstring'])
                     )))))))))))))))),
     (token.NEWLINE, '')
     ))
\end{verbatim}

Using the \function{match()} function with this pattern, extracting the
module docstring from the parse tree created previously is easy:

\begin{verbatim}
>>> found, vars = match(DOCSTRING_STMT_PATTERN, tup[1])
>>> found
1
>>> vars
{'docstring': '"""Some documentation.\012"""'}
\end{verbatim}

Once specific data can be extracted from a location where it is
expected, the question of where information can be expected
needs to be answered.  When dealing with docstrings, the answer is
fairly simple: the docstring is the first \constant{stmt} node in a code
block (\constant{file_input} or \constant{suite} node types).  A module
consists of a single \constant{file_input} node, and class and function
definitions each contain exactly one \constant{suite} node.  Classes and
functions are readily identified as subtrees of code block nodes which
start with \code{(stmt, (compound_stmt, (classdef, ...} or
\code{(stmt, (compound_stmt, (funcdef, ...}.  Note that these subtrees
cannot be matched by \function{match()} since it does not support multiple
sibling nodes to match without regard to number.  A more elaborate
matching function could be used to overcome this limitation, but this
is sufficient for the example.

Given the ability to determine whether a statement might be a
docstring and extract the actual string from the statement, some work
needs to be performed to walk the parse tree for an entire module and
extract information about the names defined in each context of the
module and associate any docstrings with the names.  The code to
perform this work is not complicated, but bears some explanation.

The public interface to the classes is straightforward and should
probably be somewhat more flexible.  Each ``major'' block of the
module is described by an object providing several methods for inquiry
and a constructor which accepts at least the subtree of the complete
parse tree which it represents.  The \class{ModuleInfo} constructor
accepts an optional \var{name} parameter since it cannot
otherwise determine the name of the module.

The public classes include \class{ClassInfo}, \class{FunctionInfo},
and \class{ModuleInfo}.  All objects provide the
methods \method{get_name()}, \method{get_docstring()},
\method{get_class_names()}, and \method{get_class_info()}.  The
\class{ClassInfo} objects support \method{get_method_names()} and
\method{get_method_info()} while the other classes provide
\method{get_function_names()} and \method{get_function_info()}.

Within each of the forms of code block that the public classes
represent, most of the required information is in the same form and is
accessed in the same way, with classes having the distinction that
functions defined at the top level are referred to as ``methods.''
Since the difference in nomenclature reflects a real semantic
distinction from functions defined outside of a class, the
implementation needs to maintain the distinction.
Hence, most of the functionality of the public classes can be
implemented in a common base class, \class{SuiteInfoBase}, with the
accessors for function and method information provided elsewhere.
Note that there is only one class which represents function and method
information; this parallels the use of the \keyword{def} statement to
define both types of elements.

Most of the accessor functions are declared in \class{SuiteInfoBase}
and do not need to be overriden by subclasses.  More importantly, the
extraction of most information from a parse tree is handled through a
method called by the \class{SuiteInfoBase} constructor.  The example
code for most of the classes is clear when read alongside the formal
grammar, but the method which recursively creates new information
objects requires further examination.  Here is the relevant part of
the \class{SuiteInfoBase} definition from \file{example.py}:

\begin{verbatim}
class SuiteInfoBase:
    _docstring = ''
    _name = ''

    def __init__(self, tree = None):
        self._class_info = {}
        self._function_info = {}
        if tree:
            self._extract_info(tree)

    def _extract_info(self, tree):
        # extract docstring
        if len(tree) == 2:
            found, vars = match(DOCSTRING_STMT_PATTERN[1], tree[1])
        else:
            found, vars = match(DOCSTRING_STMT_PATTERN, tree[3])
        if found:
            self._docstring = eval(vars['docstring'])
        # discover inner definitions
        for node in tree[1:]:
            found, vars = match(COMPOUND_STMT_PATTERN, node)
            if found:
                cstmt = vars['compound']
                if cstmt[0] == symbol.funcdef:
                    name = cstmt[2][1]
                    self._function_info[name] = FunctionInfo(cstmt)
                elif cstmt[0] == symbol.classdef:
                    name = cstmt[2][1]
                    self._class_info[name] = ClassInfo(cstmt)
\end{verbatim}

After initializing some internal state, the constructor calls the
\method{_extract_info()} method.  This method performs the bulk of the
information extraction which takes place in the entire example.  The
extraction has two distinct phases: the location of the docstring for
the parse tree passed in, and the discovery of additional definitions
within the code block represented by the parse tree.

The initial \keyword{if} test determines whether the nested suite is of
the ``short form'' or the ``long form.''  The short form is used when
the code block is on the same line as the definition of the code
block, as in

\begin{verbatim}
def square(x): "Square an argument."; return x ** 2
\end{verbatim}

while the long form uses an indented block and allows nested
definitions:

\begin{verbatim}
def make_power(exp):
    "Make a function that raises an argument to the exponent `exp'."
    def raiser(x, y=exp):
        return x ** y
    return raiser
\end{verbatim}

When the short form is used, the code block may contain a docstring as
the first, and possibly only, \constant{small_stmt} element.  The
extraction of such a docstring is slightly different and requires only
a portion of the complete pattern used in the more common case.  As
implemented, the docstring will only be found if there is only
one \constant{small_stmt} node in the \constant{simple_stmt} node.
Since most functions and methods which use the short form do not
provide a docstring, this may be considered sufficient.  The
extraction of the docstring proceeds using the \function{match()} function
as described above, and the value of the docstring is stored as an
attribute of the \class{SuiteInfoBase} object.

After docstring extraction, a simple definition discovery
algorithm operates on the \constant{stmt} nodes of the
\constant{suite} node.  The special case of the short form is not
tested; since there are no \constant{stmt} nodes in the short form,
the algorithm will silently skip the single \constant{simple_stmt}
node and correctly not discover any nested definitions.

Each statement in the code block is categorized as
a class definition, function or method definition, or
something else.  For the definition statements, the name of the
element defined is extracted and a representation object
appropriate to the definition is created with the defining subtree
passed as an argument to the constructor.  The repesentation objects
are stored in instance variables and may be retrieved by name using
the appropriate accessor methods.

The public classes provide any accessors required which are more
specific than those provided by the \class{SuiteInfoBase} class, but
the real extraction algorithm remains common to all forms of code
blocks.  A high-level function can be used to extract the complete set
of information from a source file.  (See file \file{example.py}.)

\begin{verbatim}
def get_docs(fileName):
    source = open(fileName).read()
    import os
    basename = os.path.basename(os.path.splitext(fileName)[0])
    import parser
    ast = parser.suite(source)
    tup = parser.ast2tuple(ast)
    return ModuleInfo(tup, basename)
\end{verbatim}

This provides an easy-to-use interface to the documentation of a
module.  If information is required which is not extracted by the code
of this example, the code may be extended at clearly defined points to
provide additional capabilities.

\begin{seealso}

\seemodule{symbol}{useful constants representing internal nodes of the
parse tree}

\seemodule{token}{useful constants representing leaf nodes of the
parse tree and functions for testing node values}

\end{seealso}

\section{\module{symbol} ---
         Constants used with Python parse trees}

\declaremodule{standard}{symbol}
\modulesynopsis{Constants representing internal nodes of the parse tree.}
\sectionauthor{Fred L. Drake, Jr.}{fdrake@acm.org}


This module provides constants which represent the numeric values of
internal nodes of the parse tree.  Unlike most Python constants, these
use lower-case names.  Refer to the file \file{Grammar/Grammar} in the
Python distribution for the definitions of the names in the context of
the language grammar.  The specific numeric values which the names map
to may change between Python versions.

This module also provides one additional data object:


\begin{datadesc}{sym_name}
  Dictionary mapping the numeric values of the constants defined in
  this module back to name strings, allowing more human-readable
  representation of parse trees to be generated.
\end{datadesc}


\begin{seealso}
  \seemodule{parser}{The second example for the \refmodule{parser}
                     module shows how to use the \module{symbol}
                     module.}
\end{seealso}

\section{\module{token} ---
         Constants used with Python parse trees}

\declaremodule{standard}{token}
\modulesynopsis{Constants representing terminal nodes of the parse tree.}
\sectionauthor{Fred L. Drake, Jr.}{fdrake@acm.org}


This module provides constants which represent the numeric values of
leaf nodes of the parse tree (terminal tokens).  Refer to the file
\file{Grammar/Grammar} in the Python distribution for the definitions
of the names in the context of the language grammar.  The specific
numeric values which the names map to may change between Python
versions.

This module also provides one data object and some functions.  The
functions mirror definitions in the Python C header files.



\begin{datadesc}{tok_name}
Dictionary mapping the numeric values of the constants defined in this
module back to name strings, allowing more human-readable
representation of parse trees to be generated.
\end{datadesc}

\begin{funcdesc}{ISTERMINAL}{x}
Return true for terminal token values.
\end{funcdesc}

\begin{funcdesc}{ISNONTERMINAL}{x}
Return true for non-terminal token values.
\end{funcdesc}

\begin{funcdesc}{ISEOF}{x}
Return true if \var{x} is the marker indicating the end of input.
\end{funcdesc}


\begin{seealso}
  \seemodule{parser}{The second example for the \refmodule{parser}
                     module shows how to use the \module{symbol}
                     module.}
\end{seealso}

\section{Standard Module \module{keyword}}
\label{module-keyword}
\stmodindex{keyword}

This module allows a Python program to determine if a string is a
keyword.  A single function is provided:

\begin{funcdesc}{iskeyword}{s}
Return true if \var{s} is a Python keyword.
\end{funcdesc}

\section{\module{tokenize} ---
         Tokenizer for Python source}

\declaremodule{standard}{tokenize}
\modulesynopsis{Lexical scanner for Python source code.}
\moduleauthor{Ka Ping Yee}{}
\sectionauthor{Fred L. Drake, Jr.}{fdrake@acm.org}


The \module{tokenize} module provides a lexical scanner for Python
source code, implemented in Python.  The scanner in this module
returns comments as tokens as well, making it useful for implementing
``pretty-printers,'' including colorizers for on-screen displays.

The scanner is exposed by a single function:


\begin{funcdesc}{tokenize}{readline\optional{, tokeneater}}
  The \function{tokenize()} function accepts two parameters: one
  representing the input stream, and one providing an output mechanism 
  for \function{tokenize()}.

  The first parameter, \var{readline}, must be a callable object which
  provides the same interface as the \method{readline()} method of
  built-in file objects (see section~\ref{bltin-file-objects}).  Each
  call to the function should return one line of input as a string.

  The second parameter, \var{tokeneater}, must also be a callable
  object.  It is called with five parameters: the token type, the
  token string, a tuple \code{(\var{srow}, \var{scol})} specifying the 
  row and column where the token begins in the source, a tuple
  \code{(\var{erow}, \var{ecol})} giving the ending position of the
  token, and the line on which the token was found.  The line passed
  is the \emph{logical} line; continuation lines are included.
\end{funcdesc}


All constants from the \refmodule{token} module are also exported from 
\module{tokenize}, as is one additional token type value that might be 
passed to the \var{tokeneater} function by \function{tokenize()}:

\begin{datadesc}{COMMENT}
  Token value used to indicate a comment.
\end{datadesc}

\section{\module{pyclbr} ---
         Python class browser support}

\declaremodule{standard}{pyclbr}
\modulesynopsis{Supports information extraction for a Python class
                browser.}
\sectionauthor{Fred L. Drake, Jr.}{fdrake@acm.org}


The \module{pyclbr} can be used to determine some limited information
about the classes, methods and top-level functions
defined in a module.  The information
provided is sufficient to implement a traditional three-pane class
browser.  The information is extracted from the source code rather
than by importing the module, so this module is safe to use with
untrusted source code.  This restriction makes it impossible to use
this module with modules not implemented in Python, including many
standard and optional extension modules.


\begin{funcdesc}{readmodule}{module\optional{, path}}
  % The 'inpackage' parameter appears to be for internal use only....
  Read a module and return a dictionary mapping class names to class
  descriptor objects.  The parameter \var{module} should be the name
  of a module as a string; it may be the name of a module within a
  package.  The \var{path} parameter should be a sequence, and is used
  to augment the value of \code{sys.path}, which is used to locate
  module source code.
\end{funcdesc}

\begin{funcdesc}{readmodule_ex}{module\optional{, path}}
  % The 'inpackage' parameter appears to be for internal use only....
  Like \function{readmodule()}, but the returned dictionary, in addition
  to mapping class names to class descriptor objects, also maps
  top-level function names to function descriptor objects.  Moreover, if
  the module being read is a package, the key \code{'__path__'} in the
  returned dictionary has as its value a list which contains the package
  search path.
\end{funcdesc}


\subsection{Class Descriptor Objects \label{pyclbr-class-objects}}

The class descriptor objects used as values in the dictionary returned
by \function{readmodule()} and \function{readmodule_ex()}
provide the following data members:


\begin{memberdesc}[class descriptor]{module}
  The name of the module defining the class described by the class
  descriptor.
\end{memberdesc}

\begin{memberdesc}[class descriptor]{name}
  The name of the class.
\end{memberdesc}

\begin{memberdesc}[class descriptor]{super}
  A list of class descriptors which describe the immediate base
  classes of the class being described.  Classes which are named as
  superclasses but which are not discoverable by
  \function{readmodule()} are listed as a string with the class name
  instead of class descriptors.
\end{memberdesc}

\begin{memberdesc}[class descriptor]{methods}
  A dictionary mapping method names to line numbers.
\end{memberdesc}

\begin{memberdesc}[class descriptor]{file}
  Name of the file containing the \code{class} statement defining the class.
\end{memberdesc}

\begin{memberdesc}[class descriptor]{lineno}
  The line number of the \code{class} statement within the file named by
  \member{file}.
\end{memberdesc}

\subsection{Function Descriptor Objects \label{pyclbr-function-objects}}

The function descriptor objects used as values in the dictionary returned
by \function{readmodule_ex()} provide the following data members:


\begin{memberdesc}[function descriptor]{module}
  The name of the module defining the function described by the function
  descriptor.
\end{memberdesc}

\begin{memberdesc}[function descriptor]{name}
  The name of the function.
\end{memberdesc}

\begin{memberdesc}[function descriptor]{file}
  Name of the file containing the \code{def} statement defining the function.
\end{memberdesc}

\begin{memberdesc}[function descriptor]{lineno}
  The line number of the \code{def} statement within the file named by
  \member{file}.
\end{memberdesc}

\section{Standard Module \sectcode{code}}
\label{module-code}
\stmodindex{code}

The \code{code} module defines operations pertaining to Python code
objects.

The \code{code} module defines the following functions:

\renewcommand{\indexsubitem}{(in module code)}

\begin{funcdesc}{compile_command}{source\,
\optional{filename\optional{\, symbol}}}
This function is useful for programs that want to emulate Python's
interpreter main loop (a.k.a. the read-eval-print loop).  The tricky
part is to determine when the user has entered an incomplete command
that can be completed by entering more text (as opposed to a complete
command or a syntax error).  This function \emph{almost} always makes
the same decision as the real interpreter main loop.

Arguments: \var{source} is the source string; \var{filename} is the
optional filename from which source was read, defaulting to
\code{"<input>"}; and \var{symbol} is the optional grammar start
symbol, which should be either \code{"single"} (the default) or
\code{"eval"}.

Return a code object (the same as \code{compile(\var{source},
\var{filename}, \var{symbol})}) if the command is complete and valid;
return \code{None} if the command is incomplete; raise
\code{SyntaxError} if the command is a syntax error.


\end{funcdesc}

% LaTeXed from excellent doc-string.
\section{\module{codeop} ---
         Compile Python code}

\declaremodule{standard}{codeop}
\sectionauthor{Moshe Zadka}{mzadka@geocities.com}
\modulesynopsis{Compile (possibly incomplete) Python code.}

The \module{codeop} module provides a function to compile Python code
with hints on whether it certainly complete, possible complete or
definitely incomplete.  This is used by the \refmodule{code} module
and should not normally be used directly.

The \module{codeop} module defines the following function:

\begin{funcdesc}{compile_command}
                {source\optional{, filename\optional{, symbol}}}

Try to compile \var{source}, which should be a string of Python
code. Return a code object if \var{source} is valid
Python code. In that case, the filename attribute of the code object
will be \var{filename}, which defaults to \code{'<input>'}.

Return \code{None} if \var{source} is \emph{not} valid Python
code, but is a prefix of valid Python code.

Raise an exception if there is a problem with \var{source}:
\begin{itemize}
        \item \exception{SyntaxError}
              if there is invalid Python syntax.
        \item \exception{OverflowError}
              if there is an invalid numeric constant.
\end{itemize}

The \var{symbol} argument means whether to compile it as a statement
(\code{'single'}, the default) or as an expression (\code{'eval'}).

\strong{Caveat:}
It is possible (but not likely) that the parser stops parsing
with a successful outcome before reaching the end of the source;
in this case, trailing symbols may be ignored instead of causing an
error.  For example, a backslash followed by two newlines may be
followed by arbitrary garbage.  This will be fixed once the API
for the parser is better.
\end{funcdesc}

\section{\module{pprint} ---
         Data pretty printer}

\declaremodule{standard}{pprint}
\modulesynopsis{Data pretty printer.}
\moduleauthor{Fred L. Drake, Jr.}{fdrake@acm.org}
\sectionauthor{Fred L. Drake, Jr.}{fdrake@acm.org}


The \module{pprint} module provides a capability to ``pretty-print''
arbitrary Python data structures in a form which can be used as input
to the interpreter.  If the formatted structures include objects which
are not fundamental Python types, the representation may not be
loadable.  This may be the case if objects such as files, sockets,
classes, or instances are included, as well as many other builtin
objects which are not representable as Python constants.

The formatted representation keeps objects on a single line if it can,
and breaks them onto multiple lines if they don't fit within the
allowed width.  Construct \class{PrettyPrinter} objects explicitly if
you need to adjust the width constraint.

The \module{pprint} module defines one class:


% First the implementation class:

\begin{classdesc}{PrettyPrinter}{...}
Construct a \class{PrettyPrinter} instance.  This constructor
understands several keyword parameters.  An output stream may be set
using the \var{stream} keyword; the only method used on the stream
object is the file protocol's \method{write()} method.  If not
specified, the \class{PrettyPrinter} adopts \code{sys.stdout}.  Three
additional parameters may be used to control the formatted
representation.  The keywords are \var{indent}, \var{depth}, and
\var{width}.  The amount of indentation added for each recursive level
is specified by \var{indent}; the default is one.  Other values can
cause output to look a little odd, but can make nesting easier to
spot.  The number of levels which may be printed is controlled by
\var{depth}; if the data structure being printed is too deep, the next
contained level is replaced by \samp{...}.  By default, there is no
constraint on the depth of the objects being formatted.  The desired
output width is constrained using the \var{width} parameter; the
default is eighty characters.  If a structure cannot be formatted
within the constrained width, a best effort will be made.

\begin{verbatim}
>>> import pprint, sys
>>> stuff = sys.path[:]
>>> stuff.insert(0, stuff[:])
>>> pp = pprint.PrettyPrinter(indent=4)
>>> pp.pprint(stuff)
[   [   '',
        '/usr/local/lib/python1.5',
        '/usr/local/lib/python1.5/test',
        '/usr/local/lib/python1.5/sunos5',
        '/usr/local/lib/python1.5/sharedmodules',
        '/usr/local/lib/python1.5/tkinter'],
    '',
    '/usr/local/lib/python1.5',
    '/usr/local/lib/python1.5/test',
    '/usr/local/lib/python1.5/sunos5',
    '/usr/local/lib/python1.5/sharedmodules',
    '/usr/local/lib/python1.5/tkinter']
>>>
>>> import parser
>>> tup = parser.ast2tuple(
...     parser.suite(open('pprint.py').read()))[1][1][1]
>>> pp = pprint.PrettyPrinter(depth=6)
>>> pp.pprint(tup)
(266, (267, (307, (287, (288, (...))))))
\end{verbatim}
\end{classdesc}


% Now the derivative functions:

The \class{PrettyPrinter} class supports several derivative functions:

\begin{funcdesc}{pformat}{object\optional{, indent\optional{,
width\optional{, depth}}}}
Return the formatted representation of \var{object} as a string.  \var{indent},
\var{width} and \var{depth} will be passed to the \class{PrettyPrinter}
constructor as formatting parameters.
\versionchanged[The parameters \var{indent}, \var{width} and \var{depth}
were added]{2.4}
\end{funcdesc}

\begin{funcdesc}{pprint}{object\optional{, stream\optional{,
indent\optional{, width\optional{, depth}}}}}
Prints the formatted representation of \var{object} on \var{stream},
followed by a newline.  If \var{stream} is omitted, \code{sys.stdout}
is used.  This may be used in the interactive interpreter instead of a
\keyword{print} statement for inspecting values.    \var{indent},
\var{width} and \var{depth} will be passed to the \class{PrettyPrinter}
constructor as formatting parameters.

\begin{verbatim}
>>> stuff = sys.path[:]
>>> stuff.insert(0, stuff)
>>> pprint.pprint(stuff)
[<Recursion on list with id=869440>,
 '',
 '/usr/local/lib/python1.5',
 '/usr/local/lib/python1.5/test',
 '/usr/local/lib/python1.5/sunos5',
 '/usr/local/lib/python1.5/sharedmodules',
 '/usr/local/lib/python1.5/tkinter']
\end{verbatim}
\versionchanged[The parameters \var{indent}, \var{width} and \var{depth}
were added]{2.4}
\end{funcdesc}

\begin{funcdesc}{isreadable}{object}
Determine if the formatted representation of \var{object} is
``readable,'' or can be used to reconstruct the value using
\function{eval()}\bifuncindex{eval}.  This always returns false for
recursive objects.

\begin{verbatim}
>>> pprint.isreadable(stuff)
False
\end{verbatim}
\end{funcdesc}

\begin{funcdesc}{isrecursive}{object}
Determine if \var{object} requires a recursive representation.
\end{funcdesc}


One more support function is also defined:

\begin{funcdesc}{saferepr}{object}
Return a string representation of \var{object}, protected against
recursive data structures.  If the representation of \var{object}
exposes a recursive entry, the recursive reference will be represented
as \samp{<Recursion on \var{typename} with id=\var{number}>}.  The
representation is not otherwise formatted.
\end{funcdesc}

% This example is outside the {funcdesc} to keep it from running over
% the right margin.
\begin{verbatim}
>>> pprint.saferepr(stuff)
"[<Recursion on list with id=682968>, '', '/usr/local/lib/python1.5', '/usr/loca
l/lib/python1.5/test', '/usr/local/lib/python1.5/sunos5', '/usr/local/lib/python
1.5/sharedmodules', '/usr/local/lib/python1.5/tkinter']"
\end{verbatim}


\subsection{PrettyPrinter Objects}
\label{PrettyPrinter Objects}

\class{PrettyPrinter} instances have the following methods:


\begin{methoddesc}{pformat}{object}
Return the formatted representation of \var{object}.  This takes into
account the options passed to the \class{PrettyPrinter} constructor.
\end{methoddesc}

\begin{methoddesc}{pprint}{object}
Print the formatted representation of \var{object} on the configured
stream, followed by a newline.
\end{methoddesc}

The following methods provide the implementations for the
corresponding functions of the same names.  Using these methods on an
instance is slightly more efficient since new \class{PrettyPrinter}
objects don't need to be created.

\begin{methoddesc}{isreadable}{object}
Determine if the formatted representation of the object is
``readable,'' or can be used to reconstruct the value using
\function{eval()}\bifuncindex{eval}.  Note that this returns false for
recursive objects.  If the \var{depth} parameter of the
\class{PrettyPrinter} is set and the object is deeper than allowed,
this returns false.
\end{methoddesc}

\begin{methoddesc}{isrecursive}{object}
Determine if the object requires a recursive representation.
\end{methoddesc}

This method is provided as a hook to allow subclasses to modify the
way objects are converted to strings.  The default implementation uses
the internals of the \function{saferepr()} implementation.

\begin{methoddesc}{format}{object, context, maxlevels, level}
Returns three values: the formatted version of \var{object} as a
string, a flag indicating whether the result is readable, and a flag
indicating whether recursion was detected.  The first argument is the
object to be presented.  The second is a dictionary which contains the
\function{id()} of objects that are part of the current presentation
context (direct and indirect containers for \var{object} that are
affecting the presentation) as the keys; if an object needs to be
presented which is already represented in \var{context}, the third
return value should be true.  Recursive calls to the \method{format()}
method should add additional entries for containers to this
dictionary.  The third argument, \var{maxlevels}, gives the requested
limit to recursion; this will be \code{0} if there is no requested
limit.  This argument should be passed unmodified to recursive calls.
The fourth argument, \var{level}, gives the current level; recursive
calls should be passed a value less than that of the current call.
\versionadded{2.3}
\end{methoddesc}

\section{\module{repr} ---
         Alternate \function{repr()} implementation}

\sectionauthor{Fred L. Drake, Jr.}{fdrake@acm.org}
\declaremodule{standard}{repr}
\modulesynopsis{Alternate \function{repr()} implementation with size limits.}


The \module{repr} module provides a means for producing object
representations with limits on the size of the resulting strings.
This is used in the Python debugger and may be useful in other
contexts as well.

This module provides a class, an instance, and a function:


\begin{classdesc}{Repr}{}
  Class which provides formatting services useful in implementing
  functions similar to the built-in \function{repr()}; size limits for 
  different object types are added to avoid the generation of
  representations which are excessively long.
\end{classdesc}


\begin{datadesc}{aRepr}
  This is an instance of \class{Repr} which is used to provide the
  \function{repr()} function described below.  Changing the attributes
  of this object will affect the size limits used by \function{repr()}
  and the Python debugger.
\end{datadesc}


\begin{funcdesc}{repr}{obj}
  This is the \method{repr()} method of \code{aRepr}.  It returns a
  string similar to that returned by the built-in function of the same 
  name, but with limits on most sizes.
\end{funcdesc}


\subsection{Repr Objects \label{Repr-objects}}

\class{Repr} instances provide several members which can be used to
provide size limits for the representations of different object types, 
and methods which format specific object types.


\begin{memberdesc}{maxlevel}
  Depth limit on the creation of recursive representations.  The
  default is \code{6}.
\end{memberdesc}

\begin{memberdesc}{maxdict}
\memberline{maxlist}
\memberline{maxtuple}
  Limits on the number of entries represented for the named object
  type.  The default for \member{maxdict} is \code{4}, for the others, 
  \code{6}.
\end{memberdesc}

\begin{memberdesc}{maxlong}
  Maximum number of characters in the representation for a long
  integer.  Digits are dropped from the middle.  The default is
  \code{40}.
\end{memberdesc}

\begin{memberdesc}{maxstring}
  Limit on the number of characters in the representation of the
  string.  Note that the ``normal'' representation of the string is
  used as the character source: if escape sequences are needed in the
  representation, these may be mangled when the representation is
  shortened.  The default is \code{30}.
\end{memberdesc}

\begin{memberdesc}{maxother}
  This limit is used to control the size of object types for which no
  specific formatting method is available on the \class{Repr} object.
  It is applied in a similar manner as \member{maxstring}.  The
  default is \code{20}.
\end{memberdesc}

\begin{methoddesc}{repr}{obj}
  The equivalent to the built-in \function{repr()} that uses the
  formatting imposed by the instance.
\end{methoddesc}

\begin{methoddesc}{repr1}{obj, level}
  Recursive implementation used by \method{repr()}.  This uses the
  type of \var{obj} to determine which formatting method to call,
  passing it \var{obj} and \var{level}.  The type-specific methods
  should call \method{repr1()} to perform recursive formatting, with
  \code{\var{level} - 1} for the value of \var{level} in the recursive 
  call.
\end{methoddesc}

\begin{methoddescni}{repr_\var{type}}{obj, level}
  Formatting methods for specific types are implemented as methods
  with a name based on the type name.  In the method name, \var{type}
  is replaced by
  \code{string.join(string.split(type(\var{obj}).__name__, '_'))}.
  Dispatch to these methods is handled by \method{repr1()}.
  Type-specific methods which need to recursively format a value
  should call \samp{self.repr1(\var{subobj}, \var{level} - 1)}.
\end{methoddescni}


\subsection{Subclassing Repr Objects \label{subclassing-reprs}}

The use of dynamic dispatching by \method{Repr.repr1()} allows
subclasses of \class{Repr} to add support for additional built-in
object types or to modify the handling of types already supported.
This example shows how special support for file objects could be
added:

\begin{verbatim}
import repr
import sys

class MyRepr(repr.Repr):
    def repr_file(self, obj, level):
        if obj.name in ['<stdin>', '<stdout>', '<stderr>']:
            return obj.name
        else:
            return `obj`

aRepr = MyRepr()
print aRepr.repr(sys.stdin)          # prints '<stdin>'
\end{verbatim}

\section{\module{py_compile} ---
         Compile Python source files}

% Documentation based on module docstrings, by Fred L. Drake, Jr.
% <fdrake@acm.org>

\declaremodule[pycompile]{standard}{py_compile}

\modulesynopsis{Compile Python source files to byte-code files.}


\indexii{file}{byte-code}
The \module{py_compile} module provides a single function to generate
a byte-code file from a source file.

Though not often needed, this function can be useful when installing
modules for shared use, especially if some of the users may not have
permission to write the byte-code cache files in the directory
containing the source code.


\begin{funcdesc}{compile}{file\optional{, cfile\optional{, dfile}}}
  Compile a source file to byte-code and write out the byte-code cache 
  file.  The source code is loaded from the file name \var{file}.  The 
  byte-code is written to \var{cfile}, which defaults to \var{file}
  \code{+} \code{'c'} (\code{'o'} if optimization is enabled in the
  current interpreter).  If \var{dfile} is specified, it is used as
  the name of the source file in error messages instead of \var{file}. 
\end{funcdesc}


\begin{seealso}
  \seemodule{compileall}{Utilities to compile all Python source files
                         in a directory tree.}
\end{seealso}
		% really py_compile
\section{\module{compileall} ---
         Byte-compile Python libraries}

\declaremodule{standard}{compileall}
\modulesynopsis{Tools for byte-compiling all Python source files in a
                directory tree.}


This module provides some utility functions to support installing
Python libraries.  These functions compile Python source files in a
directory tree, allowing users without permission to write to the
libraries to take advantage of cached byte-code files.

The source file for this module may also be used as a script to
compile Python sources in directories named on the command line or in
\code{sys.path}.


\begin{funcdesc}{compile_dir}{dir\optional{, maxlevels\optional{,
                              ddir\optional{, force\optional{, 
                              rx\optional{, quiet}}}}}}
  Recursively descend the directory tree named by \var{dir}, compiling
  all \file{.py} files along the way.  The \var{maxlevels} parameter
  is used to limit the depth of the recursion; it defaults to
  \code{10}.  If \var{ddir} is given, it is used as the base path from 
  which the filenames used in error messages will be generated.  If
  \var{force} is true, modules are re-compiled even if the timestamps
  are up to date. 

  If \var{rx} is given, it specifies a regular expression of file
  names to exclude from the search; that expression is searched for in
  the full path.

  If \var{quiet} is true, nothing is printed to the standard output
  in normal operation.
\end{funcdesc}

\begin{funcdesc}{compile_path}{\optional{skip_curdir\optional{,
                               maxlevels\optional{, force}}}}
  Byte-compile all the \file{.py} files found along \code{sys.path}.
  If \var{skip_curdir} is true (the default), the current directory is
  not included in the search.  The \var{maxlevels} and
  \var{force} parameters default to \code{0} and are passed to the
  \function{compile_dir()} function.
\end{funcdesc}


\begin{seealso}
  \seemodule[pycompile]{py_compile}{Byte-compile a single source file.}
\end{seealso}

\section{\module{dis} ---
         Disassembler for Python byte code}

\declaremodule{standard}{dis}
\modulesynopsis{Disassembler for Python byte code.}


The \module{dis} module supports the analysis of Python byte code by
disassembling it.  Since there is no Python assembler, this module
defines the Python assembly language.  The Python byte code which
this module takes as an input is defined in the file 
\file{Include/opcode.h} and used by the compiler and the interpreter.

Example: Given the function \function{myfunc}:

\begin{verbatim}
def myfunc(alist):
    return len(alist)
\end{verbatim}

the following command can be used to get the disassembly of
\function{myfunc()}:

\begin{verbatim}
>>> dis.dis(myfunc)
          0 SET_LINENO          1

          3 SET_LINENO          2
          6 LOAD_GLOBAL         0 (len)
          9 LOAD_FAST           0 (alist)
         12 CALL_FUNCTION       1
         15 RETURN_VALUE   
         16 LOAD_CONST          0 (None)
         19 RETURN_VALUE   
\end{verbatim}

The \module{dis} module defines the following functions and constants:

\begin{funcdesc}{dis}{\optional{bytesource}}
Disassemble the \var{bytesource} object. \var{bytesource} can denote
either a class, a method, a function, or a code object.  For a class,
it disassembles all methods.  For a single code sequence, it prints
one line per byte code instruction.  If no object is provided, it
disassembles the last traceback.
\end{funcdesc}

\begin{funcdesc}{distb}{\optional{tb}}
Disassembles the top-of-stack function of a traceback, using the last
traceback if none was passed.  The instruction causing the exception
is indicated.
\end{funcdesc}

\begin{funcdesc}{disassemble}{code\optional{, lasti}}
Disassembles a code object, indicating the last instruction if \var{lasti}
was provided.  The output is divided in the following columns:

\begin{enumerate}
\item the current instruction, indicated as \samp{-->},
\item a labelled instruction, indicated with \samp{>>},
\item the address of the instruction,
\item the operation code name,
\item operation parameters, and
\item interpretation of the parameters in parentheses.
\end{enumerate}

The parameter interpretation recognizes local and global
variable names, constant values, branch targets, and compare
operators.
\end{funcdesc}

\begin{funcdesc}{disco}{code\optional{, lasti}}
A synonym for disassemble.  It is more convenient to type, and kept
for compatibility with earlier Python releases.
\end{funcdesc}

\begin{datadesc}{opname}
Sequence of operation names, indexable using the byte code.
\end{datadesc}

\begin{datadesc}{cmp_op}
Sequence of all compare operation names.
\end{datadesc}

\begin{datadesc}{hasconst}
Sequence of byte codes that have a constant parameter.
\end{datadesc}

\begin{datadesc}{hasname}
Sequence of byte codes that access an attribute by name.
\end{datadesc}

\begin{datadesc}{hasjrel}
Sequence of byte codes that have a relative jump target.
\end{datadesc}

\begin{datadesc}{hasjabs}
Sequence of byte codes that have an absolute jump target.
\end{datadesc}

\begin{datadesc}{haslocal}
Sequence of byte codes that access a local variable.
\end{datadesc}

\begin{datadesc}{hascompare}
Sequence of byte codes of boolean operations.
\end{datadesc}

\subsection{Python Byte Code Instructions}
\label{bytecodes}

The Python compiler currently generates the following byte code
instructions.

\setindexsubitem{(byte code insns)}

\begin{opcodedesc}{STOP_CODE}{}
Indicates end-of-code to the compiler, not used by the interpreter.
\end{opcodedesc}

\begin{opcodedesc}{POP_TOP}{}
Removes the top-of-stack (TOS) item.
\end{opcodedesc}

\begin{opcodedesc}{ROT_TWO}{}
Swaps the two top-most stack items.
\end{opcodedesc}

\begin{opcodedesc}{ROT_THREE}{}
Lifts second and third stack item one position up, moves top down
to position three.
\end{opcodedesc}

\begin{opcodedesc}{ROT_FOUR}{}
Lifts second, third and forth stack item one position up, moves top down to
position four.
\end{opcodedesc}

\begin{opcodedesc}{DUP_TOP}{}
Duplicates the reference on top of the stack.
\end{opcodedesc}

Unary Operations take the top of the stack, apply the operation, and
push the result back on the stack.

\begin{opcodedesc}{UNARY_POSITIVE}{}
Implements \code{TOS = +TOS}.
\end{opcodedesc}

\begin{opcodedesc}{UNARY_NEGATIVE}{}
Implements \code{TOS = -TOS}.
\end{opcodedesc}

\begin{opcodedesc}{UNARY_NOT}{}
Implements \code{TOS = not TOS}.
\end{opcodedesc}

\begin{opcodedesc}{UNARY_CONVERT}{}
Implements \code{TOS = `TOS`}.
\end{opcodedesc}

\begin{opcodedesc}{UNARY_INVERT}{}
Implements \code{TOS = \~{}TOS}.
\end{opcodedesc}

Binary operations remove the top of the stack (TOS) and the second top-most
stack item (TOS1) from the stack.  They perform the operation, and put the
result back on the stack.

\begin{opcodedesc}{BINARY_POWER}{}
Implements \code{TOS = TOS1 ** TOS}.
\end{opcodedesc}

\begin{opcodedesc}{BINARY_MULTIPLY}{}
Implements \code{TOS = TOS1 * TOS}.
\end{opcodedesc}

\begin{opcodedesc}{BINARY_DIVIDE}{}
Implements \code{TOS = TOS1 / TOS}.
\end{opcodedesc}

\begin{opcodedesc}{BINARY_MODULO}{}
Implements \code{TOS = TOS1 \%{} TOS}.
\end{opcodedesc}

\begin{opcodedesc}{BINARY_ADD}{}
Implements \code{TOS = TOS1 + TOS}.
\end{opcodedesc}

\begin{opcodedesc}{BINARY_SUBTRACT}{}
Implements \code{TOS = TOS1 - TOS}.
\end{opcodedesc}

\begin{opcodedesc}{BINARY_SUBSCR}{}
Implements \code{TOS = TOS1[TOS]}.
\end{opcodedesc}

\begin{opcodedesc}{BINARY_LSHIFT}{}
Implements \code{TOS = TOS1 <\code{}< TOS}.
\end{opcodedesc}

\begin{opcodedesc}{BINARY_RSHIFT}{}
Implements \code{TOS = TOS1 >\code{}> TOS}.
\end{opcodedesc}

\begin{opcodedesc}{BINARY_AND}{}
Implements \code{TOS = TOS1 \&\ TOS}.
\end{opcodedesc}

\begin{opcodedesc}{BINARY_XOR}{}
Implements \code{TOS = TOS1 \^\ TOS}.
\end{opcodedesc}

\begin{opcodedesc}{BINARY_OR}{}
Implements \code{TOS = TOS1 | TOS}.
\end{opcodedesc}

In-place operations are like binary operations, in that they remove TOS and
TOS1, and push the result back on the stack, but the operation is done
in-place when TOS1 supports it, and the resulting TOS may be (but does not
have to be) the original TOS1.

\begin{opcodedesc}{INPLACE_POWER}{}
Implements in-place \code{TOS = TOS1 ** TOS}.
\end{opcodedesc}

\begin{opcodedesc}{INPLACE_MULTIPLY}{}
Implements in-place \code{TOS = TOS1 * TOS}.
\end{opcodedesc}

\begin{opcodedesc}{INPLACE_DIVIDE}{}
Implements in-place \code{TOS = TOS1 / TOS}.
\end{opcodedesc}

\begin{opcodedesc}{INPLACE_MODULO}{}
Implements in-place \code{TOS = TOS1 \%{} TOS}.
\end{opcodedesc}

\begin{opcodedesc}{INPLACE_ADD}{}
Implements in-place \code{TOS = TOS1 + TOS}.
\end{opcodedesc}

\begin{opcodedesc}{INPLACE_SUBTRACT}{}
Implements in-place \code{TOS = TOS1 - TOS}.
\end{opcodedesc}

\begin{opcodedesc}{INPLACE_LSHIFT}{}
Implements in-place \code{TOS = TOS1 <\code{}< TOS}.
\end{opcodedesc}

\begin{opcodedesc}{INPLACE_RSHIFT}{}
Implements in-place \code{TOS = TOS1 >\code{}> TOS}.
\end{opcodedesc}

\begin{opcodedesc}{INPLACE_AND}{}
Implements in-place \code{TOS = TOS1 \&\ TOS}.
\end{opcodedesc}

\begin{opcodedesc}{INPLACE_XOR}{}
Implements in-place \code{TOS = TOS1 \^\ TOS}.
\end{opcodedesc}

\begin{opcodedesc}{INPLACE_OR}{}
Implements in-place \code{TOS = TOS1 | TOS}.
\end{opcodedesc}

The slice opcodes take up to three parameters.

\begin{opcodedesc}{SLICE+0}{}
Implements \code{TOS = TOS[:]}.
\end{opcodedesc}

\begin{opcodedesc}{SLICE+1}{}
Implements \code{TOS = TOS1[TOS:]}.
\end{opcodedesc}

\begin{opcodedesc}{SLICE+2}{}
Implements \code{TOS = TOS1[:TOS1]}.
\end{opcodedesc}

\begin{opcodedesc}{SLICE+3}{}
Implements \code{TOS = TOS2[TOS1:TOS]}.
\end{opcodedesc}

Slice assignment needs even an additional parameter.  As any statement,
they put nothing on the stack.

\begin{opcodedesc}{STORE_SLICE+0}{}
Implements \code{TOS[:] = TOS1}.
\end{opcodedesc}

\begin{opcodedesc}{STORE_SLICE+1}{}
Implements \code{TOS1[TOS:] = TOS2}.
\end{opcodedesc}

\begin{opcodedesc}{STORE_SLICE+2}{}
Implements \code{TOS1[:TOS] = TOS2}.
\end{opcodedesc}

\begin{opcodedesc}{STORE_SLICE+3}{}
Implements \code{TOS2[TOS1:TOS] = TOS3}.
\end{opcodedesc}

\begin{opcodedesc}{DELETE_SLICE+0}{}
Implements \code{del TOS[:]}.
\end{opcodedesc}

\begin{opcodedesc}{DELETE_SLICE+1}{}
Implements \code{del TOS1[TOS:]}.
\end{opcodedesc}

\begin{opcodedesc}{DELETE_SLICE+2}{}
Implements \code{del TOS1[:TOS]}.
\end{opcodedesc}

\begin{opcodedesc}{DELETE_SLICE+3}{}
Implements \code{del TOS2[TOS1:TOS]}.
\end{opcodedesc}

\begin{opcodedesc}{STORE_SUBSCR}{}
Implements \code{TOS1[TOS] = TOS2}.
\end{opcodedesc}

\begin{opcodedesc}{DELETE_SUBSCR}{}
Implements \code{del TOS1[TOS]}.
\end{opcodedesc}

\begin{opcodedesc}{PRINT_EXPR}{}
Implements the expression statement for the interactive mode.  TOS is
removed from the stack and printed.  In non-interactive mode, an
expression statement is terminated with \code{POP_STACK}.
\end{opcodedesc}

\begin{opcodedesc}{PRINT_ITEM}{}
Prints TOS to the file-like object bound to \code{sys.stdout}.  There
is one such instruction for each item in the \keyword{print} statement.
\end{opcodedesc}

\begin{opcodedesc}{PRINT_ITEM_TO}{}
Like \code{PRINT_ITEM}, but prints the item second from TOS to the
file-like object at TOS.  This is used by the extended print statement.
\end{opcodedesc}

\begin{opcodedesc}{PRINT_NEWLINE}{}
Prints a new line on \code{sys.stdout}.  This is generated as the
last operation of a \keyword{print} statement, unless the statement
ends with a comma.
\end{opcodedesc}

\begin{opcodedesc}{PRINT_NEWLINE_TO}{}
Like \code{PRINT_NEWLINE}, but prints the new line on the file-like
object on the TOS.  This is used by the extended print statement.
\end{opcodedesc}

\begin{opcodedesc}{BREAK_LOOP}{}
Terminates a loop due to a \keyword{break} statement.
\end{opcodedesc}

\begin{opcodedesc}{LOAD_LOCALS}{}
Pushes a reference to the locals of the current scope on the stack.
This is used in the code for a class definition: After the class body
is evaluated, the locals are passed to the class definition.
\end{opcodedesc}

\begin{opcodedesc}{RETURN_VALUE}{}
Returns with TOS to the caller of the function.
\end{opcodedesc}

\begin{opcodedesc}{IMPORT_STAR}{}
Loads all symbols not starting with \character{_} directly from the module TOS
to the local namespace. The module is popped after loading all names.
This opcode implements \code{from module import *}.
\end{opcodedesc}

\begin{opcodedesc}{EXEC_STMT}{}
Implements \code{exec TOS2,TOS1,TOS}.  The compiler fills
missing optional parameters with \code{None}.
\end{opcodedesc}

\begin{opcodedesc}{POP_BLOCK}{}
Removes one block from the block stack.  Per frame, there is a 
stack of blocks, denoting nested loops, try statements, and such.
\end{opcodedesc}

\begin{opcodedesc}{END_FINALLY}{}
Terminates a \keyword{finally} clause.  The interpreter recalls
whether the exception has to be re-raised, or whether the function
returns, and continues with the outer-next block.
\end{opcodedesc}

\begin{opcodedesc}{BUILD_CLASS}{}
Creates a new class object.  TOS is the methods dictionary, TOS1
the tuple of the names of the base classes, and TOS2 the class name.
\end{opcodedesc}

All of the following opcodes expect arguments.  An argument is two
bytes, with the more significant byte last.

\begin{opcodedesc}{STORE_NAME}{namei}
Implements \code{name = TOS}. \var{namei} is the index of \var{name}
in the attribute \member{co_names} of the code object.
The compiler tries to use \code{STORE_LOCAL} or \code{STORE_GLOBAL}
if possible.
\end{opcodedesc}

\begin{opcodedesc}{DELETE_NAME}{namei}
Implements \code{del name}, where \var{namei} is the index into
\member{co_names} attribute of the code object.
\end{opcodedesc}

\begin{opcodedesc}{UNPACK_SEQUENCE}{count}
Unpacks TOS into \var{count} individual values, which are put onto
the stack right-to-left.
\end{opcodedesc}

%\begin{opcodedesc}{UNPACK_LIST}{count}
%This opcode is obsolete.
%\end{opcodedesc}

%\begin{opcodedesc}{UNPACK_ARG}{count}
%This opcode is obsolete.
%\end{opcodedesc}

\begin{opcodedesc}{DUP_TOPX}{count}
Duplicate \var{count} items, keeping them in the same order. Due to
implementation limits, \var{count} should be between 1 and 5 inclusive.
\end{opcodedesc}

\begin{opcodedesc}{STORE_ATTR}{namei}
Implements \code{TOS.name = TOS1}, where \var{namei} is the index
of name in \member{co_names}.
\end{opcodedesc}

\begin{opcodedesc}{DELETE_ATTR}{namei}
Implements \code{del TOS.name}, using \var{namei} as index into
\member{co_names}.
\end{opcodedesc}

\begin{opcodedesc}{STORE_GLOBAL}{namei}
Works as \code{STORE_NAME}, but stores the name as a global.
\end{opcodedesc}

\begin{opcodedesc}{DELETE_GLOBAL}{namei}
Works as \code{DELETE_NAME}, but deletes a global name.
\end{opcodedesc}

%\begin{opcodedesc}{UNPACK_VARARG}{argc}
%This opcode is obsolete.
%\end{opcodedesc}

\begin{opcodedesc}{LOAD_CONST}{consti}
Pushes \samp{co_consts[\var{consti}]} onto the stack.
\end{opcodedesc}

\begin{opcodedesc}{LOAD_NAME}{namei}
Pushes the value associated with \samp{co_names[\var{namei}]} onto the stack.
\end{opcodedesc}

\begin{opcodedesc}{BUILD_TUPLE}{count}
Creates a tuple consuming \var{count} items from the stack, and pushes
the resulting tuple onto the stack.
\end{opcodedesc}

\begin{opcodedesc}{BUILD_LIST}{count}
Works as \code{BUILD_TUPLE}, but creates a list.
\end{opcodedesc}

\begin{opcodedesc}{BUILD_MAP}{zero}
Pushes a new empty dictionary object onto the stack.  The argument is
ignored and set to zero by the compiler.
\end{opcodedesc}

\begin{opcodedesc}{LOAD_ATTR}{namei}
Replaces TOS with \code{getattr(TOS, co_names[\var{namei}]}.
\end{opcodedesc}

\begin{opcodedesc}{COMPARE_OP}{opname}
Performs a boolean operation.  The operation name can be found
in \code{cmp_op[\var{opname}]}.
\end{opcodedesc}

\begin{opcodedesc}{IMPORT_NAME}{namei}
Imports the module \code{co_names[\var{namei}]}.  The module object is
pushed onto the stack.  The current namespace is not affected: for a
proper import statement, a subsequent \code{STORE_FAST} instruction
modifies the namespace.
\end{opcodedesc}

\begin{opcodedesc}{IMPORT_FROM}{namei}
Loads the attribute \code{co_names[\var{namei}]} from the module found in
TOS. The resulting object is pushed onto the stack, to be subsequently
stored by a \code{STORE_FAST} instruction.
\end{opcodedesc}

\begin{opcodedesc}{JUMP_FORWARD}{delta}
Increments byte code counter by \var{delta}.
\end{opcodedesc}

\begin{opcodedesc}{JUMP_IF_TRUE}{delta}
If TOS is true, increment the byte code counter by \var{delta}.  TOS is
left on the stack.
\end{opcodedesc}

\begin{opcodedesc}{JUMP_IF_FALSE}{delta}
If TOS is false, increment the byte code counter by \var{delta}.  TOS
is not changed. 
\end{opcodedesc}

\begin{opcodedesc}{JUMP_ABSOLUTE}{target}
Set byte code counter to \var{target}.
\end{opcodedesc}

\begin{opcodedesc}{FOR_LOOP}{delta}
Iterate over a sequence.  TOS is the current index, TOS1 the sequence.
First, the next element is computed.  If the sequence is exhausted,
increment byte code counter by \var{delta}.  Otherwise, push the
sequence, the incremented counter, and the current item onto the stack.
\end{opcodedesc}

%\begin{opcodedesc}{LOAD_LOCAL}{namei}
%This opcode is obsolete.
%\end{opcodedesc}

\begin{opcodedesc}{LOAD_GLOBAL}{namei}
Loads the global named \code{co_names[\var{namei}]} onto the stack.
\end{opcodedesc}

%\begin{opcodedesc}{SET_FUNC_ARGS}{argc}
%This opcode is obsolete.
%\end{opcodedesc}

\begin{opcodedesc}{SETUP_LOOP}{delta}
Pushes a block for a loop onto the block stack.  The block spans
from the current instruction with a size of \var{delta} bytes.
\end{opcodedesc}

\begin{opcodedesc}{SETUP_EXCEPT}{delta}
Pushes a try block from a try-except clause onto the block stack.
\var{delta} points to the first except block.
\end{opcodedesc}

\begin{opcodedesc}{SETUP_FINALLY}{delta}
Pushes a try block from a try-except clause onto the block stack.
\var{delta} points to the finally block.
\end{opcodedesc}

\begin{opcodedesc}{LOAD_FAST}{var_num}
Pushes a reference to the local \code{co_varnames[\var{var_num}]} onto
the stack.
\end{opcodedesc}

\begin{opcodedesc}{STORE_FAST}{var_num}
Stores TOS into the local \code{co_varnames[\var{var_num}]}.
\end{opcodedesc}

\begin{opcodedesc}{DELETE_FAST}{var_num}
Deletes local \code{co_varnames[\var{var_num}]}.
\end{opcodedesc}

\begin{opcodedesc}{LOAD_CLOSURE}{i}
Pushes a reference to the cell contained in slot \var{i} of the
cell and free variable storage.  The name of the variable is 
\code{co_cellvars[\var{i}]} if \var{i} is less than the length of
\var{co_cellvars}.  Otherwise it is 
\code{co_freevars[\var{i} - len(co_cellvars)]}.
\end{opcodedesc}

\begin{opcodedesc}{LOAD_DEREF}{i}
Loads the cell contained in slot \var{i} of the cell and free variable
storage.  Pushes a reference to the object the cell contains on the
stack. 
\end{opcodedesc}

\begin{opcodedesc}{STORE_DEREF}{i}
Stores TOS into the cell contained in slot \var{i} of the cell and
free variable storage.
\end{opcodedesc}

\begin{opcodedesc}{SET_LINENO}{lineno}
Sets the current line number to \var{lineno}.
\end{opcodedesc}

\begin{opcodedesc}{RAISE_VARARGS}{argc}
Raises an exception. \var{argc} indicates the number of parameters
to the raise statement, ranging from 0 to 3.  The handler will find
the traceback as TOS2, the parameter as TOS1, and the exception
as TOS.
\end{opcodedesc}

\begin{opcodedesc}{CALL_FUNCTION}{argc}
Calls a function.  The low byte of \var{argc} indicates the number of
positional parameters, the high byte the number of keyword parameters.
On the stack, the opcode finds the keyword parameters first.  For each
keyword argument, the value is on top of the key.  Below the keyword
parameters, the positional parameters are on the stack, with the
right-most parameter on top.  Below the parameters, the function object
to call is on the stack.
\end{opcodedesc}

\begin{opcodedesc}{MAKE_FUNCTION}{argc}
Pushes a new function object on the stack.  TOS is the code associated
with the function.  The function object is defined to have \var{argc}
default parameters, which are found below TOS.
\end{opcodedesc}

\begin{opcodedesc}{MAKE_CLOSURE}{argc}
Creates a new function object, sets its \var{func_closure} slot, and
pushes it on the stack.  TOS is the code associated with the function.
If the code object has N free variables, the next N items on the stack
are the cells for these variables.  The function also has \var{argc}
default parameters, where are found before the cells.
\end{opcodedesc}

\begin{opcodedesc}{BUILD_SLICE}{argc}
Pushes a slice object on the stack.  \var{argc} must be 2 or 3.  If it
is 2, \code{slice(TOS1, TOS)} is pushed; if it is 3,
\code{slice(TOS2, TOS1, TOS)} is pushed.
See the \code{slice()}\bifuncindex{slice} built-in function for more
information.
\end{opcodedesc}

\begin{opcodedesc}{EXTENDED_ARG}{ext}
Prefixes any opcode which has an argument too big to fit into the
default two bytes.  \var{ext} holds two additional bytes which, taken
together with the subsequent opcode's argument, comprise a four-byte
argument, \var{ext} being the two most-significant bytes.
\end{opcodedesc}

\begin{opcodedesc}{CALL_FUNCTION_VAR}{argc}
Calls a function. \var{argc} is interpreted as in \code{CALL_FUNCTION}.
The top element on the stack contains the variable argument list, followed
by keyword and positional arguments.
\end{opcodedesc}

\begin{opcodedesc}{CALL_FUNCTION_KW}{argc}
Calls a function. \var{argc} is interpreted as in \code{CALL_FUNCTION}.
The top element on the stack contains the keyword arguments dictionary, 
followed by explicit keyword and positional arguments.
\end{opcodedesc}

\begin{opcodedesc}{CALL_FUNCTION_VAR_KW}{argc}
Calls a function. \var{argc} is interpreted as in
\code{CALL_FUNCTION}.  The top element on the stack contains the
keyword arguments dictionary, followed by the variable-arguments
tuple, followed by explicit keyword and positional arguments.
\end{opcodedesc}

\section{\module{new} ---
         Creation of runtime internal objects}

\declaremodule{builtin}{new}
\sectionauthor{Moshe Zadka}{moshez@zadka.site.co.il}
\modulesynopsis{Interface to the creation of runtime implementation objects.}


The \module{new} module allows an interface to the interpreter object
creation functions. This is for use primarily in marshal-type functions,
when a new object needs to be created ``magically'' and not by using the
regular creation functions. This module provides a low-level interface
to the interpreter, so care must be exercised when using this module.

The \module{new} module defines the following functions:

\begin{funcdesc}{instance}{class\optional{, dict}}
This function creates an instance of \var{class} with dictionary
\var{dict} without calling the \method{__init__()} constructor.  If
\var{dict} is omitted or \code{None}, a new, empty dictionary is
created for the new instance.  Note that there are no guarantees that
the object will be in a consistent state.
\end{funcdesc}

\begin{funcdesc}{instancemethod}{function, instance, class}
This function will return a method object, bound to \var{instance}, or
unbound if \var{instance} is \code{None}.  \var{function} must be
callable, and \var{instance} must be an instance object or
\code{None}.
\end{funcdesc}

\begin{funcdesc}{function}{code, globals\optional{, name\optional{, argdefs}}}
Returns a (Python) function with the given code and globals. If
\var{name} is given, it must be a string or \code{None}.  If it is a
string, the function will have the given name, otherwise the function
name will be taken from \code{\var{code}.co_name}.  If
\var{argdefs} is given, it must be a tuple and will be used to
determine the default values of parameters.
\end{funcdesc}

\begin{funcdesc}{code}{argcount, nlocals, stacksize, flags, codestring,
                       constants, names, varnames, filename, name, firstlineno,
                       lnotab}
This function is an interface to the \cfunction{PyCode_New()} C
function.
%XXX This is still undocumented!!!!!!!!!!!
\end{funcdesc}

\begin{funcdesc}{module}{name}
This function returns a new module object with name \var{name}.
\var{name} must be a string.
\end{funcdesc}

\begin{funcdesc}{classobj}{name, baseclasses, dict}
This function returns a new class object, with name \var{name}, derived
from \var{baseclasses} (which should be a tuple of classes) and with
namespace \var{dict}.
\end{funcdesc}

\section{\module{site} ---
         Site-specific configuration hook}

\declaremodule{standard}{site}
\modulesynopsis{A standard way to reference site-specific modules.}


\strong{This module is automatically imported during initialization.}
The automatic import can be suppressed using the interpreter's
\programopt{-S} option.

Importing this module will append site-specific paths to the module
search path.
\indexiii{module}{search}{path}

It starts by constructing up to four directories from a head and a
tail part.  For the head part, it uses \code{sys.prefix} and
\code{sys.exec_prefix}; empty heads are skipped.  For
the tail part, it uses the empty string (on Windows) or
\file{lib/python\shortversion/site-packages} (on \UNIX{} and Macintosh)
and then \file{lib/site-python}.  For each of the distinct
head-tail combinations, it sees if it refers to an existing directory,
and if so, adds it to \code{sys.path} and also inspects the newly added 
path for configuration files.
\indexii{site-python}{directory}
\indexii{site-packages}{directory}

A path configuration file is a file whose name has the form
\file{\var{package}.pth} and exists in one of the four directories
mentioned above; its contents are additional items (one
per line) to be added to \code{sys.path}.  Non-existing items are
never added to \code{sys.path}, but no check is made that the item
refers to a directory (rather than a file).  No item is added to
\code{sys.path} more than once.  Blank lines and lines beginning with
\code{\#} are skipped.  Lines starting with \code{import} are executed.
\index{package}
\indexiii{path}{configuration}{file}

For example, suppose \code{sys.prefix} and \code{sys.exec_prefix} are
set to \file{/usr/local}.  The Python \version\ library is then
installed in \file{/usr/local/lib/python\shortversion} (where only the
first three characters of \code{sys.version} are used to form the
installation path name).  Suppose this has a subdirectory
\file{/usr/local/lib/python\shortversion/site-packages} with three
subsubdirectories, \file{foo}, \file{bar} and \file{spam}, and two
path configuration files, \file{foo.pth} and \file{bar.pth}.  Assume
\file{foo.pth} contains the following:

\begin{verbatim}
# foo package configuration

foo
bar
bletch
\end{verbatim}

and \file{bar.pth} contains:

\begin{verbatim}
# bar package configuration

bar
\end{verbatim}

Then the following directories are added to \code{sys.path}, in this
order:

\begin{verbatim}
/usr/local/lib/python2.3/site-packages/bar
/usr/local/lib/python2.3/site-packages/foo
\end{verbatim}

Note that \file{bletch} is omitted because it doesn't exist; the
\file{bar} directory precedes the \file{foo} directory because
\file{bar.pth} comes alphabetically before \file{foo.pth}; and
\file{spam} is omitted because it is not mentioned in either path
configuration file.

After these path manipulations, an attempt is made to import a module
named \module{sitecustomize}\refmodindex{sitecustomize}, which can
perform arbitrary site-specific customizations.  If this import fails
with an \exception{ImportError} exception, it is silently ignored.

Note that for some non-\UNIX{} systems, \code{sys.prefix} and
\code{sys.exec_prefix} are empty, and the path manipulations are
skipped; however the import of
\module{sitecustomize}\refmodindex{sitecustomize} is still attempted.

\section{\module{user} ---
         User-specific configuration hook}

\declaremodule{standard}{user}
\modulesynopsis{A standard way to reference user-specific modules.}


\indexii{.pythonrc.py}{file}
\indexiii{user}{configuration}{file}

As a policy, Python doesn't run user-specified code on startup of
Python programs.  (Only interactive sessions execute the script
specified in the \envvar{PYTHONSTARTUP} environment variable if it
exists).

However, some programs or sites may find it convenient to allow users
to have a standard customization file, which gets run when a program
requests it.  This module implements such a mechanism.  A program
that wishes to use the mechanism must execute the statement

\begin{verbatim}
import user
\end{verbatim}

The \module{user} module looks for a file \file{.pythonrc.py} in the user's
home directory and if it can be opened, executes it (using
\function{exec()}\bifuncindex{exec}) in its own (the
module \module{user}'s) global namespace.  Errors during this phase
are not caught; that's up to the program that imports the
\module{user} module, if it wishes.  The home directory is assumed to
be named by the \envvar{HOME} environment variable; if this is not set,
the current directory is used.

The user's \file{.pythonrc.py} could conceivably test for
\code{sys.version} if it wishes to do different things depending on
the Python version.

A warning to users: be very conservative in what you place in your
\file{.pythonrc.py} file.  Since you don't know which programs will
use it, changing the behavior of standard modules or functions is
generally not a good idea.

A suggestion for programmers who wish to use this mechanism: a simple
way to let users specify options for your package is to have them
define variables in their \file{.pythonrc.py} file that you test in
your module.  For example, a module \module{spam} that has a verbosity
level can look for a variable \code{user.spam_verbose}, as follows:

\begin{verbatim}
import user

verbose = bool(getattr(user, "spam_verbose", 0))
\end{verbatim}

(The three-argument form of \function{getattr()} is used in case
the user has not defined \code{spam_verbose} in their
\file{.pythonrc.py} file.)

Programs with extensive customization needs are better off reading a
program-specific customization file.

Programs with security or privacy concerns should \emph{not} import
this module; a user can easily break into a program by placing
arbitrary code in the \file{.pythonrc.py} file.

Modules for general use should \emph{not} import this module; it may
interfere with the operation of the importing program.

\begin{seealso}
  \seemodule{site}{Site-wide customization mechanism.}
\end{seealso}

\section{\module{__builtin__} ---
         Built-in functions.}
\declaremodule[builtin]{builtin}{__builtin__}

\modulesynopsis{The set of built-in functions.}


This module provides direct access to all `built-in' identifiers of
Python; e.g. \code{__builtin__.open} is the full name for the built-in
function \code{open()}.  See section \ref{built-in-funcs}, ``Built-in
Functions.''
		% really __builtin__
\section{\module{__main__} ---
         Top-level script environment}

\declaremodule[main]{builtin}{__main__}
\modulesynopsis{The environment where the top-level script is run.}

This module represents the (otherwise anonymous) scope in which the
interpreter's main program executes --- commands read either from
standard input, from a script file, or from an interactive prompt.  It
is this environment in which the idiomatic ``conditional script''
stanza causes a script to run:

\begin{verbatim}
if __name__ == "__main__":
    main()
\end{verbatim}
			% really __main__

\chapter{String Services}

The modules described in this chapter provide a wide range of string
manipulation operations.  Here's an overview:

\begin{description}

\item[string]
--- Common string operations.

\item[re]
--- New Perl-style regular expression search and match operations.

\item[regex]
--- Regular expression search and match operations.

\item[regsub]
--- Substitution and splitting operations that use regular expressions.

\item[struct]
--- Interpret strings as packed binary data.

\item[StringIO]
--- Read and write strings as if they were files.

\item[soundex]
--- Compute hash values for English words.

\end{description}
		% String Services
\section{\module{string} ---
         Common string operations}

\declaremodule{standard}{string}
\modulesynopsis{Common string operations.}

The \module{string} module contains a number of useful constants and classes,
as well as some deprecated legacy functions that are also available as methods
on strings.  See the module \refmodule{re}\refstmodindex{re} for string
functions based on regular expressions.

\subsection{String constants}

The constants defined in this module are:

\begin{datadesc}{ascii_letters}
  The concatenation of the \constant{ascii_lowercase} and
  \constant{ascii_uppercase} constants described below.  This value is
  not locale-dependent.
\end{datadesc}

\begin{datadesc}{ascii_lowercase}
  The lowercase letters \code{'abcdefghijklmnopqrstuvwxyz'}.  This
  value is not locale-dependent and will not change.
\end{datadesc}

\begin{datadesc}{ascii_uppercase}
  The uppercase letters \code{'ABCDEFGHIJKLMNOPQRSTUVWXYZ'}.  This
  value is not locale-dependent and will not change.
\end{datadesc}

\begin{datadesc}{digits}
  The string \code{'0123456789'}.
\end{datadesc}

\begin{datadesc}{hexdigits}
  The string \code{'0123456789abcdefABCDEF'}.
\end{datadesc}

\begin{datadesc}{letters}
  The concatenation of the strings \constant{lowercase} and
  \constant{uppercase} described below.  The specific value is
  locale-dependent, and will be updated when
  \function{locale.setlocale()} is called.
\end{datadesc}

\begin{datadesc}{lowercase}
  A string containing all the characters that are considered lowercase
  letters.  On most systems this is the string
  \code{'abcdefghijklmnopqrstuvwxyz'}.  Do not change its definition ---
  the effect on the routines \function{upper()} and
  \function{swapcase()} is undefined.  The specific value is
  locale-dependent, and will be updated when
  \function{locale.setlocale()} is called.
\end{datadesc}

\begin{datadesc}{octdigits}
  The string \code{'01234567'}.
\end{datadesc}

\begin{datadesc}{punctuation}
  String of \ASCII{} characters which are considered punctuation
  characters in the \samp{C} locale.
\end{datadesc}

\begin{datadesc}{printable}
  String of characters which are considered printable.  This is a
  combination of \constant{digits}, \constant{letters},
  \constant{punctuation}, and \constant{whitespace}.
\end{datadesc}

\begin{datadesc}{uppercase}
  A string containing all the characters that are considered uppercase
  letters.  On most systems this is the string
  \code{'ABCDEFGHIJKLMNOPQRSTUVWXYZ'}.  Do not change its definition ---
  the effect on the routines \function{lower()} and
  \function{swapcase()} is undefined.  The specific value is
  locale-dependent, and will be updated when
  \function{locale.setlocale()} is called.
\end{datadesc}

\begin{datadesc}{whitespace}
  A string containing all characters that are considered whitespace.
  On most systems this includes the characters space, tab, linefeed,
  return, formfeed, and vertical tab.  Do not change its definition ---
  the effect on the routines \function{strip()} and \function{split()}
  is undefined.
\end{datadesc}

\subsection{Template strings}

Templates provide simpler string substitutions as described in \pep{292}.
Instead of the normal \samp{\%}-based substitutions, Templates support
\samp{\$}-based substitutions, using the following rules:

\begin{itemize}
\item \samp{\$\$} is an escape; it is replaced with a single \samp{\$}.

\item \samp{\$identifier} names a substitution placeholder matching a mapping
       key of "identifier".  By default, "identifier" must spell a Python
       identifier.  The first non-identifier character after the \samp{\$}
       character terminates this placeholder specification.

\item \samp{\$\{identifier\}} is equivalent to \samp{\$identifier}.  It is
      required when valid identifier characters follow the placeholder but are
      not part of the placeholder, such as "\$\{noun\}ification".
\end{itemize}

Any other appearance of \samp{\$} in the string will result in a
\exception{ValueError} being raised.

\versionadded{2.4}

The \module{string} module provides a \class{Template} class that implements
these rules.  The methods of \class{Template} are:

\begin{classdesc}{Template}{template}
The constructor takes a single argument which is the template string.
\end{classdesc}

\begin{methoddesc}[Template]{substitute}{mapping\optional{, **kws}}
Performs the template substitution, returning a new string.  \var{mapping} is
any dictionary-like object with keys that match the placeholders in the
template.  Alternatively, you can provide keyword arguments, where the
keywords are the placeholders.  When both \var{mapping} and \var{kws} are
given and there are duplicates, the placeholders from \var{kws} take
precedence.
\end{methoddesc}

\begin{methoddesc}[Template]{safe_substitute}{mapping\optional{, **kws}}
Like \method{substitute()}, except that if placeholders are missing from
\var{mapping} and \var{kws}, instead of raising a \exception{KeyError}
exception, the original placeholder will appear in the resulting string
intact.  Also, unlike with \method{substitute()}, any other appearances of the
\samp{\$} will simply return \samp{\$} instead of raising
\exception{ValueError}.

While other exceptions may still occur, this method is called ``safe'' because
substitutions always tries to return a usable string instead of raising an
exception.  In another sense, \method{safe_substitute()} may be anything other
than safe, since it will silently ignore malformed templates containing
dangling delimiters, unmatched braces, or placeholders that are not valid
Python identifiers.
\end{methoddesc}

\class{Template} instances also provide one public data attribute:

\begin{memberdesc}[string]{template}
This is the object passed to the constructor's \var{template} argument.  In
general, you shouldn't change it, but read-only access is not enforced.
\end{memberdesc}

Here is an example of how to use a Template:

\begin{verbatim}
>>> from string import Template
>>> s = Template('$who likes $what')
>>> s.substitute(who='tim', what='kung pao')
'tim likes kung pao'
>>> d = dict(who='tim')
>>> Template('Give $who $100').substitute(d)
Traceback (most recent call last):
[...]
ValueError: Invalid placeholder in string: line 1, col 10
>>> Template('$who likes $what').substitute(d)
Traceback (most recent call last):
[...]
KeyError: 'what'
>>> Template('$who likes $what').safe_substitute(d)
'tim likes $what'
\end{verbatim}

Advanced usage: you can derive subclasses of \class{Template} to customize the
placeholder syntax, delimiter character, or the entire regular expression used
to parse template strings.  To do this, you can override these class
attributes:

\begin{itemize}
\item \var{delimiter} -- This is the literal string describing a placeholder
      introducing delimiter.  The default value \samp{\$}.  Note that this
      should \emph{not} be a regular expression, as the implementation will
      call \method{re.escape()} on this string as needed.
\item \var{idpattern} -- This is the regular expression describing the pattern
      for non-braced placeholders (the braces will be added automatically as
      appropriate).  The default value is the regular expression
      \samp{[_a-z][_a-z0-9]*}.
\end{itemize}

Alternatively, you can provide the entire regular expression pattern by
overriding the class attribute \var{pattern}.  If you do this, the value must
be a regular expression object with four named capturing groups.  The
capturing groups correspond to the rules given above, along with the invalid
placeholder rule:

\begin{itemize}
\item \var{escaped} -- This group matches the escape sequence,
      e.g. \samp{\$\$}, in the default pattern.
\item \var{named} -- This group matches the unbraced placeholder name; it
      should not include the delimiter in capturing group.
\item \var{braced} -- This group matches the brace enclosed placeholder name;
      it should not include either the delimiter or braces in the capturing
      group.
\item \var{invalid} -- This group matches any other delimiter pattern (usually
      a single delimiter), and it should appear last in the regular
      expression.
\end{itemize}

\subsection{String functions}

The following functions are available to operate on string and Unicode
objects.  They are not available as string methods.

\begin{funcdesc}{capwords}{s}
  Split the argument into words using \function{split()}, capitalize
  each word using \function{capitalize()}, and join the capitalized
  words using \function{join()}.  Note that this replaces runs of
  whitespace characters by a single space, and removes leading and
  trailing whitespace.
\end{funcdesc}

\begin{funcdesc}{maketrans}{from, to}
  Return a translation table suitable for passing to
  \function{translate()}, that will map
  each character in \var{from} into the character at the same position
  in \var{to}; \var{from} and \var{to} must have the same length.

  \warning{Don't use strings derived from \constant{lowercase}
  and \constant{uppercase} as arguments; in some locales, these don't have
  the same length.  For case conversions, always use
  \function{lower()} and \function{upper()}.}
\end{funcdesc}

\subsection{Deprecated string functions}

The following list of functions are also defined as methods of string and
Unicode objects; see ``String Methods'' (section
\ref{string-methods}) for more information on those.  You should consider
these functions as deprecated, although they will not be removed until Python
3.0.  The functions defined in this module are:

\begin{funcdesc}{atof}{s}
  \deprecated{2.0}{Use the \function{float()} built-in function.}
  Convert a string to a floating point number.  The string must have
  the standard syntax for a floating point literal in Python,
  optionally preceded by a sign (\samp{+} or \samp{-}).  Note that
  this behaves identical to the built-in function
  \function{float()}\bifuncindex{float} when passed a string.

  \note{When passing in a string, values for NaN\index{NaN}
  and Infinity\index{Infinity} may be returned, depending on the
  underlying C library.  The specific set of strings accepted which
  cause these values to be returned depends entirely on the C library
  and is known to vary.}
\end{funcdesc}

\begin{funcdesc}{atoi}{s\optional{, base}}
  \deprecated{2.0}{Use the \function{int()} built-in function.}
  Convert string \var{s} to an integer in the given \var{base}.  The
  string must consist of one or more digits, optionally preceded by a
  sign (\samp{+} or \samp{-}).  The \var{base} defaults to 10.  If it
  is 0, a default base is chosen depending on the leading characters
  of the string (after stripping the sign): \samp{0x} or \samp{0X}
  means 16, \samp{0} means 8, anything else means 10.  If \var{base}
  is 16, a leading \samp{0x} or \samp{0X} is always accepted, though
  not required.  This behaves identically to the built-in function
  \function{int()} when passed a string.  (Also note: for a more
  flexible interpretation of numeric literals, use the built-in
  function \function{eval()}\bifuncindex{eval}.)
\end{funcdesc}

\begin{funcdesc}{atol}{s\optional{, base}}
  \deprecated{2.0}{Use the \function{long()} built-in function.}
  Convert string \var{s} to a long integer in the given \var{base}.
  The string must consist of one or more digits, optionally preceded
  by a sign (\samp{+} or \samp{-}).  The \var{base} argument has the
  same meaning as for \function{atoi()}.  A trailing \samp{l} or
  \samp{L} is not allowed, except if the base is 0.  Note that when
  invoked without \var{base} or with \var{base} set to 10, this
  behaves identical to the built-in function
  \function{long()}\bifuncindex{long} when passed a string.
\end{funcdesc}

\begin{funcdesc}{capitalize}{word}
  Return a copy of \var{word} with only its first character capitalized.
\end{funcdesc}

\begin{funcdesc}{expandtabs}{s\optional{, tabsize}}
  Expand tabs in a string replacing them by one or more spaces,
  depending on the current column and the given tab size.  The column
  number is reset to zero after each newline occurring in the string.
  This doesn't understand other non-printing characters or escape
  sequences.  The tab size defaults to 8.
\end{funcdesc}

\begin{funcdesc}{find}{s, sub\optional{, start\optional{,end}}}
  Return the lowest index in \var{s} where the substring \var{sub} is
  found such that \var{sub} is wholly contained in
  \code{\var{s}[\var{start}:\var{end}]}.  Return \code{-1} on failure.
  Defaults for \var{start} and \var{end} and interpretation of
  negative values is the same as for slices.
\end{funcdesc}

\begin{funcdesc}{rfind}{s, sub\optional{, start\optional{, end}}}
  Like \function{find()} but find the highest index.
\end{funcdesc}

\begin{funcdesc}{index}{s, sub\optional{, start\optional{, end}}}
  Like \function{find()} but raise \exception{ValueError} when the
  substring is not found.
\end{funcdesc}

\begin{funcdesc}{rindex}{s, sub\optional{, start\optional{, end}}}
  Like \function{rfind()} but raise \exception{ValueError} when the
  substring is not found.
\end{funcdesc}

\begin{funcdesc}{count}{s, sub\optional{, start\optional{, end}}}
  Return the number of (non-overlapping) occurrences of substring
  \var{sub} in string \code{\var{s}[\var{start}:\var{end}]}.
  Defaults for \var{start} and \var{end} and interpretation of
  negative values are the same as for slices.
\end{funcdesc}

\begin{funcdesc}{lower}{s}
  Return a copy of \var{s}, but with upper case letters converted to
  lower case.
\end{funcdesc}

\begin{funcdesc}{split}{s\optional{, sep\optional{, maxsplit}}}
  Return a list of the words of the string \var{s}.  If the optional
  second argument \var{sep} is absent or \code{None}, the words are
  separated by arbitrary strings of whitespace characters (space, tab, 
  newline, return, formfeed).  If the second argument \var{sep} is
  present and not \code{None}, it specifies a string to be used as the 
  word separator.  The returned list will then have one more item
  than the number of non-overlapping occurrences of the separator in
  the string.  The optional third argument \var{maxsplit} defaults to
  0.  If it is nonzero, at most \var{maxsplit} number of splits occur,
  and the remainder of the string is returned as the final element of
  the list (thus, the list will have at most \code{\var{maxsplit}+1}
  elements).

  The behavior of split on an empty string depends on the value of \var{sep}.
  If \var{sep} is not specified, or specified as \code{None}, the result will
  be an empty list.  If \var{sep} is specified as any string, the result will
  be a list containing one element which is an empty string.
\end{funcdesc}

\begin{funcdesc}{rsplit}{s\optional{, sep\optional{, maxsplit}}}
  Return a list of the words of the string \var{s}, scanning \var{s}
  from the end.  To all intents and purposes, the resulting list of
  words is the same as returned by \function{split()}, except when the
  optional third argument \var{maxsplit} is explicitly specified and
  nonzero.  When \var{maxsplit} is nonzero, at most \var{maxsplit}
  number of splits -- the \emph{rightmost} ones -- occur, and the remainder
  of the string is returned as the first element of the list (thus, the
  list will have at most \code{\var{maxsplit}+1} elements).
  \versionadded{2.4}
\end{funcdesc}

\begin{funcdesc}{splitfields}{s\optional{, sep\optional{, maxsplit}}}
  This function behaves identically to \function{split()}.  (In the
  past, \function{split()} was only used with one argument, while
  \function{splitfields()} was only used with two arguments.)
\end{funcdesc}

\begin{funcdesc}{join}{words\optional{, sep}}
  Concatenate a list or tuple of words with intervening occurrences of 
  \var{sep}.  The default value for \var{sep} is a single space
  character.  It is always true that
  \samp{string.join(string.split(\var{s}, \var{sep}), \var{sep})}
  equals \var{s}.
\end{funcdesc}

\begin{funcdesc}{joinfields}{words\optional{, sep}}
  This function behaves identically to \function{join()}.  (In the past, 
  \function{join()} was only used with one argument, while
  \function{joinfields()} was only used with two arguments.)
  Note that there is no \method{joinfields()} method on string
  objects; use the \method{join()} method instead.
\end{funcdesc}

\begin{funcdesc}{lstrip}{s\optional{, chars}}
Return a copy of the string with leading characters removed.  If
\var{chars} is omitted or \code{None}, whitespace characters are
removed.  If given and not \code{None}, \var{chars} must be a string;
the characters in the string will be stripped from the beginning of
the string this method is called on.
\versionchanged[The \var{chars} parameter was added.  The \var{chars}
parameter cannot be passed in earlier 2.2 versions]{2.2.3}
\end{funcdesc}

\begin{funcdesc}{rstrip}{s\optional{, chars}}
Return a copy of the string with trailing characters removed.  If
\var{chars} is omitted or \code{None}, whitespace characters are
removed.  If given and not \code{None}, \var{chars} must be a string;
the characters in the string will be stripped from the end of the
string this method is called on.
\versionchanged[The \var{chars} parameter was added.  The \var{chars}
parameter cannot be passed in earlier 2.2 versions]{2.2.3}
\end{funcdesc}

\begin{funcdesc}{strip}{s\optional{, chars}}
Return a copy of the string with leading and trailing characters
removed.  If \var{chars} is omitted or \code{None}, whitespace
characters are removed.  If given and not \code{None}, \var{chars}
must be a string; the characters in the string will be stripped from
the both ends of the string this method is called on.
\versionchanged[The \var{chars} parameter was added.  The \var{chars}
parameter cannot be passed in earlier 2.2 versions]{2.2.3}
\end{funcdesc}

\begin{funcdesc}{swapcase}{s}
  Return a copy of \var{s}, but with lower case letters
  converted to upper case and vice versa.
\end{funcdesc}

\begin{funcdesc}{translate}{s, table\optional{, deletechars}}
  Delete all characters from \var{s} that are in \var{deletechars} (if 
  present), and then translate the characters using \var{table}, which 
  must be a 256-character string giving the translation for each
  character value, indexed by its ordinal.
\end{funcdesc}

\begin{funcdesc}{upper}{s}
  Return a copy of \var{s}, but with lower case letters converted to
  upper case.
\end{funcdesc}

\begin{funcdesc}{ljust}{s, width}
\funcline{rjust}{s, width}
\funcline{center}{s, width}
  These functions respectively left-justify, right-justify and center
  a string in a field of given width.  They return a string that is at
  least \var{width} characters wide, created by padding the string
  \var{s} with spaces until the given width on the right, left or both
  sides.  The string is never truncated.
\end{funcdesc}

\begin{funcdesc}{zfill}{s, width}
  Pad a numeric string on the left with zero digits until the given
  width is reached.  Strings starting with a sign are handled
  correctly.
\end{funcdesc}

\begin{funcdesc}{replace}{str, old, new\optional{, maxreplace}}
  Return a copy of string \var{str} with all occurrences of substring
  \var{old} replaced by \var{new}.  If the optional argument
  \var{maxreplace} is given, the first \var{maxreplace} occurrences are
  replaced.
\end{funcdesc}

\section{\module{re} ---
         New Perl-style regular expression search and match operations.}
\declaremodule{standard}{re}
\moduleauthor{Andrew M. Kuchling}{akuchling@acm.org}
\sectionauthor{Andrew M. Kuchling}{akuchling@acm.org}


\modulesynopsis{New Perl-style regular expression search and match
operations.}


This module provides regular expression matching operations similar to
those found in Perl.  It's 8-bit clean: the strings being processed
may contain both null bytes and characters whose high bit is set.  Regular
expression patterns may not contain null bytes, but they may contain
characters with the high bit set.  The \module{re} module is always
available.

Regular expressions use the backslash character (\character{\e}) to
indicate special forms or to allow special characters to be used
without invoking their special meaning.  This collides with Python's
usage of the same character for the same purpose in string literals;
for example, to match a literal backslash, one might have to write
\code{'\e\e\e\e'} as the pattern string, because the regular expression
must be \samp{\e\e}, and each backslash must be expressed as
\samp{\e\e} inside a regular Python string literal. 

The solution is to use Python's raw string notation for regular
expression patterns; backslashes are not handled in any special way in
a string literal prefixed with \character{r}.  So \code{r"\e n"} is a
two-character string containing \character{\e} and \character{n},
while \code{"\e n"} is a one-character string containing a newline.
Usually patterns will be expressed in Python code using this raw
string notation.

\subsection{Regular Expression Syntax \label{re-syntax}}

A regular expression (or RE) specifies a set of strings that matches
it; the functions in this module let you check if a particular string
matches a given regular expression (or if a given regular expression
matches a particular string, which comes down to the same thing).

Regular expressions can be concatenated to form new regular
expressions; if \emph{A} and \emph{B} are both regular expressions,
then \emph{AB} is also an regular expression.  If a string \emph{p}
matches A and another string \emph{q} matches B, the string \emph{pq}
will match AB.  Thus, complex expressions can easily be constructed
from simpler primitive expressions like the ones described here.  For
details of the theory and implementation of regular expressions,
consult the Friedl book referenced below, or almost any textbook about
compiler construction.

A brief explanation of the format of regular expressions follows.  For
further information and a gentler presentation, consult the Regular
Expression HOWTO, accessible from \url{http://www.python.org/doc/howto/}.

Regular expressions can contain both special and ordinary characters.
Most ordinary characters, like \character{A}, \character{a}, or \character{0},
are the simplest regular expressions; they simply match themselves.  
You can concatenate ordinary characters, so \regexp{last} matches the
string \code{'last'}.  (In the rest of this section, we'll write RE's in
\regexp{this special style}, usually without quotes, and strings to be
matched \code{'in single quotes'}.)

Some characters, like \character{|} or \character{(}, are special.  Special
characters either stand for classes of ordinary characters, or affect
how the regular expressions around them are interpreted.

The special characters are:

\begin{list}{}{\leftmargin 0.7in \labelwidth 0.65in}

\item[\character{.}] (Dot.)  In the default mode, this matches any
character except a newline.  If the \constant{DOTALL} flag has been
specified, this matches any character including a newline.

\item[\character{\^}] (Caret.)  Matches the start of the string, and in
\constant{MULTILINE} mode also matches immediately after each newline.

\item[\character{\$}] Matches the end of the string, and in
\constant{MULTILINE} mode also matches before a newline.
\regexp{foo} matches both 'foo' and 'foobar', while the regular
expression \regexp{foo\$} matches only 'foo'.

\item[\character{*}] Causes the resulting RE to
match 0 or more repetitions of the preceding RE, as many repetitions
as are possible.  \regexp{ab*} will
match 'a', 'ab', or 'a' followed by any number of 'b's.

\item[\character{+}] Causes the
resulting RE to match 1 or more repetitions of the preceding RE.
\regexp{ab+} will match 'a' followed by any non-zero number of 'b's; it
will not match just 'a'.

\item[\character{?}] Causes the resulting RE to
match 0 or 1 repetitions of the preceding RE.  \regexp{ab?} will
match either 'a' or 'ab'.
\item[\code{*?}, \code{+?}, \code{??}] The \character{*}, \character{+}, and
\character{?} qualifiers are all \dfn{greedy}; they match as much text as
possible.  Sometimes this behaviour isn't desired; if the RE
\regexp{<.*>} is matched against \code{'<H1>title</H1>'}, it will match the
entire string, and not just \code{'<H1>'}.
Adding \character{?} after the qualifier makes it perform the match in
\dfn{non-greedy} or \dfn{minimal} fashion; as \emph{few} characters as
possible will be matched.  Using \regexp{.*?} in the previous
expression will match only \code{'<H1>'}.

\item[\code{\{\var{m},\var{n}\}}] Causes the resulting RE to match from
\var{m} to \var{n} repetitions of the preceding RE, attempting to
match as many repetitions as possible.  For example, \regexp{a\{3,5\}}
will match from 3 to 5 \character{a} characters.  Omitting \var{n}
specifies an infinite upper bound; you can't omit \var{m}.

\item[\code{\{\var{m},\var{n}\}?}] Causes the resulting RE to
match from \var{m} to \var{n} repetitions of the preceding RE,
attempting to match as \emph{few} repetitions as possible.  This is
the non-greedy version of the previous qualifier.  For example, on the
6-character string \code{'aaaaaa'}, \regexp{a\{3,5\}} will match 5
\character{a} characters, while \regexp{a\{3,5\}?} will only match 3
characters.

\item[\character{\e}] Either escapes special characters (permitting
you to match characters like \character{*}, \character{?}, and so
forth), or signals a special sequence; special sequences are discussed
below.

If you're not using a raw string to
express the pattern, remember that Python also uses the
backslash as an escape sequence in string literals; if the escape
sequence isn't recognized by Python's parser, the backslash and
subsequent character are included in the resulting string.  However,
if Python would recognize the resulting sequence, the backslash should
be repeated twice.  This is complicated and hard to understand, so
it's highly recommended that you use raw strings for all but the
simplest expressions.

\item[\code{[]}] Used to indicate a set of characters.  Characters can
be listed individually, or a range of characters can be indicated by
giving two characters and separating them by a \character{-}.  Special
characters are not active inside sets.  For example, \regexp{[akm\$]}
will match any of the characters \character{a}, \character{k},
\character{m}, or \character{\$}; \regexp{[a-z]}
will match any lowercase letter, and \code{[a-zA-Z0-9]} matches any
letter or digit.  Character classes such as \code{\e w} or \code{\e S}
(defined below) are also acceptable inside a range.  If you want to
include a \character{]} or a \character{-} inside a set, precede it with a
backslash, or place it as the first character.  The 
pattern \regexp{[]]} will match \code{']'}, for example.  

You can match the characters not within a range by \dfn{complementing}
the set.  This is indicated by including a
\character{\^} as the first character of the set; \character{\^} elsewhere will
simply match the \character{\^} character.  For example, \regexp{[{\^}5]}
will match any character except \character{5}.

\item[\character{|}]\code{A|B}, where A and B can be arbitrary REs,
creates a regular expression that will match either A or B.  This can
be used inside groups (see below) as well.  To match a literal \character{|},
use \regexp{\e|}, or enclose it inside a character class, as in  \regexp{[|]}.

\item[\code{(...)}] Matches whatever regular expression is inside the
parentheses, and indicates the start and end of a group; the contents
of a group can be retrieved after a match has been performed, and can
be matched later in the string with the \regexp{\e \var{number}} special
sequence, described below.  To match the literals \character{(} or
\character{')}, use \regexp{\e(} or \regexp{\e)}, or enclose them
inside a character class: \regexp{[(] [)]}.

\item[\code{(?...)}] This is an extension notation (a \character{?}
following a \character{(} is not meaningful otherwise).  The first
character after the \character{?} 
determines what the meaning and further syntax of the construct is.
Extensions usually do not create a new group;
\regexp{(?P<\var{name}>...)} is the only exception to this rule.
Following are the currently supported extensions.

\item[\code{(?iLmsx)}] (One or more letters from the set \character{i},
\character{L}, \character{m}, \character{s}, \character{x}.)  The group matches
the empty string; the letters set the corresponding flags
(\constant{re.I}, \constant{re.L}, \constant{re.M}, \constant{re.S},
\constant{re.X}) for the entire regular expression.  This is useful if
you wish to include the flags as part of the regular expression, instead
of passing a \var{flag} argument to the \function{compile()} function. 

\item[\code{(?:...)}] A non-grouping version of regular parentheses.
Matches whatever regular expression is inside the parentheses, but the
substring matched by the 
group \emph{cannot} be retrieved after performing a match or
referenced later in the pattern. 

\item[\code{(?P<\var{name}>...)}] Similar to regular parentheses, but
the substring matched by the group is accessible via the symbolic group
name \var{name}.  Group names must be valid Python identifiers.  A
symbolic group is also a numbered group, just as if the group were not
named.  So the group named 'id' in the example above can also be
referenced as the numbered group 1.

For example, if the pattern is
\regexp{(?P<id>[a-zA-Z_]\e w*)}, the group can be referenced by its
name in arguments to methods of match objects, such as \code{m.group('id')}
or \code{m.end('id')}, and also by name in pattern text
(e.g. \regexp{(?P=id)}) and replacement text (e.g. \code{\e g<id>}).

\item[\code{(?P=\var{name})}] Matches whatever text was matched by the
earlier group named \var{name}.

\item[\code{(?\#...)}] A comment; the contents of the parentheses are
simply ignored.

\item[\code{(?=...)}] Matches if \regexp{...} matches next, but doesn't
consume any of the string.  This is called a lookahead assertion.  For
example, \regexp{Isaac (?=Asimov)} will match \code{'Isaac~'} only if it's
followed by \code{'Asimov'}.

\item[\code{(?!...)}] Matches if \regexp{...} doesn't match next.  This
is a negative lookahead assertion.  For example,
\regexp{Isaac (?!Asimov)} will match \code{'Isaac~'} only if it's \emph{not}
followed by \code{'Asimov'}.

\end{list}

The special sequences consist of \character{\e} and a character from the
list below.  If the ordinary character is not on the list, then the
resulting RE will match the second character.  For example,
\regexp{\e\$} matches the character \character{\$}.

\begin{list}{}{\leftmargin 0.7in \labelwidth 0.65in}

%
\item[\code{\e \var{number}}] Matches the contents of the group of the
same number.  Groups are numbered starting from 1.  For example,
\regexp{(.+) \e 1} matches \code{'the the'} or \code{'55 55'}, but not
\code{'the end'} (note 
the space after the group).  This special sequence can only be used to
match one of the first 99 groups.  If the first digit of \var{number}
is 0, or \var{number} is 3 octal digits long, it will not be interpreted
as a group match, but as the character with octal value \var{number}.
Inside the \character{[} and \character{]} of a character class, all numeric
escapes are treated as characters. 
%
\item[\code{\e A}] Matches only at the start of the string.
%
\item[\code{\e b}] Matches the empty string, but only at the
beginning or end of a word.  A word is defined as a sequence of
alphanumeric characters, so the end of a word is indicated by
whitespace or a non-alphanumeric character.  Inside a character range,
\regexp{\e b} represents the backspace character, for compatibility with
Python's string literals.
%
\item[\code{\e B}] Matches the empty string, but only when it is
\emph{not} at the beginning or end of a word.
%
\item[\code{\e d}]Matches any decimal digit; this is
equivalent to the set \regexp{[0-9]}.
%
\item[\code{\e D}]Matches any non-digit character; this is
equivalent to the set \regexp{[{\^}0-9]}.
%
\item[\code{\e s}]Matches any whitespace character; this is
equivalent to the set \regexp{[ \e t\e n\e r\e f\e v]}.
%
\item[\code{\e S}]Matches any non-whitespace character; this is
equivalent to the set \regexp{[\^\ \e t\e n\e r\e f\e v]}.
%
\item[\code{\e w}]When the \constant{LOCALE} flag is not specified,
matches any alphanumeric character; this is equivalent to the set
\regexp{[a-zA-Z0-9_]}.  With \constant{LOCALE}, it will match the set
\regexp{[0-9_]} plus whatever characters are defined as letters for the
current locale.
%
\item[\code{\e W}]When the \constant{LOCALE} flag is not specified,
matches any non-alphanumeric character; this is equivalent to the set
\regexp{[{\^}a-zA-Z0-9_]}.   With \constant{LOCALE}, it will match any
character not in the set \regexp{[0-9_]}, and not defined as a letter
for the current locale.

\item[\code{\e Z}]Matches only at the end of the string.
%

\item[\code{\e \e}] Matches a literal backslash.

\end{list}


\subsection{Matching vs. Searching \label{matching-searching}}
\sectionauthor{Fred L. Drake, Jr.}{fdrake@acm.org}

\strong{XXX This section is still incomplete!}

Python offers two different primitive operations based on regular
expressions: match and search.  If you are accustomed to Perl's
semantics, the search operation is what you're looking for.  See the
\function{search()} function and corresponding method of compiled
regular expression objects.

Note that match may differ from search using a regular expression
beginning with \character{\^}:  \character{\^} matches only at the start
of the string, or in \constant{MULTILINE} mode also immediately
following a newline.  "match" succeeds only if the pattern matches at
the start of the string regardless of mode, or at the starting
position given by the optional \var{pos} argument regardless of
whether a newline precedes it.

% Examples from Tim Peters:
\begin{verbatim}
re.compile("a").match("ba", 1)           # succeeds
re.compile("^a").search("ba", 1)         # fails; 'a' not at start
re.compile("^a").search("\na", 1)        # fails; 'a' not at start
re.compile("^a", re.M).search("\na", 1)  # succeeds
re.compile("^a", re.M).search("ba", 1)   # fails; no preceding \n
\end{verbatim}


\subsection{Module Contents}
\nodename{Contents of Module re}

The module defines the following functions and constants, and an exception:


\begin{funcdesc}{compile}{pattern\optional{, flags}}
  Compile a regular expression pattern into a regular expression
  object, which can be used for matching using its \function{match()} and
  \function{search()} methods, described below.  

  The expression's behaviour can be modified by specifying a
  \var{flags} value.  Values can be any of the following variables,
  combined using bitwise OR (the \code{|} operator).

The sequence

\begin{verbatim}
prog = re.compile(pat)
result = prog.match(str)
\end{verbatim}

is equivalent to

\begin{verbatim}
result = re.match(pat, str)
\end{verbatim}

but the version using \function{compile()} is more efficient when the
expression will be used several times in a single program.
%(The compiled version of the last pattern passed to
%\function{regex.match()} or \function{regex.search()} is cached, so
%programs that use only a single regular expression at a time needn't
%worry about compiling regular expressions.)
\end{funcdesc}

\begin{datadesc}{I}
\dataline{IGNORECASE}
Perform case-insensitive matching; expressions like \regexp{[A-Z]} will match
lowercase letters, too.  This is not affected by the current locale.
\end{datadesc}

\begin{datadesc}{L}
\dataline{LOCALE}
Make \regexp{\e w}, \regexp{\e W}, \regexp{\e b},
\regexp{\e B}, dependent on the current locale. 
\end{datadesc}

\begin{datadesc}{M}
\dataline{MULTILINE}
When specified, the pattern character \character{\^} matches at the
beginning of the string and at the beginning of each line
(immediately following each newline); and the pattern character
\character{\$} matches at the end of the string and at the end of each line
(immediately preceding each newline).
By default, \character{\^} matches only at the beginning of the string, and
\character{\$} only at the end of the string and immediately before the
newline (if any) at the end of the string. 
\end{datadesc}

\begin{datadesc}{S}
\dataline{DOTALL}
Make the \character{.} special character match any character at all, including a
newline; without this flag, \character{.} will match anything \emph{except}
a newline.
\end{datadesc}

\begin{datadesc}{X}
\dataline{VERBOSE}
This flag allows you to write regular expressions that look nicer.
Whitespace within the pattern is ignored, 
except when in a character class or preceded by an unescaped
backslash, and, when a line contains a \character{\#} neither in a character
class or preceded by an unescaped backslash, all characters from the
leftmost such \character{\#} through the end of the line are ignored.
% XXX should add an example here
\end{datadesc}


\begin{funcdesc}{search}{pattern, string\optional{, flags}}
  Scan through \var{string} looking for a location where the regular
  expression \var{pattern} produces a match, and return a
  corresponding \class{MatchObject} instance.
  Return \code{None} if no
  position in the string matches the pattern; note that this is
  different from finding a zero-length match at some point in the string.
\end{funcdesc}

\begin{funcdesc}{match}{pattern, string\optional{, flags}}
  If zero or more characters at the beginning of \var{string} match
  the regular expression \var{pattern}, return a corresponding
  \class{MatchObject} instance.  Return \code{None} if the string does not
  match the pattern; note that this is different from a zero-length
  match.

  \strong{Note:}  If you want to locate a match anywhere in
  \var{string}, use \method{search()} instead.
\end{funcdesc}

\begin{funcdesc}{split}{pattern, string, \optional{, maxsplit\code{ = 0}}}
  Split \var{string} by the occurrences of \var{pattern}.  If
  capturing parentheses are used in \var{pattern}, then the text of all
  groups in the pattern are also returned as part of the resulting list.
  If \var{maxsplit} is nonzero, at most \var{maxsplit} splits
  occur, and the remainder of the string is returned as the final
  element of the list.  (Incompatibility note: in the original Python
  1.5 release, \var{maxsplit} was ignored.  This has been fixed in
  later releases.)

\begin{verbatim}
>>> re.split('\W+', 'Words, words, words.')
['Words', 'words', 'words', '']
>>> re.split('(\W+)', 'Words, words, words.')
['Words', ', ', 'words', ', ', 'words', '.', '']
>>> re.split('\W+', 'Words, words, words.', 1)
['Words', 'words, words.']
\end{verbatim}

  This function combines and extends the functionality of
  the old \function{regsub.split()} and \function{regsub.splitx()}.
\end{funcdesc}

\begin{funcdesc}{findall}{pattern, string}
\versionadded{1.5.2}
Return a list of all non-overlapping matches of \var{pattern} in
\var{string}.  If one or more groups are present in the pattern,
return a list of groups; this will be a list of tuples if the pattern
has more than one group.  Empty matches are included in the result.
\end{funcdesc}

\begin{funcdesc}{sub}{pattern, repl, string\optional{, count\code{ = 0}}}
Return the string obtained by replacing the leftmost non-overlapping
occurrences of \var{pattern} in \var{string} by the replacement
\var{repl}.  If the pattern isn't found, \var{string} is returned
unchanged.  \var{repl} can be a string or a function; if a function,
it is called for every non-overlapping occurance of \var{pattern}.
The function takes a single match object argument, and returns the
replacement string.  For example:

\begin{verbatim}
>>> def dashrepl(matchobj):
....    if matchobj.group(0) == '-': return ' '
....    else: return '-'
>>> re.sub('-{1,2}', dashrepl, 'pro----gram-files')
'pro--gram files'
\end{verbatim}

The pattern may be a string or a 
regex object; if you need to specify
regular expression flags, you must use a regex object, or use
embedded modifiers in a pattern; e.g.
\samp{sub("(?i)b+", "x", "bbbb BBBB")} returns \code{'x x'}.

The optional argument \var{count} is the maximum number of pattern
occurrences to be replaced; \var{count} must be a non-negative integer, and
the default value of 0 means to replace all occurrences.

Empty matches for the pattern are replaced only when not adjacent to a
previous match, so \samp{sub('x*', '-', 'abc')} returns \code{'-a-b-c-'}.

If \var{repl} is a string, any backslash escapes in it are processed.
That is, \samp{\e n} is converted to a single newline character,
\samp{\e r} is converted to a linefeed, and so forth.  Unknown escapes
such as \samp{\e j} are left alone.  Backreferences, such as \samp{\e 6}, are
replaced with the substring matched by group 6 in the pattern. 

In addition to character escapes and backreferences as described
above, \samp{\e g<name>} will use the substring matched by the group
named \samp{name}, as defined by the \regexp{(?P<name>...)} syntax.
\samp{\e g<number>} uses the corresponding group number; \samp{\e
g<2>} is therefore equivalent to \samp{\e 2}, but isn't ambiguous in a
replacement such as \samp{\e g<2>0}.  \samp{\e 20} would be
interpreted as a reference to group 20, not a reference to group 2
followed by the literal character \character{0}.  
\end{funcdesc}

\begin{funcdesc}{subn}{pattern, repl, string\optional{, count\code{ = 0}}}
Perform the same operation as \function{sub()}, but return a tuple
\code{(\var{new_string}, \var{number_of_subs_made})}.
\end{funcdesc}

\begin{funcdesc}{escape}{string}
  Return \var{string} with all non-alphanumerics backslashed; this is
  useful if you want to match an arbitrary literal string that may have
  regular expression metacharacters in it.
\end{funcdesc}

\begin{excdesc}{error}
  Exception raised when a string passed to one of the functions here
  is not a valid regular expression (e.g., unmatched parentheses) or
  when some other error occurs during compilation or matching.  It is
  never an error if a string contains no match for a pattern.
\end{excdesc}


\subsection{Regular Expression Objects \label{re-objects}}

Compiled regular expression objects support the following methods and
attributes:

\begin{methoddesc}[RegexObject]{search}{string\optional{, pos}\optional{,
                                        endpos}}
  Scan through \var{string} looking for a location where this regular
  expression produces a match, and return a
  corresponding \class{MatchObject} instance.  Return \code{None} if no
  position in the string matches the pattern; note that this is
  different from finding a zero-length match at some point in the string.
  
  The optional \var{pos} and \var{endpos} parameters have the same
  meaning as for the \method{match()} method.
\end{methoddesc}

\begin{methoddesc}[RegexObject]{match}{string\optional{, pos}\optional{,
                                       endpos}}
  If zero or more characters at the beginning of \var{string} match
  this regular expression, return a corresponding
  \class{MatchObject} instance.  Return \code{None} if the string does not
  match the pattern; note that this is different from a zero-length
  match.

  \strong{Note:}  If you want to locate a match anywhere in
  \var{string}, use \method{search()} instead.

  The optional second parameter \var{pos} gives an index in the string
  where the search is to start; it defaults to \code{0}.  This is not
  completely equivalent to slicing the string; the \code{'\^'} pattern
  character matches at the real beginning of the string and at positions
  just after a newline, but not necessarily at the index where the search
  is to start.

  The optional parameter \var{endpos} limits how far the string will
  be searched; it will be as if the string is \var{endpos} characters
  long, so only the characters from \var{pos} to \var{endpos} will be
  searched for a match.
\end{methoddesc}

\begin{methoddesc}[RegexObject]{split}{string, \optional{,
                                       maxsplit\code{ = 0}}}
Identical to the \function{split()} function, using the compiled pattern.
\end{methoddesc}

\begin{methoddesc}[RegexObject]{findall}{string}
Identical to the \function{findall()} function, using the compiled pattern.
\end{methoddesc}

\begin{methoddesc}[RegexObject]{sub}{repl, string\optional{, count\code{ = 0}}}
Identical to the \function{sub()} function, using the compiled pattern.
\end{methoddesc}

\begin{methoddesc}[RegexObject]{subn}{repl, string\optional{,
                                      count\code{ = 0}}}
Identical to the \function{subn()} function, using the compiled pattern.
\end{methoddesc}


\begin{memberdesc}[RegexObject]{flags}
The flags argument used when the regex object was compiled, or
\code{0} if no flags were provided.
\end{memberdesc}

\begin{memberdesc}[RegexObject]{groupindex}
A dictionary mapping any symbolic group names defined by 
\regexp{(?P<\var{id}>)} to group numbers.  The dictionary is empty if no
symbolic groups were used in the pattern.
\end{memberdesc}

\begin{memberdesc}[RegexObject]{pattern}
The pattern string from which the regex object was compiled.
\end{memberdesc}


\subsection{Match Objects \label{match-objects}}

\class{MatchObject} instances support the following methods and attributes:

\begin{methoddesc}[MatchObject]{group}{\optional{group1, group2, ...}}
Returns one or more subgroups of the match.  If there is a single
argument, the result is a single string; if there are
multiple arguments, the result is a tuple with one item per argument.
Without arguments, \var{group1} defaults to zero (i.e. the whole match
is returned).
If a \var{groupN} argument is zero, the corresponding return value is the
entire matching string; if it is in the inclusive range [1..99], it is
the string matching the the corresponding parenthesized group.  If a
group number is negative or larger than the number of groups defined
in the pattern, an \exception{IndexError} exception is raised.
If a group is contained in a part of the pattern that did not match,
the corresponding result is \code{None}.  If a group is contained in a 
part of the pattern that matched multiple times, the last match is
returned.

If the regular expression uses the \regexp{(?P<\var{name}>...)} syntax,
the \var{groupN} arguments may also be strings identifying groups by
their group name.  If a string argument is not used as a group name in 
the pattern, an \exception{IndexError} exception is raised.

A moderately complicated example:

\begin{verbatim}
m = re.match(r"(?P<int>\d+)\.(\d*)", '3.14')
\end{verbatim}

After performing this match, \code{m.group(1)} is \code{'3'}, as is
\code{m.group('int')}, and \code{m.group(2)} is \code{'14'}.
\end{methoddesc}

\begin{methoddesc}[MatchObject]{groups}{\optional{default}}
Return a tuple containing all the subgroups of the match, from 1 up to
however many groups are in the pattern.  The \var{default} argument is
used for groups that did not participate in the match; it defaults to
\code{None}.  (Incompatibility note: in the original Python 1.5
release, if the tuple was one element long, a string would be returned
instead.  In later versions (from 1.5.1 on), a singleton tuple is
returned in such cases.)
\end{methoddesc}

\begin{methoddesc}[MatchObject]{groupdict}{\optional{default}}
Return a dictionary containing all the \emph{named} subgroups of the
match, keyed by the subgroup name.  The \var{default} argument is
used for groups that did not participate in the match; it defaults to
\code{None}.
\end{methoddesc}

\begin{methoddesc}[MatchObject]{start}{\optional{group}}
\funcline{end}{\optional{group}}
Return the indices of the start and end of the substring
matched by \var{group}; \var{group} defaults to zero (meaning the whole
matched substring).
Return \code{None} if \var{group} exists but
did not contribute to the match.  For a match object
\var{m}, and a group \var{g} that did contribute to the match, the
substring matched by group \var{g} (equivalent to
\code{\var{m}.group(\var{g})}) is

\begin{verbatim}
m.string[m.start(g):m.end(g)]
\end{verbatim}

Note that
\code{m.start(\var{group})} will equal \code{m.end(\var{group})} if
\var{group} matched a null string.  For example, after \code{\var{m} =
re.search('b(c?)', 'cba')}, \code{\var{m}.start(0)} is 1,
\code{\var{m}.end(0)} is 2, \code{\var{m}.start(1)} and
\code{\var{m}.end(1)} are both 2, and \code{\var{m}.start(2)} raises
an \exception{IndexError} exception.
\end{methoddesc}

\begin{methoddesc}[MatchObject]{span}{\optional{group}}
For \class{MatchObject} \var{m}, return the 2-tuple
\code{(\var{m}.start(\var{group}), \var{m}.end(\var{group}))}.
Note that if \var{group} did not contribute to the match, this is
\code{(None, None)}.  Again, \var{group} defaults to zero.
\end{methoddesc}

\begin{memberdesc}[MatchObject]{pos}
The value of \var{pos} which was passed to the
\function{search()} or \function{match()} function.  This is the index into
the string at which the regex engine started looking for a match. 
\end{memberdesc}

\begin{memberdesc}[MatchObject]{endpos}
The value of \var{endpos} which was passed to the
\function{search()} or \function{match()} function.  This is the index into
the string beyond which the regex engine will not go.
\end{memberdesc}

\begin{memberdesc}[MatchObject]{re}
The regular expression object whose \method{match()} or
\method{search()} method produced this \class{MatchObject} instance.
\end{memberdesc}

\begin{memberdesc}[MatchObject]{string}
The string passed to \function{match()} or \function{search()}.
\end{memberdesc}

\begin{seealso}
\seetext{Jeffrey Friedl, \emph{Mastering Regular Expressions},
O'Reilly.  The Python material in this book dates from before the
\module{re} module, but it covers writing good regular expression
patterns in great detail.}
\end{seealso}


\section{Built-in Module \sectcode{regex}}
\label{module-regex}

\bimodindex{regex}
This module provides regular expression matching operations similar to
those found in Emacs.

\strong{Obsolescence note:}
This module is obsolete as of Python version 1.5; it is still being
maintained because much existing code still uses it.  All new code in
need of regular expressions should use the new \code{re} module, which
supports the more powerful and regular Perl-style regular expressions.
Existing code should be converted.  The standard library module
\code{reconvert} helps in converting \code{regex} style regular
expressions to \code{re} style regular expressions.  (For more
conversion help, see the URL
\file{http://starship.skyport.net/crew/amk/regex/regex-to-re.html}.)

By default the patterns are Emacs-style regular expressions
(with one exception).  There is
a way to change the syntax to match that of several well-known
\UNIX{} utilities.  The exception is that Emacs' \samp{\e s}
pattern is not supported, since the original implementation references
the Emacs syntax tables.

This module is 8-bit clean: both patterns and strings may contain null
bytes and characters whose high bit is set.

\strong{Please note:} There is a little-known fact about Python string
literals which means that you don't usually have to worry about
doubling backslashes, even though they are used to escape special
characters in string literals as well as in regular expressions.  This
is because Python doesn't remove backslashes from string literals if
they are followed by an unrecognized escape character.
\emph{However}, if you want to include a literal \dfn{backslash} in a
regular expression represented as a string literal, you have to
\emph{quadruple} it or enclose it in a singleton character class.
E.g.\  to extract \LaTeX\ \samp{\e section\{{\rm
\ldots}\}} headers from a document, you can use this pattern:
\code{'[\e ]section\{\e (.*\e )\}'}.  \emph{Another exception:}
the escape sequece \samp{\e b} is significant in string literals
(where it means the ASCII bell character) as well as in Emacs regular
expressions (where it stands for a word boundary), so in order to
search for a word boundary, you should use the pattern \code{'\e \e b'}.
Similarly, a backslash followed by a digit 0-7 should be doubled to
avoid interpretation as an octal escape.

\subsection{Regular Expressions}

A regular expression (or RE) specifies a set of strings that matches
it; the functions in this module let you check if a particular string
matches a given regular expression (or if a given regular expression
matches a particular string, which comes down to the same thing).

Regular expressions can be concatenated to form new regular
expressions; if \emph{A} and \emph{B} are both regular expressions,
then \emph{AB} is also an regular expression.  If a string \emph{p}
matches A and another string \emph{q} matches B, the string \emph{pq}
will match AB.  Thus, complex expressions can easily be constructed
from simpler ones like the primitives described here.  For details of
the theory and implementation of regular expressions, consult almost
any textbook about compiler construction.

% XXX The reference could be made more specific, say to 
% "Compilers: Principles, Techniques and Tools", by Alfred V. Aho, 
% Ravi Sethi, and Jeffrey D. Ullman, or some FA text.   

A brief explanation of the format of regular expressions follows.

Regular expressions can contain both special and ordinary characters.
Ordinary characters, like '\code{A}', '\code{a}', or '\code{0}', are
the simplest regular expressions; they simply match themselves.  You
can concatenate ordinary characters, so '\code{last}' matches the
characters 'last'.  (In the rest of this section, we'll write RE's in
\code{this special font}, usually without quotes, and strings to be
matched 'in single quotes'.)

Special characters either stand for classes of ordinary characters, or
affect how the regular expressions around them are interpreted.

The special characters are:
\begin{itemize}
\item[\code{.}] (Dot.)  Matches any character except a newline.
\item[\code{\^}] (Caret.)  Matches the start of the string.
\item[\code{\$}] Matches the end of the string.  
\code{foo} matches both 'foo' and 'foobar', while the regular
expression '\code{foo\$}' matches only 'foo'.
\item[\code{*}] Causes the resulting RE to
match 0 or more repetitions of the preceding RE.  \code{ab*} will
match 'a', 'ab', or 'a' followed by any number of 'b's.
\item[\code{+}] Causes the
resulting RE to match 1 or more repetitions of the preceding RE.
\code{ab+} will match 'a' followed by any non-zero number of 'b's; it
will not match just 'a'.
\item[\code{?}] Causes the resulting RE to
match 0 or 1 repetitions of the preceding RE.  \code{ab?} will
match either 'a' or 'ab'.

\item[\code{\e}] Either escapes special characters (permitting you to match
characters like '*?+\&\$'), or signals a special sequence; special
sequences are discussed below.  Remember that Python also uses the
backslash as an escape sequence in string literals; if the escape
sequence isn't recognized by Python's parser, the backslash and
subsequent character are included in the resulting string.  However,
if Python would recognize the resulting sequence, the backslash should
be repeated twice.  

\item[\code{[]}] Used to indicate a set of characters.  Characters can
be listed individually, or a range is indicated by giving two
characters and separating them by a '-'.  Special characters are
not active inside sets.  For example, \code{[akm\$]}
will match any of the characters 'a', 'k', 'm', or '\$'; \code{[a-z]} will
match any lowercase letter.  

If you want to include a \code{]} inside a
set, it must be the first character of the set; to include a \code{-},
place it as the first or last character. 

Characters \emph{not} within a range can be matched by including a
\code{\^} as the first character of the set; \code{\^} elsewhere will
simply match the '\code{\^}' character.  
\end{itemize}

The special sequences consist of '\code{\e}' and a character
from the list below.  If the ordinary character is not on the list,
then the resulting RE will match the second character.  For example,
\code{\e\$} matches the character '\$'.  Ones where the backslash
should be doubled in string literals are indicated.

\begin{itemize}
\item[\code{\e|}]\code{A\e|B}, where A and B can be arbitrary REs,
creates a regular expression that will match either A or B.  This can
be used inside groups (see below) as well.
%
\item[\code{\e( \e)}] Indicates the start and end of a group; the
contents of a group can be matched later in the string with the
\code{\e [1-9]} special sequence, described next.
%
{\fulllineitems\item[\code{\e \e 1, ... \e \e 7, \e 8, \e 9}]
Matches the contents of the group of the same
number.  For example, \code{\e (.+\e ) \e \e 1} matches 'the the' or
'55 55', but not 'the end' (note the space after the group).  This
special sequence can only be used to match one of the first 9 groups;
groups with higher numbers can be matched using the \code{\e v}
sequence.  (\code{\e 8} and \code{\e 9} don't need a double backslash
because they are not octal digits.)}
%
\item[\code{\e \e b}] Matches the empty string, but only at the
beginning or end of a word.  A word is defined as a sequence of
alphanumeric characters, so the end of a word is indicated by
whitespace or a non-alphanumeric character.
%
\item[\code{\e B}] Matches the empty string, but when it is \emph{not} at the
beginning or end of a word.
%
\item[\code{\e v}] Must be followed by a two digit decimal number, and
matches the contents of the group of the same number.  The group number must be between 1 and 99, inclusive.
%
\item[\code{\e w}]Matches any alphanumeric character; this is
equivalent to the set \code{[a-zA-Z0-9]}.
%
\item[\code{\e W}] Matches any non-alphanumeric character; this is
equivalent to the set \code{[\^a-zA-Z0-9]}.
\item[\code{\e <}] Matches the empty string, but only at the beginning of a
word.  A word is defined as a sequence of alphanumeric characters, so
the end of a word is indicated by whitespace or a non-alphanumeric 
character.
\item[\code{\e >}] Matches the empty string, but only at the end of a
word.

\item[\code{\e \e \e \e}] Matches a literal backslash.

% In Emacs, the following two are start of buffer/end of buffer.  In
% Python they seem to be synonyms for ^$.
\item[\code{\e `}] Like \code{\^}, this only matches at the start of the
string.
\item[\code{\e \e '}] Like \code{\$}, this only matches at the end of the
string.
% end of buffer
\end{itemize}

\subsection{Module Contents}

The module defines these functions, and an exception:

\renewcommand{\indexsubitem}{(in module regex)}

\begin{funcdesc}{match}{pattern\, string}
  Return how many characters at the beginning of \var{string} match
  the regular expression \var{pattern}.  Return \code{-1} if the
  string does not match the pattern (this is different from a
  zero-length match!).
\end{funcdesc}

\begin{funcdesc}{search}{pattern\, string}
  Return the first position in \var{string} that matches the regular
  expression \var{pattern}.  Return \code{-1} if no position in the string
  matches the pattern (this is different from a zero-length match
  anywhere!).
\end{funcdesc}

\begin{funcdesc}{compile}{pattern\optional{\, translate}}
  Compile a regular expression pattern into a regular expression
  object, which can be used for matching using its \code{match} and
  \code{search} methods, described below.  The optional argument
  \var{translate}, if present, must be a 256-character string
  indicating how characters (both of the pattern and of the strings to
  be matched) are translated before comparing them; the \code{i}-th
  element of the string gives the translation for the character with
  \ASCII{} code \code{i}.  This can be used to implement
  case-insensitive matching; see the \code{casefold} data item below.

  The sequence

\bcode\begin{verbatim}
prog = regex.compile(pat)
result = prog.match(str)
\end{verbatim}\ecode
%
is equivalent to

\bcode\begin{verbatim}
result = regex.match(pat, str)
\end{verbatim}\ecode
%
but the version using \code{compile()} is more efficient when multiple
regular expressions are used concurrently in a single program.  (The
compiled version of the last pattern passed to \code{regex.match()} or
\code{regex.search()} is cached, so programs that use only a single
regular expression at a time needn't worry about compiling regular
expressions.)
\end{funcdesc}

\begin{funcdesc}{set_syntax}{flags}
  Set the syntax to be used by future calls to \code{compile},
  \code{match} and \code{search}.  (Already compiled expression objects
  are not affected.)  The argument is an integer which is the OR of
  several flag bits.  The return value is the previous value of
  the syntax flags.  Names for the flags are defined in the standard
  module \code{regex_syntax}; read the file \file{regex_syntax.py} for
  more information.
\end{funcdesc}

\begin{funcdesc}{get_syntax}{}
  Returns the current value of the syntax flags as an integer.
\end{funcdesc}

\begin{funcdesc}{symcomp}{pattern\optional{\, translate}}
This is like \code{compile}, but supports symbolic group names: if a
parenthesis-enclosed group begins with a group name in angular
brackets, e.g. \code{'\e(<id>[a-z][a-z0-9]*\e)'}, the group can
be referenced by its name in arguments to the \code{group} method of
the resulting compiled regular expression object, like this:
\code{p.group('id')}.  Group names may contain alphanumeric characters
and \code{'_'} only.
\end{funcdesc}

\begin{excdesc}{error}
  Exception raised when a string passed to one of the functions here
  is not a valid regular expression (e.g., unmatched parentheses) or
  when some other error occurs during compilation or matching.  (It is
  never an error if a string contains no match for a pattern.)
\end{excdesc}

\begin{datadesc}{casefold}
A string suitable to pass as \var{translate} argument to
\code{compile} to map all upper case characters to their lowercase
equivalents.
\end{datadesc}

\noindent
Compiled regular expression objects support these methods:

\renewcommand{\indexsubitem}{(regex method)}
\begin{funcdesc}{match}{string\optional{\, pos}}
  Return how many characters at the beginning of \var{string} match
  the compiled regular expression.  Return \code{-1} if the string
  does not match the pattern (this is different from a zero-length
  match!).
  
  The optional second parameter \var{pos} gives an index in the string
  where the search is to start; it defaults to \code{0}.  This is not
  completely equivalent to slicing the string; the \code{'\^'} pattern
  character matches at the real begin of the string and at positions
  just after a newline, not necessarily at the index where the search
  is to start.
\end{funcdesc}

\begin{funcdesc}{search}{string\optional{\, pos}}
  Return the first position in \var{string} that matches the regular
  expression \code{pattern}.  Return \code{-1} if no position in the
  string matches the pattern (this is different from a zero-length
  match anywhere!).
  
  The optional second parameter has the same meaning as for the
  \code{match} method.
\end{funcdesc}

\begin{funcdesc}{group}{index\, index\, ...}
This method is only valid when the last call to the \code{match}
or \code{search} method found a match.  It returns one or more
groups of the match.  If there is a single \var{index} argument,
the result is a single string; if there are multiple arguments, the
result is a tuple with one item per argument.  If the \var{index} is
zero, the corresponding return value is the entire matching string; if
it is in the inclusive range [1..99], it is the string matching the
the corresponding parenthesized group (using the default syntax,
groups are parenthesized using \code{{\e}(} and \code{{\e})}).  If no
such group exists, the corresponding result is \code{None}.

If the regular expression was compiled by \code{symcomp} instead of
\code{compile}, the \var{index} arguments may also be strings
identifying groups by their group name.
\end{funcdesc}

\noindent
Compiled regular expressions support these data attributes:

\renewcommand{\indexsubitem}{(regex attribute)}

\begin{datadesc}{regs}
When the last call to the \code{match} or \code{search} method found a
match, this is a tuple of pairs of indices corresponding to the
beginning and end of all parenthesized groups in the pattern.  Indices
are relative to the string argument passed to \code{match} or
\code{search}.  The 0-th tuple gives the beginning and end or the
whole pattern.  When the last match or search failed, this is
\code{None}.
\end{datadesc}

\begin{datadesc}{last}
When the last call to the \code{match} or \code{search} method found a
match, this is the string argument passed to that method.  When the
last match or search failed, this is \code{None}.
\end{datadesc}

\begin{datadesc}{translate}
This is the value of the \var{translate} argument to
\code{regex.compile} that created this regular expression object.  If
the \var{translate} argument was omitted in the \code{regex.compile}
call, this is \code{None}.
\end{datadesc}

\begin{datadesc}{givenpat}
The regular expression pattern as passed to \code{compile} or
\code{symcomp}.
\end{datadesc}

\begin{datadesc}{realpat}
The regular expression after stripping the group names for regular
expressions compiled with \code{symcomp}.  Same as \code{givenpat}
otherwise.
\end{datadesc}

\begin{datadesc}{groupindex}
A dictionary giving the mapping from symbolic group names to numerical
group indices for regular expressions compiled with \code{symcomp}.
\code{None} otherwise.
\end{datadesc}

\section{\module{regsub} ---
         String operations using regular expressions}

\declaremodule{standard}{regsub}
\modulesynopsis{Substitution and splitting operations that use
                regular expressions.  \strong{Obsolete!}}


This module defines a number of functions useful for working with
regular expressions (see built-in module \refmodule{regex}).

Warning: these functions are not thread-safe.

\strong{Obsolescence note:}
This module is obsolete as of Python version 1.5; it is still being
maintained because much existing code still uses it.  All new code in
need of regular expressions should use the new \refmodule{re} module, which
supports the more powerful and regular Perl-style regular expressions.
Existing code should be converted.  The standard library module
\module{reconvert} helps in converting \refmodule{regex} style regular
expressions to \refmodule{re} style regular expressions.  (For more
conversion help, see Andrew Kuchling's\index{Kuchling, Andrew}
``regex-to-re HOWTO'' at
\url{http://www.python.org/doc/howto/regex-to-re/}.)


\begin{funcdesc}{sub}{pat, repl, str}
Replace the first occurrence of pattern \var{pat} in string
\var{str} by replacement \var{repl}.  If the pattern isn't found,
the string is returned unchanged.  The pattern may be a string or an
already compiled pattern.  The replacement may contain references
\samp{\e \var{digit}} to subpatterns and escaped backslashes.
\end{funcdesc}

\begin{funcdesc}{gsub}{pat, repl, str}
Replace all (non-overlapping) occurrences of pattern \var{pat} in
string \var{str} by replacement \var{repl}.  The same rules as for
\code{sub()} apply.  Empty matches for the pattern are replaced only
when not adjacent to a previous match, so e.g.
\code{gsub('', '-', 'abc')} returns \code{'-a-b-c-'}.
\end{funcdesc}

\begin{funcdesc}{split}{str, pat\optional{, maxsplit}}
Split the string \var{str} in fields separated by delimiters matching
the pattern \var{pat}, and return a list containing the fields.  Only
non-empty matches for the pattern are considered, so e.g.
\code{split('a:b', ':*')} returns \code{['a', 'b']} and
\code{split('abc', '')} returns \code{['abc']}.  The \var{maxsplit}
defaults to 0. If it is nonzero, only \var{maxsplit} number of splits
occur, and the remainder of the string is returned as the final
element of the list.
\end{funcdesc}

\begin{funcdesc}{splitx}{str, pat\optional{, maxsplit}}
Split the string \var{str} in fields separated by delimiters matching
the pattern \var{pat}, and return a list containing the fields as well
as the separators.  For example, \code{splitx('a:::b', ':*')} returns
\code{['a', ':::', 'b']}.  Otherwise, this function behaves the same
as \code{split}.
\end{funcdesc}

\begin{funcdesc}{capwords}{s\optional{, pat}}
Capitalize words separated by optional pattern \var{pat}.  The default
pattern uses any characters except letters, digits and underscores as
word delimiters.  Capitalization is done by changing the first
character of each word to upper case.
\end{funcdesc}

\begin{funcdesc}{clear_cache}{}
The regsub module maintains a cache of compiled regular expressions,
keyed on the regular expression string and the syntax of the regex
module at the time the expression was compiled.  This function clears
that cache.
\end{funcdesc}

\section{\module{struct} ---
         Interpret strings as packed binary data}
\declaremodule{builtin}{struct}

\modulesynopsis{Interpret strings as packed binary data.}

\indexii{C}{structures}
\indexiii{packing}{binary}{data}

This module performs conversions between Python values and C
structs represented as Python strings.  It uses \dfn{format strings}
(explained below) as compact descriptions of the lay-out of the C
structs and the intended conversion to/from Python values.  This can
be used in handling binary data stored in files or from network
connections, among other sources.

The module defines the following exception and functions:


\begin{excdesc}{error}
  Exception raised on various occasions; argument is a string
  describing what is wrong.
\end{excdesc}

\begin{funcdesc}{pack}{fmt, v1, v2, \textrm{\ldots}}
  Return a string containing the values
  \code{\var{v1}, \var{v2}, \textrm{\ldots}} packed according to the given
  format.  The arguments must match the values required by the format
  exactly.
\end{funcdesc}

\begin{funcdesc}{unpack}{fmt, string}
  Unpack the string (presumably packed by \code{pack(\var{fmt},
  \textrm{\ldots})}) according to the given format.  The result is a
  tuple even if it contains exactly one item.  The string must contain
  exactly the amount of data required by the format
  (\code{len(\var{string})} must equal \code{calcsize(\var{fmt})}).
\end{funcdesc}

\begin{funcdesc}{calcsize}{fmt}
  Return the size of the struct (and hence of the string)
  corresponding to the given format.
\end{funcdesc}

Format characters have the following meaning; the conversion between
C and Python values should be obvious given their types:

\begin{tableiv}{c|l|l|c}{samp}{Format}{C Type}{Python}{Notes}
  \lineiv{x}{pad byte}{no value}{}
  \lineiv{c}{\ctype{char}}{string of length 1}{}
  \lineiv{b}{\ctype{signed char}}{integer}{}
  \lineiv{B}{\ctype{unsigned char}}{integer}{}
  \lineiv{h}{\ctype{short}}{integer}{}
  \lineiv{H}{\ctype{unsigned short}}{integer}{}
  \lineiv{i}{\ctype{int}}{integer}{}
  \lineiv{I}{\ctype{unsigned int}}{long}{}
  \lineiv{l}{\ctype{long}}{integer}{}
  \lineiv{L}{\ctype{unsigned long}}{long}{}
  \lineiv{q}{\ctype{long long}}{long}{(1)}
  \lineiv{Q}{\ctype{unsigned long long}}{long}{(1)}
  \lineiv{f}{\ctype{float}}{float}{}
  \lineiv{d}{\ctype{double}}{float}{}
  \lineiv{s}{\ctype{char[]}}{string}{}
  \lineiv{p}{\ctype{char[]}}{string}{}
  \lineiv{P}{\ctype{void *}}{integer}{}
\end{tableiv}

\noindent
Notes:

\begin{description}
\item[(1)]
  The \character{q} and \character{Q} conversion codes are available in
  native mode only if the platform C compiler supports C \ctype{long long},
  or, on Windows, \ctype{__int64}.  They are always available in standard
  modes.
  \versionadded{2.2}
\end{description}


A format character may be preceded by an integral repeat count.  For
example, the format string \code{'4h'} means exactly the same as
\code{'hhhh'}.

Whitespace characters between formats are ignored; a count and its
format must not contain whitespace though.

For the \character{s} format character, the count is interpreted as the
size of the string, not a repeat count like for the other format
characters; for example, \code{'10s'} means a single 10-byte string, while
\code{'10c'} means 10 characters.  For packing, the string is
truncated or padded with null bytes as appropriate to make it fit.
For unpacking, the resulting string always has exactly the specified
number of bytes.  As a special case, \code{'0s'} means a single, empty
string (while \code{'0c'} means 0 characters).

The \character{p} format character can be used to encode a Pascal
string.  The first byte is the length of the stored string, with the
bytes of the string following.  If count is given, it is used as the
total number of bytes used, including the length byte.  If the string
passed in to \function{pack()} is too long, the stored representation
is truncated.  If the string is too short, padding is used to ensure
that exactly enough bytes are used to satisfy the count.

For the \character{I}, \character{L}, \character{q} and \character{Q}
format characters, the return value is a Python long integer.

For the \character{P} format character, the return value is a Python
integer or long integer, depending on the size needed to hold a
pointer when it has been cast to an integer type.  A \NULL{} pointer will
always be returned as the Python integer \code{0}. When packing pointer-sized
values, Python integer or long integer objects may be used.  For
example, the Alpha and Merced processors use 64-bit pointer values,
meaning a Python long integer will be used to hold the pointer; other
platforms use 32-bit pointers and will use a Python integer.

By default, C numbers are represented in the machine's native format
and byte order, and properly aligned by skipping pad bytes if
necessary (according to the rules used by the C compiler).

Alternatively, the first character of the format string can be used to
indicate the byte order, size and alignment of the packed data,
according to the following table:

\begin{tableiii}{c|l|l}{samp}{Character}{Byte order}{Size and alignment}
  \lineiii{@}{native}{native}
  \lineiii{=}{native}{standard}
  \lineiii{<}{little-endian}{standard}
  \lineiii{>}{big-endian}{standard}
  \lineiii{!}{network (= big-endian)}{standard}
\end{tableiii}

If the first character is not one of these, \character{@} is assumed.

Native byte order is big-endian or little-endian, depending on the
host system.  For example, Motorola and Sun processors are big-endian;
Intel and DEC processors are little-endian.

Native size and alignment are determined using the C compiler's
\keyword{sizeof} expression.  This is always combined with native byte
order.

Standard size and alignment are as follows: no alignment is required
for any type (so you have to use pad bytes);
\ctype{short} is 2 bytes;
\ctype{int} and \ctype{long} are 4 bytes;
\ctype{long long} (\ctype{__int64} on Windows) is 8 bytes;
\ctype{float} and \ctype{double} are 32-bit and 64-bit
IEEE floating point numbers, respectively.

Note the difference between \character{@} and \character{=}: both use
native byte order, but the size and alignment of the latter is
standardized.

The form \character{!} is available for those poor souls who claim they
can't remember whether network byte order is big-endian or
little-endian.

There is no way to indicate non-native byte order (force
byte-swapping); use the appropriate choice of \character{<} or
\character{>}.

The \character{P} format character is only available for the native
byte ordering (selected as the default or with the \character{@} byte
order character). The byte order character \character{=} chooses to
use little- or big-endian ordering based on the host system. The
struct module does not interpret this as native ordering, so the
\character{P} format is not available.

Examples (all using native byte order, size and alignment, on a
big-endian machine):

\begin{verbatim}
>>> from struct import *
>>> pack('hhl', 1, 2, 3)
'\x00\x01\x00\x02\x00\x00\x00\x03'
>>> unpack('hhl', '\x00\x01\x00\x02\x00\x00\x00\x03')
(1, 2, 3)
>>> calcsize('hhl')
8
\end{verbatim}

Hint: to align the end of a structure to the alignment requirement of
a particular type, end the format with the code for that type with a
repeat count of zero.  For example, the format \code{'llh0l'}
specifies two pad bytes at the end, assuming longs are aligned on
4-byte boundaries.  This only works when native size and alignment are
in effect; standard size and alignment does not enforce any alignment.

\begin{seealso}
  \seemodule{array}{Packed binary storage of homogeneous data.}
  \seemodule{xdrlib}{Packing and unpacking of XDR data.}
\end{seealso}

\section{\module{fpformat} ---
         Floating point conversions}

\declaremodule{standard}{fpformat}
\sectionauthor{Moshe Zadka}{moshez@zadka.site.co.il}
\modulesynopsis{General floating point formatting functions.}


The \module{fpformat} module defines functions for dealing with
floating point numbers representations in 100\% pure
Python. \note{This module is unneeded: everything here could
be done via the \code{\%} string interpolation operator.}

The \module{fpformat} module defines the following functions and an
exception:


\begin{funcdesc}{fix}{x, digs}
Format \var{x} as \code{[-]ddd.ddd} with \var{digs} digits after the
point and at least one digit before.
If \code{\var{digs} <= 0}, the decimal point is suppressed.

\var{x} can be either a number or a string that looks like
one. \var{digs} is an integer.

Return value is a string.
\end{funcdesc}

\begin{funcdesc}{sci}{x, digs}
Format \var{x} as \code{[-]d.dddE[+-]ddd} with \var{digs} digits after the 
point and exactly one digit before.
If \code{\var{digs} <= 0}, one digit is kept and the point is suppressed.

\var{x} can be either a real number, or a string that looks like
one. \var{digs} is an integer.

Return value is a string.
\end{funcdesc}

\begin{excdesc}{NotANumber}
Exception raised when a string passed to \function{fix()} or
\function{sci()} as the \var{x} parameter does not look like a number.
This is a subclass of \exception{ValueError} when the standard
exceptions are strings.  The exception value is the improperly
formatted string that caused the exception to be raised.
\end{excdesc}

Example:

\begin{verbatim}
>>> import fpformat
>>> fpformat.fix(1.23, 1)
'1.2'
\end{verbatim}

\section{\module{StringIO} ---
         Read and write strings as if they were files.}
\declaremodule{standard}{StringIO}


\modulesynopsis{Read and write strings as if they were files.}


This module implements a file-like class, \class{StringIO},
that reads and writes a string buffer (also known as \emph{memory
files}). See the description on file objects for operations.

\begin{classdesc}{StringIO}{\optional{buffer}}
When a \class{StringIO} object is created, it can be initialized
to an existing string by passing the string to the constructor.
If no string is given, the \class{StringIO} will start empty.
\end{classdesc}

The following methods of \class{StringIO} objects require special
mention:

\begin{methoddesc}{getvalue}{}
Retrieve the entire contents of the ``file'' at any time before the
\class{StringIO} object's \method{close()} method is called.
\end{methoddesc}

\begin{methoddesc}{close}{}
Free the memory buffer.
\end{methoddesc}


\section{\module{cStringIO} ---
         Faster version of \module{StringIO}, but not subclassable.}

\declaremodule{builtin}{cStringIO}
\modulesynopsis{Faster version of \module{StringIO}, but not subclassable.}
\sectionauthor{Fred L. Drake, Jr.}{fdrake@acm.org}

The module \module{cStringIO} provides an interface similar to that of
the \module{StringIO} module.  Heavy use of \class{StringIO.StringIO}
objects can be made more efficient by using the function
\function{StringIO()} from this module instead.

Since this module provides a factory function which returns objects of
built-in types, there's no way to build your own version using
subclassing.  Use the original \module{StringIO} module in that case.

%\section{Built-in Module \sectcode{soundex}}
\label{module-soundex}
\bimodindex{soundex}

\setindexsubitem{(in module soundex)}
The soundex algorithm takes an English word, and returns an
easily-computed hash of it; this hash is intended to be the same for
words that sound alike.  This module provides an interface to the
soundex algorithm.

Note that the soundex algorithm is quite simple-minded, and isn't
perfect by any measure.  Its main purpose is to help looking up names
in databases, when the name may be misspelled --- soundex hashes common
misspellings together.

\begin{funcdesc}{get_soundex}{string}
Return the soundex hash value for a word; it will always be a
6-character string.  \var{string} must contain the word to be hashed,
with no leading whitespace; the case of the word is ignored.
\end{funcdesc}

\begin{funcdesc}{sound_similar}{string1, string2}
Compare the word in \var{string1} with the word in \var{string2}; this
is equivalent to 
\code{get_soundex(\var{string1})} \code{==}
\code{get_soundex(\var{string2})}.
\end{funcdesc}


\chapter{Miscellaneous Services}
\label{misc}

The modules described in this chapter provide miscellaneous services
that are available in all Python versions.  Here's an overview:

\begin{description}

\item[math]
--- Mathematical functions (\function{sin()} etc.).

\item[cmath]
--- Mathematical functions for complex numbers.

\item[whrandom]
--- Floating point pseudo-random number generator.

\item[random]
--- Generate pseudo-random numbers with various common distributions.

\item[array]
--- Efficient arrays of uniformly typed numeric values.

\item[fileinput]
--- Perl-like iteration over lines from multiple input streams, with
``save in place'' capability.

\end{description}
			% Miscellaneous Services
\section{Built-in Module \sectcode{math}}
\label{module-math}

\bimodindex{math}
\renewcommand{\indexsubitem}{(in module math)}
This module is always available.
It provides access to the mathematical functions defined by the C
standard.
They are:

\begin{funcdesc}{acos}{x}
Return the arc cosine of \var{x}.
\end{funcdesc}

\begin{funcdesc}{asin}{x}
Return the arc sine of \var{x}.
\end{funcdesc}

\begin{funcdesc}{atan}{x}
Return the arc tangent of \var{x}.
\end{funcdesc}

\begin{funcdesc}{atan2}{x, y}
Return \code{atan(x / y)}.
\end{funcdesc}

\begin{funcdesc}{ceil}{x}
Return the ceiling of \var{x}.
\end{funcdesc}

\begin{funcdesc}{cos}{x}
Return the cosine of \var{x}.
\end{funcdesc}

\begin{funcdesc}{cosh}{x}
Return the hyperbolic cosine of \var{x}.
\end{funcdesc}

\begin{funcdesc}{exp}{x}
Return the exponential value $\mbox{e}^x$.
\end{funcdesc}

\begin{funcdesc}{fabs}{x}
Return the absolute value of the real \var{x}.
\end{funcdesc}

\begin{funcdesc}{floor}{x}
Return the floor of \var{x}.
\end{funcdesc}

\begin{funcdesc}{fmod}{x, y}
Return \code{x \% y}.
\end{funcdesc}

\begin{funcdesc}{frexp}{x}
Return the matissa and exponent for \var{x}.  The mantissa is
positive.
\end{funcdesc}

\begin{funcdesc}{hypot}{x, y}
Return the Euclidean distance, \code{sqrt(x*x + y*y)}.
\end{funcdesc}

\begin{funcdesc}{ldexp}{x, i}
Return $x {\times} 2^i$.
\end{funcdesc}

\begin{funcdesc}{modf}{x}
Return the fractional and integer parts of \var{x}.  Both results
carry the sign of \var{x}.
\end{funcdesc}

\begin{funcdesc}{pow}{x, y}
Return $x^y$.
\end{funcdesc}

\begin{funcdesc}{sin}{x}
Return the sine of \var{x}.
\end{funcdesc}

\begin{funcdesc}{sinh}{x}
Return the hyperbolic sine of \var{x}.
\end{funcdesc}

\begin{funcdesc}{sqrt}{x}
Return the square root of \var{x}.
\end{funcdesc}

\begin{funcdesc}{tan}{x}
Return the tangent of \var{x}.
\end{funcdesc}

\begin{funcdesc}{tanh}{x}
Return the hyperbolic tangent of \var{x}.
\end{funcdesc}

Note that \code{frexp} and \code{modf} have a different call/return
pattern than their C equivalents: they take a single argument and
return a pair of values, rather than returning their second return
value through an `output parameter' (there is no such thing in Python).

The module also defines two mathematical constants:

\begin{datadesc}{pi}
The mathematical constant \emph{pi}.
\end{datadesc}

\begin{datadesc}{e}
The mathematical constant \emph{e}.
\end{datadesc}

\begin{seealso}
  \seemodule{cmath}{Complex number versions of many of these functions.}
\end{seealso}

\section{Built-in Module \sectcode{cmath}}
\label{module-cmath}

\bimodindex{cmath}
This module is always available.
It provides access to mathematical functions for complex numbers.
The functions are:

\begin{funcdesc}{acos}{x}
Return the arc cosine of \var{x}.
\end{funcdesc}

\begin{funcdesc}{acosh}{x}
Return the hyperbolic arc cosine of \var{x}.
\end{funcdesc}

\begin{funcdesc}{asin}{x}
Return the arc sine of \var{x}.
\end{funcdesc}

\begin{funcdesc}{asinh}{x}
Return the hyperbolic arc sine of \var{x}.
\end{funcdesc}

\begin{funcdesc}{atan}{x}
Return the arc tangent of \var{x}.
\end{funcdesc}

\begin{funcdesc}{atanh}{x}
Return the hyperbolic arc tangent of \var{x}.
\end{funcdesc}

\begin{funcdesc}{cos}{x}
Return the cosine of \var{x}.
\end{funcdesc}

\begin{funcdesc}{cosh}{x}
Return the hyperbolic cosine of \var{x}.
\end{funcdesc}

\begin{funcdesc}{exp}{x}
Return the exponential value \code{e**\var{x}}.
\end{funcdesc}

\begin{funcdesc}{log}{x}
Return the natural logarithm of \var{x}.
\end{funcdesc}

\begin{funcdesc}{log10}{x}
Return the base-10 logarithm of \var{x}.
\end{funcdesc}

\begin{funcdesc}{sin}{x}
Return the sine of \var{x}.
\end{funcdesc}

\begin{funcdesc}{sinh}{x}
Return the hyperbolic sine of \var{x}.
\end{funcdesc}

\begin{funcdesc}{sqrt}{x}
Return the square root of \var{x}.
\end{funcdesc}

\begin{funcdesc}{tan}{x}
Return the tangent of \var{x}.
\end{funcdesc}

\begin{funcdesc}{tanh}{x}
Return the hyperbolic tangent of \var{x}.
\end{funcdesc}

The module also defines two mathematical constants:

\begin{datadesc}{pi}
The mathematical constant \emph{pi}, as a real.
\end{datadesc}

\begin{datadesc}{e}
The mathematical constant \emph{e}, as a real.
\end{datadesc}

Note that the selection of functions is similar, but not identical, to
that in module \code{math}\refbimodindex{math}.  The reason for having
two modules is, that some users aren't interested in complex numbers,
and perhaps don't even know what they are.  They would rather have
\code{math.sqrt(-1)} raise an exception than return a complex number.
Also note that the functions defined in \code{cmath} always return a
complex number, even if the answer can be expressed as a real number
(in which case the complex number has an imaginary part of zero).

\section{Standard Module \sectcode{random}}
\label{module-random}
\stmodindex{random}

This module implements pseudo-random number generators for various
distributions: on the real line, there are functions to compute normal
or Gaussian, lognormal, negative exponential, gamma, and beta
distributions.  For generating distribution of angles, the circular
uniform and von Mises distributions are available.

The module exports the following functions, which are exactly
equivalent to those in the \code{whrandom} module: \code{choice},
\code{randint}, \code{random}, \code{uniform}.  See the documentation
for the \code{whrandom} module for these functions.

The following functions specific to the \code{random} module are also
defined, and all return real values.  Function parameters are named
after the corresponding variables in the distribution's equation, as
used in common mathematical practice; most of these equations can be
found in any statistics text.

\renewcommand{\indexsubitem}{(in module random)}
\begin{funcdesc}{betavariate}{alpha\, beta}
Beta distribution.  Conditions on the parameters are \code{alpha>-1}
and \code{beta>-1}.
Returned values will range between 0 and 1.
\end{funcdesc}

\begin{funcdesc}{cunifvariate}{mean\, arc}
Circular uniform distribution.  \var{mean} is the mean angle, and
\var{arc} is the range of the distribution, centered around the mean
angle.  Both values must be expressed in radians, and can range
between 0 and \code{pi}.  Returned values will range between
\code{mean - arc/2} and \code{mean + arc/2}.
\end{funcdesc}

\begin{funcdesc}{expovariate}{lambd}
Exponential distribution.  \var{lambd} is 1.0 divided by the desired mean.
(The parameter would be called ``lambda'', but that's also a reserved
word in Python.)  Returned values will range from 0 to positive infinity.
\end{funcdesc}

\begin{funcdesc}{gamma}{alpha\, beta}
Gamma distribution.  (\emph{Not} the gamma function!) 
Conditions on the parameters are \code{alpha>-1} and \code{beta>0}.
\end{funcdesc}

\begin{funcdesc}{gauss}{mu\, sigma}
Gaussian distribution.  \var{mu} is the mean, and \var{sigma} is the
standard deviation.  This is slightly faster than the
\code{normalvariate} function defined below.
\end{funcdesc}

\begin{funcdesc}{lognormvariate}{mu\, sigma}
Log normal distribution.  If you take the natural logarithm of this
distribution, you'll get a normal distribution with mean \var{mu} and
standard deviation \var{sigma}  \var{mu} can have any value, and \var{sigma}
must be greater than zero.  
\end{funcdesc}

\begin{funcdesc}{normalvariate}{mu\, sigma}
Normal distribution.  \var{mu} is the mean, and \var{sigma} is the
standard deviation.
\end{funcdesc}

\begin{funcdesc}{vonmisesvariate}{mu\, kappa}
\var{mu} is the mean angle, expressed in radians between 0 and pi,
and \var{kappa} is the concentration parameter, which must be greater
then or equal to zero.  If \var{kappa} is equal to zero, this
distribution reduces to a uniform random angle over the range 0 to
\code{2*pi}.
\end{funcdesc}


\begin{seealso}
\seemodule{whrandom}{the standard Python random number generator}
\end{seealso}

\section{\module{whrandom} ---
         Pseudo-random number generator}

\declaremodule{standard}{whrandom}
\modulesynopsis{Floating point pseudo-random number generator.}

\deprecated{2.1}{Use \refmodule{random} instead.}

\note{This module was an implementation detail of the
\refmodule{random} module in releases of Python prior to 2.1.  It is
no longer used.  Please do not use this module directly; use
\refmodule{random} instead.}

This module implements a Wichmann-Hill pseudo-random number generator
class that is also named \class{whrandom}.  Instances of the
\class{whrandom} class conform to the Random Number Generator
interface described in the docs for the \refmodule{random} module.
They also offer the 
following method, specific to the Wichmann-Hill algorithm:

\begin{methoddesc}[whrandom]{seed}{\optional{x, y, z}}
  Initializes the random number generator from the integers \var{x},
  \var{y} and \var{z}.  When the module is first imported, the random
  number is initialized using values derived from the current time.
  If \var{x}, \var{y}, and \var{z} are either omitted or \code{0}, the 
  seed will be computed from the current system time.  If one or two
  of the parameters are \code{0}, but not all three, the zero values
  are replaced by ones.  This causes some apparently different seeds
  to be equal, with the corresponding result on the pseudo-random
  series produced by the generator.
\end{methoddesc}

Other supported methods include:

\begin{funcdesc}{choice}{seq}
Chooses a random element from the non-empty sequence \var{seq} and returns it.
\end{funcdesc}

\begin{funcdesc}{randint}{a, b}
Returns a random integer \var{N} such that \code{\var{a}<=\var{N}<=\var{b}}.
\end{funcdesc}

\begin{funcdesc}{random}{}
Returns the next random floating point number in the range [0.0 ... 1.0).
\end{funcdesc}

\begin{funcdesc}{seed}{x, y, z}
Initializes the random number generator from the integers \var{x},
\var{y} and \var{z}.  When the module is first imported, the random
number is initialized using values derived from the current time.
\end{funcdesc}

\begin{funcdesc}{uniform}{a, b}
Returns a random real number \var{N} such that \code{\var{a}<=\var{N}<\var{b}}.
\end{funcdesc}

When imported, the \module{whrandom} module also creates an instance of
the \class{whrandom} class, and makes the methods of that instance
available at the module level.  Therefore one can write either 
\code{N = whrandom.random()} or:

\begin{verbatim}
generator = whrandom.whrandom()
N = generator.random()
\end{verbatim}

Note that using separate instances of the generator leads to
independent sequences of pseudo-random numbers.

\begin{seealso}
  \seemodule{random}{Generators for various random distributions and
                     documentation for the Random Number Generator
                     interface.}
  \seetext{Wichmann, B. A. \& Hill, I. D., ``Algorithm AS 183: 
           An efficient and portable pseudo-random number generator'',
           \citetitle{Applied Statistics} 31 (1982) 188-190.}
\end{seealso}

%\section{Standard Module \sectcode{rand}}
\label{module-rand}
\stmodindex{rand}

The \code{rand} module simulates the C library's \code{rand()}
interface, though the results aren't necessarily compatible with any
given library's implementation.  While still supported for
compatibility, the \code{rand} module is now considered obsolete; if
possible, use the \code{whrandom} module instead.

\begin{funcdesc}{choice}{seq}
Returns a random element from the sequence \var{seq}.
\end{funcdesc}

\begin{funcdesc}{rand}{}
Return a random integer between 0 and 32767, inclusive.
\end{funcdesc}

\begin{funcdesc}{srand}{seed}
Set a starting seed value for the random number generator; \var{seed}
can be an arbitrary integer. 
\end{funcdesc}

\begin{seealso}
\seemodule{whrandom}{the standard Python random number generator}
\end{seealso}


\section{\module{bisect} ---
         Array bisection algorithm}

\declaremodule{standard}{bisect}
\modulesynopsis{Array bisection algorithms for binary searching.}
\sectionauthor{Fred L. Drake, Jr.}{fdrake@acm.org}
% LaTeX produced by Fred L. Drake, Jr. <fdrake@acm.org>, with an
% example based on the PyModules FAQ entry by Aaron Watters
% <arw@pythonpros.com>.


This module provides support for maintaining a list in sorted order
without having to sort the list after each insertion.  For long lists
of items with expensive comparison operations, this can be an
improvement over the more common approach.  The module is called
\module{bisect} because it uses a basic bisection algorithm to do its
work.  The source code may be most useful as a working example of the
algorithm (the boundary conditions are already right!).

The following functions are provided:

\begin{funcdesc}{bisect_left}{list, item\optional{, lo\optional{, hi}}}
  Locate the proper insertion point for \var{item} in \var{list} to
  maintain sorted order.  The parameters \var{lo} and \var{hi} may be
  used to specify a subset of the list which should be considered; by
  default the entire list is used.  If \var{item} is already present
  in \var{list}, the insertion point will be before (to the left of)
  any existing entries.  The return value is suitable for use as the
  first parameter to \code{\var{list}.insert()}.  This assumes that
  \var{list} is already sorted.
\versionadded{2.1}
\end{funcdesc}

\begin{funcdesc}{bisect_right}{list, item\optional{, lo\optional{, hi}}}
  Similar to \function{bisect_left()}, but returns an insertion point
  which comes after (to the right of) any existing entries of
  \var{item} in \var{list}.
\versionadded{2.1}
\end{funcdesc}

\begin{funcdesc}{bisect}{\unspecified}
  Alias for \function{bisect_right()}.
\end{funcdesc}

\begin{funcdesc}{insort_left}{list, item\optional{, lo\optional{, hi}}}
  Insert \var{item} in \var{list} in sorted order.  This is equivalent
  to \code{\var{list}.insert(bisect.bisect_left(\var{list}, \var{item},
  \var{lo}, \var{hi}), \var{item})}.  This assumes that \var{list} is
  already sorted.
\versionadded{2.1}
\end{funcdesc}

\begin{funcdesc}{insort_right}{list, item\optional{, lo\optional{, hi}}}
  Similar to \function{insort_left()}, but inserting \var{item} in
  \var{list} after any existing entries of \var{item}.
\versionadded{2.1}
\end{funcdesc}

\begin{funcdesc}{insort}{\unspecified}
  Alias for \function{insort_right()}.
\end{funcdesc}


\subsection{Examples}
\nodename{bisect-example}

The \function{bisect()} function is generally useful for categorizing
numeric data.  This example uses \function{bisect()} to look up a
letter grade for an exam total (say) based on a set of ordered numeric
breakpoints: 85 and up is an `A', 75..84 is a `B', etc.

\begin{verbatim}
>>> grades = "FEDCBA"
>>> breakpoints = [30, 44, 66, 75, 85]
>>> from bisect import bisect
>>> def grade(total):
...           return grades[bisect(breakpoints, total)]
...
>>> grade(66)
'C'
>>> map(grade, [33, 99, 77, 44, 12, 88])
['E', 'A', 'B', 'D', 'F', 'A']
\end{verbatim}

The bisect module can be used with the Queue module to implement a priority
queue (example courtesy of Fredrik Lundh): \index{Priority Queue}

\begin{verbatim}
import Queue, bisect

class PriorityQueue(Queue.Queue):
    def _put(self, item):
        bisect.insort(self.queue, item)

# usage
queue = PriorityQueue(0)
queue.put((2, "second"))
queue.put((1, "first"))
queue.put((3, "third"))
priority, value = queue.get()
\end{verbatim}

\section{\module{array} ---
         Efficient arrays of numeric values}

\declaremodule{builtin}{array}
\modulesynopsis{Efficient arrays of uniformly typed numeric values.}


This module defines a new object type which can efficiently represent
an array of basic values: characters, integers, floating point
numbers.  Arrays\index{arrays} are sequence types and behave very much
like lists, except that the type of objects stored in them is
constrained.  The type is specified at object creation time by using a
\dfn{type code}, which is a single character.  The following type
codes are defined:

\begin{tableiii}{c|l|c}{code}{Type code}{C Type}{Minimum size in bytes}
\lineiii{'c'}{character}{1}
\lineiii{'b'}{signed int}{1}
\lineiii{'B'}{unsigned int}{1}
\lineiii{'h'}{signed int}{2}
\lineiii{'H'}{unsigned int}{2}
\lineiii{'i'}{signed int}{2}
\lineiii{'I'}{unsigned int}{2}
\lineiii{'l'}{signed int}{4}
\lineiii{'L'}{unsigned int}{4}
\lineiii{'f'}{float}{4}
\lineiii{'d'}{double}{8}
\end{tableiii}

The actual representation of values is determined by the machine
architecture (strictly speaking, by the C implementation).  The actual
size can be accessed through the \member{itemsize} attribute.  The values
stored  for \code{'L'} and \code{'I'} items will be represented as
Python long integers when retrieved, because Python's plain integer
type cannot represent the full range of C's unsigned (long) integers.


The module defines the following function and type object:

\begin{funcdesc}{array}{typecode\optional{, initializer}}
Return a new array whose items are restricted by \var{typecode}, and
initialized from the optional \var{initializer} value, which must be a
list or a string.  The list or string is passed to the new array's
\method{fromlist()} or \method{fromstring()} method (see below) to add
initial items to the array.
\end{funcdesc}

\begin{datadesc}{ArrayType}
Type object corresponding to the objects returned by
\function{array()}.
\end{datadesc}


Array objects support the following data items and methods:

\begin{memberdesc}[array]{typecode}
The typecode character used to create the array.
\end{memberdesc}

\begin{memberdesc}[array]{itemsize}
The length in bytes of one array item in the internal representation.
\end{memberdesc}


\begin{methoddesc}[array]{append}{x}
Append a new item with value \var{x} to the end of the array.
\end{methoddesc}

\begin{methoddesc}[array]{buffer_info}{}
Return a tuple \code{(\var{address}, \var{length})} giving the current
memory address and the length in bytes of the buffer used to hold
array's contents.  This is occasionally useful when working with
low-level (and inherently unsafe) I/O interfaces that require memory
addresses, such as certain \cfunction{ioctl()} operations.  The returned
numbers are valid as long as the array exists and no length-changing
operations are applied to it.
\end{methoddesc}

\begin{methoddesc}[array]{byteswap}{x}
``Byteswap'' all items of the array.  This is only supported for
integer values.  It is useful when reading data from a file written
on a machine with a different byte order.
\end{methoddesc}

\begin{methoddesc}[array]{fromfile}{f, n}
Read \var{n} items (as machine values) from the file object \var{f}
and append them to the end of the array.  If less than \var{n} items
are available, \exception{EOFError} is raised, but the items that were
available are still inserted into the array.  \var{f} must be a real
built-in file object; something else with a \method{read()} method won't
do.
\end{methoddesc}

\begin{methoddesc}[array]{fromlist}{list}
Append items from the list.  This is equivalent to
\samp{for x in \var{list}:\ a.append(x)}
except that if there is a type error, the array is unchanged.
\end{methoddesc}

\begin{methoddesc}[array]{fromstring}{s}
Appends items from the string, interpreting the string as an
array of machine values (i.e. as if it had been read from a
file using the \method{fromfile()} method).
\end{methoddesc}

\begin{methoddesc}[array]{insert}{i, x}
Insert a new item with value \var{x} in the array before position
\var{i}.
\end{methoddesc}

\begin{methoddesc}[array]{read}{f, n}
\deprecated {1.5.1}
  {Use the \method{fromfile()} method.}
Read \var{n} items (as machine values) from the file object \var{f}
and append them to the end of the array.  If less than \var{n} items
are available, \exception{EOFError} is raised, but the items that were
available are still inserted into the array.  \var{f} must be a real
built-in file object; something else with a \method{read()} method won't
do.
\end{methoddesc}

\begin{methoddesc}[array]{reverse}{}
Reverse the order of the items in the array.
\end{methoddesc}

\begin{methoddesc}[array]{tofile}{f}
Write all items (as machine values) to the file object \var{f}.
\end{methoddesc}

\begin{methoddesc}[array]{tolist}{}
Convert the array to an ordinary list with the same items.
\end{methoddesc}

\begin{methoddesc}[array]{tostring}{}
Convert the array to an array of machine values and return the
string representation (the same sequence of bytes that would
be written to a file by the \method{tofile()} method.)
\end{methoddesc}

\begin{methoddesc}[array]{write}{f}
\deprecated {1.5.1}
  {Use the \method{tofile()} method.}
Write all items (as machine values) to the file object \var{f}.
\end{methoddesc}

When an array object is printed or converted to a string, it is
represented as \code{array(\var{typecode}, \var{initializer})}.  The
\var{initializer} is omitted if the array is empty, otherwise it is a
string if the \var{typecode} is \code{'c'}, otherwise it is a list of
numbers.  The string is guaranteed to be able to be converted back to
an array with the same type and value using reverse quotes
(\code{``}).  Examples:

\begin{verbatim}
array('l')
array('c', 'hello world')
array('l', [1, 2, 3, 4, 5])
array('d', [1.0, 2.0, 3.14])
\end{verbatim}


\begin{seealso}
  \seemodule{struct}{packing and unpacking of heterogeneous binary data}
  \seemodule{xdrlib{{packing and unpacking of XDR data}
\end{seealso}

\section{\module{ConfigParser} ---
         Configuration file parser}

\declaremodule{standard}{ConfigParser}
\modulesynopsis{Configuration file parser.}
\moduleauthor{Ken Manheimer}{klm@zope.com}
\moduleauthor{Barry Warsaw}{bwarsaw@python.org}
\moduleauthor{Eric S. Raymond}{esr@thyrsus.com}
\sectionauthor{Christopher G. Petrilli}{petrilli@amber.org}

This module defines the class \class{ConfigParser}.
\indexii{.ini}{file}\indexii{configuration}{file}\index{ini file}
\index{Windows ini file}
The \class{ConfigParser} class implements a basic configuration file
parser language which provides a structure similar to what you would
find on Microsoft Windows INI files.  You can use this to write Python
programs which can be customized by end users easily.

\begin{notice}[warning]
  This library does \emph{not} interpret or write the value-type
  prefixes used in the Windows Registry extended version of INI syntax.
\end{notice}

The configuration file consists of sections, led by a
\samp{[section]} header and followed by \samp{name: value} entries,
with continuations in the style of \rfc{822}; \samp{name=value} is
also accepted.  Note that leading whitespace is removed from values.
The optional values can contain format strings which refer to other
values in the same section, or values in a special
\code{DEFAULT} section.  Additional defaults can be provided on
initialization and retrieval.  Lines beginning with \character{\#} or
\character{;} are ignored and may be used to provide comments.

For example:

\begin{verbatim}
[My Section]
foodir: %(dir)s/whatever
dir=frob
\end{verbatim}

would resolve the \samp{\%(dir)s} to the value of
\samp{dir} (\samp{frob} in this case).  All reference expansions are
done on demand.

Default values can be specified by passing them into the
\class{ConfigParser} constructor as a dictionary.  Additional defaults 
may be passed into the \method{get()} method which will override all
others.

Sections are normally stored in a builtin dictionary. An alternative
dictionary type can be passed to the \class{ConfigParser} constructor.
For example, if a dictionary type is passed that sorts its keys,
the sections will be sorted on write-back, as will be the keys within
each section.

\begin{classdesc}{RawConfigParser}{\optional{defaults\optional{, dict_type}}}
The basic configuration object.  When \var{defaults} is given, it is
initialized into the dictionary of intrinsic defaults.  When \var{dict_type}
is given, it will be used to create the dictionary objects for the list
of sections, for the options within a section, and for the default values.
This class does not support the magical interpolation behavior.
\versionadded{2.3}
\versionchanged[\var{dict_type} was added]{2.6}
\end{classdesc}

\begin{classdesc}{ConfigParser}{\optional{defaults}}
Derived class of \class{RawConfigParser} that implements the magical
interpolation feature and adds optional arguments to the \method{get()}
and \method{items()} methods.  The values in \var{defaults} must be
appropriate for the \samp{\%()s} string interpolation.  Note that
\var{__name__} is an intrinsic default; its value is the section name,
and will override any value provided in \var{defaults}.

All option names used in interpolation will be passed through the
\method{optionxform()} method just like any other option name
reference.  For example, using the default implementation of
\method{optionxform()} (which converts option names to lower case),
the values \samp{foo \%(bar)s} and \samp{foo \%(BAR)s} are
equivalent.
\end{classdesc}

\begin{classdesc}{SafeConfigParser}{\optional{defaults}}
Derived class of \class{ConfigParser} that implements a more-sane
variant of the magical interpolation feature.  This implementation is
more predictable as well.
% XXX Need to explain what's safer/more predictable about it.
New applications should prefer this version if they don't need to be
compatible with older versions of Python.
\versionadded{2.3}
\end{classdesc}

\begin{excdesc}{NoSectionError}
Exception raised when a specified section is not found.
\end{excdesc}

\begin{excdesc}{DuplicateSectionError}
Exception raised if \method{add_section()} is called with the name of
a section that is already present.
\end{excdesc}

\begin{excdesc}{NoOptionError}
Exception raised when a specified option is not found in the specified 
section.
\end{excdesc}

\begin{excdesc}{InterpolationError}
Base class for exceptions raised when problems occur performing string
interpolation.
\end{excdesc}

\begin{excdesc}{InterpolationDepthError}
Exception raised when string interpolation cannot be completed because
the number of iterations exceeds \constant{MAX_INTERPOLATION_DEPTH}.
Subclass of \exception{InterpolationError}.
\end{excdesc}

\begin{excdesc}{InterpolationMissingOptionError}
Exception raised when an option referenced from a value does not exist.
Subclass of \exception{InterpolationError}.
\versionadded{2.3}
\end{excdesc}

\begin{excdesc}{InterpolationSyntaxError}
Exception raised when the source text into which substitutions are
made does not conform to the required syntax.
Subclass of \exception{InterpolationError}.
\versionadded{2.3}
\end{excdesc}

\begin{excdesc}{MissingSectionHeaderError}
Exception raised when attempting to parse a file which has no section
headers.
\end{excdesc}

\begin{excdesc}{ParsingError}
Exception raised when errors occur attempting to parse a file.
\end{excdesc}

\begin{datadesc}{MAX_INTERPOLATION_DEPTH}
The maximum depth for recursive interpolation for \method{get()} when
the \var{raw} parameter is false.  This is relevant only for the
\class{ConfigParser} class.
\end{datadesc}


\begin{seealso}
  \seemodule{shlex}{Support for a creating \UNIX{} shell-like
                    mini-languages which can be used as an alternate
                    format for application configuration files.}
\end{seealso}


\subsection{RawConfigParser Objects \label{RawConfigParser-objects}}

\class{RawConfigParser} instances have the following methods:

\begin{methoddesc}[RawConfigParser]{defaults}{}
Return a dictionary containing the instance-wide defaults.
\end{methoddesc}

\begin{methoddesc}[RawConfigParser]{sections}{}
Return a list of the sections available; \code{DEFAULT} is not
included in the list.
\end{methoddesc}

\begin{methoddesc}[RawConfigParser]{add_section}{section}
Add a section named \var{section} to the instance.  If a section by
the given name already exists, \exception{DuplicateSectionError} is
raised.
\end{methoddesc}

\begin{methoddesc}[RawConfigParser]{has_section}{section}
Indicates whether the named section is present in the
configuration. The \code{DEFAULT} section is not acknowledged.
\end{methoddesc}

\begin{methoddesc}[RawConfigParser]{options}{section}
Returns a list of options available in the specified \var{section}.
\end{methoddesc}

\begin{methoddesc}[RawConfigParser]{has_option}{section, option}
If the given section exists, and contains the given option,
return \constant{True}; otherwise return \constant{False}.
\versionadded{1.6}
\end{methoddesc}

\begin{methoddesc}[RawConfigParser]{read}{filenames}
Attempt to read and parse a list of filenames, returning a list of filenames
which were successfully parsed.  If \var{filenames} is a string or
Unicode string, it is treated as a single filename.
If a file named in \var{filenames} cannot be opened, that file will be
ignored.  This is designed so that you can specify a list of potential
configuration file locations (for example, the current directory, the
user's home directory, and some system-wide directory), and all
existing configuration files in the list will be read.  If none of the
named files exist, the \class{ConfigParser} instance will contain an
empty dataset.  An application which requires initial values to be
loaded from a file should load the required file or files using
\method{readfp()} before calling \method{read()} for any optional
files:

\begin{verbatim}
import ConfigParser, os

config = ConfigParser.ConfigParser()
config.readfp(open('defaults.cfg'))
config.read(['site.cfg', os.path.expanduser('~/.myapp.cfg')])
\end{verbatim}
\versionchanged[Returns list of successfully parsed filenames]{2.4}
\end{methoddesc}

\begin{methoddesc}[RawConfigParser]{readfp}{fp\optional{, filename}}
Read and parse configuration data from the file or file-like object in
\var{fp} (only the \method{readline()} method is used).  If
\var{filename} is omitted and \var{fp} has a \member{name} attribute,
that is used for \var{filename}; the default is \samp{<???>}.
\end{methoddesc}

\begin{methoddesc}[RawConfigParser]{get}{section, option}
Get an \var{option} value for the named \var{section}.
\end{methoddesc}

\begin{methoddesc}[RawConfigParser]{getint}{section, option}
A convenience method which coerces the \var{option} in the specified
\var{section} to an integer.
\end{methoddesc}

\begin{methoddesc}[RawConfigParser]{getfloat}{section, option}
A convenience method which coerces the \var{option} in the specified
\var{section} to a floating point number.
\end{methoddesc}

\begin{methoddesc}[RawConfigParser]{getboolean}{section, option}
A convenience method which coerces the \var{option} in the specified
\var{section} to a Boolean value.  Note that the accepted values
for the option are \code{"1"}, \code{"yes"}, \code{"true"}, and \code{"on"},
which cause this method to return \code{True}, and \code{"0"}, \code{"no"},
\code{"false"}, and \code{"off"}, which cause it to return \code{False}.  These
string values are checked in a case-insensitive manner.  Any other value will
cause it to raise \exception{ValueError}.
\end{methoddesc}

\begin{methoddesc}[RawConfigParser]{items}{section}
Return a list of \code{(\var{name}, \var{value})} pairs for each
option in the given \var{section}.
\end{methoddesc}

\begin{methoddesc}[RawConfigParser]{set}{section, option, value}
If the given section exists, set the given option to the specified
value; otherwise raise \exception{NoSectionError}.  While it is
possible to use \class{RawConfigParser} (or \class{ConfigParser} with
\var{raw} parameters set to true) for \emph{internal} storage of
non-string values, full functionality (including interpolation and
output to files) can only be achieved using string values.
\versionadded{1.6}
\end{methoddesc}

\begin{methoddesc}[RawConfigParser]{write}{fileobject}
Write a representation of the configuration to the specified file
object.  This representation can be parsed by a future \method{read()}
call.
\versionadded{1.6}
\end{methoddesc}

\begin{methoddesc}[RawConfigParser]{remove_option}{section, option}
Remove the specified \var{option} from the specified \var{section}.
If the section does not exist, raise \exception{NoSectionError}. 
If the option existed to be removed, return \constant{True};
otherwise return \constant{False}.
\versionadded{1.6}
\end{methoddesc}

\begin{methoddesc}[RawConfigParser]{remove_section}{section}
Remove the specified \var{section} from the configuration.
If the section in fact existed, return \code{True}.
Otherwise return \code{False}.
\end{methoddesc}

\begin{methoddesc}[RawConfigParser]{optionxform}{option}
Transforms the option name \var{option} as found in an input file or
as passed in by  client code to the form that should be used in the
internal structures.  The default implementation returns a lower-case
version of \var{option}; subclasses may override this or client code
can set an attribute of this name on instances to affect this
behavior.  Setting this to \function{str()}, for example, would make
option names case sensitive.
\end{methoddesc}


\subsection{ConfigParser Objects \label{ConfigParser-objects}}

The \class{ConfigParser} class extends some methods of the
\class{RawConfigParser} interface, adding some optional arguments.

\begin{methoddesc}[ConfigParser]{get}{section, option\optional{, raw\optional{, vars}}}
Get an \var{option} value for the named \var{section}.  All the
\character{\%} interpolations are expanded in the return values, based
on the defaults passed into the constructor, as well as the options
\var{vars} provided, unless the \var{raw} argument is true.
\end{methoddesc}

\begin{methoddesc}[ConfigParser]{items}{section\optional{, raw\optional{, vars}}}
Return a list of \code{(\var{name}, \var{value})} pairs for each
option in the given \var{section}. Optional arguments have the
same meaning as for the \method{get()} method.
\versionadded{2.3}
\end{methoddesc}


\subsection{SafeConfigParser Objects \label{SafeConfigParser-objects}}

The \class{SafeConfigParser} class implements the same extended
interface as \class{ConfigParser}, with the following addition:

\begin{methoddesc}[SafeConfigParser]{set}{section, option, value}
If the given section exists, set the given option to the specified
value; otherwise raise \exception{NoSectionError}.  \var{value} must
be a string (\class{str} or \class{unicode}); if not,
\exception{TypeError} is raised.
\versionadded{2.4}
\end{methoddesc}

\section{\module{fileinput} ---
         Iterate over lines from multiple input streams}
\declaremodule{standard}{fileinput}
\moduleauthor{Guido van Rossum}{guido@python.org}
\sectionauthor{Fred L. Drake, Jr.}{fdrake@acm.org}

\modulesynopsis{Perl-like iteration over lines from multiple input
streams, with ``save in place'' capability.}


This module implements a helper class and functions to quickly write a
loop over standard input or a list of files.

The typical use is:

\begin{verbatim}
import fileinput
for line in fileinput.input():
    process(line)
\end{verbatim}

This iterates over the lines of all files listed in
\code{sys.argv[1:]}, defaulting to \code{sys.stdin} if the list is
empty.  If a filename is \code{'-'}, it is also replaced by
\code{sys.stdin}.  To specify an alternative list of filenames, pass
it as the first argument to \function{input()}.  A single file name is
also allowed.

All files are opened in text mode.  If an I/O error occurs during
opening or reading a file, \exception{IOError} is raised.

If \code{sys.stdin} is used more than once, the second and further use
will return no lines, except perhaps for interactive use, or if it has
been explicitly reset (e.g. using \code{sys.stdin.seek(0)}).

Empty files are opened and immediately closed; the only time their
presence in the list of filenames is noticeable at all is when the
last file opened is empty.

It is possible that the last line of a file does not end in a newline
character; lines are returned including the trailing newline when it
is present.

The following function is the primary interface of this module:

\begin{funcdesc}{input}{\optional{files\optional{,
                       inplace\optional{, backup}}}}
  Create an instance of the \class{FileInput} class.  The instance
  will be used as global state for the functions of this module, and
  is also returned to use during iteration.
\end{funcdesc}


The following functions use the global state created by
\function{input()}; if there is no active state,
\exception{RuntimeError} is raised.

\begin{funcdesc}{filename}{}
  Return the name of the file currently being read.  Before the first
  line has been read, returns \code{None}.
\end{funcdesc}

\begin{funcdesc}{lineno}{}
  Return the cumulative line number of the line that has just been
  read.  Before the first line has been read, returns \code{0}.  After
  the last line of the last file has been read, returns the line
  number of that line.
\end{funcdesc}

\begin{funcdesc}{filelineno}{}
  Return the line number in the current file.  Before the first line
  has been read, returns \code{0}.  After the last line of the last
  file has been read, returns the line number of that line within the
  file.
\end{funcdesc}

\begin{funcdesc}{isfirstline}{}
  Returns true the line just read is the first line of its file,
  otherwise returns false.
\end{funcdesc}

\begin{funcdesc}{isstdin}{}
  Returns true if the last line was read from \code{sys.stdin},
  otherwise returns false.
\end{funcdesc}

\begin{funcdesc}{nextfile}{}
  Close the current file so that the next iteration will read the
  first line from the next file (if any); lines not read from the file
  will not count towards the cumulative line count.  The filename is
  not changed until after the first line of the next file has been
  read.  Before the first line has been read, this function has no
  effect; it cannot be used to skip the first file.  After the last
  line of the last file has been read, this function has no effect.
\end{funcdesc}

\begin{funcdesc}{close}{}
  Close the sequence.
\end{funcdesc}


The class which implements the sequence behavior provided by the
module is available for subclassing as well:

\begin{classdesc}{FileInput}{\optional{files\optional{,
                             inplace\optional{, backup}}}}
  Class \class{FileInput} is the implementation; its methods
  \method{filename()}, \method{lineno()}, \method{fileline()},
  \method{isfirstline()}, \method{isstdin()}, \method{nextfile()} and
  \method{close()} correspond to the functions of the same name in the
  module.  In addition it has a \method{readline()} method which
  returns the next input line, and a \method{__getitem__()} method
  which implements the sequence behavior.  The sequence must be
  accessed in strictly sequential order; random access and
  \method{readline()} cannot be mixed.
\end{classdesc}

\strong{Optional in-place filtering:} if the keyword argument
\code{\var{inplace}=1} is passed to \function{input()} or to the
\class{FileInput} constructor, the file is moved to a backup file and
standard output is directed to the input file.
This makes it possible to write a filter that rewrites its input file
in place.  If the keyword argument \code{\var{backup}='.<some
extension>'} is also given, it specifies the extension for the backup
file, and the backup file remains around; by default, the extension is
\code{'.bak'} and it is deleted when the output file is closed.  In-place
filtering is disabled when standard input is read.

\strong{Caveat:} The current implementation does not work for MS-DOS
8+3 filesystems.

% This section was contributed by Drew Csillag <drew_csillag@geocities.com>.

\section{\module{calendar} ---
         Functions that emulate the \UNIX{} \program{cal} program.}
\declaremodule{standard}{calendar}

\modulesynopsis{Functions that emulate the \UNIX{} \program{cal}
program.}


This module allows you to output calendars like the \UNIX{}
\manpage{cal}{1} program.

\begin{funcdesc}{isleap}{year}
Returns \code{1} if \var{year} is a leap year.
\end{funcdesc}

\begin{funcdesc}{leapdays}{year1, year2}
Return the number of leap years in the range
[\var{year1}\ldots\var{year2}].
\end{funcdesc}

\begin{funcdesc}{weekday}{year, month, day}
Returns the day of the week (\code{0} is Monday) for \var{year}
(\code{1970}--\ldots), \var{month} (\code{1}--\code{12}), \var{day}
(\code{1}--\code{31}).
\end{funcdesc}

\begin{funcdesc}{monthrange}{year, month}
Returns weekday of first day of the month and number of days in month, 
for the specified \var{year} and \var{month}.
\end{funcdesc}

\begin{funcdesc}{monthcalendar}{year, month}
Returns a matrix representing a month's calendar.  Each row represents
a week; days outside of the month a represented by zeros.
\end{funcdesc}

\begin{funcdesc}{prmonth}{year, month\optional{, width\optional{, length}}}
Prints a month's calendar.  If \var{width} is provided, it specifies
the width of the columns that the numbers are centered in.  If
\var{length} is given, it specifies the number of lines that each
week will use.
\end{funcdesc}

\begin{funcdesc}{prcal}{year}
Prints the calendar for the year \var{year}.
\end{funcdesc}

\section{\module{cmd} ---
         Support for line-oriented command interpreters}

\declaremodule{standard}{cmd}
\sectionauthor{Eric S. Raymond}{esr@snark.thyrsus.com}
\modulesynopsis{Build line-oriented command interpreters.}


The \class{Cmd} class provides a simple framework for writing
line-oriented command interpreters.  These are often useful for
test harnesses, administrative tools, and prototypes that will
later be wrapped in a more sophisticated interface.

\begin{classdesc}{Cmd}{\optional{completekey\optional{,
                       stdin\optional{, stdout}}}}
A \class{Cmd} instance or subclass instance is a line-oriented
interpreter framework.  There is no good reason to instantiate
\class{Cmd} itself; rather, it's useful as a superclass of an
interpreter class you define yourself in order to inherit
\class{Cmd}'s methods and encapsulate action methods.

The optional argument \var{completekey} is the \refmodule{readline} name
of a completion key; it defaults to \kbd{Tab}. If \var{completekey} is
not \constant{None} and \refmodule{readline} is available, command completion
is done automatically.

The optional arguments \var{stdin} and \var{stdout} specify the 
input and output file objects that the Cmd instance or subclass 
instance will use for input and output. If not specified, they
will default to \var{sys.stdin} and \var{sys.stdout}.

\versionchanged[The \var{stdin} and \var{stdout} parameters were added]{2.3}
\end{classdesc}

\subsection{Cmd Objects}
\label{Cmd-objects}

A \class{Cmd} instance has the following methods:

\begin{methoddesc}{cmdloop}{\optional{intro}}
Repeatedly issue a prompt, accept input, parse an initial prefix off
the received input, and dispatch to action methods, passing them the
remainder of the line as argument.

The optional argument is a banner or intro string to be issued before the
first prompt (this overrides the \member{intro} class member).

If the \refmodule{readline} module is loaded, input will automatically
inherit \program{bash}-like history-list editing (e.g. \kbd{Control-P}
scrolls back to the last command, \kbd{Control-N} forward to the next
one, \kbd{Control-F} moves the cursor to the right non-destructively,
\kbd{Control-B} moves the cursor to the left non-destructively, etc.).

An end-of-file on input is passed back as the string \code{'EOF'}.

An interpreter instance will recognize a command name \samp{foo} if
and only if it has a method \method{do_foo()}.  As a special case,
a line beginning with the character \character{?} is dispatched to
the method \method{do_help()}.  As another special case, a line
beginning with the character \character{!} is dispatched to the
method \method{do_shell()} (if such a method is defined).

This method will return when the \method{postcmd()} method returns a
true value.  The \var{stop} argument to \method{postcmd()} is the
return value from the command's corresponding \method{do_*()} method.

If completion is enabled, completing commands will be done
automatically, and completing of commands args is done by calling
\method{complete_foo()} with arguments \var{text}, \var{line},
\var{begidx}, and \var{endidx}.  \var{text} is the string prefix we
are attempting to match: all returned matches must begin with it.
\var{line} is the current input line with leading whitespace removed,
\var{begidx} and \var{endidx} are the beginning and ending indexes
of the prefix text, which could be used to provide different
completion depending upon which position the argument is in.

All subclasses of \class{Cmd} inherit a predefined \method{do_help()}.
This method, called with an argument \code{'bar'}, invokes the
corresponding method \method{help_bar()}.  With no argument,
\method{do_help()} lists all available help topics (that is, all
commands with corresponding \method{help_*()} methods), and also lists
any undocumented commands.
\end{methoddesc}

\begin{methoddesc}{onecmd}{str}
Interpret the argument as though it had been typed in response to the
prompt.  This may be overridden, but should not normally need to be;
see the \method{precmd()} and \method{postcmd()} methods for useful
execution hooks.  The return value is a flag indicating whether
interpretation of commands by the interpreter should stop.  If there
is a \method{do_*()} method for the command \var{str}, the return
value of that method is returned, otherwise the return value from the
\method{default()} method is returned.
\end{methoddesc}

\begin{methoddesc}{emptyline}{}
Method called when an empty line is entered in response to the prompt.
If this method is not overridden, it repeats the last nonempty command
entered.  
\end{methoddesc}

\begin{methoddesc}{default}{line}
Method called on an input line when the command prefix is not
recognized. If this method is not overridden, it prints an
error message and returns.
\end{methoddesc}

\begin{methoddesc}{completedefault}{text, line, begidx, endidx}
Method called to complete an input line when no command-specific
\method{complete_*()} method is available.  By default, it returns an
empty list.
\end{methoddesc}

\begin{methoddesc}{precmd}{line}
Hook method executed just before the command line \var{line} is
interpreted, but after the input prompt is generated and issued.  This
method is a stub in \class{Cmd}; it exists to be overridden by
subclasses.  The return value is used as the command which will be
executed by the \method{onecmd()} method; the \method{precmd()}
implementation may re-write the command or simply return \var{line}
unchanged.
\end{methoddesc}

\begin{methoddesc}{postcmd}{stop, line}
Hook method executed just after a command dispatch is finished.  This
method is a stub in \class{Cmd}; it exists to be overridden by
subclasses.  \var{line} is the command line which was executed, and
\var{stop} is a flag which indicates whether execution will be
terminated after the call to \method{postcmd()}; this will be the
return value of the \method{onecmd()} method.  The return value of
this method will be used as the new value for the internal flag which
corresponds to \var{stop}; returning false will cause interpretation
to continue.
\end{methoddesc}

\begin{methoddesc}{preloop}{}
Hook method executed once when \method{cmdloop()} is called.  This
method is a stub in \class{Cmd}; it exists to be overridden by
subclasses.
\end{methoddesc}

\begin{methoddesc}{postloop}{}
Hook method executed once when \method{cmdloop()} is about to return.
This method is a stub in \class{Cmd}; it exists to be overridden by
subclasses.
\end{methoddesc}

Instances of \class{Cmd} subclasses have some public instance variables:

\begin{memberdesc}{prompt}
The prompt issued to solicit input.
\end{memberdesc}

\begin{memberdesc}{identchars}
The string of characters accepted for the command prefix.
\end{memberdesc}

\begin{memberdesc}{lastcmd}
The last nonempty command prefix seen. 
\end{memberdesc}

\begin{memberdesc}{intro}
A string to issue as an intro or banner.  May be overridden by giving
the \method{cmdloop()} method an argument.
\end{memberdesc}

\begin{memberdesc}{doc_header}
The header to issue if the help output has a section for documented
commands.
\end{memberdesc}

\begin{memberdesc}{misc_header}
The header to issue if the help output has a section for miscellaneous 
help topics (that is, there are \method{help_*()} methods without
corresponding \method{do_*()} methods).
\end{memberdesc}

\begin{memberdesc}{undoc_header}
The header to issue if the help output has a section for undocumented 
commands (that is, there are \method{do_*()} methods without
corresponding \method{help_*()} methods).
\end{memberdesc}

\begin{memberdesc}{ruler}
The character used to draw separator lines under the help-message
headers.  If empty, no ruler line is drawn.  It defaults to
\character{=}.
\end{memberdesc}

\section{\module{shlex} ---
         Simple lexical analysis}

\declaremodule{standard}{shlex}
\modulesynopsis{Simple lexical analysis for \UNIX{} shell-like languages.}
\moduleauthor{Eric S. Raymond}{esr@snark.thyrsus.com}
\sectionauthor{Eric S. Raymond}{esr@snark.thyrsus.com}

\versionadded{1.5.2}

The \class{shlex} class makes it easy to write lexical analyzers for
simple syntaxes resembling that of the \UNIX{} shell.  This will often
be useful for writing minilanguages, e.g.\ in run control files for
Python applications.

\begin{classdesc}{shlex}{\optional{stream\optional{, file}}}
A \class{shlex} instance or subclass instance is a lexical analyzer
object.  The initialization argument, if present, specifies where to
read characters from. It must be a file- or stream-like object with
\method{read()} and \method{readline()} methods.  If no argument is given,
input will be taken from \code{sys.stdin}.  The second optional 
argument is a filename string, which sets the initial value of the
\member{infile} member.  If the stream argument is omitted or
equal to \code{sys.stdin}, this second argument defaults to ``stdin''.
\end{classdesc}


\begin{seealso}
  \seemodule{ConfigParser}{Parser for configuration files similar to the
                           Windows \file{.ini} files.}
\end{seealso}


\subsection{shlex Objects \label{shlex-objects}}

A \class{shlex} instance has the following methods:


\begin{methoddesc}{get_token}{}
Return a token.  If tokens have been stacked using
\method{push_token()}, pop a token off the stack.  Otherwise, read one
from the input stream.  If reading encounters an immediate
end-of-file, an empty string is returned. 
\end{methoddesc}

\begin{methoddesc}{push_token}{str}
Push the argument onto the token stack.
\end{methoddesc}

\begin{methoddesc}{read_token}{}
Read a raw token.  Ignore the pushback stack, and do not interpret source
requests.  (This is not ordinarily a useful entry point, and is
documented here only for the sake of completeness.)
\end{methoddesc}

\begin{methoddesc}{sourcehook}{filename}
When \class{shlex} detects a source request (see
\member{source} below) this method is given the following token as
argument, and expected to return a tuple consisting of a filename and
an open file-like object.

Normally, this method first strips any quotes off the argument.  If
the result is an absolute pathname, or there was no previous source
request in effect, or the previous source was a stream
(e.g. \code{sys.stdin}), the result is left alone.  Otherwise, if the
result is a relative pathname, the directory part of the name of the
file immediately before it on the source inclusion stack is prepended
(this behavior is like the way the C preprocessor handles
\code{\#include "file.h"}).

The result of the manipulations is treated as a filename, and returned
as the first component of the tuple, with
\function{open()} called on it to yield the second component. (Note:
this is the reverse of the order of arguments in instance initialization!)

This hook is exposed so that you can use it to implement directory
search paths, addition of file extensions, and other namespace hacks.
There is no corresponding `close' hook, but a shlex instance will call
the \method{close()} method of the sourced input stream when it
returns \EOF.

For more explicit control of source stacking, use the
\method{push_source()} and \method{pop_source()} methods. 
\end{methoddesc}

\begin{methoddesc}{push_source}{stream\optional{, filename}}
Push an input source stream onto the input stack.  If the filename
argument is specified it will later be available for use in error
messages.  This is the same method used internally by the
\method{sourcehook} method.
\versionadded{2.1}
\end{methoddesc}

\begin{methoddesc}{pop_source}{}
Pop the last-pushed input source from the input stack.
This is the same method used internally when the lexer reaches
\EOF on a stacked input stream.
\versionadded{2.1}
\end{methoddesc}

\begin{methoddesc}{error_leader}{\optional{file\optional{, line}}}
This method generates an error message leader in the format of a
\UNIX{} C compiler error label; the format is \code{'"\%s", line \%d: '},
where the \samp{\%s} is replaced with the name of the current source
file and the \samp{\%d} with the current input line number (the
optional arguments can be used to override these).

This convenience is provided to encourage \module{shlex} users to
generate error messages in the standard, parseable format understood
by Emacs and other \UNIX{} tools.
\end{methoddesc}

Instances of \class{shlex} subclasses have some public instance
variables which either control lexical analysis or can be used for
debugging:

\begin{memberdesc}{commenters}
The string of characters that are recognized as comment beginners.
All characters from the comment beginner to end of line are ignored.
Includes just \character{\#} by default.   
\end{memberdesc}

\begin{memberdesc}{wordchars}
The string of characters that will accumulate into multi-character
tokens.  By default, includes all \ASCII{} alphanumerics and
underscore.
\end{memberdesc}

\begin{memberdesc}{whitespace}
Characters that will be considered whitespace and skipped.  Whitespace
bounds tokens.  By default, includes space, tab, linefeed and
carriage-return.
\end{memberdesc}

\begin{memberdesc}{quotes}
Characters that will be considered string quotes.  The token
accumulates until the same quote is encountered again (thus, different
quote types protect each other as in the shell.)  By default, includes
\ASCII{} single and double quotes.
\end{memberdesc}

\begin{memberdesc}{infile}
The name of the current input file, as initially set at class
instantiation time or stacked by later source requests.  It may
be useful to examine this when constructing error messages.
\end{memberdesc}

\begin{memberdesc}{instream}
The input stream from which this \class{shlex} instance is reading
characters.
\end{memberdesc}

\begin{memberdesc}{source}
This member is \code{None} by default.  If you assign a string to it,
that string will be recognized as a lexical-level inclusion request
similar to the \samp{source} keyword in various shells.  That is, the
immediately following token will opened as a filename and input taken
from that stream until \EOF, at which point the \method{close()}
method of that stream will be called and the input source will again
become the original input stream. Source requests may be stacked any
number of levels deep.
\end{memberdesc}

\begin{memberdesc}{debug}
If this member is numeric and \code{1} or more, a \class{shlex}
instance will print verbose progress output on its behavior.  If you
need to use this, you can read the module source code to learn the
details.
\end{memberdesc}

Note that any character not declared to be a word character,
whitespace, or a quote will be returned as a single-character token.

Quote and comment characters are not recognized within words.  Thus,
the bare words \samp{ain't} and \samp{ain\#t} would be returned as single
tokens by the default parser.

\begin{memberdesc}{lineno}
Source line number (count of newlines seen so far plus one).
\end{memberdesc}

\begin{memberdesc}{token}
The token buffer.  It may be useful to examine this when catching
exceptions.
\end{memberdesc}


\chapter{Generic Operating System Services}
\label{allos}

The modules described in this chapter provide interfaces to operating
system features that are available on (almost) all operating systems,
such as files and a clock.  The interfaces are generally modelled
after the \UNIX{} or \C{} interfaces but they are available on most
other systems as well.  Here's an overview:

\begin{description}

\item[os]
--- Miscellaneous OS interfaces.

\item[time]
--- Time access and conversions.

\item[getopt]
--- Parser for command line options.

\item[tempfile]
--- Generate temporary file names.

\item[errno]
--- Standard errno system symbols.

\item[glob]
--- \UNIX{} shell style pathname pattern expansion.

\item[fnmatch]
--- \UNIX{} shell style pathname pattern matching.

\item[locale]
--- Internationalization services.

\end{description}
		% Generic Operating System Services
\section{\module{os} ---
         Miscellaneous operating system interfaces}

\declaremodule{standard}{os}
\modulesynopsis{Miscellaneous operating system interfaces.}


This module provides a more portable way of using operating system
dependent functionality than importing a operating system dependent
built-in module like \refmodule{posix} or \module{nt}.

This module searches for an operating system dependent built-in module like
\module{mac} or \refmodule{posix} and exports the same functions and data
as found there.  The design of all Python's built-in operating system dependent
modules is such that as long as the same functionality is available,
it uses the same interface; for example, the function
\code{os.stat(\var{path})} returns stat information about \var{path} in
the same format (which happens to have originated with the
\POSIX{} interface).

Extensions peculiar to a particular operating system are also
available through the \module{os} module, but using them is of course a
threat to portability!

Note that after the first time \module{os} is imported, there is
\emph{no} performance penalty in using functions from \module{os}
instead of directly from the operating system dependent built-in module,
so there should be \emph{no} reason not to use \module{os}!


% Frank Stajano <fstajano@uk.research.att.com> complained that it
% wasn't clear that the entries described in the subsections were all
% available at the module level (most uses of subsections are
% different); I think this is only a problem for the HTML version,
% where the relationship may not be as clear.
%
\ifhtml
The \module{os} module contains many functions and data values.
The items below and in the following sub-sections are all available
directly from the \module{os} module.
\fi


\begin{excdesc}{error}
This exception is raised when a function returns a system-related
error (not for illegal argument types or other incidental errors).
This is also known as the built-in exception \exception{OSError}.  The
accompanying value is a pair containing the numeric error code from
\cdata{errno} and the corresponding string, as would be printed by the
C function \cfunction{perror()}.  See the module
\refmodule{errno}\refbimodindex{errno}, which contains names for the
error codes defined by the underlying operating system.

When exceptions are classes, this exception carries two attributes,
\member{errno} and \member{strerror}.  The first holds the value of
the C \cdata{errno} variable, and the latter holds the corresponding
error message from \cfunction{strerror()}.  For exceptions that
involve a file system path (such as \function{chdir()} or
\function{unlink()}), the exception instance will contain a third
attribute, \member{filename}, which is the file name passed to the
function.
\end{excdesc}

\begin{datadesc}{name}
The name of the operating system dependent module imported.  The
following names have currently been registered: \code{'posix'},
\code{'nt'}, \code{'mac'}, \code{'os2'}, \code{'ce'},
\code{'java'}, \code{'riscos'}.
\end{datadesc}

\begin{datadesc}{path}
The corresponding operating system dependent standard module for pathname
operations, such as \module{posixpath} or \module{macpath}.  Thus,
given the proper imports, \code{os.path.split(\var{file})} is
equivalent to but more portable than
\code{posixpath.split(\var{file})}.  Note that this is also an
importable module: it may be imported directly as
\refmodule{os.path}.
\end{datadesc}



\subsection{Process Parameters \label{os-procinfo}}

These functions and data items provide information and operate on the
current process and user.

\begin{datadesc}{environ}
A mapping object representing the string environment. For example,
\code{environ['HOME']} is the pathname of your home directory (on some
platforms), and is equivalent to \code{getenv("HOME")} in C.

This mapping is captured the first time the \module{os} module is
imported, typically during Python startup as part of processing
\file{site.py}.  Changes to the environment made after this time are
not reflected in \code{os.environ}, except for changes made by modifying
\code{os.environ} directly.

If the platform supports the \function{putenv()} function, this
mapping may be used to modify the environment as well as query the
environment.  \function{putenv()} will be called automatically when
the mapping is modified.
\note{Calling \function{putenv()} directly does not change
\code{os.environ}, so it's better to modify \code{os.environ}.}
\note{On some platforms, including FreeBSD and Mac OS X, setting
\code{environ} may cause memory leaks.  Refer to the system documentation
for \cfunction{putenv()}.}

If \function{putenv()} is not provided, a modified copy of this mapping 
may be passed to the appropriate process-creation functions to cause 
child processes to use a modified environment.

If the platform supports the \function{unsetenv()} function, you can 
delete items in this mapping to unset environment variables.
\function{unsetenv()} will be called automatically when an item is
deleted from \code{os.environ}.

\end{datadesc}

\begin{funcdescni}{chdir}{path}
\funclineni{fchdir}{fd}
\funclineni{getcwd}{}
These functions are described in ``Files and Directories'' (section
\ref{os-file-dir}).
\end{funcdescni}

\begin{funcdesc}{ctermid}{}
Return the filename corresponding to the controlling terminal of the
process.
Availability: \UNIX.
\end{funcdesc}

\begin{funcdesc}{getegid}{}
Return the effective group id of the current process.  This
corresponds to the `set id' bit on the file being executed in the
current process.
Availability: \UNIX.
\end{funcdesc}

\begin{funcdesc}{geteuid}{}
\index{user!effective id}
Return the current process' effective user id.
Availability: \UNIX.
\end{funcdesc}

\begin{funcdesc}{getgid}{}
\index{process!group}
Return the real group id of the current process.
Availability: \UNIX.
\end{funcdesc}

\begin{funcdesc}{getgroups}{}
Return list of supplemental group ids associated with the current
process.
Availability: \UNIX.
\end{funcdesc}

\begin{funcdesc}{getlogin}{}
Return the name of the user logged in on the controlling terminal of
the process.  For most purposes, it is more useful to use the
environment variable \envvar{LOGNAME} to find out who the user is,
or \code{pwd.getpwuid(os.getuid())[0]} to get the login name
of the currently effective user ID.
Availability: \UNIX.
\end{funcdesc}

\begin{funcdesc}{getpgid}{pid}
Return the process group id of the process with process id \var{pid}.
If \var{pid} is 0, the process group id of the current process is
returned. Availability: \UNIX.
\versionadded{2.3}
\end{funcdesc}

\begin{funcdesc}{getpgrp}{}
\index{process!group}
Return the id of the current process group.
Availability: \UNIX.
\end{funcdesc}

\begin{funcdesc}{getpid}{}
\index{process!id}
Return the current process id.
Availability: \UNIX, Windows.
\end{funcdesc}

\begin{funcdesc}{getppid}{}
\index{process!id of parent}
Return the parent's process id.
Availability: \UNIX.
\end{funcdesc}

\begin{funcdesc}{getuid}{}
\index{user!id}
Return the current process' user id.
Availability: \UNIX.
\end{funcdesc}

\begin{funcdesc}{getenv}{varname\optional{, value}}
Return the value of the environment variable \var{varname} if it
exists, or \var{value} if it doesn't.  \var{value} defaults to
\code{None}.
Availability: most flavors of \UNIX, Windows.
\end{funcdesc}

\begin{funcdesc}{putenv}{varname, value}
\index{environment variables!setting}
Set the environment variable named \var{varname} to the string
\var{value}.  Such changes to the environment affect subprocesses
started with \function{os.system()}, \function{popen()} or
\function{fork()} and \function{execv()}.
Availability: most flavors of \UNIX, Windows.

\note{On some platforms, including FreeBSD and Mac OS X,
setting \code{environ} may cause memory leaks.
Refer to the system documentation for putenv.}

When \function{putenv()} is
supported, assignments to items in \code{os.environ} are automatically
translated into corresponding calls to \function{putenv()}; however,
calls to \function{putenv()} don't update \code{os.environ}, so it is
actually preferable to assign to items of \code{os.environ}.
\end{funcdesc}

\begin{funcdesc}{setegid}{egid}
Set the current process's effective group id.
Availability: \UNIX.
\end{funcdesc}

\begin{funcdesc}{seteuid}{euid}
Set the current process's effective user id.
Availability: \UNIX.
\end{funcdesc}

\begin{funcdesc}{setgid}{gid}
Set the current process' group id.
Availability: \UNIX.
\end{funcdesc}

\begin{funcdesc}{setgroups}{groups}
Set the list of supplemental group ids associated with the current
process to \var{groups}. \var{groups} must be a sequence, and each
element must be an integer identifying a group. This operation is
typical available only to the superuser.
Availability: \UNIX.
\versionadded{2.2}
\end{funcdesc}

\begin{funcdesc}{setpgrp}{}
Calls the system call \cfunction{setpgrp()} or \cfunction{setpgrp(0,
0)} depending on which version is implemented (if any).  See the
\UNIX{} manual for the semantics.
Availability: \UNIX.
\end{funcdesc}

\begin{funcdesc}{setpgid}{pid, pgrp} Calls the system call
\cfunction{setpgid()} to set the process group id of the process with
id \var{pid} to the process group with id \var{pgrp}.  See the \UNIX{}
manual for the semantics.
Availability: \UNIX.
\end{funcdesc}

\begin{funcdesc}{setreuid}{ruid, euid}
Set the current process's real and effective user ids.
Availability: \UNIX.
\end{funcdesc}

\begin{funcdesc}{setregid}{rgid, egid}
Set the current process's real and effective group ids.
Availability: \UNIX.
\end{funcdesc}

\begin{funcdesc}{getsid}{pid}
Calls the system call \cfunction{getsid()}.  See the \UNIX{} manual
for the semantics.
Availability: \UNIX. \versionadded{2.4}
\end{funcdesc}

\begin{funcdesc}{setsid}{}
Calls the system call \cfunction{setsid()}.  See the \UNIX{} manual
for the semantics.
Availability: \UNIX.
\end{funcdesc}

\begin{funcdesc}{setuid}{uid}
\index{user!id, setting}
Set the current process' user id.
Availability: \UNIX.
\end{funcdesc}

% placed in this section since it relates to errno.... a little weak
\begin{funcdesc}{strerror}{code}
Return the error message corresponding to the error code in
\var{code}.
Availability: \UNIX, Windows.
\end{funcdesc}

\begin{funcdesc}{umask}{mask}
Set the current numeric umask and returns the previous umask.
Availability: \UNIX, Windows.
\end{funcdesc}

\begin{funcdesc}{uname}{}
Return a 5-tuple containing information identifying the current
operating system.  The tuple contains 5 strings:
\code{(\var{sysname}, \var{nodename}, \var{release}, \var{version},
\var{machine})}.  Some systems truncate the nodename to 8
characters or to the leading component; a better way to get the
hostname is \function{socket.gethostname()}
\withsubitem{(in module socket)}{\ttindex{gethostname()}}
or even
\withsubitem{(in module socket)}{\ttindex{gethostbyaddr()}}
\code{socket.gethostbyaddr(socket.gethostname())}.
Availability: recent flavors of \UNIX.
\end{funcdesc}

\begin{funcdesc}{unsetenv}{varname}
\index{environment variables!deleting}
Unset (delete) the environment variable named \var{varname}. Such
changes to the environment affect subprocesses started with
\function{os.system()}, \function{popen()} or \function{fork()} and
\function{execv()}. Availability: most flavors of \UNIX, Windows.

When \function{unsetenv()} is
supported, deletion of items in \code{os.environ} is automatically
translated into a corresponding call to \function{unsetenv()}; however,
calls to \function{unsetenv()} don't update \code{os.environ}, so it is
actually preferable to delete items of \code{os.environ}.
\end{funcdesc}

\subsection{File Object Creation \label{os-newstreams}}

These functions create new file objects.


\begin{funcdesc}{fdopen}{fd\optional{, mode\optional{, bufsize}}}
Return an open file object connected to the file descriptor \var{fd}.
\index{I/O control!buffering}
The \var{mode} and \var{bufsize} arguments have the same meaning as
the corresponding arguments to the built-in \function{open()}
function.
Availability: Macintosh, \UNIX, Windows.

\versionchanged[When specified, the \var{mode} argument must now start
  with one of the letters \character{r}, \character{w}, or \character{a},
  otherwise a \exception{ValueError} is raised]{2.3}
\end{funcdesc}

\begin{funcdesc}{popen}{command\optional{, mode\optional{, bufsize}}}
Open a pipe to or from \var{command}.  The return value is an open
file object connected to the pipe, which can be read or written
depending on whether \var{mode} is \code{'r'} (default) or \code{'w'}.
The \var{bufsize} argument has the same meaning as the corresponding
argument to the built-in \function{open()} function.  The exit status of
the command (encoded in the format specified for \function{wait()}) is
available as the return value of the \method{close()} method of the file
object, except that when the exit status is zero (termination without
errors), \code{None} is returned.
Availability: Macintosh, \UNIX, Windows.

\versionchanged[This function worked unreliably under Windows in
  earlier versions of Python.  This was due to the use of the
  \cfunction{_popen()} function from the libraries provided with
  Windows.  Newer versions of Python do not use the broken
  implementation from the Windows libraries]{2.0}
\end{funcdesc}

\begin{funcdesc}{tmpfile}{}
Return a new file object opened in update mode (\samp{w+b}).  The file
has no directory entries associated with it and will be automatically
deleted once there are no file descriptors for the file.
Availability: Macintosh, \UNIX, Windows.
\end{funcdesc}


For each of the following \function{popen()} variants, if \var{bufsize} is
specified, it specifies the buffer size for the I/O pipes.
\var{mode}, if provided, should be the string \code{'b'} or
\code{'t'}; on Windows this is needed to determine whether the file
objects should be opened in binary or text mode.  The default value
for \var{mode} is \code{'t'}.

Also, for each of these variants, on \UNIX, \var{cmd} may be a sequence, in
which case arguments will be passed directly to the program without shell
intervention (as with \function{os.spawnv()}). If \var{cmd} is a string it will
be passed to the shell (as with \function{os.system()}).

These methods do not make it possible to retrieve the exit status from
the child processes.  The only way to control the input and output
streams and also retrieve the return codes is to use the
\class{Popen3} and \class{Popen4} classes from the \refmodule{popen2}
module; these are only available on \UNIX.

For a discussion of possible deadlock conditions related to the use
of these functions, see ``\ulink{Flow Control
Issues}{popen2-flow-control.html}''
(section~\ref{popen2-flow-control}).

\begin{funcdesc}{popen2}{cmd\optional{, mode\optional{, bufsize}}}
Executes \var{cmd} as a sub-process.  Returns the file objects
\code{(\var{child_stdin}, \var{child_stdout})}.
Availability: Macintosh, \UNIX, Windows.
\versionadded{2.0}
\end{funcdesc}

\begin{funcdesc}{popen3}{cmd\optional{, mode\optional{, bufsize}}}
Executes \var{cmd} as a sub-process.  Returns the file objects
\code{(\var{child_stdin}, \var{child_stdout}, \var{child_stderr})}.
Availability: Macintosh, \UNIX, Windows.
\versionadded{2.0}
\end{funcdesc}

\begin{funcdesc}{popen4}{cmd\optional{, mode\optional{, bufsize}}}
Executes \var{cmd} as a sub-process.  Returns the file objects
\code{(\var{child_stdin}, \var{child_stdout_and_stderr})}.
Availability: Macintosh, \UNIX, Windows.
\versionadded{2.0}
\end{funcdesc}

(Note that \code{\var{child_stdin}, \var{child_stdout}, and
\var{child_stderr}} are named from the point of view of the child
process, i.e. \var{child_stdin} is the child's standard input.)

This functionality is also available in the \refmodule{popen2} module
using functions of the same names, but the return values of those
functions have a different order.


\subsection{File Descriptor Operations \label{os-fd-ops}}

These functions operate on I/O streams referenced using file
descriptors.  

File descriptors are small integers corresponding to a file that has
been opened by the current process.  For example, standard input is
usually file descriptor 0, standard output is 1, and standard error is
2.  Further files opened by a process will then be assigned 3, 4, 5,
and so forth.  The name ``file descriptor'' is slightly deceptive; on
{\UNIX} platforms, sockets and pipes are also referenced by file descriptors.


\begin{funcdesc}{close}{fd}
Close file descriptor \var{fd}.
Availability: Macintosh, \UNIX, Windows.

\begin{notice}
This function is intended for low-level I/O and must be applied
to a file descriptor as returned by \function{open()} or
\function{pipe()}.  To close a ``file object'' returned by the
built-in function \function{open()} or by \function{popen()} or
\function{fdopen()}, use its \method{close()} method.
\end{notice}
\end{funcdesc}

\begin{funcdesc}{dup}{fd}
Return a duplicate of file descriptor \var{fd}.
Availability: Macintosh, \UNIX, Windows.
\end{funcdesc}

\begin{funcdesc}{dup2}{fd, fd2}
Duplicate file descriptor \var{fd} to \var{fd2}, closing the latter
first if necessary.
Availability: Macintosh, \UNIX, Windows.
\end{funcdesc}

\begin{funcdesc}{fdatasync}{fd}
Force write of file with filedescriptor \var{fd} to disk.
Does not force update of metadata.
Availability: \UNIX.
\end{funcdesc}

\begin{funcdesc}{fpathconf}{fd, name}
Return system configuration information relevant to an open file.
\var{name} specifies the configuration value to retrieve; it may be a
string which is the name of a defined system value; these names are
specified in a number of standards (\POSIX.1, \UNIX{} 95, \UNIX{} 98, and
others).  Some platforms define additional names as well.  The names
known to the host operating system are given in the
\code{pathconf_names} dictionary.  For configuration variables not
included in that mapping, passing an integer for \var{name} is also
accepted.
Availability: Macintosh, \UNIX.

If \var{name} is a string and is not known, \exception{ValueError} is
raised.  If a specific value for \var{name} is not supported by the
host system, even if it is included in \code{pathconf_names}, an
\exception{OSError} is raised with \constant{errno.EINVAL} for the
error number.
\end{funcdesc}

\begin{funcdesc}{fstat}{fd}
Return status for file descriptor \var{fd}, like \function{stat()}.
Availability: Macintosh, \UNIX, Windows.
\end{funcdesc}

\begin{funcdesc}{fstatvfs}{fd}
Return information about the filesystem containing the file associated
with file descriptor \var{fd}, like \function{statvfs()}.
Availability: \UNIX.
\end{funcdesc}

\begin{funcdesc}{fsync}{fd}
Force write of file with filedescriptor \var{fd} to disk.  On \UNIX,
this calls the native \cfunction{fsync()} function; on Windows, the
MS \cfunction{_commit()} function.

If you're starting with a Python file object \var{f}, first do
\code{\var{f}.flush()}, and then do \code{os.fsync(\var{f}.fileno())},
to ensure that all internal buffers associated with \var{f} are written
to disk.
Availability: Macintosh, \UNIX, and Windows starting in 2.2.3.
\end{funcdesc}

\begin{funcdesc}{ftruncate}{fd, length}
Truncate the file corresponding to file descriptor \var{fd},
so that it is at most \var{length} bytes in size.
Availability: Macintosh, \UNIX.
\end{funcdesc}

\begin{funcdesc}{isatty}{fd}
Return \code{True} if the file descriptor \var{fd} is open and
connected to a tty(-like) device, else \code{False}.
Availability: Macintosh, \UNIX.
\end{funcdesc}

\begin{funcdesc}{lseek}{fd, pos, how}
Set the current position of file descriptor \var{fd} to position
\var{pos}, modified by \var{how}: \code{0} to set the position
relative to the beginning of the file; \code{1} to set it relative to
the current position; \code{2} to set it relative to the end of the
file.
Availability: Macintosh, \UNIX, Windows.
\end{funcdesc}

\begin{funcdesc}{open}{file, flags\optional{, mode}}
Open the file \var{file} and set various flags according to
\var{flags} and possibly its mode according to \var{mode}.
The default \var{mode} is \code{0777} (octal), and the current umask
value is first masked out.  Return the file descriptor for the newly
opened file.
Availability: Macintosh, \UNIX, Windows.

For a description of the flag and mode values, see the C run-time
documentation; flag constants (like \constant{O_RDONLY} and
\constant{O_WRONLY}) are defined in this module too (see below).

\begin{notice}
This function is intended for low-level I/O.  For normal usage,
use the built-in function \function{open()}, which returns a ``file
object'' with \method{read()} and \method{write()} methods (and many
more).
\end{notice}
\end{funcdesc}

\begin{funcdesc}{openpty}{}
Open a new pseudo-terminal pair. Return a pair of file descriptors
\code{(\var{master}, \var{slave})} for the pty and the tty,
respectively. For a (slightly) more portable approach, use the
\refmodule{pty}\refstmodindex{pty} module.
Availability: Macintosh, Some flavors of \UNIX.
\end{funcdesc}

\begin{funcdesc}{pipe}{}
Create a pipe.  Return a pair of file descriptors \code{(\var{r},
\var{w})} usable for reading and writing, respectively.
Availability: Macintosh, \UNIX, Windows.
\end{funcdesc}

\begin{funcdesc}{read}{fd, n}
Read at most \var{n} bytes from file descriptor \var{fd}.
Return a string containing the bytes read.  If the end of the file
referred to by \var{fd} has been reached, an empty string is
returned.
Availability: Macintosh, \UNIX, Windows.

\begin{notice}
This function is intended for low-level I/O and must be applied
to a file descriptor as returned by \function{open()} or
\function{pipe()}.  To read a ``file object'' returned by the
built-in function \function{open()} or by \function{popen()} or
\function{fdopen()}, or \code{sys.stdin}, use its
\method{read()} or \method{readline()} methods.
\end{notice}
\end{funcdesc}

\begin{funcdesc}{tcgetpgrp}{fd}
Return the process group associated with the terminal given by
\var{fd} (an open file descriptor as returned by \function{open()}).
Availability: Macintosh, \UNIX.
\end{funcdesc}

\begin{funcdesc}{tcsetpgrp}{fd, pg}
Set the process group associated with the terminal given by
\var{fd} (an open file descriptor as returned by \function{open()})
to \var{pg}.
Availability: Macintosh, \UNIX.
\end{funcdesc}

\begin{funcdesc}{ttyname}{fd}
Return a string which specifies the terminal device associated with
file-descriptor \var{fd}.  If \var{fd} is not associated with a terminal
device, an exception is raised.
Availability:Macintosh,  \UNIX.
\end{funcdesc}

\begin{funcdesc}{write}{fd, str}
Write the string \var{str} to file descriptor \var{fd}.
Return the number of bytes actually written.
Availability: Macintosh, \UNIX, Windows.

\begin{notice}
This function is intended for low-level I/O and must be applied
to a file descriptor as returned by \function{open()} or
\function{pipe()}.  To write a ``file object'' returned by the
built-in function \function{open()} or by \function{popen()} or
\function{fdopen()}, or \code{sys.stdout} or \code{sys.stderr}, use
its \method{write()} method.
\end{notice}
\end{funcdesc}


The following data items are available for use in constructing the
\var{flags} parameter to the \function{open()} function.  Some items will
not be available on all platforms.  For descriptions of their availability
and use, consult \manpage{open}{2}.

\begin{datadesc}{O_RDONLY}
\dataline{O_WRONLY}
\dataline{O_RDWR}
\dataline{O_APPEND}
\dataline{O_CREAT}
\dataline{O_EXCL}
\dataline{O_TRUNC}
Options for the \var{flag} argument to the \function{open()} function.
These can be bit-wise OR'd together.
Availability: Macintosh, \UNIX, Windows.
\end{datadesc}

\begin{datadesc}{O_DSYNC}
\dataline{O_RSYNC}
\dataline{O_SYNC}
\dataline{O_NDELAY}
\dataline{O_NONBLOCK}
\dataline{O_NOCTTY}
\dataline{O_SHLOCK}
\dataline{O_EXLOCK}
More options for the \var{flag} argument to the \function{open()} function.
Availability: Macintosh, \UNIX.
\end{datadesc}

\begin{datadesc}{O_BINARY}
Option for the \var{flag} argument to the \function{open()} function.
This can be bit-wise OR'd together with those listed above.
Availability: Windows.
% XXX need to check on the availability of this one.
\end{datadesc}

\begin{datadesc}{O_NOINHERIT}
\dataline{O_SHORT_LIVED}
\dataline{O_TEMPORARY}
\dataline{O_RANDOM}
\dataline{O_SEQUENTIAL}
\dataline{O_TEXT}
Options for the \var{flag} argument to the \function{open()} function.
These can be bit-wise OR'd together.
Availability: Windows.
\end{datadesc}

\begin{datadesc}{SEEK_SET}
\dataline{SEEK_CUR}
\dataline{SEEK_END}
Parameters to the \function{lseek()} function.
Their values are 0, 1, and 2, respectively.
Availability: Windows, Macintosh, \UNIX.
\versionadded{2.5}
\end{datadesc}

\subsection{Files and Directories \label{os-file-dir}}

\begin{funcdesc}{access}{path, mode}
Use the real uid/gid to test for access to \var{path}.  Note that most
operations will use the effective uid/gid, therefore this routine can
be used in a suid/sgid environment to test if the invoking user has the
specified access to \var{path}.  \var{mode} should be \constant{F_OK}
to test the existence of \var{path}, or it can be the inclusive OR of
one or more of \constant{R_OK}, \constant{W_OK}, and \constant{X_OK} to
test permissions.  Return \constant{True} if access is allowed,
\constant{False} if not.
See the \UNIX{} man page \manpage{access}{2} for more information.
Availability: Macintosh, \UNIX, Windows.

\note{Using \function{access()} to check if a user is authorized to e.g.
open a file before actually doing so using \function{open()} creates a 
security hole, because the user might exploit the short time interval 
between checking and opening the file to manipulate it.}

\note{I/O operations may fail even when \function{access()}
indicates that they would succeed, particularly for operations
on network filesystems which may have permissions semantics
beyond the usual \POSIX{} permission-bit model.}
\end{funcdesc}

\begin{datadesc}{F_OK}
  Value to pass as the \var{mode} parameter of \function{access()} to
  test the existence of \var{path}.
\end{datadesc}

\begin{datadesc}{R_OK}
  Value to include in the \var{mode} parameter of \function{access()}
  to test the readability of \var{path}.
\end{datadesc}

\begin{datadesc}{W_OK}
  Value to include in the \var{mode} parameter of \function{access()}
  to test the writability of \var{path}.
\end{datadesc}

\begin{datadesc}{X_OK}
  Value to include in the \var{mode} parameter of \function{access()}
  to determine if \var{path} can be executed.
\end{datadesc}

\begin{funcdesc}{chdir}{path}
\index{directory!changing}
Change the current working directory to \var{path}.
Availability: Macintosh, \UNIX, Windows.
\end{funcdesc}

\begin{funcdesc}{fchdir}{fd}
Change the current working directory to the directory represented by
the file descriptor \var{fd}.  The descriptor must refer to an opened
directory, not an open file.
Availability: \UNIX.
\versionadded{2.3}
\end{funcdesc}

\begin{funcdesc}{getcwd}{}
Return a string representing the current working directory.
Availability: Macintosh, \UNIX, Windows.
\end{funcdesc}

\begin{funcdesc}{getcwdu}{}
Return a Unicode object representing the current working directory.
Availability: Macintosh, \UNIX, Windows.
\versionadded{2.3}
\end{funcdesc}

\begin{funcdesc}{chroot}{path}
Change the root directory of the current process to \var{path}.
Availability: Macintosh, \UNIX.
\versionadded{2.2}
\end{funcdesc}

\begin{funcdesc}{chmod}{path, mode}
Change the mode of \var{path} to the numeric \var{mode}.
\var{mode} may take one of the following values
(as defined in the \module{stat} module) or bitwise or-ed
combinations of them:
\begin{itemize}
  \item \code{S_ISUID}
  \item \code{S_ISGID}
  \item \code{S_ENFMT}
  \item \code{S_ISVTX}
  \item \code{S_IREAD}
  \item \code{S_IWRITE}
  \item \code{S_IEXEC}
  \item \code{S_IRWXU}
  \item \code{S_IRUSR}
  \item \code{S_IWUSR}
  \item \code{S_IXUSR}
  \item \code{S_IRWXG}
  \item \code{S_IRGRP}
  \item \code{S_IWGRP}
  \item \code{S_IXGRP}
  \item \code{S_IRWXO}
  \item \code{S_IROTH}
  \item \code{S_IWOTH}
  \item \code{S_IXOTH}
\end{itemize}
Availability: Macintosh, \UNIX, Windows.

\note{Although Windows supports \function{chmod()}, you can only 
set the file's read-only flag with it (via the \code{S_IWRITE} 
and \code{S_IREAD} constants or a corresponding integer value). 
All other bits are ignored.}
\end{funcdesc}

\begin{funcdesc}{chown}{path, uid, gid}
Change the owner and group id of \var{path} to the numeric \var{uid}
and \var{gid}. To leave one of the ids unchanged, set it to -1.
Availability: Macintosh, \UNIX.
\end{funcdesc}

\begin{funcdesc}{lchown}{path, uid, gid}
Change the owner and group id of \var{path} to the numeric \var{uid}
and gid. This function will not follow symbolic links.
Availability: Macintosh, \UNIX.
\versionadded{2.3}
\end{funcdesc}

\begin{funcdesc}{link}{src, dst}
Create a hard link pointing to \var{src} named \var{dst}.
Availability: Macintosh, \UNIX.
\end{funcdesc}

\begin{funcdesc}{listdir}{path}
Return a list containing the names of the entries in the directory.
The list is in arbitrary order.  It does not include the special
entries \code{'.'} and \code{'..'} even if they are present in the
directory.
Availability: Macintosh, \UNIX, Windows.

\versionchanged[On Windows NT/2k/XP and \UNIX, if \var{path} is a Unicode
object, the result will be a list of Unicode objects.]{2.3}
\end{funcdesc}

\begin{funcdesc}{lstat}{path}
Like \function{stat()}, but do not follow symbolic links.
Availability: Macintosh, \UNIX.
\end{funcdesc}

\begin{funcdesc}{mkfifo}{path\optional{, mode}}
Create a FIFO (a named pipe) named \var{path} with numeric mode
\var{mode}.  The default \var{mode} is \code{0666} (octal).  The current
umask value is first masked out from the mode.
Availability: Macintosh, \UNIX.

FIFOs are pipes that can be accessed like regular files.  FIFOs exist
until they are deleted (for example with \function{os.unlink()}).
Generally, FIFOs are used as rendezvous between ``client'' and
``server'' type processes: the server opens the FIFO for reading, and
the client opens it for writing.  Note that \function{mkfifo()}
doesn't open the FIFO --- it just creates the rendezvous point.
\end{funcdesc}

\begin{funcdesc}{mknod}{filename\optional{, mode=0600, device}}
Create a filesystem node (file, device special file or named pipe)
named \var{filename}. \var{mode} specifies both the permissions to use and
the type of node to be created, being combined (bitwise OR) with one
of S_IFREG, S_IFCHR, S_IFBLK, and S_IFIFO (those constants are
available in \module{stat}). For S_IFCHR and S_IFBLK, \var{device}
defines the newly created device special file (probably using
\function{os.makedev()}), otherwise it is ignored.
\versionadded{2.3}
\end{funcdesc}

\begin{funcdesc}{major}{device}
Extracts the device major number from a raw device number (usually
the \member{st_dev} or \member{st_rdev} field from \ctype{stat}).
\versionadded{2.3}
\end{funcdesc}

\begin{funcdesc}{minor}{device}
Extracts the device minor number from a raw device number (usually
the \member{st_dev} or \member{st_rdev} field from \ctype{stat}).
\versionadded{2.3}
\end{funcdesc}

\begin{funcdesc}{makedev}{major, minor}
Composes a raw device number from the major and minor device numbers.
\versionadded{2.3}
\end{funcdesc}

\begin{funcdesc}{mkdir}{path\optional{, mode}}
Create a directory named \var{path} with numeric mode \var{mode}.
The default \var{mode} is \code{0777} (octal).  On some systems,
\var{mode} is ignored.  Where it is used, the current umask value is
first masked out.
Availability: Macintosh, \UNIX, Windows.
\end{funcdesc}

\begin{funcdesc}{makedirs}{path\optional{, mode}}
Recursive directory creation function.\index{directory!creating}
\index{UNC paths!and \function{os.makedirs()}}
Like \function{mkdir()},
but makes all intermediate-level directories needed to contain the
leaf directory.  Throws an \exception{error} exception if the leaf
directory already exists or cannot be created.  The default \var{mode}
is \code{0777} (octal).  On some systems, \var{mode} is ignored.
Where it is used, the current umask value is first masked out.
\note{\function{makedirs()} will become confused if the path elements
to create include \var{os.pardir}.}
\versionadded{1.5.2}
\versionchanged[This function now handles UNC paths correctly]{2.3}
\end{funcdesc}

\begin{funcdesc}{pathconf}{path, name}
Return system configuration information relevant to a named file.
\var{name} specifies the configuration value to retrieve; it may be a
string which is the name of a defined system value; these names are
specified in a number of standards (\POSIX.1, \UNIX{} 95, \UNIX{} 98, and
others).  Some platforms define additional names as well.  The names
known to the host operating system are given in the
\code{pathconf_names} dictionary.  For configuration variables not
included in that mapping, passing an integer for \var{name} is also
accepted.
Availability: Macintosh, \UNIX.

If \var{name} is a string and is not known, \exception{ValueError} is
raised.  If a specific value for \var{name} is not supported by the
host system, even if it is included in \code{pathconf_names}, an
\exception{OSError} is raised with \constant{errno.EINVAL} for the
error number.
\end{funcdesc}

\begin{datadesc}{pathconf_names}
Dictionary mapping names accepted by \function{pathconf()} and
\function{fpathconf()} to the integer values defined for those names
by the host operating system.  This can be used to determine the set
of names known to the system.
Availability: Macintosh, \UNIX.
\end{datadesc}

\begin{funcdesc}{readlink}{path}
Return a string representing the path to which the symbolic link
points.  The result may be either an absolute or relative pathname; if
it is relative, it may be converted to an absolute pathname using
\code{os.path.join(os.path.dirname(\var{path}), \var{result})}.
Availability: Macintosh, \UNIX.
\end{funcdesc}

\begin{funcdesc}{remove}{path}
Remove the file \var{path}.  If \var{path} is a directory,
\exception{OSError} is raised; see \function{rmdir()} below to remove
a directory.  This is identical to the \function{unlink()} function
documented below.  On Windows, attempting to remove a file that is in
use causes an exception to be raised; on \UNIX, the directory entry is
removed but the storage allocated to the file is not made available
until the original file is no longer in use.
Availability: Macintosh, \UNIX, Windows.
\end{funcdesc}

\begin{funcdesc}{removedirs}{path}
\index{directory!deleting}
Removes directories recursively.  Works like
\function{rmdir()} except that, if the leaf directory is
successfully removed, \function{removedirs()} 
tries to successively remove every parent directory mentioned in 
\var{path} until an error is raised (which is ignored, because
it generally means that a parent directory is not empty).
For example, \samp{os.removedirs('foo/bar/baz')} will first remove
the directory \samp{'foo/bar/baz'}, and then remove \samp{'foo/bar'}
and \samp{'foo'} if they are empty.
Raises \exception{OSError} if the leaf directory could not be 
successfully removed.
\versionadded{1.5.2}
\end{funcdesc}

\begin{funcdesc}{rename}{src, dst}
Rename the file or directory \var{src} to \var{dst}.  If \var{dst} is
a directory, \exception{OSError} will be raised.  On \UNIX, if
\var{dst} exists and is a file, it will be removed silently if the
user has permission.  The operation may fail on some \UNIX{} flavors
if \var{src} and \var{dst} are on different filesystems.  If
successful, the renaming will be an atomic operation (this is a
\POSIX{} requirement).  On Windows, if \var{dst} already exists,
\exception{OSError} will be raised even if it is a file; there may be
no way to implement an atomic rename when \var{dst} names an existing
file.
Availability: Macintosh, \UNIX, Windows.
\end{funcdesc}

\begin{funcdesc}{renames}{old, new}
Recursive directory or file renaming function.
Works like \function{rename()}, except creation of any intermediate
directories needed to make the new pathname good is attempted first.
After the rename, directories corresponding to rightmost path segments
of the old name will be pruned away using \function{removedirs()}.
\versionadded{1.5.2}

\begin{notice}
This function can fail with the new directory structure made if
you lack permissions needed to remove the leaf directory or file.
\end{notice}
\end{funcdesc}

\begin{funcdesc}{rmdir}{path}
Remove the directory \var{path}.
Availability: Macintosh, \UNIX, Windows.
\end{funcdesc}

\begin{funcdesc}{stat}{path}
Perform a \cfunction{stat()} system call on the given path.  The
return value is an object whose attributes correspond to the members of
the \ctype{stat} structure, namely:
\member{st_mode} (protection bits),
\member{st_ino} (inode number),
\member{st_dev} (device),
\member{st_nlink} (number of hard links),
\member{st_uid} (user ID of owner),
\member{st_gid} (group ID of owner),
\member{st_size} (size of file, in bytes),
\member{st_atime} (time of most recent access),
\member{st_mtime} (time of most recent content modification),
\member{st_ctime}
(platform dependent; time of most recent metadata change on \UNIX, or
the time of creation on Windows):

\begin{verbatim}
>>> import os
>>> statinfo = os.stat('somefile.txt')
>>> statinfo
(33188, 422511L, 769L, 1, 1032, 100, 926L, 1105022698,1105022732, 1105022732)
>>> statinfo.st_size
926L
>>>
\end{verbatim}

\versionchanged [If \function{stat_float_times} returns true, the time
values are floats, measuring seconds. Fractions of a second may be
reported if the system supports that. On Mac OS, the times are always
floats. See \function{stat_float_times} for further discussion. ]{2.3}

On some \UNIX{} systems (such as Linux), the following attributes may
also be available:
\member{st_blocks} (number of blocks allocated for file),
\member{st_blksize} (filesystem blocksize),
\member{st_rdev} (type of device if an inode device).
\member{st_flags} (user defined flags for file).

On other \UNIX{} systems (such as FreeBSD), the following attributes
may be available (but may be only filled out of root tries to
use them:
\member{st_gen} (file generation number),
\member{st_birthtime} (time of file creation).

On Mac OS systems, the following attributes may also be available:
\member{st_rsize},
\member{st_creator},
\member{st_type}.

On RISCOS systems, the following attributes are also available:
\member{st_ftype} (file type),
\member{st_attrs} (attributes),
\member{st_obtype} (object type).

For backward compatibility, the return value of \function{stat()} is
also accessible as a tuple of at least 10 integers giving the most
important (and portable) members of the \ctype{stat} structure, in the
order
\member{st_mode},
\member{st_ino},
\member{st_dev},
\member{st_nlink},
\member{st_uid},
\member{st_gid},
\member{st_size},
\member{st_atime},
\member{st_mtime},
\member{st_ctime}.
More items may be added at the end by some implementations.
The standard module \refmodule{stat}\refstmodindex{stat} defines
functions and constants that are useful for extracting information
from a \ctype{stat} structure.
(On Windows, some items are filled with dummy values.)

\note{The exact meaning and resolution of the \member{st_atime},
 \member{st_mtime}, and \member{st_ctime} members depends on the
 operating system and the file system.  For example, on Windows systems
 using the FAT or FAT32 file systems, \member{st_mtime} has 2-second
 resolution, and \member{st_atime} has only 1-day resolution.  See
 your operating system documentation for details.}

Availability: Macintosh, \UNIX, Windows.

\versionchanged
[Added access to values as attributes of the returned object]{2.2}
\versionchanged[Added st_gen, st_birthtime]{2.5}
\end{funcdesc}

\begin{funcdesc}{stat_float_times}{\optional{newvalue}}
Determine whether \class{stat_result} represents time stamps as float
objects.  If newval is True, future calls to stat() return floats, if
it is False, future calls return ints.  If newval is omitted, return
the current setting.

For compatibility with older Python versions, accessing
\class{stat_result} as a tuple always returns integers.

\versionchanged[Python now returns float values by default. Applications
which do not work correctly with floating point time stamps can use
this function to restore the old behaviour]{2.5}

The resolution of the timestamps (i.e. the smallest possible fraction)
depends on the system. Some systems only support second resolution;
on these systems, the fraction will always be zero.

It is recommended that this setting is only changed at program startup
time in the \var{__main__} module; libraries should never change this
setting. If an application uses a library that works incorrectly if
floating point time stamps are processed, this application should turn
the feature off until the library has been corrected.

\end{funcdesc}

\begin{funcdesc}{statvfs}{path}
Perform a \cfunction{statvfs()} system call on the given path.  The
return value is an object whose attributes describe the filesystem on
the given path, and correspond to the members of the
\ctype{statvfs} structure, namely:
\member{f_frsize},
\member{f_blocks},
\member{f_bfree},
\member{f_bavail},
\member{f_files},
\member{f_ffree},
\member{f_favail},
\member{f_flag},
\member{f_namemax}.
Availability: \UNIX.

For backward compatibility, the return value is also accessible as a
tuple whose values correspond to the attributes, in the order given above.
The standard module \refmodule{statvfs}\refstmodindex{statvfs}
defines constants that are useful for extracting information
from a \ctype{statvfs} structure when accessing it as a sequence; this
remains useful when writing code that needs to work with versions of
Python that don't support accessing the fields as attributes.

\versionchanged
[Added access to values as attributes of the returned object]{2.2}
\end{funcdesc}

\begin{funcdesc}{symlink}{src, dst}
Create a symbolic link pointing to \var{src} named \var{dst}.
Availability: \UNIX.
\end{funcdesc}

\begin{funcdesc}{tempnam}{\optional{dir\optional{, prefix}}}
Return a unique path name that is reasonable for creating a temporary
file.  This will be an absolute path that names a potential directory
entry in the directory \var{dir} or a common location for temporary
files if \var{dir} is omitted or \code{None}.  If given and not
\code{None}, \var{prefix} is used to provide a short prefix to the
filename.  Applications are responsible for properly creating and
managing files created using paths returned by \function{tempnam()};
no automatic cleanup is provided.
On \UNIX, the environment variable \envvar{TMPDIR} overrides
\var{dir}, while on Windows the \envvar{TMP} is used.  The specific
behavior of this function depends on the C library implementation;
some aspects are underspecified in system documentation.
\warning{Use of \function{tempnam()} is vulnerable to symlink attacks;
consider using \function{tmpfile()} (section \ref{os-newstreams})
instead.}  Availability: Macintosh, \UNIX, Windows.
\end{funcdesc}

\begin{funcdesc}{tmpnam}{}
Return a unique path name that is reasonable for creating a temporary
file.  This will be an absolute path that names a potential directory
entry in a common location for temporary files.  Applications are
responsible for properly creating and managing files created using
paths returned by \function{tmpnam()}; no automatic cleanup is
provided.
\warning{Use of \function{tmpnam()} is vulnerable to symlink attacks;
consider using \function{tmpfile()} (section \ref{os-newstreams})
instead.}  Availability: \UNIX, Windows.  This function probably
shouldn't be used on Windows, though: Microsoft's implementation of
\function{tmpnam()} always creates a name in the root directory of the
current drive, and that's generally a poor location for a temp file
(depending on privileges, you may not even be able to open a file
using this name).
\end{funcdesc}

\begin{datadesc}{TMP_MAX}
The maximum number of unique names that \function{tmpnam()} will
generate before reusing names.
\end{datadesc}

\begin{funcdesc}{unlink}{path}
Remove the file \var{path}.  This is the same function as
\function{remove()}; the \function{unlink()} name is its traditional
\UNIX{} name.
Availability: Macintosh, \UNIX, Windows.
\end{funcdesc}

\begin{funcdesc}{utime}{path, times}
Set the access and modified times of the file specified by \var{path}.
If \var{times} is \code{None}, then the file's access and modified
times are set to the current time.  Otherwise, \var{times} must be a
2-tuple of numbers, of the form \code{(\var{atime}, \var{mtime})}
which is used to set the access and modified times, respectively.
Whether a directory can be given for \var{path} depends on whether the
operating system implements directories as files (for example, Windows
does not).  Note that the exact times you set here may not be returned
by a subsequent \function{stat()} call, depending on the resolution
with which your operating system records access and modification times;
see \function{stat()}.
\versionchanged[Added support for \code{None} for \var{times}]{2.0}
Availability: Macintosh, \UNIX, Windows.
\end{funcdesc}

\begin{funcdesc}{walk}{top\optional{, topdown\code{=True}
                       \optional{, onerror\code{=None}}}}
\index{directory!walking}
\index{directory!traversal}
\function{walk()} generates the file names in a directory tree, by
walking the tree either top down or bottom up.
For each directory in the tree rooted at directory \var{top} (including
\var{top} itself), it yields a 3-tuple
\code{(\var{dirpath}, \var{dirnames}, \var{filenames})}.

\var{dirpath} is a string, the path to the directory.  \var{dirnames} is
a list of the names of the subdirectories in \var{dirpath}
(excluding \code{'.'} and \code{'..'}).  \var{filenames} is a list of
the names of the non-directory files in \var{dirpath}.  Note that the
names in the lists contain no path components.  To get a full
path (which begins with \var{top}) to a file or directory in
\var{dirpath}, do \code{os.path.join(\var{dirpath}, \var{name})}.

If optional argument \var{topdown} is true or not specified, the triple
for a directory is generated before the triples for any of its
subdirectories (directories are generated top down).  If \var{topdown} is
false, the triple for a directory is generated after the triples for all
of its subdirectories (directories are generated bottom up).

When \var{topdown} is true, the caller can modify the \var{dirnames} list
in-place (perhaps using \keyword{del} or slice assignment), and
\function{walk()} will only recurse into the subdirectories whose names
remain in \var{dirnames}; this can be used to prune the search,
impose a specific order of visiting, or even to inform \function{walk()}
about directories the caller creates or renames before it resumes
\function{walk()} again.  Modifying \var{dirnames} when \var{topdown} is
false is ineffective, because in bottom-up mode the directories in
\var{dirnames} are generated before \var{dirnames} itself is generated.

By default errors from the \code{os.listdir()} call are ignored.  If
optional argument \var{onerror} is specified, it should be a function;
it will be called with one argument, an os.error instance.  It can
report the error to continue with the walk, or raise the exception
to abort the walk.  Note that the filename is available as the
\code{filename} attribute of the exception object.

\begin{notice}
If you pass a relative pathname, don't change the current working
directory between resumptions of \function{walk()}.  \function{walk()}
never changes the current directory, and assumes that its caller
doesn't either.
\end{notice}

\begin{notice}
On systems that support symbolic links, links to subdirectories appear
in \var{dirnames} lists, but \function{walk()} will not visit them
(infinite loops are hard to avoid when following symbolic links).
To visit linked directories, you can identify them with
\code{os.path.islink(\var{path})}, and invoke \code{walk(\var{path})}
on each directly.
\end{notice}

This example displays the number of bytes taken by non-directory files
in each directory under the starting directory, except that it doesn't
look under any CVS subdirectory:

\begin{verbatim}
import os
from os.path import join, getsize
for root, dirs, files in os.walk('python/Lib/email'):
    print root, "consumes",
    print sum(getsize(join(root, name)) for name in files),
    print "bytes in", len(files), "non-directory files"
    if 'CVS' in dirs:
        dirs.remove('CVS')  # don't visit CVS directories
\end{verbatim}

In the next example, walking the tree bottom up is essential:
\function{rmdir()} doesn't allow deleting a directory before the
directory is empty:

\begin{verbatim}
# Delete everything reachable from the directory named in 'top',
# assuming there are no symbolic links.
# CAUTION:  This is dangerous!  For example, if top == '/', it
# could delete all your disk files.
import os
for root, dirs, files in os.walk(top, topdown=False):
    for name in files:
        os.remove(os.path.join(root, name))
    for name in dirs:
        os.rmdir(os.path.join(root, name))
\end{verbatim}

\versionadded{2.3}
\end{funcdesc}

\subsection{Process Management \label{os-process}}

These functions may be used to create and manage processes.

The various \function{exec*()} functions take a list of arguments for
the new program loaded into the process.  In each case, the first of
these arguments is passed to the new program as its own name rather
than as an argument a user may have typed on a command line.  For the
C programmer, this is the \code{argv[0]} passed to a program's
\cfunction{main()}.  For example, \samp{os.execv('/bin/echo', ['foo',
'bar'])} will only print \samp{bar} on standard output; \samp{foo}
will seem to be ignored.


\begin{funcdesc}{abort}{}
Generate a \constant{SIGABRT} signal to the current process.  On
\UNIX, the default behavior is to produce a core dump; on Windows, the
process immediately returns an exit code of \code{3}.  Be aware that
programs which use \function{signal.signal()} to register a handler
for \constant{SIGABRT} will behave differently.
Availability: Macintosh, \UNIX, Windows.
\end{funcdesc}

\begin{funcdesc}{execl}{path, arg0, arg1, \moreargs}
\funcline{execle}{path, arg0, arg1, \moreargs, env}
\funcline{execlp}{file, arg0, arg1, \moreargs}
\funcline{execlpe}{file, arg0, arg1, \moreargs, env}
\funcline{execv}{path, args}
\funcline{execve}{path, args, env}
\funcline{execvp}{file, args}
\funcline{execvpe}{file, args, env}
These functions all execute a new program, replacing the current
process; they do not return.  On \UNIX, the new executable is loaded
into the current process, and will have the same process ID as the
caller.  Errors will be reported as \exception{OSError} exceptions.

The \character{l} and \character{v} variants of the
\function{exec*()} functions differ in how command-line arguments are
passed.  The \character{l} variants are perhaps the easiest to work
with if the number of parameters is fixed when the code is written;
the individual parameters simply become additional parameters to the
\function{execl*()} functions.  The \character{v} variants are good
when the number of parameters is variable, with the arguments being
passed in a list or tuple as the \var{args} parameter.  In either
case, the arguments to the child process should start with the name of
the command being run, but this is not enforced.

The variants which include a \character{p} near the end
(\function{execlp()}, \function{execlpe()}, \function{execvp()},
and \function{execvpe()}) will use the \envvar{PATH} environment
variable to locate the program \var{file}.  When the environment is
being replaced (using one of the \function{exec*e()} variants,
discussed in the next paragraph), the
new environment is used as the source of the \envvar{PATH} variable.
The other variants, \function{execl()}, \function{execle()},
\function{execv()}, and \function{execve()}, will not use the
\envvar{PATH} variable to locate the executable; \var{path} must
contain an appropriate absolute or relative path.

For \function{execle()}, \function{execlpe()}, \function{execve()},
and \function{execvpe()} (note that these all end in \character{e}),
the \var{env} parameter must be a mapping which is used to define the
environment variables for the new process; the \function{execl()},
\function{execlp()}, \function{execv()}, and \function{execvp()}
all cause the new process to inherit the environment of the current
process.
Availability: Macintosh, \UNIX, Windows.
\end{funcdesc}

\begin{funcdesc}{_exit}{n}
Exit to the system with status \var{n}, without calling cleanup
handlers, flushing stdio buffers, etc.
Availability: Macintosh, \UNIX, Windows.

\begin{notice}
The standard way to exit is \code{sys.exit(\var{n})}.
\function{_exit()} should normally only be used in the child process
after a \function{fork()}.
\end{notice}
\end{funcdesc}

The following exit codes are a defined, and can be used with
\function{_exit()}, although they are not required.  These are
typically used for system programs written in Python, such as a
mail server's external command delivery program.
\note{Some of these may not be available on all \UNIX{} platforms,
since there is some variation.  These constants are defined where they
are defined by the underlying platform.}

\begin{datadesc}{EX_OK}
Exit code that means no error occurred.
Availability: Macintosh, \UNIX.
\versionadded{2.3}
\end{datadesc}

\begin{datadesc}{EX_USAGE}
Exit code that means the command was used incorrectly, such as when
the wrong number of arguments are given.
Availability: Macintosh, \UNIX.
\versionadded{2.3}
\end{datadesc}

\begin{datadesc}{EX_DATAERR}
Exit code that means the input data was incorrect.
Availability: Macintosh, \UNIX.
\versionadded{2.3}
\end{datadesc}

\begin{datadesc}{EX_NOINPUT}
Exit code that means an input file did not exist or was not readable.
Availability: Macintosh, \UNIX.
\versionadded{2.3}
\end{datadesc}

\begin{datadesc}{EX_NOUSER}
Exit code that means a specified user did not exist.
Availability: Macintosh, \UNIX.
\versionadded{2.3}
\end{datadesc}

\begin{datadesc}{EX_NOHOST}
Exit code that means a specified host did not exist.
Availability: Macintosh, \UNIX.
\versionadded{2.3}
\end{datadesc}

\begin{datadesc}{EX_UNAVAILABLE}
Exit code that means that a required service is unavailable.
Availability: Macintosh, \UNIX.
\versionadded{2.3}
\end{datadesc}

\begin{datadesc}{EX_SOFTWARE}
Exit code that means an internal software error was detected.
Availability: Macintosh, \UNIX.
\versionadded{2.3}
\end{datadesc}

\begin{datadesc}{EX_OSERR}
Exit code that means an operating system error was detected, such as
the inability to fork or create a pipe.
Availability: Macintosh, \UNIX.
\versionadded{2.3}
\end{datadesc}

\begin{datadesc}{EX_OSFILE}
Exit code that means some system file did not exist, could not be
opened, or had some other kind of error.
Availability: Macintosh, \UNIX.
\versionadded{2.3}
\end{datadesc}

\begin{datadesc}{EX_CANTCREAT}
Exit code that means a user specified output file could not be created.
Availability: Macintosh, \UNIX.
\versionadded{2.3}
\end{datadesc}

\begin{datadesc}{EX_IOERR}
Exit code that means that an error occurred while doing I/O on some file.
Availability: Macintosh, \UNIX.
\versionadded{2.3}
\end{datadesc}

\begin{datadesc}{EX_TEMPFAIL}
Exit code that means a temporary failure occurred.  This indicates
something that may not really be an error, such as a network
connection that couldn't be made during a retryable operation.
Availability: Macintosh, \UNIX.
\versionadded{2.3}
\end{datadesc}

\begin{datadesc}{EX_PROTOCOL}
Exit code that means that a protocol exchange was illegal, invalid, or
not understood.
Availability: Macintosh, \UNIX.
\versionadded{2.3}
\end{datadesc}

\begin{datadesc}{EX_NOPERM}
Exit code that means that there were insufficient permissions to
perform the operation (but not intended for file system problems).
Availability: Macintosh, \UNIX.
\versionadded{2.3}
\end{datadesc}

\begin{datadesc}{EX_CONFIG}
Exit code that means that some kind of configuration error occurred.
Availability: Macintosh, \UNIX.
\versionadded{2.3}
\end{datadesc}

\begin{datadesc}{EX_NOTFOUND}
Exit code that means something like ``an entry was not found''.
Availability: Macintosh, \UNIX.
\versionadded{2.3}
\end{datadesc}

\begin{funcdesc}{fork}{}
Fork a child process.  Return \code{0} in the child, the child's
process id in the parent.
Availability: Macintosh, \UNIX.
\end{funcdesc}

\begin{funcdesc}{forkpty}{}
Fork a child process, using a new pseudo-terminal as the child's
controlling terminal. Return a pair of \code{(\var{pid}, \var{fd})},
where \var{pid} is \code{0} in the child, the new child's process id
in the parent, and \var{fd} is the file descriptor of the master end
of the pseudo-terminal.  For a more portable approach, use the
\refmodule{pty} module.
Availability: Macintosh, Some flavors of \UNIX.
\end{funcdesc}

\begin{funcdesc}{kill}{pid, sig}
\index{process!killing}
\index{process!signalling}
Send signal \var{sig} to the process \var{pid}.  Constants for the
specific signals available on the host platform are defined in the
\refmodule{signal} module.
Availability: Macintosh, \UNIX.
\end{funcdesc}

\begin{funcdesc}{killpg}{pgid, sig}
\index{process!killing}
\index{process!signalling}
Send the signal \var{sig} to the process group \var{pgid}.
Availability: Macintosh, \UNIX.
\versionadded{2.3}
\end{funcdesc}

\begin{funcdesc}{nice}{increment}
Add \var{increment} to the process's ``niceness''.  Return the new
niceness.
Availability: Macintosh, \UNIX.
\end{funcdesc}

\begin{funcdesc}{plock}{op}
Lock program segments into memory.  The value of \var{op}
(defined in \code{<sys/lock.h>}) determines which segments are locked.
Availability: Macintosh, \UNIX.
\end{funcdesc}

\begin{funcdescni}{popen}{\unspecified}
\funclineni{popen2}{\unspecified}
\funclineni{popen3}{\unspecified}
\funclineni{popen4}{\unspecified}
Run child processes, returning opened pipes for communications.  These
functions are described in section \ref{os-newstreams}.
\end{funcdescni}

\begin{funcdesc}{spawnl}{mode, path, \moreargs}
\funcline{spawnle}{mode, path, \moreargs, env}
\funcline{spawnlp}{mode, file, \moreargs}
\funcline{spawnlpe}{mode, file, \moreargs, env}
\funcline{spawnv}{mode, path, args}
\funcline{spawnve}{mode, path, args, env}
\funcline{spawnvp}{mode, file, args}
\funcline{spawnvpe}{mode, file, args, env}
Execute the program \var{path} in a new process.  If \var{mode} is
\constant{P_NOWAIT}, this function returns the process ID of the new
process; if \var{mode} is \constant{P_WAIT}, returns the process's
exit code if it exits normally, or \code{-\var{signal}}, where
\var{signal} is the signal that killed the process.  On Windows, the
process ID will actually be the process handle, so can be used with
the \function{waitpid()} function.

The \character{l} and \character{v} variants of the
\function{spawn*()} functions differ in how command-line arguments are
passed.  The \character{l} variants are perhaps the easiest to work
with if the number of parameters is fixed when the code is written;
the individual parameters simply become additional parameters to the
\function{spawnl*()} functions.  The \character{v} variants are good
when the number of parameters is variable, with the arguments being
passed in a list or tuple as the \var{args} parameter.  In either
case, the arguments to the child process must start with the name of
the command being run.

The variants which include a second \character{p} near the end
(\function{spawnlp()}, \function{spawnlpe()}, \function{spawnvp()},
and \function{spawnvpe()}) will use the \envvar{PATH} environment
variable to locate the program \var{file}.  When the environment is
being replaced (using one of the \function{spawn*e()} variants,
discussed in the next paragraph), the new environment is used as the
source of the \envvar{PATH} variable.  The other variants,
\function{spawnl()}, \function{spawnle()}, \function{spawnv()}, and
\function{spawnve()}, will not use the \envvar{PATH} variable to
locate the executable; \var{path} must contain an appropriate absolute
or relative path.

For \function{spawnle()}, \function{spawnlpe()}, \function{spawnve()},
and \function{spawnvpe()} (note that these all end in \character{e}),
the \var{env} parameter must be a mapping which is used to define the
environment variables for the new process; the \function{spawnl()},
\function{spawnlp()}, \function{spawnv()}, and \function{spawnvp()}
all cause the new process to inherit the environment of the current
process.

As an example, the following calls to \function{spawnlp()} and
\function{spawnvpe()} are equivalent:

\begin{verbatim}
import os
os.spawnlp(os.P_WAIT, 'cp', 'cp', 'index.html', '/dev/null')

L = ['cp', 'index.html', '/dev/null']
os.spawnvpe(os.P_WAIT, 'cp', L, os.environ)
\end{verbatim}

Availability: \UNIX, Windows.  \function{spawnlp()},
\function{spawnlpe()}, \function{spawnvp()} and \function{spawnvpe()}
are not available on Windows.
\versionadded{1.6}
\end{funcdesc}

\begin{datadesc}{P_NOWAIT}
\dataline{P_NOWAITO}
Possible values for the \var{mode} parameter to the \function{spawn*()}
family of functions.  If either of these values is given, the
\function{spawn*()} functions will return as soon as the new process
has been created, with the process ID as the return value.
Availability: Macintosh, \UNIX, Windows.
\versionadded{1.6}
\end{datadesc}

\begin{datadesc}{P_WAIT}
Possible value for the \var{mode} parameter to the \function{spawn*()}
family of functions.  If this is given as \var{mode}, the
\function{spawn*()} functions will not return until the new process
has run to completion and will return the exit code of the process the
run is successful, or \code{-\var{signal}} if a signal kills the
process.
Availability: Macintosh, \UNIX, Windows.
\versionadded{1.6}
\end{datadesc}

\begin{datadesc}{P_DETACH}
\dataline{P_OVERLAY}
Possible values for the \var{mode} parameter to the
\function{spawn*()} family of functions.  These are less portable than
those listed above.
\constant{P_DETACH} is similar to \constant{P_NOWAIT}, but the new
process is detached from the console of the calling process.
If \constant{P_OVERLAY} is used, the current process will be replaced;
the \function{spawn*()} function will not return.
Availability: Windows.
\versionadded{1.6}
\end{datadesc}

\begin{funcdesc}{startfile}{path}
Start a file with its associated application.  This acts like
double-clicking the file in Windows Explorer, or giving the file name
as an argument to the \program{start} command from the interactive
command shell: the file is opened with whatever application (if any)
its extension is associated.

\function{startfile()} returns as soon as the associated application
is launched.  There is no option to wait for the application to close,
and no way to retrieve the application's exit status.  The \var{path}
parameter is relative to the current directory.  If you want to use an
absolute path, make sure the first character is not a slash
(\character{/}); the underlying Win32 \cfunction{ShellExecute()}
function doesn't work if it is.  Use the \function{os.path.normpath()}
function to ensure that the path is properly encoded for Win32.
Availability: Windows.
\versionadded{2.0}
\end{funcdesc}

\begin{funcdesc}{system}{command}
Execute the command (a string) in a subshell.  This is implemented by
calling the Standard C function \cfunction{system()}, and has the
same limitations.  Changes to \code{posix.environ}, \code{sys.stdin},
etc.\ are not reflected in the environment of the executed command.

On \UNIX, the return value is the exit status of the process encoded in the
format specified for \function{wait()}.  Note that \POSIX{} does not
specify the meaning of the return value of the C \cfunction{system()}
function, so the return value of the Python function is system-dependent.

On Windows, the return value is that returned by the system shell after
running \var{command}, given by the Windows environment variable
\envvar{COMSPEC}: on \program{command.com} systems (Windows 95, 98 and ME)
this is always \code{0}; on \program{cmd.exe} systems (Windows NT, 2000
and XP) this is the exit status of the command run; on systems using
a non-native shell, consult your shell documentation.

Availability: Macintosh, \UNIX, Windows.
\end{funcdesc}

\begin{funcdesc}{times}{}
Return a 5-tuple of floating point numbers indicating accumulated
(processor or other)
times, in seconds.  The items are: user time, system time, children's
user time, children's system time, and elapsed real time since a fixed
point in the past, in that order.  See the \UNIX{} manual page
\manpage{times}{2} or the corresponding Windows Platform API
documentation.
Availability: Macintosh, \UNIX, Windows.
\end{funcdesc}

\begin{funcdesc}{wait}{}
Wait for completion of a child process, and return a tuple containing
its pid and exit status indication: a 16-bit number, whose low byte is
the signal number that killed the process, and whose high byte is the
exit status (if the signal number is zero); the high bit of the low
byte is set if a core file was produced.
Availability: Macintosh, \UNIX.
\end{funcdesc}

\begin{funcdesc}{waitpid}{pid, options}
The details of this function differ on \UNIX{} and Windows.

On \UNIX:
Wait for completion of a child process given by process id \var{pid},
and return a tuple containing its process id and exit status
indication (encoded as for \function{wait()}).  The semantics of the
call are affected by the value of the integer \var{options}, which
should be \code{0} for normal operation.

If \var{pid} is greater than \code{0}, \function{waitpid()} requests
status information for that specific process.  If \var{pid} is
\code{0}, the request is for the status of any child in the process
group of the current process.  If \var{pid} is \code{-1}, the request
pertains to any child of the current process.  If \var{pid} is less
than \code{-1}, status is requested for any process in the process
group \code{-\var{pid}} (the absolute value of \var{pid}).

On Windows:
Wait for completion of a process given by process handle \var{pid},
and return a tuple containing \var{pid},
and its exit status shifted left by 8 bits (shifting makes cross-platform
use of the function easier).
A \var{pid} less than or equal to \code{0} has no special meaning on
Windows, and raises an exception.
The value of integer \var{options} has no effect.
\var{pid} can refer to any process whose id is known, not necessarily a
child process.
The \function{spawn()} functions called with \constant{P_NOWAIT}
return suitable process handles.
\end{funcdesc}

\begin{datadesc}{WNOHANG}
The option for \function{waitpid()} to return immediately if no child
process status is available immediately. The function returns
\code{(0, 0)} in this case.
Availability: Macintosh, \UNIX.
\end{datadesc}

\begin{datadesc}{WCONTINUED}
This option causes child processes to be reported if they have been
continued from a job control stop since their status was last
reported.
Availability: Some \UNIX{} systems.
\versionadded{2.3}
\end{datadesc}

\begin{datadesc}{WUNTRACED}
This option causes child processes to be reported if they have been
stopped but their current state has not been reported since they were
stopped.
Availability: Macintosh, \UNIX.
\versionadded{2.3}
\end{datadesc}

The following functions take a process status code as returned by
\function{system()}, \function{wait()}, or \function{waitpid()} as a
parameter.  They may be used to determine the disposition of a
process.

\begin{funcdesc}{WCOREDUMP}{status}
Returns \code{True} if a core dump was generated for the process,
otherwise it returns \code{False}.
Availability: Macintosh, \UNIX.
\versionadded{2.3}
\end{funcdesc}

\begin{funcdesc}{WIFCONTINUED}{status}
Returns \code{True} if the process has been continued from a job
control stop, otherwise it returns \code{False}.
Availability: \UNIX.
\versionadded{2.3}
\end{funcdesc}

\begin{funcdesc}{WIFSTOPPED}{status}
Returns \code{True} if the process has been stopped, otherwise it
returns \code{False}.
Availability: \UNIX.
\end{funcdesc}

\begin{funcdesc}{WIFSIGNALED}{status}
Returns \code{True} if the process exited due to a signal, otherwise
it returns \code{False}.
Availability: Macintosh, \UNIX.
\end{funcdesc}

\begin{funcdesc}{WIFEXITED}{status}
Returns \code{True} if the process exited using the \manpage{exit}{2}
system call, otherwise it returns \code{False}.
Availability: Macintosh, \UNIX.
\end{funcdesc}

\begin{funcdesc}{WEXITSTATUS}{status}
If \code{WIFEXITED(\var{status})} is true, return the integer
parameter to the \manpage{exit}{2} system call.  Otherwise, the return
value is meaningless.
Availability: Macintosh, \UNIX.
\end{funcdesc}

\begin{funcdesc}{WSTOPSIG}{status}
Return the signal which caused the process to stop.
Availability: Macintosh, \UNIX.
\end{funcdesc}

\begin{funcdesc}{WTERMSIG}{status}
Return the signal which caused the process to exit.
Availability: Macintosh, \UNIX.
\end{funcdesc}


\subsection{Miscellaneous System Information \label{os-path}}


\begin{funcdesc}{confstr}{name}
Return string-valued system configuration values.
\var{name} specifies the configuration value to retrieve; it may be a
string which is the name of a defined system value; these names are
specified in a number of standards (\POSIX, \UNIX{} 95, \UNIX{} 98, and
others).  Some platforms define additional names as well.  The names
known to the host operating system are given in the
\code{confstr_names} dictionary.  For configuration variables not
included in that mapping, passing an integer for \var{name} is also
accepted.
Availability: Macintosh, \UNIX.

If the configuration value specified by \var{name} isn't defined, the
empty string is returned.

If \var{name} is a string and is not known, \exception{ValueError} is
raised.  If a specific value for \var{name} is not supported by the
host system, even if it is included in \code{confstr_names}, an
\exception{OSError} is raised with \constant{errno.EINVAL} for the
error number.
\end{funcdesc}

\begin{datadesc}{confstr_names}
Dictionary mapping names accepted by \function{confstr()} to the
integer values defined for those names by the host operating system.
This can be used to determine the set of names known to the system.
Availability: Macintosh, \UNIX.
\end{datadesc}

\begin{funcdesc}{getloadavg}{}
Return the number of processes in the system run queue averaged over
the last 1, 5, and 15 minutes or raises OSError if the load average
was unobtainable.

\versionadded{2.3}
\end{funcdesc}

\begin{funcdesc}{sysconf}{name}
Return integer-valued system configuration values.
If the configuration value specified by \var{name} isn't defined,
\code{-1} is returned.  The comments regarding the \var{name}
parameter for \function{confstr()} apply here as well; the dictionary
that provides information on the known names is given by
\code{sysconf_names}.
Availability: Macintosh, \UNIX.
\end{funcdesc}

\begin{datadesc}{sysconf_names}
Dictionary mapping names accepted by \function{sysconf()} to the
integer values defined for those names by the host operating system.
This can be used to determine the set of names known to the system.
Availability: Macintosh, \UNIX.
\end{datadesc}


The follow data values are used to support path manipulation
operations.  These are defined for all platforms.

Higher-level operations on pathnames are defined in the
\refmodule{os.path} module.


\begin{datadesc}{curdir}
The constant string used by the operating system to refer to the current
directory.
For example: \code{'.'} for \POSIX{} or \code{':'} for Mac OS 9.
Also available via \module{os.path}.
\end{datadesc}

\begin{datadesc}{pardir}
The constant string used by the operating system to refer to the parent
directory.
For example: \code{'..'} for \POSIX{} or \code{'::'} for Mac OS 9.
Also available via \module{os.path}.
\end{datadesc}

\begin{datadesc}{sep}
The character used by the operating system to separate pathname components,
for example, \character{/} for \POSIX{} or \character{:} for
Mac OS 9.  Note that knowing this is not sufficient to be able to
parse or concatenate pathnames --- use \function{os.path.split()} and
\function{os.path.join()} --- but it is occasionally useful.
Also available via \module{os.path}.
\end{datadesc}

\begin{datadesc}{altsep}
An alternative character used by the operating system to separate pathname
components, or \code{None} if only one separator character exists.  This is
set to \character{/} on Windows systems where \code{sep} is a
backslash.
Also available via \module{os.path}.
\end{datadesc}

\begin{datadesc}{extsep}
The character which separates the base filename from the extension;
for example, the \character{.} in \file{os.py}.
Also available via \module{os.path}.
\versionadded{2.2}
\end{datadesc}

\begin{datadesc}{pathsep}
The character conventionally used by the operating system to separate
search path components (as in \envvar{PATH}), such as \character{:} for
\POSIX{} or \character{;} for Windows.
Also available via \module{os.path}.
\end{datadesc}

\begin{datadesc}{defpath}
The default search path used by \function{exec*p*()} and
\function{spawn*p*()} if the environment doesn't have a \code{'PATH'}
key.
Also available via \module{os.path}.
\end{datadesc}

\begin{datadesc}{linesep}
The string used to separate (or, rather, terminate) lines on the
current platform.  This may be a single character, such as \code{'\e
n'} for \POSIX{} or \code{'\e r'} for Mac OS, or multiple characters,
for example, \code{'\e r\e n'} for Windows.
\end{datadesc}

\begin{datadesc}{devnull}
The file path of the null device.
For example: \code{'/dev/null'} for \POSIX{} or \code{'Dev:Nul'} for
Mac OS 9.
Also available via \module{os.path}.
\versionadded{2.4}
\end{datadesc}


\subsection{Miscellaneous Functions \label{os-miscfunc}}

\begin{funcdesc}{urandom}{n}
Return a string of \var{n} random bytes suitable for cryptographic use.

This function returns random bytes from an OS-specific
randomness source.  The returned data should be unpredictable enough for
cryptographic applications, though its exact quality depends on the OS
implementation.  On a UNIX-like system this will query /dev/urandom, and
on Windows it will use CryptGenRandom.  If a randomness source is not
found, \exception{NotImplementedError} will be raised.
\versionadded{2.4}
\end{funcdesc}





\section{\module{os.path} ---
         Common pathname manipulations}
\declaremodule{standard}{os.path}

\modulesynopsis{Common pathname manipulations.}

This module implements some useful functions on pathnames.
\index{path!operations}

\warning{On Windows, many of these functions do not properly
support UNC pathnames.  \function{splitunc()} and \function{ismount()}
do handle them correctly.}


\begin{funcdesc}{abspath}{path}
Return a normalized absolutized version of the pathname \var{path}.
On most platforms, this is equivalent to
\code{normpath(join(os.getcwd(), \var{path}))}.
\versionadded{1.5.2}
\end{funcdesc}

\begin{funcdesc}{basename}{path}
Return the base name of pathname \var{path}.  This is the second half
of the pair returned by \code{split(\var{path})}.  Note that the
result of this function is different from the
\UNIX{} \program{basename} program; where \program{basename} for
\code{'/foo/bar/'} returns \code{'bar'}, the \function{basename()}
function returns an empty string (\code{''}).
\end{funcdesc}

\begin{funcdesc}{commonprefix}{list}
Return the longest path prefix (taken character-by-character) that is a
prefix of all paths in 
\var{list}.  If \var{list} is empty, return the empty string
(\code{''}).  Note that this may return invalid paths because it works a
character at a time.
\end{funcdesc}

\begin{funcdesc}{dirname}{path}
Return the directory name of pathname \var{path}.  This is the first
half of the pair returned by \code{split(\var{path})}.
\end{funcdesc}

\begin{funcdesc}{exists}{path}
Return \code{True} if \var{path} refers to an existing path.
Returns \code{False} for broken symbolic links.
\end{funcdesc}

\begin{funcdesc}{lexists}{path}
Return \code{True} if \var{path} refers to an existing path.
Returns \code{True} for broken symbolic links.  
Equivalent to \function{exists()} on platforms lacking
\function{os.lstat()}.
\versionadded{2.4}
\end{funcdesc}

\begin{funcdesc}{expanduser}{path}
On \UNIX, return the argument with an initial component of \samp{\~} or
\samp{\~\var{user}} replaced by that \var{user}'s home directory.
An initial \samp{\~} is replaced by the environment variable
\envvar{HOME} if it is set; otherwise the current user's home directory
is looked up in the password directory through the built-in module
\refmodule{pwd}\refbimodindex{pwd}.
An initial \samp{\~\var{user}} is looked up directly in the
password directory.

On Windows, only \samp{\~} is supported; it is replaced by the
environment variable \envvar{HOME} or by a combination of
\envvar{HOMEDRIVE} and \envvar{HOMEPATH}.

If the expansion fails or if the
path does not begin with a tilde, the path is returned unchanged.
\end{funcdesc}

\begin{funcdesc}{expandvars}{path}
Return the argument with environment variables expanded.  Substrings
of the form \samp{\$\var{name}} or \samp{\$\{\var{name}\}} are
replaced by the value of environment variable \var{name}.  Malformed
variable names and references to non-existing variables are left
unchanged.
\end{funcdesc}

\begin{funcdesc}{getatime}{path}
Return the time of last access of \var{path}.  The return
value is a number giving the number of seconds since the epoch (see the 
\refmodule{time} module).  Raise \exception{os.error} if the file does
not exist or is inaccessible.
\versionadded{1.5.2}
\versionchanged[If \function{os.stat_float_times()} returns True, the result is a floating point number]{2.3}
\end{funcdesc}

\begin{funcdesc}{getmtime}{path}
Return the time of last modification of \var{path}.  The return
value is a number giving the number of seconds since the epoch (see the 
\refmodule{time} module).  Raise \exception{os.error} if the file does
not exist or is inaccessible.
\versionadded{1.5.2}
\versionchanged[If \function{os.stat_float_times()} returns True, the result is a floating point number]{2.3}
\end{funcdesc}

\begin{funcdesc}{getctime}{path}
Return the system's ctime which, on some systems (like \UNIX) is the
time of the last change, and, on others (like Windows), is the
creation time for \var{path}.  The return
value is a number giving the number of seconds since the epoch (see the 
\refmodule{time} module).  Raise \exception{os.error} if the file does
not exist or is inaccessible.
\versionadded{2.3}
\end{funcdesc}

\begin{funcdesc}{getsize}{path}
Return the size, in bytes, of \var{path}.  Raise
\exception{os.error} if the file does not exist or is inaccessible.
\versionadded{1.5.2}
\end{funcdesc}

\begin{funcdesc}{isabs}{path}
Return \code{True} if \var{path} is an absolute pathname (begins with a
slash).
\end{funcdesc}

\begin{funcdesc}{isfile}{path}
Return \code{True} if \var{path} is an existing regular file.  This follows
symbolic links, so both \function{islink()} and \function{isfile()}
can be true for the same path.
\end{funcdesc}

\begin{funcdesc}{isdir}{path}
Return \code{True} if \var{path} is an existing directory.  This follows
symbolic links, so both \function{islink()} and \function{isdir()} can
be true for the same path.
\end{funcdesc}

\begin{funcdesc}{islink}{path}
Return \code{True} if \var{path} refers to a directory entry that is a
symbolic link.  Always \code{False} if symbolic links are not supported.
\end{funcdesc}

\begin{funcdesc}{ismount}{path}
Return \code{True} if pathname \var{path} is a \dfn{mount point}: a point in
a file system where a different file system has been mounted.  The
function checks whether \var{path}'s parent, \file{\var{path}/..}, is
on a different device than \var{path}, or whether \file{\var{path}/..}
and \var{path} point to the same i-node on the same device --- this
should detect mount points for all \UNIX{} and \POSIX{} variants.
\end{funcdesc}

\begin{funcdesc}{join}{path1\optional{, path2\optional{, ...}}}
Joins one or more path components intelligently.  If any component is
an absolute path, all previous components are thrown away, and joining
continues.  The return value is the concatenation of \var{path1}, and
optionally \var{path2}, etc., with exactly one directory separator
(\code{os.sep}) inserted between components, unless \var{path2} is
empty.  Note that on Windows, since there is a current directory for
each drive, \function{os.path.join("c:", "foo")} represents a path
relative to the current directory on drive \file{C:} (\file{c:foo}), not
\file{c:\textbackslash\textbackslash foo}.
\end{funcdesc}

\begin{funcdesc}{normcase}{path}
Normalize the case of a pathname.  On \UNIX, this returns the path
unchanged; on case-insensitive filesystems, it converts the path to
lowercase.  On Windows, it also converts forward slashes to backward
slashes.
\end{funcdesc}

\begin{funcdesc}{normpath}{path}
Normalize a pathname.  This collapses redundant separators and
up-level references so that \code{A//B}, \code{A/./B} and
\code{A/foo/../B} all become \code{A/B}.  It does not normalize the
case (use \function{normcase()} for that).  On Windows, it converts
forward slashes to backward slashes. It should be understood that this may
change the meaning of the path if it contains symbolic links! 
\end{funcdesc}

\begin{funcdesc}{realpath}{path}
Return the canonical path of the specified filename, eliminating any
symbolic links encountered in the path.
Availability:  \UNIX.
\versionadded{2.2}
\end{funcdesc}

\begin{funcdesc}{samefile}{path1, path2}
Return \code{True} if both pathname arguments refer to the same file or
directory (as indicated by device number and i-node number).
Raise an exception if a \function{os.stat()} call on either pathname
fails.
Availability:  Macintosh, \UNIX.
\end{funcdesc}

\begin{funcdesc}{sameopenfile}{fp1, fp2}
Return \code{True} if the file objects \var{fp1} and \var{fp2} refer to the
same file.  The two file objects may represent different file
descriptors.
Availability:  Macintosh, \UNIX.
\end{funcdesc}

\begin{funcdesc}{samestat}{stat1, stat2}
Return \code{True} if the stat tuples \var{stat1} and \var{stat2} refer to
the same file.  These structures may have been returned by
\function{fstat()}, \function{lstat()}, or \function{stat()}.  This
function implements the underlying comparison used by
\function{samefile()} and \function{sameopenfile()}.
Availability:  Macintosh, \UNIX.
\end{funcdesc}

\begin{funcdesc}{split}{path}
Split the pathname \var{path} into a pair, \code{(\var{head},
\var{tail})} where \var{tail} is the last pathname component and
\var{head} is everything leading up to that.  The \var{tail} part will
never contain a slash; if \var{path} ends in a slash, \var{tail} will
be empty.  If there is no slash in \var{path}, \var{head} will be
empty.  If \var{path} is empty, both \var{head} and \var{tail} are
empty.  Trailing slashes are stripped from \var{head} unless it is the
root (one or more slashes only).  In nearly all cases,
\code{join(\var{head}, \var{tail})} equals \var{path} (the only
exception being when there were multiple slashes separating \var{head}
from \var{tail}).
\end{funcdesc}

\begin{funcdesc}{splitdrive}{path}
Split the pathname \var{path} into a pair \code{(\var{drive},
\var{tail})} where \var{drive} is either a drive specification or the
empty string.  On systems which do not use drive specifications,
\var{drive} will always be the empty string.  In all cases,
\code{\var{drive} + \var{tail}} will be the same as \var{path}.
\versionadded{1.3}
\end{funcdesc}

\begin{funcdesc}{splitext}{path}
Split the pathname \var{path} into a pair \code{(\var{root}, \var{ext})} 
such that \code{\var{root} + \var{ext} == \var{path}},
and \var{ext} is empty or begins with a period and contains
at most one period.
\end{funcdesc}

\begin{funcdesc}{walk}{path, visit, arg}
Calls the function \var{visit} with arguments
\code{(\var{arg}, \var{dirname}, \var{names})} for each directory in the
directory tree rooted at \var{path} (including \var{path} itself, if it
is a directory).  The argument \var{dirname} specifies the visited
directory, the argument \var{names} lists the files in the directory
(gotten from \code{os.listdir(\var{dirname})}).
The \var{visit} function may modify \var{names} to
influence the set of directories visited below \var{dirname}, e.g. to
avoid visiting certain parts of the tree.  (The object referred to by
\var{names} must be modified in place, using \keyword{del} or slice
assignment.)

\begin{notice}
Symbolic links to directories are not treated as subdirectories, and
that \function{walk()} therefore will not visit them. To visit linked
directories you must identify them with
\code{os.path.islink(\var{file})} and
\code{os.path.isdir(\var{file})}, and invoke \function{walk()} as
necessary.
\end{notice}

\note{The newer \function{\refmodule{os}.walk()} generator supplies
      similar functionality and can be easier to use.}
\end{funcdesc}

\begin{datadesc}{supports_unicode_filenames}
True if arbitrary Unicode strings can be used as file names (within
limitations imposed by the file system), and if
\function{os.listdir()} returns Unicode strings for a Unicode
argument.
\versionadded{2.3}
\end{datadesc}
		% os.path
\section{\module{dircache} ---
         Cached directory listings}

\declaremodule{standard}{dircache}
\sectionauthor{Moshe Zadka}{moshez@zadka.site.co.il}
\modulesynopsis{Return directory listing, with cache mechanism.}

The \module{dircache} module defines a function for reading directory listing
using a cache, and cache invalidation using the \var{mtime} of the directory.
Additionally, it defines a function to annotate directories by appending
a slash.

The \module{dircache} module defines the following functions:

\begin{funcdesc}{reset}{}
Resets the directory cache.
\end{funcdesc}

\begin{funcdesc}{listdir}{path}
Return a directory listing of \var{path}, as gotten from
\function{os.listdir()}. Note that unless \var{path} changes, further call
to \function{listdir()} will not re-read the directory structure.

Note that the list returned should be regarded as read-only. (Perhaps
a future version should change it to return a tuple?)
\end{funcdesc}

\begin{funcdesc}{opendir}{path}
Same as \function{listdir()}. Defined for backwards compatibility.
\end{funcdesc}

\begin{funcdesc}{annotate}{head, list}
Assume \var{list} is a list of paths relative to \var{head}, and append,
in place, a \character{/} to each path which points to a directory.
\end{funcdesc}

\begin{verbatim}
>>> import dircache
>>> a = dircache.listdir('/')
>>> a = a[:] # Copy the return value so we can change 'a'
>>> a
['bin', 'boot', 'cdrom', 'dev', 'etc', 'floppy', 'home', 'initrd', 'lib', 'lost+
found', 'mnt', 'proc', 'root', 'sbin', 'tmp', 'usr', 'var', 'vmlinuz']
>>> dircache.annotate('/', a)
>>> a
['bin/', 'boot/', 'cdrom/', 'dev/', 'etc/', 'floppy/', 'home/', 'initrd/', 'lib/
', 'lost+found/', 'mnt/', 'proc/', 'root/', 'sbin/', 'tmp/', 'usr/', 'var/', 'vm
linuz']
\end{verbatim}

\section{\module{stat} ---
         Interpreting \function{stat()} results}

\declaremodule{standard}{stat}
\modulesynopsis{Utilities for interpreting the results of
  \function{os.stat()}, \function{os.lstat()} and \function{os.fstat()}.}
\sectionauthor{Skip Montanaro}{skip@automatrix.com}


The \module{stat} module defines constants and functions for
interpreting the results of \function{os.stat()},
\function{os.fstat()} and \function{os.lstat()} (if they exist).  For
complete details about the \cfunction{stat()}, \cfunction{fstat()} and
\cfunction{lstat()} calls, consult the documentation for your system.

The \module{stat} module defines the following functions to test for
specific file types:


\begin{funcdesc}{S_ISDIR}{mode}
Return non-zero if the mode is from a directory.
\end{funcdesc}

\begin{funcdesc}{S_ISCHR}{mode}
Return non-zero if the mode is from a character special device file.
\end{funcdesc}

\begin{funcdesc}{S_ISBLK}{mode}
Return non-zero if the mode is from a block special device file.
\end{funcdesc}

\begin{funcdesc}{S_ISREG}{mode}
Return non-zero if the mode is from a regular file.
\end{funcdesc}

\begin{funcdesc}{S_ISFIFO}{mode}
Return non-zero if the mode is from a FIFO (named pipe).
\end{funcdesc}

\begin{funcdesc}{S_ISLNK}{mode}
Return non-zero if the mode is from a symbolic link.
\end{funcdesc}

\begin{funcdesc}{S_ISSOCK}{mode}
Return non-zero if the mode is from a socket.
\end{funcdesc}

Two additional functions are defined for more general manipulation of
the file's mode:

\begin{funcdesc}{S_IMODE}{mode}
Return the portion of the file's mode that can be set by
\function{os.chmod()}---that is, the file's permission bits, plus the
sticky bit, set-group-id, and set-user-id bits (on systems that support
them).
\end{funcdesc}

\begin{funcdesc}{S_IFMT}{mode}
Return the portion of the file's mode that describes the file type (used
by the \function{S_IS*()} functions above).
\end{funcdesc}

Normally, you would use the \function{os.path.is*()} functions for
testing the type of a file; the functions here are useful when you are
doing multiple tests of the same file and wish to avoid the overhead of
the \cfunction{stat()} system call for each test.  These are also
useful when checking for information about a file that isn't handled
by \refmodule{os.path}, like the tests for block and character
devices.

All the variables below are simply symbolic indexes into the 10-tuple
returned by \function{os.stat()}, \function{os.fstat()} or
\function{os.lstat()}.

\begin{datadesc}{ST_MODE}
Inode protection mode.
\end{datadesc}

\begin{datadesc}{ST_INO}
Inode number.
\end{datadesc}

\begin{datadesc}{ST_DEV}
Device inode resides on.
\end{datadesc}

\begin{datadesc}{ST_NLINK}
Number of links to the inode.
\end{datadesc}

\begin{datadesc}{ST_UID}
User id of the owner.
\end{datadesc}

\begin{datadesc}{ST_GID}
Group id of the owner.
\end{datadesc}

\begin{datadesc}{ST_SIZE}
Size in bytes of a plain file; amount of data waiting on some special
files.
\end{datadesc}

\begin{datadesc}{ST_ATIME}
Time of last access.
\end{datadesc}

\begin{datadesc}{ST_MTIME}
Time of last modification.
\end{datadesc}

\begin{datadesc}{ST_CTIME}
Time of last status change (see manual pages for details).
\end{datadesc}

The interpretation of ``file size'' changes according to the file
type.  For plain files this is the size of the file in bytes.  For
FIFOs and sockets under most Unixes (including Linux in particular),
the ``size'' is the number of bytes waiting to be read at the time of
the call to \function{os.stat()}, \function{os.fstat()}, or
\function{os.lstat()}; this can sometimes be useful, especially for
polling one of these special files after a non-blocking open.  The
meaning of the size field for other character and block devices varies
more, depending on the implementation of the underlying system call.

Example:

\begin{verbatim}
import os, sys
from stat import *

def walktree(dir, callback):
    '''recursively descend the directory rooted at dir,
       calling the callback function for each regular file'''

    for f in os.listdir(dir):
        pathname = '%s/%s' % (dir, f)
        mode = os.stat(pathname)[ST_MODE]
        if S_ISDIR(mode):
            # It's a directory, recurse into it
            walktree(pathname, callback)
        elif S_ISREG(mode):
            # It's a file, call the callback function
            callback(pathname)
        else:
            # Unknown file type, print a message
            print 'Skipping %s' % pathname

def visitfile(file):
    print 'visiting', file

if __name__ == '__main__':
    walktree(sys.argv[1], visitfile)
\end{verbatim}

\section{\module{statcache} ---
         An optimization of \function{os.stat()}}

\declaremodule{standard}{statcache}
\sectionauthor{Moshe Zadka}{mzadka@geocities.com}
\modulesynopsis{Stat files, and remember results.}

The \module{statcache} module provides a simple optimization to
\function{os.stat()}: remembering the values of previous invocations.

The \module{statcache} module defines the following functions:

\begin{funcdesc}{stat}{path}
This is the main module entry-point.
Identical for \function{os.stat()}, except for remembering the result
for future invocations of the function.
\end{funcdesc}

The rest of the functions are used to clear the cache, or parts of
it.

\begin{funcdesc}{reset}{}
Clear the cache: forget all results of previous \code{stat}s.
\end{funcdesc}

\begin{funcdesc}{forget}{path}
Forget the result of \code{stat(\var{path})}, if any.
\end{funcdesc}

\begin{funcdesc}{forget_prefix}{prefix}
Forget all results of \code{stat(\var{path})} for \var{path} starting
with \var{prefix}.
\end{funcdesc}

\begin{funcdesc}{forget_dir}{prefix}
Forget all results of \code{stat(\var{path})} for \var{path} a file in
the directory \var{prefix}, including \code{stat(\var{prefix})}.
\end{funcdesc}

\begin{funcdesc}{forget_except_prefix}{prefix}
Similar to \function{forget_prefix()}, but for all \var{path}
\emph{not} starting with \var{prefix}.
\end{funcdesc}

Example:

\begin{verbatim}
>>> import os, statcache
>>> statcache.stat('.')
(16893, 2049, 772, 18, 1000, 1000, 2048, 929609777, 929609777, 929609777)
>>> os.stat('.')
(16893, 2049, 772, 18, 1000, 1000, 2048, 929609777, 929609777, 929609777)
\end{verbatim}

\section{\module{statvfs} ---
         Constants used with \function{os.statvfs()}}

\declaremodule{standard}{statvfs}
% LaTeX'ed from comments in module
\sectionauthor{Moshe Zadka}{mzadka@geocities.com}
\modulesynopsis{Constants for interpreting the result of
                \function{os.statvfs()}.}

The \module{statvfs} module defines constants so interpreting the result
if \function{os.statvfs()}, which returns a tuple, can be made without
remembering ``magic numbers.''  Each of the constants defined in this
module is the \emph{index} of the entry in the tuple returned by
\function{os.statvfs()} that contains the specified information.


\begin{datadesc}{F_BSIZE}
Preferred file system block size.
\end{datadesc}

\begin{datadesc}{F_FRSIZE}
Fundamental file system block size.
\end{datadesc}

\begin{datadesc}{F_BFREE}
Total number of free blocks.
\end{datadesc}

\begin{datadesc}{F_BAVAIL}
Free blocks available to non-super user.
\end{datadesc}

\begin{datadesc}{F_FILES}
Total number of file nodes.
\end{datadesc}

\begin{datadesc}{F_FFREE}
Total number of free file nodes.
\end{datadesc}

\begin{datadesc}{F_FAVAIL}
Free nodes available to non-superuser.
\end{datadesc}

\begin{datadesc}{F_FLAG}
Flags. System dependant: see \cfunction{statvfs()} man page.
\end{datadesc}

\begin{datadesc}{F_NAMEMAX}
Maximum file name length.
\end{datadesc}

\section{\module{filecmp} ---
         File and Directory Comparisons}

\declaremodule{standard}{filecmp}
\sectionauthor{Moshe Zadka}{moshez@zadka.site.co.il}
\modulesynopsis{Compare files efficiently.}


The \module{filecmp} module defines functions to compare files and directories,
with various optional time/correctness trade-offs.

The \module{filecmp} module defines the following function:

\begin{funcdesc}{cmp}{f1, f2\optional{, shallow\optional{, use_statcache}}}
Compare the files named \var{f1} and \var{f2}, returning \code{1} if
they seem equal, \code{0} otherwise.

Unless \var{shallow} is given and is false, files with identical
\function{os.stat()} signatures are taken to be equal.  If
\var{use_statcache} is given and is true,
\function{statcache.stat()} will be called rather then
\function{os.stat()}; the default is to use \function{os.stat()}.

Files that were compared using this function will not be compared again
unless their \function{os.stat()} signature changes. Note that using
\var{use_statcache} true will cause the cache invalidation mechanism to 
fail --- the stale stat value will be used from \refmodule{statcache}'s 
cache.

Note that no external programs are called from this function, giving it
portability and efficiency.
\end{funcdesc}

\begin{funcdesc}{cmpfiles}{dir1, dir2, common\optional{,
                           shallow\optional{, use_statcache}}}
Returns three lists of file names: \var{match}, \var{mismatch},
\var{errors}.  \var{match} contains the list of files match in both
directories, \var{mismatch} includes the names of those that don't,
and \var{errros} lists the names of files which could not be
compared.  Files may be listed in \var{errors} because the user may
lack permission to read them or many other reasons, but always that
the comparison could not be done for some reason.

The \var{shallow} and \var{use_statcache} parameters have the same
meanings and default values as for \function{filecmp.cmp()}.
\end{funcdesc}

Example:

\begin{verbatim}
>>> import filecmp
>>> filecmp.cmp('libundoc.tex', 'libundoc.tex')
1
>>> filecmp.cmp('libundoc.tex', 'lib.tex')
0
\end{verbatim}


\subsection{The \protect\class{dircmp} class \label{dircmp-objects}}

\begin{classdesc}{dircmp}{a, b\optional{, ignore\optional{, hide}}}
Construct a new directory comparison object, to compare the
directories \var{a} and \var{b}. \var{ignore} is a list of names to
ignore, and defaults to \code{['RCS', 'CVS', 'tags']}. \var{hide} is a
list of names to hid, and defaults to \code{[os.curdir, os.pardir]}.
\end{classdesc}

\begin{methoddesc}[dircmp]{report}{}
Print (to \code{sys.stdout}) a comparison between \var{a} and \var{b}.
\end{methoddesc}

\begin{methoddesc}[dircmp]{report_partial_closure}{}
Print a comparison between \var{a} and \var{b} and common immediate
subdirctories.
\end{methoddesc}

\begin{methoddesc}[dircmp]{report_full_closure}{}
Print a comparison between \var{a} and \var{b} and common 
subdirctories (recursively).
\end{methoddesc}

\begin{memberdesc}[dircmp]{left_list}
Files and subdirectories in \var{a}, filtered by \var{hide} and
\var{ignore}.
\end{memberdesc}

\begin{memberdesc}[dircmp]{right_list}
Files and subdirectories in \var{b}, filtered by \var{hide} and
\var{ignore}.
\end{memberdesc}

\begin{memberdesc}[dircmp]{common}
Files and subdirectories in both \var{a} and \var{b}.
\end{memberdesc}

\begin{memberdesc}[dircmp]{left_only}
Files and subdirectories only in \var{a}.
\end{memberdesc}

\begin{memberdesc}[dircmp]{right_only}
Files and subdirectories only in \var{b}.
\end{memberdesc}

\begin{memberdesc}[dircmp]{common_dirs}
Subdirectories in both \var{a} and \var{b}.
\end{memberdesc}

\begin{memberdesc}[dircmp]{common_files}
Files in both \var{a} and \var{b}
\end{memberdesc}

\begin{memberdesc}[dircmp]{common_funny}
Names in both \var{a} and \var{b}, such that the type differs between
the directories, or names for which \function{os.stat()} reports an
error.
\end{memberdesc}

\begin{memberdesc}[dircmp]{same_files}
Files which are identical in both \var{a} and \var{b}.
\end{memberdesc}

\begin{memberdesc}[dircmp]{diff_files}
Files which are in both \var{a} and \var{b}, whose contents differ.
\end{memberdesc}

\begin{memberdesc}[dircmp]{funny_files}
Files which are in both \var{a} and \var{b}, but could not be
compared.
\end{memberdesc}

\begin{memberdesc}[dircmp]{subdirs}
A dictionary mapping names in \member{common_dirs} to
\class{dircmp} objects.
\end{memberdesc}

Note that via \method{__getattr__()} hooks, all attributes are
computed lazilly, so there is no speed penalty if only those
attributes which are lightweight to compute are used.

%\section{\module{cmp} ---
         File comparisons}

\declaremodule{standard}{cmp}
\sectionauthor{Moshe Zadka}{mzadka@geocities.com}
\modulesynopsis{Compare files very efficiently.}

The \module{cmp} module defines a function to compare files, taking all
sort of short-cuts to make it a highly efficient operation.

The \module{cmp} module defines the following function:

\begin{funcdesc}{cmp}{f1, f2}
Compare two files given as names. The following tricks are used to
optimize the comparisons:

\begin{itemize}
        \item Files with identical type, size and mtime are assumed equal.
        \item Files with different type or size are never equal.
        \item The module only compares files it already compared if their
        signature (type, size and mtime) changed.
        \item No external programs are called.
\end{itemize}
\end{funcdesc}

Example:

\begin{verbatim}
>>> import cmp
>>> cmp.cmp('libundoc.tex', 'libundoc.tex')
1
>>> cmp.cmp('libundoc.tex', 'lib.tex')
0
\end{verbatim}

%\section{\module{cmpcache} ---
         Efficient file comparisons}

\declaremodule{standard}{cmpcache}
\sectionauthor{Moshe Zadka}{mzadka@geocities.com}
\modulesynopsis{Compare files very efficiently.}

% XXX check version number before release!
\deprecated{1.5.3}{Use the \module{filecmp} module instead.}

The \module{cmpcache} module provides an identical interface and similar
functionality as the \refmodule{cmp} module, but can be a bit more efficient
as it uses \function{statcache.stat()} instead of \function{os.stat()}
(see the \refmodule{statcache} module for information on the
difference).

\strong{Note:}  Using the \refmodule{statcache} module to provide
\function{stat()} information results in trashing the cache
invalidation mechanism: results are not as reliable.  To ensure
``current'' results, use \function{cmp.cmp()} instead of the version
defined in this module, or use \function{statcache.forget()} to
invalidate the appropriate entries.

\section{\module{time} ---
         Time access and conversions}

\declaremodule{builtin}{time}
\modulesynopsis{Time access and conversions.}


This module provides various time-related functions.
It is always available, but not all functions are available
on all platforms.

An explanation of some terminology and conventions is in order.

\begin{itemize}

\item
The \dfn{epoch}\index{epoch} is the point where the time starts.  On
January 1st of that year, at 0 hours, the ``time since the epoch'' is
zero.  For \UNIX{}, the epoch is 1970.  To find out what the epoch is,
look at \code{gmtime(0)}.

\item
The functions in this module do not handle dates and times before the
epoch or far in the future.  The cut-off point in the future is
determined by the C library; for \UNIX{}, it is typically in
2038\index{Year 2038}.

\item
\strong{Year 2000 (Y2K) issues}:\index{Year 2000}\index{Y2K}  Python
depends on the platform's C library, which generally doesn't have year
2000 issues, since all dates and times are represented internally as
seconds since the epoch.  Functions accepting a time tuple (see below)
generally require a 4-digit year.  For backward compatibility, 2-digit
years are supported if the module variable \code{accept2dyear} is a
non-zero integer; this variable is initialized to \code{1} unless the
environment variable \envvar{PYTHONY2K} is set to a non-empty string,
in which case it is initialized to \code{0}.  Thus, you can set
\envvar{PYTHONY2K} to a non-empty string in the environment to require 4-digit
years for all year input.  When 2-digit years are accepted, they are
converted according to the \POSIX{} or X/Open standard: values 69-99
are mapped to 1969-1999, and values 0--68 are mapped to 2000--2068.
Values 100--1899 are always illegal.  Note that this is new as of
Python 1.5.2(a2); earlier versions, up to Python 1.5.1 and 1.5.2a1,
would add 1900 to year values below 1900.

\item
UTC\index{UTC} is Coordinated Universal Time\index{Coordinated
Universal Time} (formerly known as Greenwich Mean
Time,\index{Greenwich Mean Time} or GMT).  The acronym UTC is not a
mistake but a compromise between English and French.

\item
DST is Daylight Saving Time,\index{Daylight Saving Time} an adjustment
of the timezone by (usually) one hour during part of the year.  DST
rules are magic (determined by local law) and can change from year to
year.  The C library has a table containing the local rules (often it
is read from a system file for flexibility) and is the only source of
True Wisdom in this respect.

\item
The precision of the various real-time functions may be less than
suggested by the units in which their value or argument is expressed.
E.g.\ on most \UNIX{} systems, the clock ``ticks'' only 50 or 100 times a
second, and on the Mac, times are only accurate to whole seconds.

\item
On the other hand, the precision of \function{time()} and
\function{sleep()} is better than their \UNIX{} equivalents: times are
expressed as floating point numbers, \function{time()} returns the
most accurate time available (using \UNIX{} \cfunction{gettimeofday()}
where available), and \function{sleep()} will accept a time with a
nonzero fraction (\UNIX{} \cfunction{select()} is used to implement
this, where available).

\item

The time tuple as returned by \function{gmtime()},
\function{localtime()}, and \function{strptime()}, and accepted by
\function{asctime()}, \function{mktime()} and \function{strftime()},
is a tuple of 9 integers:

\begin{tableiii}{r|l|l}{textrm}{Index}{Field}{Values}
  \lineiii{0}{year}{(e.g.\ 1993)}
  \lineiii{1}{month}{range [1,12]}
  \lineiii{2}{day}{range [1,31]}
  \lineiii{3}{hour}{range [0,23]}
  \lineiii{4}{minute}{range [0,59]}
  \lineiii{5}{second}{range [0,61]; see \strong{(1)} in \function{strftime()} description}
  \lineiii{6}{weekday}{range [0,6], monday is 0}
  \lineiii{7}{Julian day}{range [1,366]}
  \lineiii{8}{daylight savings flag}{0, 1 or -1; see below}
\end{tableiii}

Note that unlike the C structure, the month value is a
range of 1-12, not 0-11.  A year value will be handled as described
under ``Year 2000 (Y2K) issues'' above.  A \code{-1} argument as
daylight savings flag, passed to \function{mktime()} will usually
result in the correct daylight savings state to be filled in.

\end{itemize}

The module defines the following functions and data items:


\begin{datadesc}{accept2dyear}
Boolean value indicating whether two-digit year values will be
accepted.  This is true by default, but will be set to false if the
environment variable \envvar{PYTHONY2K} has been set to a non-empty
string.  It may also be modified at run time.
\end{datadesc}

\begin{datadesc}{altzone}
The offset of the local DST timezone, in seconds west of UTC, if one
is defined.  This is negative if the local DST timezone is east of UTC
(as in Western Europe, including the UK).  Only use this if
\code{daylight} is nonzero.
\end{datadesc}

\begin{funcdesc}{asctime}{tuple}
Convert a tuple representing a time as returned by \function{gmtime()}
or \function{localtime()} to a 24-character string of the following form:
\code{'Sun Jun 20 23:21:05 1993'}.  Note: unlike the C function of
the same name, there is no trailing newline.
\end{funcdesc}

\begin{funcdesc}{clock}{}
Return the current CPU time as a floating point number expressed in
seconds.  The precision, and in fact the very definiton of the meaning
of ``CPU time''\index{CPU time}, depends on that of the C function
of the same name, but in any case, this is the function to use for
benchmarking\index{benchmarking} Python or timing algorithms.
\end{funcdesc}

\begin{funcdesc}{ctime}{secs}
Convert a time expressed in seconds since the epoch to a string
representing local time.  \code{ctime(\var{secs})} is equivalent to
\code{asctime(localtime(\var{secs}))}.
\end{funcdesc}

\begin{datadesc}{daylight}
Nonzero if a DST timezone is defined.
\end{datadesc}

\begin{funcdesc}{gmtime}{secs}
Convert a time expressed in seconds since the epoch to a time tuple
in UTC in which the dst flag is always zero.  Fractions of a second are
ignored.  See above for a description of the tuple lay-out.
\end{funcdesc}

\begin{funcdesc}{localtime}{secs}
Like \function{gmtime()} but converts to local time.  The dst flag is
set to \code{1} when DST applies to the given time.
\end{funcdesc}

\begin{funcdesc}{mktime}{tuple}
This is the inverse function of \function{localtime()}.  Its argument
is the full 9-tuple (since the dst flag is needed --- pass \code{-1}
as the dst flag if it is unknown) which expresses the time in
\emph{local} time, not UTC.  It returns a floating point number, for
compatibility with \function{time()}.  If the input value cannot be
represented as a valid time, \exception{OverflowError} is raised.
\end{funcdesc}

\begin{funcdesc}{sleep}{secs}
Suspend execution for the given number of seconds.  The argument may
be a floating point number to indicate a more precise sleep time.
The actual suspension time may be less than that requested because any
caught signal will terminate the \function{sleep()} following
execution of that signal's catching routine.  Also, the suspension
time may be longer than requested by an arbitrary amount because of
the scheduling of other activity in the system.
\end{funcdesc}

\begin{funcdesc}{strftime}{format, tuple}
Convert a tuple representing a time as returned by \function{gmtime()}
or \function{localtime()} to a string as specified by the \var{format}
argument.  \var{format} must be a string.

The following directives can be embedded in the \var{format} string.
They are shown without the optional field width and precision
specification, and are replaced by the indicated characters in the
\function{strftime()} result:

\begin{tableiii}{c|p{24em}|c}{code}{Directive}{Meaning}{Notes}
  \lineiii{\%a}{Locale's abbreviated weekday name.}{}
  \lineiii{\%A}{Locale's full weekday name.}{}
  \lineiii{\%b}{Locale's abbreviated month name.}{}
  \lineiii{\%B}{Locale's full month name.}{}
  \lineiii{\%c}{Locale's appropriate date and time representation.}{}
  \lineiii{\%d}{Day of the month as a decimal number [01,31].}{}
  \lineiii{\%H}{Hour (24-hour clock) as a decimal number [00,23].}{}
  \lineiii{\%I}{Hour (12-hour clock) as a decimal number [01,12].}{}
  \lineiii{\%j}{Day of the year as a decimal number [001,366].}{}
  \lineiii{\%m}{Month as a decimal number [01,12].}{}
  \lineiii{\%M}{Minute as a decimal number [00,59].}{}
  \lineiii{\%p}{Locale's equivalent of either AM or PM.}{}
  \lineiii{\%S}{Second as a decimal number [00,61].}{(1)}
  \lineiii{\%U}{Week number of the year (Sunday as the first day of the
                week) as a decimal number [00,53].  All days in a new year
                preceding the first Sunday are considered to be in week 0.}{}
  \lineiii{\%w}{Weekday as a decimal number [0(Sunday),6].}{}
  \lineiii{\%W}{Week number of the year (Monday as the first day of the
                week) as a decimal number [00,53].  All days in a new year
                preceding the first Sunday are considered to be in week 0.}{}
  \lineiii{\%x}{Locale's appropriate date representation.}{}
  \lineiii{\%X}{Locale's appropriate time representation.}{}
  \lineiii{\%y}{Year without century as a decimal number [00,99].}{}
  \lineiii{\%Y}{Year with century as a decimal number.}{}
  \lineiii{\%Z}{Time zone name (or by no characters if no time zone exists).}{}
  \lineiii{\%\%}{A literal \character{\%} character.}{}
\end{tableiii}

\noindent
Notes:

\begin{description}
  \item[(1)]
    The range really is \code{0} to \code{61}; this accounts for leap
    seconds and the (very rare) double leap seconds.
\end{description}

Additional directives may be supported on certain platforms, but
only the ones listed here have a meaning standardized by ANSI C.

On some platforms, an optional field width and precision
specification can immediately follow the initial \character{\%} of a
directive in the following order; this is also not portable.
The field width is normally 2 except for \code{\%j} where it is 3.
\end{funcdesc}

\begin{funcdesc}{strptime}{string\optional{, format}}
Parse a string representing a time according to a format.  The return 
value is a tuple as returned by \function{gmtime()} or
\function{localtime()}.  The \var{format} parameter uses the same
directives as those used by \function{strftime()}; it defaults to
\code{"\%a \%b \%d \%H:\%M:\%S \%Y"} which matches the formatting
returned by \function{ctime()}.  The same platform caveats apply; see
the local \UNIX{} documentation for restrictions or additional
supported directives.  If \var{string} cannot be parsed according to
\var{format}, \exception{ValueError} is raised.

Availability: Most modern \UNIX{} systems.
\end{funcdesc}

\begin{funcdesc}{time}{}
Return the time as a floating point number expressed in seconds since
the epoch, in UTC.  Note that even though the time is always returned
as a floating point number, not all systems provide time with a better
precision than 1 second.
\end{funcdesc}

\begin{datadesc}{timezone}
The offset of the local (non-DST) timezone, in seconds west of UTC
(i.e. negative in most of Western Europe, positive in the US, zero in
the UK).
\end{datadesc}

\begin{datadesc}{tzname}
A tuple of two strings: the first is the name of the local non-DST
timezone, the second is the name of the local DST timezone.  If no DST
timezone is defined, the second string should not be used.
\end{datadesc}


\begin{seealso}
  \seemodule{locale}{Internationalization services.  The locale
                     settings can affect the return values for some of 
                     the functions in the \module{time} module.}
\end{seealso}

\section{\module{sched} ---
         Event scheduler}

% LaTeXed and enhanced from comments in file

\declaremodule{standard}{sched}
\sectionauthor{Moshe Zadka}{moshez@zadka.site.co.il}
\modulesynopsis{General purpose event scheduler.}

The \module{sched} module defines a class which implements a general
purpose event scheduler:\index{event scheduling}

\begin{classdesc}{scheduler}{timefunc, delayfunc}
The \class{scheduler} class defines a generic interface to scheduling
events. It needs two functions to actually deal with the ``outside world''
--- \var{timefunc} should be callable without arguments, and return 
a number (the ``time'', in any units whatsoever).  The \var{delayfunc}
function should be callable with one argument, compatible with the output
of \var{timefunc}, and should delay that many time units.
\var{delayfunc} will also be called with the argument \code{0} after
each event is run to allow other threads an opportunity to run in
multi-threaded applications.
\end{classdesc}

Example:

\begin{verbatim}
>>> import sched, time
>>> s=sched.scheduler(time.time, time.sleep)
>>> def print_time(): print "From print_time", time.time()
...
>>> def print_some_times():
...     print time.time()
...     s.enter(5, 1, print_time, ())
...     s.enter(10, 1, print_time, ())
...     s.run()
...     print time.time()
...
>>> print_some_times()
930343690.257
From print_time 930343695.274
From print_time 930343700.273
930343700.276
\end{verbatim}


\subsection{Scheduler Objects \label{scheduler-objects}}

\class{scheduler} instances have the following methods:

\begin{methoddesc}{enterabs}{time, priority, action, argument}
Schedule a new event. The \var{time} argument should be a numeric type
compatible with the return value of the \var{timefunc} function passed 
to the constructor. Events scheduled for
the same \var{time} will be executed in the order of their
\var{priority}.

Executing the event means executing
\code{\var{action}(*\var{argument})}.  \var{argument} must be a
sequence holding the parameters for \var{action}.

Return value is an event which may be used for later cancellation of
the event (see \method{cancel()}).
\end{methoddesc}

\begin{methoddesc}{enter}{delay, priority, action, argument}
Schedule an event for \var{delay} more time units. Other then the
relative time, the other arguments, the effect and the return value
are the same as those for \method{enterabs()}.
\end{methoddesc}

\begin{methoddesc}{cancel}{event}
Remove the event from the queue. If \var{event} is not an event
currently in the queue, this method will raise a
\exception{RuntimeError}.
\end{methoddesc}

\begin{methoddesc}{empty}{}
Return true if the event queue is empty.
\end{methoddesc}

\begin{methoddesc}{run}{}
Run all scheduled events. This function will wait 
(using the \function{delayfunc} function passed to the constructor)
for the next event, then execute it and so on until there are no more
scheduled events.

Either \var{action} or \var{delayfunc} can raise an exception.  In
either case, the scheduler will maintain a consistent state and
propagate the exception.  If an exception is raised by \var{action},
the event will not be attempted in future calls to \method{run()}.

If a sequence of events takes longer to run than the time available
before the next event, the scheduler will simply fall behind.  No
events will be dropped; the calling code is responsible for canceling 
events which are no longer pertinent.
\end{methoddesc}

\section{\module{getpass}
         --- Portable password input}

\declaremodule{standard}{getpass}
\modulesynopsis{Portable reading of passwords and retrieval of the userid.}
\moduleauthor{Piers Lauder}{piers@cs.su.oz.au}
% Windows (& Mac?) support by Guido van Rossum.
\sectionauthor{Fred L. Drake, Jr.}{fdrake@acm.org}


The \module{getpass} module provides two functions:


\begin{funcdesc}{getpass}{\optional{prompt}}
  Prompt the user for a password without echoing.  The user is
  prompted using the string \var{prompt}, which defaults to
  \code{'Password: '}.
  Availability: Macintosh, \UNIX, Windows.
\end{funcdesc}


\begin{funcdesc}{getuser}{}
  Return the ``login name'' of the user.
  Availability: \UNIX, Windows.

  This function checks the environment variables \envvar{LOGNAME},
  \envvar{USER}, \envvar{LNAME} and \envvar{USERNAME}, in order, and
  returns the value of the first one which is set to a non-empty
  string.  If none are set, the login name from the password database
  is returned on systems which support the \refmodule{pwd} module,
  otherwise, an exception is raised.
\end{funcdesc}

\section{\module{curses} ---
         Terminal handling for character-cell displays}

\declaremodule{standard}{curses}
\sectionauthor{Moshe Zadka}{moshez@zadka.site.co.il}
\sectionauthor{Eric Raymond}{esr@thyrsus.com}
\modulesynopsis{An interface to the curses library, providing portable
                terminal handling.}

\versionchanged[Added support for the \code{ncurses} library and
                converted to a package]{1.6}

The \module{curses} module provides an interface to the curses
library, the de-facto standard for portable advanced terminal
handling.

While curses is most widely used in the \UNIX{} environment, versions
are available for DOS, OS/2, and possibly other systems as well.  This
extension module is designed to match the API of ncurses, an
open-source curses library hosted on Linux and the BSD variants of
\UNIX.

\begin{seealso}
  \seemodule{curses.ascii}{Utilities for working with \ASCII{}
                           characters, regardless of your locale
                           settings.}
  \seemodule{curses.panel}{A panel stack extension that adds depth to 
                           curses windows.}
  \seemodule{curses.textpad}{Editable text widget for curses supporting 
			     \program{Emacs}-like bindings.}
  \seemodule{curses.wrapper}{Convenience function to ensure proper
                             terminal setup and resetting on
                             application entry and exit.}
  \seetitle[http://www.python.org/doc/howto/curses/curses.html]{Curses
            Programming with Python}{Tutorial material on using curses
            with Python, by Andrew Kuchling, is available on the
            Python Web site.}
   \seetitle[Demo/curses]{}{Some example programs.}
\end{seealso}


\subsection{Functions \label{curses-functions}}

The module \module{curses} defines the following exception:

\begin{excdesc}{error}
Exception raised when a curses library function returns an error.
\end{excdesc}

\strong{Note:} Whenever \var{x} or \var{y} arguments to a function
or a method are optional, they default to the current cursor location.
Whenever \var{attr} is optional, it defaults to \constant{A_NORMAL}.

The module \module{curses} defines the following functions:

\begin{funcdesc}{baudrate}{}
Returns the output speed of the terminal in bits per second.  On
software terminal emulators it will have a fixed high value.
Included for historical reasons; in former times, it was used to 
write output loops for time delays and occasionally to change
interfaces depending on the line speed.
\end{funcdesc}

\begin{funcdesc}{beep}{}
Emit a short attention sound.
\end{funcdesc}

\begin{funcdesc}{can_change_color}{}
Returns true or false, depending on whether the programmer can change
the colors displayed by the terminal.
\end{funcdesc}

\begin{funcdesc}{cbreak}{}
Enter cbreak mode.  In cbreak mode (sometimes called ``rare'' mode)
normal tty line buffering is turned off and characters are available
to be read one by one.  However, unlike raw mode, special characters
(interrupt, quit, suspend, and flow control) retain their effects on
the tty driver and calling program.  Calling first \function{raw()}
then \function{cbreak()} leaves the terminal in cbreak mode.
\end{funcdesc}

\begin{funcdesc}{color_content}{color_number}
Returns the intensity of the red, green, and blue (RGB) components in
the color \var{color_number}, which must be between \code{0} and
\constant{COLORS}.  A 3-tuple is returned, containing the R,G,B values
for the given color, which will be between \code{0} (no component) and
\code{1000} (maximum amount of component).
\end{funcdesc}

\begin{funcdesc}{color_pair}{color_number}
Returns the attribute value for displaying text in the specified
color.  This attribute value can be combined with
\constant{A_STANDOUT}, \constant{A_REVERSE}, and the other
\constant{A_*} attributes.  \function{pair_number()} is the
counterpart to this function.
\end{funcdesc}

\begin{funcdesc}{curs_set}{visibility}
Sets the cursor state.  \var{visibility} can be set to 0, 1, or 2, for
invisible, normal, or very visible.  If the terminal supports the
visibility requested, the previous cursor state is returned;
otherwise, an exception is raised.  On many terminals, the ``visible''
mode is an underline cursor and the ``very visible'' mode is a block cursor.
\end{funcdesc}

\begin{funcdesc}{def_prog_mode}{}
Saves the current terminal mode as the ``program'' mode, the mode when
the running program is using curses.  (Its counterpart is the
``shell'' mode, for when the program is not in curses.)  Subsequent calls
to \function{reset_prog_mode()} will restore this mode.
\end{funcdesc}

\begin{funcdesc}{def_shell_mode}{}
Saves the current terminal mode as the ``shell'' mode, the mode when
the running program is not using curses.  (Its counterpart is the
``program'' mode, when the program is using curses capabilities.)
Subsequent calls
to \function{reset_shell_mode()} will restore this mode.
\end{funcdesc}

\begin{funcdesc}{delay_output}{ms}
Inserts an \var{ms} millisecond pause in output.  
\end{funcdesc}

\begin{funcdesc}{doupdate}{}
Update the physical screen.  The curses library keeps two data
structures, one representing the current physical screen contents
and a virtual screen representing the desired next state.  The
\function{doupdate()} ground updates the physical screen to match the
virtual screen.

The virtual screen may be updated by a \method{noutrefresh()} call
after write operations such as \method{addstr()} have been performed
on a window.  The normal \method{refresh()} call is simply
\method{noutrefresh()} followed by \function{doupdate()}; if you have
to update multiple windows, you can speed performance and perhaps
reduce screen flicker by issuing \method{noutrefresh()} calls on
all windows, followed by a single \function{doupdate()}.
\end{funcdesc}

\begin{funcdesc}{echo}{}
Enter echo mode.  In echo mode, each character input is echoed to the
screen as it is entered.  
\end{funcdesc}

\begin{funcdesc}{endwin}{}
De-initialize the library, and return terminal to normal status.
\end{funcdesc}

\begin{funcdesc}{erasechar}{}
Returns the user's current erase character.  Under Unix operating
systems this is a property of the controlling tty of the curses
program, and is not set by the curses library itself.
\end{funcdesc}

\begin{funcdesc}{filter}{}
The \function{filter()} routine, if used, must be called before
\function{initscr()} is  called.  The effect is that, during those
calls, LINES is set to 1; the capabilities clear, cup, cud, cud1,
cuu1, cuu, vpa are disabled; and the home string is set to the value of cr.
The effect is that the cursor is confined to the current line, and so
are screen updates.  This may be used for enabling cgaracter-at-a-time 
line editing without touching the rest of the screen.
\end{funcdesc}

\begin{funcdesc}{flash}{}
Flash the screen.  That is, change it to reverse-video and then change
it back in a short interval.  Some people prefer such as `visible bell'
to the audible attention signal produced by \function{beep()}.
\end{funcdesc}

\begin{funcdesc}{flushinp}{}
Flush all input buffers.  This throws away any  typeahead  that  has
been typed by the user and has not yet been processed by the program.
\end{funcdesc}

\begin{funcdesc}{getmouse}{}
After \method{getch()} returns \constant{KEY_MOUSE} to signal a mouse
event, this method should be call to retrieve the queued mouse event,
represented as a 5-tuple
\code{(\var{id}, \var{x}, \var{y}, \var{z}, \var{bstate})}.
\var{id} is an ID value used to distinguish multiple devices,
and \var{x}, \var{y}, \var{z} are the event's coordinates.  (\var{z}
is currently unused.).  \var{bstate} is an integer value whose bits
will be set to indicate the type of event, and will be the bitwise OR
of one or more of the following constants, where \var{n} is the button
number from 1 to 4:
\constant{BUTTON\var{n}_PRESSED},
\constant{BUTTON\var{n}_RELEASED},
\constant{BUTTON\var{n}_CLICKED},
\constant{BUTTON\var{n}_DOUBLE_CLICKED},
\constant{BUTTON\var{n}_TRIPLE_CLICKED},
\constant{BUTTON_SHIFT},
\constant{BUTTON_CTRL},
\constant{BUTTON_ALT}.
\end{funcdesc}

\begin{funcdesc}{getsyx}{}
Returns the current coordinates of the virtual screen cursor in y and
x.  If leaveok is currently true, then -1,-1 is returned.
\end{funcdesc}

\begin{funcdesc}{getwin}{file}
Reads window related data stored in the file by an earlier
\function{putwin()} call.  The routine then creates and initializes a
new window using that data, returning the new window object.
\end{funcdesc}

\begin{funcdesc}{has_colors}{}
Returns true if the terminal can display colors; otherwise, it
returns false. 
\end{funcdesc}

\begin{funcdesc}{has_ic}{}
Returns true if the terminal has insert- and delete- character
capabilities.  This function is included for historical reasons only,
as all modern software terminal emulators have such capabilities.
\end{funcdesc}

\begin{funcdesc}{has_il}{}
Returns true if the terminal has insert- and
delete-line  capabilities,  or  can  simulate  them  using
scrolling regions. This function is included for historical reasons only,
as all modern software terminal emulators have such capabilities.
\end{funcdesc}

\begin{funcdesc}{has_key}{ch}
Takes a key value \var{ch}, and returns true if the current terminal
type recognizes a key with that value.
\end{funcdesc}

\begin{funcdesc}{halfdelay}{tenths}
Used for half-delay mode, which is similar to cbreak mode in that
characters typed by the user are immediately available to the program.
However, after blocking for \var{tenths} tenths of seconds, an
exception is raised if nothing has been typed.  The value of
\var{tenths} must be a number between 1 and 255.  Use
\function{nocbreak()} to leave half-delay mode.
\end{funcdesc}

\begin{funcdesc}{init_color}{color_number, r, g, b}
Changes the definition of a color, taking the number of the color to
be changed followed by three RGB values (for the amounts of red,
green, and blue components).  The value of \var{color_number} must be
between \code{0} and \constant{COLORS}.  Each of \var{r}, \var{g},
\var{b}, must be a value between \code{0} and \code{1000}.  When
\function{init_color()} is used, all occurrences of that color on the
screen immediately change to the new definition.  This function is a
no-op on most terminals; it is active only if
\function{can_change_color()} returns \code{1}.
\end{funcdesc}

\begin{funcdesc}{init_pair}{pair_number, fg, bg}
Changes the definition of a color-pair.  It takes three arguments: the
number of the color-pair to be changed, the foreground color number,
and the background color number.  The value of \var{pair_number} must
be between \code{1} and \code{COLOR_PAIRS - 1} (the \code{0} color
pair is wired to white on black and cannot be changed).  The value of
\var{fg} and \var{bg} arguments must be between \code{0} and
\constant{COLORS}.  If the color-pair was previously initialized, the
screen is refreshed and all occurrences of that color-pair are changed
to the new definition.
\end{funcdesc}

\begin{funcdesc}{initscr}{}
Initialize the library. Returns a \class{WindowObject} which represents
the whole screen.
\end{funcdesc}

\begin{funcdesc}{isendwin}{}
Returns true if \function{endwin()} has been called (that is, the 
curses library has been deinitialized).
\end{funcdesc}

\begin{funcdesc}{keyname}{k}
Return the name of the key numbered \var{k}.  The name of a key
generating printable ASCII character is the key's character.  The name
of a control-key combination is a two-character string consisting of a
caret followed by the corresponding printable ASCII character.  The
name of an alt-key combination (128-255) is a string consisting of the
prefix `M-' followed by the name of the corresponding ASCII character.
\end{funcdesc}

\begin{funcdesc}{killchar}{}
Returns the user's current line kill character. Under Unix operating
systems this is a property of the controlling tty of the curses
program, and is not set by the curses library itself.
\end{funcdesc}

\begin{funcdesc}{longname}{}
Returns a string containing the terminfo long name field describing the current
terminal.  The maximum length of a verbose description is 128
characters.  It is defined only after the call to
\function{initscr()}.
\end{funcdesc}

\begin{funcdesc}{meta}{yes}
If \var{yes} is 1, allow 8-bit characters to be input. If \var{yes} is 0, 
allow only 7-bit chars.
\end{funcdesc}

\begin{funcdesc}{mouseinterval}{interval}
Sets the maximum time in milliseconds that can elapse between press and
release events in order for them to be recognized as a click, and
returns the previous interval value.  The default value is 200 msec,
or one fifth of a second.
\end{funcdesc}

\begin{funcdesc}{mousemask}{mousemask}
Sets the mouse events to be reported, and returns a tuple
\code{(\var{availmask}, \var{oldmask})}.  
\var{availmask} indicates which of the
specified mouse events can be reported; on complete failure it returns
0.  \var{oldmask} is the previous value of the given window's mouse
event mask.  If this function is never called, no mouse events are
ever reported.
\end{funcdesc}

\begin{funcdesc}{napms}{ms}
Sleep for \var{ms} milliseconds.
\end{funcdesc}

\begin{funcdesc}{newpad}{nlines, ncols}
Creates and returns a pointer to a new pad data structure with the
given number of lines and columns.  A pad is returned as a
window object.

A pad is like a window, except that it is not restricted by the screen
size, and is not necessarily associated with a particular part of the
screen.  Pads can be used when a large window is needed, and only a
part of the window will be on the screen at one time.  Automatic
refreshes of pads (e.g., from scrolling or echoing of input) do not
occur.  The \method{refresh()} and \method{noutrefresh()} methods of a
pad require 6 arguments to specify the part of the pad to be
displayed and the location on the screen to be used for the display.
The arguments are pminrow, pmincol, sminrow, smincol, smaxrow,
smaxcol; the p arguments refer to the upper left corner of the the pad
region to be displayed and the s arguments define a clipping box on
the screen within which the pad region is to be displayed.
\end{funcdesc}

\begin{funcdesc}{newwin}{\optional{nlines, ncols,} begin_y, begin_x}
Return a new window, whose left-upper corner is at 
\code{(\var{begin_y}, \var{begin_x})}, and whose height/width is 
\var{nlines}/\var{ncols}.  

By default, the window will extend from the 
specified position to the lower right corner of the screen.
\end{funcdesc}

\begin{funcdesc}{nl}{}
Enter newline mode.  This mode translates the return key into newline
on input, and translates newline into return and line-feed on output.
Newline mode is initially on.
\end{funcdesc}

\begin{funcdesc}{nocbreak}{}
Leave cbreak mode.  Return to normal ``cooked'' mode with line buffering.
\end{funcdesc}

\begin{funcdesc}{noecho}{}
Leave echo mode.  Echoing of input characters is turned off,
\end{funcdesc}

\begin{funcdesc}{nonl}{}
Leave newline mode.  Disable translation of return into newline on
input, and disable low-level translation of newline into
newline/return on output (but this does not change the behavior of
\code{addch('\e n')}, which always does the equivalent of return and
line feed on the virtual screen).  With translation off, curses can
sometimes speed up vertical motion a little; also, it will be able to
detect the return key on input.
\end{funcdesc}

\begin{funcdesc}{noqiflush}{}
When the noqiflush routine is used, normal flush of input and
output queues associated with the INTR, QUIT and SUSP
characters will not be done.  You may want to call
\function{noqiflush()} in a signal handler if you want output
to continue as though the interrupt had not occurred, after the
handler exits.
\end{funcdesc}

\begin{funcdesc}{noraw}{}
Leave raw mode. Return to normal ``cooked'' mode with line buffering.
\end{funcdesc}

\begin{funcdesc}{pair_content}{pair_number}
Returns a tuple \var{(fg,bg)} containing the colors for the requested
color pair.  The value of \var{pair_number} must be between 0 and
COLOR_PAIRS-1.
\end{funcdesc}

\begin{funcdesc}{pair_number}{attr}
Returns the number of the color-pair set by the attribute value \var{attr}.
\function{color_pair()} is the counterpart to this function.
\end{funcdesc}

\begin{funcdesc}{putp}{string}
Equivalent to \code{tputs(str, 1, putchar)}; emits the value of a
specified terminfo capability for the current terminal.  Note that the
output of putp always goes to standard output.
\end{funcdesc}

\begin{funcdesc}{qiflush}{ \optional{flag} }
If \var{flag} is false, the effect is the same as calling
\function{noqiflush()}. If \var{flag} is true, or no argument is
provided, the queues will be flushed when these control characters are
read.
\end{funcdesc}

\begin{funcdesc}{raw}{}
Enter raw mode.  In raw mode, normal line buffering and 
processing of interrupt, quit, suspend, and flow control keys are
turned off; characters are presented to curses input functions one
by one.
\end{funcdesc}

\begin{funcdesc}{reset_prog_mode}{}
Restores the  terminal  to ``program'' mode, as previously saved 
by \function{def_prog_mode()}.
\end{funcdesc}

\begin{funcdesc}{reset_shell_mode}{}
Restores the  terminal  to ``shell'' mode, as previously saved 
by \function{def_shell_mode()}.
\end{funcdesc}

\begin{funcdesc}{setsyx}{y, x}
Sets the virtual screen cursor to \var{y}, \var{x}.
If \var{y} and \var{x} are both -1, then leaveok is set.  
\end{funcdesc}

\begin{funcdesc}{setupterm}{\optional{termstr, fd}}
Initializes the terminal.  \var{termstr} is a string giving the
terminal name; if omitted, the value of the TERM environment variable
will be used.  \var{fd} is the file descriptor to which any
initialization sequences will be sent; if not supplied, the file
descriptor for \code{sys.stdout} will be used.
\end{funcdesc}

\begin{funcdesc}{start_color}{}
Must be called if the programmer wants to use colors, and before any
other color manipulation routine is called.  It is good
practice to call this routine right after \function{initscr()}.

\function{start_color()} initializes eight basic colors (black, red, 
green, yellow, blue, magenta, cyan, and white), and two global
variables in the \module{curses} module, \constant{COLORS} and
\constant{COLOR_PAIRS}, containing the maximum number of colors and
color-pairs the terminal can support.  It also restores the colors on
the terminal to the values they had when the terminal was just turned
on.
\end{funcdesc}

\begin{funcdesc}{termattrs}{}
Returns a logical OR of all video attributes supported by the
terminal.  This information is useful when a curses program needs
complete control over the appearance of the screen.
\end{funcdesc}

\begin{funcdesc}{termname}{}
Returns the value of the environment variable TERM, truncated to 14
characters.
\end{funcdesc}

\begin{funcdesc}{tigetflag}{capname}
Returns the value of the Boolean capability corresponding to the
terminfo capability name \var{capname}.  The value \code{-1} is
returned if \var{capname} is not a Boolean capability, or \code{0} if
it is canceled or absent from the terminal description.
\end{funcdesc}

\begin{funcdesc}{tigetnum}{capname}
Returns the value of the numeric capability corresponding to the
terminfo capability name \var{capname}.  The value \code{-2} is
returned if \var{capname} is not a numeric capability, or \code{-1} if
it is canceled or absent from the terminal description.  
\end{funcdesc}

\begin{funcdesc}{tigetstr}{capname}
Returns the value of the string capability corresponding to the
terminfo capability name \var{capname}.  \code{None} is returned if
\var{capname} is not a string capability, or is canceled or absent
from the terminal description.
\end{funcdesc}

\begin{funcdesc}{tparm}{str\optional{,...}}
Instantiates the string \var{str} with the supplied parameters, where 
\var{str} should be a parameterized string obtained from the terminfo 
database.  E.g. \code{tparm(tigetstr("cup"), 5, 3)} could result in 
\code{'\e{}033[6;4H'}, the exact result depending on terminal type.
\end{funcdesc}

\begin{funcdesc}{typeahead}{fd}
Specifies that the file descriptor \var{fd} be used for typeahead
checking.  If \var{fd} is \code{-1}, then no typeahead checking is
done.

The curses library does ``line-breakout optimization'' by looking for
typeahead periodically while updating the screen.  If input is found,
and it is coming from a tty, the current update is postponed until
refresh or doupdate is called again, allowing faster response to
commands typed in advance. This function allows specifying a different
file descriptor for typeahead checking.
\end{funcdesc}

\begin{funcdesc}{unctrl}{ch}
Returns a string which is a printable representation of the character
\var{ch}.  Control characters are displayed as a caret followed by the
character, for example as \verb|^C|. Printing characters are left as they
are.
\end{funcdesc}

\begin{funcdesc}{ungetch}{ch}
Push \var{ch} so the next \method{getch()} will return it.
\strong{Note:} only one \var{ch} can be pushed before \method{getch()}
is called.
\end{funcdesc}

\begin{funcdesc}{ungetmouse}{id, x, y, z, bstate}
Push a \constant{KEY_MOUSE} event onto the input queue, associating
the given state data with it.
\end{funcdesc}

\begin{funcdesc}{use_env}{flag}
If used, this function should be called before \function{initscr()} or
newterm are called.  When \var{flag} is false, the values of
lines and columns specified in the terminfo database will be
used, even if environment variables \envvar{LINES} and
\envvar{COLUMNS} (used by default) are set, or if curses is running in
a window (in which case default behavior would be to use the window
size if \envvar{LINES} and \envvar{COLUMNS} are not set).
\end{funcdesc}

\subsection{Window Objects \label{curses-window-objects}}

Window objects, as returned by \function{initscr()} and
\function{newwin()} above, have the
following methods:

\begin{methoddesc}{addch}{\optional{y, x,} ch\optional{, attr}}
\strong{Note:} A \emph{character} means a C character (i.e., an
\ASCII{} code), rather then a Python character (a string of length 1).
(This note is true whenever the documentation mentions a character.)
The builtin \function{ord()} is handy for conveying strings to codes.

Paint character \var{ch} at \code{(\var{y}, \var{x})} with attributes
\var{attr}, overwriting any character previously painter at that
location.  By default, the character position and attributes are the
current settings for the window object.
\end{methoddesc}

\begin{methoddesc}{addnstr}{\optional{y, x,} str, n\optional{, attr}}
Paint at most \var{n} characters of the 
string \var{str} at \code{(\var{y}, \var{x})} with attributes
\var{attr}, overwriting anything previously on the display.
\end{methoddesc}

\begin{methoddesc}{addstr}{\optional{y, x,} str\optional{, attr}}
Paint the string \var{str} at \code{(\var{y}, \var{x})} with attributes
\var{attr}, overwriting anything previously on the display.
\end{methoddesc}

\begin{methoddesc}{attroff}{attr}
Remove attribute \var{attr} from the ``background'' set applied to all
writes to the current window.
\end{methoddesc}

\begin{methoddesc}{attron}{attr}
Add attribute \var{attr} from the ``background'' set applied to all
writes to the current window.
\end{methoddesc}

\begin{methoddesc}{attrset}{attr}
Set the ``background'' set of attributes to \var{attr}.  This set is
initially 0 (no attributes).
\end{methoddesc}

\begin{methoddesc}{bkgd}{ch\optional{, attr}}
Sets the background property of the window to the character \var{ch},
with attributes \var{attr}.  The change is then applied to every
character position in that window:
\begin{itemize}
\item  
The attribute of every character in the window  is
changed to the new background attribute.
\item
Wherever  the  former background character appears,
it is changed to the new background character.
\end{itemize}

\end{methoddesc}

\begin{methoddesc}{bkgdset}{ch\optional{, attr}}
Sets the window's background.  A window's background consists of a
character and any combination of attributes.  The attribute part of
the background is combined (OR'ed) with all non-blank characters that
are written into the window.  Both the character and attribute parts
of the background are combined with the blank characters.  The
background becomes a property of the character and moves with the
character through any scrolling and insert/delete line/character
operations.
\end{methoddesc}

\begin{methoddesc}{border}{\optional{ls\optional{, rs\optional{, ts\optional{,
                           bs\optional{, tl\optional{, tr\optional{,
                           bl\optional{, br}}}}}}}}}
Draw a border around the edges of the window. Each parameter specifies 
the character to use for a specific part of the border; see the table
below for more details.  The characters must be specified as integers;
using one-character strings will cause \exception{TypeError} to be
raised.

\strong{Note:} A \code{0} value for any parameter will cause the
default character to be used for that parameter.  Keyword parameters
can \emph{not} be used.  The defaults are listed in this table:

\begin{tableiii}{l|l|l}{var}{Parameter}{Description}{Default value}
  \lineiii{ls}{Left side}{\constant{ACS_VLINE}}
  \lineiii{rs}{Right side}{\constant{ACS_VLINE}}
  \lineiii{ts}{Top}{\constant{ACS_HLINE}}
  \lineiii{bs}{Bottom}{\constant{ACS_HLINE}}
  \lineiii{tl}{Upper-left corner}{\constant{ACS_ULCORNER}}
  \lineiii{tr}{Upper-right corner}{\constant{ACS_URCORNER}}
  \lineiii{bl}{Bottom-left corner}{\constant{ACS_BLCORNER}}
  \lineiii{br}{Bottom-right corner}{\constant{ACS_BRCORNER}}
\end{tableiii}
\end{methoddesc}

\begin{methoddesc}{box}{\optional{vertch, horch}}
Similar to \method{border()}, but both \var{ls} and \var{rs} are
\var{vertch} and both \var{ts} and {bs} are \var{horch}.  The default
corner characters are always used by this function.
\end{methoddesc}

\begin{methoddesc}{clear}{}
Like \method{erase()}, but also causes the whole window to be repainted
upon next call to \method{refresh()}.
\end{methoddesc}

\begin{methoddesc}{clearok}{yes}
If \var{yes} is 1, the next call to \method{refresh()}
will clear the window completely.
\end{methoddesc}

\begin{methoddesc}{clrtobot}{}
Erase from cursor to the end of the window: all lines below the cursor
are deleted, and then the equivalent of \method{clrtoeol()} is performed.
\end{methoddesc}

\begin{methoddesc}{clrtoeol}{}
Erase from cursor to the end of the line.
\end{methoddesc}

\begin{methoddesc}{cursyncup}{}
Updates the current cursor position of all the ancestors of the window
to reflect the current cursor position of the window.
\end{methoddesc}

\begin{methoddesc}{delch}{\optional{x, y}}
Delete any character at \code{(\var{y}, \var{x})}.
\end{methoddesc}

\begin{methoddesc}{deleteln}{}
Delete the line under the cursor. All following lines are moved up
by 1 line.
\end{methoddesc}

\begin{methoddesc}{derwin}{\optional{nlines, ncols,} begin_y, begin_y}
An abbreviation for ``derive window'', \method{derwin()} is the same
as calling \method{subwin()}, except that \var{begin_y} and
\var{begin_x} are relative to the origin of the window, rather than
relative to the entire screen.  Returns a window object for the
derived window.
\end{methoddesc}

\begin{methoddesc}{echochar}{ch\optional{, attr}}
Add character \var{ch} with attribute \var{attr}, and immediately 
call \method{refresh()} on the window.
\end{methoddesc}

\begin{methoddesc}{enclose}{y, x}
Tests whether the given pair of screen-relative character-cell
coordinates are enclosed by the given window, returning true or
false.  It is useful for determining what subset of the screen
windows enclose the location of a mouse event.
\end{methoddesc}

\begin{methoddesc}{erase}{}
Clear the window.
\end{methoddesc}

\begin{methoddesc}{getbegyx}{}
Return a tuple \code{(\var{y}, \var{x})} of co-ordinates of upper-left
corner.
\end{methoddesc}

\begin{methoddesc}{getch}{\optional{x, y}}
Get a character. Note that the integer returned does \emph{not} have to
be in \ASCII{} range: function keys, keypad keys and so on return numbers
higher than 256. In no-delay mode, an exception is raised if there is 
no input.
\end{methoddesc}

\begin{methoddesc}{getkey}{\optional{x, y}}
Get a character, returning a string instead of an integer, as
\method{getch()} does. Function keys, keypad keys and so on return a
multibyte string containing the key name.  In no-delay mode, an
exception is raised if there is no input.
\end{methoddesc}

\begin{methoddesc}{getmaxyx}{}
Return a tuple \code{(\var{y}, \var{x})} of the height and width of
the window.
\end{methoddesc}

\begin{methoddesc}{getparyx}{}
Returns the beginning coordinates of this window relative to its
parent window into two integer variables y and x.  Returns
\code{-1,-1} if this window has no parent.
\end{methoddesc}

\begin{methoddesc}{getstr}{\optional{x, y}}
Read a string from the user, with primitive line editing capacity.
\end{methoddesc}

\begin{methoddesc}{getyx}{}
Return a tuple \code{(\var{y}, \var{x})} of current cursor position 
relative to the window's upper-left corner.
\end{methoddesc}

\begin{methoddesc}{hline}{\optional{y, x,} ch, n}
Display a horizontal line starting at \code{(\var{y}, \var{x})} with
length \var{n} consisting of the character \var{ch}.
\end{methoddesc}

\begin{methoddesc}{idcok}{flag}
If \var{flag} is false, curses no longer considers using the hardware
insert/delete character feature of the terminal; if \var{flag} is
true, use of character insertion and deletion is enabled.  When curses
is first initialized, use of character insert/delete is enabled by
default.
\end{methoddesc}

\begin{methoddesc}{idlok}{yes}
If called with \var{yes} equal to 1, \module{curses} will try and use
hardware line editing facilities. Otherwise, line insertion/deletion
are disabled.
\end{methoddesc}

\begin{methoddesc}{immedok}{flag}
If \var{flag} is true, any change in the window image
automatically causes the window to be refreshed; you no longer
have to call \method{refresh()} yourself.  However, it may
degrade performance considerably, due to repeated calls to
wrefresh.  This option is disabled by default.
\end{methoddesc}

\begin{methoddesc}{inch}{\optional{x, y}}
Return the character at the given position in the window. The bottom
8 bits are the character proper, and upper bits are the attributes.
\end{methoddesc}

\begin{methoddesc}{insch}{\optional{y, x,} ch\optional{, attr}}
Paint character \var{ch} at \code{(\var{y}, \var{x})} with attributes
\var{attr}, moving the line from position \var{x} right by one
character.
\end{methoddesc}

\begin{methoddesc}{insdelln}{nlines}
Inserts \var{nlines} lines into the specified window above the current
line.  The \var{nlines} bottom lines are lost.  For negative
\var{nlines}, delete \var{nlines} lines starting with the one under
the cursor, and move the remaining lines up.  The bottom \var{nlines}
lines are cleared.  The current cursor position remains the same.
\end{methoddesc}

\begin{methoddesc}{insertln}{}
Insert a blank line under the cursor. All following lines are moved
down by 1 line.
\end{methoddesc}

\begin{methoddesc}{insnstr}{\optional{y, x, } str, n \optional{, attr}}
Insert a character string (as many characters as will fit on the line)
before the character under the cursor, up to \var{n} characters.  
If \var{n} is zero or negative,
the entire string is inserted.
All characters to the right of
the cursor are shifted right, with the the rightmost characters on the
line being lost.  The cursor position does not change (after moving to
\var{y}, \var{x}, if specified). 
\end{methoddesc}

\begin{methoddesc}{insstr}{\optional{y, x, } str \optional{, attr}}
Insert a character string (as many characters as will fit on the line)
before the character under the cursor.  All characters to the right of
the cursor are shifted right, with the the rightmost characters on the
line being lost.  The cursor position does not change (after moving to
\var{y}, \var{x}, if specified). 
\end{methoddesc}

\begin{methoddesc}{instr}{\optional{y, x} \optional{, n}}
Returns a string of characters, extracted from the window starting at
the current cursor position, or at \var{y}, \var{x} if specified.
Attributes are stripped from the characters.  If \var{n} is specified,
\method{instr()} returns return a string at most \var{n} characters
long (exclusive of the trailing NUL).
\end{methoddesc}

\begin{methoddesc}{is_linetouched}{\var{line}}
Returns true if the specified line was modified since the last call to
\method{refresh()}; otherwise returns false.  Raises a
\exception{curses.error} exception if \var{line} is not valid
for the given window.
\end{methoddesc}

\begin{methoddesc}{is_wintouched}{}
Returns true if the specified window was modified since the last call to
\method{refresh()}; otherwise returns false.
\end{methoddesc}

\begin{methoddesc}{keypad}{yes}
If \var{yes} is 1, escape sequences generated by some keys (keypad, 
function keys) will be interpreted by \module{curses}.
If \var{yes} is 0, escape sequences will be left as is in the input
stream.
\end{methoddesc}

\begin{methoddesc}{leaveok}{yes}
If \var{yes} is 1, cursor is left where it is on update, instead of
being at ``cursor position.''  This reduces cursor movement where
possible. If possible the cursor will be made invisible.

If \var{yes} is 0, cursor will always be at ``cursor position'' after
an update.
\end{methoddesc}

\begin{methoddesc}{move}{new_y, new_x}
Move cursor to \code{(\var{new_y}, \var{new_x})}.
\end{methoddesc}

\begin{methoddesc}{mvderwin}{y, x}
Moves the window inside its parent window.  The screen-relative
parameters of the window are not changed.  This routine is used to
display different parts of the parent window at the same physical
position on the screen.
\end{methoddesc}

\begin{methoddesc}{mvwin}{new_y, new_x}
Move the window so its upper-left corner is at
\code{(\var{new_y}, \var{new_x})}.
\end{methoddesc}

\begin{methoddesc}{nodelay}{yes}
If \var{yes} is \code{1}, \method{getch()} will be non-blocking.
\end{methoddesc}

\begin{methoddesc}{notimeout}{yes}
If \var{yes} is \code{1}, escape sequences will not be timed out.

If \var{yes} is \code{0}, after a few milliseconds, an escape sequence
will not be interpreted, and will be left in the input stream as is.
\end{methoddesc}

\begin{methoddesc}{noutrefresh}{}
Mark for refresh but wait.  This function updates the data structure
representing the desired state of the window, but does not force
an update of the physical screen.  To accomplish that, call 
\function{doupdate()}.
\end{methoddesc}

\begin{methoddesc}{overlay}{destwin\optional{, sminrow, smincol,
                            dminrow, dmincol, dmaxrow, dmaxcol}}
Overlay the window on top of \var{destwin}. The windows need not be
the same size, only the overlapping region is copied. This copy is
non-destructive, which means that the current background character
does not overwrite the old contents of \var{destwin}.

To get fine-grained control over the copied region, the second form
of \method{overlay()} can be used. \var{sminrow} and \var{smincol} are
the upper-left coordinates of the source window, and the other variables
mark a rectangle in the destination window.
\end{methoddesc}

\begin{methoddesc}{overwrite}{destwin\optional{, sminrow, smincol,
                              dminrow, dmincol, dmaxrow, dmaxcol}}
Overwrite the window on top of \var{destwin}. The windows need not be
the same size, in which case only the overlapping region is
copied. This copy is destructive, which means that the current
background character overwrites the old contents of \var{destwin}.

To get fine-grained control over the copied region, the second form
of \method{overwrite()} can be used. \var{sminrow} and \var{smincol} are
the upper-left coordinates of the source window, the other variables
mark a rectangle in the destination window.
\end{methoddesc}

\begin{methoddesc}{putwin}{file}
Writes all data associated with the window into the provided file
object.  This information can be later retrieved using the
\function{getwin()} function.

\end{methoddesc}

\begin{methoddesc}{redrawln}{beg, num}
Indicates that the \var{num} screen lines, starting at line \var{beg},
are corrupted and should be completely redrawn on the next
\method{refresh()} call.
\end{methoddesc}

\begin{methoddesc}{redrawwin}{}
Touches the entire window, causing it to be completely redrawn on the
next \method{refresh()} call.
\end{methoddesc}

\begin{methoddesc}{refresh}{\optional{pminrow, pmincol, sminrow,
                                      smincol, smaxrow, smaxcol}}
Update the display immediately (sync actual screen with previous
drawing/deleting methods).

The 6 optional arguments can only be specified when the window is a
pad created with \function{newpad()}.  The additional parameters are
needed to indicate what part of the pad and screen are involved.
\var{pminrow} and \var{pmincol} specify the upper left-hand corner of the
rectangle to be displayed in the pad.  \var{sminrow}, \var{smincol},
\var{smaxrow}, and \var{smaxcol} specify the edges of the rectangle to
be displayed on the screen.  The lower right-hand corner of the
rectangle to be displayed in the pad is calculated from the screen
coordinates, since the rectangles must be the same size.  Both
rectangles must be entirely contained within their respective
structures.  Negative values of \var{pminrow}, \var{pmincol},
\var{sminrow}, or \var{smincol} are treated as if they were zero.
\end{methoddesc}

\begin{methoddesc}{scroll}{\optional{lines\code{ = 1}}}
Scroll the screen or scrolling region upward by \var{lines} lines.
\end{methoddesc}

\begin{methoddesc}{scrollok}{flag}
Controls what happens when the cursor of a window is moved off the
edge of the window or scrolling region, either as a result of a
newline action on the bottom line, or typing the last character
of the last line.  If \var{flag} is false, the cursor is left
on the bottom line.  If \var{flag} is true, the window is
scrolled up one line.  Note that in order to get the physical
scrolling effect on the terminal, it is also necessary to call
\method{idlok()}.
\end{methoddesc}

\begin{methoddesc}{setscrreg}{top, bottom}
Set the scrolling region from line \var{top} to line \var{bottom}. All
scrolling actions will take place in this region.
\end{methoddesc}

\begin{methoddesc}{standend}{}
Turn off the standout attribute.  On some terminals this has the
side effect of turning off all attributes.
\end{methoddesc}

\begin{methoddesc}{standout}{}
Turn on attribute \var{A_STANDOUT}.
\end{methoddesc}

\begin{methoddesc}{subpad}{\optional{nlines, ncols,} begin_y, begin_y}
Return a sub-window, whose upper-left corner is at
\code{(\var{begin_y}, \var{begin_x})}, and whose width/height is
\var{ncols}/\var{nlines}.
\end{methoddesc}

\begin{methoddesc}{subwin}{\optional{nlines, ncols,} begin_y, begin_y}
Return a sub-window, whose upper-left corner is at
\code{(\var{begin_y}, \var{begin_x})}, and whose width/height is
\var{ncols}/\var{nlines}.

By default, the sub-window will extend from the
specified position to the lower right corner of the window.
\end{methoddesc}

\begin{methoddesc}{syncdown}{}
Touches each location in the window that has been touched in any of
its ancestor windows.  This routine is called by \method{refresh()},
so it should almost never be necessary to call it manually.
\end{methoddesc}

\begin{methoddesc}{syncok}{flag}
If called with \var{flag} set to true, then \method{syncup()} is
called automatically whenever there is a change in the window.
\end{methoddesc}

\begin{methoddesc}{syncup}{}
Touches all locations in ancestors of the window that have been changed in 
the window.  
\end{methoddesc}

\begin{methoddesc}{timeout}{delay}
Sets blocking or non-blocking read behavior for the window.  If
\var{delay} is negative, blocking read is used, which will wait
indefinitely for input).  If \var{delay} is zero, then non-blocking
read is used, and -1 will be returned by \method{getch()} if no input
is waiting.  If \var{delay} is positive, then \method{getch()} will
block for \var{delay} milliseconds, and return -1 if there is still no
input at the end of that time.
\end{methoddesc}

\begin{methoddesc}{touchline}{start, count}
Pretend \var{count} lines have been changed, starting with line
\var{start}.
\end{methoddesc}

\begin{methoddesc}{touchwin}{}
Pretend the whole window has been changed, for purposes of drawing
optimizations.
\end{methoddesc}

\begin{methoddesc}{untouchwin}{}
Marks all lines in  the  window  as unchanged since the last call to
\method{refresh()}. 
\end{methoddesc}

\begin{methoddesc}{vline}{\optional{y, x,} ch, n}
Display a vertical line starting at \code{(\var{y}, \var{x})} with
length \var{n} consisting of the character \var{ch}.
\end{methoddesc}

\subsection{Constants}

The \module{curses} module defines the following data members:

\begin{datadesc}{ERR}
Some curses routines  that  return  an integer, such as 
\function{getch()}, return \constant{ERR} upon failure.  
\end{datadesc}

\begin{datadesc}{OK}
Some curses routines  that  return  an integer, such as 
\function{napms()}, return \constant{OK} upon success.  
\end{datadesc}

\begin{datadesc}{version}
A string representing the current version of the module. 
Also available as \constant{__version__}.
\end{datadesc}

Several constants are available to specify character cell attributes:

\begin{tableii}{l|l}{code}{Attribute}{Meaning}
  \lineii{A_ALTCHARSET}{Alternate character set mode.}
  \lineii{A_BLINK}{Blink mode.}
  \lineii{A_BOLD}{Bold mode.}
  \lineii{A_DIM}{Dim mode.}
  \lineii{A_NORMAL}{Normal attribute.}
  \lineii{A_STANDOUT}{Standout mode.}
  \lineii{A_UNDERLINE}{Underline mode.}
\end{tableii}

Keys are referred to by integer constants with names starting with 
\samp{KEY_}.   The exact keycaps available are system dependent.

% XXX this table is far too large!
% XXX should this table be alphabetized?

\begin{longtableii}{l|l}{code}{Key constant}{Key}
  \lineii{KEY_MIN}{Minimum key value}
  \lineii{KEY_BREAK}{ Break key (unreliable) }
  \lineii{KEY_DOWN}{ Down-arrow }
  \lineii{KEY_UP}{ Up-arrow }
  \lineii{KEY_LEFT}{ Left-arrow }
  \lineii{KEY_RIGHT}{ Right-arrow }
  \lineii{KEY_HOME}{ Home key (upward+left arrow) }
  \lineii{KEY_BACKSPACE}{ Backspace (unreliable) }
  \lineii{KEY_F0}{ Function keys.  Up to 64 function keys are supported. }
  \lineii{KEY_F\var{n}}{ Value of function key \var{n} }
  \lineii{KEY_DL}{ Delete line }
  \lineii{KEY_IL}{ Insert line }
  \lineii{KEY_DC}{ Delete character }
  \lineii{KEY_IC}{ Insert char or enter insert mode }
  \lineii{KEY_EIC}{ Exit insert char mode }
  \lineii{KEY_CLEAR}{ Clear screen }
  \lineii{KEY_EOS}{ Clear to end of screen }
  \lineii{KEY_EOL}{ Clear to end of line }
  \lineii{KEY_SF}{ Scroll 1 line forward }
  \lineii{KEY_SR}{ Scroll 1 line backward (reverse) }
  \lineii{KEY_NPAGE}{ Next page }
  \lineii{KEY_PPAGE}{ Previous page }
  \lineii{KEY_STAB}{ Set tab }
  \lineii{KEY_CTAB}{ Clear tab }
  \lineii{KEY_CATAB}{ Clear all tabs }
  \lineii{KEY_ENTER}{ Enter or send (unreliable) }
  \lineii{KEY_SRESET}{ Soft (partial) reset (unreliable) }
  \lineii{KEY_RESET}{ Reset or hard reset (unreliable) }
  \lineii{KEY_PRINT}{ Print }
  \lineii{KEY_LL}{ Home down or bottom (lower left) }
  \lineii{KEY_A1}{ Upper left of keypad }
  \lineii{KEY_A3}{ Upper right of keypad }
  \lineii{KEY_B2}{ Center of keypad }
  \lineii{KEY_C1}{ Lower left of keypad }
  \lineii{KEY_C3}{ Lower right of keypad }
  \lineii{KEY_BTAB}{ Back tab }
  \lineii{KEY_BEG}{ Beg (beginning) }
  \lineii{KEY_CANCEL}{ Cancel }
  \lineii{KEY_CLOSE}{ Close }
  \lineii{KEY_COMMAND}{ Cmd (command) }
  \lineii{KEY_COPY}{ Copy }
  \lineii{KEY_CREATE}{ Create }
  \lineii{KEY_END}{ End }
  \lineii{KEY_EXIT}{ Exit }
  \lineii{KEY_FIND}{ Find }
  \lineii{KEY_HELP}{ Help }
  \lineii{KEY_MARK}{ Mark }
  \lineii{KEY_MESSAGE}{ Message }
  \lineii{KEY_MOVE}{ Move }
  \lineii{KEY_NEXT}{ Next }
  \lineii{KEY_OPEN}{ Open }
  \lineii{KEY_OPTIONS}{ Options }
  \lineii{KEY_PREVIOUS}{ Prev (previous) }
  \lineii{KEY_REDO}{ Redo }
  \lineii{KEY_REFERENCE}{ Ref (reference) }
  \lineii{KEY_REFRESH}{ Refresh }
  \lineii{KEY_REPLACE}{ Replace }
  \lineii{KEY_RESTART}{ Restart }
  \lineii{KEY_RESUME}{ Resume }
  \lineii{KEY_SAVE}{ Save }
  \lineii{KEY_SBEG}{ Shifted Beg (beginning) }
  \lineii{KEY_SCANCEL}{ Shifted Cancel }
  \lineii{KEY_SCOMMAND}{ Shifted Command }
  \lineii{KEY_SCOPY}{ Shifted Copy }
  \lineii{KEY_SCREATE}{ Shifted Create }
  \lineii{KEY_SDC}{ Shifted Delete char }
  \lineii{KEY_SDL}{ Shifted Delete line }
  \lineii{KEY_SELECT}{ Select }
  \lineii{KEY_SEND}{ Shifted End }
  \lineii{KEY_SEOL}{ Shifted Clear line }
  \lineii{KEY_SEXIT}{ Shifted Dxit }
  \lineii{KEY_SFIND}{ Shifted Find }
  \lineii{KEY_SHELP}{ Shifted Help }
  \lineii{KEY_SHOME}{ Shifted Home }
  \lineii{KEY_SIC}{ Shifted Input }
  \lineii{KEY_SLEFT}{ Shifted Left arrow }
  \lineii{KEY_SMESSAGE}{ Shifted Message }
  \lineii{KEY_SMOVE}{ Shifted Move }
  \lineii{KEY_SNEXT}{ Shifted Next }
  \lineii{KEY_SOPTIONS}{ Shifted Options }
  \lineii{KEY_SPREVIOUS}{ Shifted Prev }
  \lineii{KEY_SPRINT}{ Shifted Print }
  \lineii{KEY_SREDO}{ Shifted Redo }
  \lineii{KEY_SREPLACE}{ Shifted Replace }
  \lineii{KEY_SRIGHT}{ Shifted Right arrow }
  \lineii{KEY_SRSUME}{ Shifted Resume }
  \lineii{KEY_SSAVE}{ Shifted Save }
  \lineii{KEY_SSUSPEND}{ Shifted Suspend }
  \lineii{KEY_SUNDO}{ Shifted Undo }
  \lineii{KEY_SUSPEND}{ Suspend }
  \lineii{KEY_UNDO}{ Undo }
  \lineii{KEY_MOUSE}{ Mouse event has occurred }
  \lineii{KEY_RESIZE}{ Terminal resize event }
  \lineii{KEY_MAX}{Maximum key value}
\end{longtableii}

On VT100s and their software emulations, such as X terminal emulators,
there are normally at least four function keys (\constant{KEY_F1},
\constant{KEY_F2}, \constant{KEY_F3}, \constant{KEY_F4}) available,
and the arrow keys mapped to \constant{KEY_UP}, \constant{KEY_DOWN},
\constant{KEY_LEFT} and \constant{KEY_RIGHT} in the obvious way.  If
your machine has a PC keybboard, it is safe to expect arrow keys and
twelve function keys (older PC keyboards may have only ten function
keys); also, the following keypad mappings are standard:

\begin{tableii}{l|l}{kbd}{Keycap}{Constant}
   \lineii{Insert}{KEY_IC}
   \lineii{Delete}{KEY_DC}
   \lineii{Home}{KEY_HOME}
   \lineii{End}{KEY_END}
   \lineii{Page Up}{KEY_NPAGE}
   \lineii{Page Down}{KEY_PPAGE}
\end{tableii}

The following table lists characters from the alternate character set.
These are inherited from the VT100 terminal, and will generally be 
available on software emulations such as X terminals.  When there
is no graphic available, curses falls back on a crude printable ASCII
approximation.
\strong{Note:} These are available only after \function{initscr()} has 
been called.

\begin{longtableii}{l|l}{code}{ACS code}{Meaning}
  \lineii{ACS_BBSS}{alternate name for upper right corner}
  \lineii{ACS_BLOCK}{solid square block}
  \lineii{ACS_BOARD}{board of squares}
  \lineii{ACS_BSBS}{alternate name for horizontal line}
  \lineii{ACS_BSSB}{alternate name for upper left corner}
  \lineii{ACS_BSSS}{alternate name for top tee}
  \lineii{ACS_BTEE}{bottom tee}
  \lineii{ACS_BULLET}{bullet}
  \lineii{ACS_CKBOARD}{checker board (stipple)}
  \lineii{ACS_DARROW}{arrow pointing down}
  \lineii{ACS_DEGREE}{degree symbol}
  \lineii{ACS_DIAMOND}{diamond}
  \lineii{ACS_GEQUAL}{greater-than-or-equal-to}
  \lineii{ACS_HLINE}{horizontal line}
  \lineii{ACS_LANTERN}{lantern symbol}
  \lineii{ACS_LARROW}{left arrow}
  \lineii{ACS_LEQUAL}{less-than-or-equal-to}
  \lineii{ACS_LLCORNER}{lower left-hand corner}
  \lineii{ACS_LRCORNER}{lower right-hand corner}
  \lineii{ACS_LTEE}{left tee}
  \lineii{ACS_NEQUAL}{not-equal sign}
  \lineii{ACS_PI}{letter pi}
  \lineii{ACS_PLMINUS}{plus-or-minus sign}
  \lineii{ACS_PLUS}{big plus sign}
  \lineii{ACS_RARROW}{right arrow}
  \lineii{ACS_RTEE}{right tee}
  \lineii{ACS_S1}{scan line 1}
  \lineii{ACS_S3}{scan line 3}
  \lineii{ACS_S7}{scan line 7}
  \lineii{ACS_S9}{scan line 9}
  \lineii{ACS_SBBS}{alternate name for lower right corner}
  \lineii{ACS_SBSB}{alternate name for vertical line}
  \lineii{ACS_SBSS}{alternate name for right tee}
  \lineii{ACS_SSBB}{alternate name for lower left corner}
  \lineii{ACS_SSBS}{alternate name for bottom tee}
  \lineii{ACS_SSSB}{alternate name for left tee}
  \lineii{ACS_SSSS}{alternate name for crossover or big plus}
  \lineii{ACS_STERLING}{pound sterling}
  \lineii{ACS_TTEE}{top tee}
  \lineii{ACS_UARROW}{up arrow}
  \lineii{ACS_ULCORNER}{upper left corner}
  \lineii{ACS_URCORNER}{upper right corner}
  \lineii{ACS_VLINE}{vertical line}
\end{longtableii}

The following table lists the predefined colors:

\begin{tableii}{l|l}{code}{Constant}{Color}
  \lineii{COLOR_BLACK}{Black}
  \lineii{COLOR_BLUE}{Blue}
  \lineii{COLOR_CYAN}{Cyan (light greenish blue)}
  \lineii{COLOR_GREEN}{Green}
  \lineii{COLOR_MAGENTA}{Magenta (purplish red)}
  \lineii{COLOR_RED}{Red}
  \lineii{COLOR_WHITE}{White}
  \lineii{COLOR_YELLOW}{Yellow}
\end{tableii}

\section{\module{curses.textpad} ---
         Text input widget for curses programs}

\declaremodule{standard}{curses.textpad}
\sectionauthor{Eric Raymond}{esr@thyrsus.com}
\moduleauthor{Eric Raymond}{esr@thyrsus.com}
\modulesynopsis{Emacs-like input editing in a curses window.}
\versionadded{1.6}

The \module{curses.textpad} module provides a \class{Textbox} class
that handles elementary text editing in a curses window, supporting a
set of keybindings resembling those of Emacs (thus, also of Netscape
Navigator, BBedit 6.x, FrameMaker, and many other programs).  The
module also provides a rectangle-drawing function useful for framing
text boxes or for other purposes.

The module \module{curses.textpad} defines the following function:

\begin{funcdesc}{rectangle}{win, uly, ulx, lry, lrx}
Draw a rectangle.  The first argument must be a window object; the
remaining arguments are coordinates relative to that window.  The
second and third arguments are the y and x coordinates of the upper
left hand corner of the rectangle To be drawn; the fourth and fifth
arguments are the y and x coordinates of the lower right hand corner.
The rectangle will be drawn using VT100/IBM PC forms characters on
terminals that make this possible (including xterm and most other
software terminal emulators).  Otherwise it will be drawn with ASCII 
dashes, vertical bars, and plus signs.
\end{funcdesc}


\subsection{Textbox objects \label{curses-textpad-objects}}

You can instantiate a \class{Textbox} object as follows:

\begin{classdesc}{Textbox}{win}
Return a textbox widget object.  The \var{win} argument should be a
curses \class{WindowObject} in which the textbox is to be contained.
The edit cursor of the textbox is initially located at the upper left
hand corner of the containin window, with coordinates \code{(0, 0)}.
The instance's \member{stripspaces} flag is initially on.
\end{classdesc}

\class{Textbox} objects have the following methods:

\begin{methoddesc}{edit}{\optional{validator}}
This is the entry point you will normally use.  It accepts editing
keystrokes until one of the termination keystrokes is entered.  If
\var{validator} is supplied, it must be a function.  It will be called
for each keystroke entered with the keystroke as a parameter; command
dispatch is done on the result. This method returns the window
contents as a string; whether blanks in the window are included is
affected by the \member{stripspaces} member.
\end{methoddesc}

\begin{methoddesc}{do_command}{ch}
Process a single command keystroke.  Here are the supported special
keystrokes: 

\begin{tableii}{l|l}{kbd}{Keystroke}{Action}
  \lineii{Ctrl-A}{Go to left edge of window.}
  \lineii{Ctrl-B}{Cursor left, wrapping to previous line if appropriate.}
  \lineii{Ctrl-D}{Delete character under cursor.}
  \lineii{Ctrl-E}{Go to right edge (stripspaces off) or end of line
                  (stripspaces on).}
  \lineii{Ctrl-F}{Cursor right, wrapping to next line when appropriate.}
  \lineii{Ctrl-G}{Terminate, returning the window contents.}
  \lineii{Ctrl-H}{Delete character backward.}
  \lineii{Ctrl-J}{Terminate if the window is 1 line, otherwise insert newline.}
  \lineii{Ctrl-K}{If line is blank, delete it, otherwise clear to end of line.}
  \lineii{Ctrl-L}{Refresh screen.}
  \lineii{Ctrl-N}{Cursor down; move down one line.}
  \lineii{Ctrl-O}{Insert a blank line at cursor location.}
  \lineii{Ctrl-P}{Cursor up; move up one line.}
\end{tableii}

Move operations do nothing if the cursor is at an edge where the
movement is not possible.  The following synonyms are supported where
possible:

\begin{tableii}{l|l}{constant}{Constant}{Keystroke}
  \lineii{KEY_LEFT}{\kbd{Ctrl-B}}
  \lineii{KEY_RIGHT}{\kbd{Ctrl-F}}
  \lineii{KEY_UP}{\kbd{Ctrl-P}}
  \lineii{KEY_DOWN}{\kbd{Ctrl-N}}
  \lineii{KEY_BACKSPACE}{\kbd{Ctrl-h}}
\end{tableii}

All other keystrokes are treated as a command to insert the given
character and move right (with line wrapping).
\end{methoddesc}

\begin{methoddesc}{gather}{}
This method returns the window contents as a string; whether blanks in
the window are included is affected by the \member{stripspaces}
member.
\end{methoddesc}

\begin{memberdesc}{stripspaces}
This data member is a flag which controls the interpretation of blanks in
the window.  When it is on, trailing blanks on each line are ignored;
any cursor motion that would land the cursor on a trailing blank goes
to the end of that line instead, and trailing blanks are stripped when
the window contents is gathered.
\end{memberdesc}


\section{\module{curses.wrapper} ---
         Terminal handler for curses programs}

\declaremodule{standard}{curses.wrapper}
\sectionauthor{Eric Raymond}{esr@thyrsus.com}
\moduleauthor{Eric Raymond}{esr@thyrsus.com}
\modulesynopsis{Terminal configuration wrapper for curses programs.}
\versionadded{1.6}

This module supplies one function, \function{wrapper()}, which runs
another function which should be the rest of your curses-using
application.  If the application raises an exception,
\function{wrapper()} will restore the terminal to a sane state before
passing it further up the stack and generating a traceback.

\begin{funcdesc}{wrapper}{func, \moreargs}
Wrapper function that initializes curses and calls another function,
\var{func}, restoring normal keyboard/screen behavior on error.
The callable object \var{func} is then passed the main window 'stdscr'
as its first argument, followed by any other arguments passed to
\function{wrapper()}.
\end{funcdesc}

Before calling the hook function, \function{wrapper()} turns on cbreak
mode, turns off echo, enables the terminal keypad, and initializes
colors if the terminal has color support.  On exit (whether normally
or by exception) it restores cooked mode, turns on echo, and disables
the terminal keypad.


\section{\module{getopt} ---
         Parser for command line options}

\declaremodule{standard}{getopt}
\modulesynopsis{Portable parser for command line options; support both
                short and long option names.}


This module helps scripts to parse the command line arguments in
\code{sys.argv}.
It supports the same conventions as the \UNIX{} \cfunction{getopt()}
function (including the special meanings of arguments of the form
`\code{-}' and `\code{-}\code{-}').
% That's to fool latex2html into leaving the two hyphens alone!
Long options similar to those supported by
GNU software may be used as well via an optional third argument.
This module provides a single function and an exception:

\begin{funcdesc}{getopt}{args, options\optional{, long_options}}
Parses command line options and parameter list.  \var{args} is the
argument list to be parsed, without the leading reference to the
running program. Typically, this means \samp{sys.argv[1:]}.
\var{options} is the string of option letters that the script wants to
recognize, with options that require an argument followed by a colon
(\character{:}; i.e., the same format that \UNIX{}
\cfunction{getopt()} uses).

\note{Unlike GNU \cfunction{getopt()}, after a non-option
argument, all further arguments are considered also non-options.
This is similar to the way non-GNU \UNIX{} systems work.}

\var{long_options}, if specified, must be a list of strings with the
names of the long options which should be supported.  The leading
\code{'-}\code{-'} characters should not be included in the option
name.  Long options which require an argument should be followed by an
equal sign (\character{=}).  To accept only long options,
\var{options} should be an empty string.  Long options on the command
line can be recognized so long as they provide a prefix of the option
name that matches exactly one of the accepted options.  For example,
if \var{long_options} is \code{['foo', 'frob']}, the option
\longprogramopt{fo} will match as \longprogramopt{foo}, but
\longprogramopt{f} will not match uniquely, so \exception{GetoptError}
will be raised.

The return value consists of two elements: the first is a list of
\code{(\var{option}, \var{value})} pairs; the second is the list of
program arguments left after the option list was stripped (this is a
trailing slice of \var{args}).  Each option-and-value pair returned
has the option as its first element, prefixed with a hyphen for short
options (e.g., \code{'-x'}) or two hyphens for long options (e.g.,
\code{'-}\code{-long-option'}), and the option argument as its second
element, or an empty string if the option has no argument.  The
options occur in the list in the same order in which they were found,
thus allowing multiple occurrences.  Long and short options may be
mixed.
\end{funcdesc}

\begin{funcdesc}{gnu_getopt}{args, options\optional{, long_options}}
This function works like \function{getopt()}, except that GNU style
scanning mode is used by default. This means that option and
non-option arguments may be intermixed. The \function{getopt()}
function stops processing options as soon as a non-option argument is
encountered.

If the first character of the option string is `+', or if the
environment variable POSIXLY_CORRECT is set, then option processing
stops as soon as a non-option argument is encountered.

\versionadded{2.3}
\end{funcdesc}

\begin{excdesc}{GetoptError}
This is raised when an unrecognized option is found in the argument
list or when an option requiring an argument is given none.
The argument to the exception is a string indicating the cause of the
error.  For long options, an argument given to an option which does
not require one will also cause this exception to be raised.  The
attributes \member{msg} and \member{opt} give the error message and
related option; if there is no specific option to which the exception
relates, \member{opt} is an empty string.

\versionchanged[Introduced \exception{GetoptError} as a synonym for
                \exception{error}]{1.6}
\end{excdesc}

\begin{excdesc}{error}
Alias for \exception{GetoptError}; for backward compatibility.
\end{excdesc}


An example using only \UNIX{} style options:

\begin{verbatim}
>>> import getopt
>>> args = '-a -b -cfoo -d bar a1 a2'.split()
>>> args
['-a', '-b', '-cfoo', '-d', 'bar', 'a1', 'a2']
>>> optlist, args = getopt.getopt(args, 'abc:d:')
>>> optlist
[('-a', ''), ('-b', ''), ('-c', 'foo'), ('-d', 'bar')]
>>> args
['a1', 'a2']
\end{verbatim}

Using long option names is equally easy:

\begin{verbatim}
>>> s = '--condition=foo --testing --output-file abc.def -x a1 a2'
>>> args = s.split()
>>> args
['--condition=foo', '--testing', '--output-file', 'abc.def', '-x', 'a1', 'a2']
>>> optlist, args = getopt.getopt(args, 'x', [
...     'condition=', 'output-file=', 'testing'])
>>> optlist
[('--condition', 'foo'), ('--testing', ''), ('--output-file', 'abc.def'), ('-x',
 '')]
>>> args
['a1', 'a2']
\end{verbatim}

In a script, typical usage is something like this:

\begin{verbatim}
import getopt, sys

def main():
    try:
        opts, args = getopt.getopt(sys.argv[1:], "ho:v", ["help", "output="])
    except getopt.GetoptError:
        # print help information and exit:
        usage()
        sys.exit(2)
    output = None
    verbose = False
    for o, a in opts:
        if o == "-v":
            verbose = True
        if o in ("-h", "--help"):
            usage()
            sys.exit()
        if o in ("-o", "--output"):
            output = a
    # ...

if __name__ == "__main__":
    main()
\end{verbatim}

\begin{seealso}
  \seemodule{optparse}{More object-oriented command line option parsing.}
\end{seealso}

\section{\module{tempfile} ---
         Generate temporary file names.}
\declaremodule{standard}{tempfile}

\modulesynopsis{Generate temporary file names.}

\indexii{temporary}{file name}
\indexii{temporary}{file}


This module generates temporary file names.  It is not \UNIX{} specific,
but it may require some help on non-\UNIX{} systems.

Note: the modules does not create temporary files, nor does it
automatically remove them when the current process exits or dies.

The module defines a single user-callable function:

\begin{funcdesc}{mktemp}{}
Return a unique temporary filename.  This is an absolute pathname of a
file that does not exist at the time the call is made.  No two calls
will return the same filename.
\end{funcdesc}

The module uses two global variables that tell it how to construct a
temporary name.  The caller may assign values to them; by default they
are initialized at the first call to \function{mktemp()}.

\begin{datadesc}{tempdir}
When set to a value other than \code{None}, this variable defines the
directory in which filenames returned by \function{mktemp()} reside.
The default is taken from the environment variable \envvar{TMPDIR}; if
this is not set, either \file{/usr/tmp} is used (on \UNIX{}), or the
current working directory (all other systems).  No check is made to
see whether its value is valid.
\end{datadesc}

\begin{datadesc}{template}
When set to a value other than \code{None}, this variable defines the
prefix of the final component of the filenames returned by
\function{mktemp()}.  A string of decimal digits is added to generate
unique filenames.  The default is either \file{@\var{pid}.} where
\var{pid} is the current process ID (on \UNIX{}), or \file{tmp} (all
other systems).
\end{datadesc}

\strong{Warning:} if a \UNIX{} process uses \code{mktemp()}, then
calls \function{fork()} and both parent and child continue to use
\function{mktemp()}, the processes will generate conflicting temporary
names.  To resolve this, the child process should assign \code{None}
to \code{template}, to force recomputing the default on the next call
to \function{mktemp()}.

\section{Standard Module \sectcode{errno}}
\stmodindex{errno}

\renewcommand{\indexsubitem}{(in module errno)}

This module makes available standard errno system symbols.
The value of each symbol is the corresponding integer value.
The names and descriptions are borrowed from linux/include/errno.h,
which should be pretty all-inclusive.  Of the following list, symbols
that are not used on the current platform are not defined by the
module.

Symbols available can include:
\begin{datadesc}{EPERM} Operation not permitted \end{datadesc}
\begin{datadesc}{ENOENT} No such file or directory \end{datadesc}
\begin{datadesc}{ESRCH} No such process \end{datadesc}
\begin{datadesc}{EINTR} Interrupted system call \end{datadesc}
\begin{datadesc}{EIO} I/O error \end{datadesc}
\begin{datadesc}{ENXIO} No such device or address \end{datadesc}
\begin{datadesc}{E2BIG} Arg list too long \end{datadesc}
\begin{datadesc}{ENOEXEC} Exec format error \end{datadesc}
\begin{datadesc}{EBADF} Bad file number \end{datadesc}
\begin{datadesc}{ECHILD} No child processes \end{datadesc}
\begin{datadesc}{EAGAIN} Try again \end{datadesc}
\begin{datadesc}{ENOMEM} Out of memory \end{datadesc}
\begin{datadesc}{EACCES} Permission denied \end{datadesc}
\begin{datadesc}{EFAULT} Bad address \end{datadesc}
\begin{datadesc}{ENOTBLK} Block device required \end{datadesc}
\begin{datadesc}{EBUSY} Device or resource busy \end{datadesc}
\begin{datadesc}{EEXIST} File exists \end{datadesc}
\begin{datadesc}{EXDEV} Cross-device link \end{datadesc}
\begin{datadesc}{ENODEV} No such device \end{datadesc}
\begin{datadesc}{ENOTDIR} Not a directory \end{datadesc}
\begin{datadesc}{EISDIR} Is a directory \end{datadesc}
\begin{datadesc}{EINVAL} Invalid argument \end{datadesc}
\begin{datadesc}{ENFILE} File table overflow \end{datadesc}
\begin{datadesc}{EMFILE} Too many open files \end{datadesc}
\begin{datadesc}{ENOTTY} Not a typewriter \end{datadesc}
\begin{datadesc}{ETXTBSY} Text file busy \end{datadesc}
\begin{datadesc}{EFBIG} File too large \end{datadesc}
\begin{datadesc}{ENOSPC} No space left on device \end{datadesc}
\begin{datadesc}{ESPIPE} Illegal seek \end{datadesc}
\begin{datadesc}{EROFS} Read-only file system \end{datadesc}
\begin{datadesc}{EMLINK} Too many links \end{datadesc}
\begin{datadesc}{EPIPE} Broken pipe \end{datadesc}
\begin{datadesc}{EDOM} Math argument out of domain of func \end{datadesc}
\begin{datadesc}{ERANGE} Math result not representable \end{datadesc}
\begin{datadesc}{EDEADLK} Resource deadlock would occur \end{datadesc}
\begin{datadesc}{ENAMETOOLONG} File name too long \end{datadesc}
\begin{datadesc}{ENOLCK} No record locks available \end{datadesc}
\begin{datadesc}{ENOSYS} Function not implemented \end{datadesc}
\begin{datadesc}{ENOTEMPTY} Directory not empty \end{datadesc}
\begin{datadesc}{ELOOP} Too many symbolic links encountered \end{datadesc}
\begin{datadesc}{EWOULDBLOCK} Operation would block \end{datadesc}
\begin{datadesc}{ENOMSG} No message of desired type \end{datadesc}
\begin{datadesc}{EIDRM} Identifier removed \end{datadesc}
\begin{datadesc}{ECHRNG} Channel number out of range \end{datadesc}
\begin{datadesc}{EL2NSYNC} Level 2 not synchronized \end{datadesc}
\begin{datadesc}{EL3HLT} Level 3 halted \end{datadesc}
\begin{datadesc}{EL3RST} Level 3 reset \end{datadesc}
\begin{datadesc}{ELNRNG} Link number out of range \end{datadesc}
\begin{datadesc}{EUNATCH} Protocol driver not attached \end{datadesc}
\begin{datadesc}{ENOCSI} No CSI structure available \end{datadesc}
\begin{datadesc}{EL2HLT} Level 2 halted \end{datadesc}
\begin{datadesc}{EBADE} Invalid exchange \end{datadesc}
\begin{datadesc}{EBADR} Invalid request descriptor \end{datadesc}
\begin{datadesc}{EXFULL} Exchange full \end{datadesc}
\begin{datadesc}{ENOANO} No anode \end{datadesc}
\begin{datadesc}{EBADRQC} Invalid request code \end{datadesc}
\begin{datadesc}{EBADSLT} Invalid slot \end{datadesc}
\begin{datadesc}{EDEADLOCK} File locking deadlock error \end{datadesc}
\begin{datadesc}{EBFONT} Bad font file format \end{datadesc}
\begin{datadesc}{ENOSTR} Device not a stream \end{datadesc}
\begin{datadesc}{ENODATA} No data available \end{datadesc}
\begin{datadesc}{ETIME} Timer expired \end{datadesc}
\begin{datadesc}{ENOSR} Out of streams resources \end{datadesc}
\begin{datadesc}{ENONET} Machine is not on the network \end{datadesc}
\begin{datadesc}{ENOPKG} Package not installed \end{datadesc}
\begin{datadesc}{EREMOTE} Object is remote \end{datadesc}
\begin{datadesc}{ENOLINK} Link has been severed \end{datadesc}
\begin{datadesc}{EADV} Advertise error \end{datadesc}
\begin{datadesc}{ESRMNT} Srmount error \end{datadesc}
\begin{datadesc}{ECOMM} Communication error on send \end{datadesc}
\begin{datadesc}{EPROTO} Protocol error \end{datadesc}
\begin{datadesc}{EMULTIHOP} Multihop attempted \end{datadesc}
\begin{datadesc}{EDOTDOT} RFS specific error \end{datadesc}
\begin{datadesc}{EBADMSG} Not a data message \end{datadesc}
\begin{datadesc}{EOVERFLOW} Value too large for defined data type \end{datadesc}
\begin{datadesc}{ENOTUNIQ} Name not unique on network \end{datadesc}
\begin{datadesc}{EBADFD} File descriptor in bad state \end{datadesc}
\begin{datadesc}{EREMCHG} Remote address changed \end{datadesc}
\begin{datadesc}{ELIBACC} Can not access a needed shared library \end{datadesc}
\begin{datadesc}{ELIBBAD} Accessing a corrupted shared library \end{datadesc}
\begin{datadesc}{ELIBSCN} .lib section in a.out corrupted \end{datadesc}
\begin{datadesc}{ELIBMAX} Attempting to link in too many shared libraries \end{datadesc}
\begin{datadesc}{ELIBEXEC} Cannot exec a shared library directly \end{datadesc}
\begin{datadesc}{EILSEQ} Illegal byte sequence \end{datadesc}
\begin{datadesc}{ERESTART} Interrupted system call should be restarted \end{datadesc}
\begin{datadesc}{ESTRPIPE} Streams pipe error \end{datadesc}
\begin{datadesc}{EUSERS} Too many users \end{datadesc}
\begin{datadesc}{ENOTSOCK} Socket operation on non-socket \end{datadesc}
\begin{datadesc}{EDESTADDRREQ} Destination address required \end{datadesc}
\begin{datadesc}{EMSGSIZE} Message too long \end{datadesc}
\begin{datadesc}{EPROTOTYPE} Protocol wrong type for socket \end{datadesc}
\begin{datadesc}{ENOPROTOOPT} Protocol not available \end{datadesc}
\begin{datadesc}{EPROTONOSUPPORT} Protocol not supported \end{datadesc}
\begin{datadesc}{ESOCKTNOSUPPORT} Socket type not supported \end{datadesc}
\begin{datadesc}{EOPNOTSUPP} Operation not supported on transport endpoint \end{datadesc}
\begin{datadesc}{EPFNOSUPPORT} Protocol family not supported \end{datadesc}
\begin{datadesc}{EAFNOSUPPORT} Address family not supported by protocol \end{datadesc}
\begin{datadesc}{EADDRINUSE} Address already in use \end{datadesc}
\begin{datadesc}{EADDRNOTAVAIL} Cannot assign requested address \end{datadesc}
\begin{datadesc}{ENETDOWN} Network is down \end{datadesc}
\begin{datadesc}{ENETUNREACH} Network is unreachable \end{datadesc}
\begin{datadesc}{ENETRESET} Network dropped connection because of reset \end{datadesc}
\begin{datadesc}{ECONNABORTED} Software caused connection abort \end{datadesc}
\begin{datadesc}{ECONNRESET} Connection reset by peer \end{datadesc}
\begin{datadesc}{ENOBUFS} No buffer space available \end{datadesc}
\begin{datadesc}{EISCONN} Transport endpoint is already connected \end{datadesc}
\begin{datadesc}{ENOTCONN} Transport endpoint is not connected \end{datadesc}
\begin{datadesc}{ESHUTDOWN} Cannot send after transport endpoint shutdown \end{datadesc}
\begin{datadesc}{ETOOMANYREFS} Too many references: cannot splice \end{datadesc}
\begin{datadesc}{ETIMEDOUT} Connection timed out \end{datadesc}
\begin{datadesc}{ECONNREFUSED} Connection refused \end{datadesc}
\begin{datadesc}{EHOSTDOWN} Host is down \end{datadesc}
\begin{datadesc}{EHOSTUNREACH} No route to host \end{datadesc}
\begin{datadesc}{EALREADY} Operation already in progress \end{datadesc}
\begin{datadesc}{EINPROGRESS} Operation now in progress \end{datadesc}
\begin{datadesc}{ESTALE} Stale NFS file handle \end{datadesc}
\begin{datadesc}{EUCLEAN} Structure needs cleaning \end{datadesc}
\begin{datadesc}{ENOTNAM} Not a XENIX named type file \end{datadesc}
\begin{datadesc}{ENAVAIL} No XENIX semaphores available \end{datadesc}
\begin{datadesc}{EISNAM} Is a named type file \end{datadesc}
\begin{datadesc}{EREMOTEIO} Remote I/O error \end{datadesc}
\begin{datadesc}{EDQUOT} Quota exceeded \end{datadesc}


\section{Standard Module \sectcode{glob}}
\stmodindex{glob}
\renewcommand{\indexsubitem}{(in module glob)}

The \code{glob} module finds all the pathnames matching a specified
pattern according to the rules used by the \UNIX{} shell.  No tilde
expansion is done, but \verb\*\, \verb\?\, and character ranges
expressed with \verb\[]\ will be correctly matched.  This is done by
using the \code{os.listdir()} and \code{fnmatch.fnmatch()} functions
in concert, and not by actually invoking a subshell.  (For tilde and
shell variable expansion, use \code{os.path.expanduser(}) and
\code{os.path.expandvars()}.)

\begin{funcdesc}{glob}{pathname}
Returns a possibly-empty list of path names that match \var{pathname},
which must be a string containing a path specification.
\var{pathname} can be either absolute (like
\file{/usr/src/Python1.4/Makefile}) or relative (like
\file{../../Tools/*.gif}), and can contain shell-style wildcards.
\end{funcdesc}

For example, consider a directory containing only the following files:
\file{1.gif}, \file{2.txt}, and \file{card.gif}.  \code{glob.glob()}
will produce the following results.  Notice how any leading components
of the path are preserved.

\begin{verbatim}
>>> import glob
>>> glob.glob('./[0-9].*')
['./1.gif', './2.txt']
>>> glob.glob('*.gif')
['1.gif', 'card.gif']
>>> glob.glob('?.gif')
['1.gif']
\end{verbatim}

\section{Standard Module \sectcode{fnmatch}}
\label{module-fnmatch}
\stmodindex{fnmatch}

This module provides support for Unix shell-style wildcards, which are
\emph{not} the same as Python's regular expressions (which are
documented in the \code{re} module).  The special characters used
in shell-style wildcards are:
\begin{itemize}
\item[\code{*}] matches everything
\item[\code{?}]	matches any single character
\item[\code{[}\var{seq}\code{]}] matches any character in \var{seq}
\item[\code{[!}\var{seq}\code{]}] matches any character not in \var{seq}
\end{itemize}

Note that the filename separator (\code{'/'} on Unix) is \emph{not}
special to this module.  See module \code{glob} for pathname expansion
(\code{glob} uses \code{fnmatch} to match filename segments).

\renewcommand{\indexsubitem}{(in module fnmatch)}

\begin{funcdesc}{fnmatch}{filename\, pattern}
Test whether the \var{filename} string matches the \var{pattern}
string, returning true or false.  If the operating system is
case-insensitive, then both parameters will be normalized to all
lower- or upper-case before the comparision is performed.  If you
require a case-sensitive comparision regardless of whether that's
standard for your operating system, use \code{fnmatchcase()} instead.
\end{funcdesc}

\begin{funcdesc}{fnmatchcase}{}
Test whether \var{filename} matches \var{pattern}, returning true or
false; the comparision is case-sensitive.
\end{funcdesc}

\begin{funcdesc}{translate}{pattern}
Translate a shell pattern into a corresponding regular expression,
returning a string describing the pattern.  It does not compile the
expression.  
\end{funcdesc}


\section{\module{shutil} ---
         High-level file operations}

\declaremodule{standard}{shutil}
\modulesynopsis{High-level file operations, including copying.}
\sectionauthor{Fred L. Drake, Jr.}{fdrake@acm.org}
% partly based on the docstrings


The \module{shutil} module offers a number of high-level operations on
files and collections of files.  In particular, functions are provided 
which support file copying and removal.
\index{file!copying}
\index{copying files}

\strong{Caveat:}  On MacOS, the resource fork and other metadata are
not used.  For file copies, this means that resources will be lost and 
file type and creator codes will not be correct.


\begin{funcdesc}{copyfile}{src, dst}
  Copy the contents of \var{src} to \var{dst}.  If \var{dst} exists,
  it will be replaced, otherwise it will be created.
\end{funcdesc}

\begin{funcdesc}{copyfileobj}{fsrc, fdst\optional{, length}}
  Copy the contents of the file-like object \var{fsrc} to the
  file-like object \var{fdst}.  The integer \var{length}, if given,
  is the buffer size. In particular, a negative \var{length} value
  means to copy the data without looping over the source data in
  chunks; by default the data is read in chunks to avoid uncontrolled
  memory consumption.
\end{funcdesc}

\begin{funcdesc}{copymode}{src, dst}
  Copy the permission bits from \var{src} to \var{dst}.  The file
  contents, owner, and group are unaffected.
\end{funcdesc}

\begin{funcdesc}{copystat}{src, dst}
  Copy the permission bits, last access time, and last modification
  time from \var{src} to \var{dst}.  The file contents, owner, and
  group are unaffected.
\end{funcdesc}

\begin{funcdesc}{copy}{src, dst}
  Copy the file \var{src} to the file or directory \var{dst}.  If
  \var{dst} is a directory, a file with the same basename as \var{src} 
  is created (or overwritten) in the directory specified.  Permission
  bits are copied.
\end{funcdesc}

\begin{funcdesc}{copy2}{src, dst}
  Similar to \function{copy()}, but last access time and last
  modification time are copied as well.  This is similar to the
  \UNIX{} command \program{cp} \programopt{-p}.
\end{funcdesc}

\begin{funcdesc}{copytree}{src, dst\optional{, symlinks}}
  Recursively copy an entire directory tree rooted at \var{src}.  The
  destination directory, named by \var{dst}, must not already exist;
  it will be created.  Individual files are copied using
  \function{copy2()}.  If \var{symlinks} is true, symbolic links in
  the source tree are represented as symbolic links in the new tree;
  if false or omitted, the contents of the linked files are copied to
  the new tree.  Errors are reported to standard output.

  The source code for this should be considered an example rather than 
  a tool.
\end{funcdesc}

\begin{funcdesc}{rmtree}{path\optional{, ignore_errors\optional{, onerror}}}
\index{directory!deleting}
  Delete an entire directory tree.  If \var{ignore_errors} is true,
  errors will be ignored; if false or omitted, errors are handled by
  calling a handler specified by \var{onerror} or raise an exception.

  If \var{onerror} is provided, it must be a callable that accepts
  three parameters: \var{function}, \var{path}, and \var{excinfo}.
  The first parameter, \var{function}, is the function which raised
  the exception; it will be \function{os.remove()} or
  \function{os.rmdir()}.  The second parameter, \var{path}, will be
  the path name passed to \var{function}.  The third parameter,
  \var{excinfo}, will be the exception information return by
  \function{sys.exc_info()}.  Exceptions raised by \var{onerror} will
  not be caught.
\end{funcdesc}


\subsection{Example \label{shutil-example}}

This example is the implementation of the \function{copytree()}
function, described above, with the docstring omitted.  It
demonstrates many of the other functions provided by this module.

\begin{verbatim}
def copytree(src, dst, symlinks=0):
    names = os.listdir(src)
    os.mkdir(dst)
    for name in names:
        srcname = os.path.join(src, name)
        dstname = os.path.join(dst, name)
        try:
            if symlinks and os.path.islink(srcname):
                linkto = os.readlink(srcname)
                os.symlink(linkto, dstname)
            elif os.path.isdir(srcname):
                copytree(srcname, dstname)
            else:
                copy2(srcname, dstname)
            # XXX What about devices, sockets etc.?
        except (IOError, os.error), why:
            print "Can't copy %s to %s: %s" % (`srcname`, `dstname`, str(why))
\end{verbatim}

\section{\module{locale} ---
         Internationalization services}

\declaremodule{standard}{locale}
\modulesynopsis{Internationalization services.}
\moduleauthor{Martin von L\"owis}{martin@v.loewis.de}
\sectionauthor{Martin von L\"owis}{martin@v.loewis.de}


The \module{locale} module opens access to the \POSIX{} locale
database and functionality. The \POSIX{} locale mechanism allows
programmers to deal with certain cultural issues in an application,
without requiring the programmer to know all the specifics of each
country where the software is executed.

The \module{locale} module is implemented on top of the
\module{_locale}\refbimodindex{_locale} module, which in turn uses an
ANSI C locale implementation if available.

The \module{locale} module defines the following exception and
functions:


\begin{excdesc}{Error}
  Exception raised when \function{setlocale()} fails.
\end{excdesc}

\begin{funcdesc}{setlocale}{category\optional{, locale}}
  If \var{locale} is specified, it may be a string, a tuple of the
  form \code{(\var{language code}, \var{encoding})}, or \code{None}.
  If it is a tuple, it is converted to a string using the locale
  aliasing engine.  If \var{locale} is given and not \code{None},
  \function{setlocale()} modifies the locale setting for the
  \var{category}.  The available categories are listed in the data
  description below.  The value is the name of a locale.  An empty
  string specifies the user's default settings. If the modification of
  the locale fails, the exception \exception{Error} is raised.  If
  successful, the new locale setting is returned.

  If \var{locale} is omitted or \code{None}, the current setting for
  \var{category} is returned.

  \function{setlocale()} is not thread safe on most systems.
  Applications typically start with a call of

\begin{verbatim}
import locale
locale.setlocale(locale.LC_ALL, '')
\end{verbatim}

  This sets the locale for all categories to the user's default
  setting (typically specified in the \envvar{LANG} environment
  variable).  If the locale is not changed thereafter, using
  multithreading should not cause problems.

  \versionchanged[Added support for tuple values of the \var{locale}
                  parameter]{2.0}
\end{funcdesc}

\begin{funcdesc}{localeconv}{}
  Returns the database of the local conventions as a dictionary.
  This dictionary has the following strings as keys:

  \begin{tableiii}{l|l|p{3in}}{constant}{Key}{Category}{Meaning}
    \lineiii{LC_NUMERIC}{\code{'decimal_point'}}
            {Decimal point character.}
    \lineiii{}{\code{'grouping'}}
            {Sequence of numbers specifying which relative positions
             the \code{'thousands_sep'} is expected.  If the sequence is
             terminated with \constant{CHAR_MAX}, no further grouping
             is performed. If the sequence terminates with a \code{0}, 
             the last group size is repeatedly used.}
    \lineiii{}{\code{'thousands_sep'}}
            {Character used between groups.}\hline
    \lineiii{LC_MONETARY}{\code{'int_curr_symbol'}}
            {International currency symbol.}
    \lineiii{}{\code{'currency_symbol'}}
            {Local currency symbol.}
    \lineiii{}{\code{'mon_decimal_point'}}
            {Decimal point used for monetary values.}
    \lineiii{}{\code{'mon_thousands_sep'}}
            {Group separator used for monetary values.}
    \lineiii{}{\code{'mon_grouping'}}
            {Equivalent to \code{'grouping'}, used for monetary
             values.}
    \lineiii{}{\code{'positive_sign'}}
            {Symbol used to annotate a positive monetary value.}
    \lineiii{}{\code{'negative_sign'}}
            {Symbol used to annotate a nnegative monetary value.}
    \lineiii{}{\code{'frac_digits'}}
            {Number of fractional digits used in local formatting
             of monetary values.}
    \lineiii{}{\code{'int_frac_digits'}}
            {Number of fractional digits used in international
             formatting of monetary values.}
  \end{tableiii}

  The possible values for \code{'p_sign_posn'} and
  \code{'n_sign_posn'} are given below.

  \begin{tableii}{c|l}{code}{Value}{Explanation}
    \lineii{0}{Currency and value are surrounded by parentheses.}
    \lineii{1}{The sign should precede the value and currency symbol.}
    \lineii{2}{The sign should follow the value and currency symbol.}
    \lineii{3}{The sign should immediately precede the value.}
    \lineii{4}{The sign should immediately follow the value.}
    \lineii{\constant{LC_MAX}}{Nothing is specified in this locale.}
  \end{tableii}
\end{funcdesc}

\begin{funcdesc}{nl_langinfo}{option}

Return some locale-specific information as a string. This function is
not available on all systems, and the set of possible options might
also vary across platforms. The possible argument values are numbers,
for which symbolic constants are available in the locale module.

\end{funcdesc}

\begin{funcdesc}{getdefaultlocale}{\optional{envvars}}
  Tries to determine the default locale settings and returns
  them as a tuple of the form \code{(\var{language code},
  \var{encoding})}.

  According to \POSIX, a program which has not called
  \code{setlocale(LC_ALL, '')} runs using the portable \code{'C'}
  locale.  Calling \code{setlocale(LC_ALL, '')} lets it use the
  default locale as defined by the \envvar{LANG} variable.  Since we
  do not want to interfere with the current locale setting we thus
  emulate the behavior in the way described above.

  To maintain compatibility with other platforms, not only the
  \envvar{LANG} variable is tested, but a list of variables given as
  envvars parameter.  The first found to be defined will be
  used.  \var{envvars} defaults to the search path used in GNU gettext;
  it must always contain the variable name \samp{LANG}.  The GNU
  gettext search path contains \code{'LANGUAGE'}, \code{'LC_ALL'},
  \code{'LC_CTYPE'}, and \code{'LANG'}, in that order.

  Except for the code \code{'C'}, the language code corresponds to
  \rfc{1766}.  \var{language code} and \var{encoding} may be
  \code{None} if their values cannot be determined.
  \versionadded{2.0}
\end{funcdesc}

\begin{funcdesc}{getlocale}{\optional{category}}
  Returns the current setting for the given locale category as
  sequence containing \var{language code}, \var{encoding}.
  \var{category} may be one of the \constant{LC_*} values except
  \constant{LC_ALL}.  It defaults to \constant{LC_CTYPE}.

  Except for the code \code{'C'}, the language code corresponds to
  \rfc{1766}.  \var{language code} and \var{encoding} may be
  \code{None} if their values cannot be determined.
  \versionadded{2.0}
\end{funcdesc}

\begin{funcdesc}{getpreferredencoding}{\optional{do_setlocale}}
  Return the encoding used for text data, according to user
  preferences.  User preferences are expressed differently on
  different systems, and might not be available programmatically on
  some systems, so this function only returns a guess.

  On some systems, it is necessary to invoke \function{setlocale}
  to obtain the user preferences, so this function is not thread-safe.
  If invoking setlocale is not necessary or desired, \var{do_setlocale}
  should be set to \code{False}.

  \versionadded{2.3}
\end{funcdesc}

\begin{funcdesc}{normalize}{localename}
  Returns a normalized locale code for the given locale name.  The
  returned locale code is formatted for use with
  \function{setlocale()}.  If normalization fails, the original name
  is returned unchanged.

  If the given encoding is not known, the function defaults to
  the default encoding for the locale code just like
  \function{setlocale()}.
  \versionadded{2.0}
\end{funcdesc}

\begin{funcdesc}{resetlocale}{\optional{category}}
  Sets the locale for \var{category} to the default setting.

  The default setting is determined by calling
  \function{getdefaultlocale()}.  \var{category} defaults to
  \constant{LC_ALL}.
  \versionadded{2.0}
\end{funcdesc}

\begin{funcdesc}{strcoll}{string1, string2}
  Compares two strings according to the current
  \constant{LC_COLLATE} setting. As any other compare function,
  returns a negative, or a positive value, or \code{0}, depending on
  whether \var{string1} collates before or after \var{string2} or is
  equal to it.
\end{funcdesc}

\begin{funcdesc}{strxfrm}{string}
  Transforms a string to one that can be used for the built-in
  function \function{cmp()}\bifuncindex{cmp}, and still returns
  locale-aware results.  This function can be used when the same
  string is compared repeatedly, e.g. when collating a sequence of
  strings.
\end{funcdesc}

\begin{funcdesc}{format}{format, val\optional{, grouping}}
  Formats a number \var{val} according to the current
  \constant{LC_NUMERIC} setting.  The format follows the conventions
  of the \code{\%} operator.  For floating point values, the decimal
  point is modified if appropriate.  If \var{grouping} is true, also
  takes the grouping into account.
\end{funcdesc}

\begin{funcdesc}{str}{float}
  Formats a floating point number using the same format as the
  built-in function \code{str(\var{float})}, but takes the decimal
  point into account.
\end{funcdesc}

\begin{funcdesc}{atof}{string}
  Converts a string to a floating point number, following the
  \constant{LC_NUMERIC} settings.
\end{funcdesc}

\begin{funcdesc}{atoi}{string}
  Converts a string to an integer, following the
  \constant{LC_NUMERIC} conventions.
\end{funcdesc}

\begin{datadesc}{LC_CTYPE}
\refstmodindex{string}
  Locale category for the character type functions.  Depending on the
  settings of this category, the functions of module
  \refmodule{string} dealing with case change their behaviour.
\end{datadesc}

\begin{datadesc}{LC_COLLATE}
  Locale category for sorting strings.  The functions
  \function{strcoll()} and \function{strxfrm()} of the
  \module{locale} module are affected.
\end{datadesc}

\begin{datadesc}{LC_TIME}
  Locale category for the formatting of time.  The function
  \function{time.strftime()} follows these conventions.
\end{datadesc}

\begin{datadesc}{LC_MONETARY}
  Locale category for formatting of monetary values.  The available
  options are available from the \function{localeconv()} function.
\end{datadesc}

\begin{datadesc}{LC_MESSAGES}
  Locale category for message display. Python currently does not
  support application specific locale-aware messages.  Messages
  displayed by the operating system, like those returned by
  \function{os.strerror()} might be affected by this category.
\end{datadesc}

\begin{datadesc}{LC_NUMERIC}
  Locale category for formatting numbers.  The functions
  \function{format()}, \function{atoi()}, \function{atof()} and
  \function{str()} of the \module{locale} module are affected by that
  category.  All other numeric formatting operations are not
  affected.
\end{datadesc}

\begin{datadesc}{LC_ALL}
  Combination of all locale settings.  If this flag is used when the
  locale is changed, setting the locale for all categories is
  attempted. If that fails for any category, no category is changed at
  all.  When the locale is retrieved using this flag, a string
  indicating the setting for all categories is returned. This string
  can be later used to restore the settings.
\end{datadesc}

\begin{datadesc}{CHAR_MAX}
  This is a symbolic constant used for different values returned by
  \function{localeconv()}.
\end{datadesc}

The \function{nl_langinfo} function accepts one of the following keys.
Most descriptions are taken from the corresponding description in the
GNU C library.

\begin{datadesc}{CODESET}
Return a string with the name of the character encoding used in the
selected locale.
\end{datadesc}

\begin{datadesc}{D_T_FMT}
Return a string that can be used as a format string for strftime(3) to
represent time and date in a locale-specific way.
\end{datadesc}

\begin{datadesc}{D_FMT}
Return a string that can be used as a format string for strftime(3) to
represent a date in a locale-specific way.
\end{datadesc}

\begin{datadesc}{T_FMT}
Return a string that can be used as a format string for strftime(3) to
represent a time in a locale-specific way.
\end{datadesc}

\begin{datadesc}{T_FMT_AMPM}
The return value can be used as a format string for `strftime' to
represent time in the am/pm format.
\end{datadesc}

\begin{datadesc}{DAY_1 ... DAY_7}
Return name of the n-th day of the week. \warning{This
follows the US convention of \constant{DAY_1} being Sunday, not the
international convention (ISO 8601) that Monday is the first day of
the week.}
\end{datadesc}

\begin{datadesc}{ABDAY_1 ... ABDAY_7}
Return abbreviated name of the n-th day of the week.
\end{datadesc}

\begin{datadesc}{MON_1 ... MON_12}
Return name of the n-th month.
\end{datadesc}

\begin{datadesc}{ABMON_1 ... ABMON_12}
Return abbreviated name of the n-th month.
\end{datadesc}

\begin{datadesc}{RADIXCHAR}
Return radix character (decimal dot, decimal comma, etc.)
\end{datadesc}

\begin{datadesc}{THOUSEP}
Return separator character for thousands (groups of three digits).
\end{datadesc}

\begin{datadesc}{YESEXPR}
Return a regular expression that can be used with the regex
function to recognize a positive response to a yes/no question.
\warning{The expression is in the syntax suitable for the
\cfunction{regex()} function from the C library, which might differ
from the syntax used in \refmodule{re}.}
\end{datadesc}

\begin{datadesc}{NOEXPR}
Return a regular expression that can be used with the regex(3)
function to recognize a negative response to a yes/no question.
\end{datadesc}

\begin{datadesc}{CRNCYSTR}
Return the currency symbol, preceded by "-" if the symbol should
appear before the value, "+" if the symbol should appear after the
value, or "." if the symbol should replace the radix character.
\end{datadesc}

\begin{datadesc}{ERA}
The return value represents the era used in the current locale.

Most locales do not define this value.  An example of a locale which
does define this value is the Japanese one.  In Japan, the traditional
representation of dates includes the name of the era corresponding to
the then-emperor's reign.

Normally it should not be necessary to use this value directly.
Specifying the \code{E} modifier in their format strings causes the
\function{strftime} function to use this information.  The format of the
returned string is not specified, and therefore you should not assume
knowledge of it on different systems.
\end{datadesc}

\begin{datadesc}{ERA_YEAR}
The return value gives the year in the relevant era of the locale.
\end{datadesc}

\begin{datadesc}{ERA_D_T_FMT}
This return value can be used as a format string for
\function{strftime} to represent dates and times in a locale-specific
era-based way.
\end{datadesc}

\begin{datadesc}{ERA_D_FMT}
This return value can be used as a format string for
\function{strftime} to represent time in a locale-specific era-based
way.
\end{datadesc}

\begin{datadesc}{ALT_DIGITS}
The return value is a representation of up to 100 values used to
represent the values 0 to 99.
\end{datadesc}

Example:

\begin{verbatim}
>>> import locale
>>> loc = locale.setlocale(locale.LC_ALL) # get current locale
>>> locale.setlocale(locale.LC_ALL, 'de_DE') # use German locale; name might vary with platform
>>> locale.strcoll('f\xe4n', 'foo') # compare a string containing an umlaut 
>>> locale.setlocale(locale.LC_ALL, '') # use user's preferred locale
>>> locale.setlocale(locale.LC_ALL, 'C') # use default (C) locale
>>> locale.setlocale(locale.LC_ALL, loc) # restore saved locale
\end{verbatim}


\subsection{Background, details, hints, tips and caveats}

The C standard defines the locale as a program-wide property that may
be relatively expensive to change.  On top of that, some
implementation are broken in such a way that frequent locale changes
may cause core dumps.  This makes the locale somewhat painful to use
correctly.

Initially, when a program is started, the locale is the \samp{C} locale, no
matter what the user's preferred locale is.  The program must
explicitly say that it wants the user's preferred locale settings by
calling \code{setlocale(LC_ALL, '')}.

It is generally a bad idea to call \function{setlocale()} in some library
routine, since as a side effect it affects the entire program.  Saving
and restoring it is almost as bad: it is expensive and affects other
threads that happen to run before the settings have been restored.

If, when coding a module for general use, you need a locale
independent version of an operation that is affected by the locale
(such as \function{string.lower()}, or certain formats used with
\function{time.strftime()}), you will have to find a way to do it
without using the standard library routine.  Even better is convincing
yourself that using locale settings is okay.  Only as a last resort
should you document that your module is not compatible with
non-\samp{C} locale settings.

The case conversion functions in the
\refmodule{string}\refstmodindex{string} module are affected by the
locale settings.  When a call to the \function{setlocale()} function
changes the \constant{LC_CTYPE} settings, the variables
\code{string.lowercase}, \code{string.uppercase} and
\code{string.letters} are recalculated.  Note that this code that uses
these variable through `\keyword{from} ... \keyword{import} ...',
e.g.\ \code{from string import letters}, is not affected by subsequent
\function{setlocale()} calls.

The only way to perform numeric operations according to the locale
is to use the special functions defined by this module:
\function{atof()}, \function{atoi()}, \function{format()},
\function{str()}.

\subsection{For extension writers and programs that embed Python
            \label{embedding-locale}}

Extension modules should never call \function{setlocale()}, except to
find out what the current locale is.  But since the return value can
only be used portably to restore it, that is not very useful (except
perhaps to find out whether or not the locale is \samp{C}).

When Python code uses the \module{locale} module to change the locale,
this also affects the embedding application.  If the embedding
application doesn't want this to happen, it should remove the
\module{_locale} extension module (which does all the work) from the
table of built-in modules in the \file{config.c} file, and make sure
that the \module{_locale} module is not accessible as a shared library.


\subsection{Access to message catalogs \label{locale-gettext}}

The locale module exposes the C library's gettext interface on systems
that provide this interface.  It consists of the functions
\function{gettext()}, \function{dgettext()}, \function{dcgettext()},
\function{textdomain()}, \function{bindtextdomain()}, and
\function{bind_textdomain_codeset()}.  These are similar to the same
functions in the \refmodule{gettext} module, but use the C library's
binary format for message catalogs, and the C library's search
algorithms for locating message catalogs. 

Python applications should normally find no need to invoke these
functions, and should use \refmodule{gettext} instead.  A known
exception to this rule are applications that link use additional C
libraries which internally invoke \cfunction{gettext()} or
\function{dcgettext()}.  For these applications, it may be necessary to
bind the text domain, so that the libraries can properly locate their
message catalogs.

\section{\module{mutex} ---
         Mutual exclusion support}

\declaremodule{standard}{mutex}
\sectionauthor{Moshe Zadka}{moshez@zadka.site.co.il}
\modulesynopsis{Lock and queue for mutual exclusion.}

The \module{mutex} module defines a class that allows mutual-exclusion
via acquiring and releasing locks. It does not require (or imply)
threading or multi-tasking, though it could be useful for
those purposes.

The \module{mutex} module defines the following class:

\begin{classdesc}{mutex}{}
Create a new (unlocked) mutex.

A mutex has two pieces of state --- a ``locked'' bit and a queue.
When the mutex is not locked, the queue is empty.
Otherwise, the queue contains zero or more 
\code{(\var{function}, \var{argument})} pairs
representing functions (or methods) waiting to acquire the lock.
When the mutex is unlocked while the queue is not empty,
the first queue entry is removed and its 
\code{\var{function}(\var{argument})} pair called,
implying it now has the lock.

Of course, no multi-threading is implied -- hence the funny interface
for \method{lock()}, where a function is called once the lock is
acquired.
\end{classdesc}


\subsection{Mutex Objects \label{mutex-objects}}

\class{mutex} objects have following methods:

\begin{methoddesc}[mutex]{test}{}
Check whether the mutex is locked.
\end{methoddesc}

\begin{methoddesc}[mutex]{testandset}{}
``Atomic'' test-and-set, grab the lock if it is not set,
and return \code{True}, otherwise, return \code{False}.
\end{methoddesc}

\begin{methoddesc}[mutex]{lock}{function, argument}
Execute \code{\var{function}(\var{argument})}, unless the mutex is locked.
In the case it is locked, place the function and argument on the queue.
See \method{unlock} for explanation of when
\code{\var{function}(\var{argument})} is executed in that case.
\end{methoddesc}

\begin{methoddesc}[mutex]{unlock}{}
Unlock the mutex if queue is empty, otherwise execute the first element
in the queue.
\end{methoddesc}


\chapter{Optional Operating System Services}
\label{someos}

The modules described in this chapter provide interfaces to operating
system features that are available on selected operating systems only.
The interfaces are generally modeled after the \UNIX{} or \C{}
interfaces but they are available on some other systems as well
(e.g. Windows or NT).  Here's an overview:

\localmoduletable
		% Optional Operating System Services
\section{\module{signal} ---
         Set handlers for asynchronous events}

\declaremodule{builtin}{signal}
\modulesynopsis{Set handlers for asynchronous events.}


This module provides mechanisms to use signal handlers in Python.
Some general rules for working with signals and their handlers:

\begin{itemize}

\item
A handler for a particular signal, once set, remains installed until
it is explicitly reset (Python emulates the BSD style interface
regardless of the underlying implementation), with the exception of
the handler for \constant{SIGCHLD}, which follows the underlying
implementation.

\item
There is no way to ``block'' signals temporarily from critical
sections (since this is not supported by all \UNIX{} flavors).

\item
Although Python signal handlers are called asynchronously as far as
the Python user is concerned, they can only occur between the
``atomic'' instructions of the Python interpreter.  This means that
signals arriving during long calculations implemented purely in C
(such as regular expression matches on large bodies of text) may be
delayed for an arbitrary amount of time.

\item
When a signal arrives during an I/O operation, it is possible that the
I/O operation raises an exception after the signal handler returns.
This is dependent on the underlying \UNIX{} system's semantics regarding
interrupted system calls.

\item
Because the \C{} signal handler always returns, it makes little sense to
catch synchronous errors like \constant{SIGFPE} or \constant{SIGSEGV}.

\item
Python installs a small number of signal handlers by default:
\constant{SIGPIPE} is ignored (so write errors on pipes and sockets can be
reported as ordinary Python exceptions) and \constant{SIGINT} is translated
into a \exception{KeyboardInterrupt} exception.  All of these can be
overridden.

\item
Some care must be taken if both signals and threads are used in the
same program.  The fundamental thing to remember in using signals and
threads simultaneously is:\ always perform \function{signal()} operations
in the main thread of execution.  Any thread can perform an
\function{alarm()}, \function{getsignal()}, or \function{pause()};
only the main thread can set a new signal handler, and the main thread
will be the only one to receive signals (this is enforced by the
Python \module{signal} module, even if the underlying thread
implementation supports sending signals to individual threads).  This
means that signals can't be used as a means of inter-thread
communication.  Use locks instead.

\end{itemize}

The variables defined in the \module{signal} module are:

\begin{datadesc}{SIG_DFL}
  This is one of two standard signal handling options; it will simply
  perform the default function for the signal.  For example, on most
  systems the default action for \constant{SIGQUIT} is to dump core
  and exit, while the default action for \constant{SIGCLD} is to
  simply ignore it.
\end{datadesc}

\begin{datadesc}{SIG_IGN}
  This is another standard signal handler, which will simply ignore
  the given signal.
\end{datadesc}

\begin{datadesc}{SIG*}
  All the signal numbers are defined symbolically.  For example, the
  hangup signal is defined as \constant{signal.SIGHUP}; the variable names
  are identical to the names used in C programs, as found in
  \code{<signal.h>}.
  The \UNIX{} man page for `\cfunction{signal()}' lists the existing
  signals (on some systems this is \manpage{signal}{2}, on others the
  list is in \manpage{signal}{7}).
  Note that not all systems define the same set of signal names; only
  those names defined by the system are defined by this module.
\end{datadesc}

\begin{datadesc}{NSIG}
  One more than the number of the highest signal number.
\end{datadesc}

The \module{signal} module defines the following functions:

\begin{funcdesc}{alarm}{time}
  If \var{time} is non-zero, this function requests that a
  \constant{SIGALRM} signal be sent to the process in \var{time} seconds.
  Any previously scheduled alarm is canceled (only one alarm can
  be scheduled at any time).  The returned value is then the number of
  seconds before any previously set alarm was to have been delivered.
  If \var{time} is zero, no alarm id scheduled, and any scheduled
  alarm is canceled.  The return value is the number of seconds
  remaining before a previously scheduled alarm.  If the return value
  is zero, no alarm is currently scheduled.  (See the \UNIX{} man page
  \manpage{alarm}{2}.)
  Availability: \UNIX.
\end{funcdesc}

\begin{funcdesc}{getsignal}{signalnum}
  Return the current signal handler for the signal \var{signalnum}.
  The returned value may be a callable Python object, or one of the
  special values \constant{signal.SIG_IGN}, \constant{signal.SIG_DFL} or
  \constant{None}.  Here, \constant{signal.SIG_IGN} means that the
  signal was previously ignored, \constant{signal.SIG_DFL} means that the
  default way of handling the signal was previously in use, and
  \code{None} means that the previous signal handler was not installed
  from Python.
\end{funcdesc}

\begin{funcdesc}{pause}{}
  Cause the process to sleep until a signal is received; the
  appropriate handler will then be called.  Returns nothing.  Not on
  Windows. (See the \UNIX{} man page \manpage{signal}{2}.)
\end{funcdesc}

\begin{funcdesc}{signal}{signalnum, handler}
  Set the handler for signal \var{signalnum} to the function
  \var{handler}.  \var{handler} can be a callable Python object
  taking two arguments (see below), or
  one of the special values \constant{signal.SIG_IGN} or
  \constant{signal.SIG_DFL}.  The previous signal handler will be returned
  (see the description of \function{getsignal()} above).  (See the
  \UNIX{} man page \manpage{signal}{2}.)

  When threads are enabled, this function can only be called from the
  main thread; attempting to call it from other threads will cause a
  \exception{ValueError} exception to be raised.

  The \var{handler} is called with two arguments: the signal number
  and the current stack frame (\code{None} or a frame object;
  for a description of frame objects, see the reference manual section
  on the standard type hierarchy or see the attribute descriptions in
  the \refmodule{inspect} module).
\end{funcdesc}

\subsection{Example}
\nodename{Signal Example}

Here is a minimal example program. It uses the \function{alarm()}
function to limit the time spent waiting to open a file; this is
useful if the file is for a serial device that may not be turned on,
which would normally cause the \function{os.open()} to hang
indefinitely.  The solution is to set a 5-second alarm before opening
the file; if the operation takes too long, the alarm signal will be
sent, and the handler raises an exception.

\begin{verbatim}
import signal, os

def handler(signum, frame):
    print 'Signal handler called with signal', signum
    raise IOError, "Couldn't open device!"

# Set the signal handler and a 5-second alarm
signal.signal(signal.SIGALRM, handler)
signal.alarm(5)

# This open() may hang indefinitely
fd = os.open('/dev/ttyS0', os.O_RDWR)  

signal.alarm(0)          # Disable the alarm
\end{verbatim}

\section{\module{socket} ---
         Low-level networking interface}

\declaremodule{builtin}{socket}
\modulesynopsis{Low-level networking interface.}


This module provides access to the BSD \emph{socket} interface.
It is available on \UNIX{} systems that support this interface.

For an introduction to socket programming (in C), see the following
papers: \emph{An Introductory 4.3BSD Interprocess Communication
Tutorial}, by Stuart Sechrest and \emph{An Advanced 4.3BSD Interprocess
Communication Tutorial}, by Samuel J.  Leffler et al, both in the
\UNIX{} Programmer's Manual, Supplementary Documents 1 (sections PS1:7
and PS1:8).  The \UNIX{} manual pages for the various socket-related
system calls are also a valuable source of information on the details of
socket semantics.

The Python interface is a straightforward transliteration of the
\UNIX{} system call and library interface for sockets to Python's
object-oriented style: the \function{socket()} function returns a
\dfn{socket object}\obindex{socket} whose methods implement the
various socket system calls.  Parameter types are somewhat
higher-level than in the C interface: as with \method{read()} and
\method{write()} operations on Python files, buffer allocation on
receive operations is automatic, and buffer length is implicit on send
operations.

Socket addresses are represented as a single string for the
\constant{AF_UNIX} address family and as a pair
\code{(\var{host}, \var{port})} for the \constant{AF_INET} address
family, where \var{host} is a string representing
either a hostname in Internet domain notation like
\code{'daring.cwi.nl'} or an IP address like \code{'100.50.200.5'},
and \var{port} is an integral port number.  Other address families are
currently not supported.  The address format required by a particular
socket object is automatically selected based on the address family
specified when the socket object was created.

For IP addresses, two special forms are accepted instead of a host
address: the empty string represents \constant{INADDR_ANY}, and the string
\code{'<broadcast>'} represents \constant{INADDR_BROADCAST}.

All errors raise exceptions.  The normal exceptions for invalid
argument types and out-of-memory conditions can be raised; errors
related to socket or address semantics raise the error
\exception{socket.error}.

Non-blocking mode is supported through the
\method{setblocking()} method.

The module \module{socket} exports the following constants and functions:


\begin{excdesc}{error}
This exception is raised for socket- or address-related errors.
The accompanying value is either a string telling what went wrong or a
pair \code{(\var{errno}, \var{string})}
representing an error returned by a system
call, similar to the value accompanying \exception{os.error}.
See the module \refmodule{errno}\refbimodindex{errno}, which contains
names for the error codes defined by the underlying operating system.
\end{excdesc}

\begin{datadesc}{AF_UNIX}
\dataline{AF_INET}
These constants represent the address (and protocol) families,
used for the first argument to \function{socket()}.  If the
\constant{AF_UNIX} constant is not defined then this protocol is
unsupported.
\end{datadesc}

\begin{datadesc}{SOCK_STREAM}
\dataline{SOCK_DGRAM}
\dataline{SOCK_RAW}
\dataline{SOCK_RDM}
\dataline{SOCK_SEQPACKET}
These constants represent the socket types,
used for the second argument to \function{socket()}.
(Only \constant{SOCK_STREAM} and
\constant{SOCK_DGRAM} appear to be generally useful.)
\end{datadesc}

\begin{datadesc}{SO_*}
\dataline{SOMAXCONN}
\dataline{MSG_*}
\dataline{SOL_*}
\dataline{IPPROTO_*}
\dataline{IPPORT_*}
\dataline{INADDR_*}
\dataline{IP_*}
Many constants of these forms, documented in the \UNIX{} documentation on
sockets and/or the IP protocol, are also defined in the socket module.
They are generally used in arguments to the \method{setsockopt()} and
\method{getsockopt()} methods of socket objects.  In most cases, only
those symbols that are defined in the \UNIX{} header files are defined;
for a few symbols, default values are provided.
\end{datadesc}

\begin{funcdesc}{gethostbyname}{hostname}
Translate a host name to IP address format.  The IP address is
returned as a string, e.g.,  \code{'100.50.200.5'}.  If the host name
is an IP address itself it is returned unchanged.  See
\function{gethostbyname_ex()} for a more complete interface.
\end{funcdesc}

\begin{funcdesc}{gethostbyname_ex}{hostname}
Translate a host name to IP address format, extended interface.
Return a triple \code{(hostname, aliaslist, ipaddrlist)} where
\code{hostname} is the primary host name responding to the given
\var{ip_address}, \code{aliaslist} is a (possibly empty) list of
alternative host names for the same address, and \code{ipaddrlist} is
a list of IP addresses for the same interface on the same
host (often but not always a single address).
\end{funcdesc}

\begin{funcdesc}{gethostname}{}
Return a string containing the hostname of the machine where 
the Python interpreter is currently executing.  If you want to know the
current machine's IP address, use \code{gethostbyname(gethostname())}.
Note: \function{gethostname()} doesn't always return the fully qualified
domain name; use \code{gethostbyaddr(gethostname())}
(see below).
\end{funcdesc}

\begin{funcdesc}{gethostbyaddr}{ip_address}
Return a triple \code{(\var{hostname}, \var{aliaslist},
\var{ipaddrlist})} where \var{hostname} is the primary host name
responding to the given \var{ip_address}, \var{aliaslist} is a
(possibly empty) list of alternative host names for the same address,
and \var{ipaddrlist} is a list of IP addresses for the same interface
on the same host (most likely containing only a single address).
To find the fully qualified domain name, check \var{hostname} and the
items of \var{aliaslist} for an entry containing at least one period.
\end{funcdesc}

\begin{funcdesc}{getprotobyname}{protocolname}
Translate an Internet protocol name (e.g.\ \code{'icmp'}) to a constant
suitable for passing as the (optional) third argument to the
\function{socket()} function.  This is usually only needed for sockets
opened in ``raw'' mode (\constant{SOCK_RAW}); for the normal socket
modes, the correct protocol is chosen automatically if the protocol is
omitted or zero.
\end{funcdesc}

\begin{funcdesc}{getservbyname}{servicename, protocolname}
Translate an Internet service name and protocol name to a port number
for that service.  The protocol name should be \code{'tcp'} or
\code{'udp'}.
\end{funcdesc}

\begin{funcdesc}{socket}{family, type\optional{, proto}}
Create a new socket using the given address family, socket type and
protocol number.  The address family should be \constant{AF_INET} or
\constant{AF_UNIX}.  The socket type should be \constant{SOCK_STREAM},
\constant{SOCK_DGRAM} or perhaps one of the other \samp{SOCK_} constants.
The protocol number is usually zero and may be omitted in that case.
\end{funcdesc}

\begin{funcdesc}{fromfd}{fd, family, type\optional{, proto}}
Build a socket object from an existing file descriptor (an integer as
returned by a file object's \method{fileno()} method).  Address family,
socket type and protocol number are as for the \function{socket()} function
above.  The file descriptor should refer to a socket, but this is not
checked --- subsequent operations on the object may fail if the file
descriptor is invalid.  This function is rarely needed, but can be
used to get or set socket options on a socket passed to a program as
standard input or output (e.g.\ a server started by the \UNIX{} inet
daemon).
\end{funcdesc}

\begin{funcdesc}{ntohl}{x}
Convert 32-bit integers from network to host byte order.  On machines
where the host byte order is the same as network byte order, this is a
no-op; otherwise, it performs a 4-byte swap operation.
\end{funcdesc}

\begin{funcdesc}{ntohs}{x}
Convert 16-bit integers from network to host byte order.  On machines
where the host byte order is the same as network byte order, this is a
no-op; otherwise, it performs a 2-byte swap operation.
\end{funcdesc}

\begin{funcdesc}{htonl}{x}
Convert 32-bit integers from host to network byte order.  On machines
where the host byte order is the same as network byte order, this is a
no-op; otherwise, it performs a 4-byte swap operation.
\end{funcdesc}

\begin{funcdesc}{htons}{x}
Convert 16-bit integers from host to network byte order.  On machines
where the host byte order is the same as network byte order, this is a
no-op; otherwise, it performs a 2-byte swap operation.
\end{funcdesc}

\begin{datadesc}{SocketType}
This is a Python type object that represents the socket object type.
It is the same as \code{type(socket(...))}.
\end{datadesc}

\subsection{Socket Objects}

Socket objects have the following methods.  Except for
\method{makefile()} these correspond to \UNIX{} system calls
applicable to sockets.

\begin{methoddesc}[socket]{accept}{}
Accept a connection.
The socket must be bound to an address and listening for connections.
The return value is a pair \code{(\var{conn}, \var{address})}
where \var{conn} is a \emph{new} socket object usable to send and
receive data on the connection, and \var{address} is the address bound
to the socket on the other end of the connection.
\end{methoddesc}

\begin{methoddesc}[socket]{bind}{address}
Bind the socket to \var{address}.  The socket must not already be bound.
(The format of \var{address} depends on the address family --- see above.)
\end{methoddesc}

\begin{methoddesc}[socket]{close}{}
Close the socket.  All future operations on the socket object will fail.
The remote end will receive no more data (after queued data is flushed).
Sockets are automatically closed when they are garbage-collected.
\end{methoddesc}

\begin{methoddesc}[socket]{connect}{address}
Connect to a remote socket at \var{address}.
(The format of \var{address} depends on the address family --- see
above.)
\end{methoddesc}

\begin{methoddesc}[socket]{connect_ex}{address}
Like \code{connect(\var{address})}, but return an error indicator
instead of raising an exception for errors returned by the C-level
\cfunction{connect()} call (other problems, such as ``host not found,''
can still raise exceptions).  The error indicator is \code{0} if the
operation succeeded, otherwise the value of the \cdata{errno}
variable.  This is useful, e.g., for asynchronous connects.
\end{methoddesc}

\begin{methoddesc}[socket]{fileno}{}
Return the socket's file descriptor (a small integer).  This is useful
with \function{select.select()}.
\end{methoddesc}

\begin{methoddesc}[socket]{getpeername}{}
Return the remote address to which the socket is connected.  This is
useful to find out the port number of a remote IP socket, for instance.
(The format of the address returned depends on the address family ---
see above.)  On some systems this function is not supported.
\end{methoddesc}

\begin{methoddesc}[socket]{getsockname}{}
Return the socket's own address.  This is useful to find out the port
number of an IP socket, for instance.
(The format of the address returned depends on the address family ---
see above.)
\end{methoddesc}

\begin{methoddesc}[socket]{getsockopt}{level, optname\optional{, buflen}}
Return the value of the given socket option (see the \UNIX{} man page
\manpage{getsockopt}{2}).  The needed symbolic constants
(\constant{SO_*} etc.) are defined in this module.  If \var{buflen}
is absent, an integer option is assumed and its integer value
is returned by the function.  If \var{buflen} is present, it specifies
the maximum length of the buffer used to receive the option in, and
this buffer is returned as a string.  It is up to the caller to decode
the contents of the buffer (see the optional built-in module
\refmodule{struct} for a way to decode C structures encoded as strings).
\end{methoddesc}

\begin{methoddesc}[socket]{listen}{backlog}
Listen for connections made to the socket.  The \var{backlog} argument
specifies the maximum number of queued connections and should be at
least 1; the maximum value is system-dependent (usually 5).
\end{methoddesc}

\begin{methoddesc}[socket]{makefile}{\optional{mode\optional{, bufsize}}}
Return a \dfn{file object} associated with the socket.  (File objects
were described earlier in \ref{bltin-file-objects}, ``File Objects.'')
The file object references a \cfunction{dup()}ped version of the
socket file descriptor, so the file object and socket object may be
closed or garbage-collected independently.  The optional \var{mode}
and \var{bufsize} arguments are interpreted the same way as by the
built-in \function{open()} function.
\end{methoddesc}

\begin{methoddesc}[socket]{recv}{bufsize\optional{, flags}}
Receive data from the socket.  The return value is a string representing
the data received.  The maximum amount of data to be received
at once is specified by \var{bufsize}.  See the \UNIX{} manual page
\manpage{recv}{2} for the meaning of the optional argument
\var{flags}; it defaults to zero.
\end{methoddesc}

\begin{methoddesc}[socket]{recvfrom}{bufsize\optional{, flags}}
Receive data from the socket.  The return value is a pair
\code{(\var{string}, \var{address})} where \var{string} is a string
representing the data received and \var{address} is the address of the
socket sending the data.  The optional \var{flags} argument has the
same meaning as for \method{recv()} above.
(The format of \var{address} depends on the address family --- see above.)
\end{methoddesc}

\begin{methoddesc}[socket]{send}{string\optional{, flags}}
Send data to the socket.  The socket must be connected to a remote
socket.  The optional \var{flags} argument has the same meaning as for
\method{recv()} above.  Returns the number of bytes sent.
\end{methoddesc}

\begin{methoddesc}[socket]{sendto}{string\optional{, flags}, address}
Send data to the socket.  The socket should not be connected to a
remote socket, since the destination socket is specified by
\var{address}.  The optional \var{flags} argument has the same
meaning as for \method{recv()} above.  Return the number of bytes sent.
(The format of \var{address} depends on the address family --- see above.)
\end{methoddesc}

\begin{methoddesc}[socket]{setblocking}{flag}
Set blocking or non-blocking mode of the socket: if \var{flag} is 0,
the socket is set to non-blocking, else to blocking mode.  Initially
all sockets are in blocking mode.  In non-blocking mode, if a
\method{recv()} call doesn't find any data, or if a
\method{send()} call can't immediately dispose of the data, a
\exception{error} exception is raised; in blocking mode, the calls
block until they can proceed.
\end{methoddesc}

\begin{methoddesc}[socket]{setsockopt}{level, optname, value}
Set the value of the given socket option (see the \UNIX{} man page
\manpage{setsockopt}{2}).  The needed symbolic constants are defined in
the \module{socket} module (\code{SO_*} etc.).  The value can be an
integer or a string representing a buffer.  In the latter case it is
up to the caller to ensure that the string contains the proper bits
(see the optional built-in module
\refmodule{struct}\refbimodindex{struct} for a way to encode C
structures as strings). 
\end{methoddesc}

\begin{methoddesc}[socket]{shutdown}{how}
Shut down one or both halves of the connection.  If \var{how} is
\code{0}, further receives are disallowed.  If \var{how} is \code{1},
further sends are disallowed.  If \var{how} is \code{2}, further sends
and receives are disallowed.
\end{methoddesc}

Note that there are no methods \method{read()} or \method{write()};
use \method{recv()} and \method{send()} without \var{flags} argument
instead.

\subsection{Example}
\nodename{Socket Example}

Here are two minimal example programs using the TCP/IP protocol:\ a
server that echoes all data that it receives back (servicing only one
client), and a client using it.  Note that a server must perform the
sequence \function{socket()}, \method{bind()}, \method{listen()},
\method{accept()} (possibly repeating the \method{accept()} to service
more than one client), while a client only needs the sequence
\function{socket()}, \method{connect()}.  Also note that the server
does not \method{send()}/\method{recv()} on the 
socket it is listening on but on the new socket returned by
\method{accept()}.

\begin{verbatim}
# Echo server program
from socket import *
HOST = ''                 # Symbolic name meaning the local host
PORT = 50007              # Arbitrary non-privileged server
s = socket(AF_INET, SOCK_STREAM)
s.bind(HOST, PORT)
s.listen(1)
conn, addr = s.accept()
print 'Connected by', addr
while 1:
    data = conn.recv(1024)
    if not data: break
    conn.send(data)
conn.close()
\end{verbatim}

\begin{verbatim}
# Echo client program
from socket import *
HOST = 'daring.cwi.nl'    # The remote host
PORT = 50007              # The same port as used by the server
s = socket(AF_INET, SOCK_STREAM)
s.connect(HOST, PORT)
s.send('Hello, world')
data = s.recv(1024)
s.close()
print 'Received', `data`
\end{verbatim}

\begin{seealso}
\seemodule{SocketServer}{classes that simplify writing network servers}
\end{seealso}

\section{\module{select} ---
         Waiting for I/O completion}

\declaremodule{builtin}{select}
\modulesynopsis{Wait for I/O completion on multiple streams.}


This module provides access to the \cfunction{select()}
and \cfunction{poll()} functions
available in most operating systems.  Note that on Windows, it only
works for sockets; on other operating systems, it also works for other
file types (in particular, on \UNIX{}, it works on pipes).  It cannot
be used on regular files to determine whether a file has grown since
it was last read.

The module defines the following:

\begin{excdesc}{error}
The exception raised when an error occurs.  The accompanying value is
a pair containing the numeric error code from \cdata{errno} and the
corresponding string, as would be printed by the \C{} function
\cfunction{perror()}.
\end{excdesc}

\begin{funcdesc}{poll}{}
(Not supported by all operating systems.)  Returns a polling object, 
which supports registering and unregistering file descriptors, and
then polling them for I/O events; 
see section~\ref{poll-objects} below for the methods supported by 
polling objects.
\end{funcdesc}

\begin{funcdesc}{select}{iwtd, owtd, ewtd\optional{, timeout}}
This is a straightforward interface to the \UNIX{} \cfunction{select()}
system call.  The first three arguments are lists of `waitable
objects': either integers representing file descriptors or
objects with a parameterless method named \method{fileno()} returning
such an integer.  The three lists of waitable objects are for input,
output and `exceptional conditions', respectively.  Empty lists are
allowed, but acceptance of three empty lists is platform-dependent.
(It is known to work on \UNIX{} but not on Windows.)  The optional
\var{timeout} argument specifies a time-out as a floating point number
in seconds.  When the \var{timeout} argument is omitted the function
blocks until at least one file descriptor is ready.  A time-out value
of zero specifies a poll and never blocks.

The return value is a triple of lists of objects that are ready:
subsets of the first three arguments.  When the time-out is reached
without a file descriptor becoming ready, three empty lists are
returned.

Amongst the acceptable object types in the lists are Python file
objects (e.g. \code{sys.stdin}, or objects returned by
\function{open()} or \function{os.popen()}), socket objects
returned by \function{socket.socket()},%
\withsubitem{(in module socket)}{\ttindex{socket()}}
\withsubitem{(in module os)}{\ttindex{popen()}}.
You may also define a \dfn{wrapper} class yourself, as long as it has
an appropriate \method{fileno()} method (that really returns a file
descriptor, not just a random integer).
\strong{Note:}\index{WinSock}  File objects on Windows are not
acceptable, but sockets are.  On Windows, the underlying
\cfunction{select()} function is provided by the WinSock library, and
does not handle file desciptors that don't originate from WinSock.
\end{funcdesc}

\subsection{Polling Objects
            \label{poll-objects}}

The \cfunction{poll()} system call, supported on most Unix systems,
provides better scalability for network servers that service many,
many clients at the same time.
\cfunction{poll()} scales better because the system call only
requires listing the file descriptors of interest, while \cfunction{select()}
builds a bitmap, turns on bits for the fds of interest, and then
afterward the whole bitmap has to be linearly scanned again.
\cfunction{select()} is O(highest file descriptor), while
\cfunction{poll()} is O(number of file descriptors).

\begin{methoddesc}{register}{fd\optional{, eventmask}}
Register a file descriptor with the polling object.  Future calls to
the \method{poll()} method will then check whether the file descriptor
has any pending I/O events.  \var{fd} can be either an integer, or an
object with a \method{fileno()} method that returns an integer.  File
objects implement
\method{fileno()}, so they can also be used as the argument.

\var{eventmask} is an optional bitmask describing the type of events you
want to check for, and can be a combination of the constants
\constant{POLLIN}, \constant{POLLPRI}, and \constant{POLLOUT},
described in the table below.  If not specified, the default value
used will check for all 3 types of events.

\begin{tableii}{l|l}{constant}{Constant}{Meaning}
  \lineii{POLLIN}{There is data to read}
  \lineii{POLLPRI}{There is urgent data to read}
  \lineii{POLLOUT}{Ready for output: writing will not block}
  \lineii{POLLERR}{Error condition of some sort}
  \lineii{POLLHUP}{Hung up}
  \lineii{POLLNVAL}{Invalid request: descriptor not open}
\end{tableii}

Registering a file descriptor that's already registered is not an
error, and has the same effect as registering the descriptor exactly
once. 
 
\end{methoddesc}

\begin{methoddesc}{unregister}{fd}
Remove a file descriptor being tracked by a polling object.  Just like
the \method{register()} method, \var{fd} can be an integer or an
object with a \method{fileno()} method that returns an integer.

Attempting to remove a file descriptor that was never registered
causes a \exception{KeyError} exception to be raised.
\end{methoddesc}

\begin{methoddesc}{poll}{\optional{timeout}}
Polls the set of registered file descriptors, and returns a
possibly-empty list containing \code{(\var{fd}, \var{event})} 2-tuples
for the descriptors that have events or errors to report.
\var{fd} is the file descriptor, and \var{event} is a bitmask 
with bits set for the reported events for that descriptor
--- \constant{POLLIN} for waiting input, 
\constant{POLLOUT} to indicate that the descriptor can be written to, and
so forth.
An empty list indicates that the call timed out and no file
descriptors had any events to report.
If \var{timeout} is given, it specifies the length of time in
milliseconds which the system will wait for events before returning.
If \var{timeout} is omitted, negative, or \code{None}, the call will
block until there is an event for this poll object.
\end{methoddesc}



\section{\module{thread} ---
         Multiple threads of control}

\declaremodule{builtin}{thread}
\modulesynopsis{Create multiple threads of control within one interpreter.}


This module provides low-level primitives for working with multiple
threads (a.k.a.\ \dfn{light-weight processes} or \dfn{tasks}) --- multiple
threads of control sharing their global data space.  For
synchronization, simple locks (a.k.a.\ \dfn{mutexes} or \dfn{binary
semaphores}) are provided.
\index{light-weight processes}
\index{processes, light-weight}
\index{binary semaphores}
\index{semaphores, binary}

The module is optional.  It is supported on Windows NT and '95, SGI
IRIX, Solaris 2.x, as well as on systems that have a \POSIX{} thread
(a.k.a. ``pthread'') implementation.
\index{pthreads}
\indexii{threads}{\POSIX{}}

It defines the following constant and functions:

\begin{excdesc}{error}
Raised on thread-specific errors.
\end{excdesc}

\begin{datadesc}{LockType}
This is the type of lock objects.
\end{datadesc}

\begin{funcdesc}{start_new_thread}{function, args\optional{, kwargs}}
Start a new thread.  The thread executes the function \var{function}
with the argument list \var{args} (which must be a tuple).  The
optional \var{kwargs} argument specifies a dictionary of keyword
arguments.  When the
function returns, the thread silently exits.  When the function
terminates with an unhandled exception, a stack trace is printed and
then the thread exits (but other threads continue to run).
\end{funcdesc}

\begin{funcdesc}{exit}{}
Raise the \exception{SystemExit} exception.  When not caught, this
will cause the thread to exit silently.
\end{funcdesc}

\begin{funcdesc}{exit_thread}{}
\deprecated{1.5.2}{Use \function{exit()}.}
This is an obsolete synonym for \function{exit()}.
\end{funcdesc}

%\begin{funcdesc}{exit_prog}{status}
%Exit all threads and report the value of the integer argument
%\var{status} as the exit status of the entire program.
%\strong{Caveat:} code in pending \keyword{finally} clauses, in this thread
%or in other threads, is not executed.
%\end{funcdesc}

\begin{funcdesc}{allocate_lock}{}
Return a new lock object.  Methods of locks are described below.  The
lock is initially unlocked.
\end{funcdesc}

\begin{funcdesc}{get_ident}{}
Return the `thread identifier' of the current thread.  This is a
nonzero integer.  Its value has no direct meaning; it is intended as a
magic cookie to be used e.g. to index a dictionary of thread-specific
data.  Thread identifiers may be recycled when a thread exits and
another thread is created.
\end{funcdesc}


Lock objects have the following methods:

\begin{methoddesc}[lock]{acquire}{\optional{waitflag}}
Without the optional argument, this method acquires the lock
unconditionally, if necessary waiting until it is released by another
thread (only one thread at a time can acquire a lock --- that's their
reason for existence), and returns \code{None}.  If the integer
\var{waitflag} argument is present, the action depends on its
value: if it is zero, the lock is only acquired if it can be acquired
immediately without waiting, while if it is nonzero, the lock is
acquired unconditionally as before.  If an argument is present, the
return value is \code{1} if the lock is acquired successfully,
\code{0} if not.
\end{methoddesc}

\begin{methoddesc}[lock]{release}{}
Releases the lock.  The lock must have been acquired earlier, but not
necessarily by the same thread.
\end{methoddesc}

\begin{methoddesc}[lock]{locked}{}
Return the status of the lock:\ \code{1} if it has been acquired by
some thread, \code{0} if not.
\end{methoddesc}

\strong{Caveats:}

\begin{itemize}
\item
Threads interact strangely with interrupts: the
\exception{KeyboardInterrupt} exception will be received by an
arbitrary thread.  (When the \refmodule{signal}\refbimodindex{signal}
module is available, interrupts always go to the main thread.)

\item
Calling \function{sys.exit()} or raising the \exception{SystemExit}
exception is equivalent to calling \function{exit()}.

\item
Not all built-in functions that may block waiting for I/O allow other
threads to run.  (The most popular ones (\function{time.sleep()},
\method{\var{file}.read()}, \function{select.select()}) work as
expected.)

\item
It is not possible to interrupt the \method{acquire()} method on a lock
--- the \exception{KeyboardInterrupt} exception will happen after the
lock has been acquired.

\item
When the main thread exits, it is system defined whether the other
threads survive.  On SGI IRIX using the native thread implementation,
they survive.  On most other systems, they are killed without
executing \keyword{try} ... \keyword{finally} clauses or executing
object destructors.
\indexii{threads}{IRIX}

\item
When the main thread exits, it does not do any of its usual cleanup
(except that \keyword{try} ... \keyword{finally} clauses are honored),
and the standard I/O files are not flushed.

\end{itemize}

\section{\module{threading} ---
         Higher-level threading interface}

\declaremodule{standard}{threading}
\modulesynopsis{Higher-level threading interface.}


This module constructs higher-level threading interfaces on top of the 
lower level \refmodule{thread} module.

The \refmodule[dummythreading]{dummy_threading} module is provided for
situations where \module{threading} cannot be used because
\refmodule{thread} is missing.

This module defines the following functions and objects:

\begin{funcdesc}{activeCount}{}
Return the number of currently active \class{Thread} objects.
The returned count is equal to the length of the list returned by
\function{enumerate()}.
A function that returns the number of currently active threads.
\end{funcdesc}

\begin{funcdesc}{Condition}{}
A factory function that returns a new condition variable object.
A condition variable allows one or more threads to wait until they
are notified by another thread.
\end{funcdesc}

\begin{funcdesc}{currentThread}{}
Return the current \class{Thread} object, corresponding to the
caller's thread of control.  If the caller's thread of control was not
created through the
\module{threading} module, a dummy thread object with limited functionality
is returned.
\end{funcdesc}

\begin{funcdesc}{enumerate}{}
Return a list of all currently active \class{Thread} objects.
The list includes daemonic threads, dummy thread objects created
by \function{currentThread()}, and the main thread.  It excludes terminated
threads and threads that have not yet been started.
\end{funcdesc}

\begin{funcdesc}{Event}{}
A factory function that returns a new event object.  An event manages
a flag that can be set to true with the \method{set()} method and
reset to false with the \method{clear()} method.  The \method{wait()}
method blocks until the flag is true.
\end{funcdesc}

\begin{classdesc*}{local}{}
A class that represents thread-local data.  Thread-local data are data
whose values are thread specific.  To manage thread-local data, just
create an instance of \class{local} (or a subclass) and store
attributes on it:

\begin{verbatim}
mydata = threading.local()
mydata.x = 1
\end{verbatim}

The instance's values will be different for separate threads.

For more details and extensive examples, see the documentation string
of the \module{_threading_local} module.

\versionadded{2.4}
\end{classdesc*}

\begin{funcdesc}{Lock}{}
A factory function that returns a new primitive lock object.  Once
a thread has acquired it, subsequent attempts to acquire it block,
until it is released; any thread may release it.
\end{funcdesc}

\begin{funcdesc}{RLock}{}
A factory function that returns a new reentrant lock object.
A reentrant lock must be released by the thread that acquired it.
Once a thread has acquired a reentrant lock, the same thread may
acquire it again without blocking; the thread must release it once
for each time it has acquired it.
\end{funcdesc}

\begin{funcdesc}{Semaphore}{\optional{value}}
A factory function that returns a new semaphore object.  A
semaphore manages a counter representing the number of \method{release()}
calls minus the number of \method{acquire()} calls, plus an initial value.
The \method{acquire()} method blocks if necessary until it can return
without making the counter negative.  If not given, \var{value} defaults to
1. 
\end{funcdesc}

\begin{funcdesc}{BoundedSemaphore}{\optional{value}}
A factory function that returns a new bounded semaphore object.  A bounded
semaphore checks to make sure its current value doesn't exceed its initial
value.  If it does, \exception{ValueError} is raised. In most situations
semaphores are used to guard resources with limited capacity.  If the
semaphore is released too many times it's a sign of a bug.  If not given,
\var{value} defaults to 1. 
\end{funcdesc}

\begin{classdesc*}{Thread}{}
A class that represents a thread of control.  This class can be safely
subclassed in a limited fashion.
\end{classdesc*}

\begin{classdesc*}{Timer}{}
A thread that executes a function after a specified interval has passed.
\end{classdesc*}

\begin{funcdesc}{settrace}{func}
Set a trace function\index{trace function} for all threads started
from the \module{threading} module.  The \var{func} will be passed to 
\function{sys.settrace()} for each thread, before its \method{run()}
method is called.
\versionadded{2.3}
\end{funcdesc}

\begin{funcdesc}{setprofile}{func}
Set a profile function\index{profile function} for all threads started
from the \module{threading} module.  The \var{func} will be passed to 
\function{sys.setprofile()} for each thread, before its \method{run()}
method is called.
\versionadded{2.3}
\end{funcdesc}

\begin{funcdesc}{stack_size}{\optional{size}}
Return the thread stack size used when creating new threads.  The
optional \var{size} argument specifies the stack size to be used for
subsequently created threads, and must be 0 (use platform or
configured default) or a positive integer value of at least 32,768 (32kB).
If changing the thread stack size is unsupported, or the specified size
is invalid, a RuntimeWarning is issued and the stack size is unmodified.
32kB is currently the minimum supported stack size value, to guarantee
sufficient stack space for the interpreter itself.
Note that some platforms may have particular restrictions on values for
the stack size, such as requiring allocation in multiples of the system
memory page size - platform documentation should be referred to for more
information (4kB pages are common; using multiples of 4096 for the
stack size is the suggested approach in the absence of more specific
information).
Availability: Windows, systems with \POSIX{} threads.
\versionadded{2.5}
\end{funcdesc}

Detailed interfaces for the objects are documented below.  

The design of this module is loosely based on Java's threading model.
However, where Java makes locks and condition variables basic behavior
of every object, they are separate objects in Python.  Python's \class{Thread}
class supports a subset of the behavior of Java's Thread class;
currently, there are no priorities, no thread groups, and threads
cannot be destroyed, stopped, suspended, resumed, or interrupted.  The
static methods of Java's Thread class, when implemented, are mapped to
module-level functions.

All of the methods described below are executed atomically.


\subsection{Lock Objects \label{lock-objects}}

A primitive lock is a synchronization primitive that is not owned
by a particular thread when locked.  In Python, it is currently
the lowest level synchronization primitive available, implemented
directly by the \refmodule{thread} extension module.

A primitive lock is in one of two states, ``locked'' or ``unlocked''.
It is created in the unlocked state.  It has two basic methods,
\method{acquire()} and \method{release()}.  When the state is
unlocked, \method{acquire()} changes the state to locked and returns
immediately.  When the state is locked, \method{acquire()} blocks
until a call to \method{release()} in another thread changes it to
unlocked, then the \method{acquire()} call resets it to locked and
returns.  The \method{release()} method should only be called in the
locked state; it changes the state to unlocked and returns
immediately.  When more than one thread is blocked in
\method{acquire()} waiting for the state to turn to unlocked, only one
thread proceeds when a \method{release()} call resets the state to
unlocked; which one of the waiting threads proceeds is not defined,
and may vary across implementations.

All methods are executed atomically.

\begin{methoddesc}{acquire}{\optional{blocking\code{ = 1}}}
Acquire a lock, blocking or non-blocking.

When invoked without arguments, block until the lock is
unlocked, then set it to locked, and return true.  

When invoked with the \var{blocking} argument set to true, do the
same thing as when called without arguments, and return true.

When invoked with the \var{blocking} argument set to false, do not
block.  If a call without an argument would block, return false
immediately; otherwise, do the same thing as when called
without arguments, and return true.
\end{methoddesc}

\begin{methoddesc}{release}{}
Release a lock.

When the lock is locked, reset it to unlocked, and return.  If
any other threads are blocked waiting for the lock to become
unlocked, allow exactly one of them to proceed.

Do not call this method when the lock is unlocked.

There is no return value.
\end{methoddesc}


\subsection{RLock Objects \label{rlock-objects}}

A reentrant lock is a synchronization primitive that may be
acquired multiple times by the same thread.  Internally, it uses
the concepts of ``owning thread'' and ``recursion level'' in
addition to the locked/unlocked state used by primitive locks.  In
the locked state, some thread owns the lock; in the unlocked
state, no thread owns it.

To lock the lock, a thread calls its \method{acquire()} method; this
returns once the thread owns the lock.  To unlock the lock, a
thread calls its \method{release()} method.
\method{acquire()}/\method{release()} call pairs may be nested; only
the final \method{release()} (the \method{release()} of the outermost
pair) resets the lock to unlocked and allows another thread blocked in
\method{acquire()} to proceed.

\begin{methoddesc}{acquire}{\optional{blocking\code{ = 1}}}
Acquire a lock, blocking or non-blocking.

When invoked without arguments: if this thread already owns
the lock, increment the recursion level by one, and return
immediately.  Otherwise, if another thread owns the lock,
block until the lock is unlocked.  Once the lock is unlocked
(not owned by any thread), then grab ownership, set the
recursion level to one, and return.  If more than one thread
is blocked waiting until the lock is unlocked, only one at a
time will be able to grab ownership of the lock.  There is no
return value in this case.

When invoked with the \var{blocking} argument set to true, do the
same thing as when called without arguments, and return true.

When invoked with the \var{blocking} argument set to false, do not
block.  If a call without an argument would block, return false
immediately; otherwise, do the same thing as when called
without arguments, and return true.
\end{methoddesc}

\begin{methoddesc}{release}{}
Release a lock, decrementing the recursion level.  If after the
decrement it is zero, reset the lock to unlocked (not owned by any
thread), and if any other threads are blocked waiting for the lock to
become unlocked, allow exactly one of them to proceed.  If after the
decrement the recursion level is still nonzero, the lock remains
locked and owned by the calling thread.

Only call this method when the calling thread owns the lock.
Do not call this method when the lock is unlocked.

There is no return value.
\end{methoddesc}


\subsection{Condition Objects \label{condition-objects}}

A condition variable is always associated with some kind of lock;
this can be passed in or one will be created by default.  (Passing
one in is useful when several condition variables must share the
same lock.)

A condition variable has \method{acquire()} and \method{release()}
methods that call the corresponding methods of the associated lock.
It also has a \method{wait()} method, and \method{notify()} and
\method{notifyAll()} methods.  These three must only be called when
the calling thread has acquired the lock.

The \method{wait()} method releases the lock, and then blocks until it
is awakened by a \method{notify()} or \method{notifyAll()} call for
the same condition variable in another thread.  Once awakened, it
re-acquires the lock and returns.  It is also possible to specify a
timeout.

The \method{notify()} method wakes up one of the threads waiting for
the condition variable, if any are waiting.  The \method{notifyAll()}
method wakes up all threads waiting for the condition variable.

Note: the \method{notify()} and \method{notifyAll()} methods don't
release the lock; this means that the thread or threads awakened will
not return from their \method{wait()} call immediately, but only when
the thread that called \method{notify()} or \method{notifyAll()}
finally relinquishes ownership of the lock.

Tip: the typical programming style using condition variables uses the
lock to synchronize access to some shared state; threads that are
interested in a particular change of state call \method{wait()}
repeatedly until they see the desired state, while threads that modify
the state call \method{notify()} or \method{notifyAll()} when they
change the state in such a way that it could possibly be a desired
state for one of the waiters.  For example, the following code is a
generic producer-consumer situation with unlimited buffer capacity:

\begin{verbatim}
# Consume one item
cv.acquire()
while not an_item_is_available():
    cv.wait()
get_an_available_item()
cv.release()

# Produce one item
cv.acquire()
make_an_item_available()
cv.notify()
cv.release()
\end{verbatim}

To choose between \method{notify()} and \method{notifyAll()}, consider
whether one state change can be interesting for only one or several
waiting threads.  E.g. in a typical producer-consumer situation,
adding one item to the buffer only needs to wake up one consumer
thread.

\begin{classdesc}{Condition}{\optional{lock}}
If the \var{lock} argument is given and not \code{None}, it must be a
\class{Lock} or \class{RLock} object, and it is used as the underlying
lock.  Otherwise, a new \class{RLock} object is created and used as
the underlying lock.
\end{classdesc}

\begin{methoddesc}{acquire}{*args}
Acquire the underlying lock.
This method calls the corresponding method on the underlying
lock; the return value is whatever that method returns.
\end{methoddesc}

\begin{methoddesc}{release}{}
Release the underlying lock.
This method calls the corresponding method on the underlying
lock; there is no return value.
\end{methoddesc}

\begin{methoddesc}{wait}{\optional{timeout}}
Wait until notified or until a timeout occurs.
This must only be called when the calling thread has acquired the
lock.

This method releases the underlying lock, and then blocks until it is
awakened by a \method{notify()} or \method{notifyAll()} call for the
same condition variable in another thread, or until the optional
timeout occurs.  Once awakened or timed out, it re-acquires the lock
and returns.

When the \var{timeout} argument is present and not \code{None}, it
should be a floating point number specifying a timeout for the
operation in seconds (or fractions thereof).

When the underlying lock is an \class{RLock}, it is not released using
its \method{release()} method, since this may not actually unlock the
lock when it was acquired multiple times recursively.  Instead, an
internal interface of the \class{RLock} class is used, which really
unlocks it even when it has been recursively acquired several times.
Another internal interface is then used to restore the recursion level
when the lock is reacquired.
\end{methoddesc}

\begin{methoddesc}{notify}{}
Wake up a thread waiting on this condition, if any.
This must only be called when the calling thread has acquired the
lock.

This method wakes up one of the threads waiting for the condition
variable, if any are waiting; it is a no-op if no threads are waiting.

The current implementation wakes up exactly one thread, if any are
waiting.  However, it's not safe to rely on this behavior.  A future,
optimized implementation may occasionally wake up more than one
thread.

Note: the awakened thread does not actually return from its
\method{wait()} call until it can reacquire the lock.  Since
\method{notify()} does not release the lock, its caller should.
\end{methoddesc}

\begin{methoddesc}{notifyAll}{}
Wake up all threads waiting on this condition.  This method acts like
\method{notify()}, but wakes up all waiting threads instead of one.
\end{methoddesc}


\subsection{Semaphore Objects \label{semaphore-objects}}

This is one of the oldest synchronization primitives in the history of
computer science, invented by the early Dutch computer scientist
Edsger W. Dijkstra (he used \method{P()} and \method{V()} instead of
\method{acquire()} and \method{release()}).

A semaphore manages an internal counter which is decremented by each
\method{acquire()} call and incremented by each \method{release()}
call.  The counter can never go below zero; when \method{acquire()}
finds that it is zero, it blocks, waiting until some other thread
calls \method{release()}.

\begin{classdesc}{Semaphore}{\optional{value}}
The optional argument gives the initial value for the internal
counter; it defaults to \code{1}.
\end{classdesc}

\begin{methoddesc}{acquire}{\optional{blocking}}
Acquire a semaphore.

When invoked without arguments: if the internal counter is larger than
zero on entry, decrement it by one and return immediately.  If it is
zero on entry, block, waiting until some other thread has called
\method{release()} to make it larger than zero.  This is done with
proper interlocking so that if multiple \method{acquire()} calls are
blocked, \method{release()} will wake exactly one of them up.  The
implementation may pick one at random, so the order in which blocked
threads are awakened should not be relied on.  There is no return
value in this case.

When invoked with \var{blocking} set to true, do the same thing as
when called without arguments, and return true.

When invoked with \var{blocking} set to false, do not block.  If a
call without an argument would block, return false immediately;
otherwise, do the same thing as when called without arguments, and
return true.
\end{methoddesc}

\begin{methoddesc}{release}{}
Release a semaphore,
incrementing the internal counter by one.  When it was zero on
entry and another thread is waiting for it to become larger
than zero again, wake up that thread.
\end{methoddesc}


\subsubsection{\class{Semaphore} Example \label{semaphore-examples}}

Semaphores are often used to guard resources with limited capacity, for
example, a database server.  In any situation where the size of the resource
size is fixed, you should use a bounded semaphore.  Before spawning any
worker threads, your main thread would initialize the semaphore:

\begin{verbatim}
maxconnections = 5
...
pool_sema = BoundedSemaphore(value=maxconnections)
\end{verbatim}

Once spawned, worker threads call the semaphore's acquire and release
methods when they need to connect to the server:

\begin{verbatim}
pool_sema.acquire()
conn = connectdb()
... use connection ...
conn.close()
pool_sema.release()
\end{verbatim}

The use of a bounded semaphore reduces the chance that a programming error
which causes the semaphore to be released more than it's acquired will go
undetected.


\subsection{Event Objects \label{event-objects}}

This is one of the simplest mechanisms for communication between
threads: one thread signals an event and other threads wait for it.

An event object manages an internal flag that can be set to true with
the \method{set()} method and reset to false with the \method{clear()}
method.  The \method{wait()} method blocks until the flag is true.


\begin{classdesc}{Event}{}
The internal flag is initially false.
\end{classdesc}

\begin{methoddesc}{isSet}{}
Return true if and only if the internal flag is true.
\end{methoddesc}

\begin{methoddesc}{set}{}
Set the internal flag to true.
All threads waiting for it to become true are awakened.
Threads that call \method{wait()} once the flag is true will not block
at all.
\end{methoddesc}

\begin{methoddesc}{clear}{}
Reset the internal flag to false.
Subsequently, threads calling \method{wait()} will block until
\method{set()} is called to set the internal flag to true again.
\end{methoddesc}

\begin{methoddesc}{wait}{\optional{timeout}}
Block until the internal flag is true.
If the internal flag is true on entry, return immediately.  Otherwise,
block until another thread calls \method{set()} to set the flag to
true, or until the optional timeout occurs.

When the timeout argument is present and not \code{None}, it should be a
floating point number specifying a timeout for the operation in
seconds (or fractions thereof).
\end{methoddesc}


\subsection{Thread Objects \label{thread-objects}}

This class represents an activity that is run in a separate thread
of control.  There are two ways to specify the activity: by
passing a callable object to the constructor, or by overriding the
\method{run()} method in a subclass.  No other methods (except for the
constructor) should be overridden in a subclass.  In other words, 
\emph{only}  override the \method{__init__()} and \method{run()}
methods of this class.

Once a thread object is created, its activity must be started by
calling the thread's \method{start()} method.  This invokes the
\method{run()} method in a separate thread of control.

Once the thread's activity is started, the thread is considered
'alive' and 'active' (these concepts are almost, but not quite
exactly, the same; their definition is intentionally somewhat
vague).  It stops being alive and active when its \method{run()}
method terminates -- either normally, or by raising an unhandled
exception.  The \method{isAlive()} method tests whether the thread is
alive.

Other threads can call a thread's \method{join()} method.  This blocks
the calling thread until the thread whose \method{join()} method is
called is terminated.

A thread has a name.  The name can be passed to the constructor,
set with the \method{setName()} method, and retrieved with the
\method{getName()} method.

A thread can be flagged as a ``daemon thread''.  The significance
of this flag is that the entire Python program exits when only
daemon threads are left.  The initial value is inherited from the
creating thread.  The flag can be set with the \method{setDaemon()}
method and retrieved with the \method{isDaemon()} method.

There is a ``main thread'' object; this corresponds to the
initial thread of control in the Python program.  It is not a
daemon thread.

There is the possibility that ``dummy thread objects'' are
created.  These are thread objects corresponding to ``alien
threads''.  These are threads of control started outside the
threading module, such as directly from C code.  Dummy thread objects
have limited functionality; they are always considered alive,
active, and daemonic, and cannot be \method{join()}ed.  They are never 
deleted, since it is impossible to detect the termination of alien
threads.


\begin{classdesc}{Thread}{group=None, target=None, name=None,
                          args=(), kwargs=\{\}}
This constructor should always be called with keyword
arguments.  Arguments are:

\var{group} should be \code{None}; reserved for future extension when
a \class{ThreadGroup} class is implemented.

\var{target} is the callable object to be invoked by the
\method{run()} method.  Defaults to \code{None}, meaning nothing is
called.

\var{name} is the thread name.  By default, a unique name is
constructed of the form ``Thread-\var{N}'' where \var{N} is a small
decimal number.

\var{args} is the argument tuple for the target invocation.  Defaults
to \code{()}.

\var{kwargs} is a dictionary of keyword arguments for the target
invocation.  Defaults to \code{\{\}}.

If the subclass overrides the constructor, it must make sure
to invoke the base class constructor (\code{Thread.__init__()})
before doing anything else to the thread.
\end{classdesc}

\begin{methoddesc}{start}{}
Start the thread's activity.

This must be called at most once per thread object.  It
arranges for the object's \method{run()} method to be invoked in a
separate thread of control.
\end{methoddesc}

\begin{methoddesc}{run}{}
Method representing the thread's activity.

You may override this method in a subclass.  The standard
\method{run()} method invokes the callable object passed to the
object's constructor as the \var{target} argument, if any, with
sequential and keyword arguments taken from the \var{args} and
\var{kwargs} arguments, respectively.
\end{methoddesc}

\begin{methoddesc}{join}{\optional{timeout}}
Wait until the thread terminates.
This blocks the calling thread until the thread whose \method{join()}
method is called terminates -- either normally or through an
unhandled exception -- or until the optional timeout occurs.

When the \var{timeout} argument is present and not \code{None}, it
should be a floating point number specifying a timeout for the
operation in seconds (or fractions thereof). As \method{join()} always 
returns \code{None}, you must call \method{isAlive()} to decide whether 
a timeout happened.

When the \var{timeout} argument is not present or \code{None}, the
operation will block until the thread terminates.

A thread can be \method{join()}ed many times.

A thread cannot join itself because this would cause a
deadlock.

It is an error to attempt to \method{join()} a thread before it has
been started.
\end{methoddesc}

\begin{methoddesc}{getName}{}
Return the thread's name.
\end{methoddesc}

\begin{methoddesc}{setName}{name}
Set the thread's name.

The name is a string used for identification purposes only.
It has no semantics.  Multiple threads may be given the same
name.  The initial name is set by the constructor.
\end{methoddesc}

\begin{methoddesc}{isAlive}{}
Return whether the thread is alive.

Roughly, a thread is alive from the moment the \method{start()} method
returns until its \method{run()} method terminates.
\end{methoddesc}

\begin{methoddesc}{isDaemon}{}
Return the thread's daemon flag.
\end{methoddesc}

\begin{methoddesc}{setDaemon}{daemonic}
Set the thread's daemon flag to the Boolean value \var{daemonic}.
This must be called before \method{start()} is called.

The initial value is inherited from the creating thread.

The entire Python program exits when no active non-daemon
threads are left.
\end{methoddesc}


\subsection{Timer Objects \label{timer-objects}}

This class represents an action that should be run only after a
certain amount of time has passed --- a timer.  \class{Timer} is a
subclass of \class{Thread} and as such also functions as an example of
creating custom threads.

Timers are started, as with threads, by calling their \method{start()}
method.  The timer can be stopped (before its action has begun) by
calling the \method{cancel()} method.  The interval the timer will
wait before executing its action may not be exactly the same as the
interval specified by the user.

For example:
\begin{verbatim}
def hello():
    print "hello, world"

t = Timer(30.0, hello)
t.start() # after 30 seconds, "hello, world" will be printed
\end{verbatim}

\begin{classdesc}{Timer}{interval, function, args=[], kwargs=\{\}}
Create a timer that will run \var{function} with arguments \var{args} and 
keyword arguments \var{kwargs}, after \var{interval} seconds have passed.
\end{classdesc}

\begin{methoddesc}{cancel}{}
Stop the timer, and cancel the execution of the timer's action.  This
will only work if the timer is still in its waiting stage.
\end{methoddesc}

\subsection{Using locks, conditions, and semaphores in the \keyword{with}
statement \label{with-locks}}

All of the objects provided by this module that have \method{acquire()} and
\method{release()} methods can be used as context managers for a \keyword{with}
statement.  The \method{acquire()} method will be called when the block is
entered, and \method{release()} will be called when the block is exited.

Currently, \class{Lock}, \class{RLock}, \class{Condition}, \class{Semaphore},
and \class{BoundedSemaphore} objects may be used as \keyword{with}
statement context managers.  For example:

\begin{verbatim}
from __future__ import with_statement
import threading

some_rlock = threading.RLock()

with some_rlock:
    print "some_rlock is locked while this executes"
\end{verbatim}


\section{\module{Queue} ---
         A synchronized queue class.}
\declaremodule{standard}{Queue}

\modulesynopsis{A synchronized queue class.}



The \module{Queue} module implements a multi-producer, multi-consumer
FIFO queue.  It is especially useful in threads programming when
information must be exchanged safely between multiple threads.  The
\class{Queue} class in this module implements all the required locking
semantics.  It depends on the availability of thread support in
Python.

The \module{Queue} module defines the following class and exception:


\begin{classdesc}{Queue}{maxsize}
Constructor for the class.  \var{maxsize} is an integer that sets the
upperbound limit on the number of items that can be placed in the
queue.  Insertion will block once this size has been reached, until
queue items are consumed.  If \var{maxsize} is less than or equal to
zero, the queue size is infinite.
\end{classdesc}

\begin{excdesc}{Empty}
Exception raised when non-blocking get (e.g. \method{get_nowait()}) is
called on a \class{Queue} object which is empty, or for which the
emptyiness cannot be determined (i.e. because the appropriate locks
cannot be acquired).
\end{excdesc}

\subsection{Queue Objects}
\label{QueueObjects}

Class \class{Queue} implements queue objects and has the methods
described below.  This class can be derived from in order to implement
other queue organizations (e.g. stack) but the inheritable interface
is not described here.  See the source code for details.  The public
methods are:

\begin{methoddesc}{qsize}{}
Returns the approximate size of the queue.  Because of multithreading
semantics, this number is not reliable.
\end{methoddesc}

\begin{methoddesc}{empty}{}
Returns \code{1} if the queue is empty, \code{0} otherwise.  Because
of multithreading semantics, this is not reliable.
\end{methoddesc}

\begin{methoddesc}{full}{}
Returns \code{1} if the queue is full, \code{0} otherwise.  Because of
multithreading semantics, this is not reliable.
\end{methoddesc}

\begin{methoddesc}{put}{item}
Puts \var{item} into the queue.
\end{methoddesc}

\begin{methoddesc}{get}{}
Gets and returns an item from the queue, blocking if necessary until
one is available.
\end{methoddesc}

\begin{methoddesc}{get_nowait}{}
Gets and returns an item from the queue if one is immediately
available.  Raises an \exception{Empty} exception if the queue is
empty or if the queue's emptiness cannot be determined.
\end{methoddesc}

\section{\module{anydbm} ---
         Generic access to DBM-style databases}

\declaremodule{standard}{anydbm}
\modulesynopsis{Generic interface to DBM-style database modules.}


\module{anydbm} is a generic interface to variants of the DBM
database --- \refmodule{dbhash}\refstmodindex{dbhash} (requires
\refmodule{bsddb}\refbimodindex{bsddb}),
\refmodule{gdbm}\refbimodindex{gdbm}, or
\refmodule{dbm}\refbimodindex{dbm}.  If none of these modules is
installed, the slow-but-simple implementation in module
\refmodule{dumbdbm}\refstmodindex{dumbdbm} will be used.

\begin{funcdesc}{open}{filename\optional{, flag\optional{, mode}}}
Open the database file \var{filename} and return a corresponding object.

If the database file already exists, the \refmodule{whichdb} module is 
used to determine its type and the appropriate module is used; if it
does not exist, the first module listed above that can be imported is
used.

The optional \var{flag} argument can be
\code{'r'} to open an existing database for reading only,
\code{'w'} to open an existing database for reading and writing,
\code{'c'} to create the database if it doesn't exist, or
\code{'n'}, which will always create a new empty database.  If not
specified, the default value is \code{'r'}.

The optional \var{mode} argument is the \UNIX{} mode of the file, used
only when the database has to be created.  It defaults to octal
\code{0666} (and will be modified by the prevailing umask).
\end{funcdesc}

\begin{excdesc}{error}
A tuple containing the exceptions that can be raised by each of the
supported modules, with a unique exception \exception{anydbm.error} as
the first item --- the latter is used when \exception{anydbm.error} is
raised.
\end{excdesc}

The object returned by \function{open()} supports most of the same
functionality as dictionaries; keys and their corresponding values can
be stored, retrieved, and deleted, and the \method{has_key()} and
\method{keys()} methods are available.  Keys and values must always be
strings.


\begin{seealso}
  \seemodule{anydbm}{Generic interface to \code{dbm}-style databases.}
  \seemodule{dbhash}{BSD \code{db} database interface.}
  \seemodule{dbm}{Standard \UNIX{} database interface.}
  \seemodule{dumbdbm}{Portable implementation of the \code{dbm} interface.}
  \seemodule{gdbm}{GNU database interface, based on the \code{dbm} interface.}
  \seemodule{shelve}{General object persistence built on top of 
                     the Python \code{dbm} interface.}
  \seemodule{whichdb}{Utility module used to determine the type of an
                      existing database.}
\end{seealso}


\section{\module{dumbdbm} ---
         Portable DBM implementation}

\declaremodule{standard}{dumbdbm}
\modulesynopsis{Portable implementation of the simple DBM interface.}


A simple and slow database implemented entirely in Python.  This
should only be used when no other DBM-style database is available.


\begin{funcdesc}{open}{filename\optional{, flag\optional{, mode}}}
Open the database file \var{filename} and return a corresponding object.
The optional \var{flag} argument can be
\code{'r'} to open an existing database for reading only,
\code{'w'} to open an existing database for reading and writing,
\code{'c'} to create the database if it doesn't exist, or
\code{'n'}, which will always create a new empty database.  If not
specified, the default value is \code{'r'}.

The optional \var{mode} argument is the \UNIX{} mode of the file, used
only when the database has to be created.  It defaults to octal
\code{0666} (and will be modified by the prevailing umask).
\end{funcdesc}

\begin{excdesc}{error}
Raised for errors not reported as \exception{KeyError} errors.
\end{excdesc}


\begin{seealso}
  \seemodule{anydbm}{Generic interface to \code{dbm}-style databases.}
  \seemodule{whichdb}{Utility module used to determine the type of an
                      existing database.}
\end{seealso}

\section{\module{dbhash} ---
         DBM-style interface to the BSD database library}

\declaremodule{standard}{dbhash}
  \platform{Unix, Windows}
\modulesynopsis{DBM-style interface to the BSD database library.}
\sectionauthor{Fred L. Drake, Jr.}{fdrake@acm.org}


The \module{dbhash} module provides a function to open databases using
the BSD \code{db} library.  This module mirrors the interface of the
other Python database modules that provide access to DBM-style
databases.  The \refmodule{bsddb}\refbimodindex{bsddb} module is required 
to use \module{dbhash}.

This module provides an exception and a function:


\begin{excdesc}{error}
  Exception raised on database errors other than
  \exception{KeyError}.  It is a synonym for \exception{bsddb.error}.
\end{excdesc}

\begin{funcdesc}{open}{path\optional{, flag\optional{, mode}}}
  Open a \code{db} database and return the database object.  The
  \var{path} argument is the name of the database file.

  The \var{flag} argument can be
  \code{'r'} (the default), \code{'w'},
  \code{'c'} (which creates the database if it doesn't exist), or
  \code{'n'} (which always creates a new empty database).
  For platforms on which the BSD \code{db} library supports locking,
  an \character{l} can be appended to indicate that locking should be
  used.

  The optional \var{mode} parameter is used to indicate the \UNIX{}
  permission bits that should be set if a new database must be
  created; this will be masked by the current umask value for the
  process.
\end{funcdesc}


\begin{seealso}
  \seemodule{anydbm}{Generic interface to \code{dbm}-style databases.}
  \seemodule{bsddb}{Lower-level interface to the BSD \code{db} library.}
  \seemodule{whichdb}{Utility module used to determine the type of an
                      existing database.}
\end{seealso}


\subsection{Database Objects \label{dbhash-objects}}

The database objects returned by \function{open()} provide the methods 
common to all the DBM-style databases and mapping objects.  The following
methods are available in addition to the standard methods.

\begin{methoddesc}[dbhash]{first}{}
  It's possible to loop over every key/value pair in the database using
  this method   and the \method{next()} method.  The traversal is ordered by
  the databases internal hash values, and won't be sorted by the key
  values.  This method returns the starting key.
\end{methoddesc}

\begin{methoddesc}[dbhash]{last}{}
  Return the last key/value pair in a database traversal.  This may be used to
  begin a reverse-order traversal; see \method{previous()}.
\end{methoddesc}

\begin{methoddesc}[dbhash]{next}{}
  Returns the key next key/value pair in a database traversal.  The
  following code prints every key in the database \code{db}, without
  having to create a list in memory that contains them all:

\begin{verbatim}
print db.first()
for i in xrange(1, len(db)):
    print db.next()
\end{verbatim}
\end{methoddesc}

\begin{methoddesc}[dbhash]{previous}{}
  Returns the previous key/value pair in a forward-traversal of the database.
  In conjunction with \method{last()}, this may be used to implement
  a reverse-order traversal.
\end{methoddesc}

\begin{methoddesc}[dbhash]{sync}{}
  This method forces any unwritten data to be written to the disk.
\end{methoddesc}

\section{Standard Module \module{whichdb}}
\declaremodule{standard}{whichdb}

\modulesynopsis{Guess which DBM-style module created a given database.}


The single function in this module attempts to guess which of the
several simple database modules available--dbm, gdbm, or
dbhash--should be used to open a given file.

\begin{funcdesc}{whichdb}{filename}
Returns one of the following values: \code{None} if the file can't be
opened because it's unreadable or doesn't exist; the empty string
(\code{""}) if the file's format can't be guessed; or a string
containing the required module name, such as \code{"dbm"} or
\code{"gdbm"}.
\end{funcdesc}


\section{\module{bsddb} ---
         Interface to Berkeley DB library}

\declaremodule{extension}{bsddb}
\modulesynopsis{Interface to Berkeley DB database library}
\sectionauthor{Skip Montanaro}{skip@mojam.com}


The \module{bsddb} module provides an interface to the Berkeley DB
library.  Users can create hash, btree or record based library files
using the appropriate open call. Bsddb objects behave generally like
dictionaries.  Keys and values must be strings, however, so to use
other objects as keys or to store other kinds of objects the user must
serialize them somehow, typically using \function{marshal.dumps()} or 
\function{pickle.dumps()}.

The \module{bsddb} module requires a Berkeley DB library version from
3.3 thru 4.5.

\begin{seealso}
  \seeurl{http://pybsddb.sourceforge.net/}
         {The website with documentation for the \module{bsddb.db}
          Python Berkeley DB interface that closely mirrors the object
          oriented interface provided in Berkeley DB 3 and 4.}

  \seeurl{http://www.oracle.com/database/berkeley-db/}
         {The Berkeley DB library.}
\end{seealso}

A more modern DB, DBEnv and DBSequence object interface is available in the
\module{bsddb.db} module which closely matches the Berkeley DB C API
documented at the above URLs.  Additional features provided by the
\module{bsddb.db} API include fine tuning, transactions, logging, and
multiprocess concurrent database access.

The following is a description of the legacy \module{bsddb} interface
compatible with the old Python bsddb module.  Starting in Python 2.5 this
interface should be safe for multithreaded access.  The \module{bsddb.db}
API is recommended for threading users as it provides better control.

The \module{bsddb} module defines the following functions that create
objects that access the appropriate type of Berkeley DB file.  The
first two arguments of each function are the same.  For ease of
portability, only the first two arguments should be used in most
instances.

\begin{funcdesc}{hashopen}{filename\optional{, flag\optional{,
                           mode\optional{, pgsize\optional{,
                           ffactor\optional{, nelem\optional{,
                           cachesize\optional{, lorder\optional{,
                           hflags}}}}}}}}}
Open the hash format file named \var{filename}.  Files never intended
to be preserved on disk may be created by passing \code{None} as the 
\var{filename}.  The optional
\var{flag} identifies the mode used to open the file.  It may be
\character{r} (read only), \character{w} (read-write) ,
\character{c} (read-write - create if necessary; the default) or
\character{n} (read-write - truncate to zero length).  The other
arguments are rarely used and are just passed to the low-level
\cfunction{dbopen()} function.  Consult the Berkeley DB documentation
for their use and interpretation.
\end{funcdesc}

\begin{funcdesc}{btopen}{filename\optional{, flag\optional{,
mode\optional{, btflags\optional{, cachesize\optional{, maxkeypage\optional{,
minkeypage\optional{, pgsize\optional{, lorder}}}}}}}}}

Open the btree format file named \var{filename}.  Files never intended 
to be preserved on disk may be created by passing \code{None} as the 
\var{filename}.  The optional
\var{flag} identifies the mode used to open the file.  It may be
\character{r} (read only), \character{w} (read-write),
\character{c} (read-write - create if necessary; the default) or
\character{n} (read-write - truncate to zero length).  The other
arguments are rarely used and are just passed to the low-level dbopen
function.  Consult the Berkeley DB documentation for their use and
interpretation.
\end{funcdesc}

\begin{funcdesc}{rnopen}{filename\optional{, flag\optional{, mode\optional{,
rnflags\optional{, cachesize\optional{, pgsize\optional{, lorder\optional{,
rlen\optional{, delim\optional{, source\optional{, pad}}}}}}}}}}}

Open a DB record format file named \var{filename}.  Files never intended 
to be preserved on disk may be created by passing \code{None} as the 
\var{filename}.  The optional
\var{flag} identifies the mode used to open the file.  It may be
\character{r} (read only), \character{w} (read-write),
\character{c} (read-write - create if necessary; the default) or
\character{n} (read-write - truncate to zero length).  The other
arguments are rarely used and are just passed to the low-level dbopen
function.  Consult the Berkeley DB documentation for their use and
interpretation.
\end{funcdesc}


\begin{notice}
Beginning in 2.3 some \UNIX{} versions of Python may have a \module{bsddb185}
module.  This is present \emph{only} to allow backwards compatibility with
systems which ship with the old Berkeley DB 1.85 database library.  The
\module{bsddb185} module should never be used directly in new code.
\end{notice}


\begin{seealso}
  \seemodule{dbhash}{DBM-style interface to the \module{bsddb}}
\end{seealso}

\subsection{Hash, BTree and Record Objects \label{bsddb-objects}}

Once instantiated, hash, btree and record objects support
the same methods as dictionaries.  In addition, they support
the methods listed below.
\versionchanged[Added dictionary methods]{2.3.1}

\begin{methoddesc}{close}{}
Close the underlying file.  The object can no longer be accessed.  Since
there is no open \method{open} method for these objects, to open the file
again a new \module{bsddb} module open function must be called.
\end{methoddesc}

\begin{methoddesc}{keys}{}
Return the list of keys contained in the DB file.  The order of the list is
unspecified and should not be relied on.  In particular, the order of the
list returned is different for different file formats.
\end{methoddesc}

\begin{methoddesc}{has_key}{key}
Return \code{1} if the DB file contains the argument as a key.
\end{methoddesc}

\begin{methoddesc}{set_location}{key}
Set the cursor to the item indicated by \var{key} and return a tuple
containing the key and its value.  For binary tree databases (opened
using \function{btopen()}), if \var{key} does not actually exist in
the database, the cursor will point to the next item in sorted order
and return that key and value.  For other databases,
\exception{KeyError} will be raised if \var{key} is not found in the
database.
\end{methoddesc}

\begin{methoddesc}{first}{}
Set the cursor to the first item in the DB file and return it.  The order of 
keys in the file is unspecified, except in the case of B-Tree databases.
This method raises \exception{bsddb.error} if the database is empty.
\end{methoddesc}

\begin{methoddesc}{next}{}
Set the cursor to the next item in the DB file and return it.  The order of 
keys in the file is unspecified, except in the case of B-Tree databases.
\end{methoddesc}

\begin{methoddesc}{previous}{}
Set the cursor to the previous item in the DB file and return it.  The
order of keys in the file is unspecified, except in the case of B-Tree
databases.  This is not supported on hashtable databases (those opened
with \function{hashopen()}).
\end{methoddesc}

\begin{methoddesc}{last}{}
Set the cursor to the last item in the DB file and return it.  The
order of keys in the file is unspecified.  This is not supported on
hashtable databases (those opened with \function{hashopen()}).
This method raises \exception{bsddb.error} if the database is empty.
\end{methoddesc}

\begin{methoddesc}{sync}{}
Synchronize the database on disk.
\end{methoddesc}

Example:

\begin{verbatim}
>>> import bsddb
>>> db = bsddb.btopen('/tmp/spam.db', 'c')
>>> for i in range(10): db['%d'%i] = '%d'% (i*i)
... 
>>> db['3']
'9'
>>> db.keys()
['0', '1', '2', '3', '4', '5', '6', '7', '8', '9']
>>> db.first()
('0', '0')
>>> db.next()
('1', '1')
>>> db.last()
('9', '81')
>>> db.set_location('2')
('2', '4')
>>> db.previous() 
('1', '1')
>>> for k, v in db.iteritems():
...     print k, v
0 0
1 1
2 4
3 9
4 16
5 25
6 36
7 49
8 64
9 81
>>> '8' in db
True
>>> db.sync()
0
\end{verbatim}

% XXX The module has been extended (by Jeremy) but this documentation
% hasn't been updated yet

\section{Built-in Module \module{zlib}}
\label{module-zlib}
\bimodindex{zlib}

For applications that require data compression, the functions in this
module allow compression and decompression, using the zlib library,
which is based on the GNU \program{gzip} program.  The zlib library
has its own home page at
\url{http://www.cdrom.com/pub/infozip/zlib/}.  Version 1.0.4 is the
most recent version as of December, 1997; use a later version if one
is available.

The available exception and functions in this module are:

\begin{excdesc}{error}
  Exception raised on compression and decompression errors.
\end{excdesc}


\begin{funcdesc}{adler32}{string\optional{, value}}
   Computes a Adler-32 checksum of \var{string}.  (An Adler-32
   checksum is almost as reliable as a CRC32 but can be computed much
   more quickly.)  If \var{value} is present, it is used as the
   starting value of the checksum; otherwise, a fixed default value is
   used.  This allows computing a running checksum over the
   concatenation of several input strings.  The algorithm is not
   cryptographically strong, and should not be used for
   authentication or digital signatures.
\end{funcdesc}

\begin{funcdesc}{compress}{string\optional{, level}}
Compresses the data in \var{string}, returning a string contained
compressed data.  \var{level} is an integer from \code{1} to \code{9}
controlling the level of compression; \code{1} is fastest and produces
the least compression, \code{9} is slowest and produces the most.  The
default value is \code{6}.  Raises the \exception{error}
exception if any error occurs.
\end{funcdesc}

\begin{funcdesc}{compressobj}{\optional{level}}
  Returns a compression object, to be used for compressing data streams
  that won't fit into memory at once.  \var{level} is an integer from
  \code{1} to \code{9} controlling the level of compression; \code{1} is
  fastest and produces the least compression, \code{9} is slowest and
  produces the most.  The default value is \code{6}.
\end{funcdesc}

\begin{funcdesc}{crc32}{string\optional{, value}}
  Computes a CRC (Cyclic Redundancy Check)%
  \index{Cyclic Redundancy Check}
  \index{checksum!Cyclic Redundancy Check}
  checksum of \var{string}. If
  \var{value} is present, it is used as the starting value of the
  checksum; otherwise, a fixed default value is used.  This allows
  computing a running checksum over the concatenation of several
  input strings.  The algorithm is not cryptographically strong, and
  should not be used for authentication or digital signatures.
\end{funcdesc}

\begin{funcdesc}{decompress}{string}
Decompresses the data in \var{string}, returning a string containing
the uncompressed data.  Raises the \exception{error} exception if any
error occurs.
\end{funcdesc}

\begin{funcdesc}{decompressobj}{\optional{wbits}}
Returns a compression object, to be used for decompressing data streams
that won't fit into memory at once.  The \var{wbits} parameter
controls the size of the window buffer; usually this can be left
alone.
\end{funcdesc}

Compression objects support the following methods:

\begin{methoddesc}[Compress]{compress}{string}
Compress \var{string}, returning a string containing compressed data
for at least part of the data in \var{string}.  This data should be
concatenated to the output produced by any preceding calls to the
\method{compress()} method.  Some input may be kept in internal buffers
for later processing.
\end{methoddesc}

\begin{methoddesc}[Compress]{flush}{}
All pending input is processed, and an string containing the remaining
compressed output is returned.  After calling \method{flush()}, the
\method{compress()} method cannot be called again; the only realistic
action is to delete the object.
\end{methoddesc}

Decompression objects support the following methods:

\begin{methoddesc}[Decompress]{decompress}{string}
Decompress \var{string}, returning a string containing the
uncompressed data corresponding to at least part of the data in
\var{string}.  This data should be concatenated to the output produced
by any preceding calls to the
\method{decompress()} method.  Some of the input data may be preserved
in internal buffers for later processing.
\end{methoddesc}

\begin{methoddesc}[Decompress]{flush}{}
All pending input is processed, and a string containing the remaining
uncompressed output is returned.  After calling \method{flush()}, the
\method{decompress()} method cannot be called again; the only realistic
action is to delete the object.
\end{methoddesc}

\begin{seealso}
\seemodule{gzip}{reading and writing \program{gzip}-format files}
\end{seealso}



\section{Built-in Module \sectcode{gzip}}
\label{module-gzip}
\bimodindex{gzip}

The data compression provided by the \code{zlib} module is compatible
with that used by the GNU compression program \file{gzip}.
Accordingly, the \code{gzip} module provides the \code{GzipFile} class
to read and write \file{gzip}-format files, automatically compressing
or decompressing the data so it looks like an ordinary file object.

\code{GzipFile} objects simulate most of the methods of a file
object, though it's not possible to use the \code{seek()} and
\code{tell()} methods to access the file randomly.

\setindexsubitem{(in module gzip)}
\begin{funcdesc}{open}{fileobj\optional{\, filename\optional{\, mode\, compresslevel}}}
  Returns a new \code{GzipFile} object on top of \var{fileobj}, which
  can be a regular file, a \code{StringIO} object, or any object which
  simulates a file.

  The \file{gzip} file format includes the original filename of the
  uncompressed file; when opening a \code{GzipFile} object for
  writing, it can be set by the \var{filename} argument.  The default
  value is an empty string.

  \var{mode} can be either \code{'r'} or \code{'w'} depending on
  whether the file will be read or written.  \var{compresslevel} is an
  integer from 1 to 9 controlling the level of compression; 1 is
  fastest and produces the least compression, and 9 is slowest and
  produces the most compression.  The default value of
  \var{compresslevel} is 9.

  Calling a \code{GzipFile} object's \code{close()} method does not
  close \var{fileobj}, since you might wish to append more material
  after the compressed data.  This also allows you to pass a
  \code{StringIO} object opened for writing as \var{fileobj}, and
  retrieve the resulting memory buffer using the \code{StringIO}
  object's \code{getvalue()} method.
\end{funcdesc}

\begin{seealso}
\seemodule{zlib}{the basic data compression module}
\end{seealso}


\section{\module{rlcompleter} ---
         Completion function for GNU readline}

\declaremodule{standard}{rlcompleter}
  \platform{Unix}
\sectionauthor{Moshe Zadka}{moshez@zadka.site.co.il}
\modulesynopsis{Python identifier completion for the GNU readline library.}

The \module{rlcompleter} module defines a completion function for
the \refmodule{readline} module by completing valid Python identifiers
and keywords.

This module is \UNIX-specific due to it's dependence on the
\refmodule{readline} module.

The \module{rlcompleter} module defines the \class{Completer} class.

Example:

\begin{verbatim}
>>> import rlcompleter
>>> import readline
>>> readline.parse_and_bind("tab: complete")
>>> readline. <TAB PRESSED>
readline.__doc__          readline.get_line_buffer  readline.read_init_file
readline.__file__         readline.insert_text      readline.set_completer
readline.__name__         readline.parse_and_bind
>>> readline.
\end{verbatim}

The \module{rlcompleter} module is designed for use with Python's
interactive mode.  A user can add the following lines to his or her
initialization file (identified by the \envvar{PYTHONSTARTUP}
environment variable) to get automatic \kbd{Tab} completion:

\begin{verbatim}
try:
    import readline
except ImportError:
    print "Module readline not available."
else:
    import rlcompleter
    readline.parse_and_bind("tab: complete")
\end{verbatim}


\subsection{Completer Objects \label{completer-objects}}

Completer objects have the following method:

\begin{methoddesc}[Completer]{complete}{text, state}
Return the \var{state}th completion for \var{text}.

If called for \var{text} that doesn't include a period character
(\character{.}), it will complete from names currently defined in
\refmodule[main]{__main__}, \refmodule[builtin]{__builtin__} and
keywords (as defined by the \refmodule{keyword} module).

If called for a dotted name, it will try to evaluate anything without
obvious side-effects (i.e., functions will not be evaluated, but it
can generate calls to \method{__getattr__()}) upto the last part, and
find matches for the rest via the \function{dir()} function.
\end{methoddesc}


\chapter{Unix Specific Services}
\label{unix}

The modules described in this chapter provide interfaces to features
that are unique to the \UNIX{} operating system, or in some cases to
some or many variants of it.  Here's an overview:

\localmoduletable
			% UNIX Specific Services
\section{\module{posix} ---
         The most common \POSIX{} system calls}

\declaremodule{builtin}{posix}
  \platform{Unix}
\modulesynopsis{The most common \POSIX\ system calls (normally used
                via module \refmodule{os}).}


This module provides access to operating system functionality that is
standardized by the C Standard and the \POSIX{} standard (a thinly
disguised \UNIX{} interface).

\strong{Do not import this module directly.}  Instead, import the
module \refmodule{os}, which provides a \emph{portable} version of this
interface.  On \UNIX{}, the \refmodule{os} module provides a superset of
the \module{posix} interface.  On non-\UNIX{} operating systems the
\module{posix} module is not available, but a subset is always
available through the \refmodule{os} interface.  Once \refmodule{os} is
imported, there is \emph{no} performance penalty in using it instead
of \module{posix}.  In addition, \refmodule{os}\refstmodindex{os}
provides some additional functionality, such as automatically calling
\function{putenv()} when an entry in \code{os.environ} is changed.

The descriptions below are very terse; refer to the corresponding
\UNIX{} manual (or \POSIX{} documentation) entry for more information.
Arguments called \var{path} refer to a pathname given as a string.

Errors are reported as exceptions; the usual exceptions are given for
type errors, while errors reported by the system calls raise
\exception{error} (a synonym for the standard exception
\exception{OSError}), described below.


\subsection{Large File Support \label{posix-large-files}}
\sectionauthor{Steve Clift}{clift@mail.anacapa.net}
\index{large files}
\index{file!large files}


Several operating systems (including AIX, HPUX, Irix and Solaris)
provide support for files that are larger than 2 Gb from a C
programming model where \ctype{int} and \ctype{long} are 32-bit
values. This is typically accomplished by defining the relevant size
and offset types as 64-bit values. Such files are sometimes referred
to as \dfn{large files}.

Large file support is enabled in Python when the size of an
\ctype{off_t} is larger than a \ctype{long} and the \ctype{long long}
type is available and is at least as large as an \ctype{off_t}. Python
longs are then used to represent file sizes, offsets and other values
that can exceed the range of a Python int. It may be necessary to
configure and compile Python with certain compiler flags to enable
this mode. For example, it is enabled by default with recent versions
of Irix, but with Solaris 2.6 and 2.7 you need to do something like:

\begin{verbatim}
CFLAGS="`getconf LFS_CFLAGS`" OPT="-g -O2 $CFLAGS" \
        ./configure
\end{verbatim} % $ <-- bow to font-lock

On large-file-capable Linux systems, this might work:

\begin{verbatim}
CFLAGS='-D_LARGEFILE64_SOURCE -D_FILE_OFFSET_BITS=64' OPT="-g -O2 $CFLAGS" \
        ./configure
\end{verbatim} % $ <-- bow to font-lock


\subsection{Module Contents \label{posix-contents}}


Module \module{posix} defines the following data item:

\begin{datadesc}{environ}
A dictionary representing the string environment at the time the
interpreter was started. For example, \code{environ['HOME']} is the
pathname of your home directory, equivalent to
\code{getenv("HOME")} in C.

Modifying this dictionary does not affect the string environment
passed on by \function{execv()}, \function{popen()} or
\function{system()}; if you need to change the environment, pass
\code{environ} to \function{execve()} or add variable assignments and
export statements to the command string for \function{system()} or
\function{popen()}.

\note{The \refmodule{os} module provides an alternate
implementation of \code{environ} which updates the environment on
modification.  Note also that updating \code{os.environ} will render
this dictionary obsolete.  Use of the \refmodule{os} module version of
this is recommended over direct access to the \module{posix} module.}
\end{datadesc}

Additional contents of this module should only be accessed via the
\refmodule{os} module; refer to the documentation for that module for
further information.

\section{\module{pwd} ---
         The password database}

\declaremodule{builtin}{pwd}
  \platform{Unix}
\modulesynopsis{The password database (\function{getpwnam()} and friends).}

This module provides access to the \UNIX{} user account and password
database.  It is available on all \UNIX{} versions.

Password database entries are reported as a tuple-like object, whose
attributes correspond to the members of the \code{passwd} structure
(Attribute field below, see \code{<pwd.h>}):

\begin{tableiii}{r|l|l}{textrm}{Index}{Attribute}{Meaning}
  \lineiii{0}{\code{pw_name}}{Login name}
  \lineiii{1}{\code{pw_passwd}}{Optional encrypted password}
  \lineiii{2}{\code{pw_uid}}{Numerical user ID}
  \lineiii{3}{\code{pw_gid}}{Numerical group ID}
  \lineiii{4}{\code{pw_gecos}}{User name or comment field}
  \lineiii{5}{\code{pw_dir}}{User home directory}
  \lineiii{6}{\code{pw_shell}}{User command interpreter}
\end{tableiii}

The uid and gid items are integers, all others are strings.
\exception{KeyError} is raised if the entry asked for cannot be found.

\note{In traditional \UNIX{} the field \code{pw_passwd} usually
contains a password encrypted with a DES derived algorithm (see module
\refmodule{crypt}\refbimodindex{crypt}).  However most modern unices 
use a so-called \emph{shadow password} system.  On those unices the
\var{pw_passwd} field only contains an asterisk (\code{'*'}) or the 
letter \character{x} where the encrypted password is stored in a file
\file{/etc/shadow} which is not world readable.  Whether the \var{pw_passwd}
field contains anything useful is system-dependent.}

It defines the following items:

\begin{funcdesc}{getpwuid}{uid}
Return the password database entry for the given numeric user ID.
\end{funcdesc}

\begin{funcdesc}{getpwnam}{name}
Return the password database entry for the given user name.
\end{funcdesc}

\begin{funcdesc}{getpwall}{}
Return a list of all available password database entries, in arbitrary order.
\end{funcdesc}


\begin{seealso}
  \seemodule{grp}{An interface to the group database, similar to this.}
\end{seealso}

\section{Built-in Module \module{grp}}
\declaremodule{builtin}{grp}


\modulesynopsis{The group database (\function{getgrnam()} and friends).}

This module provides access to the \UNIX{} group database.
It is available on all \UNIX{} versions.

Group database entries are reported as 4-tuples containing the
following items from the group database (see \file{<grp.h>}), in order:
\code{gr_name},
\code{gr_passwd},
\code{gr_gid},
\code{gr_mem}.
The gid is an integer, name and password are strings, and the member
list is a list of strings.
(Note that most users are not explicitly listed as members of the
group they are in according to the password database.)
A \code{KeyError} exception is raised if the entry asked for cannot be found.

It defines the following items:

\begin{funcdesc}{getgrgid}{gid}
Return the group database entry for the given numeric group ID.
\end{funcdesc}

\begin{funcdesc}{getgrnam}{name}
Return the group database entry for the given group name.
\end{funcdesc}

\begin{funcdesc}{getgrall}{}
Return a list of all available group entries, in arbitrary order.
\end{funcdesc}

\section{Built-in Module \sectcode{crypt}}
\label{module-crypt}
\bimodindex{crypt}

This module implements an interface to the \manpage{crypt}{3} routine,
which is a one-way hash function based upon a modified DES algorithm;
see the \UNIX{} man page for further details.  Possible uses include
allowing Python scripts to accept typed passwords from the user, or
attempting to crack \UNIX{} passwords with a dictionary.
\index{crypt(3)}

\setindexsubitem{(in module crypt)}
\begin{funcdesc}{crypt}{word\, salt} 
\var{word} will usually be a user's password.  \var{salt} is a
2-character string which will be used to select one of 4096 variations
of DES\indexii{cipher}{DES}.  The characters in \var{salt} must be
either \code{.}, \code{/}, or an alphanumeric character.  Returns the
hashed password as a string, which will be composed of characters from
the same alphabet as the salt.
\end{funcdesc}

The module and documentation were written by Steve Majewski.
\index{Majewski, Steve}

\section{\module{dl} ---
         Call C functions in shared objects}
\declaremodule{extension}{dl}
  \platform{Unix} %?????????? Anyone????????????
\sectionauthor{Moshe Zadka}{moshez@zadka.site.co.il}
\modulesynopsis{Call C functions in shared objects.}

The \module{dl} module defines an interface to the
\cfunction{dlopen()} function, which is the most common interface on
\UNIX{} platforms for handling dynamically linked libraries. It allows
the program to call arbitrary functions in such a library.

\warning{The \module{dl} module bypasses the Python type system and 
error handling. If used incorrectly it may cause segmentation faults,
crashes or other incorrect behaviour.}

\note{This module will not work unless
\code{sizeof(int) == sizeof(long) == sizeof(char *)}
If this is not the case, \exception{SystemError} will be raised on
import.}

The \module{dl} module defines the following function:

\begin{funcdesc}{open}{name\optional{, mode\code{ = RTLD_LAZY}}}
Open a shared object file, and return a handle. Mode
signifies late binding (\constant{RTLD_LAZY}) or immediate binding
(\constant{RTLD_NOW}). Default is \constant{RTLD_LAZY}. Note that some
systems do not support \constant{RTLD_NOW}.

Return value is a \class{dlobject}.
\end{funcdesc}

The \module{dl} module defines the following constants:

\begin{datadesc}{RTLD_LAZY}
Useful as an argument to \function{open()}.
\end{datadesc}

\begin{datadesc}{RTLD_NOW}
Useful as an argument to \function{open()}.  Note that on systems
which do not support immediate binding, this constant will not appear
in the module. For maximum portability, use \function{hasattr()} to
determine if the system supports immediate binding.
\end{datadesc}

The \module{dl} module defines the following exception:

\begin{excdesc}{error}
Exception raised when an error has occurred inside the dynamic loading
and linking routines.
\end{excdesc}

Example:

\begin{verbatim}
>>> import dl, time
>>> a=dl.open('/lib/libc.so.6')
>>> a.call('time'), time.time()
(929723914, 929723914.498)
\end{verbatim}

This example was tried on a Debian GNU/Linux system, and is a good
example of the fact that using this module is usually a bad alternative.

\subsection{Dl Objects \label{dl-objects}}

Dl objects, as returned by \function{open()} above, have the
following methods:

\begin{methoddesc}{close}{}
Free all resources, except the memory.
\end{methoddesc}

\begin{methoddesc}{sym}{name}
Return the pointer for the function named \var{name}, as a number, if
it exists in the referenced shared object, otherwise \code{None}. This
is useful in code like:

\begin{verbatim}
>>> if a.sym('time'): 
...     a.call('time')
... else: 
...     time.time()
\end{verbatim}

(Note that this function will return a non-zero number, as zero is the
\NULL{} pointer)
\end{methoddesc}

\begin{methoddesc}{call}{name\optional{, arg1\optional{, arg2\ldots}}}
Call the function named \var{name} in the referenced shared object.
The arguments must be either Python integers, which will be 
passed as is, Python strings, to which a pointer will be passed, 
or \code{None}, which will be passed as \NULL.  Note that 
strings should only be passed to functions as \ctype{const char*}, as
Python will not like its string mutated.

There must be at most 10 arguments, and arguments not given will be
treated as \code{None}. The function's return value must be a C
\ctype{long}, which is a Python integer.
\end{methoddesc}

\section{Built-in Module \sectcode{dbm}}
\label{module-dbm}
\bimodindex{dbm}

The \code{dbm} module provides an interface to the \UNIX{}
\code{(n)dbm} library.  Dbm objects behave like mappings
(dictionaries), except that keys and values are always strings.
Printing a dbm object doesn't print the keys and values, and the
\code{items()} and \code{values()} methods are not supported.

See also the \code{gdbm} module, which provides a similar interface
using the GNU GDBM library.
\refbimodindex{gdbm}

The module defines the following constant and functions:

\renewcommand{\indexsubitem}{(in module dbm)}
\begin{excdesc}{error}
Raised on dbm-specific errors, such as I/O errors. \code{KeyError} is
raised for general mapping errors like specifying an incorrect key.
\end{excdesc}

\begin{funcdesc}{open}{filename\, \optional{flag\, \optional{mode}}}
Open a dbm database and return a dbm object.  The \var{filename}
argument is the name of the database file (without the \file{.dir} or
\file{.pag} extensions).

The optional \var{flag} argument can be
\code{'r'} (to open an existing database for reading only --- default),
\code{'w'} (to open an existing database for reading and writing),
\code{'c'} (which creates the database if it doesn't exist), or
\code{'n'} (which always creates a new empty database).

The optional \var{mode} argument is the \UNIX{} mode of the file, used
only when the database has to be created.  It defaults to octal
\code{0666}.
\end{funcdesc}

\section{Built-in Module \module{gdbm}}
\label{module-gdbm}
\bimodindex{gdbm}

% Note that if this section appears on the same page as the first
% paragraph of the dbm module section, makeindex will produce the
% warning:
%
% ## Warning (input = lib.idx, line = 1184; output = lib.ind, line = 852):
%    -- Conflicting entries: multiple encaps for the same page under same key.
%
% This is because the \bimodindex{gdbm} and \refbimodindex{gdbm}
% entries in the .idx file are slightly different (the \bimodindex{}
% version includes "|textbf" at the end to make the defining occurance 
% bold).  There doesn't appear to be anything that can be done about
% this; it's just a little annoying.  The warning can be ignored, but
% the index produced uses the non-bold version.

This module is quite similar to the \code{dbm} module, but uses \code{gdbm}
instead to provide some additional functionality.  Please note that
the file formats created by \code{gdbm} and \code{dbm} are incompatible.
\refbimodindex{dbm}

The \code{gdbm} module provides an interface to the GNU DBM
library.  \code{gdbm} objects behave like mappings
(dictionaries), except that keys and values are always strings.
Printing a \code{gdbm} object doesn't print the keys and values, and the
\code{items()} and \code{values()} methods are not supported.

The module defines the following constant and functions:

\begin{excdesc}{error}
Raised on \code{gdbm}-specific errors, such as I/O errors. \code{KeyError} is
raised for general mapping errors like specifying an incorrect key.
\end{excdesc}

\begin{funcdesc}{open}{filename, \optional{flag, \optional{mode}}}
Open a \code{gdbm} database and return a \code{gdbm} object.  The
\var{filename} argument is the name of the database file.

The optional \var{flag} argument can be
\code{'r'} (to open an existing database for reading only --- default),
\code{'w'} (to open an existing database for reading and writing),
\code{'c'} (which creates the database if it doesn't exist), or
\code{'n'} (which always creates a new empty database).

Appending \code{f} to the flag opens the database in fast mode;
altered data will not automatically be written to the disk after every
change.  This results in faster writes to the database, but may result
in an inconsistent database if the program crashes while the database
is still open.  Use the \code{sync()} method to force any unwritten
data to be written to the disk.

The optional \var{mode} argument is the \UNIX{} mode of the file, used
only when the database has to be created.  It defaults to octal
\code{0666}.
\end{funcdesc}

In addition to the dictionary-like methods, \code{gdbm} objects have the
following methods:

\begin{funcdesc}{firstkey}{}
It's possible to loop over every key in the database using this method
and the \code{nextkey()} method.  The traversal is ordered by \code{gdbm}'s
internal hash values, and won't be sorted by the key values.  This
method returns the starting key.
\end{funcdesc}

\begin{funcdesc}{nextkey}{key}
Returns the key that follows \var{key} in the traversal.  The
following code prints every key in the database \code{db}, without having to
create a list in memory that contains them all:
\begin{verbatim}
k=db.firstkey()
while k!=None:
    print k
    k=db.nextkey(k)
\end{verbatim}
\end{funcdesc}

\begin{funcdesc}{reorganize}{}
If you have carried out a lot of deletions and would like to shrink
the space used by the \code{gdbm} file, this routine will reorganize the
database.  \code{gdbm} will not shorten the length of a database file except
by using this reorganization; otherwise, deleted file space will be
kept and reused as new (key,value) pairs are added.
\end{funcdesc}

\begin{funcdesc}{sync}{}
When the database has been opened in fast mode, this method forces any
unwritten data to be written to the disk.
\end{funcdesc}


\section{Built-in Module \sectcode{termios}}
\label{module-termios}
\bimodindex{termios}
\indexii{Posix}{I/O control}
\indexii{tty}{I/O control}

\renewcommand{\indexsubitem}{(in module termios)}

This module provides an interface to the Posix calls for tty I/O
control.  For a complete description of these calls, see the Posix or
\UNIX{} manual pages.  It is only available for those \UNIX{} versions
that support Posix \code{termios} style tty I/O control (and then
only if configured at installation time).

All functions in this module take a file descriptor \var{fd} as their
first argument.  This must be an integer file descriptor, such as
returned by \code{sys.stdin.fileno()}.

This module should be used in conjunction with the \code{TERMIOS}
module, which defines the relevant symbolic constants (see the next
section).

The module defines the following functions:

\begin{funcdesc}{tcgetattr}{fd}
Return a list containing the tty attributes for file descriptor
\var{fd}, as follows: \code{[\var{iflag}, \var{oflag}, \var{cflag},
\var{lflag}, \var{ispeed}, \var{ospeed}, \var{cc}]} where \var{cc} is
a list of the tty special characters (each a string of length 1,
except the items with indices \code{VMIN} and \code{VTIME}, which are
integers when these fields are defined).  The interpretation of the
flags and the speeds as well as the indexing in the \var{cc} array
must be done using the symbolic constants defined in the
\code{TERMIOS} module.
\end{funcdesc}

\begin{funcdesc}{tcsetattr}{fd\, when\, attributes}
Set the tty attributes for file descriptor \var{fd} from the
\var{attributes}, which is a list like the one returned by
\code{tcgetattr()}.  The \var{when} argument determines when the
attributes are changed: \code{TERMIOS.TCSANOW} to change immediately,
\code{TERMIOS.TCSADRAIN} to change after transmitting all queued
output, or \code{TERMIOS.TCSAFLUSH} to change after transmitting all
queued output and discarding all queued input.
\end{funcdesc}

\begin{funcdesc}{tcsendbreak}{fd\, duration}
Send a break on file descriptor \var{fd}.  A zero \var{duration} sends
a break for 0.25--0.5 seconds; a nonzero \var{duration} has a system
dependent meaning.
\end{funcdesc}

\begin{funcdesc}{tcdrain}{fd}
Wait until all output written to file descriptor \var{fd} has been
transmitted.
\end{funcdesc}

\begin{funcdesc}{tcflush}{fd\, queue}
Discard queued data on file descriptor \var{fd}.  The \var{queue}
selector specifies which queue: \code{TERMIOS.TCIFLUSH} for the input
queue, \code{TERMIOS.TCOFLUSH} for the output queue, or
\code{TERMIOS.TCIOFLUSH} for both queues.
\end{funcdesc}

\begin{funcdesc}{tcflow}{fd\, action}
Suspend or resume input or output on file descriptor \var{fd}.  The
\var{action} argument can be \code{TERMIOS.TCOOFF} to suspend output,
\code{TERMIOS.TCOON} to restart output, \code{TERMIOS.TCIOFF} to
suspend input, or \code{TERMIOS.TCION} to restart input.
\end{funcdesc}

\subsection{Example}
\nodename{termios Example}

Here's a function that prompts for a password with echoing turned off.
Note the technique using a separate \code{termios.tcgetattr()} call
and a \code{try \ldots{} finally} statement to ensure that the old tty
attributes are restored exactly no matter what happens:

\bcode\begin{verbatim}
def getpass(prompt = "Password: "):
    import termios, TERMIOS, sys
    fd = sys.stdin.fileno()
    old = termios.tcgetattr(fd)
    new = termios.tcgetattr(fd)
    new[3] = new[3] & \~TERMIOS.ECHO          # lflags
    try:
        termios.tcsetattr(fd, TERMIOS.TCSADRAIN, new)
        passwd = raw_input(prompt)
    finally:
        termios.tcsetattr(fd, TERMIOS.TCSADRAIN, old)
    return passwd
\end{verbatim}\ecode
%
\section{Standard Module \sectcode{TERMIOS}}
\stmodindex{TERMIOS}
\indexii{Posix}{I/O control}
\indexii{tty}{I/O control}

\renewcommand{\indexsubitem}{(in module TERMIOS)}

This module defines the symbolic constants required to use the
\code{termios} module (see the previous section).  See the Posix or
\UNIX{} manual pages (or the source) for a list of those constants.
\refbimodindex{termios}

Note: this module resides in a system-dependent subdirectory of the
Python library directory.  You may have to generate it for your
particular system using the script \file{Tools/scripts/h2py.py}.

\section{\module{tty} ---
         Terminal control functions}

\declaremodule{standard}{tty}
  \platform{Unix}
\moduleauthor{Steen Lumholt}{}
\sectionauthor{Moshe Zadka}{moshez@zadka.site.co.il}
\modulesynopsis{Utility functions that perform common terminal control
                operations.}

The \module{tty} module defines functions for putting the tty into
cbreak and raw modes.

Because it requires the \refmodule{termios} module, it will work
only on \UNIX.

The \module{tty} module defines the following functions:

\begin{funcdesc}{setraw}{fd\optional{, when}}
Change the mode of the file descriptor \var{fd} to raw. If \var{when}
is omitted, it defaults to \constant{termios.TCSAFLUSH}, and is passed
to \function{termios.tcsetattr()}.
\end{funcdesc}

\begin{funcdesc}{setcbreak}{fd\optional{, when}}
Change the mode of file descriptor \var{fd} to cbreak. If \var{when}
is omitted, it defaults to \constant{termios.TCSAFLUSH}, and is passed
to \function{termios.tcsetattr()}.
\end{funcdesc}


\begin{seealso}
  \seemodule{termios}{Low-level terminal control interface.}
\end{seealso}

\section{\module{pty} ---
         Pseudo-terminal utilities}
\declaremodule{standard}{pty}
  \platform{IRIX, Linux}
\modulesynopsis{Pseudo-Terminal Handling for SGI and Linux.}
\moduleauthor{Steen Lumholt}{}
\sectionauthor{Moshe Zadka}{mzadka@geocities.com}


The \module{pty} module defines operations for handling the
pseudo-terminal concept: starting another process and being able to
write to and read from its controlling terminal programmatically.

Because pseudo-terminal handling is highly platform dependant, there
is code to do it only for SGI and Linux. (The Linux code is supposed
to work on other platforms, but hasn't been tested yet.)

The \module{pty} module defines the following functions:

\begin{funcdesc}{fork}{}
Fork. Connect the child's controlling terminal to a pseudo-terminal.
Return value is \code{(\var{pid}, \var{fd})}. Note that the child 
gets \var{pid} 0, and the \var{fd} is \emph{invalid}. The parent's
return value is the \var{pid} of the child, and \var{fd} is a file
descriptor connected to the child's controlling terminal (and also
to the child's standard input and output.
\end{funcdesc}

\begin{funcdesc}{openpty}{}
Open a new pseudo-terminal pair, using \function{os.openpty()} if
possible, or emulation code for SGI and generic \UNIX{} systems.
Return a pair of file descriptors \code{(\var{master}, \var{slave})},
for the master and the slave end, respectively.
\end{funcdesc}

\begin{funcdesc}{spawn}{argv\optional{, master_read\optional{, stdin_read}}}
Spawn a process, and connect its controlling terminal with the current 
process's standard io. This is often used to baffle programs which
insist on reading from the controlling terminal.

The functions \var{master_read} and \var{stdin_read} should be
functions which read from a file-descriptor. The defaults try to read
1024 bytes each time they are called.
\end{funcdesc}

\section{\module{fcntl} ---
         The \function{fcntl()} and \function{ioctl()} system calls}

\declaremodule{builtin}{fcntl}
  \platform{Unix}
\modulesynopsis{The \function{fcntl()} and \function{ioctl()} system calls.}
\sectionauthor{Jaap Vermeulen}{}

\indexii{UNIX@\UNIX{}}{file control}
\indexii{UNIX@\UNIX{}}{I/O control}

This module performs file control and I/O control on file descriptors.
It is an interface to the \cfunction{fcntl()} and \cfunction{ioctl()}
\UNIX{} routines.

All functions in this module take a file descriptor \var{fd} as their
first argument.  This can be an integer file descriptor, such as
returned by \code{sys.stdin.fileno()}, or a file object, such as
\code{sys.stdin} itself.

The module defines the following functions:


\begin{funcdesc}{fcntl}{fd, op\optional{, arg}}
  Perform the requested operation on file descriptor \var{fd}.
  The operation is defined by \var{op} and is operating system
  dependent.  These codes are also found in the \module{fcntl}
  module. The argument \var{arg} is optional, and defaults to the
  integer value \code{0}.  When present, it can either be an integer
  value, or a string.  With the argument missing or an integer value,
  the return value of this function is the integer return value of the
  C \cfunction{fcntl()} call.  When the argument is a string it
  represents a binary structure, e.g.\ created by
  \function{struct.pack()}. The binary data is copied to a buffer
  whose address is passed to the C \cfunction{fcntl()} call.  The
  return value after a successful call is the contents of the buffer,
  converted to a string object.  The length of the returned string
  will be the same as the length of the \var{arg} argument.  This is
  limited to 1024 bytes.  If the information returned in the buffer by
  the operating system is larger than 1024 bytes, this is most likely
  to result in a segmentation violation or a more subtle data
  corruption.

  If the \cfunction{fcntl()} fails, an \exception{IOError} is
  raised.
\end{funcdesc}

\begin{funcdesc}{ioctl}{fd, op, arg}
  This function is identical to the \function{fcntl()} function, except
  that the operations are typically defined in the library module
  \module{IOCTL}.
\end{funcdesc}

\begin{funcdesc}{flock}{fd, op}
Perform the lock operation \var{op} on file descriptor \var{fd}.
See the \UNIX{} manual \manpage{flock}{3} for details.  (On some
systems, this function is emulated using \cfunction{fcntl()}.)
\end{funcdesc}

\begin{funcdesc}{lockf}{fd, operation,
    \optional{len, \optional{start, \optional{whence}}}}
This is essentially a wrapper around the \function{fcntl()} locking
calls.  \var{fd} is the file descriptor of the file to lock or unlock,
and \var{operation} is one of the following values:

\begin{itemize}
\item \constant{LOCK_UN} -- unlock
\item \constant{LOCK_SH} -- acquire a shared lock
\item \constant{LOCK_EX} -- acquire an exclusive lock
\end{itemize}

When \var{operation} is \constant{LOCK_SH} or \constant{LOCK_EX}, it
can also be bit-wise OR'd with \constant{LOCK_NB} to avoid blocking on
lock acquisition.  If \constant{LOCK_NB} is used and the lock cannot
be acquired, an \exception{IOError} will be raised and the exception
will have an \var{errno} attribute set to \constant{EACCES} or
\constant{EAGAIN} (depending on the operating system; for portability,
check for both values).

\var{length} is the number of bytes to lock, \var{start} is the byte
offset at which the lock starts, relative to \var{whence}, and
\var{whence} is as with \function{fileobj.seek()}, specifically:

\begin{itemize}
\item \constant{0} -- relative to the start of the file
      (\constant{SEEK_SET})
\item \constant{1} -- relative to the current buffer position
      (\constant{SEEK_CUR})
\item \constant{2} -- relative to the end of the file
      (\constant{SEEK_END})
\end{itemize}

The default for \var{start} is 0, which means to start at the
beginning of the file.  The default for \var{length} is 0 which means
to lock to the end of the file.  The default for \var{whence} is also
0.
\end{funcdesc}

Examples (all on a SVR4 compliant system):

\begin{verbatim}
import struct, fcntl

file = open(...)
rv = fcntl(file, fcntl.F_SETFL, os.O_NDELAY)

lockdata = struct.pack('hhllhh', fcntl.F_WRLCK, 0, 0, 0, 0, 0)
rv = fcntl.fcntl(file, fcntl.F_SETLKW, lockdata)
\end{verbatim}

Note that in the first example the return value variable \var{rv} will
hold an integer value; in the second example it will hold a string
value.  The structure lay-out for the \var{lockdata} variable is
system dependent --- therefore using the \function{flock()} call may be
better.

\section{\module{pipes} ---
         Interface to shell pipelines}

\declaremodule{standard}{pipes}
  \platform{Unix}
\sectionauthor{Moshe Zadka}{mzadka@geocities.com}
\modulesynopsis{A Python interface to \UNIX{} shell pipelines.}


The \module{pipes} module defines a class to abstract the concept of
a \emph{pipeline} --- a sequence of convertors from one file to 
another.

Because the module uses \program{/bin/sh} command lines, a \POSIX{} or
compatible shell for \function{os.system()} and \function{os.popen()}
is required.

The \module{pipes} module defines the following class:

\begin{classdesc}{Template}{}
An abstraction of a pipeline.
\end{classdesc}

Example:

\begin{verbatim}
>>> import pipes
>>> t=pipes.Template()
>>> t.append('tr a-z A-Z', '--')
>>> f=t.open('/tmp/1', 'w')
>>> f.write('hello world')
>>> f.close()
>>> open('/tmp/1').read()
'HELLO WORLD'
\end{verbatim}


\subsection{Template Objects \label{template-objects}}

Template objects following methods:

\begin{methoddesc}{reset}{}
Restore a pipeline template to its initial state.
\end{methoddesc}

\begin{methoddesc}{clone}{}
Return a new, equivalent, pipeline template.
\end{methoddesc}

\begin{methoddesc}{debug}{flag}
If \var{flag} is true, turn debugging on. Otherwise, turn debugging
off. When debugging is on, commands to be executed are printed, and
the shell is given \code{set -x} command to be more verbose.
\end{methoddesc}

\begin{methoddesc}{append}{cmd, kind}
Append a new action at the end. The \var{cmd} variable must be a valid
bourne shell command. The \var{kind} variable consists of two letters.

The first letter can be either of \code{'-'} (which means the command
reads its standard input), \code{'f'} (which means the commands reads
a given file on the command line) or \code{'.'} (which means the commands
reads no input, and hence must be first.)

Similarily, the second letter can be either of \code{'-'} (which means 
the command writes to standard output), \code{'f'} (which means the 
command writes a file on the command line) or \code{'.'} (which means
the command does not write anything, and hence must be last.)
\end{methoddesc}

\begin{methoddesc}{prepend}{cmd, kind}
Add a new action at the beginning. See \method{append()} for explanations
of the arguments.
\end{methoddesc}

\begin{methoddesc}{open}{file, mode}
Return a file-like object, open to \var{file}, but read from or
written to by the pipeline.  Note that only one of \code{'r'},
\code{'w'} may be given.
\end{methoddesc}

\begin{methoddesc}{copy}{infile, outfile}
Copy \var{infile} to \var{outfile} through the pipe.
\end{methoddesc}

% Manual text and implementation by Jaap Vermeulen
\section{Standard Module \sectcode{posixfile}}
\label{module-posixfile}
\bimodindex{posixfile}
\indexii{\POSIX{}}{file object}

\emph{Note:} This module will become obsolete in a future release.
The locking operation that it provides is done better and more
portably by the \function{fcntl.lockf()} call.%
\withsubitem{(in module fcntl)}{\ttindex{lockf()}}

This module implements some additional functionality over the built-in
file objects.  In particular, it implements file locking, control over
the file flags, and an easy interface to duplicate the file object.
The module defines a new file object, the posixfile object.  It
has all the standard file object methods and adds the methods
described below.  This module only works for certain flavors of
\UNIX{}, since it uses \function{fcntl.fcntl()} for file locking.%
\withsubitem{(in module fcntl)}{\ttindex{fcntl()}}

To instantiate a posixfile object, use the \function{open()} function
in the \module{posixfile} module.  The resulting object looks and
feels roughly the same as a standard file object.

The \module{posixfile} module defines the following constants:


\begin{datadesc}{SEEK_SET}
Offset is calculated from the start of the file.
\end{datadesc}

\begin{datadesc}{SEEK_CUR}
Offset is calculated from the current position in the file.
\end{datadesc}

\begin{datadesc}{SEEK_END}
Offset is calculated from the end of the file.
\end{datadesc}

The \module{posixfile} module defines the following functions:


\begin{funcdesc}{open}{filename\optional{, mode\optional{, bufsize}}}
 Create a new posixfile object with the given filename and mode.  The
 \var{filename}, \var{mode} and \var{bufsize} arguments are
 interpreted the same way as by the built-in \function{open()}
 function.
\end{funcdesc}

\begin{funcdesc}{fileopen}{fileobject}
 Create a new posixfile object with the given standard file object.
 The resulting object has the same filename and mode as the original
 file object.
\end{funcdesc}

The posixfile object defines the following additional methods:

\setindexsubitem{(posixfile method)}
\begin{funcdesc}{lock}{fmt, \optional{len\optional{, start\optional{, whence}}}}
 Lock the specified section of the file that the file object is
 referring to.  The format is explained
 below in a table.  The \var{len} argument specifies the length of the
 section that should be locked. The default is \code{0}. \var{start}
 specifies the starting offset of the section, where the default is
 \code{0}.  The \var{whence} argument specifies where the offset is
 relative to. It accepts one of the constants \constant{SEEK_SET},
 \constant{SEEK_CUR} or \constant{SEEK_END}.  The default is
 \constant{SEEK_SET}.  For more information about the arguments refer
 to the \manpage{fcntl}{2} manual page on your system.
\end{funcdesc}

\begin{funcdesc}{flags}{\optional{flags}}
 Set the specified flags for the file that the file object is referring
 to.  The new flags are ORed with the old flags, unless specified
 otherwise.  The format is explained below in a table.  Without
 the \var{flags} argument
 a string indicating the current flags is returned (this is
 the same as the \samp{?} modifier).  For more information about the
 flags refer to the \manpage{fcntl}{2} manual page on your system.
\end{funcdesc}

\begin{funcdesc}{dup}{}
 Duplicate the file object and the underlying file pointer and file
 descriptor.  The resulting object behaves as if it were newly
 opened.
\end{funcdesc}

\begin{funcdesc}{dup2}{fd}
 Duplicate the file object and the underlying file pointer and file
 descriptor.  The new object will have the given file descriptor.
 Otherwise the resulting object behaves as if it were newly opened.
\end{funcdesc}

\begin{funcdesc}{file}{}
 Return the standard file object that the posixfile object is based
 on.  This is sometimes necessary for functions that insist on a
 standard file object.
\end{funcdesc}

All methods raise \exception{IOError} when the request fails.

Format characters for the \method{lock()} method have the following
meaning:

\begin{tableii}{|c|l|}{samp}{Format}{Meaning}
  \lineii{u}{unlock the specified region}
  \lineii{r}{request a read lock for the specified section}
  \lineii{w}{request a write lock for the specified section}
\end{tableii}

In addition the following modifiers can be added to the format:

\begin{tableiii}{|c|l|c|}{samp}{Modifier}{Meaning}{Notes}
  \lineiii{|}{wait until the lock has been granted}{}
  \lineiii{?}{return the first lock conflicting with the requested lock, or
              \code{None} if there is no conflict.}{(1)} 
\end{tableiii}

Note:

(1) The lock returned is in the format \code{(\var{mode}, \var{len},
\var{start}, \var{whence}, \var{pid})} where \var{mode} is a character
representing the type of lock ('r' or 'w').  This modifier prevents a
request from being granted; it is for query purposes only.

Format characters for the \method{flags()} method have the following
meanings:

\begin{tableii}{|c|l|}{samp}{Format}{Meaning}
  \lineii{a}{append only flag}
  \lineii{c}{close on exec flag}
  \lineii{n}{no delay flag (also called non-blocking flag)}
  \lineii{s}{synchronization flag}
\end{tableii}

In addition the following modifiers can be added to the format:

\begin{tableiii}{|c|l|c|}{samp}{Modifier}{Meaning}{Notes}
  \lineiii{!}{turn the specified flags 'off', instead of the default 'on'}{(1)}
  \lineiii{=}{replace the flags, instead of the default 'OR' operation}{(1)}
  \lineiii{?}{return a string in which the characters represent the flags that
  are set.}{(2)}
\end{tableiii}

Note:

(1) The \code{!} and \code{=} modifiers are mutually exclusive.

(2) This string represents the flags after they may have been altered
by the same call.

Examples:

\begin{verbatim}
import posixfile

file = posixfile.open('/tmp/test', 'w')
file.lock('w|')
...
file.lock('u')
file.close()
\end{verbatim}

\section{Built-in Module \module{resource}}
\label{module-resource}

\bimodindex{resource}
This module provides basic mechanisms for measuring and controlling
system resources utilized by a program.

Symbolic constants are used to specify particular system resources and
to request usage information about either the current process or its
children.

A single exception is defined for errors:


\begin{excdesc}{error}
  The functions described below may raise this error if the underlying
  system call failures unexpectedly.
\end{excdesc}

\subsection{Resource Limits}

Resources usage can be limited using the \function{setrlimit()} function
described below. Each resource is controlled by a pair of limits: a
soft limit and a hard limit. The soft limit is the current limit, and
may be lowered or raised by a process over time. The soft limit can
never exceed the hard limit. The hard limit can be lowered to any
value greater than the soft limit, but not raised. (Only processes with
the effective UID of the super-user can raise a hard limit.)

The specific resources that can be limited are system dependent. They
are described in the \manpage{getrlimit}{2} man page.  The resources
listed below are supported when the underlying operating system
supports them; resources which cannot be checked or controlled by the
operating system are not defined in this module for those platforms.

\begin{funcdesc}{getrlimit}{resource}
  Returns a tuple \code{(\var{soft}, \var{hard})} with the current
  soft and hard limits of \var{resource}. Raises \exception{ValueError} if
  an invalid resource is specified, or \exception{error} if the
  underyling system call fails unexpectedly.
\end{funcdesc}

\begin{funcdesc}{setrlimit}{resource, limits}
  Sets new limits of consumption of \var{resource}. The \var{limits}
  argument must be a tuple \code{(\var{soft}, \var{hard})} of two
  integers describing the new limits. A value of \code{-1} can be used to
  specify the maximum possible upper limit.

  Raises \exception{ValueError} if an invalid resource is specified,
  if the new soft limit exceeds the hard limit, or if a process tries
  to raise its hard limit (unless the process has an effective UID of
  super-user).  Can also raise \exception{error} if the underyling
  system call fails.
\end{funcdesc}

These symbols define resources whose consumption can be controlled
using the \function{setrlimit()} and \function{getrlimit()} functions
described below. The values of these symbols are exactly the constants
used by \C{} programs.

The \UNIX{} man page for \manpage{getrlimit}{2} lists the available
resources.  Note that not all systems use the same symbol or same
value to denote the same resource.

\begin{datadesc}{RLIMIT_CORE}
  The maximum size (in bytes) of a core file that the current process
  can create.  This may result in the creation of a partial core file
  if a larger core would be required to contain the entire process
  image.
\end{datadesc}

\begin{datadesc}{RLIMIT_CPU}
  The maximum amount of CPU time (in seconds) that a process can
  use. If this limit is exceeded, a \constant{SIGXCPU} signal is sent to
  the process. (See the \module{signal} module documentation for
  information about how to catch this signal and do something useful,
  e.g. flush open files to disk.)
\end{datadesc}

\begin{datadesc}{RLIMIT_FSIZE}
  The maximum size of a file which the process may create.  This only
  affects the stack of the main thread in a multi-threaded process.
\end{datadesc}

\begin{datadesc}{RLIMIT_DATA}
  The maximum size (in bytes) of the process's heap.
\end{datadesc}

\begin{datadesc}{RLIMIT_STACK}
  The maximum size (in bytes) of the call stack for the current
  process.
\end{datadesc}

\begin{datadesc}{RLIMIT_RSS}
  The maximum resident set size that should be made available to the
  process.
\end{datadesc}

\begin{datadesc}{RLIMIT_NPROC}
  The maximum number of processes the current process may create.
\end{datadesc}

\begin{datadesc}{RLIMIT_NOFILE}
  The maximum number of open file descriptors for the current
  process.
\end{datadesc}

\begin{datadesc}{RLIMIT_OFILE}
  The BSD name for \constant{RLIMIT_NOFILE}.
\end{datadesc}

\begin{datadesc}{RLIMIT_MEMLOC}
  The maximm address space which may be locked in memory.
\end{datadesc}

\begin{datadesc}{RLIMIT_VMEM}
  The largest area of mapped memory which the process may occupy.
\end{datadesc}

\begin{datadesc}{RLIMIT_AS}
  The maximum area (in bytes) of address space which may be taken by
  the process.
\end{datadesc}

\subsection{Resource Usage}

These functiona are used to retrieve resource usage information:

\begin{funcdesc}{getrusage}{who}
  This function returns a large tuple that describes the resources
  consumed by either the current process or its children, as specified
  by the \var{who} parameter.  The \var{who} parameter should be
  specified using one of the \code{RUSAGE_*} constants described
  below.

  The elements of the return value each
  describe how a particular system resource has been used, e.g. amount
  of time spent running is user mode or number of times the process was
  swapped out of main memory. Some values are dependent on the clock
  tick internal, e.g. the amount of memory the process is using.

  The first two elements of the return value are floating point values
  representing the amount of time spent executing in user mode and the
  amount of time spent executing in system mode, respectively. The
  remaining values are integers. Consult the \manpage{getrusage}{2}
  man page for detailed information about these values. A brief
  summary is presented here:

\begin{tableii}{|r|l|}{code}{Offset}{Resource}
  \lineii{0}{time in user mode (float)}
  \lineii{1}{time in system mode (float)}
  \lineii{2}{maximum resident set size}
  \lineii{3}{shared memory size}
  \lineii{4}{unshared memory size}
  \lineii{5}{unshared stack size}
  \lineii{6}{page faults not requiring I/O}
  \lineii{7}{page faults requiring I/O}
  \lineii{8}{number of swap outs}
  \lineii{9}{block input operations}
  \lineii{10}{block output operations}
  \lineii{11}{messages sent}
  \lineii{12}{messages received}
  \lineii{13}{signals received}
  \lineii{14}{voluntary context switches}
  \lineii{15}{involuntary context switches}
\end{tableii}

  This function will raise a \exception{ValueError} if an invalid
  \var{who} parameter is specified. It may also raise
  \exception{error} exception in unusual circumstances.
\end{funcdesc}

\begin{funcdesc}{getpagesize}{}
  Returns the number of bytes in a system page. (This need not be the
  same as the hardware page size.) This function is useful for
  determining the number of bytes of memory a process is using. The
  third element of the tuple returned by \function{getrusage()} describes
  memory usage in pages; multiplying by page size produces number of
  bytes. 
\end{funcdesc}

The following \code{RUSAGE_*} symbols are passed to the
\function{getrusage()} function to specify which processes information
should be provided for.

\begin{datadesc}{RUSAGE_SELF}
  \constant{RUSAGE_SELF} should be used to
  request information pertaining only to the process itself.
\end{datadesc}

\begin{datadesc}{RUSAGE_CHILDREN}
  Pass to \function{getrusage()} to request resource information for
  child processes of the calling process.
\end{datadesc}

\begin{datadesc}{RUSAGE_BOTH}
  Pass to \function{getrusage()} to request resources consumed by both
  the current process and child processes.  May not be available on all
  systems.
\end{datadesc}

\section{\module{nis} ---
         Interface to Sun's NIS (Yellow Pages)}

\declaremodule{extension}{nis}
  \platform{UNIX}
\moduleauthor{Fred Gansevles}{Fred.Gansevles@cs.utwente.nl}
\sectionauthor{Moshe Zadka}{mzadka@geocities.com}
\modulesynopsis{Interface to Sun's N.I.S. (a.k.a. Yellow Pages) library.}

The \module{nis} module gives a thin wrapper around the NIS library, useful
for central administration of several hosts.

Because NIS exists only on \UNIX{} systems, this module is
only available for \UNIX{}.

The \module{nis} module defines the following functions:

\begin{funcdesc}{match}{key, mapname}
Return the match for \var{key} in map \var{mapname}, or raise an
error (\exception{nis.error}) if there is none.
Both should be strings, \var{key} is 8-bit clean.
Return value is an arbitary array of bytes (i.e., may contain \code{NULL}
and other joys).

Note that \var{mapname} is first checked if it is an alias to another name.
XXX Describe list of all aliases? Internal for the C code, so
    I'm not sure it's a good idea.
\end{funcdesc}

\begin{funcdesc}{cat}{mapname}
Return a dictionary mapping \var{key} to \var{value} such that
\code{match(\var{key}, \var{mapname})==\var{value}}.
Note that both keys and values of the dictionary are arbitary
arrays of bytes.

Note that \var{mapname} is first checked if it is an alias to another name.
\end{funcdesc}

\begin{funcdesc}{maps}{}
Return a list of all valid maps.
\end{funcdesc}


The \module{nis} module defines the following exception:

\begin{excdesc}{error}
An error raised when a NIS function returns an error code.
\end{excdesc}

\section{Built-in Module \sectcode{syslog}}
\bimodindex{syslog}

This module provides an interface to the Unix \code{syslog} library
routines.  Refer to the \UNIX{} manual pages for a detailed description
of the \code{syslog} facility.

The module defines the following functions:

\begin{funcdesc}{syslog}{\optional{priority\,} message}
Send the string \var{message} to the system logger.
A trailing newline is added if necessary.
Each message is tagged with a priority composed of a \var{facility} and
a \var{level}.
The optional \var{priority} argument, which defaults to
\code{(LOG_USER | LOG_INFO)}, determines the message priority.
\end{funcdesc}

\begin{funcdesc}{openlog}{ident\, \optional{logopt\, \optional{facility}}}
Logging options other than the defaults can be set by explicitly opening
the log file with \code{openlog()} prior to calling \code{syslog()}.
The defaults are (usually) \var{ident} = \samp{syslog}, \var{logopt} = 0,
\var{facility} = \code{LOG_USER}.
The \var{ident} argument is a string which is prepended to every message.
The optional \var{logopt} argument is a bit field - see below for possible
values to combine.
The optional \var{facility} argument sets the default facility for messages
which do not have a facility explicitly encoded.
\end{funcdesc}

\begin{funcdesc}{closelog}{}
Close the log file.
\end{funcdesc}

\begin{funcdesc}{setlogmask}{maskpri}
This function set the priority mask to \var{maskpri} and returns the
previous mask value.
Calls to \code{syslog} with a priority level not set in \var{maskpri}
are ignored.
The default is to log all priorities.
The function \code{LOG_MASK(\var{pri})} calculates the mask for the
individual priority \var{pri}.
The function \code{LOG_UPTO(\var{pri})} calculates the mask for all priorities
up to and including \var{pri}.
\end{funcdesc}

The module defines the following constants:

\begin{description}

\item[Priority levels (high to low):]

\code{LOG_EMERG}, \code{LOG_ALERT}, \code{LOG_CRIT}, \code{LOG_ERR},
\code{LOG_WARNING}, \code{LOG_NOTICE}, \code{LOG_INFO}, \code{LOG_DEBUG}.

\item[Facilities:]

\code{LOG_KERN}, \code{LOG_USER}, \code{LOG_MAIL}, \code{LOG_DAEMON},
\code{LOG_AUTH}, \code{LOG_LPR}, \code{LOG_NEWS}, \code{LOG_UUCP},
\code{LOG_CRON} and \code{LOG_LOCAL0} to \code{LOG_LOCAL7}.

\item[Log options:]

\code{LOG_PID}, \code{LOG_CONS}, \code{LOG_NDELAY}, \code{LOG_NOWAIT}
and \code{LOG_PERROR} if defined in \file{syslog.h}.

\end{description}

\section{\module{popen2} ---
         Subprocesses with accessible I/O streams}

\declaremodule{standard}{popen2}
  \platform{Unix, Windows}
\modulesynopsis{Subprocesses with accessible standard I/O streams.}
\sectionauthor{Drew Csillag}{drew_csillag@geocities.com}


This module allows you to spawn processes and connect to their
input/output/error pipes and obtain their return codes under
\UNIX{} and Windows.

Note that starting with Python 2.0, this functionality is available
using functions from the \refmodule{os} module which have the same
names as the factory functions here, but the order of the return
values is more intuitive in the \refmodule{os} module variants.

The primary interface offered by this module is a trio of factory
functions.  For each of these, if \var{bufsize} is specified, 
it specifies the buffer size for the I/O pipes.  \var{mode}, if
provided, should be the string \code{'b'} or \code{'t'}; on Windows
this is needed to determine whether the file objects should be opened
in binary or text mode.  The default value for \var{mode} is
\code{'t'}.

On \UNIX, \var{cmd} may be a sequence, in which case arguments will be passed
directly to the program without shell intervention (as with
\function{os.spawnv()}). If \var{cmd} is a string it will be passed to the
shell (as with \function{os.system()}).

The only way to retrieve the return codes for the child processes is
by using the \method{poll()} or \method{wait()} methods on the
\class{Popen3} and \class{Popen4} classes; these are only available on
\UNIX.  This information is not available when using the
\function{popen2()}, \function{popen3()}, and \function{popen4()}
functions, or the equivalent functions in the \refmodule{os} module.
(Note that the tuples returned by the \refmodule{os} module's functions
are in a different order from the ones returned by the \module{popen2}
module.)

\begin{funcdesc}{popen2}{cmd\optional{, bufsize\optional{, mode}}}
Executes \var{cmd} as a sub-process.  Returns the file objects
\code{(\var{child_stdout}, \var{child_stdin})}.
\end{funcdesc}

\begin{funcdesc}{popen3}{cmd\optional{, bufsize\optional{, mode}}}
Executes \var{cmd} as a sub-process.  Returns the file objects
\code{(\var{child_stdout}, \var{child_stdin}, \var{child_stderr})}.
\end{funcdesc}

\begin{funcdesc}{popen4}{cmd\optional{, bufsize\optional{, mode}}}
Executes \var{cmd} as a sub-process.  Returns the file objects
\code{(\var{child_stdout_and_stderr}, \var{child_stdin})}.
\versionadded{2.0}
\end{funcdesc}


On \UNIX, a class defining the objects returned by the factory
functions is also available.  These are not used for the Windows
implementation, and are not available on that platform.

\begin{classdesc}{Popen3}{cmd\optional{, capturestderr\optional{, bufsize}}}
This class represents a child process.  Normally, \class{Popen3}
instances are created using the \function{popen2()} and
\function{popen3()} factory functions described above.

If not using one of the helper functions to create \class{Popen3}
objects, the parameter \var{cmd} is the shell command to execute in a
sub-process.  The \var{capturestderr} flag, if true, specifies that
the object should capture standard error output of the child process.
The default is false.  If the \var{bufsize} parameter is specified, it
specifies the size of the I/O buffers to/from the child process.
\end{classdesc}

\begin{classdesc}{Popen4}{cmd\optional{, bufsize}}
Similar to \class{Popen3}, but always captures standard error into the
same file object as standard output.  These are typically created
using \function{popen4()}.
\versionadded{2.0}
\end{classdesc}

\subsection{Popen3 and Popen4 Objects \label{popen3-objects}}

Instances of the \class{Popen3} and \class{Popen4} classes have the
following methods:

\begin{methoddesc}{poll}{}
Returns \code{-1} if child process hasn't completed yet, or its return 
code otherwise.
\end{methoddesc}

\begin{methoddesc}{wait}{}
Waits for and returns the status code of the child process.  The
status code encodes both the return code of the process and
information about whether it exited using the \cfunction{exit()}
system call or died due to a signal.  Functions to help interpret the
status code are defined in the \refmodule{os} module; see section
\ref{os-process} for the \function{W\var{*}()} family of functions.
\end{methoddesc}


The following attributes are also available: 

\begin{memberdesc}{fromchild}
A file object that provides output from the child process.  For
\class{Popen4} instances, this will provide both the standard output
and standard error streams.
\end{memberdesc}

\begin{memberdesc}{tochild}
A file object that provides input to the child process.
\end{memberdesc}

\begin{memberdesc}{childerr}
A file object that provides error output from the child process, if
\var{capturestderr} was true for the constructor, otherwise
\code{None}.  This will always be \code{None} for \class{Popen4}
instances.
\end{memberdesc}

\begin{memberdesc}{pid}
The process ID of the child process.
\end{memberdesc}


\subsection{Flow Control Issues \label{popen2-flow-control}}

Any time you are working with any form of inter-process communication,
control flow needs to be carefully thought out.  This remains the case
with the file objects provided by this module (or the \refmodule{os}
module equivalents).

% Example explanation and suggested work-arounds substantially stolen
% from Martin von L�wis:
% http://mail.python.org/pipermail/python-dev/2000-September/009460.html

When reading output from a child process that writes a lot of data to
standard error while the parent is reading from the child's standard
output, a deadlock can occur.  A similar situation can occur with other
combinations of reads and writes.  The essential factors are that more
than \constant{_PC_PIPE_BUF} bytes are being written by one process in
a blocking fashion, while the other process is reading from the other
process, also in a blocking fashion.

There are several ways to deal with this situation.

The simplest application change, in many cases, will be to follow this
model in the parent process:

\begin{verbatim}
import popen2

r, w, e = popen2.popen3('python slave.py')
e.readlines()
r.readlines()
r.close()
e.close()
w.close()
\end{verbatim}

with code like this in the child:

\begin{verbatim}
import os
import sys

# note that each of these print statements
# writes a single long string

print >>sys.stderr, 400 * 'this is a test\n'
os.close(sys.stderr.fileno())
print >>sys.stdout, 400 * 'this is another test\n'
\end{verbatim}

In particular, note that \code{sys.stderr} must be closed after
writing all data, or \method{readlines()} won't return.  Also note
that \function{os.close()} must be used, as \code{sys.stderr.close()}
won't close \code{stderr} (otherwise assigning to \code{sys.stderr}
will silently close it, so no further errors can be printed).

Applications which need to support a more general approach should
integrate I/O over pipes with their \function{select()} loops, or use
separate threads to read each of the individual files provided by
whichever \function{popen*()} function or \class{Popen*} class was
used.

\section{Standard module \sectcode{commands}}	% If implemented in Python
\label{module-commands}
\stmodindex{commands}

The \code{commands} module contains wrapper functions for \code{os.popen()} 
which take a system command as a string and return any output generated by 
the command, and optionally, the exit status.

The \code{commands} module is only usable on systems which support 
\code{popen()} (currently \UNIX{}).

The \code{commands} module defines the following functions:

\renewcommand{\indexsubitem}{(in module commands)}
\begin{funcdesc}{getstatusoutput}{cmd}
Execute the string \var{cmd} in a shell with \code{os.popen()} and return
a 2-tuple (status, output).  \var{cmd} is actually run as
\code{\{ cmd ; \} 2>\&1}, so that the returned output will contain output
or error messages. A trailing newline is stripped from the output.
The exit status for the  command can be interpreted according to the
rules for the \C{} function \code{wait()}.  
\end{funcdesc}

\begin{funcdesc}{getoutput}{cmd}
Like \code{getstatusoutput()}, except the exit status is ignored and
the return value is a string containing the command's output.  
\end{funcdesc}

\begin{funcdesc}{getstatus}{file}
Return the output of \samp{ls -ld \var{file}} as a string.  This
function uses the \code{getoutput()} function, and properly escapes
backslashes and dollar signs in the argument.
\end{funcdesc}

Example:

\begin{verbatim}
>>> import commands
>>> commands.getstatusoutput('ls /bin/ls')
(0, '/bin/ls')
>>> commands.getstatusoutput('cat /bin/junk')
(256, 'cat: /bin/junk: No such file or directory')
>>> commands.getstatusoutput('/bin/junk')
(256, 'sh: /bin/junk: not found')
>>> commands.getoutput('ls /bin/ls')
'/bin/ls'
>>> commands.getstatus('/bin/ls')
'-rwxr-xr-x  1 root        13352 Oct 14  1994 /bin/ls'
\end{verbatim}


\chapter{The Python Debugger}

\declaremodule{standard}{pdb}
\modulesynopsis{The Python debugger for interactive interpreters.}


The module \module{pdb} defines an interactive source code
debugger\index{debugging} for Python programs.  It supports setting
(conditional) breakpoints and single stepping at the source line
level, inspection of stack frames, source code listing, and evaluation
of arbitrary Python code in the context of any stack frame.  It also
supports post-mortem debugging and can be called under program
control.

The debugger is extensible --- it is actually defined as the class
\class{Pdb}\withsubitem{(class in pdb)}{\ttindex{Pdb}}.
This is currently undocumented but easily understood by reading the
source.  The extension interface uses the modules
\module{bdb}\refstmodindex{bdb} (undocumented) and
\refmodule{cmd}\refstmodindex{cmd}.

A primitive windowing version of the debugger also exists --- this is
module \module{wdb}\refstmodindex{wdb}, which requires
\module{stdwin}\refbimodindex{stdwin}.

The debugger's prompt is \samp{(Pdb) }.
Typical usage to run a program under control of the debugger is:

\begin{verbatim}
>>> import pdb
>>> import mymodule
>>> pdb.run('mymodule.test()')
> <string>(0)?()
(Pdb) continue
> <string>(1)?()
(Pdb) continue
NameError: 'spam'
> <string>(1)?()
(Pdb) 
\end{verbatim}

\file{pdb.py} can also be invoked as
a script to debug other scripts.  For example:

\begin{verbatim}
python /usr/local/lib/python1.5/pdb.py myscript.py
\end{verbatim}

Typical usage to inspect a crashed program is:

\begin{verbatim}
>>> import pdb
>>> import mymodule
>>> mymodule.test()
Traceback (innermost last):
  File "<stdin>", line 1, in ?
  File "./mymodule.py", line 4, in test
    test2()
  File "./mymodule.py", line 3, in test2
    print spam
NameError: spam
>>> pdb.pm()
> ./mymodule.py(3)test2()
-> print spam
(Pdb) 
\end{verbatim}

The module defines the following functions; each enters the debugger
in a slightly different way:

\begin{funcdesc}{run}{statement\optional{, globals\optional{, locals}}}
Execute the \var{statement} (given as a string) under debugger
control.  The debugger prompt appears before any code is executed; you
can set breakpoints and type \samp{continue}, or you can step through
the statement using \samp{step} or \samp{next} (all these commands are
explained below).  The optional \var{globals} and \var{locals}
arguments specify the environment in which the code is executed; by
default the dictionary of the module \module{__main__} is used.  (See
the explanation of the \keyword{exec} statement or the
\function{eval()} built-in function.)
\end{funcdesc}

\begin{funcdesc}{runeval}{expression\optional{, globals\optional{, locals}}}
Evaluate the \var{expression} (given as a a string) under debugger
control.  When \function{runeval()} returns, it returns the value of the
expression.  Otherwise this function is similar to
\function{run()}.
\end{funcdesc}

\begin{funcdesc}{runcall}{function\optional{, argument, ...}}
Call the \var{function} (a function or method object, not a string)
with the given arguments.  When \function{runcall()} returns, it returns
whatever the function call returned.  The debugger prompt appears as
soon as the function is entered.
\end{funcdesc}

\begin{funcdesc}{set_trace}{}
Enter the debugger at the calling stack frame.  This is useful to
hard-code a breakpoint at a given point in a program, even if the code
is not otherwise being debugged (e.g. when an assertion fails).
\end{funcdesc}

\begin{funcdesc}{post_mortem}{traceback}
Enter post-mortem debugging of the given \var{traceback} object.
\end{funcdesc}

\begin{funcdesc}{pm}{}
Enter post-mortem debugging of the traceback found in
\code{sys.last_traceback}.
\end{funcdesc}


\section{Debugger Commands \label{debugger-commands}}

The debugger recognizes the following commands.  Most commands can be
abbreviated to one or two letters; e.g. \samp{h(elp)} means that
either \samp{h} or \samp{help} can be used to enter the help
command (but not \samp{he} or \samp{hel}, nor \samp{H} or
\samp{Help} or \samp{HELP}).  Arguments to commands must be
separated by whitespace (spaces or tabs).  Optional arguments are
enclosed in square brackets (\samp{[]}) in the command syntax; the
square brackets must not be typed.  Alternatives in the command syntax
are separated by a vertical bar (\samp{|}).

Entering a blank line repeats the last command entered.  Exception: if
the last command was a \samp{list} command, the next 11 lines are
listed.

Commands that the debugger doesn't recognize are assumed to be Python
statements and are executed in the context of the program being
debugged.  Python statements can also be prefixed with an exclamation
point (\samp{!}).  This is a powerful way to inspect the program
being debugged; it is even possible to change a variable or call a
function.  When an
exception occurs in such a statement, the exception name is printed
but the debugger's state is not changed.

Multiple commands may be entered on a single line, separated by
\samp{;;}.  (A single \samp{;} is not used as it is
the separator for multiple commands in a line that is passed to
the Python parser.)
No intelligence is applied to separating the commands;
the input is split at the first \samp{;;} pair, even if it is in
the middle of a quoted string.

The debugger supports aliases.  Aliases can have parameters which
allows one a certain level of adaptability to the context under
examination.

If a file \file{.pdbrc}
\indexii{.pdbrc}{file}\indexiii{debugger}{configuration}{file}
exists in the user's home directory or in the current directory, it is
read in and executed as if it had been typed at the debugger prompt.
This is particularly useful for aliases.  If both files exist, the one
in the home directory is read first and aliases defined there can be
overridden by the local file.

\begin{description}

\item[h(elp) \optional{\var{command}}]

Without argument, print the list of available commands.  With a
\var{command} as argument, print help about that command.  \samp{help
pdb} displays the full documentation file; if the environment variable
\envvar{PAGER} is defined, the file is piped through that command
instead.  Since the \var{command} argument must be an identifier,
\samp{help exec} must be entered to get help on the \samp{!} command.

\item[w(here)]

Print a stack trace, with the most recent frame at the bottom.  An
arrow indicates the current frame, which determines the context of
most commands.

\item[d(own)]

Move the current frame one level down in the stack trace
(to an newer frame).

\item[u(p)]

Move the current frame one level up in the stack trace
(to a older frame).

\item[b(reak) \optional{\optional{\var{filename}:}\var{lineno}\code{\Large{|}}\var{function}\optional{, \var{condition}}}]

With a \var{lineno} argument, set a break there in the current
file.  With a \var{function} argument, set a break at the first
executable statement within that function.
The line number may be prefixed with a filename and a colon,
to specify a breakpoint in another file (probably one that
hasn't been loaded yet).  The file is searched on \code{sys.path}.
Note that each breakpoint is assigned a number to which all the other
breakpoint commands refer.

If a second argument is present, it is an expression which must
evaluate to true before the breakpoint is honored.

Without argument, list all breaks, including for each breakpoint,
the number of times that breakpoint has been hit, the current
ignore count, and the associated condition if any.

\item[tbreak \optional{\optional{\var{filename}:}\var{lineno}\code{\Large{|}}\var{function}\optional{, \var{condition}}}]

Temporary breakpoint, which is removed automatically when it is
first hit.  The arguments are the same as break.

\item[cl(ear) \optional{\var{bpnumber} \optional{\var{bpnumber ...}}}]

With a space separated list of breakpoint numbers, clear those
breakpoints.  Without argument, clear all breaks (but first
ask confirmation).

\item[disable \optional{\var{bpnumber} \optional{\var{bpnumber ...}}}]

Disables the breakpoints given as a space separated list of
breakpoint numbers.  Disabling a breakpoint means it cannot cause
the program to stop execution, but unlike clearing a breakpoint, it
remains in the list of breakpoints and can be (re-)enabled.

\item[enable \optional{\var{bpnumber} \optional{\var{bpnumber ...}}}]

Enables the breakpoints specified.

\item[ignore \var{bpnumber} \optional{\var{count}}]

Sets the ignore count for the given breakpoint number.  If
count is omitted, the ignore count is set to 0.  A breakpoint
becomes active when the ignore count is zero.  When non-zero,
the count is decremented each time the breakpoint is reached
and the breakpoint is not disabled and any associated condition
evaluates to true.

\item[condition \var{bpnumber} \optional{\var{condition}}]

Condition is an expression which must evaluate to true before
the breakpoint is honored.  If condition is absent, any existing
condition is removed; i.e., the breakpoint is made unconditional.

\item[s(tep)]

Execute the current line, stop at the first possible occasion
(either in a function that is called or on the next line in the
current function).

\item[n(ext)]

Continue execution until the next line in the current function
is reached or it returns.  (The difference between \samp{next} and
\samp{step} is that \samp{step} stops inside a called function, while
\samp{next} executes called functions at (nearly) full speed, only
stopping at the next line in the current function.)

\item[r(eturn)]

Continue execution until the current function returns.

\item[c(ont(inue))]

Continue execution, only stop when a breakpoint is encountered.

\item[l(ist) \optional{\var{first\optional{, last}}}]

List source code for the current file.  Without arguments, list 11
lines around the current line or continue the previous listing.  With
one argument, list 11 lines around at that line.  With two arguments,
list the given range; if the second argument is less than the first,
it is interpreted as a count.

\item[a(rgs)]

Print the argument list of the current function.

\item[p \var{expression}]

Evaluate the \var{expression} in the current context and print its
value.  (Note: \samp{print} can also be used, but is not a debugger
command --- this executes the Python \keyword{print} statement.)

\item[alias \optional{\var{name} \optional{command}}]

Creates an alias called \var{name} that executes \var{command}.  The
command must \emph{not} be enclosed in quotes.  Replaceable parameters
can be indicated by \samp{\%1}, \samp{\%2}, and so on, while \samp{\%*} is
replaced by all the parameters.  If no command is given, the current
alias for \var{name} is shown. If no arguments are given, all
aliases are listed.

Aliases may be nested and can contain anything that can be
legally typed at the pdb prompt.  Note that internal pdb commands
\emph{can} be overridden by aliases.  Such a command is
then hidden until the alias is removed.  Aliasing is recursively
applied to the first word of the command line; all other words
in the line are left alone.

As an example, here are two useful aliases (especially when placed
in the \file{.pdbrc} file):

\begin{verbatim}
#Print instance variables (usage "pi classInst")
alias pi for k in %1.__dict__.keys(): print "%1.",k,"=",%1.__dict__[k]
#Print instance variables in self
alias ps pi self
\end{verbatim}
		
\item[unalias \var{name}]

Deletes the specified alias.

\item[\optional{!}\var{statement}]

Execute the (one-line) \var{statement} in the context of
the current stack frame.
The exclamation point can be omitted unless the first word
of the statement resembles a debugger command.
To set a global variable, you can prefix the assignment
command with a \samp{global} command on the same line, e.g.:

\begin{verbatim}
(Pdb) global list_options; list_options = ['-l']
(Pdb)
\end{verbatim}

\item[q(uit)]

Quit from the debugger.
The program being executed is aborted.

\end{description}

\section{How It Works}

Some changes were made to the interpreter:

\begin{itemize}
\item \code{sys.settrace(\var{func})} sets the global trace function
\item there can also a local trace function (see later)
\end{itemize}

Trace functions have three arguments: \var{frame}, \var{event}, and
\var{arg}. \var{frame} is the current stack frame.  \var{event} is a
string: \code{'call'}, \code{'line'}, \code{'return'} or
\code{'exception'}.  \var{arg} depends on the event type.

The global trace function is invoked (with \var{event} set to
\code{'call'}) whenever a new local scope is entered; it should return
a reference to the local trace function to be used that scope, or
\code{None} if the scope shouldn't be traced.

The local trace function should return a reference to itself (or to
another function for further tracing in that scope), or \code{None} to
turn off tracing in that scope.

Instance methods are accepted (and very useful!) as trace functions.

The events have the following meaning:

\begin{description}

\item[\code{'call'}]
A function is called (or some other code block entered).  The global
trace function is called; arg is the argument list to the function;
the return value specifies the local trace function.

\item[\code{'line'}]
The interpreter is about to execute a new line of code (sometimes
multiple line events on one line exist).  The local trace function is
called; arg in None; the return value specifies the new local trace
function.

\item[\code{'return'}]
A function (or other code block) is about to return.  The local trace
function is called; arg is the value that will be returned.  The trace
function's return value is ignored.

\item[\code{'exception'}]
An exception has occurred.  The local trace function is called; arg is
a triple (exception, value, traceback); the return value specifies the
new local trace function

\end{description}

Note that as an exception is propagated down the chain of callers, an
\code{'exception'} event is generated at each level.

For more information on code and frame objects, refer to the
\citetitle[../ref/ref.html]{Python Reference Manual}.
			% The Python Debugger

\chapter{The Python Profilers \label{profile}}

\sectionauthor{James Roskind}{}

Copyright \copyright{} 1994, by InfoSeek Corporation, all rights reserved.
\index{InfoSeek Corporation}

Written by James Roskind.\footnote{
  Updated and converted to \LaTeX\ by Guido van Rossum.
  Further updated by Armin Rigo to integrate the documentation for the new
  \module{cProfile} module of Python 2.5.}

Permission to use, copy, modify, and distribute this Python software
and its associated documentation for any purpose (subject to the
restriction in the following sentence) without fee is hereby granted,
provided that the above copyright notice appears in all copies, and
that both that copyright notice and this permission notice appear in
supporting documentation, and that the name of InfoSeek not be used in
advertising or publicity pertaining to distribution of the software
without specific, written prior permission.  This permission is
explicitly restricted to the copying and modification of the software
to remain in Python, compiled Python, or other languages (such as C)
wherein the modified or derived code is exclusively imported into a
Python module.

INFOSEEK CORPORATION DISCLAIMS ALL WARRANTIES WITH REGARD TO THIS
SOFTWARE, INCLUDING ALL IMPLIED WARRANTIES OF MERCHANTABILITY AND
FITNESS. IN NO EVENT SHALL INFOSEEK CORPORATION BE LIABLE FOR ANY
SPECIAL, INDIRECT OR CONSEQUENTIAL DAMAGES OR ANY DAMAGES WHATSOEVER
RESULTING FROM LOSS OF USE, DATA OR PROFITS, WHETHER IN AN ACTION OF
CONTRACT, NEGLIGENCE OR OTHER TORTIOUS ACTION, ARISING OUT OF OR IN
CONNECTION WITH THE USE OR PERFORMANCE OF THIS SOFTWARE.


The profiler was written after only programming in Python for 3 weeks.
As a result, it is probably clumsy code, but I don't know for sure yet
'cause I'm a beginner :-).  I did work hard to make the code run fast,
so that profiling would be a reasonable thing to do.  I tried not to
repeat code fragments, but I'm sure I did some stuff in really awkward
ways at times.  Please send suggestions for improvements to:
\email{jar@netscape.com}.  I won't promise \emph{any} support.  ...but
I'd appreciate the feedback.


\section{Introduction to the profilers}
\nodename{Profiler Introduction}

A \dfn{profiler} is a program that describes the run time performance
of a program, providing a variety of statistics.  This documentation
describes the profiler functionality provided in the modules
\module{profile} and \module{pstats}.  This profiler provides
\dfn{deterministic profiling} of any Python programs.  It also
provides a series of report generation tools to allow users to rapidly
examine the results of a profile operation.
\index{deterministic profiling}
\index{profiling, deterministic}

The Python standard library provides three different profilers:

\begin{enumerate}
\item \module{profile}, a pure Python module, described in the sequel.
  Copyright \copyright{} 1994, by InfoSeek Corporation.
  \versionchanged[also reports the time spent in calls to built-in
  functions and methods]{2.4}

\item \module{cProfile}, a module written in C, with a reasonable
  overhead that makes it suitable for profiling long-running programs.
  Based on \module{lsprof}, contributed by Brett Rosen and Ted Czotter.
  \versionadded{2.5}

\item \module{hotshot}, a C module focusing on minimizing the overhead
  while profiling, at the expense of long data post-processing times.
  \versionchanged[the results should be more meaningful than in the
  past: the timing core contained a critical bug]{2.5}
\end{enumerate}

The \module{profile} and \module{cProfile} modules export the same
interface, so they are mostly interchangeables; \module{cProfile} has a
much lower overhead but is not so far as well-tested and might not be
available on all systems.  \module{cProfile} is really a compatibility
layer on top of the internal \module{_lsprof} module.  The
\module{hotshot} module is reserved to specialized usages.

%\section{How Is This Profiler Different From The Old Profiler?}
%\nodename{Profiler Changes}
%
%(This section is of historical importance only; the old profiler
%discussed here was last seen in Python 1.1.)
%
%The big changes from old profiling module are that you get more
%information, and you pay less CPU time.  It's not a trade-off, it's a
%trade-up.
%
%To be specific:
%
%\begin{description}
%
%\item[Bugs removed:]
%Local stack frame is no longer molested, execution time is now charged
%to correct functions.
%
%\item[Accuracy increased:]
%Profiler execution time is no longer charged to user's code,
%calibration for platform is supported, file reads are not done \emph{by}
%profiler \emph{during} profiling (and charged to user's code!).
%
%\item[Speed increased:]
%Overhead CPU cost was reduced by more than a factor of two (perhaps a
%factor of five), lightweight profiler module is all that must be
%loaded, and the report generating module (\module{pstats}) is not needed
%during profiling.
%
%\item[Recursive functions support:]
%Cumulative times in recursive functions are correctly calculated;
%recursive entries are counted.
%
%\item[Large growth in report generating UI:]
%Distinct profiles runs can be added together forming a comprehensive
%report; functions that import statistics take arbitrary lists of
%files; sorting criteria is now based on keywords (instead of 4 integer
%options); reports shows what functions were profiled as well as what
%profile file was referenced; output format has been improved.
%
%\end{description}


\section{Instant User's Manual \label{profile-instant}}

This section is provided for users that ``don't want to read the
manual.'' It provides a very brief overview, and allows a user to
rapidly perform profiling on an existing application.

To profile an application with a main entry point of \function{foo()},
you would add the following to your module:

\begin{verbatim}
import cProfile
cProfile.run('foo()')
\end{verbatim}

(Use \module{profile} instead of \module{cProfile} if the latter is not
available on your system.)

The above action would cause \function{foo()} to be run, and a series of
informative lines (the profile) to be printed.  The above approach is
most useful when working with the interpreter.  If you would like to
save the results of a profile into a file for later examination, you
can supply a file name as the second argument to the \function{run()}
function:

\begin{verbatim}
import cProfile
cProfile.run('foo()', 'fooprof')
\end{verbatim}

The file \file{cProfile.py} can also be invoked as
a script to profile another script.  For example:

\begin{verbatim}
python -m cProfile myscript.py
\end{verbatim}

\file{cProfile.py} accepts two optional arguments on the command line:

\begin{verbatim}
cProfile.py [-o output_file] [-s sort_order]
\end{verbatim}

\programopt{-s} only applies to standard output (\programopt{-o} is
not supplied).  Look in the \class{Stats} documentation for valid sort
values.

When you wish to review the profile, you should use the methods in the
\module{pstats} module.  Typically you would load the statistics data as
follows:

\begin{verbatim}
import pstats
p = pstats.Stats('fooprof')
\end{verbatim}

The class \class{Stats} (the above code just created an instance of
this class) has a variety of methods for manipulating and printing the
data that was just read into \code{p}.  When you ran
\function{cProfile.run()} above, what was printed was the result of three
method calls:

\begin{verbatim}
p.strip_dirs().sort_stats(-1).print_stats()
\end{verbatim}

The first method removed the extraneous path from all the module
names. The second method sorted all the entries according to the
standard module/line/name string that is printed.
%(this is to comply with the semantics of the old profiler).
The third method printed out
all the statistics.  You might try the following sort calls:

\begin{verbatim}
p.sort_stats('name')
p.print_stats()
\end{verbatim}

The first call will actually sort the list by function name, and the
second call will print out the statistics.  The following are some
interesting calls to experiment with:

\begin{verbatim}
p.sort_stats('cumulative').print_stats(10)
\end{verbatim}

This sorts the profile by cumulative time in a function, and then only
prints the ten most significant lines.  If you want to understand what
algorithms are taking time, the above line is what you would use.

If you were looking to see what functions were looping a lot, and
taking a lot of time, you would do:

\begin{verbatim}
p.sort_stats('time').print_stats(10)
\end{verbatim}

to sort according to time spent within each function, and then print
the statistics for the top ten functions.

You might also try:

\begin{verbatim}
p.sort_stats('file').print_stats('__init__')
\end{verbatim}

This will sort all the statistics by file name, and then print out
statistics for only the class init methods (since they are spelled
with \code{__init__} in them).  As one final example, you could try:

\begin{verbatim}
p.sort_stats('time', 'cum').print_stats(.5, 'init')
\end{verbatim}

This line sorts statistics with a primary key of time, and a secondary
key of cumulative time, and then prints out some of the statistics.
To be specific, the list is first culled down to 50\% (re: \samp{.5})
of its original size, then only lines containing \code{init} are
maintained, and that sub-sub-list is printed.

If you wondered what functions called the above functions, you could
now (\code{p} is still sorted according to the last criteria) do:

\begin{verbatim}
p.print_callers(.5, 'init')
\end{verbatim}

and you would get a list of callers for each of the listed functions.

If you want more functionality, you're going to have to read the
manual, or guess what the following functions do:

\begin{verbatim}
p.print_callees()
p.add('fooprof')
\end{verbatim}

Invoked as a script, the \module{pstats} module is a statistics
browser for reading and examining profile dumps.  It has a simple
line-oriented interface (implemented using \refmodule{cmd}) and
interactive help.

\section{What Is Deterministic Profiling?}
\nodename{Deterministic Profiling}

\dfn{Deterministic profiling} is meant to reflect the fact that all
\emph{function call}, \emph{function return}, and \emph{exception} events
are monitored, and precise timings are made for the intervals between
these events (during which time the user's code is executing).  In
contrast, \dfn{statistical profiling} (which is not done by this
module) randomly samples the effective instruction pointer, and
deduces where time is being spent.  The latter technique traditionally
involves less overhead (as the code does not need to be instrumented),
but provides only relative indications of where time is being spent.

In Python, since there is an interpreter active during execution, the
presence of instrumented code is not required to do deterministic
profiling.  Python automatically provides a \dfn{hook} (optional
callback) for each event.  In addition, the interpreted nature of
Python tends to add so much overhead to execution, that deterministic
profiling tends to only add small processing overhead in typical
applications.  The result is that deterministic profiling is not that
expensive, yet provides extensive run time statistics about the
execution of a Python program.

Call count statistics can be used to identify bugs in code (surprising
counts), and to identify possible inline-expansion points (high call
counts).  Internal time statistics can be used to identify ``hot
loops'' that should be carefully optimized.  Cumulative time
statistics should be used to identify high level errors in the
selection of algorithms.  Note that the unusual handling of cumulative
times in this profiler allows statistics for recursive implementations
of algorithms to be directly compared to iterative implementations.


\section{Reference Manual -- \module{profile} and \module{cProfile}}

\declaremodule{standard}{profile}
\declaremodule{standard}{cProfile}
\modulesynopsis{Python profiler}



The primary entry point for the profiler is the global function
\function{profile.run()} (resp. \function{cProfile.run()}).
It is typically used to create any profile
information.  The reports are formatted and printed using methods of
the class \class{pstats.Stats}.  The following is a description of all
of these standard entry points and functions.  For a more in-depth
view of some of the code, consider reading the later section on
Profiler Extensions, which includes discussion of how to derive
``better'' profilers from the classes presented, or reading the source
code for these modules.

\begin{funcdesc}{run}{command\optional{, filename}}

This function takes a single argument that has can be passed to the
\keyword{exec} statement, and an optional file name.  In all cases this
routine attempts to \keyword{exec} its first argument, and gather profiling
statistics from the execution. If no file name is present, then this
function automatically prints a simple profiling report, sorted by the
standard name string (file/line/function-name) that is presented in
each line.  The following is a typical output from such a call:

\begin{verbatim}
      2706 function calls (2004 primitive calls) in 4.504 CPU seconds

Ordered by: standard name

ncalls  tottime  percall  cumtime  percall filename:lineno(function)
     2    0.006    0.003    0.953    0.477 pobject.py:75(save_objects)
  43/3    0.533    0.012    0.749    0.250 pobject.py:99(evaluate)
 ...
\end{verbatim}

The first line indicates that 2706 calls were
monitored.  Of those calls, 2004 were \dfn{primitive}.  We define
\dfn{primitive} to mean that the call was not induced via recursion.
The next line: \code{Ordered by:\ standard name}, indicates that
the text string in the far right column was used to sort the output.
The column headings include:

\begin{description}

\item[ncalls ]
for the number of calls,

\item[tottime ]
for the total time spent in the given function (and excluding time
made in calls to sub-functions),

\item[percall ]
is the quotient of \code{tottime} divided by \code{ncalls}

\item[cumtime ]
is the total time spent in this and all subfunctions (from invocation
till exit). This figure is accurate \emph{even} for recursive
functions.

\item[percall ]
is the quotient of \code{cumtime} divided by primitive calls

\item[filename:lineno(function) ]
provides the respective data of each function

\end{description}

When there are two numbers in the first column (for example,
\samp{43/3}), then the latter is the number of primitive calls, and
the former is the actual number of calls.  Note that when the function
does not recurse, these two values are the same, and only the single
figure is printed.

\end{funcdesc}

\begin{funcdesc}{runctx}{command, globals, locals\optional{, filename}}
This function is similar to \function{run()}, with added
arguments to supply the globals and locals dictionaries for the
\var{command} string.
\end{funcdesc}

Analysis of the profiler data is done using this class from the
\module{pstats} module:

% now switch modules....
% (This \stmodindex use may be hard to change ;-( )
\stmodindex{pstats}

\begin{classdesc}{Stats}{filename\optional{, \moreargs}}
This class constructor creates an instance of a ``statistics object''
from a \var{filename} (or set of filenames).  \class{Stats} objects are
manipulated by methods, in order to print useful reports.

The file selected by the above constructor must have been created by
the corresponding version of \module{profile} or \module{cProfile}.
To be specific, there is
\emph{no} file compatibility guaranteed with future versions of this
profiler, and there is no compatibility with files produced by other
profilers.
%(such as the old system profiler).

If several files are provided, all the statistics for identical
functions will be coalesced, so that an overall view of several
processes can be considered in a single report.  If additional files
need to be combined with data in an existing \class{Stats} object, the
\method{add()} method can be used.
\end{classdesc}


\subsection{The \class{Stats} Class \label{profile-stats}}

\class{Stats} objects have the following methods:

\begin{methoddesc}[Stats]{strip_dirs}{}
This method for the \class{Stats} class removes all leading path
information from file names.  It is very useful in reducing the size
of the printout to fit within (close to) 80 columns.  This method
modifies the object, and the stripped information is lost.  After
performing a strip operation, the object is considered to have its
entries in a ``random'' order, as it was just after object
initialization and loading.  If \method{strip_dirs()} causes two
function names to be indistinguishable (they are on the same
line of the same filename, and have the same function name), then the
statistics for these two entries are accumulated into a single entry.
\end{methoddesc}


\begin{methoddesc}[Stats]{add}{filename\optional{, \moreargs}}
This method of the \class{Stats} class accumulates additional
profiling information into the current profiling object.  Its
arguments should refer to filenames created by the corresponding
version of \function{profile.run()} or \function{cProfile.run()}.
Statistics for identically named
(re: file, line, name) functions are automatically accumulated into
single function statistics.
\end{methoddesc}

\begin{methoddesc}[Stats]{dump_stats}{filename}
Save the data loaded into the \class{Stats} object to a file named
\var{filename}.  The file is created if it does not exist, and is
overwritten if it already exists.  This is equivalent to the method of
the same name on the \class{profile.Profile} and
\class{cProfile.Profile} classes.
\versionadded{2.3}
\end{methoddesc}

\begin{methoddesc}[Stats]{sort_stats}{key\optional{, \moreargs}}
This method modifies the \class{Stats} object by sorting it according
to the supplied criteria.  The argument is typically a string
identifying the basis of a sort (example: \code{'time'} or
\code{'name'}).

When more than one key is provided, then additional keys are used as
secondary criteria when there is equality in all keys selected
before them.  For example, \code{sort_stats('name', 'file')} will sort
all the entries according to their function name, and resolve all ties
(identical function names) by sorting by file name.

Abbreviations can be used for any key names, as long as the
abbreviation is unambiguous.  The following are the keys currently
defined:

\begin{tableii}{l|l}{code}{Valid Arg}{Meaning}
  \lineii{'calls'}{call count}
  \lineii{'cumulative'}{cumulative time}
  \lineii{'file'}{file name}
  \lineii{'module'}{file name}
  \lineii{'pcalls'}{primitive call count}
  \lineii{'line'}{line number}
  \lineii{'name'}{function name}
  \lineii{'nfl'}{name/file/line}
  \lineii{'stdname'}{standard name}
  \lineii{'time'}{internal time}
\end{tableii}

Note that all sorts on statistics are in descending order (placing
most time consuming items first), where as name, file, and line number
searches are in ascending order (alphabetical). The subtle
distinction between \code{'nfl'} and \code{'stdname'} is that the
standard name is a sort of the name as printed, which means that the
embedded line numbers get compared in an odd way.  For example, lines
3, 20, and 40 would (if the file names were the same) appear in the
string order 20, 3 and 40.  In contrast, \code{'nfl'} does a numeric
compare of the line numbers.  In fact, \code{sort_stats('nfl')} is the
same as \code{sort_stats('name', 'file', 'line')}.

%For compatibility with the old profiler,
For backward-compatibility reasons, the numeric arguments
\code{-1}, \code{0}, \code{1}, and \code{2} are permitted.  They are
interpreted as \code{'stdname'}, \code{'calls'}, \code{'time'}, and
\code{'cumulative'} respectively.  If this old style format (numeric)
is used, only one sort key (the numeric key) will be used, and
additional arguments will be silently ignored.
\end{methoddesc}


\begin{methoddesc}[Stats]{reverse_order}{}
This method for the \class{Stats} class reverses the ordering of the basic
list within the object.  %This method is provided primarily for
%compatibility with the old profiler.
Note that by default ascending vs descending order is properly selected
based on the sort key of choice.
\end{methoddesc}

\begin{methoddesc}[Stats]{print_stats}{\optional{restriction, \moreargs}}
This method for the \class{Stats} class prints out a report as described
in the \function{profile.run()} definition.

The order of the printing is based on the last \method{sort_stats()}
operation done on the object (subject to caveats in \method{add()} and
\method{strip_dirs()}).

The arguments provided (if any) can be used to limit the list down to
the significant entries.  Initially, the list is taken to be the
complete set of profiled functions.  Each restriction is either an
integer (to select a count of lines), or a decimal fraction between
0.0 and 1.0 inclusive (to select a percentage of lines), or a regular
expression (to pattern match the standard name that is printed; as of
Python 1.5b1, this uses the Perl-style regular expression syntax
defined by the \refmodule{re} module).  If several restrictions are
provided, then they are applied sequentially.  For example:

\begin{verbatim}
print_stats(.1, 'foo:')
\end{verbatim}

would first limit the printing to first 10\% of list, and then only
print functions that were part of filename \file{.*foo:}.  In
contrast, the command:

\begin{verbatim}
print_stats('foo:', .1)
\end{verbatim}

would limit the list to all functions having file names \file{.*foo:},
and then proceed to only print the first 10\% of them.
\end{methoddesc}


\begin{methoddesc}[Stats]{print_callers}{\optional{restriction, \moreargs}}
This method for the \class{Stats} class prints a list of all functions
that called each function in the profiled database.  The ordering is
identical to that provided by \method{print_stats()}, and the definition
of the restricting argument is also identical.  Each caller is reported on
its own line.  The format differs slightly depending on the profiler that
produced the stats:

\begin{itemize}
\item With \module{profile}, a number is shown in parentheses after each
  caller to show how many times this specific call was made.  For
  convenience, a second non-parenthesized number repeats the cumulative
  time spent in the function at the right.

\item With \module{cProfile}, each caller is preceeded by three numbers:
  the number of times this specific call was made, and the total and
  cumulative times spent in the current function while it was invoked by
  this specific caller.
\end{itemize}
\end{methoddesc}

\begin{methoddesc}[Stats]{print_callees}{\optional{restriction, \moreargs}}
This method for the \class{Stats} class prints a list of all function
that were called by the indicated function.  Aside from this reversal
of direction of calls (re: called vs was called by), the arguments and
ordering are identical to the \method{print_callers()} method.
\end{methoddesc}


\section{Limitations \label{profile-limits}}

One limitation has to do with accuracy of timing information.
There is a fundamental problem with deterministic profilers involving
accuracy.  The most obvious restriction is that the underlying ``clock''
is only ticking at a rate (typically) of about .001 seconds.  Hence no
measurements will be more accurate than the underlying clock.  If
enough measurements are taken, then the ``error'' will tend to average
out. Unfortunately, removing this first error induces a second source
of error.

The second problem is that it ``takes a while'' from when an event is
dispatched until the profiler's call to get the time actually
\emph{gets} the state of the clock.  Similarly, there is a certain lag
when exiting the profiler event handler from the time that the clock's
value was obtained (and then squirreled away), until the user's code
is once again executing.  As a result, functions that are called many
times, or call many functions, will typically accumulate this error.
The error that accumulates in this fashion is typically less than the
accuracy of the clock (less than one clock tick), but it
\emph{can} accumulate and become very significant.

The problem is more important with \module{profile} than with the
lower-overhead \module{cProfile}.  For this reason, \module{profile}
provides a means of calibrating itself for a given platform so that
this error can be probabilistically (on the average) removed.
After the profiler is calibrated, it will be more accurate (in a least
square sense), but it will sometimes produce negative numbers (when
call counts are exceptionally low, and the gods of probability work
against you :-). )  Do \emph{not} be alarmed by negative numbers in
the profile.  They should \emph{only} appear if you have calibrated
your profiler, and the results are actually better than without
calibration.


\section{Calibration \label{profile-calibration}}

The profiler of the \module{profile} module subtracts a constant from each
event handling time to compensate for the overhead of calling the time
function, and socking away the results.  By default, the constant is 0.
The following procedure can
be used to obtain a better constant for a given platform (see discussion
in section Limitations above).

\begin{verbatim}
import profile
pr = profile.Profile()
for i in range(5):
    print pr.calibrate(10000)
\end{verbatim}

The method executes the number of Python calls given by the argument,
directly and again under the profiler, measuring the time for both.
It then computes the hidden overhead per profiler event, and returns
that as a float.  For example, on an 800 MHz Pentium running
Windows 2000, and using Python's time.clock() as the timer,
the magical number is about 12.5e-6.

The object of this exercise is to get a fairly consistent result.
If your computer is \emph{very} fast, or your timer function has poor
resolution, you might have to pass 100000, or even 1000000, to get
consistent results.

When you have a consistent answer,
there are three ways you can use it:\footnote{Prior to Python 2.2, it
  was necessary to edit the profiler source code to embed the bias as
  a literal number.  You still can, but that method is no longer
  described, because no longer needed.}

\begin{verbatim}
import profile

# 1. Apply computed bias to all Profile instances created hereafter.
profile.Profile.bias = your_computed_bias

# 2. Apply computed bias to a specific Profile instance.
pr = profile.Profile()
pr.bias = your_computed_bias

# 3. Specify computed bias in instance constructor.
pr = profile.Profile(bias=your_computed_bias)
\end{verbatim}

If you have a choice, you are better off choosing a smaller constant, and
then your results will ``less often'' show up as negative in profile
statistics.


\section{Extensions --- Deriving Better Profilers}
\nodename{Profiler Extensions}

The \class{Profile} class of both modules, \module{profile} and
\module{cProfile}, were written so that
derived classes could be developed to extend the profiler.  The details
are not described here, as doing this successfully requires an expert
understanding of how the \class{Profile} class works internally.  Study
the source code of the module carefully if you want to
pursue this.

If all you want to do is change how current time is determined (for
example, to force use of wall-clock time or elapsed process time),
pass the timing function you want to the \class{Profile} class
constructor:

\begin{verbatim}
pr = profile.Profile(your_time_func)
\end{verbatim}

The resulting profiler will then call \function{your_time_func()}.

\begin{description}
\item[\class{profile.Profile}]
\function{your_time_func()} should return a single number, or a list of
numbers whose sum is the current time (like what \function{os.times()}
returns).  If the function returns a single time number, or the list of
returned numbers has length 2, then you will get an especially fast
version of the dispatch routine.

Be warned that you should calibrate the profiler class for the
timer function that you choose.  For most machines, a timer that
returns a lone integer value will provide the best results in terms of
low overhead during profiling.  (\function{os.times()} is
\emph{pretty} bad, as it returns a tuple of floating point values).  If
you want to substitute a better timer in the cleanest fashion,
derive a class and hardwire a replacement dispatch method that best
handles your timer call, along with the appropriate calibration
constant.

\item[\class{cProfile.Profile}]
\function{your_time_func()} should return a single number.  If it returns
plain integers, you can also invoke the class constructor with a second
argument specifying the real duration of one unit of time.  For example,
if \function{your_integer_time_func()} returns times measured in thousands
of seconds, you would constuct the \class{Profile} instance as follows:

\begin{verbatim}
pr = profile.Profile(your_integer_time_func, 0.001)
\end{verbatim}

As the \module{cProfile.Profile} class cannot be calibrated, custom
timer functions should be used with care and should be as fast as
possible.  For the best results with a custom timer, it might be
necessary to hard-code it in the C source of the internal
\module{_lsprof} module.

\end{description}
		% The Python Profiler

\chapter{Internet Protocols and Support \label{internet}}

\index{WWW}
\index{Internet}
\index{World-Wide Web}

The modules described in this chapter implement Internet protocols and 
support for related technology.  They are all implemented in Python.
Some of these modules require the presence of the system-dependent
module \module{socket}\refbimodindex{socket}, which is currently only
fully supported on \UNIX{} and Windows NT.  Here is an overview:

\localmoduletable
		% Internet Protocols
\section{Standard Module \sectcode{cgi}}
\label{module-cgi}
\stmodindex{cgi}
\indexii{WWW}{server}
\indexii{CGI}{protocol}
\indexii{HTTP}{protocol}
\indexii{MIME}{headers}
\index{URL}

\renewcommand{\indexsubitem}{(in module cgi)}

Support module for CGI (Common Gateway Interface) scripts.

This module defines a number of utilities for use by CGI scripts
written in Python.

\subsection{Introduction}
\nodename{Introduction to the CGI module}

A CGI script is invoked by an HTTP server, usually to process user
input submitted through an HTML \code{<FORM>} or \code{<ISINPUT>} element.

Most often, CGI scripts live in the server's special \code{cgi-bin}
directory.  The HTTP server places all sorts of information about the
request (such as the client's hostname, the requested URL, the query
string, and lots of other goodies) in the script's shell environment,
executes the script, and sends the script's output back to the client.

The script's input is connected to the client too, and sometimes the
form data is read this way; at other times the form data is passed via
the ``query string'' part of the URL.  This module (\code{cgi.py}) is intended
to take care of the different cases and provide a simpler interface to
the Python script.  It also provides a number of utilities that help
in debugging scripts, and the latest addition is support for file
uploads from a form (if your browser supports it -- Grail 0.3 and
Netscape 2.0 do).

The output of a CGI script should consist of two sections, separated
by a blank line.  The first section contains a number of headers,
telling the client what kind of data is following.  Python code to
generate a minimal header section looks like this:

\bcode\begin{verbatim}
print "Content-type: text/html"     # HTML is following
print                               # blank line, end of headers
\end{verbatim}\ecode
%
The second section is usually HTML, which allows the client software
to display nicely formatted text with header, in-line images, etc.
Here's Python code that prints a simple piece of HTML:

\bcode\begin{verbatim}
print "<TITLE>CGI script output</TITLE>"
print "<H1>This is my first CGI script</H1>"
print "Hello, world!"
\end{verbatim}\ecode
%
(It may not be fully legal HTML according to the letter of the
standard, but any browser will understand it.)

\subsection{Using the cgi module}
\nodename{Using the cgi module}

Begin by writing \code{import cgi}.  Don't use \code{from cgi import *} -- the
module defines all sorts of names for its own use or for backward 
compatibility that you don't want in your namespace.

It's best to use the \code{FieldStorage} class.  The other classes define in this 
module are provided mostly for backward compatibility.  Instantiate it 
exactly once, without arguments.  This reads the form contents from 
standard input or the environment (depending on the value of various 
environment variables set according to the CGI standard).  Since it may 
consume standard input, it should be instantiated only once.

The \code{FieldStorage} instance can be accessed as if it were a Python 
dictionary.  For instance, the following code (which assumes that the 
\code{Content-type} header and blank line have already been printed) checks that 
the fields \code{name} and \code{addr} are both set to a non-empty string:

\bcode\begin{verbatim}
form = cgi.FieldStorage()
form_ok = 0
if form.has_key("name") and form.has_key("addr"):
    if form["name"].value != "" and form["addr"].value != "":
        form_ok = 1
if not form_ok:
    print "<H1>Error</H1>"
    print "Please fill in the name and addr fields."
    return
...further form processing here...
\end{verbatim}\ecode
%
Here the fields, accessed through \code{form[key]}, are themselves instances
of \code{FieldStorage} (or \code{MiniFieldStorage}, depending on the form encoding).

If the submitted form data contains more than one field with the same
name, the object retrieved by \code{form[key]} is not a \code{(Mini)FieldStorage}
instance but a list of such instances.  If you expect this possibility
(i.e., when your HTML form comtains multiple fields with the same
name), use the \code{type()} function to determine whether you have a single
instance or a list of instances.  For example, here's code that
concatenates any number of username fields, separated by commas:

\bcode\begin{verbatim}
username = form["username"]
if type(username) is type([]):
    # Multiple username fields specified
    usernames = ""
    for item in username:
        if usernames:
            # Next item -- insert comma
            usernames = usernames + "," + item.value
        else:
            # First item -- don't insert comma
            usernames = item.value
else:
    # Single username field specified
    usernames = username.value
\end{verbatim}\ecode
%
If a field represents an uploaded file, the value attribute reads the 
entire file in memory as a string.  This may not be what you want.  You can 
test for an uploaded file by testing either the filename attribute or the 
file attribute.  You can then read the data at leasure from the file 
attribute:

\bcode\begin{verbatim}
fileitem = form["userfile"]
if fileitem.file:
    # It's an uploaded file; count lines
    linecount = 0
    while 1:
        line = fileitem.file.readline()
        if not line: break
        linecount = linecount + 1
\end{verbatim}\ecode
%
The file upload draft standard entertains the possibility of uploading
multiple files from one field (using a recursive \code{multipart/*}
encoding).  When this occurs, the item will be a dictionary-like
FieldStorage item.  This can be determined by testing its type
attribute, which should have the value \code{multipart/form-data} (or
perhaps another string beginning with \code{multipart/}  It this case, it
can be iterated over recursively just like the top-level form object.

When a form is submitted in the ``old'' format (as the query string or as a 
single data part of type \code{application/x-www-form-urlencoded}), the items 
will actually be instances of the class \code{MiniFieldStorage}.  In this case,
the list, file and filename attributes are always \code{None}.


\subsection{Old classes}

These classes, present in earlier versions of the \code{cgi} module, are still 
supported for backward compatibility.  New applications should use the
FieldStorage class.

\code{SvFormContentDict}: single value form content as dictionary; assumes each 
field name occurs in the form only once.

\code{FormContentDict}: multiple value form content as dictionary (the form
items are lists of values).  Useful if your form contains multiple
fields with the same name.

Other classes (\code{FormContent}, \code{InterpFormContentDict}) are present for
backwards compatibility with really old applications only.  If you still 
use these and would be inconvenienced when they disappeared from a next 
version of this module, drop me a note.


\subsection{Functions}
\nodename{Functions in cgi module}

These are useful if you want more control, or if you want to employ
some of the algorithms implemented in this module in other
circumstances.

\begin{funcdesc}{parse}{fp}: Parse a query in the environment or from a file (default \code{sys.stdin}).
\end{funcdesc}

\begin{funcdesc}{parse_qs}{qs}: parse a query string given as a string argument (data of type 
\code{application/x-www-form-urlencoded}).
\end{funcdesc}

\begin{funcdesc}{parse_multipart}{fp\, pdict}: parse input of type \code{multipart/form-data} (for 
file uploads).  Arguments are \code{fp} for the input file and 
    \code{pdict} for the dictionary containing other parameters of \code{content-type} header

    Returns a dictionary just like \code{parse_qs()}: keys are the field names, each 
    value is a list of values for that field.  This is easy to use but not 
    much good if you are expecting megabytes to be uploaded -- in that case, 
    use the \code{FieldStorage} class instead which is much more flexible.  Note 
    that \code{content-type} is the raw, unparsed contents of the \code{content-type} 
    header.

    Note that this does not parse nested multipart parts -- use \code{FieldStorage} for 
    that.
\end{funcdesc}

\begin{funcdesc}{parse_header}{string}: parse a header like \code{Content-type} into a main
content-type and a dictionary of parameters.
\end{funcdesc}

\begin{funcdesc}{test}{}: robust test CGI script, usable as main program.
    Writes minimal HTTP headers and formats all information provided to
    the script in HTML form.
\end{funcdesc}

\begin{funcdesc}{print_environ}{}: format the shell environment in HTML.
\end{funcdesc}

\begin{funcdesc}{print_form}{form}: format a form in HTML.
\end{funcdesc}

\begin{funcdesc}{print_directory}{}: format the current directory in HTML.
\end{funcdesc}

\begin{funcdesc}{print_environ_usage}{}: print a list of useful (used by CGI) environment variables in
HTML.
\end{funcdesc}

\begin{funcdesc}{escape}{}: convert the characters ``\code{\&}'', ``\code{<}'' and ``\code{>}'' to HTML-safe
sequences.  Use this if you need to display text that might contain
such characters in HTML.  To translate URLs for inclusion in the HREF
attribute of an \code{<A>} tag, use \code{urllib.quote()}.
\end{funcdesc}


\subsection{Caring about security}

There's one important rule: if you invoke an external program (e.g.
via the \code{os.system()} or \code{os.popen()} functions), make very sure you don't
pass arbitrary strings received from the client to the shell.  This is
a well-known security hole whereby clever hackers anywhere on the web
can exploit a gullible CGI script to invoke arbitrary shell commands.
Even parts of the URL or field names cannot be trusted, since the
request doesn't have to come from your form!

To be on the safe side, if you must pass a string gotten from a form
to a shell command, you should make sure the string contains only
alphanumeric characters, dashes, underscores, and periods.


\subsection{Installing your CGI script on a Unix system}

Read the documentation for your HTTP server and check with your local
system administrator to find the directory where CGI scripts should be
installed; usually this is in a directory \code{cgi-bin} in the server tree.

Make sure that your script is readable and executable by ``others''; the
Unix file mode should be 755 (use \code{chmod 755 filename}).  Make sure
that the first line of the script contains \code{\#!} starting in column 1
followed by the pathname of the Python interpreter, for instance:

\bcode\begin{verbatim}
#!/usr/local/bin/python
\end{verbatim}\ecode
%
Make sure the Python interpreter exists and is executable by ``others''.

Make sure that any files your script needs to read or write are
readable or writable, respectively, by ``others'' -- their mode should
be 644 for readable and 666 for writable.  This is because, for
security reasons, the HTTP server executes your script as user
``nobody'', without any special privileges.  It can only read (write,
execute) files that everybody can read (write, execute).  The current
directory at execution time is also different (it is usually the
server's cgi-bin directory) and the set of environment variables is
also different from what you get at login.  in particular, don't count
on the shell's search path for executables (\code{\$PATH}) or the Python
module search path (\code{\$PYTHONPATH}) to be set to anything interesting.

If you need to load modules from a directory which is not on Python's
default module search path, you can change the path in your script,
before importing other modules, e.g.:

\bcode\begin{verbatim}
import sys
sys.path.insert(0, "/usr/home/joe/lib/python")
sys.path.insert(0, "/usr/local/lib/python")
\end{verbatim}\ecode
%
(This way, the directory inserted last will be searched first!)

Instructions for non-Unix systems will vary; check your HTTP server's
documentation (it will usually have a section on CGI scripts).


\subsection{Testing your CGI script}

Unfortunately, a CGI script will generally not run when you try it
from the command line, and a script that works perfectly from the
command line may fail mysteriously when run from the server.  There's
one reason why you should still test your script from the command
line: if it contains a syntax error, the python interpreter won't
execute it at all, and the HTTP server will most likely send a cryptic
error to the client.

Assuming your script has no syntax errors, yet it does not work, you
have no choice but to read the next section:


\subsection{Debugging CGI scripts}

First of all, check for trivial installation errors -- reading the
section above on installing your CGI script carefully can save you a
lot of time.  If you wonder whether you have understood the
installation procedure correctly, try installing a copy of this module
file (\code{cgi.py}) as a CGI script.  When invoked as a script, the file
will dump its environment and the contents of the form in HTML form.
Give it the right mode etc, and send it a request.  If it's installed
in the standard \code{cgi-bin} directory, it should be possible to send it a
request by entering a URL into your browser of the form:

\bcode\begin{verbatim}
http://yourhostname/cgi-bin/cgi.py?name=Joe+Blow&addr=At+Home
\end{verbatim}\ecode
%
If this gives an error of type 404, the server cannot find the script
-- perhaps you need to install it in a different directory.  If it
gives another error (e.g.  500), there's an installation problem that
you should fix before trying to go any further.  If you get a nicely
formatted listing of the environment and form content (in this
example, the fields should be listed as ``addr'' with value ``At Home''
and ``name'' with value ``Joe Blow''), the \code{cgi.py} script has been
installed correctly.  If you follow the same procedure for your own
script, you should now be able to debug it.

The next step could be to call the \code{cgi} module's test() function from
your script: replace its main code with the single statement

\bcode\begin{verbatim}
cgi.test()
\end{verbatim}\ecode
%
This should produce the same results as those gotten from installing
the \code{cgi.py} file itself.

When an ordinary Python script raises an unhandled exception
(e.g. because of a typo in a module name, a file that can't be opened,
etc.), the Python interpreter prints a nice traceback and exits.
While the Python interpreter will still do this when your CGI script
raises an exception, most likely the traceback will end up in one of
the HTTP server's log file, or be discarded altogether.

Fortunately, once you have managed to get your script to execute
*some* code, it is easy to catch exceptions and cause a traceback to
be printed.  The \code{test()} function below in this module is an example.
Here are the rules:

\begin{enumerate}
	\item Import the traceback module (before entering the
	   try-except!)
	
	\item Make sure you finish printing the headers and the blank
	   line early
	
	\item Assign \code{sys.stderr} to \code{sys.stdout}
	
	\item Wrap all remaining code in a try-except statement
	
	\item In the except clause, call \code{traceback.print_exc()}
\end{enumerate}

For example:

\bcode\begin{verbatim}
import sys
import traceback
print "Content-type: text/html"
print
sys.stderr = sys.stdout
try:
    ...your code here...
except:
    print "\n\n<PRE>"
    traceback.print_exc()
\end{verbatim}\ecode
%
Notes: The assignment to \code{sys.stderr} is needed because the traceback
prints to \code{sys.stderr}.  The \code{print "$\backslash$n$\backslash$n<PRE>"} statement is necessary to
disable the word wrapping in HTML.

If you suspect that there may be a problem in importing the traceback
module, you can use an even more robust approach (which only uses
built-in modules):

\bcode\begin{verbatim}
import sys
sys.stderr = sys.stdout
print "Content-type: text/plain"
print
...your code here...
\end{verbatim}\ecode
%
This relies on the Python interpreter to print the traceback.  The
content type of the output is set to plain text, which disables all
HTML processing.  If your script works, the raw HTML will be displayed
by your client.  If it raises an exception, most likely after the
first two lines have been printed, a traceback will be displayed.
Because no HTML interpretation is going on, the traceback will
readable.


\subsection{Common problems and solutions}

\begin{itemize}
\item Most HTTP servers buffer the output from CGI scripts until the
script is completed.  This means that it is not possible to display a
progress report on the client's display while the script is running.

\item Check the installation instructions above.

\item Check the HTTP server's log files.  (\code{tail -f logfile} in a separate
window may be useful!)

\item Always check a script for syntax errors first, by doing something
like \code{python script.py}.

\item When using any of the debugging techniques, don't forget to add
\code{import sys} to the top of the script.

\item When invoking external programs, make sure they can be found.
Usually, this means using absolute path names -- \code{\$PATH} is usually not
set to a very useful value in a CGI script.

\item When reading or writing external files, make sure they can be read
or written by every user on the system.

\item Don't try to give a CGI script a set-uid mode.  This doesn't work on
most systems, and is a security liability as well.
\end{itemize}


\section{\module{urllib} ---
         Open an arbitrary resource by URL}

\declaremodule{standard}{urllib}
\modulesynopsis{Open an arbitrary network resource by URL (requires sockets).}

\index{WWW}
\index{World-Wide Web}
\index{URL}


This module provides a high-level interface for fetching data across
the World-Wide Web.  In particular, the \function{urlopen()} function
is similar to the built-in function \function{open()}, but accepts
Universal Resource Locators (URLs) instead of filenames.  Some
restrictions apply --- it can only open URLs for reading, and no seek
operations are available.

It defines the following public functions:

\begin{funcdesc}{urlopen}{url\optional{, data}}
Open a network object denoted by a URL for reading.  If the URL does
not have a scheme identifier, or if it has \file{file:} as its scheme
identifier, this opens a local file; otherwise it opens a socket to a
server somewhere on the network.  If the connection cannot be made, or
if the server returns an error code, the \exception{IOError} exception
is raised.  If all went well, a file-like object is returned.  This
supports the following methods: \method{read()}, \method{readline()},
\method{readlines()}, \method{fileno()}, \method{close()},
\method{info()} and \method{geturl()}.

Except for the \method{info()} and \method{geturl()} methods,
these methods have the same interface as for
file objects --- see section \ref{bltin-file-objects} in this
manual.  (It is not a built-in file object, however, so it can't be
used at those few places where a true built-in file object is
required.)

The \method{info()} method returns an instance of the class
\class{mimetools.Message} containing meta-information associated
with the URL.  When the method is HTTP, these headers are those
returned by the server at the head of the retrieved HTML page
(including Content-Length and Content-Type).  When the method is FTP,
a Content-Length header will be present if (as is now usual) the
server passed back a file length in response to the FTP retrieval
request.  When the method is local-file, returned headers will include
a Date representing the file's last-modified time, a Content-Length
giving file size, and a Content-Type containing a guess at the file's
type. See also the description of the
\refmodule{mimetools}\refstmodindex{mimetools} module.

The \method{geturl()} method returns the real URL of the page.  In
some cases, the HTTP server redirects a client to another URL.  The
\function{urlopen()} function handles this transparently, but in some
cases the caller needs to know which URL the client was redirected
to.  The \method{geturl()} method can be used to get at this
redirected URL.

If the \var{url} uses the \file{http:} scheme identifier, the optional
\var{data} argument may be given to specify a \code{POST} request
(normally the request type is \code{GET}).  The \var{data} argument
must in standard \file{application/x-www-form-urlencoded} format;
see the \function{urlencode()} function below.

The \function{urlopen()} function works transparently with proxies.
In a \UNIX{} or Windows environment, set the \envvar{http_proxy},
\envvar{ftp_proxy} or \envvar{gopher_proxy} environment variables to a
URL that identifies the proxy server before starting the Python
interpreter.  For example (the \character{\%} is the command prompt):

\begin{verbatim}
% http_proxy="http://www.someproxy.com:3128"
% export http_proxy
% python
...
\end{verbatim}

In a Macintosh environment, \function{urlopen()} will retrieve proxy
information from Internet\index{Internet Config} Config.

The \function{urlopen()} function works transparently with proxies.
In a \UNIX{} or Windows environment, set the \envvar{http_proxy},
\envvar{ftp_proxy} or \envvar{gopher_proxy} environment variables to a
URL that identifies the proxy server before starting the Python
interpreter, e.g.:

\begin{verbatim}
% http_proxy="http://www.someproxy.com:3128"
% export http_proxy
% python
...
\end{verbatim}

In a Macintosh environment, \function{urlopen()} will retrieve proxy
information from Internet Config.
\end{funcdesc}

\begin{funcdesc}{urlretrieve}{url\optional{, filename\optional{, hook}}}
Copy a network object denoted by a URL to a local file, if necessary.
If the URL points to a local file, or a valid cached copy of the
object exists, the object is not copied.  Return a tuple
\code{(\var{filename}, \var{headers})} where \var{filename} is the
local file name under which the object can be found, and \var{headers}
is either \code{None} (for a local object) or whatever the
\method{info()} method of the object returned by \function{urlopen()}
returned (for a remote object, possibly cached).  Exceptions are the
same as for \function{urlopen()}.

The second argument, if present, specifies the file location to copy
to (if absent, the location will be a tempfile with a generated name).
The third argument, if present, is a hook function that will be called
once on establishment of the network connection and once after each
block read thereafter.  The hook will be passed three arguments; a
count of blocks transferred so far, a block size in bytes, and the
total size of the file.  The third argument may be \code{-1} on older
FTP servers which do not return a file size in response to a retrieval 
request.

If the \var{url} uses the \file{http:} scheme identifier, the optional
\var{data} argument may be given to specify a \code{POST} request
(normally the request type is \code{GET}).  The \var{data} argument
must in standard \file{application/x-www-form-urlencoded} format;
see the \function{urlencode()} function below.
\end{funcdesc}

\begin{funcdesc}{urlcleanup}{}
Clear the cache that may have been built up by previous calls to
\function{urlretrieve()}.
\end{funcdesc}

\begin{funcdesc}{quote}{string\optional{, safe}}
Replace special characters in \var{string} using the \samp{\%xx} escape.
Letters, digits, and the characters \character{_,.-} are never quoted.
The optional \var{safe} parameter specifies additional characters
that should not be quoted --- its default value is \code{'/'}.

Example: \code{quote('/\~{}connolly/')} yields \code{'/\%7econnolly/'}.
\end{funcdesc}

\begin{funcdesc}{quote_plus}{string\optional{, safe}}
Like \function{quote()}, but also replaces spaces by plus signs, as
required for quoting HTML form values.  Plus signs in the original
string are escaped unless they are included in \var{safe}.
\end{funcdesc}

\begin{funcdesc}{unquote}{string}
Replace \samp{\%xx} escapes by their single-character equivalent.

Example: \code{unquote('/\%7Econnolly/')} yields \code{'/\~{}connolly/'}.
\end{funcdesc}

\begin{funcdesc}{unquote_plus}{string}
Like \function{unquote()}, but also replaces plus signs by spaces, as
required for unquoting HTML form values.
\end{funcdesc}

\begin{funcdesc}{urlencode}{dict}
Convert a dictionary to a ``url-encoded'' string, suitable to pass to
\function{urlopen()} above as the optional \var{data} argument.  This
is useful to pass a dictionary of form fields to a \code{POST}
request.  The resulting string is a series of
\code{\var{key}=\var{value}} pairs separated by \character{\&}
characters, where both \var{key} and \var{value} are quoted using
\function{quote_plus()} above.
\end{funcdesc}

The public functions \function{urlopen()} and \function{urlretrieve()}
create an instance of the \class{FancyURLopener} class and use it to perform
their requested actions.  To override this functionality, programmers can
create a subclass of \class{URLopener} or \class{FancyURLopener}, then
assign that class to the \var{urllib._urlopener} variable before calling the
desired function.  For example, applications may want to specify a different
\code{user-agent} header than \class{URLopener} defines.  This can be
accomplished with the following code:

\begin{verbatim}
class AppURLopener(urllib.FancyURLopener):
    def __init__(self, *args):
        apply(urllib.FancyURLopener.__init__, (self,) + args)
        self.version = "App/1.7"

urllib._urlopener = AppURLopener()
\end{verbatim}

\begin{classdesc}{URLopener}{\optional{proxies\optional{, **x509}}}
Base class for opening and reading URLs.  Unless you need to support
opening objects using schemes other than \file{http:}, \file{ftp:},
\file{gopher:} or \file{file:}, you probably want to use
\class{FancyURLopener}.

By default, the \class{URLopener} class sends a
\code{user-agent} header of \samp{urllib/\var{VVV}}, where
\var{VVV} is the \module{urllib} version number.  Applications can
define their own \code{user-agent} header by subclassing
\class{URLopener} or \class{FancyURLopener} and setting the instance
attribute \var{version} to an appropriate string value before the
\method{open()} method is called.

Additional keyword parameters, collected in \var{x509}, are used for
authentication with the \file{https:} scheme.  The keywords
\var{key_file} and \var{cert_file} are supported; both are needed to
actually retrieve a resource at an \file{https:} URL.
\end{classdesc}

\begin{classdesc}{FancyURLopener}{...}
\class{FancyURLopener} subclasses \class{URLopener} providing default
handling for the following HTTP response codes: 301, 302 or 401.  For
301 and 302 response codes, the \code{location} header is used to
fetch the actual URL.  For 401 response codes (authentication
required), basic HTTP authentication is performed.

The parameters to the constructor are the same as those for
\class{URLopener}.
\end{classdesc}

Restrictions:

\begin{itemize}

\item
Currently, only the following protocols are supported: HTTP, (versions
0.9 and 1.0), Gopher (but not Gopher-+), FTP, and local files.
\indexii{HTTP}{protocol}
\indexii{Gopher}{protocol}
\indexii{FTP}{protocol}

\item
The caching feature of \function{urlretrieve()} has been disabled
until I find the time to hack proper processing of Expiration time
headers.

\item
There should be a function to query whether a particular URL is in
the cache.

\item
For backward compatibility, if a URL appears to point to a local file
but the file can't be opened, the URL is re-interpreted using the FTP
protocol.  This can sometimes cause confusing error messages.

\item
The \function{urlopen()} and \function{urlretrieve()} functions can
cause arbitrarily long delays while waiting for a network connection
to be set up.  This means that it is difficult to build an interactive
web client using these functions without using threads.

\item
The data returned by \function{urlopen()} or \function{urlretrieve()}
is the raw data returned by the server.  This may be binary data
(e.g. an image), plain text or (for example) HTML\index{HTML}.  The
HTTP\indexii{HTTP}{protocol} protocol provides type information in the
reply header, which can be inspected by looking at the
\code{content-type} header.  For the Gopher\indexii{Gopher}{protocol}
protocol, type information is encoded in the URL; there is currently
no easy way to extract it.  If the returned data is HTML, you can use
the module \refmodule{htmllib}\refstmodindex{htmllib} to parse it.

\item
Although the \module{urllib} module contains (undocumented) routines
to parse and unparse URL strings, the recommended interface for URL
manipulation is in module \refmodule{urlparse}\refstmodindex{urlparse}.

\end{itemize}


\subsection{URLopener Objects \label{urlopener-objs}}
\sectionauthor{Skip Montanaro}{skip@mojam.com}

\class{URLopener} and \class{FancyURLopener} objects have the
following methods.

\begin{methoddesc}{open}{fullurl\optional{, data}}
Open \var{fullurl} using the appropriate protocol.  This method sets 
up cache and proxy information, then calls the appropriate open method with
its input arguments.  If the scheme is not recognized,
\method{open_unknown()} is called.  The \var{data} argument 
has the same meaning as the \var{data} argument of \function{urlopen()}.
\end{methoddesc}

\begin{methoddesc}{open_unknown}{fullurl\optional{, data}}
Overridable interface to open unknown URL types.
\end{methoddesc}

\begin{methoddesc}{retrieve}{url\optional{, filename\optional{, reporthook}}}
Retrieves the contents of \var{url} and places it in \var{filename}.  The
return value is a tuple consisting of a local filename and either a
\class{mimetools.Message} object containing the response headers (for remote
URLs) or None (for local URLs).  The caller must then open and read the
contents of \var{filename}.  If \var{filename} is not given and the URL
refers to a local file, the input filename is returned.  If the URL is
non-local and \var{filename} is not given, the filename is the output of
\function{tempfile.mktemp()} with a suffix that matches the suffix of the last
path component of the input URL.  If \var{reporthook} is given, it must be
a function accepting three numeric parameters.  It will be called after each
chunk of data is read from the network.  \var{reporthook} is ignored for
local URLs.

If the \var{url} uses the \file{http:} scheme identifier, the optional
\var{data} argument may be given to specify a \code{POST} request
(normally the request type is \code{GET}).  The \var{data} argument
must in standard \file{application/x-www-form-urlencoded} format;
see the \function{urlencode()} function below.
\end{methoddesc}


\subsection{Examples}
\nodename{Urllib Examples}

Here is an example session that uses the \samp{GET} method to retrieve
a URL containing parameters:

\begin{verbatim}
>>> import urllib
>>> params = urllib.urlencode({'spam': 1, 'eggs': 2, 'bacon': 0})
>>> f = urllib.urlopen("http://www.musi-cal.com/cgi-bin/query?%s" % params)
>>> print f.read()
\end{verbatim}

The following example uses the \samp{POST} method instead:

\begin{verbatim}
>>> import urllib
>>> params = urllib.urlencode({'spam': 1, 'eggs': 2, 'bacon': 0})
>>> f = urllib.urlopen("http://www.musi-cal.com/cgi-bin/query", params)
>>> print f.read()
\end{verbatim}

\section{\module{httplib} ---
         HTTP protocol client}

\declaremodule{standard}{httplib}
\modulesynopsis{HTTP and HTTPS protocol client (requires sockets).}

\indexii{HTTP}{protocol}
\index{HTTP!\module{httplib} (standard module)}

This module defines classes which implement the client side of the
HTTP and HTTPS protocols.  It is normally not used directly --- the
module \refmodule{urllib}\refstmodindex{urllib} uses it to handle URLs
that use HTTP and HTTPS.

\begin{notice}
  HTTPS support is only available if the \refmodule{socket} module was
  compiled with SSL support.
\end{notice}

\begin{notice}
  The public interface for this module changed substantially in Python
  2.0.  The \class{HTTP} class is retained only for backward
  compatibility with 1.5.2.  It should not be used in new code.  Refer
  to the online docstrings for usage.
\end{notice}

The module provides the following classes:

\begin{classdesc}{HTTPConnection}{host\optional{, port}}
An \class{HTTPConnection} instance represents one transaction with an HTTP
server.  It should be instantiated passing it a host and optional port number.
If no port number is passed, the port is extracted from the host string if it
has the form \code{\var{host}:\var{port}}, else the default HTTP port (80) is
used.  For example, the following calls all create instances that connect to
the server at the same host and port:

\begin{verbatim}
>>> h1 = httplib.HTTPConnection('www.cwi.nl')
>>> h2 = httplib.HTTPConnection('www.cwi.nl:80')
>>> h3 = httplib.HTTPConnection('www.cwi.nl', 80)
\end{verbatim}
\versionadded{2.0}
\end{classdesc}

\begin{classdesc}{HTTPSConnection}{host\optional{, port, key_file, cert_file}}
A subclass of \class{HTTPConnection} that uses SSL for communication with
secure servers.  Default port is \code{443}.
\var{key_file} is
the name of a PEM formatted file that contains your private
key. \var{cert_file} is a PEM formatted certificate chain file.

\warning{This does not do any certificate verification!}

\versionadded{2.0}
\end{classdesc}

\begin{classdesc}{HTTPResponse}{sock\optional{, debuglevel=0}\optional{, strict=0}}
Class whose instances are returned upon successful connection.  Not
instantiated directly by user.
\versionadded{2.0}
\end{classdesc}

The following exceptions are raised as appropriate:

\begin{excdesc}{HTTPException}
The base class of the other exceptions in this module.  It is a
subclass of \exception{Exception}.
\versionadded{2.0}
\end{excdesc}

\begin{excdesc}{NotConnected}
A subclass of \exception{HTTPException}.
\versionadded{2.0}
\end{excdesc}

\begin{excdesc}{InvalidURL}
A subclass of \exception{HTTPException}, raised if a port is given and is
either non-numeric or empty.
\versionadded{2.3}
\end{excdesc}

\begin{excdesc}{UnknownProtocol}
A subclass of \exception{HTTPException}.
\versionadded{2.0}
\end{excdesc}

\begin{excdesc}{UnknownTransferEncoding}
A subclass of \exception{HTTPException}.
\versionadded{2.0}
\end{excdesc}

\begin{excdesc}{UnimplementedFileMode}
A subclass of \exception{HTTPException}.
\versionadded{2.0}
\end{excdesc}

\begin{excdesc}{IncompleteRead}
A subclass of \exception{HTTPException}.
\versionadded{2.0}
\end{excdesc}

\begin{excdesc}{ImproperConnectionState}
A subclass of \exception{HTTPException}.
\versionadded{2.0}
\end{excdesc}

\begin{excdesc}{CannotSendRequest}
A subclass of \exception{ImproperConnectionState}.
\versionadded{2.0}
\end{excdesc}

\begin{excdesc}{CannotSendHeader}
A subclass of \exception{ImproperConnectionState}.
\versionadded{2.0}
\end{excdesc}

\begin{excdesc}{ResponseNotReady}
A subclass of \exception{ImproperConnectionState}.
\versionadded{2.0}
\end{excdesc}

\begin{excdesc}{BadStatusLine}
A subclass of \exception{HTTPException}.  Raised if a server responds with a
HTTP status code that we don't understand.
\versionadded{2.0}
\end{excdesc}

The constants defined in this module are:

\begin{datadesc}{HTTP_PORT}
  The default port for the HTTP protocol (always \code{80}).
\end{datadesc}

\begin{datadesc}{HTTPS_PORT}
  The default port for the HTTPS protocol (always \code{443}).
\end{datadesc}

and also the following constants for integer status codes:

\begin{tableiii}{l|c|l}{constant}{Constant}{Value}{Definition}
  \lineiii{CONTINUE}{\code{100}}
    {HTTP/1.1, \ulink{RFC 2616, Section 10.1.1}
      {http://www.w3.org/Protocols/rfc2616/rfc2616-sec10.html#sec10.1.1}}
  \lineiii{SWITCHING_PROTOCOLS}{\code{101}}
    {HTTP/1.1, \ulink{RFC 2616, Section 10.1.2}
      {http://www.w3.org/Protocols/rfc2616/rfc2616-sec10.html#sec10.1.2}}
  \lineiii{PROCESSING}{\code{102}}
    {WEBDAV, \ulink{RFC 2518, Section 10.1}
      {http://www.webdav.org/specs/rfc2518.html#STATUS_102}}

  \lineiii{OK}{\code{200}}
    {HTTP/1.1, \ulink{RFC 2616, Section 10.2.1}
      {http://www.w3.org/Protocols/rfc2616/rfc2616-sec10.html#sec10.2.1}}
  \lineiii{CREATED}{\code{201}}
    {HTTP/1.1, \ulink{RFC 2616, Section 10.2.2}
      {http://www.w3.org/Protocols/rfc2616/rfc2616-sec10.html#sec10.2.2}}
  \lineiii{ACCEPTED}{\code{202}}
    {HTTP/1.1, \ulink{RFC 2616, Section 10.2.3}
      {http://www.w3.org/Protocols/rfc2616/rfc2616-sec10.html#sec10.2.3}}
  \lineiii{NON_AUTHORITATIVE_INFORMATION}{\code{203}}
    {HTTP/1.1, \ulink{RFC 2616, Section 10.2.4}
      {http://www.w3.org/Protocols/rfc2616/rfc2616-sec10.html#sec10.2.4}}
  \lineiii{NO_CONTENT}{\code{204}}
    {HTTP/1.1, \ulink{RFC 2616, Section 10.2.5}
      {http://www.w3.org/Protocols/rfc2616/rfc2616-sec10.html#sec10.2.5}}
  \lineiii{RESET_CONTENT}{\code{205}}
    {HTTP/1.1, \ulink{RFC 2616, Section 10.2.6}
      {http://www.w3.org/Protocols/rfc2616/rfc2616-sec10.html#sec10.2.6}}
  \lineiii{PARTIAL_CONTENT}{\code{206}}
    {HTTP/1.1, \ulink{RFC 2616, Section 10.2.7}
      {http://www.w3.org/Protocols/rfc2616/rfc2616-sec10.html#sec10.2.7}}
  \lineiii{MULTI_STATUS}{\code{207}}
    {WEBDAV \ulink{RFC 2518, Section 10.2}
      {http://www.webdav.org/specs/rfc2518.html#STATUS_207}}
  \lineiii{IM_USED}{\code{226}}
    {Delta encoding in HTTP, \rfc{3229}, Section 10.4.1}

  \lineiii{MULTIPLE_CHOICES}{\code{300}}
    {HTTP/1.1, \ulink{RFC 2616, Section 10.3.1}
      {http://www.w3.org/Protocols/rfc2616/rfc2616-sec10.html#sec10.3.1}}
  \lineiii{MOVED_PERMANENTLY}{\code{301}}
    {HTTP/1.1, \ulink{RFC 2616, Section 10.3.2}
      {http://www.w3.org/Protocols/rfc2616/rfc2616-sec10.html#sec10.3.2}}
  \lineiii{FOUND}{\code{302}}
    {HTTP/1.1, \ulink{RFC 2616, Section 10.3.3}
      {http://www.w3.org/Protocols/rfc2616/rfc2616-sec10.html#sec10.3.3}}
  \lineiii{SEE_OTHER}{\code{303}}
    {HTTP/1.1, \ulink{RFC 2616, Section 10.3.4}
      {http://www.w3.org/Protocols/rfc2616/rfc2616-sec10.html#sec10.3.4}}
  \lineiii{NOT_MODIFIED}{\code{304}}
    {HTTP/1.1, \ulink{RFC 2616, Section 10.3.5}
      {http://www.w3.org/Protocols/rfc2616/rfc2616-sec10.html#sec10.3.5}}
  \lineiii{USE_PROXY}{\code{305}}
    {HTTP/1.1, \ulink{RFC 2616, Section 10.3.6}
      {http://www.w3.org/Protocols/rfc2616/rfc2616-sec10.html#sec10.3.6}}
  \lineiii{TEMPORARY_REDIRECT}{\code{307}}
    {HTTP/1.1, \ulink{RFC 2616, Section 10.3.8}
      {http://www.w3.org/Protocols/rfc2616/rfc2616-sec10.html#sec10.3.8}}

  \lineiii{BAD_REQUEST}{\code{400}}
    {HTTP/1.1, \ulink{RFC 2616, Section 10.4.1}
      {http://www.w3.org/Protocols/rfc2616/rfc2616-sec10.html#sec10.4.1}}
  \lineiii{UNAUTHORIZED}{\code{401}}
    {HTTP/1.1, \ulink{RFC 2616, Section 10.4.2}
      {http://www.w3.org/Protocols/rfc2616/rfc2616-sec10.html#sec10.4.2}}
  \lineiii{PAYMENT_REQUIRED}{\code{402}}
    {HTTP/1.1, \ulink{RFC 2616, Section 10.4.3}
      {http://www.w3.org/Protocols/rfc2616/rfc2616-sec10.html#sec10.4.3}}
  \lineiii{FORBIDDEN}{\code{403}}
    {HTTP/1.1, \ulink{RFC 2616, Section 10.4.4}
      {http://www.w3.org/Protocols/rfc2616/rfc2616-sec10.html#sec10.4.4}}
  \lineiii{NOT_FOUND}{\code{404}}
    {HTTP/1.1, \ulink{RFC 2616, Section 10.4.5}
      {http://www.w3.org/Protocols/rfc2616/rfc2616-sec10.html#sec10.4.5}}
  \lineiii{METHOD_NOT_ALLOWED}{\code{405}}
    {HTTP/1.1, \ulink{RFC 2616, Section 10.4.6}
      {http://www.w3.org/Protocols/rfc2616/rfc2616-sec10.html#sec10.4.6}}
  \lineiii{NOT_ACCEPTABLE}{\code{406}}
    {HTTP/1.1, \ulink{RFC 2616, Section 10.4.7}
      {http://www.w3.org/Protocols/rfc2616/rfc2616-sec10.html#sec10.4.7}}
  \lineiii{PROXY_AUTHENTICATION_REQUIRED}
    {\code{407}}{HTTP/1.1, \ulink{RFC 2616, Section 10.4.8}
      {http://www.w3.org/Protocols/rfc2616/rfc2616-sec10.html#sec10.4.8}}
  \lineiii{REQUEST_TIMEOUT}{\code{408}}
    {HTTP/1.1, \ulink{RFC 2616, Section 10.4.9}
      {http://www.w3.org/Protocols/rfc2616/rfc2616-sec10.html#sec10.4.9}}
  \lineiii{CONFLICT}{\code{409}}
    {HTTP/1.1, \ulink{RFC 2616, Section 10.4.10}
      {http://www.w3.org/Protocols/rfc2616/rfc2616-sec10.html#sec10.4.10}}
  \lineiii{GONE}{\code{410}}
    {HTTP/1.1, \ulink{RFC 2616, Section 10.4.11}
      {http://www.w3.org/Protocols/rfc2616/rfc2616-sec10.html#sec10.4.11}}
  \lineiii{LENGTH_REQUIRED}{\code{411}}
    {HTTP/1.1, \ulink{RFC 2616, Section 10.4.12}
      {http://www.w3.org/Protocols/rfc2616/rfc2616-sec10.html#sec10.4.12}}
  \lineiii{PRECONDITION_FAILED}{\code{412}}
    {HTTP/1.1, \ulink{RFC 2616, Section 10.4.13}
      {http://www.w3.org/Protocols/rfc2616/rfc2616-sec10.html#sec10.4.13}}
  \lineiii{REQUEST_ENTITY_TOO_LARGE}
    {\code{413}}{HTTP/1.1, \ulink{RFC 2616, Section 10.4.14}
      {http://www.w3.org/Protocols/rfc2616/rfc2616-sec10.html#sec10.4.14}}
  \lineiii{REQUEST_URI_TOO_LONG}{\code{414}}
    {HTTP/1.1, \ulink{RFC 2616, Section 10.4.15}
      {http://www.w3.org/Protocols/rfc2616/rfc2616-sec10.html#sec10.4.15}}
  \lineiii{UNSUPPORTED_MEDIA_TYPE}{\code{415}}
    {HTTP/1.1, \ulink{RFC 2616, Section 10.4.16}
      {http://www.w3.org/Protocols/rfc2616/rfc2616-sec10.html#sec10.4.16}}
  \lineiii{REQUESTED_RANGE_NOT_SATISFIABLE}{\code{416}}
    {HTTP/1.1, \ulink{RFC 2616, Section 10.4.17}
      {http://www.w3.org/Protocols/rfc2616/rfc2616-sec10.html#sec10.4.17}}
  \lineiii{EXPECTATION_FAILED}{\code{417}}
    {HTTP/1.1, \ulink{RFC 2616, Section 10.4.18}
      {http://www.w3.org/Protocols/rfc2616/rfc2616-sec10.html#sec10.4.18}}
  \lineiii{UNPROCESSABLE_ENTITY}{\code{422}}
    {WEBDAV, \ulink{RFC 2518, Section 10.3}
      {http://www.webdav.org/specs/rfc2518.html#STATUS_422}}
  \lineiii{LOCKED}{\code{423}}
    {WEBDAV \ulink{RFC 2518, Section 10.4}
      {http://www.webdav.org/specs/rfc2518.html#STATUS_423}}
  \lineiii{FAILED_DEPENDENCY}{\code{424}}
    {WEBDAV, \ulink{RFC 2518, Section 10.5}
      {http://www.webdav.org/specs/rfc2518.html#STATUS_424}}
  \lineiii{UPGRADE_REQUIRED}{\code{426}}
    {HTTP Upgrade to TLS, \rfc{2817}, Section 6}

  \lineiii{INTERNAL_SERVER_ERROR}{\code{500}}
    {HTTP/1.1, \ulink{RFC 2616, Section 10.5.1}
      {http://www.w3.org/Protocols/rfc2616/rfc2616-sec10.html#sec10.5.1}}
  \lineiii{NOT_IMPLEMENTED}{\code{501}}
    {HTTP/1.1, \ulink{RFC 2616, Section 10.5.2}
      {http://www.w3.org/Protocols/rfc2616/rfc2616-sec10.html#sec10.5.2}}
  \lineiii{BAD_GATEWAY}{\code{502}}
    {HTTP/1.1 \ulink{RFC 2616, Section 10.5.3}
      {http://www.w3.org/Protocols/rfc2616/rfc2616-sec10.html#sec10.5.3}}
  \lineiii{SERVICE_UNAVAILABLE}{\code{503}}
    {HTTP/1.1, \ulink{RFC 2616, Section 10.5.4}
      {http://www.w3.org/Protocols/rfc2616/rfc2616-sec10.html#sec10.5.4}}
  \lineiii{GATEWAY_TIMEOUT}{\code{504}}
    {HTTP/1.1 \ulink{RFC 2616, Section 10.5.5}
      {http://www.w3.org/Protocols/rfc2616/rfc2616-sec10.html#sec10.5.5}}
  \lineiii{HTTP_VERSION_NOT_SUPPORTED}{\code{505}}
    {HTTP/1.1, \ulink{RFC 2616, Section 10.5.6}
      {http://www.w3.org/Protocols/rfc2616/rfc2616-sec10.html#sec10.5.6}}
  \lineiii{INSUFFICIENT_STORAGE}{\code{507}}
    {WEBDAV, \ulink{RFC 2518, Section 10.6}
      {http://www.webdav.org/specs/rfc2518.html#STATUS_507}}
  \lineiii{NOT_EXTENDED}{\code{510}}
    {An HTTP Extension Framework, \rfc{2774}, Section 7}
\end{tableiii}

\subsection{HTTPConnection Objects \label{httpconnection-objects}}

\class{HTTPConnection} instances have the following methods:

\begin{methoddesc}{request}{method, url\optional{, body\optional{, headers}}}
This will send a request to the server using the HTTP request method
\var{method} and the selector \var{url}.  If the \var{body} argument is
present, it should be a string of data to send after the headers are finished.
The header Content-Length is automatically set to the correct value.
The \var{headers} argument should be a mapping of extra HTTP headers to send
with the request.
\end{methoddesc}

\begin{methoddesc}{getresponse}{}
Should be called after a request is sent to get the response from the server.
Returns an \class{HTTPResponse} instance.
\note{Note that you must have read the whole response before you can send a new
request to the server.}
\end{methoddesc}

\begin{methoddesc}{set_debuglevel}{level}
Set the debugging level (the amount of debugging output printed).
The default debug level is \code{0}, meaning no debugging output is
printed.
\end{methoddesc}

\begin{methoddesc}{connect}{}
Connect to the server specified when the object was created.
\end{methoddesc}

\begin{methoddesc}{close}{}
Close the connection to the server.
\end{methoddesc}

As an alternative to using the \method{request()} method described above,
you can also send your request step by step, by using the four functions
below.

\begin{methoddesc}{putrequest}{request, selector\optional{,
skip\_host\optional{, skip_accept_encoding}}}
This should be the first call after the connection to the server has
been made.  It sends a line to the server consisting of the
\var{request} string, the \var{selector} string, and the HTTP version
(\code{HTTP/1.1}).  To disable automatic sending of \code{Host:} or
\code{Accept-Encoding:} headers (for example to accept additional
content encodings), specify \var{skip_host} or \var{skip_accept_encoding}
with non-False values.
\versionchanged[\var{skip_accept_encoding} argument added]{2.4}
\end{methoddesc}

\begin{methoddesc}{putheader}{header, argument\optional{, ...}}
Send an \rfc{822}-style header to the server.  It sends a line to the
server consisting of the header, a colon and a space, and the first
argument.  If more arguments are given, continuation lines are sent,
each consisting of a tab and an argument.
\end{methoddesc}

\begin{methoddesc}{endheaders}{}
Send a blank line to the server, signalling the end of the headers.
\end{methoddesc}

\begin{methoddesc}{send}{data}
Send data to the server.  This should be used directly only after the
\method{endheaders()} method has been called and before
\method{getresponse()} is called.
\end{methoddesc}

\subsection{HTTPResponse Objects \label{httpresponse-objects}}

\class{HTTPResponse} instances have the following methods and attributes:

\begin{methoddesc}{read}{\optional{amt}}
Reads and returns the response body, or up to the next \var{amt} bytes.
\end{methoddesc}

\begin{methoddesc}{getheader}{name\optional{, default}}
Get the contents of the header \var{name}, or \var{default} if there is no
matching header.
\end{methoddesc}

\begin{methoddesc}{getheaders}{}
Return a list of (header, value) tuples. \versionadded{2.4}
\end{methoddesc}

\begin{datadesc}{msg}
  A \class{mimetools.Message} instance containing the response headers.
\end{datadesc}

\begin{datadesc}{version}
  HTTP protocol version used by server.  10 for HTTP/1.0, 11 for HTTP/1.1.
\end{datadesc}

\begin{datadesc}{status}
  Status code returned by server.
\end{datadesc}

\begin{datadesc}{reason}
  Reason phrase returned by server.
\end{datadesc}


\subsection{Examples \label{httplib-examples}}

Here is an example session that uses the \samp{GET} method:

\begin{verbatim}
>>> import httplib
>>> conn = httplib.HTTPConnection("www.python.org")
>>> conn.request("GET", "/index.html")
>>> r1 = conn.getresponse()
>>> print r1.status, r1.reason
200 OK
>>> data1 = r1.read()
>>> conn.request("GET", "/parrot.spam")
>>> r2 = conn.getresponse()
>>> print r2.status, r2.reason
404 Not Found
>>> data2 = r2.read()
>>> conn.close()
\end{verbatim}

Here is an example session that shows how to \samp{POST} requests:

\begin{verbatim}
>>> import httplib, urllib
>>> params = urllib.urlencode({'spam': 1, 'eggs': 2, 'bacon': 0})
>>> headers = {"Content-type": "application/x-www-form-urlencoded",
...            "Accept": "text/plain"}
>>> conn = httplib.HTTPConnection("musi-cal.mojam.com:80")
>>> conn.request("POST", "/cgi-bin/query", params, headers)
>>> response = conn.getresponse()
>>> print response.status, response.reason
200 OK
>>> data = response.read()
>>> conn.close()
\end{verbatim}

\section{Standard Module \module{ftplib}}
\label{module-ftplib}
\stmodindex{ftplib}
\indexii{FTP}{protocol}


This module defines the class \class{FTP} and a few related items.
The \class{FTP} class implements the client side of the FTP protocol.
You can use this to write Python programs that perform a variety of
automated FTP jobs, such as mirroring other ftp servers.  It is also
used by the module \module{urllib} to handle URLs that use FTP.  For
more information on FTP (File Transfer Protocol), see Internet
\rfc{959}.

Here's a sample session using the \module{ftplib} module:

\begin{verbatim}
>>> from ftplib import FTP
>>> ftp = FTP('ftp.cwi.nl')   # connect to host, default port
>>> ftp.login()               # user anonymous, passwd user@hostname
>>> ftp.retrlines('LIST')     # list directory contents
total 24418
drwxrwsr-x   5 ftp-usr  pdmaint     1536 Mar 20 09:48 .
dr-xr-srwt 105 ftp-usr  pdmaint     1536 Mar 21 14:32 ..
-rw-r--r--   1 ftp-usr  pdmaint     5305 Mar 20 09:48 INDEX
 .
 .
 .
>>> ftp.quit()
\end{verbatim}

The module defines the following items:

\begin{classdesc}{FTP}{\optional{host\optional{, user\optional{,
                       passwd\optional{, acct}}}}}
Return a new instance of the \class{FTP} class.  When
\var{host} is given, the method call \code{connect(\var{host})} is
made.  When \var{user} is given, additionally the method call
\code{login(\var{user}, \var{passwd}, \var{acct})} is made (where
\var{passwd} and \var{acct} default to the empty string when not given).
\end{classdesc}

\begin{datadesc}{all_errors}
The set of all exceptions (as a tuple) that methods of \class{FTP}
instances may raise as a result of problems with the FTP connection
(as opposed to programming errors made by the caller).  This set
includes the four exceptions listed below as well as
\exception{socket.error} and \exception{IOError}.
\end{datadesc}

\begin{excdesc}{error_reply}
Exception raised when an unexpected reply is received from the server.
\end{excdesc}

\begin{excdesc}{error_temp}
Exception raised when an error code in the range 400--499 is received.
\end{excdesc}

\begin{excdesc}{error_perm}
Exception raised when an error code in the range 500--599 is received.
\end{excdesc}

\begin{excdesc}{error_proto}
Exception raised when a reply is received from the server that does
not begin with a digit in the range 1--5.
\end{excdesc}


\subsection{FTP Objects}
\label{ftp-objects}

\class{FTP} instances have the following methods:

\begin{methoddesc}{set_debuglevel}{level}
Set the instance's debugging level.  This controls the amount of
debugging output printed.  The default, \code{0}, produces no
debugging output.  A value of \code{1} produces a moderate amount of
debugging output, generally a single line per request.  A value of
\code{2} or higher produces the maximum amount of debugging output,
logging each line sent and received on the control connection.
\end{methoddesc}

\begin{methoddesc}{connect}{host\optional{, port}}
Connect to the given host and port.  The default port number is \code{21}, as
specified by the FTP protocol specification.  It is rarely needed to
specify a different port number.  This function should be called only
once for each instance; it should not be called at all if a host was
given when the instance was created.  All other methods can only be
used after a connection has been made.
\end{methoddesc}

\begin{methoddesc}{getwelcome}{}
Return the welcome message sent by the server in reply to the initial
connection.  (This message sometimes contains disclaimers or help
information that may be relevant to the user.)
\end{methoddesc}

\begin{methoddesc}{login}{\optional{user\optional{, passwd\optional{, acct}}}}
Log in as the given \var{user}.  The \var{passwd} and \var{acct}
parameters are optional and default to the empty string.  If no
\var{user} is specified, it defaults to \code{'anonymous'}.  If
\var{user} is \code{anonymous}, the default \var{passwd} is
\samp{\var{realuser}@\var{host}} where \var{realuser} is the real user
name (glanced from the \envvar{LOGNAME} or \envvar{USER} environment
variable) and \var{host} is the hostname as returned by
\function{socket.gethostname()}.  This function should be called only
once for each instance, after a connection has been established; it
should not be called at all if a host and user were given when the
instance was created.  Most FTP commands are only allowed after the
client has logged in.
\end{methoddesc}

\begin{methoddesc}{abort}{}
Abort a file transfer that is in progress.  Using this does not always
work, but it's worth a try.
\end{methoddesc}

\begin{methoddesc}{sendcmd}{command}
Send a simple command string to the server and return the response
string.
\end{methoddesc}

\begin{methoddesc}{voidcmd}{command}
Send a simple command string to the server and handle the response.
Return nothing if a response code in the range 200--299 is received.
Raise an exception otherwise.
\end{methoddesc}

\begin{methoddesc}{retrbinary}{command, callback\optional{, maxblocksize}}
Retrieve a file in binary transfer mode.  \var{command} should be an
appropriate \samp{RETR} command, i.e.\ \code{'RETR \var{filename}'}.
The \var{callback} function is called for each block of data received,
with a single string argument giving the data block.
The optional \var{maxblocksize} argument specifies the maximum chunk size to
read on the low-level socket object created to do the actual transfer
(which will also be the largest size of the data blocks passed to
\var{callback}).  A reasonable default is chosen.
\end{methoddesc}

\begin{methoddesc}{retrlines}{command\optional{, callback}}
Retrieve a file or directory listing in \ASCII{} transfer mode.
\var{command} should be an appropriate \samp{RETR} command (see
\method{retrbinary()} or a \samp{LIST} command (usually just the string
\code{'LIST'}).  The \var{callback} function is called for each line,
with the trailing CRLF stripped.  The default \var{callback} prints
the line to \code{sys.stdout}.
\end{methoddesc}

\begin{methoddesc}{storbinary}{command, file, blocksize}
Store a file in binary transfer mode.  \var{command} should be an
appropriate \samp{STOR} command, i.e.\ \code{"STOR \var{filename}"}.
\var{file} is an open file object which is read until \EOF{} using its
\method{read()} method in blocks of size \var{blocksize} to provide the
data to be stored.
\end{methoddesc}

\begin{methoddesc}{storlines}{command, file}
Store a file in \ASCII{} transfer mode.  \var{command} should be an
appropriate \samp{STOR} command (see \method{storbinary()}).  Lines are
read until \EOF{} from the open file object \var{file} using its
\method{readline()} method to privide the data to be stored.
\end{methoddesc}

\begin{methoddesc}{transfercmd}{cmd}
Initiate a transfer over the data connection.  If the transfer is
active, send a \samp{PORT} command and the transfer command specified
by \var{cmd}, and accept the connection.  If the server is passive,
send a \samp{PASV} command, connect to it, and start the transfer
command.  Either way, return the socket for the connection.
\end{methoddesc}

\begin{methoddesc}{ntransfercmd}{cmd}
Like \method{transfercmd()}, but returns a tuple of the data
connection and the expected size of the data.  If the expected size
could not be computed, \code{None} will be returned as the expected
size.
\end{methoddesc}

\begin{methoddesc}{nlst}{argument\optional{, \ldots}}
Return a list of files as returned by the \samp{NLST} command.  The
optional \var{argument} is a directory to list (default is the current
server directory).  Multiple arguments can be used to pass
non-standard options to the \samp{NLST} command.
\end{methoddesc}

\begin{methoddesc}{dir}{argument\optional{, \ldots}}
Return a directory listing as returned by the \samp{LIST} command, as
a list of lines.  The optional \var{argument} is a directory to list
(default is the current server directory).  Multiple arguments can be
used to pass non-standard options to the \samp{LIST} command.  If the
last argument is a function, it is used as a \var{callback} function
as for \method{retrlines()}.
\end{methoddesc}

\begin{methoddesc}{rename}{fromname, toname}
Rename file \var{fromname} on the server to \var{toname}.
\end{methoddesc}

\begin{methoddesc}{delete}{filename}
Remove the file named \var{filename} from the server.  If successful,
returns the text of the response, otherwise raises
\exception{error_perm} on permission errors or \exception{error_reply} 
on other errors.
\end{methoddesc}

\begin{methoddesc}{cwd}{pathname}
Set the current directory on the server.
\end{methoddesc}

\begin{methoddesc}{mkd}{pathname}
Create a new directory on the server.
\end{methoddesc}

\begin{methoddesc}{pwd}{}
Return the pathname of the current directory on the server.
\end{methoddesc}

\begin{methoddesc}{rmd}{dirname}
Remove the directory named \var{dirname} on the server.
\end{methoddesc}

\begin{methoddesc}{size}{filename}
Request the size of the file named \var{filename} on the server.  On
success, the size of the file is returned as an integer, otherwise
\code{None} is returned.  Note that the \samp{SIZE} command is not 
standardized, but is supported by many common server implementations.
\end{methoddesc}

\begin{methoddesc}{quit}{}
Send a \samp{QUIT} command to the server and close the connection.
This is the ``polite'' way to close a connection, but it may raise an
exception of the server reponds with an error to the \samp{QUIT}
command.
\end{methoddesc}

\begin{methoddesc}{close}{}
Close the connection unilaterally.  This should not be applied to an
already closed connection (e.g.\ after a successful call to
\method{quit()}.
\end{methoddesc}

\section{\module{gopherlib} ---
         Gopher protocol client}

\declaremodule{standard}{gopherlib}
\modulesynopsis{Gopher protocol client (requires sockets).}

\deprecated{2.5}{The \code{gopher} protocol is not in active use
                 anymore.}

\indexii{Gopher}{protocol}

This module provides a minimal implementation of client side of the
Gopher protocol.  It is used by the module \refmodule{urllib} to
handle URLs that use the Gopher protocol.

The module defines the following functions:

\begin{funcdesc}{send_selector}{selector, host\optional{, port}}
Send a \var{selector} string to the gopher server at \var{host} and
\var{port} (default \code{70}).  Returns an open file object from
which the returned document can be read.
\end{funcdesc}

\begin{funcdesc}{send_query}{selector, query, host\optional{, port}}
Send a \var{selector} string and a \var{query} string to a gopher
server at \var{host} and \var{port} (default \code{70}).  Returns an
open file object from which the returned document can be read.
\end{funcdesc}

Note that the data returned by the Gopher server can be of any type,
depending on the first character of the selector string.  If the data
is text (first character of the selector is \samp{0}), lines are
terminated by CRLF, and the data is terminated by a line consisting of
a single \samp{.}, and a leading \samp{.} should be stripped from
lines that begin with \samp{..}.  Directory listings (first character
of the selector is \samp{1}) are transferred using the same protocol.

\section{\module{poplib} ---
         POP3 protocol client}

\declaremodule{standard}{poplib}
\modulesynopsis{POP3 protocol client (requires sockets).}

%By Andrew T. Csillag
%Even though I put it into LaTeX, I cannot really claim that I wrote
%it since I just stole most of it from the poplib.py source code and
%the imaplib ``chapter''.
%Revised by ESR, January 2000

\indexii{POP3}{protocol}

This module defines a class, \class{POP3}, which encapsulates a
connection to an POP3 server and implements the protocol as defined in
\rfc{1725}.  The \class{POP3} class supports both the minimal and
optional command sets.

Note that POP3, though widely supported, is obsolescent.  The
implementation quality of POP3 servers varies widely, and too many are
quite poor. If your mailserver supports IMAP, you would be better off
using the \code{\refmodule{imaplib}.\class{IMAP4}} class, as IMAP
servers tend to be better implemented.

A single class is provided by the \module{poplib} module:

\begin{classdesc}{POP3}{host\optional{, port}}
This class implements the actual POP3 protocol.  The connection is
created when the instance is initialized.
If \var{port} is omitted, the standard POP3 port (110) is used.
\end{classdesc}

One exception is defined as an attribute of the \module{poplib} module:

\begin{excdesc}{error_proto}
Exception raised on any errors.  The reason for the exception is
passed to the constructor as a string.
\end{excdesc}

\begin{seealso}
  \seemodule{imaplib}{The standard Python IMAP module.}
  \seetitle[http://www.tuxedo.org/\~{}esr/fetchmail/fetchmail-FAQ.html]
        {Frequently Asked Questions About Fetchmail}
        {The FAQ for the \program{fetchmail} POP/IMAP client collects
         information on POP3 server variations and RFC noncompliance
         that may be useful if you need to write an application based
         on the POP protocol.}
\end{seealso}


\subsection{POP3 Objects \label{pop3-objects}}

All POP3 commands are represented by methods of the same name,
in lower-case; most return the response text sent by the server.

An \class{POP3} instance has the following methods:


\begin{methoddesc}{set_debuglevel}{level}
Set the instance's debugging level.  This controls the amount of
debugging output printed.  The default, \code{0}, produces no
debugging output.  A value of \code{1} produces a moderate amount of
debugging output, generally a single line per request.  A value of
\code{2} or higher produces the maximum amount of debugging output,
logging each line sent and received on the control connection.
\end{methoddesc}

\begin{methoddesc}{getwelcome}{}
Returns the greeting string sent by the POP3 server.
\end{methoddesc}

\begin{methoddesc}{user}{username}
Send user command, response should indicate that a password is required.
\end{methoddesc}

\begin{methoddesc}{pass_}{password}
Send password, response includes message count and mailbox size.
Note: the mailbox on the server is locked until \method{quit()} is
called.
\end{methoddesc}

\begin{methoddesc}{apop}{user, secret}
Use the more secure APOP authentication to log into the POP3 server.
\end{methoddesc}

\begin{methoddesc}{rpop}{user}
Use RPOP authentication (similar to UNIX r-commands) to log into POP3 server.
\end{methoddesc}

\begin{methoddesc}{stat}{}
Get mailbox status.  The result is a tuple of 2 integers:
\code{(\var{message count}, \var{mailbox size})}.
\end{methoddesc}

\begin{methoddesc}{list}{\optional{which}}
Request message list, result is in the form
\code{(\var{response}, ['mesg_num octets', ...])}.  If \var{which} is
set, it is the message to list.
\end{methoddesc}

\begin{methoddesc}{retr}{which}
Retrieve whole message number \var{which}, and set its seen flag.
Result is in form  \code{(\var{response}, ['line', ...], \var{octets})}.
\end{methoddesc}

\begin{methoddesc}{dele}{which}
Flag message number \var{which} for deletion.  On most servers
deletions are not actually performed until QUIT (the major exception is
Eudora QPOP, which deliberately violates the RFCs by doing pending
deletes on any disconnect).
\end{methoddesc}

\begin{methoddesc}{rset}{}
Remove any deletion marks for the mailbox.
\end{methoddesc}

\begin{methoddesc}{noop}{}
Do nothing.  Might be used as a keep-alive.
\end{methoddesc}

\begin{methoddesc}{quit}{}
Signoff:  commit changes, unlock mailbox, drop connection.
\end{methoddesc}

\begin{methoddesc}{top}{which, howmuch}
Retrieves the message header plus \var{howmuch} lines of the message
after the header of message number \var{which}. Result is in form
\code{(\var{response}, ['line', ...], \var{octets})}.

The POP3 TOP command this method uses, unlike the RETR command,
doesn't set the message's seen flag; unfortunately, TOP is poorly
specified in the RFCs and is frequently broken in off-brand servers.
Test this method by hand against the POP3 servers you will use before
trusting it.
\end{methoddesc}

\begin{methoddesc}{uidl}{\optional{which}}
Return message digest (unique id) list.
If \var{which} is specified, result contains the unique id for that
message in the form \code{'\var{response}\ \var{mesgnum}\ \var{uid}},
otherwise result is list \code{(\var{response}, ['mesgnum uid', ...],
\var{octets})}.
\end{methoddesc}


\subsection{POP3 Example \label{pop3-example}}

Here is a minimal example (without error checking) that opens a
mailbox and retrieves and prints all messages:

\begin{verbatim}
import getpass, poplib

M = poplib.POP3('localhost')
M.user(getpass.getuser())
M.pass_(getpass.getpass())
numMessages = len(M.list()[1])
for i in range(numMessages):
    for j in M.retr(i+1)[1]:
        print j
\end{verbatim}

At the end of the module, there is a test section that contains a more
extensive example of usage.

\section{\module{imaplib} ---
         IMAP4 protocol client}

\declaremodule{standard}{imaplib}
\modulesynopsis{IMAP4 protocol client (requires sockets).}
\moduleauthor{Piers Lauder}{piers@communitysolutions.com.au}
\sectionauthor{Piers Lauder}{piers@communitysolutions.com.au}

% Based on HTML documentation by Piers Lauder <piers@communitysolutions.com.au>;
% converted by Fred L. Drake, Jr. <fdrake@acm.org>.
% Revised by ESR, January 2000.
% Changes for IMAP4_SSL by Tino Lange <Tino.Lange@isg.de>, March 2002 
% Changes for IMAP4_stream by Piers Lauder <piers@communitysolutions.com.au>, November 2002 

\indexii{IMAP4}{protocol}
\indexii{IMAP4_SSL}{protocol}
\indexii{IMAP4_stream}{protocol}

This module defines three classes, \class{IMAP4}, \class{IMAP4_SSL} and \class{IMAP4_stream}, which encapsulate a
connection to an IMAP4 server and implement a large subset of the
IMAP4rev1 client protocol as defined in \rfc{2060}. It is backward
compatible with IMAP4 (\rfc{1730}) servers, but note that the
\samp{STATUS} command is not supported in IMAP4.

Three classes are provided by the \module{imaplib} module, \class{IMAP4} is the base class:

\begin{classdesc}{IMAP4}{\optional{host\optional{, port}}}
This class implements the actual IMAP4 protocol.  The connection is
created and protocol version (IMAP4 or IMAP4rev1) is determined when
the instance is initialized.
If \var{host} is not specified, \code{''} (the local host) is used.
If \var{port} is omitted, the standard IMAP4 port (143) is used.
\end{classdesc}

Three exceptions are defined as attributes of the \class{IMAP4} class:

\begin{excdesc}{IMAP4.error}
Exception raised on any errors.  The reason for the exception is
passed to the constructor as a string.
\end{excdesc}

\begin{excdesc}{IMAP4.abort}
IMAP4 server errors cause this exception to be raised.  This is a
sub-class of \exception{IMAP4.error}.  Note that closing the instance
and instantiating a new one will usually allow recovery from this
exception.
\end{excdesc}

\begin{excdesc}{IMAP4.readonly}
This exception is raised when a writable mailbox has its status changed by the server.  This is a
sub-class of \exception{IMAP4.error}.  Some other client now has write permission,
and the mailbox will need to be re-opened to re-obtain write permission.
\end{excdesc}

There's also a subclass for secure connections:

\begin{classdesc}{IMAP4_SSL}{\optional{host\optional{, port\optional{, keyfile\optional{, certfile}}}}}
This is a subclass derived from \class{IMAP4} that connects over an SSL encrypted socket 
(to use this class you need a socket module that was compiled with SSL support).
If \var{host} is not specified, \code{''} (the local host) is used.
If \var{port} is omitted, the standard IMAP4-over-SSL port (993) is used.
\var{keyfile} and \var{certfile} are also optional - they can contain a PEM formatted
private key and certificate chain file for the SSL connection. 
\end{classdesc}

The second subclass allows for connections created by a child process:

\begin{classdesc}{IMAP4_stream}{command}
This is a subclass derived from \class{IMAP4} that connects
to the \code{stdin/stdout} file descriptors created by passing \var{command} to \code{os.popen2()}.
\versionadded{2.3}
\end{classdesc}

The following utility functions are defined:

\begin{funcdesc}{Internaldate2tuple}{datestr}
  Converts an IMAP4 INTERNALDATE string to Coordinated Universal
  Time. Returns a \refmodule{time} module tuple.
\end{funcdesc}

\begin{funcdesc}{Int2AP}{num}
  Converts an integer into a string representation using characters
  from the set [\code{A} .. \code{P}].
\end{funcdesc}

\begin{funcdesc}{ParseFlags}{flagstr}
  Converts an IMAP4 \samp{FLAGS} response to a tuple of individual
  flags.
\end{funcdesc}

\begin{funcdesc}{Time2Internaldate}{date_time}
  Converts a \refmodule{time} module tuple to an IMAP4
  \samp{INTERNALDATE} representation.  Returns a string in the form:
  \code{"DD-Mmm-YYYY HH:MM:SS +HHMM"} (including double-quotes).
\end{funcdesc}


Note that IMAP4 message numbers change as the mailbox changes; in
particular, after an \samp{EXPUNGE} command performs deletions the
remaining messages are renumbered. So it is highly advisable to use
UIDs instead, with the UID command.

At the end of the module, there is a test section that contains a more
extensive example of usage.

\begin{seealso}
  \seetext{Documents describing the protocol, and sources and binaries 
           for servers implementing it, can all be found at the
           University of Washington's \emph{IMAP Information Center}
           (\url{http://www.cac.washington.edu/imap/}).}
\end{seealso}


\subsection{IMAP4 Objects \label{imap4-objects}}

All IMAP4rev1 commands are represented by methods of the same name,
either upper-case or lower-case.

All arguments to commands are converted to strings, except for
\samp{AUTHENTICATE}, and the last argument to \samp{APPEND} which is
passed as an IMAP4 literal.  If necessary (the string contains IMAP4
protocol-sensitive characters and isn't enclosed with either
parentheses or double quotes) each string is quoted. However, the
\var{password} argument to the \samp{LOGIN} command is always quoted.
If you want to avoid having an argument string quoted
(eg: the \var{flags} argument to \samp{STORE}) then enclose the string in
parentheses (eg: \code{r'(\e Deleted)'}).

Each command returns a tuple: \code{(\var{type}, [\var{data},
...])} where \var{type} is usually \code{'OK'} or \code{'NO'},
and \var{data} is either the text from the command response, or
mandated results from the command. Each \var{data}
is either a string, or a tuple. If a tuple, then the first part
is the header of the response, and the second part contains
the data (ie: 'literal' value).

An \class{IMAP4} instance has the following methods:


\begin{methoddesc}{append}{mailbox, flags, date_time, message}
  Append \var{message} to named mailbox. 
\end{methoddesc}

\begin{methoddesc}{authenticate}{mechanism, authobject}
  Authenticate command --- requires response processing.

  \var{mechanism} specifies which authentication mechanism is to
  be used - it should appear in the instance variable \code{capabilities} in the
  form \code{AUTH=mechanism}.

  \var{authobject} must be a callable object:

\begin{verbatim}
data = authobject(response)
\end{verbatim}

  It will be called to process server continuation responses.
  It should return \code{data} that will be encoded and sent to server.
  It should return \code{None} if the client abort response \samp{*} should
  be sent instead.
\end{methoddesc}

\begin{methoddesc}{check}{}
  Checkpoint mailbox on server. 
\end{methoddesc}

\begin{methoddesc}{close}{}
  Close currently selected mailbox. Deleted messages are removed from
  writable mailbox. This is the recommended command before
  \samp{LOGOUT}.
\end{methoddesc}

\begin{methoddesc}{copy}{message_set, new_mailbox}
  Copy \var{message_set} messages onto end of \var{new_mailbox}. 
\end{methoddesc}

\begin{methoddesc}{create}{mailbox}
  Create new mailbox named \var{mailbox}.
\end{methoddesc}

\begin{methoddesc}{delete}{mailbox}
  Delete old mailbox named \var{mailbox}.
\end{methoddesc}

\begin{methoddesc}{deleteacl}{mailbox, who}
  Delete the ACLs (remove any rights) set for who on mailbox.
\versionadded{2.4}
\end{methoddesc}

\begin{methoddesc}{expunge}{}
  Permanently remove deleted items from selected mailbox. Generates an
  \samp{EXPUNGE} response for each deleted message. Returned data
  contains a list of \samp{EXPUNGE} message numbers in order
  received.
\end{methoddesc}

\begin{methoddesc}{fetch}{message_set, message_parts}
  Fetch (parts of) messages.  \var{message_parts} should be
  a string of message part names enclosed within parentheses,
  eg: \samp{"(UID BODY[TEXT])"}.  Returned data are tuples
  of message part envelope and data.
\end{methoddesc}

\begin{methoddesc}{getacl}{mailbox}
  Get the \samp{ACL}s for \var{mailbox}.
  The method is non-standard, but is supported by the \samp{Cyrus} server.
\end{methoddesc}

\begin{methoddesc}{getquota}{root}
  Get the \samp{quota} \var{root}'s resource usage and limits.
  This method is part of the IMAP4 QUOTA extension defined in rfc2087.
\versionadded{2.3}
\end{methoddesc}

\begin{methoddesc}{getquotaroot}{mailbox}
  Get the list of \samp{quota} \samp{roots} for the named \var{mailbox}.
  This method is part of the IMAP4 QUOTA extension defined in rfc2087.
\versionadded{2.3}
\end{methoddesc}

\begin{methoddesc}{list}{\optional{directory\optional{, pattern}}}
  List mailbox names in \var{directory} matching
  \var{pattern}.  \var{directory} defaults to the top-level mail
  folder, and \var{pattern} defaults to match anything.  Returned data
  contains a list of \samp{LIST} responses.
\end{methoddesc}

\begin{methoddesc}{login}{user, password}
  Identify the client using a plaintext password.
  The \var{password} will be quoted.
\end{methoddesc}

\begin{methoddesc}{login_cram_md5}{user, password}
  Force use of \samp{CRAM-MD5} authentication when identifying the client to protect the password.
  Will only work if the server \samp{CAPABILITY} response includes the phrase \samp{AUTH=CRAM-MD5}.
\versionadded{2.3}
\end{methoddesc}

\begin{methoddesc}{logout}{}
  Shutdown connection to server. Returns server \samp{BYE} response.
\end{methoddesc}

\begin{methoddesc}{lsub}{\optional{directory\optional{, pattern}}}
  List subscribed mailbox names in directory matching pattern.
  \var{directory} defaults to the top level directory and
  \var{pattern} defaults to match any mailbox.
  Returned data are tuples of message part envelope and data.
\end{methoddesc}

\begin{methoddes}{myrights}{mailbox}
  Show my ACLs for a mailbox (i.e. the rights that I have on mailbox).
\versionadded{2.4}
\end{methoddesc}

\begin{methoddesc}{namespace}{}
  Returns IMAP namespaces as defined in RFC2342.
\versionadded{2.3}
\end{methoddesc}

\begin{methoddesc}{noop}{}
  Send \samp{NOOP} to server.
\end{methoddesc}

\begin{methoddesc}{open}{host, port}
  Opens socket to \var{port} at \var{host}.
  The connection objects established by this method
  will be used in the \code{read}, \code{readline}, \code{send}, and \code{shutdown} methods.
  You may override this method.
\end{methoddesc}

\begin{methoddesc}{partial}{message_num, message_part, start, length}
  Fetch truncated part of a message.
  Returned data is a tuple of message part envelope and data.
\end{methoddesc}

\begin{methoddesc}{proxyauth}{user}
  Assume authentication as \var{user}.
  Allows an authorised administrator to proxy into any user's mailbox.
\versionadded{2.3}
\end{methoddesc}

\begin{methoddesc}{read}{size}
  Reads \var{size} bytes from the remote server.
  You may override this method.
\end{methoddesc}

\begin{methoddesc}{readline}{}
  Reads one line from the remote server.
  You may override this method.
\end{methoddesc}

\begin{methoddesc}{recent}{}
  Prompt server for an update. Returned data is \code{None} if no new
  messages, else value of \samp{RECENT} response.
\end{methoddesc}

\begin{methoddesc}{rename}{oldmailbox, newmailbox}
  Rename mailbox named \var{oldmailbox} to \var{newmailbox}.
\end{methoddesc}

\begin{methoddesc}{response}{code}
  Return data for response \var{code} if received, or
  \code{None}. Returns the given code, instead of the usual type.
\end{methoddesc}

\begin{methoddesc}{search}{charset, criterion\optional{, ...}}
  Search mailbox for matching messages.  Returned data contains a space
  separated list of matching message numbers.  \var{charset} may be
  \code{None}, in which case no \samp{CHARSET} will be specified in the
  request to the server.  The IMAP protocol requires that at least one
  criterion be specified; an exception will be raised when the server
  returns an error.

  Example:

\begin{verbatim}
# M is a connected IMAP4 instance...
msgnums = M.search(None, 'FROM', '"LDJ"')

# or:
msgnums = M.search(None, '(FROM "LDJ")')
\end{verbatim}
\end{methoddesc}

\begin{methoddesc}{select}{\optional{mailbox\optional{, readonly}}}
  Select a mailbox. Returned data is the count of messages in
  \var{mailbox} (\samp{EXISTS} response).  The default \var{mailbox}
  is \code{'INBOX'}.  If the \var{readonly} flag is set, modifications
  to the mailbox are not allowed.
\end{methoddesc}

\begin{methoddesc}{send}{data}
  Sends \code{data} to the remote server.
  You may override this method.
\end{methoddesc}

\begin{methoddesc}{setacl}{mailbox, who, what}
  Set an \samp{ACL} for \var{mailbox}.
  The method is non-standard, but is supported by the \samp{Cyrus} server.
\end{methoddesc}

\begin{methoddesc}{setquota}{root, limits}
  Set the \samp{quota} \var{root}'s resource \var{limits}.
  This method is part of the IMAP4 QUOTA extension defined in rfc2087.
\versionadded{2.3}
\end{methoddesc}

\begin{methoddesc}{shutdown}{}
  Close connection established in \code{open}.
  You may override this method.
\end{methoddesc}

\begin{methoddesc}{socket}{}
  Returns socket instance used to connect to server.
\end{methoddesc}

\begin{methoddesc}{sort}{sort_criteria, charset, search_criterion\optional{, ...}}
  The \code{sort} command is a variant of \code{search} with sorting semantics for
  the results.  Returned data contains a space
  separated list of matching message numbers.

  Sort has two arguments before the \var{search_criterion}
  argument(s); a parenthesized list of \var{sort_criteria}, and the searching \var{charset}.
  Note that unlike \code{search}, the searching \var{charset} argument is mandatory.
  There is also a \code{uid sort} command which corresponds to \code{sort} the way
  that \code{uid search} corresponds to \code{search}.
  The \code{sort} command first searches the mailbox for messages that
  match the given searching criteria using the charset argument for
  the interpretation of strings in the searching criteria.  It then
  returns the numbers of matching messages.

  This is an \samp{IMAP4rev1} extension command.
\end{methoddesc}

\begin{methoddesc}{status}{mailbox, names}
  Request named status conditions for \var{mailbox}. 
\end{methoddesc}

\begin{methoddesc}{store}{message_set, command, flag_list}
  Alters flag dispositions for messages in mailbox.
\end{methoddesc}

\begin{methoddesc}{subscribe}{mailbox}
  Subscribe to new mailbox.
\end{methoddesc}

\begin{methoddesc}{thread}{threading_algorithm, charset, search_criterion\optional{, ...}}
  The \code{thread} command is a variant of \code{search} with threading semantics for
  the results.  Returned data contains a space
  separated list of thread members.

  Thread members consist of zero or more messages numbers, delimited by spaces,
  indicating successive parent and child.

  Thread has two arguments before the \var{search_criterion}
  argument(s); a \var{threading_algorithm}, and the searching \var{charset}.
  Note that unlike \code{search}, the searching \var{charset} argument is mandatory.
  There is also a \code{uid thread} command which corresponds to \code{thread} the way
  that \code{uid search} corresponds to \code{search}.
  The \code{thread} command first searches the mailbox for messages that
  match the given searching criteria using the charset argument for
  the interpretation of strings in the searching criteria. It thren
  returns the matching messages threaded according to the specified
  threading algorithm.

  This is an \samp{IMAP4rev1} extension command. \versionadded{2.4}
\end{methoddesc}

\begin{methoddesc}{uid}{command, arg\optional{, ...}}
  Execute command args with messages identified by UID, rather than
  message number.  Returns response appropriate to command.  At least
  one argument must be supplied; if none are provided, the server will
  return an error and an exception will be raised.
\end{methoddesc}

\begin{methoddesc}{unsubscribe}{mailbox}
  Unsubscribe from old mailbox.
\end{methoddesc}

\begin{methoddesc}{xatom}{name\optional{, arg\optional{, ...}}}
  Allow simple extension commands notified by server in
  \samp{CAPABILITY} response.
\end{methoddesc}


Instances of \class{IMAP4_SSL} have just one additional method:

\begin{methoddesc}{ssl}{}
  Returns SSLObject instance used for the secure connection with the server.
\end{methoddesc}


The following attributes are defined on instances of \class{IMAP4}:


\begin{memberdesc}{PROTOCOL_VERSION}
The most recent supported protocol in the
\samp{CAPABILITY} response from the server.
\end{memberdesc}

\begin{memberdesc}{debug}
Integer value to control debugging output.  The initialize value is
taken from the module variable \code{Debug}.  Values greater than
three trace each command.
\end{memberdesc}


\subsection{IMAP4 Example \label{imap4-example}}

Here is a minimal example (without error checking) that opens a
mailbox and retrieves and prints all messages:

\begin{verbatim}
import getpass, imaplib

M = imaplib.IMAP4()
M.login(getpass.getuser(), getpass.getpass())
M.select()
typ, data = M.search(None, 'ALL')
for num in data[0].split():
    typ, data = M.fetch(num, '(RFC822)')
    print 'Message %s\n%s\n' % (num, data[0][1])
M.logout()
\end{verbatim}

\section{Standard Module \module{nntplib}}
\label{module-nntplib}
\stmodindex{nntplib}
\indexii{NNTP}{protocol}
\index{Network News Transfer Protocol}


This module defines the class \class{NNTP} which implements the client
side of the NNTP protocol.  It can be used to implement a news reader
or poster, or automated news processors.  For more information on NNTP
(Network News Transfer Protocol), see Internet \rfc{977}.

Here are two small examples of how it can be used.  To list some
statistics about a newsgroup and print the subjects of the last 10
articles:

\begin{verbatim}
>>> s = NNTP('news.cwi.nl')
>>> resp, count, first, last, name = s.group('comp.lang.python')
>>> print 'Group', name, 'has', count, 'articles, range', first, 'to', last
Group comp.lang.python has 59 articles, range 3742 to 3803
>>> resp, subs = s.xhdr('subject', first + '-' + last)
>>> for id, sub in subs[-10:]: print id, sub
... 
3792 Re: Removing elements from a list while iterating...
3793 Re: Who likes Info files?
3794 Emacs and doc strings
3795 a few questions about the Mac implementation
3796 Re: executable python scripts
3797 Re: executable python scripts
3798 Re: a few questions about the Mac implementation 
3799 Re: PROPOSAL: A Generic Python Object Interface for Python C Modules
3802 Re: executable python scripts 
3803 Re: \POSIX{} wait and SIGCHLD
>>> s.quit()
'205 news.cwi.nl closing connection.  Goodbye.'
\end{verbatim}

To post an article from a file (this assumes that the article has
valid headers):

\begin{verbatim}
>>> s = NNTP('news.cwi.nl')
>>> f = open('/tmp/article')
>>> s.post(f)
'240 Article posted successfully.'
>>> s.quit()
'205 news.cwi.nl closing connection.  Goodbye.'
\end{verbatim}
%
The module itself defines the following items:

\begin{classdesc}{NNTP}{host\optional{, port}}
Return a new instance of the \class{NNTP} class, representing a
connection to the NNTP server running on host \var{host}, listening at
port \var{port}.  The default \var{port} is 119.
\end{classdesc}

\begin{excdesc}{error_reply}
Exception raised when an unexpected reply is received from the server.
\end{excdesc}

\begin{excdesc}{error_temp}
Exception raised when an error code in the range 400--499 is received.
\end{excdesc}

\begin{excdesc}{error_perm}
Exception raised when an error code in the range 500--599 is received.
\end{excdesc}

\begin{excdesc}{error_proto}
Exception raised when a reply is received from the server that does
not begin with a digit in the range 1--5.
\end{excdesc}


\subsection{NNTP Objects}
\label{nntp-objects}

NNTP instances have the following methods.  The \var{response} that is
returned as the first item in the return tuple of almost all methods
is the server's response: a string beginning with a three-digit code.
If the server's response indicates an error, the method raises one of
the above exceptions.


\begin{methoddesc}{getwelcome}{}
Return the welcome message sent by the server in reply to the initial
connection.  (This message sometimes contains disclaimers or help
information that may be relevant to the user.)
\end{methoddesc}

\begin{methoddesc}{set_debuglevel}{level}
Set the instance's debugging level.  This controls the amount of
debugging output printed.  The default, \code{0}, produces no debugging
output.  A value of \code{1} produces a moderate amount of debugging
output, generally a single line per request or response.  A value of
\code{2} or higher produces the maximum amount of debugging output,
logging each line sent and received on the connection (including
message text).
\end{methoddesc}

\begin{methoddesc}{newgroups}{date, time}
Send a \samp{NEWGROUPS} command.  The \var{date} argument should be a
string of the form \code{"\var{yy}\var{mm}\var{dd}"} indicating the
date, and \var{time} should be a string of the form
\code{"\var{hh}\var{mm}\var{ss}"} indicating the time.  Return a pair
\code{(\var{response}, \var{groups})} where \var{groups} is a list of
group names that are new since the given date and time.
\end{methoddesc}

\begin{methoddesc}{newnews}{group, date, time}
Send a \samp{NEWNEWS} command.  Here, \var{group} is a group name or
\code{'*'}, and \var{date} and \var{time} have the same meaning as for
\method{newgroups()}.  Return a pair \code{(\var{response},
\var{articles})} where \var{articles} is a list of article ids.
\end{methoddesc}

\begin{methoddesc}{list}{}
Send a \samp{LIST} command.  Return a pair \code{(\var{response},
\var{list})} where \var{list} is a list of tuples.  Each tuple has the
form \code{(\var{group}, \var{last}, \var{first}, \var{flag})}, where
\var{group} is a group name, \var{last} and \var{first} are the last
and first article numbers (as strings), and \var{flag} is \code{'y'}
if posting is allowed, \code{'n'} if not, and \code{'m'} if the
newsgroup is moderated.  (Note the ordering: \var{last}, \var{first}.)
\end{methoddesc}

\begin{methoddesc}{group}{name}
Send a \samp{GROUP} command, where \var{name} is the group name.
Return a tuple \code{(}\var{response}\code{,} \var{count}\code{,}
\var{first}\code{,} \var{last}\code{,} \var{name}\code{)} where
\var{count} is the (estimated) number of articles in the group,
\var{first} is the first article number in the group, \var{last} is
the last article number in the group, and \var{name} is the group
name.  The numbers are returned as strings.
\end{methoddesc}

\begin{methoddesc}{help}{}
Send a \samp{HELP} command.  Return a pair \code{(\var{response},
\var{list})} where \var{list} is a list of help strings.
\end{methoddesc}

\begin{methoddesc}{stat}{id}
Send a \samp{STAT} command, where \var{id} is the message id (enclosed
in \character{<} and \character{>}) or an article number (as a string).
Return a triple \code{(\var{response}, \var{number}, \var{id})} where
\var{number} is the article number (as a string) and \var{id} is the
article id  (enclosed in \character{<} and \character{>}).
\end{methoddesc}

\begin{methoddesc}{next}{}
Send a \samp{NEXT} command.  Return as for \method{stat()}.
\end{methoddesc}

\begin{methoddesc}{last}{}
Send a \samp{LAST} command.  Return as for \method{stat()}.
\end{methoddesc}

\begin{methoddesc}{head}{id}
Send a \samp{HEAD} command, where \var{id} has the same meaning as for
\method{stat()}.  Return a tuple
\code{(\var{response}, \var{number}, \var{id}, \var{list})}
where the first three are the same as for \method{stat()},
and \var{list} is a list of the article's headers (an uninterpreted
list of lines, without trailing newlines).
\end{methoddesc}

\begin{methoddesc}{body}{id}
Send a \samp{BODY} command, where \var{id} has the same meaning as for
\method{stat()}.  Return as for \method{head()}.
\end{methoddesc}

\begin{methoddesc}{article}{id}
Send a \samp{ARTICLE} command, where \var{id} has the same meaning as
for \method{stat()}.  Return as for \method{head()}.
\end{methoddesc}

\begin{methoddesc}{slave}{}
Send a \samp{SLAVE} command.  Return the server's \var{response}.
\end{methoddesc}

\begin{methoddesc}{xhdr}{header, string}
Send an \samp{XHDR} command.  This command is not defined in the RFC
but is a common extension.  The \var{header} argument is a header
keyword, e.g. \code{'subject'}.  The \var{string} argument should have
the form \code{"\var{first}-\var{last}"} where \var{first} and
\var{last} are the first and last article numbers to search.  Return a
pair \code{(\var{response}, \var{list})}, where \var{list} is a list of
pairs \code{(\var{id}, \var{text})}, where \var{id} is an article id
(as a string) and \var{text} is the text of the requested header for
that article.
\end{methoddesc}

\begin{methoddesc}{post}{file}
Post an article using the \samp{POST} command.  The \var{file}
argument is an open file object which is read until EOF using its
\method{readline()} method.  It should be a well-formed news article,
including the required headers.  The \method{post()} method
automatically escapes lines beginning with \samp{.}.
\end{methoddesc}

\begin{methoddesc}{ihave}{id, file}
Send an \samp{IHAVE} command.  If the response is not an error, treat
\var{file} exactly as for the \method{post()} method.
\end{methoddesc}

\begin{methoddesc}{date}{}
Return a triple \code{(\var{response}, \var{date}, \var{time})},
containing the current date and time in a form suitable for the
\method{newnews()} and \method{newgroups()} methods.
This is an optional NNTP extension, and may not be supported by all
servers.
\end{methoddesc}

\begin{methoddesc}{xgtitle}{name}
Process an \samp{XGTITLE} command, returning a pair \code{(\var{response},
\var{list})}, where \var{list} is a list of tuples containing
\code{(\var{name}, \var{title})}.
% XXX huh?  Should that be name, description?
This is an optional NNTP extension, and may not be supported by all
servers.
\end{methoddesc}

\begin{methoddesc}{xover}{start, end}
Return a pair \code{(\var{resp}, \var{list})}.  \var{list} is a list
of tuples, one for each article in the range delimited by the \var{start}
and \var{end} article numbers.  Each tuple is of the form
\code{(}\var{article number}\code{,} \var{subject}\code{,}
\var{poster}\code{,} \var{date}\code{,} \var{id}\code{,}
\var{references}\code{,} \var{size}\code{,} \var{lines}\code{)}.
This is an optional NNTP extension, and may not be supported by all
servers.
\end{methoddesc}

\begin{methoddesc}{xpath}{id}
Return a pair \code{(\var{resp}, \var{path})}, where \var{path} is the
directory path to the article with message ID \var{id}.  This is an
optional NNTP extension, and may not be supported by all servers.
\end{methoddesc}

\begin{methoddesc}{quit}{}
Send a \samp{QUIT} command and close the connection.  Once this method
has been called, no other methods of the NNTP object should be called.
\end{methoddesc}

\section{\module{smtplib} ---
         SMTP protocol client.}
\declaremodule{standard}{smtplib}
\sectionauthor{Eric S. Raymond}{esr@snark.thyrsus.com}

\modulesynopsis{SMTP protocol client (requires sockets).}

\indexii{SMTP}{protocol}
\index{Simple Mail Transfer Protocol}

The \module{smtplib} module defines an SMTP client session object that
can be used to send mail to any Internet machine with an SMTP or ESMTP
listener daemon.  For details of SMTP and ESMTP operation, consult
\rfc{821} (\emph{Simple Mail Transfer Protocol}) and \rfc{1869}
(\emph{SMTP Service Extensions}).

\begin{classdesc}{SMTP}{\optional{host\optional{, port}}}
A \class{SMTP} instance encapsulates an SMTP connection.  It has
methods that support a full repertoire of SMTP and ESMTP
operations. If the optional host and port parameters are given, the
SMTP \method{connect()} method is called with those parameters during
initialization.

For normal use, you should only require the initialization/connect,
\method{sendmail()}, and \method{quit()} methods  An example is
included below.
\end{classdesc}


\subsection{SMTP Objects}
\label{SMTP-objects}

An \class{SMTP} instance has the following methods:

\begin{methoddesc}{set_debuglevel}{level}
Set the debug output level.  A true value for \var{level} results in
debug messages for connection and for all messages sent to and
received from the server.
\end{methoddesc}

\begin{methoddesc}{connect}{\optional{host\optional{, port}}}
Connect to a host on a given port.  The defaults are to connect to the 
local host at the standard SMTP port (25).

If the hostname ends with a colon (\character{:}) followed by a
number, that suffix will be stripped off and the number interpreted as 
the port number to use.

Note:  This method is automatically invoked by the constructor if a
host is specified during instantiation.
\end{methoddesc}

\begin{methoddesc}{docmd}{cmd, \optional{, argstring}}
Send a command \var{cmd} to the server.  The optional argument
\var{argstring} is simply concatenated to the command, separated by a
space.

This returns a 2-tuple composed of a numeric response code and the
actual response line (multiline responses are joined into one long
line.)

In normal operation it should not be necessary to call this method
explicitly.  It is used to implement other methods and may be useful
for testing private extensions.
\end{methoddesc}

\begin{methoddesc}{helo}{\optional{hostname}}
Identify yourself to the SMTP server using \samp{HELO}.  The hostname
argument defaults to the fully qualified domain name of the local
host.

In normal operation it should not be necessary to call this method
explicitly.  It will be implicitly called by the \method{sendmail()}
when necessary.
\end{methoddesc}

\begin{methoddesc}{ehlo}{\optional{hostname}}
Identify yourself to an ESMTP server using \samp{HELO}.  The hostname
argument defaults to the fully qualified domain name of the local
host.  Examine the response for ESMTP option and store them for use by
\method{has_option()}.

Unless you wish to use \method{has_option()} before sending
mail, it should not be necessary to call this method explicitly.  It
will be implicitly called by \method{sendmail()} when necessary.
\end{methoddesc}

\begin{methoddesc}{has_option}{name}
Return \code{1} if \var{name} is in the set of ESMTP options returned
by the server, \code{0} otherwise.  Case is ignored.
\end{methoddesc}

\begin{methoddesc}{verify}{address}
Check the validity of an address on this server using SMTP \samp{VRFY}.
Returns a tuple consisting of code 250 and a full \rfc{822} address
(including human name) if the user address is valid. Otherwise returns
an SMTP error code of 400 or greater and an error string.

Note: many sites disable SMTP \samp{VRFY} in order to foil spammers.
\end{methoddesc}

\begin{methoddesc}{sendmail}{from_addr, to_addrs, msg\optional{, options}}
Send mail.  The required arguments are an \rfc{822} from-address
string, a list of \rfc{822} to-address strings, and a message string.
The caller may pass a list of ESMTP options to be used in \samp{MAIL
FROM} commands as \var{options}.

If there has been no previous \samp{EHLO} or \samp{HELO} command this
session, this method tries ESMTP \samp{EHLO} first. If the server does
ESMTP, message size and each of the specified options will be passed
to it (if the option is in the feature set the server advertises).  If
\samp{EHLO} fails, \samp{HELO} will be tried and ESMTP options
suppressed.

This method will return normally if the mail is accepted for at least 
one recipient. Otherwise it will throw an exception (either
\exception{SMTPSenderRefused}, \exception{SMTPRecipientsRefused}, or
\exception{SMTPDataError}).  That is, if this method does not throw an
exception, then someone should get your mail.  If this method does not
throw an exception, it returns a dictionary, with one entry for each
recipient that was refused.
\end{methoddesc}

\begin{methoddesc}{quit}{}
Terminate the SMTP session and close the connection.
\end{methoddesc}

Low-level methods corresponding to the standard SMTP/ESMTP commands
\samp{HELP}, \samp{RSET}, \samp{NOOP}, \samp{MAIL}, \samp{RCPT}, and
\samp{DATA} are also supported.  Normally these do not need to be
called directly, so they are not documented here.  For details,
consult the module code.


\subsection{SMTP Example \label{SMTP-example}}

% really need a little description here...

\begin{verbatim}
import rfc822, string, sys
import smtplib

def prompt(prompt):
    sys.stdout.write(prompt + ": ")
    return string.strip(sys.stdin.readline())

fromaddr = prompt("From")
toaddrs  = string.splitfields(prompt("To"), ',')
print "Enter message, end with ^D:"
msg = ''
while 1:
    line = sys.stdin.readline()
    if not line:
        break
    msg = msg + line
print "Message length is " + `len(msg)`

server = smtplib.SMTP('localhost')
server.set_debuglevel(1)
server.sendmail(fromaddr, toaddrs, msg)
server.quit()
\end{verbatim}


\begin{seealso}
\seetext{\rfc{821}, \emph{Simple Mail Transfer Protocol}.  Available
online at \url{http://info.internet.isi.edu/in-notes/rfc/files/rfc821.txt}.}

\seetext{\rfc{1869}, \emph{SMTP Service Extensions}.  Available online 
at \url{http://info.internet.isi.edu/in-notes/rfc/files/rfc1869.txt}.}
\end{seealso}

% LaTeX'ized from the comments in the module by Skip Montanaro
% <skip@mojam.com>.

\section{\module{telnetlib} ---
         Telnet client}

\declaremodule{standard}{telnetlib}
\modulesynopsis{Telnet client class.}


The \module{telnetlib} module provides a \class{Telnet} class that
implements the Telnet protocol.  See \rfc{854} for details about the
protocol.


\begin{classdesc}{Telnet}{\optional{host\optional{, port=0}}}
\class{Telnet} represents a connection to a telnet server. The
instance is initially not connected; the \method{open()} method must
be used to establish a connection.  Alternatively, the host name and
optional port number can be passed to the constructor, too.

Do not reopen an already connected instance.

This class has many \method{read_*()} methods.  Note that some of them 
raise \exception{EOFError} when the end of the connection is read,
because they can return an empty string for other reasons.  See the
individual doc strings.
\end{classdesc}


\subsection{Telnet Objects \label{telnet-objects}}

\class{Telnet} instances have the following methods:


\begin{methoddesc}[Telnet]{read_until}{expected\optional{, timeout}}
Read until a given string is encountered or until timeout.

When no match is found, return whatever is available instead,
possibly the empty string.  Raise \exception{EOFError} if the connection
is closed and no cooked data is available.
\end{methoddesc}

\begin{methoddesc}[Telnet]{read_all}{}
Read all data until EOF; block until connection closed.
\end{methoddesc}

\begin{methoddesc}[Telnet]{read_some}{}
Read at least one byte of cooked data unless EOF is hit.

Return \code{''} if EOF is hit.  Block if no data is immediately available.
\end{methoddesc}

\begin{methoddesc}[Telnet]{read_very_eager}{}
Read everything that's possible without blocking in I/O (eager).

Raise \exception{EOFError} if connection closed and no cooked data
available.  Return \code{''} if no cooked data available otherwise.
Don't block unless in the midst of an IAC sequence.
\end{methoddesc}

\begin{methoddesc}[Telnet]{read_eager}{}
Read readily available data.

Raise \exception{EOFError} if connection closed and no cooked data
available.  Return \code{''} if no cooked data available otherwise.
Don't block unless in the midst of an IAC sequence.
\end{methoddesc}

\begin{methoddesc}[Telnet]{read_lazy}{}
Process and return data that's already in the queues (lazy).

Raise \exception{EOFError} if connection closed and no data available.
Return \code{''} if no cooked data available otherwise.  Don't block
unless in the midst of an IAC sequence.
\end{methoddesc}

\begin{methoddesc}[Telnet]{read_very_lazy}{}
Return any data available in the cooked queue (very lazy).

Raise \exception{EOFError} if connection closed and no data available.
Return \code{''} if no cooked data available otherwise.  Don't block.
\end{methoddesc}

\begin{methoddesc}[Telnet]{open}{host\optional{, port=0}}
Connect to a host.

The optional second argument is the port number, which
defaults to the standard telnet port (23).

Don't try to reopen an already connected instance.
\end{methoddesc}

\begin{methoddesc}[Telnet]{msg}{msg\optional{, *args}}
Print a debug message, when the debug level is > 0.

If extra arguments are present, they are substituted in the
message using the standard string formatting operator.
\end{methoddesc}

\begin{methoddesc}[Telnet]{set_debuglevel}{debuglevel}
Set the debug level.

The higher it is, the more debug output you get (on sys.stdout).
\end{methoddesc}

\begin{methoddesc}[Telnet]{close}{}
Close the connection.
\end{methoddesc}

\begin{methoddesc}[Telnet]{get_socket}{}
Return the socket object used internally.
\end{methoddesc}

\begin{methoddesc}[Telnet]{fileno}{}
Return the fileno() of the socket object used internally.
\end{methoddesc}

\begin{methoddesc}[Telnet]{write}{buffer}
Write a string to the socket, doubling any IAC characters.

Can block if the connection is blocked.  May raise
socket.error if the connection is closed.
\end{methoddesc}

\begin{methoddesc}[Telnet]{interact}{}
Interaction function, emulates a very dumb telnet client.
\end{methoddesc}

\begin{methoddesc}[Telnet]{mt_interact}{}
Multithreaded version of \method{interact}.
\end{methoddesc}

\begin{methoddesc}[Telnet]{expect}{list, timeout=None}
Read until one from a list of a regular expressions matches.

The first argument is a list of regular expressions, either
compiled (\class{re.RegexObject} instances) or uncompiled (strings).
The optional second argument is a timeout, in seconds; default
is no timeout.

Return a tuple of three items: the index in the list of the
first regular expression that matches; the match object
returned; and the text read up till and including the match.

If end of file is found and no text was read, raise
\exception{EOFError}.  Otherwise, when nothing matches, return
\code{(-1, None, \var{text})} where \var{text} is the text received so
far (may be the empty string if a timeout happened).

If a regular expression ends with a greedy match (e.g. \regexp{.*})
or if more than one expression can match the same input, the
results are undeterministic, and may depend on the I/O timing.
\end{methoddesc}

\section{\module{urlparse} ---
         Parse URLs into components}
\declaremodule{standard}{urlparse}

\modulesynopsis{Parse URLs into components.}

\index{WWW}
\index{World Wide Web}
\index{URL}
\indexii{URL}{parsing}
\indexii{relative}{URL}


This module defines a standard interface to break Uniform Resource
Locator (URL) strings up in components (addressing scheme, network
location, path etc.), to combine the components back into a URL
string, and to convert a ``relative URL'' to an absolute URL given a
``base URL.''

The module has been designed to match the Internet RFC on Relative
Uniform Resource Locators (and discovered a bug in an earlier
draft!). It supports the following URL schemes:
\code{file}, \code{ftp}, \code{gopher}, \code{hdl}, \code{http}, 
\code{https}, \code{imap}, \code{mailto}, \code{mms}, \code{news}, 
\code{nntp}, \code{prospero}, \code{rsync}, \code{rtsp}, \code{rtspu}, 
\code{sftp}, \code{shttp}, \code{sip}, \code{snews}, \code{svn}, 
\code{svn+ssh}, \code{telnet}, \code{wais}.

The \module{urlparse} module defines the following functions:

\begin{funcdesc}{urlparse}{urlstring\optional{, default_scheme\optional{, allow_fragments}}}
Parse a URL into 6 components, returning a 6-tuple: (addressing
scheme, network location, path, parameters, query, fragment
identifier).  This corresponds to the general structure of a URL:
\code{\var{scheme}://\var{netloc}/\var{path};\var{parameters}?\var{query}\#\var{fragment}}.
Each tuple item is a string, possibly empty.
The components are not broken up in smaller parts (e.g. the network
location is a single string), and \% escapes are not expanded.
The delimiters as shown above are not part of the tuple items,
except for a leading slash in the \var{path} component, which is
retained if present.

Example:

\begin{verbatim}
urlparse('http://www.cwi.nl:80/%7Eguido/Python.html')
\end{verbatim}

yields the tuple

\begin{verbatim}
('http', 'www.cwi.nl:80', '/%7Eguido/Python.html', '', '', '')
\end{verbatim}

If the \var{default_scheme} argument is specified, it gives the
default addressing scheme, to be used only if the URL string does not
specify one.  The default value for this argument is the empty string.

If the \var{allow_fragments} argument is zero, fragment identifiers
are not allowed, even if the URL's addressing scheme normally does
support them.  The default value for this argument is \code{1}.
\end{funcdesc}

\begin{funcdesc}{urlunparse}{tuple}
Construct a URL string from a tuple as returned by \code{urlparse()}.
This may result in a slightly different, but equivalent URL, if the
URL that was parsed originally had redundant delimiters, e.g. a ? with
an empty query (the draft states that these are equivalent).
\end{funcdesc}

\begin{funcdesc}{urlsplit}{urlstring\optional{,
                           default_scheme\optional{, allow_fragments}}}
This is similar to \function{urlparse()}, but does not split the
params from the URL.  This should generally be used instead of
\function{urlparse()} if the more recent URL syntax allowing
parameters to be applied to each segment of the \var{path} portion of
the URL (see \rfc{2396}) is wanted.  A separate function is needed to
separate the path segments and parameters.  This function returns a
5-tuple: (addressing scheme, network location, path, query, fragment
identifier).
\versionadded{2.2}
\end{funcdesc}

\begin{funcdesc}{urlunsplit}{tuple}
Combine the elements of a tuple as returned by \function{urlsplit()}
into a complete URL as a string.
\versionadded{2.2}
\end{funcdesc}

\begin{funcdesc}{urljoin}{base, url\optional{, allow_fragments}}
Construct a full (``absolute'') URL by combining a ``base URL''
(\var{base}) with a ``relative URL'' (\var{url}).  Informally, this
uses components of the base URL, in particular the addressing scheme,
the network location and (part of) the path, to provide missing
components in the relative URL.

Example:

\begin{verbatim}
urljoin('http://www.cwi.nl/%7Eguido/Python.html', 'FAQ.html')
\end{verbatim}

yields the string

\begin{verbatim}
'http://www.cwi.nl/%7Eguido/FAQ.html'
\end{verbatim}

The \var{allow_fragments} argument has the same meaning as for
\code{urlparse()}.
\end{funcdesc}

\begin{funcdesc}{urldefrag}{url}
If \var{url} contains a fragment identifier, returns a modified
version of \var{url} with no fragment identifier, and the fragment
identifier as a separate string.  If there is no fragment identifier
in \var{url}, returns \var{url} unmodified and an empty string.
\end{funcdesc}


\begin{seealso}
  \seerfc{1738}{Uniform Resource Locators (URL)}{
        This specifies the formal syntax and semantics of absolute
        URLs.}
  \seerfc{1808}{Relative Uniform Resource Locators}{
        This Request For Comments includes the rules for joining an
        absolute and a relative URL, including a fair number of
        ``Abnormal Examples'' which govern the treatment of border
        cases.}
  \seerfc{2396}{Uniform Resource Identifiers (URI): Generic Syntax}{
        Document describing the generic syntactic requirements for
        both Uniform Resource Names (URNs) and Uniform Resource
        Locators (URLs).}
\end{seealso}

\section{\module{SocketServer} ---
         A framework for network servers.}
\declaremodule{standard}{SocketServer}

\modulesynopsis{A framework for network servers.}


The \module{SocketServer} module simplifies the task of writing network
servers.

There are four basic server classes: \class{TCPServer} uses the
Internet TCP protocol, which provides for continuous streams of data
between the client and server.  \class{UDPServer} uses datagrams, which
are discrete packets of information that may arrive out of order or be
lost while in transit.  The more infrequently used
\class{UnixStreamServer} and \class{UnixDatagramServer} classes are
similar, but use \UNIX{} domain sockets; they're not available on
non-\UNIX{} platforms.  For more details on network programming, consult
a book such as W. Richard Steven's \emph{UNIX Network Programming}
or Ralph Davis's \emph{Win32 Network Programming}.

These four classes process requests \dfn{synchronously}; each request
must be completed before the next request can be started.  This isn't
suitable if each request takes a long time to complete, because it
requires a lot of computation, or because it returns a lot of data
which the client is slow to process.  The solution is to create a
separate process or thread to handle each request; the
\class{ForkingMixIn} and \class{ThreadingMixIn} mix-in classes can be
used to support asynchronous behaviour.

Creating a server requires several steps.  First, you must create a
request handler class by subclassing the \class{BaseRequestHandler}
class and overriding its \method{handle()} method; this method will
process incoming requests.  Second, you must instantiate one of the
server classes, passing it the server's address and the request
handler class.  Finally, call the \method{handle_request()} or
\method{serve_forever()} method of the server object to process one or
many requests.

Server classes have the same external methods and attributes, no
matter what network protocol they use:

\setindexsubitem{(SocketServer protocol)}

%XXX should data and methods be intermingled, or separate?
% how should the distinction between class and instance variables be
% drawn?

\begin{funcdesc}{fileno}{}
Return an integer file descriptor for the socket on which the server
is listening.  This function is most commonly passed to
\function{select.select()}, to allow monitoring multiple servers in the
same process.
\end{funcdesc}

\begin{funcdesc}{handle_request}{}
Process a single request.  This function calls the following methods
in order: \method{get_request()}, \method{verify_request()}, and
\method{process_request()}.  If the user-provided \method{handle()}
method of the handler class raises an exception, the server's
\method{handle_error()} method will be called.
\end{funcdesc}

\begin{funcdesc}{serve_forever}{}
Handle an infinite number of requests.  This simply calls
\method{handle_request()} inside an infinite loop.
\end{funcdesc}

\begin{datadesc}{address_family}
The family of protocols to which the server's socket belongs.
\constant{socket.AF_INET} and \constant{socket.AF_UNIX} are two
possible values.
\end{datadesc}

\begin{datadesc}{RequestHandlerClass}
The user-provided request handler class; an instance of this class is
created for each request.
\end{datadesc}

\begin{datadesc}{server_address}
The address on which the server is listening.  The format of addresses
varies depending on the protocol family; see the documentation for the
socket module for details.  For Internet protocols, this is a tuple
containing a string giving the address, and an integer port number:
\code{('127.0.0.1', 80)}, for example.
\end{datadesc}

\begin{datadesc}{socket}
The socket object on which the server will listen for incoming requests.
\end{datadesc}

% XXX should class variables be covered before instance variables, or
% vice versa?

The server classes support the following class variables:

\begin{datadesc}{request_queue_size}
The size of the request queue.  If it takes a long time to process a
single request, any requests that arrive while the server is busy are
placed into a queue, up to \member{request_queue_size} requests.  Once
the queue is full, further requests from clients will get a
``Connection denied'' error.  The default value is usually 5, but this
can be overridden by subclasses.
\end{datadesc}

\begin{datadesc}{socket_type}
The type of socket used by the server; \constant{socket.SOCK_STREAM}
and \constant{socket.SOCK_DGRAM} are two possible values.
\end{datadesc}

There are various server methods that can be overridden by subclasses
of base server classes like \class{TCPServer}; these methods aren't
useful to external users of the server object.

% should the default implementations of these be documented, or should
% it be assumed that the user will look at SocketServer.py?

\begin{funcdesc}{finish_request}{}
Actually processes the request by instantiating
\member{RequestHandlerClass} and calling its \method{handle()} method.
\end{funcdesc}

\begin{funcdesc}{get_request}{}
Must accept a request from the socket, and return a 2-tuple containing
the \emph{new} socket object to be used to communicate with the
client, and the client's address.
\end{funcdesc}

\begin{funcdesc}{handle_error}{request, client_address}
This function is called if the \member{RequestHandlerClass}'s
\method{handle()} method raises an exception.  The default action is
to print the traceback to standard output and continue handling
further requests.
\end{funcdesc}

\begin{funcdesc}{process_request}{request, client_address}
Calls \method{finish_request()} to create an instance of the
\member{RequestHandlerClass}.  If desired, this function can create a
new process or thread to handle the request; the \class{ForkingMixIn}
and \class{ThreadingMixIn} classes do this.
\end{funcdesc}

% Is there any point in documenting the following two functions?
% What would the purpose of overriding them be: initializing server
% instance variables, adding new network families?

\begin{funcdesc}{server_activate}{}
Called by the server's constructor to activate the server.
May be overridden.
\end{funcdesc}

\begin{funcdesc}{server_bind}{}
Called by the server's constructor to bind the socket to the desired
address.  May be overridden.
\end{funcdesc}

\begin{funcdesc}{verify_request}{request, client_address}
Must return a Boolean value; if the value is true, the request will be
processed, and if it's false, the request will be denied.
This function can be overridden to implement access controls for a server.
The default implementation always return true.
\end{funcdesc}

The request handler class must define a new \method{handle()} method,
and can override any of the following methods.  A new instance is
created for each request.

\begin{funcdesc}{finish}{}
Called after the \method{handle()} method to perform any clean-up
actions required.  The default implementation does nothing.  If
\method{setup()} or \method{handle()} raise an exception, this
function will not be called.
\end{funcdesc}

\begin{funcdesc}{handle}{}
This function must do all the work required to service a request.
Several instance attributes are available to it; the request is
available as \member{self.request}; the client address as
\member{self.client_address}; and the server instance as
\member{self.server}, in case it needs access to per-server
information.

The type of \member{self.request} is different for datagram or stream
services.  For stream services, \member{self.request} is a socket
object; for datagram services, \member{self.request} is a string.
However, this can be hidden by using the mix-in request handler
classes
\class{StreamRequestHandler} or \class{DatagramRequestHandler}, which
override the \method{setup()} and \method{finish()} methods, and
provides \member{self.rfile} and \member{self.wfile} attributes.
\member{self.rfile} and \member{self.wfile} can be read or written,
respectively, to get the request data or return data to the client.
\end{funcdesc}

\begin{funcdesc}{setup}{}
Called before the \method{handle()} method to perform any
initialization actions required.  The default implementation does
nothing.
\end{funcdesc}

\section{\module{BaseHTTPServer} ---
         Basic HTTP server}

\declaremodule{standard}{BaseHTTPServer}
\modulesynopsis{Basic HTTP server (base class for
                \class{SimpleHTTPServer} and \class{CGIHTTPServer}).}


\indexii{WWW}{server}
\indexii{HTTP}{protocol}
\index{URL}
\index{httpd}

This module defines two classes for implementing HTTP servers
(Web servers). Usually, this module isn't used directly, but is used
as a basis for building functioning Web servers. See the
\refmodule{SimpleHTTPServer}\refstmodindex{SimpleHTTPServer} and
\refmodule{CGIHTTPServer}\refstmodindex{CGIHTTPServer} modules.

The first class, \class{HTTPServer}, is a
\class{SocketServer.TCPServer} subclass.  It creates and listens at the
HTTP socket, dispatching the requests to a handler.  Code to create and
run the server looks like this:

\begin{verbatim}
def run(server_class=BaseHTTPServer.HTTPServer,
        handler_class=BaseHTTPServer.BaseHTTPRequestHandler):
    server_address = ('', 8000)
    httpd = server_class(server_address, handler_class)
    httpd.serve_forever()
\end{verbatim}

\begin{classdesc}{HTTPServer}{server_address, RequestHandlerClass}
This class builds on the \class{TCPServer} class by
storing the server address as instance
variables named \member{server_name} and \member{server_port}. The
server is accessible by the handler, typically through the handler's
\member{server} instance variable.
\end{classdesc}

\begin{classdesc}{BaseHTTPRequestHandler}{request, client_address, server}
This class is used
to handle the HTTP requests that arrive at the server. By itself,
it cannot respond to any actual HTTP requests; it must be subclassed
to handle each request method (e.g. GET or POST).
\class{BaseHTTPRequestHandler} provides a number of class and instance
variables, and methods for use by subclasses.

The handler will parse the request and the headers, then call a
method specific to the request type. The method name is constructed
from the request. For example, for the request method \samp{SPAM}, the
\method{do_SPAM()} method will be called with no arguments. All of
the relevant information is stored in instance variables of the
handler.  Subclasses should not need to override or extend the
\method{__init__()} method.
\end{classdesc}


\class{BaseHTTPRequestHandler} has the following instance variables:

\begin{memberdesc}{client_address}
Contains a tuple of the form \code{(\var{host}, \var{port})} referring
to the client's address.
\end{memberdesc}

\begin{memberdesc}{command}
Contains the command (request type). For example, \code{'GET'}.
\end{memberdesc}

\begin{memberdesc}{path}
Contains the request path.
\end{memberdesc}

\begin{memberdesc}{request_version}
Contains the version string from the request. For example,
\code{'HTTP/1.0'}.
\end{memberdesc}

\begin{memberdesc}{headers}
Holds an instance of the class specified by the \member{MessageClass}
class variable. This instance parses and manages the headers in
the HTTP request.
\end{memberdesc}

\begin{memberdesc}{rfile}
Contains an input stream, positioned at the start of the optional
input data.
\end{memberdesc}

\begin{memberdesc}{wfile}
Contains the output stream for writing a response back to the client.
Proper adherence to the HTTP protocol must be used when writing
to this stream.
\end{memberdesc}


\class{BaseHTTPRequestHandler} has the following class variables:

\begin{memberdesc}{server_version}
Specifies the server software version.  You may want to override
this.
The format is multiple whitespace-separated strings,
where each string is of the form name[/version].
For example, \code{'BaseHTTP/0.2'}.
\end{memberdesc}

\begin{memberdesc}{sys_version}
Contains the Python system version, in a form usable by the
\member{version_string} method and the \member{server_version} class
variable. For example, \code{'Python/1.4'}.
\end{memberdesc}

\begin{memberdesc}{error_message_format}
Specifies a format string for building an error response to the
client. It uses parenthesized, keyed format specifiers, so the
format operand must be a dictionary. The \var{code} key should
be an integer, specifying the numeric HTTP error code value.
\var{message} should be a string containing a (detailed) error
message of what occurred, and \var{explain} should be an
explanation of the error code number. Default \var{message}
and \var{explain} values can found in the \var{responses}
class variable.
\end{memberdesc}

\begin{memberdesc}{protocol_version}
This specifies the HTTP protocol version used in responses.  If set
to \code{'HTTP/1.1'}, the server will permit HTTP persistent
connections; however, your server \emph{must} then include an
accurate \code{Content-Length} header (using \method{send_header()})
in all of its responses to clients.  For backwards compatibility,
the setting defaults to \code{'HTTP/1.0'}.
\end{memberdesc}

\begin{memberdesc}{MessageClass}
Specifies a \class{rfc822.Message}-like class to parse HTTP
headers. Typically, this is not overridden, and it defaults to
\class{mimetools.Message}.
\withsubitem{(in module mimetools)}{\ttindex{Message}}
\end{memberdesc}

\begin{memberdesc}{responses}
This variable contains a mapping of error code integers to two-element
tuples containing a short and long message. For example,
\code{\{\var{code}: (\var{shortmessage}, \var{longmessage})\}}. The
\var{shortmessage} is usually used as the \var{message} key in an
error response, and \var{longmessage} as the \var{explain} key
(see the \member{error_message_format} class variable).
\end{memberdesc}


A \class{BaseHTTPRequestHandler} instance has the following methods:

\begin{methoddesc}{handle}{}
Calls \method{handle_one_request()} once (or, if persistent connections
are enabled, multiple times) to handle incoming HTTP requests.
You should never need to override it; instead, implement appropriate
\method{do_*()} methods.
\end{methoddesc}

\begin{methoddesc}{handle_one_request}{}
This method will parse and dispatch
the request to the appropriate \method{do_*()} method.  You should
never need to override it.
\end{methoddesc}

\begin{methoddesc}{send_error}{code\optional{, message}}
Sends and logs a complete error reply to the client. The numeric
\var{code} specifies the HTTP error code, with \var{message} as
optional, more specific text. A complete set of headers is sent,
followed by text composed using the \member{error_message_format}
class variable.
\end{methoddesc}

\begin{methoddesc}{send_response}{code\optional{, message}}
Sends a response header and logs the accepted request. The HTTP
response line is sent, followed by \emph{Server} and \emph{Date}
headers. The values for these two headers are picked up from the
\method{version_string()} and \method{date_time_string()} methods,
respectively.
\end{methoddesc}

\begin{methoddesc}{send_header}{keyword, value}
Writes a specific HTTP header to the output stream. \var{keyword}
should specify the header keyword, with \var{value} specifying
its value.
\end{methoddesc}

\begin{methoddesc}{end_headers}{}
Sends a blank line, indicating the end of the HTTP headers in
the response.
\end{methoddesc}

\begin{methoddesc}{log_request}{\optional{code\optional{, size}}}
Logs an accepted (successful) request. \var{code} should specify
the numeric HTTP code associated with the response. If a size of
the response is available, then it should be passed as the
\var{size} parameter.
\end{methoddesc}

\begin{methoddesc}{log_error}{...}
Logs an error when a request cannot be fulfilled. By default,
it passes the message to \method{log_message()}, so it takes the
same arguments (\var{format} and additional values).
\end{methoddesc}

\begin{methoddesc}{log_message}{format, ...}
Logs an arbitrary message to \code{sys.stderr}. This is typically
overridden to create custom error logging mechanisms. The
\var{format} argument is a standard printf-style format string,
where the additional arguments to \method{log_message()} are applied
as inputs to the formatting. The client address and current date
and time are prefixed to every message logged.
\end{methoddesc}

\begin{methoddesc}{version_string}{}
Returns the server software's version string. This is a combination
of the \member{server_version} and \member{sys_version} class variables.
\end{methoddesc}

\begin{methoddesc}{date_time_string}{}
Returns the current date and time, formatted for a message header.
\end{methoddesc}

\begin{methoddesc}{log_data_time_string}{}
Returns the current date and time, formatted for logging.
\end{methoddesc}

\begin{methoddesc}{address_string}{}
Returns the client address, formatted for logging. A name lookup
is performed on the client's IP address.
\end{methoddesc}


\begin{seealso}
  \seemodule{CGIHTTPServer}{Extended request handler that supports CGI
                            scripts.}

  \seemodule{SimpleHTTPServer}{Basic request handler that limits response
                               to files actually under the document root.}
\end{seealso}

\section{\module{SimpleHTTPServer} ---
         Simple HTTP request handler}

\declaremodule{standard}{SimpleHTTPServer}
\sectionauthor{Moshe Zadka}{moshez@zadka.site.co.il}
\modulesynopsis{This module provides a basic request handler for HTTP
                servers.}


The \module{SimpleHTTPServer} module defines a request-handler class,
interface-compatible with \class{BaseHTTPServer.BaseHTTPRequestHandler},
that serves files only from a base directory.

The \module{SimpleHTTPServer} module defines the following class:

\begin{classdesc}{SimpleHTTPRequestHandler}{request, client_address, server}
This class is used to serve files from the current directory and below,
directly mapping the directory structure to HTTP requests.

A lot of the work, such as parsing the request, is done by the base
class \class{BaseHTTPServer.BaseHTTPRequestHandler}.  This class
implements the \function{do_GET()} and \function{do_HEAD()} functions.
\end{classdesc}

The \class{SimpleHTTPRequestHandler} defines the following member
variables:

\begin{memberdesc}{server_version}
This will be \code{"SimpleHTTP/" + __version__}, where \code{__version__}
is defined in the module.
\end{memberdesc}

\begin{memberdesc}{extensions_map}
A dictionary mapping suffixes into MIME types. The default is signified
by an empty string, and is considered to be \code{application/octet-stream}.
The mapping is used case-insensitively, and so should contain only
lower-cased keys.
\end{memberdesc}

The \class{SimpleHTTPRequestHandler} defines the following methods:

\begin{methoddesc}{do_HEAD}{}
This method serves the \code{'HEAD'} request type: it sends the
headers it would send for the equivalent \code{GET} request. See the
\method{do_GET()} method for a more complete explanation of the possible
headers.
\end{methoddesc}

\begin{methoddesc}{do_GET}{}
The request is mapped to a local file by interpreting the request as
a path relative to the current working directory.

If the request was mapped to a directory, the directory is checked for
a file named \code{index.html} or \code{index.htm} (in that order).
If found, the file's contents are returned; otherwise a directory
listing is generated by calling the \method{list_directory()} method.
This method uses \function{os.listdir()} to scan the directory, and
returns a \code{404} error response if the \function{listdir()} fails.

If the request was mapped to a file, it is opened and the contents are
returned.  Any \exception{IOError} exception in opening the requested
file is mapped to a \code{404}, \code{'File not found'}
error. Otherwise, the content type is guessed by calling the
\method{guess_type()} method, which in turn uses the
\var{extensions_map} variable.

A \code{'Content-type:'} header with the guessed content type is
output, followed by a \code{'Content-Length:'} header with the file's
size and a \code{'Last-Modified:'} header with the file's modification
time.

Then follows a blank line signifying the end of the headers,
and then the contents of the file are output. If the file's MIME type
starts with \code{text/} the file is opened in text mode; otherwise
binary mode is used.

For example usage, see the implementation of the \function{test()}
function.
\versionadded[The \code{'Last-Modified'} header]{2.5}
\end{methoddesc}


\begin{seealso}
  \seemodule{BaseHTTPServer}{Base class implementation for Web server
                             and request handler.}
\end{seealso}

\section{\module{CGIHTTPServer} ---
         A Do-Something Request Handler}


\declaremodule{standard}{CGIHTTPServer}
  \platform{Unix}
\sectionauthor{Moshe Zadka}{mzadka@geocities.com}
\modulesynopsis{This module provides a request handler for HTTP servers
                which can run CGI scripts.}


The \module{CGIHTTPServer} module defines a request-handler class,
interface compatible with
\class{BaseHTTPServer.BaseHTTPRequestHandler} and inherits behavior
from \class{SimpleHTTPServer.SimpleHTTPRequestHandler} but can also
run CGI scripts.

\strong{Note:}  This module is \UNIX{} dependent since it creates the
CGI process using \function{os.fork()} and \function{os.exec()}.

The \module{CGIHTTPServer} module defines the following class:

\begin{classdesc}{CGIHTTPRequestHandler}{request, client_address, server}
This class is used to serve either files or output of CGI scripts from 
the current directory and below. Note that mapping HTTP hierarchic
structure to local directory structure is exactly as in
\class{SimpleHTTPServer.SimpleHTTPRequestHandler}.

The class will however, run the CGI script, instead of serving it as a
file, if it guesses it to be a CGI script. Only directory-based CGI
are used --- the other common server configuration is to treat special
extensions as denoting CGI scripts.

The \function{do_GET()} and \function{do_HEAD()} functions are
modified to run CGI scripts and serve the output, instead of serving
files, if the request leads to somewhere below the
\code{cgi_directories} path.
\end{classdesc}

The \class{CGIHTTPRequestHandler} defines the following data member:

\begin{memberdesc}{cgi_directories}
This defaults to \code{['/cgi-bin', '/htbin']} and describes
directories to treat as containing CGI scripts.
\end{memberdesc}

The \class{CGIHTTPRequestHandler} defines the following methods:

\begin{methoddesc}{do_POST}{}
This method serves the \code{'POST'} request type, only allowed for
CGI scripts.  Error 501, "Can only POST to CGI scripts", is output
when trying to POST to a non-CGI url.
\end{methoddesc}

Note that CGI scripts will be run with UID of user nobody, for security
reasons. Problems with the CGI script will be translated to error 403.

For example usage, see the implementation of the \function{test()}
function.


\begin{seealso}
  \seemodule{BaseHTTPServer}{Base class implementation for Web server
                             and request handler.}
\end{seealso}

\section{\module{asyncore} ---
         Asynchronous socket handler}

\declaremodule{builtin}{asyncore}
\modulesynopsis{A base class for developing asynchronous socket 
                handling services.}
\moduleauthor{Sam Rushing}{rushing@nightmare.com}
\sectionauthor{Christopher Petrilli}{petrilli@amber.org}
\sectionauthor{Steve Holden}{sholden@holdenweb.com}
% Heavily adapted from original documentation by Sam Rushing.

This module provides the basic infrastructure for writing asynchronous 
socket service clients and servers.

There are only two ways to have a program on a single processor do 
``more than one thing at a time.'' Multi-threaded programming is the 
simplest and most popular way to do it, but there is another very 
different technique, that lets you have nearly all the advantages of 
multi-threading, without actually using multiple threads.  It's really 
only practical if your program is largely I/O bound.  If your program 
is processor bound, then pre-emptive scheduled threads are probably what 
you really need. Network servers are rarely processor bound, however.

If your operating system supports the \cfunction{select()} system call 
in its I/O library (and nearly all do), then you can use it to juggle 
multiple communication channels at once; doing other work while your 
I/O is taking place in the ``background.''  Although this strategy can 
seem strange and complex, especially at first, it is in many ways 
easier to understand and control than multi-threaded programming.  
The \module{asyncore} module solves many of the difficult problems for 
you, making the task of building sophisticated high-performance 
network servers and clients a snap. For ``conversational'' applications
and protocols the companion  \refmodule{asynchat} module is invaluable.

The basic idea behind both modules is to create one or more network
\emph{channels}, instances of class \class{asyncore.dispatcher} and
\class{asynchat.async_chat}. Creating the channels adds them to a global
map, used by the \function{loop()} function if you do not provide it
with your own \var{map}.

Once the initial channel(s) is(are) created, calling the \function{loop()}
function activates channel service, which continues until the last
channel (including any that have been added to the map during asynchronous
service) is closed.

\begin{funcdesc}{loop}{\optional{timeout\optional{, use_poll\optional{,
                       map\optional{,count}}}}}
  Enter a polling loop that terminates after count passes or all open
  channels have been closed.  All arguments are optional.  The \var(count)
  parameter defaults to None, resulting in the loop terminating only
  when all channels have been closed.  The \var{timeout} argument sets the
  timeout parameter for the appropriate \function{select()} or
  \function{poll()} call, measured in seconds; the default is 30 seconds.
  The \var{use_poll} parameter, if true, indicates that \function{poll()}
  should be used in preference to \function{select()} (the default is
  \code{False}).  The \var{map} parameter is a dictionary whose items are
  the channels to watch.  As channels are closed they are deleted from their
  map.  If \var{map} is omitted, a global map is used (this map is updated
  by the default class \method{__init__()} -- make sure you extend, rather
  than override, \method{__init__()} if you want to retain this behavior).

  Channels (instances of \class{asyncore.dispatcher}, \class{asynchat.async_chat}
  and subclasses thereof) can freely be mixed in the map.
\end{funcdesc}

\begin{classdesc}{dispatcher}{}
  The \class{dispatcher} class is a thin wrapper around a low-level socket object.
  To make it more useful, it has a few methods for event-handling  which are called
  from the asynchronous loop.  
  Otherwise, it can be treated as a normal non-blocking socket object.

  Two class attributes can be modified, to improve performance,
  or possibly even to conserve memory.

  \begin{datadesc}{ac_in_buffer_size}
  The asynchronous input buffer size (default \code{4096}).
  \end{datadesc}

  \begin{datadesc}{ac_out_buffer_size}
  The asynchronous output buffer size (default \code{4096}).
  \end{datadesc}

  The firing of low-level events at certain times or in certain connection
  states tells the asynchronous loop that certain higher-level events have
  taken place. For example, if we have asked for a socket to connect to
  another host, we know that the connection has been made when the socket
  becomes writable for the first time (at this point you know that you may
  write to it with the expectation of success). The implied higher-level
  events are:

  \begin{tableii}{l|l}{code}{Event}{Description}
    \lineii{handle_connect()}{Implied by the first write event}
    \lineii{handle_close()}{Implied by a read event with no data available}
    \lineii{handle_accept()}{Implied by a read event on a listening socket}
  \end{tableii}

  During asynchronous processing, each mapped channel's \method{readable()}
  and \method{writable()} methods are used to determine whether the channel's
  socket should be added to the list of channels \cfunction{select()}ed or
  \cfunction{poll()}ed for read and write events.

\end{classdesc}

Thus, the set of channel events is larger than the basic socket events.
The full set of methods that can be overridden in your subclass follows:

\begin{methoddesc}{handle_read}{}
  Called when the asynchronous loop detects that a \method{read()}
  call on the channel's socket will succeed.
\end{methoddesc}

\begin{methoddesc}{handle_write}{}
  Called when the asynchronous loop detects that a writable socket
  can be written.  
  Often this method will implement the necessary buffering for 
  performance.  For example:

\begin{verbatim}
def handle_write(self):
    sent = self.send(self.buffer)
    self.buffer = self.buffer[sent:]
\end{verbatim}
\end{methoddesc}

\begin{methoddesc}{handle_expt}{}
  Called when there is out of band (OOB) data for a socket 
  connection.  This will almost never happen, as OOB is 
  tenuously supported and rarely used.
\end{methoddesc}

\begin{methoddesc}{handle_connect}{}
  Called when the active opener's socket actually makes a connection.
  Might send a ``welcome'' banner, or initiate a protocol
  negotiation with the remote endpoint, for example.
\end{methoddesc}

\begin{methoddesc}{handle_close}{}
  Called when the socket is closed.
\end{methoddesc}

\begin{methoddesc}{handle_error}{}
  Called when an exception is raised and not otherwise handled.  The default
  version prints a condensed traceback.
\end{methoddesc}

\begin{methoddesc}{handle_accept}{}
  Called on listening channels (passive openers) when a  
  connection can be established with a new remote endpoint that
  has issued a \method{connect()} call for the local endpoint.
\end{methoddesc}

\begin{methoddesc}{readable}{}
  Called each time around the asynchronous loop to determine whether a
  channel's socket should be added to the list on which read events can
  occur.  The default method simply returns \code{True}, 
  indicating that by default, all channels will be interested in
  read events.
\end{methoddesc}

\begin{methoddesc}{writable}{}
  Called each time around the asynchronous loop to determine whether a
  channel's socket should be added to the list on which write events can
  occur.  The default method simply returns \code{True}, 
  indicating that by default, all channels will be interested in
  write events.
\end{methoddesc}

In addition, each channel delegates or extends many of the socket methods.
Most of these are nearly identical to their socket partners.

\begin{methoddesc}{create_socket}{family, type}
  This is identical to the creation of a normal socket, and 
  will use the same options for creation.  Refer to the
  \refmodule{socket} documentation for information on creating
  sockets.
\end{methoddesc}

\begin{methoddesc}{connect}{address}
  As with the normal socket object, \var{address} is a 
  tuple with the first element the host to connect to, and the 
  second the port number.
\end{methoddesc}

\begin{methoddesc}{send}{data}
  Send \var{data} to the remote end-point of the socket.
\end{methoddesc}

\begin{methoddesc}{recv}{buffer_size}
  Read at most \var{buffer_size} bytes from the socket's remote end-point.
  An empty string implies that the channel has been closed from the other
  end.
\end{methoddesc}

\begin{methoddesc}{listen}{backlog}
  Listen for connections made to the socket.  The \var{backlog}
  argument specifies the maximum number of queued connections
  and should be at least 1; the maximum value is
  system-dependent (usually 5).
\end{methoddesc}

\begin{methoddesc}{bind}{address}
  Bind the socket to \var{address}.  The socket must not already
  be bound.  (The format of \var{address} depends on the address
  family --- see above.)
\end{methoddesc}

\begin{methoddesc}{accept}{}
  Accept a connection.  The socket must be bound to an address
  and listening for connections.  The return value is a pair
  \code{(\var{conn}, \var{address})} where \var{conn} is a
  \emph{new} socket object usable to send and receive data on
  the connection, and \var{address} is the address bound to the
  socket on the other end of the connection.
\end{methoddesc}

\begin{methoddesc}{close}{}
  Close the socket.  All future operations on the socket object
  will fail.  The remote end-point will receive no more data (after
  queued data is flushed).  Sockets are automatically closed
  when they are garbage-collected.
\end{methoddesc}


\subsection{asyncore Example basic HTTP client \label{asyncore-example}}

As a basic example, below is a very basic HTTP client that uses the 
\class{dispatcher} class to implement its socket handling:

\begin{verbatim}
class http_client(asyncore.dispatcher):
    def __init__(self, host,path):
        asyncore.dispatcher.__init__(self)
        self.path = path
        self.create_socket(socket.AF_INET, socket.SOCK_STREAM)
        self.connect( (host, 80) )
        self.buffer = 'GET %s HTTP/1.0\r\n\r\n' % self.path
        
    def handle_connect(self):
        pass
        
    def handle_read(self):
        data = self.recv(8192)
        print data
        
    def writable(self):
        return (len(self.buffer) > 0)
    
    def handle_write(self):
        sent = self.send(self.buffer)
        self.buffer = self.buffer[sent:]
\end{verbatim}


\chapter{Internet Data Handling \label{netdata}}

This chapter describes modules which support handling data formats
commonly used on the internet.  Some, like SGML and XML, may be useful 
for other applications as well.

\localmoduletable
			% Internet Data Handling
\section{\module{sgmllib} ---
         Simple SGML parser}

\declaremodule{standard}{sgmllib}
\modulesynopsis{Only as much of an SGML parser as needed to parse HTML.}

\index{SGML}

This module defines a class \class{SGMLParser} which serves as the
basis for parsing text files formatted in SGML (Standard Generalized
Mark-up Language).  In fact, it does not provide a full SGML parser
--- it only parses SGML insofar as it is used by HTML, and the module
only exists as a base for the \refmodule{htmllib}\refstmodindex{htmllib}
module.


\begin{classdesc}{SGMLParser}{}
The \class{SGMLParser} class is instantiated without arguments.
The parser is hardcoded to recognize the following
constructs:

\begin{itemize}
\item
Opening and closing tags of the form
\samp{<\var{tag} \var{attr}="\var{value}" ...>} and
\samp{</\var{tag}>}, respectively.

\item
Numeric character references of the form \samp{\&\#\var{name};}.

\item
Entity references of the form \samp{\&\var{name};}.

\item
SGML comments of the form \samp{<!--\var{text}-->}.  Note that
spaces, tabs, and newlines are allowed between the trailing
\samp{>} and the immediately preceeding \samp{--}.

\end{itemize}
\end{classdesc}

\class{SGMLParser} instances have the following interface methods:


\begin{methoddesc}{reset}{}
Reset the instance.  Loses all unprocessed data.  This is called
implicitly at instantiation time.
\end{methoddesc}

\begin{methoddesc}{setnomoretags}{}
Stop processing tags.  Treat all following input as literal input
(CDATA).  (This is only provided so the HTML tag
\code{<PLAINTEXT>} can be implemented.)
\end{methoddesc}

\begin{methoddesc}{setliteral}{}
Enter literal mode (CDATA mode).
\end{methoddesc}

\begin{methoddesc}{feed}{data}
Feed some text to the parser.  It is processed insofar as it consists
of complete elements; incomplete data is buffered until more data is
fed or \method{close()} is called.
\end{methoddesc}

\begin{methoddesc}{close}{}
Force processing of all buffered data as if it were followed by an
end-of-file mark.  This method may be redefined by a derived class to
define additional processing at the end of the input, but the
redefined version should always call \method{close()}.
\end{methoddesc}

\begin{methoddesc}{get_starttag_text}{}
Return the text of the most recently opened start tag.  This should
not normally be needed for structured processing, but may be useful in
dealing with HTML ``as deployed'' or for re-generating input with
minimal changes (whitespace between attributes can be preserved,
etc.).
\end{methoddesc}

\begin{methoddesc}{handle_starttag}{tag, method, attributes}
This method is called to handle start tags for which either a
\method{start_\var{tag}()} or \method{do_\var{tag}()} method has been
defined.  The \var{tag} argument is the name of the tag converted to
lower case, and the \var{method} argument is the bound method which
should be used to support semantic interpretation of the start tag.
The \var{attributes} argument is a list of \code{(\var{name},
\var{value})} pairs containing the attributes found inside the tag's
\code{<>} brackets.  The \var{name} has been translated to lower case
and double quotes and backslashes in the \var{value} have been interpreted.
For instance, for the tag \code{<A HREF="http://www.cwi.nl/">}, this
method would be called as \samp{unknown_starttag('a', [('href',
'http://www.cwi.nl/')])}.  The base implementation simply calls
\var{method} with \var{attributes} as the only argument.
\end{methoddesc}

\begin{methoddesc}{handle_endtag}{tag, method}
This method is called to handle endtags for which an
\method{end_\var{tag}()} method has been defined.  The
\var{tag} argument is the name of the tag converted to lower case, and
the \var{method} argument is the bound method which should be used to
support semantic interpretation of the end tag.  If no
\method{end_\var{tag}()} method is defined for the closing element,
this handler is not called.  The base implementation simply calls
\var{method}.
\end{methoddesc}

\begin{methoddesc}{handle_data}{data}
This method is called to process arbitrary data.  It is intended to be
overridden by a derived class; the base class implementation does
nothing.
\end{methoddesc}

\begin{methoddesc}{handle_charref}{ref}
This method is called to process a character reference of the form
\samp{\&\#\var{ref};}.  In the base implementation, \var{ref} must
be a decimal number in the
range 0-255.  It translates the character to \ASCII{} and calls the
method \method{handle_data()} with the character as argument.  If
\var{ref} is invalid or out of range, the method
\code{unknown_charref(\var{ref})} is called to handle the error.  A
subclass must override this method to provide support for named
character entities.
\end{methoddesc}

\begin{methoddesc}{handle_entityref}{ref}
This method is called to process a general entity reference of the
form \samp{\&\var{ref};} where \var{ref} is an general entity
reference.  It looks for \var{ref} in the instance (or class)
variable \member{entitydefs} which should be a mapping from entity
names to corresponding translations.  If a translation is found, it
calls the method \method{handle_data()} with the translation;
otherwise, it calls the method \code{unknown_entityref(\var{ref})}.
The default \member{entitydefs} defines translations for
\code{\&amp;}, \code{\&apos}, \code{\&gt;}, \code{\&lt;}, and
\code{\&quot;}.
\end{methoddesc}

\begin{methoddesc}{handle_comment}{comment}
This method is called when a comment is encountered.  The
\var{comment} argument is a string containing the text between the
\samp{<!--} and \samp{-->} delimiters, but not the delimiters
themselves.  For example, the comment \samp{<!--text-->} will
cause this method to be called with the argument \code{'text'}.  The
default method does nothing.
\end{methoddesc}

\begin{methoddesc}{report_unbalanced}{tag}
This method is called when an end tag is found which does not
correspond to any open element.
\end{methoddesc}

\begin{methoddesc}{unknown_starttag}{tag, attributes}
This method is called to process an unknown start tag.  It is intended
to be overridden by a derived class; the base class implementation
does nothing.
\end{methoddesc}

\begin{methoddesc}{unknown_endtag}{tag}
This method is called to process an unknown end tag.  It is intended
to be overridden by a derived class; the base class implementation
does nothing.
\end{methoddesc}

\begin{methoddesc}{unknown_charref}{ref}
This method is called to process unresolvable numeric character
references.  Refer to \method{handle_charref()} to determine what is
handled by default.  It is intended to be overridden by a derived
class; the base class implementation does nothing.
\end{methoddesc}

\begin{methoddesc}{unknown_entityref}{ref}
This method is called to process an unknown entity reference.  It is
intended to be overridden by a derived class; the base class
implementation does nothing.
\end{methoddesc}

Apart from overriding or extending the methods listed above, derived
classes may also define methods of the following form to define
processing of specific tags.  Tag names in the input stream are case
independent; the \var{tag} occurring in method names must be in lower
case:

\begin{methoddescni}{start_\var{tag}}{attributes}
This method is called to process an opening tag \var{tag}.  It has
preference over \method{do_\var{tag}()}.  The
\var{attributes} argument has the same meaning as described for
\method{handle_starttag()} above.
\end{methoddescni}

\begin{methoddescni}{do_\var{tag}}{attributes}
This method is called to process an opening tag \var{tag} that does
not come with a matching closing tag.  The \var{attributes} argument
has the same meaning as described for \method{handle_starttag()} above.
\end{methoddescni}

\begin{methoddescni}{end_\var{tag}}{}
This method is called to process a closing tag \var{tag}.
\end{methoddescni}

Note that the parser maintains a stack of open elements for which no
end tag has been found yet.  Only tags processed by
\method{start_\var{tag}()} are pushed on this stack.  Definition of an
\method{end_\var{tag}()} method is optional for these tags.  For tags
processed by \method{do_\var{tag}()} or by \method{unknown_tag()}, no
\method{end_\var{tag}()} method must be defined; if defined, it will
not be used.  If both \method{start_\var{tag}()} and
\method{do_\var{tag}()} methods exist for a tag, the
\method{start_\var{tag}()} method takes precedence.

\section{\module{htmllib} ---
         A parser for HTML documents}

\declaremodule{standard}{htmllib}
\modulesynopsis{A parser for HTML documents.}

\index{HTML}
\index{hypertext}


This module defines a class which can serve as a base for parsing text
files formatted in the HyperText Mark-up Language (HTML).  The class
is not directly concerned with I/O --- it must be provided with input
in string form via a method, and makes calls to methods of a
``formatter'' object in order to produce output.  The
\class{HTMLParser} class is designed to be used as a base class for
other classes in order to add functionality, and allows most of its
methods to be extended or overridden.  In turn, this class is derived
from and extends the \class{SGMLParser} class defined in module
\refmodule{sgmllib}\refstmodindex{sgmllib}.  The \class{HTMLParser}
implementation supports the HTML 2.0 language as described in
\rfc{1866}.  Two implementations of formatter objects are provided in
the \refmodule{formatter}\refstmodindex{formatter}\ module; refer to the
documentation for that module for information on the formatter
interface.
\withsubitem{(in module sgmllib)}{\ttindex{SGMLParser}}

The following is a summary of the interface defined by
\class{sgmllib.SGMLParser}:

\begin{itemize}

\item
The interface to feed data to an instance is through the \method{feed()}
method, which takes a string argument.  This can be called with as
little or as much text at a time as desired; \samp{p.feed(a);
p.feed(b)} has the same effect as \samp{p.feed(a+b)}.  When the data
contains complete HTML tags, these are processed immediately;
incomplete elements are saved in a buffer.  To force processing of all
unprocessed data, call the \method{close()} method.

For example, to parse the entire contents of a file, use:
\begin{verbatim}
parser.feed(open('myfile.html').read())
parser.close()
\end{verbatim}

\item
The interface to define semantics for HTML tags is very simple: derive
a class and define methods called \method{start_\var{tag}()},
\method{end_\var{tag}()}, or \method{do_\var{tag}()}.  The parser will
call these at appropriate moments: \method{start_\var{tag}} or
\method{do_\var{tag}()} is called when an opening tag of the form
\code{<\var{tag} ...>} is encountered; \method{end_\var{tag}()} is called
when a closing tag of the form \code{<\var{tag}>} is encountered.  If
an opening tag requires a corresponding closing tag, like \code{<H1>}
... \code{</H1>}, the class should define the \method{start_\var{tag}()}
method; if a tag requires no closing tag, like \code{<P>}, the class
should define the \method{do_\var{tag}()} method.

\end{itemize}

The module defines a single class:

\begin{classdesc}{HTMLParser}{formatter}
This is the basic HTML parser class.  It supports all entity names
required by the HTML 2.0 specification (\rfc{1866}).  It also defines
handlers for all HTML 2.0 and many HTML 3.0 and 3.2 elements.
\end{classdesc}


\begin{seealso}
  \seemodule{formatter}{Interface definition for transforming an
                        abstract flow of formatting events into
                        specific output events on writer objects.}
  \seemodule{HTMLParser}{Alternate HTML parser that offers a slightly
                         lower-level view of the input, but is
                         designed to work with XHTML, and does not
                         implement some of the SGML syntax not used in
                         ``HTML as deployed'' and which isn't legal
                         for XHTML.}
  \seemodule{htmlentitydefs}{Definition of replacement text for HTML
                             2.0 entities.}
  \seemodule{sgmllib}{Base class for \class{HTMLParser}.}
\end{seealso}


\subsection{HTMLParser Objects \label{html-parser-objects}}

In addition to tag methods, the \class{HTMLParser} class provides some
additional methods and instance variables for use within tag methods.

\begin{memberdesc}{formatter}
This is the formatter instance associated with the parser.
\end{memberdesc}

\begin{memberdesc}{nofill}
Boolean flag which should be true when whitespace should not be
collapsed, or false when it should be.  In general, this should only
be true when character data is to be treated as ``preformatted'' text,
as within a \code{<PRE>} element.  The default value is false.  This
affects the operation of \method{handle_data()} and \method{save_end()}.
\end{memberdesc}


\begin{methoddesc}{anchor_bgn}{href, name, type}
This method is called at the start of an anchor region.  The arguments
correspond to the attributes of the \code{<A>} tag with the same
names.  The default implementation maintains a list of hyperlinks
(defined by the \code{HREF} attribute for \code{<A>} tags) within the
document.  The list of hyperlinks is available as the data attribute
\member{anchorlist}.
\end{methoddesc}

\begin{methoddesc}{anchor_end}{}
This method is called at the end of an anchor region.  The default
implementation adds a textual footnote marker using an index into the
list of hyperlinks created by \method{anchor_bgn()}.
\end{methoddesc}

\begin{methoddesc}{handle_image}{source, alt\optional{, ismap\optional{, align\optional{, width\optional{, height}}}}}
This method is called to handle images.  The default implementation
simply passes the \var{alt} value to the \method{handle_data()}
method.
\end{methoddesc}

\begin{methoddesc}{save_bgn}{}
Begins saving character data in a buffer instead of sending it to the
formatter object.  Retrieve the stored data via \method{save_end()}.
Use of the \method{save_bgn()} / \method{save_end()} pair may not be
nested.
\end{methoddesc}

\begin{methoddesc}{save_end}{}
Ends buffering character data and returns all data saved since the
preceding call to \method{save_bgn()}.  If the \member{nofill} flag is
false, whitespace is collapsed to single spaces.  A call to this
method without a preceding call to \method{save_bgn()} will raise a
\exception{TypeError} exception.
\end{methoddesc}



\section{\module{htmlentitydefs} ---
         Definitions of HTML general entities}

\declaremodule{standard}{htmlentitydefs}
\modulesynopsis{Definitions of HTML general entities.}
\sectionauthor{Fred L. Drake, Jr.}{fdrake@acm.org}

This module defines three dictionaries, \code{name2codepoint},
\code{codepoint2name}, and \code{entitydefs}. \code{entitydefs} is
used by the \refmodule{htmllib} module to provide the
\member{entitydefs} member of the \class{HTMLParser} class.  The
definition provided here contains all the entities defined by XHTML 1.0 
that can be handled using simple textual substitution in the Latin-1
character set (ISO-8859-1).


\begin{datadesc}{entitydefs}
  A dictionary mapping XHTML 1.0 entity definitions to their
  replacement text in ISO Latin-1.

\end{datadesc}

\begin{datadesc}{name2codepoint}
  A dictionary that maps HTML entity names to the Unicode codepoints.
  \versionadded{2.3}
\end{datadesc}

\begin{datadesc}{codepoint2name}
  A dictionary that maps Unicode codepoints to HTML entity names.
  \versionadded{2.3}
\end{datadesc}

\section{Standard Module \sectcode{xmllib}}
% Author: Sjoerd Mullender
\label{module-xmllib}
\stmodindex{xmllib}
\index{XML}

This module defines a class \code{XMLParser} which serves as the basis 
for parsing text files formatted in XML (eXtended Markup Language).

The \code{XMLParser} class must be instantiated without arguments.  It 
has the following interface methods:

\setindexsubitem{(XMLParser method)}

\begin{funcdesc}{reset}{}
Reset the instance.  Loses all unprocessed data.  This is called
implicitly at the instantiation time.
\end{funcdesc}

\begin{funcdesc}{setnomoretags}{}
Stop processing tags.  Treat all following input as literal input
(CDATA).
\end{funcdesc}

\begin{funcdesc}{setliteral}{}
Enter literal mode (CDATA mode).
\end{funcdesc}

\begin{funcdesc}{feed}{data}
Feed some text to the parser.  It is processed insofar as it consists
of complete elements; incomplete data is buffered until more data is
fed or \code{close()} is called.
\end{funcdesc}

\begin{funcdesc}{close}{}
Force processing of all buffered data as if it were followed by an
end-of-file mark.  This method may be redefined by a derived class to
define additional processing at the end of the input, but the
redefined version should always call \code{XMLParser.close()}.
\end{funcdesc}

\begin{funcdesc}{translate_references}{data}
Translate all entity and character references in \code{data} and
returns the translated string.
\end{funcdesc}

\begin{funcdesc}{handle_xml}{encoding\, standalone}
This method is called when the \code{<?xml ...?>} tag is processed.
The arguments are the values of the encoding and standalone attributes 
in the tag.  Both encoding and standalone are optional.  The values
passed to \code{handle_xml} default to \code{None} and the string
\code{'no'} respectively.
\end{funcdesc}

\begin{funcdesc}{handle_doctype}{tag\, data}
This method is called when the \code{<!DOCTYPE...>} tag is processed.
The arguments are the name of the root element and the uninterpreted
contents of the tag, starting after the white space after the name of
the root element.
\end{funcdesc}

\begin{funcdesc}{handle_starttag}{tag\, method\, attributes}
This method is called to handle start tags for which a
\code{start_\var{tag}()} method has been defined.  The \code{tag}
argument is the name of the tag, and the \code{method} argument is the
bound method which should be used to support semantic interpretation
of the start tag.  The \var{attributes} argument is a dictionary of
attributes, the key being the \var{name} and the value being the
\var{value} of the attribute found inside the tag's \code{<>} brackets.
Character and entity references in the \var{value} have
been interpreted.  For instance, for the tag
\code{<A HREF="http://www.cwi.nl/">}, this method would be called as
\code{handle_starttag('A', self.start_A, \{'HREF': 'http://www.cwi.nl/'\})}.
The base implementation simply calls \code{method} with \code{attributes}
as the only argument.
\end{funcdesc}

\begin{funcdesc}{handle_endtag}{tag\, method}
This method is called to handle endtags for which an
\code{end_\var{tag}()} method has been defined.  The \code{tag}
argument is the name of the tag, and the
\code{method} argument is the bound method which should be used to
support semantic interpretation of the end tag.  If no
\code{end_\var{tag}()} method is defined for the closing element, this
handler is not called.  The base implementation simply calls
\code{method}.
\end{funcdesc}

\begin{funcdesc}{handle_data}{data}
This method is called to process arbitrary data.  It is intended to be
overridden by a derived class; the base class implementation does
nothing.
\end{funcdesc}

\begin{funcdesc}{handle_charref}{ref}
This method is called to process a character reference of the form
\samp{\&\#\var{ref};}.  \var{ref} can either be a decimal number,
or a hexadecimal number when preceded by \code{x}.
In the base implementation, \var{ref} must be a number in the
range 0-255.  It translates the character to \ASCII{} and calls the
method \code{handle_data()} with the character as argument.  If
\var{ref} is invalid or out of range, the method
\code{unknown_charref(\var{ref})} is called to handle the error.  A
subclass must override this method to provide support for character
references outside of the \ASCII{} range.
\end{funcdesc}

\begin{funcdesc}{handle_entityref}{ref}
This method is called to process a general entity reference of the form
\samp{\&\var{ref};} where \var{ref} is an general entity
reference.  It looks for \var{ref} in the instance (or class)
variable \code{entitydefs} which should be a mapping from entity names
to corresponding translations.
If a translation is found, it calls the method \code{handle_data()}
with the translation; otherwise, it calls the method
\code{unknown_entityref(\var{ref})}.  The default \code{entitydefs}
defines translations for \code{\&amp;}, \code{\&apos}, \code{\&gt;},
\code{\&lt;}, and \code{\&quot;}.
\end{funcdesc}

\begin{funcdesc}{handle_comment}{comment}
This method is called when a comment is encountered.  The
\code{comment} argument is a string containing the text between the
\samp{<!--} and \samp{-->} delimiters, but not the delimiters
themselves.  For example, the comment \samp{<!--text-->} will
cause this method to be called with the argument \code{'text'}.  The
default method does nothing.
\end{funcdesc}

\begin{funcdesc}{handle_cdata}{data}
This method is called when a CDATA element is encountered.  The
\code{data} argument is a string containing the text between the
\samp{<![CDATA[} and \samp{]]>} delimiters, but not the delimiters
themselves.  For example, the entity \samp{<![CDATA[text]]>} will
cause this method to be called with the argument \code{'text'}.  The
default method does nothing.
\end{funcdesc}

\begin{funcdesc}{handle_proc}{name\, data}
This method is called when a processing instruction (PI) is encountered.  The
\code{name} is the PI target, and the \code{data} argument is a
string containing the text between the PI target and the closing delimiter,
but not the delimiter itself.  For example, the instruction
\samp{<?XML text?>} will cause this method to be called with the
arguments \code{'XML'} and \code{'text'}.  The default method does
nothing.  Note that if a document starts with a \code{<?xml ...?>}
tag, \code{handle_xml} is called to handle it.
\end{funcdesc}

\begin{funcdesc}{handle_special}{data}
This method is called when a declaration is encountered.  The
\code{data} argument is a string containing the text between the
\samp{<!} and \samp{>} delimiters, but not the delimiters
themselves.  For example, the entity \samp{<!ENTITY text>} will
cause this method to be called with the argument \code{'ENTITY text'}.  The
default method does nothing.  Note that \code{<!DOCTYPE ...>} is
handled separately if it is located at the start of the document.
\end{funcdesc}

\begin{funcdesc}{syntax_error}{message}
This method is called when a syntax error is encountered.  The
\code{message} is a description of what was wrong.  The default method 
raises a \code{RuntimeError} exception.  If this method is overridden, 
it is permissable for it to return.  This method is only called when
the error can be recovered from.  Unrecoverable errors raise a
\code{RuntimeError} without first calling \code{syntax_error}.
\end{funcdesc}

\begin{funcdesc}{unknown_starttag}{tag\, attributes}
This method is called to process an unknown start tag.  It is intended
to be overridden by a derived class; the base class implementation
does nothing.
\end{funcdesc}

\begin{funcdesc}{unknown_endtag}{tag}
This method is called to process an unknown end tag.  It is intended
to be overridden by a derived class; the base class implementation
does nothing.
\end{funcdesc}

\begin{funcdesc}{unknown_charref}{ref}
This method is called to process unresolvable numeric character
references.  It is intended to be overridden by a derived class; the
base class implementation does nothing.
\end{funcdesc}

\begin{funcdesc}{unknown_entityref}{ref}
This method is called to process an unknown entity reference.  It is
intended to be overridden by a derived class; the base class
implementation does nothing.
\end{funcdesc}

Apart from overriding or extending the methods listed above, derived
classes may also define methods and variables of the following form to
define processing of specific tags.  Tag names in the input stream are
case dependent; the \var{tag} occurring in method names must be in the
correct case:

\begin{funcdescni}{start_\var{tag}}{attributes}
This method is called to process an opening tag \var{tag}.  The
\var{attributes} argument has the same meaning as described for
\code{handle_starttag()} above.  In fact, the base implementation of
\code{handle_starttag()} calls this method.
\end{funcdescni}

\begin{funcdescni}{end_\var{tag}}{}
This method is called to process a closing tag \var{tag}.
\end{funcdescni}

\begin{datadescni}{\var{tag}_attributes}
If a class or instance variable \code{\var{tag}_attributes} exists, it 
should be a list or a dictionary.  If a list, the elements of the list 
are the valid attributes for the element \var{tag}; if a dictionary,
the keys are the valid attributes for the element \var{tag}, and the
values the default values of the attributes, or \code{None} if there
is no default.
In addition to the attributes that were present in the tag, the
attribute dictionary that is passed to \code{handle_starttag()} and
\code{unknown_starttag()} contains values for all attributes that have a
default value.
\end{datadescni}

\section{Standard Module \sectcode{formatter}}
\label{module-formatter}
\stmodindex{formatter}


This module supports two interface definitions, each with mulitple
implementations.  The \emph{formatter} interface is used by the
\class{HTMLParser} class of the \module{htmllib} module, and the
\emph{writer} interface is required by the formatter interface.
\withsubitem{(im module htmllib)}{\ttindex{HTMLParser}}

Formatter objects transform an abstract flow of formatting events into
specific output events on writer objects.  Formatters manage several
stack structures to allow various properties of a writer object to be
changed and restored; writers need not be able to handle relative
changes nor any sort of ``change back'' operation.  Specific writer
properties which may be controlled via formatter objects are
horizontal alignment, font, and left margin indentations.  A mechanism
is provided which supports providing arbitrary, non-exclusive style
settings to a writer as well.  Additional interfaces facilitate
formatting events which are not reversible, such as paragraph
separation.

Writer objects encapsulate device interfaces.  Abstract devices, such
as file formats, are supported as well as physical devices.  The
provided implementations all work with abstract devices.  The
interface makes available mechanisms for setting the properties which
formatter objects manage and inserting data into the output.


\subsection{The Formatter Interface}

Interfaces to create formatters are dependent on the specific
formatter class being instantiated.  The interfaces described below
are the required interfaces which all formatters must support once
initialized.

One data element is defined at the module level:


\begin{datadesc}{AS_IS}
Value which can be used in the font specification passed to the
\code{push_font()} method described below, or as the new value to any
other \code{push_\var{property}()} method.  Pushing the \code{AS_IS}
value allows the corresponding \code{pop_\var{property}()} method to
be called without having to track whether the property was changed.
\end{datadesc}

The following attributes are defined for formatter instance objects:


\begin{memberdesc}[formatter]{writer}
The writer instance with which the formatter interacts.
\end{memberdesc}


\begin{methoddesc}[formatter]{end_paragraph}{blanklines}
Close any open paragraphs and insert at least \var{blanklines}
before the next paragraph.
\end{methoddesc}

\begin{methoddesc}[formatter]{add_line_break}{}
Add a hard line break if one does not already exist.  This does not
break the logical paragraph.
\end{methoddesc}

\begin{methoddesc}[formatter]{add_hor_rule}{*args, **kw}
Insert a horizontal rule in the output.  A hard break is inserted if
there is data in the current paragraph, but the logical paragraph is
not broken.  The arguments and keywords are passed on to the writer's
\method{send_line_break()} method.
\end{methoddesc}

\begin{methoddesc}[formatter]{add_flowing_data}{data}
Provide data which should be formatted with collapsed whitespaces.
Whitespace from preceeding and successive calls to
\method{add_flowing_data()} is considered as well when the whitespace
collapse is performed.  The data which is passed to this method is
expected to be word-wrapped by the output device.  Note that any
word-wrapping still must be performed by the writer object due to the
need to rely on device and font information.
\end{methoddesc}

\begin{methoddesc}[formatter]{add_literal_data}{data}
Provide data which should be passed to the writer unchanged.
Whitespace, including newline and tab characters, are considered legal
in the value of \var{data}.  
\end{methoddesc}

\begin{methoddesc}[formatter]{add_label_data}{format, counter}
Insert a label which should be placed to the left of the current left
margin.  This should be used for constructing bulleted or numbered
lists.  If the \var{format} value is a string, it is interpreted as a
format specification for \var{counter}, which should be an integer.
The result of this formatting becomes the value of the label; if
\var{format} is not a string it is used as the label value directly.
The label value is passed as the only argument to the writer's
\method{send_label_data()} method.  Interpretation of non-string label
values is dependent on the associated writer.

Format specifications are strings which, in combination with a counter
value, are used to compute label values.  Each character in the format
string is copied to the label value, with some characters recognized
to indicate a transform on the counter value.  Specifically, the
character \character{1} represents the counter value formatter as an
arabic number, the characters \character{A} and \character{a}
represent alphabetic representations of the counter value in upper and
lower case, respectively, and \character{I} and \character{i}
represent the counter value in Roman numerals, in upper and lower
case.  Note that the alphabetic and roman transforms require that the
counter value be greater than zero.
\end{methoddesc}

\begin{methoddesc}[formatter]{flush_softspace}{}
Send any pending whitespace buffered from a previous call to
\method{add_flowing_data()} to the associated writer object.  This
should be called before any direct manipulation of the writer object.
\end{methoddesc}

\begin{methoddesc}[formatter]{push_alignment}{align}
Push a new alignment setting onto the alignment stack.  This may be
\constant{AS_IS} if no change is desired.  If the alignment value is
changed from the previous setting, the writer's \method{new_alignment()}
method is called with the \var{align} value.
\end{methoddesc}

\begin{methoddesc}[formatter]{pop_alignment}{}
Restore the previous alignment.
\end{methoddesc}

\begin{methoddesc}[formatter]{push_font}{\code{(}size, italic, bold, teletype\code{)}}
Change some or all font properties of the writer object.  Properties
which are not set to \constant{AS_IS} are set to the values passed in
while others are maintained at their current settings.  The writer's
\method{new_font()} method is called with the fully resolved font
specification.
\end{methoddesc}

\begin{methoddesc}[formatter]{pop_font}{}
Restore the previous font.
\end{methoddesc}

\begin{methoddesc}[formatter]{push_margin}{margin}
Increase the number of left margin indentations by one, associating
the logical tag \var{margin} with the new indentation.  The initial
margin level is \code{0}.  Changed values of the logical tag must be
true values; false values other than \constant{AS_IS} are not
sufficient to change the margin.
\end{methoddesc}

\begin{methoddesc}[formatter]{pop_margin}{}
Restore the previous margin.
\end{methoddesc}

\begin{methoddesc}[formatter]{push_style}{*styles}
Push any number of arbitrary style specifications.  All styles are
pushed onto the styles stack in order.  A tuple representing the
entire stack, including \constant{AS_IS} values, is passed to the
writer's \method{new_styles()} method.
\end{methoddesc}

\begin{methoddesc}[formatter]{pop_style}{\optional{n\code{ = 1}}}
Pop the last \var{n} style specifications passed to
\method{push_style()}.  A tuple representing the revised stack,
including \constant{AS_IS} values, is passed to the writer's
\method{new_styles()} method.
\end{methoddesc}

\begin{methoddesc}[formatter]{set_spacing}{spacing}
Set the spacing style for the writer.
\end{methoddesc}

\begin{methoddesc}[formatter]{assert_line_data}{\optional{flag\code{ = 1}}}
Inform the formatter that data has been added to the current paragraph
out-of-band.  This should be used when the writer has been manipulated
directly.  The optional \var{flag} argument can be set to false if
the writer manipulations produced a hard line break at the end of the
output.
\end{methoddesc}


\subsection{Formatter Implementations}

Two implementations of formatter objects are provided by this module.
Most applications may use one of these classes without modification or
subclassing.

\begin{classdesc}{NullFormatter}{\optional{writer}}
A formatter which does nothing.  If \var{writer} is omitted, a
\class{NullWriter} instance is created.  No methods of the writer are
called by \class{NullFormatter} instances.  Implementations should
inherit from this class if implementing a writer interface but don't
need to inherit any implementation.
\end{classdesc}

\begin{classdesc}{AbstractFormatter}{writer}
The standard formatter.  This implementation has demonstrated wide
applicability to many writers, and may be used directly in most
circumstances.  It has been used to implement a full-featured
world-wide web browser.
\end{classdesc}



\subsection{The Writer Interface}

Interfaces to create writers are dependent on the specific writer
class being instantiated.  The interfaces described below are the
required interfaces which all writers must support once initialized.
Note that while most applications can use the
\class{AbstractFormatter} class as a formatter, the writer must
typically be provided by the application.


\begin{methoddesc}[writer]{flush}{}
Flush any buffered output or device control events.
\end{methoddesc}

\begin{methoddesc}[writer]{new_alignment}{align}
Set the alignment style.  The \var{align} value can be any object,
but by convention is a string or \code{None}, where \code{None}
indicates that the writer's ``preferred'' alignment should be used.
Conventional \var{align} values are \code{'left'}, \code{'center'},
\code{'right'}, and \code{'justify'}.
\end{methoddesc}

\begin{methoddesc}[writer]{new_font}{font}
Set the font style.  The value of \var{font} will be \code{None},
indicating that the device's default font should be used, or a tuple
of the form \code{(}\var{size}, \var{italic}, \var{bold},
\var{teletype}\code{)}.  Size will be a string indicating the size of
font that should be used; specific strings and their interpretation
must be defined by the application.  The \var{italic}, \var{bold}, and
\var{teletype} values are boolean indicators specifying which of those
font attributes should be used.
\end{methoddesc}

\begin{methoddesc}[writer]{new_margin}{margin, level}
Set the margin level to the integer \var{level} and the logical tag
to \var{margin}.  Interpretation of the logical tag is at the
writer's discretion; the only restriction on the value of the logical
tag is that it not be a false value for non-zero values of
\var{level}.
\end{methoddesc}

\begin{methoddesc}[writer]{new_spacing}{spacing}
Set the spacing style to \var{spacing}.
\end{methoddesc}

\begin{methoddesc}[writer]{new_styles}{styles}
Set additional styles.  The \var{styles} value is a tuple of
arbitrary values; the value \constant{AS_IS} should be ignored.  The
\var{styles} tuple may be interpreted either as a set or as a stack
depending on the requirements of the application and writer
implementation.
\end{methoddesc}

\begin{methoddesc}[writer]{send_line_break}{}
Break the current line.
\end{methoddesc}

\begin{methoddesc}[writer]{send_paragraph}{blankline}
Produce a paragraph separation of at least \var{blankline} blank
lines, or the equivelent.  The \var{blankline} value will be an
integer.
\end{methoddesc}

\begin{methoddesc}[writer]{send_hor_rule}{*args, **kw}
Display a horizontal rule on the output device.  The arguments to this
method are entirely application- and writer-specific, and should be
interpreted with care.  The method implementation may assume that a
line break has already been issued via \method{send_line_break()}.
\end{methoddesc}

\begin{methoddesc}[writer]{send_flowing_data}{data}
Output character data which may be word-wrapped and re-flowed as
needed.  Within any sequence of calls to this method, the writer may
assume that spans of multiple whitespace characters have been
collapsed to single space characters.
\end{methoddesc}

\begin{methoddesc}[writer]{send_literal_data}{data}
Output character data which has already been formatted
for display.  Generally, this should be interpreted to mean that line
breaks indicated by newline characters should be preserved and no new
line breaks should be introduced.  The data may contain embedded
newline and tab characters, unlike data provided to the
\method{send_formatted_data()} interface.
\end{methoddesc}

\begin{methoddesc}[writer]{send_label_data}{data}
Set \var{data} to the left of the current left margin, if possible.
The value of \var{data} is not restricted; treatment of non-string
values is entirely application- and writer-dependent.  This method
will only be called at the beginning of a line.
\end{methoddesc}


\subsection{Writer Implementations}

Three implementations of the writer object interface are provided as
examples by this module.  Most applications will need to derive new
writer classes from the \class{NullWriter} class.

\begin{classdesc}{NullWriter}{}
A writer which only provides the interface definition; no actions are
taken on any methods.  This should be the base class for all writers
which do not need to inherit any implementation methods.
\end{classdesc}

\begin{classdesc}{AbstractWriter}{}
A writer which can be used in debugging formatters, but not much
else.  Each method simply announces itself by printing its name and
arguments on standard output.
\end{classdesc}

\begin{classdesc}{DumbWriter}{\optional{file\optional{, maxcol\code{ = 72}}}}
Simple writer class which writes output on the file object passed in
as \var{file} or, if \var{file} is omitted, on standard output.  The
output is simply word-wrapped to the number of columns specified by
\var{maxcol}.  This class is suitable for reflowing a sequence of
paragraphs.
\end{classdesc}

\section{\module{rfc822} ---
         Parse RFC 2822 mail headers}

\declaremodule{standard}{rfc822}
\modulesynopsis{Parse \rfc{2822} style mail messages.}

This module defines a class, \class{Message}, which represents an
``email message'' as defined by the Internet standard
\rfc{2822}.\footnote{This module originally conformed to \rfc{822},
hence the name.  Since then, \rfc{2822} has been released as an
update to \rfc{822}.  This module should be considered
\rfc{2822}-conformant, especially in cases where the
syntax or semantics have changed since \rfc{822}.}  Such messages
consist of a collection of message headers, and a message body.  This
module also defines a helper class
\class{AddressList} for parsing \rfc{2822} addresses.  Please refer to
the RFC for information on the specific syntax of \rfc{2822} messages.

The \refmodule{mailbox}\refstmodindex{mailbox} module provides classes 
to read mailboxes produced by various end-user mail programs.

\begin{classdesc}{Message}{file\optional{, seekable}}
A \class{Message} instance is instantiated with an input object as
parameter.  Message relies only on the input object having a
\method{readline()} method; in particular, ordinary file objects
qualify.  Instantiation reads headers from the input object up to a
delimiter line (normally a blank line) and stores them in the
instance.  The message body, following the headers, is not consumed.

This class can work with any input object that supports a
\method{readline()} method.  If the input object has seek and tell
capability, the \method{rewindbody()} method will work; also, illegal
lines will be pushed back onto the input stream.  If the input object
lacks seek but has an \method{unread()} method that can push back a
line of input, \class{Message} will use that to push back illegal
lines.  Thus this class can be used to parse messages coming from a
buffered stream.

The optional \var{seekable} argument is provided as a workaround for
certain stdio libraries in which \cfunction{tell()} discards buffered
data before discovering that the \cfunction{lseek()} system call
doesn't work.  For maximum portability, you should set the seekable
argument to zero to prevent that initial \method{tell()} when passing
in an unseekable object such as a a file object created from a socket
object.

Input lines as read from the file may either be terminated by CR-LF or
by a single linefeed; a terminating CR-LF is replaced by a single
linefeed before the line is stored.

All header matching is done independent of upper or lower case;
e.g.\ \code{\var{m}['From']}, \code{\var{m}['from']} and
\code{\var{m}['FROM']} all yield the same result.
\end{classdesc}

\begin{classdesc}{AddressList}{field}
You may instantiate the \class{AddressList} helper class using a single
string parameter, a comma-separated list of \rfc{2822} addresses to be
parsed.  (The parameter \code{None} yields an empty list.)
\end{classdesc}

\begin{funcdesc}{quote}{str}
Return a new string with backslashes in \var{str} replaced by two
backslashes and double quotes replaced by backslash-double quote.
\end{funcdesc}

\begin{funcdesc}{unquote}{str}
Return a new string which is an \emph{unquoted} version of \var{str}.
If \var{str} ends and begins with double quotes, they are stripped
off.  Likewise if \var{str} ends and begins with angle brackets, they
are stripped off.
\end{funcdesc}

\begin{funcdesc}{parseaddr}{address}
Parse \var{address}, which should be the value of some address-containing
field such as \code{To:} or \code{Cc:}, into its constituent
``realname'' and ``email address'' parts.  Returns a tuple of that
information, unless the parse fails, in which case a 2-tuple of
\code{(None, None)} is returned.
\end{funcdesc}

\begin{funcdesc}{dump_address_pair}{pair}
The inverse of \method{parseaddr()}, this takes a 2-tuple of the form
\code{(realname, email_address)} and returns the string value suitable
for a \code{To:} or \code{Cc:} header.  If the first element of
\var{pair} is false, then the second element is returned unmodified.
\end{funcdesc}

\begin{funcdesc}{parsedate}{date}
Attempts to parse a date according to the rules in \rfc{2822}.
however, some mailers don't follow that format as specified, so
\function{parsedate()} tries to guess correctly in such cases. 
\var{date} is a string containing an \rfc{2822} date, such as 
\code{'Mon, 20 Nov 1995 19:12:08 -0500'}.  If it succeeds in parsing
the date, \function{parsedate()} returns a 9-tuple that can be passed
directly to \function{time.mktime()}; otherwise \code{None} will be
returned.  Note that fields 6, 7, and 8 of the result tuple are not
usable.
\end{funcdesc}

\begin{funcdesc}{parsedate_tz}{date}
Performs the same function as \function{parsedate()}, but returns
either \code{None} or a 10-tuple; the first 9 elements make up a tuple
that can be passed directly to \function{time.mktime()}, and the tenth
is the offset of the date's timezone from UTC (which is the official
term for Greenwich Mean Time).  (Note that the sign of the timezone
offset is the opposite of the sign of the \code{time.timezone}
variable for the same timezone; the latter variable follows the
\POSIX{} standard while this module follows \rfc{2822}.)  If the input
string has no timezone, the last element of the tuple returned is
\code{None}.  Note that fields 6, 7, and 8 of the result tuple are not
usable.
\end{funcdesc}

\begin{funcdesc}{mktime_tz}{tuple}
Turn a 10-tuple as returned by \function{parsedate_tz()} into a UTC
timestamp.  It the timezone item in the tuple is \code{None}, assume
local time.  Minor deficiency: this first interprets the first 8
elements as a local time and then compensates for the timezone
difference; this may yield a slight error around daylight savings time
switch dates.  Not enough to worry about for common use.
\end{funcdesc}


\begin{seealso}
  \seemodule{mailbox}{Classes to read various mailbox formats produced 
                      by end-user mail programs.}
  \seemodule{mimetools}{Subclass of rfc.Message that handles MIME encoded
		        messages.} 
\end{seealso}


\subsection{Message Objects \label{message-objects}}

A \class{Message} instance has the following methods:

\begin{methoddesc}{rewindbody}{}
Seek to the start of the message body.  This only works if the file
object is seekable.
\end{methoddesc}

\begin{methoddesc}{isheader}{line}
Returns a line's canonicalized fieldname (the dictionary key that will
be used to index it) if the line is a legal \rfc{2822} header; otherwise
returns None (implying that parsing should stop here and the line be
pushed back on the input stream).  It is sometimes useful to override
this method in a subclass.
\end{methoddesc}

\begin{methoddesc}{islast}{line}
Return true if the given line is a delimiter on which Message should
stop.  The delimiter line is consumed, and the file object's read
location positioned immediately after it.  By default this method just
checks that the line is blank, but you can override it in a subclass.
\end{methoddesc}

\begin{methoddesc}{iscomment}{line}
Return true if the given line should be ignored entirely, just skipped.
By default this is a stub that always returns false, but you can
override it in a subclass.
\end{methoddesc}

\begin{methoddesc}{getallmatchingheaders}{name}
Return a list of lines consisting of all headers matching
\var{name}, if any.  Each physical line, whether it is a continuation
line or not, is a separate list item.  Return the empty list if no
header matches \var{name}.
\end{methoddesc}

\begin{methoddesc}{getfirstmatchingheader}{name}
Return a list of lines comprising the first header matching
\var{name}, and its continuation line(s), if any.  Return
\code{None} if there is no header matching \var{name}.
\end{methoddesc}

\begin{methoddesc}{getrawheader}{name}
Return a single string consisting of the text after the colon in the
first header matching \var{name}.  This includes leading whitespace,
the trailing linefeed, and internal linefeeds and whitespace if there
any continuation line(s) were present.  Return \code{None} if there is
no header matching \var{name}.
\end{methoddesc}

\begin{methoddesc}{getheader}{name\optional{, default}}
Like \code{getrawheader(\var{name})}, but strip leading and trailing
whitespace.  Internal whitespace is not stripped.  The optional
\var{default} argument can be used to specify a different default to
be returned when there is no header matching \var{name}.
\end{methoddesc}

\begin{methoddesc}{get}{name\optional{, default}}
An alias for \method{getheader()}, to make the interface more compatible 
with regular dictionaries.
\end{methoddesc}

\begin{methoddesc}{getaddr}{name}
Return a pair \code{(\var{full name}, \var{email address})} parsed
from the string returned by \code{getheader(\var{name})}.  If no
header matching \var{name} exists, return \code{(None, None)};
otherwise both the full name and the address are (possibly empty)
strings.

Example: If \var{m}'s first \code{From} header contains the string
\code{'jack@cwi.nl (Jack Jansen)'}, then
\code{m.getaddr('From')} will yield the pair
\code{('Jack Jansen', 'jack@cwi.nl')}.
If the header contained
\code{'Jack Jansen <jack@cwi.nl>'} instead, it would yield the
exact same result.
\end{methoddesc}

\begin{methoddesc}{getaddrlist}{name}
This is similar to \code{getaddr(\var{list})}, but parses a header
containing a list of email addresses (e.g.\ a \code{To} header) and
returns a list of \code{(\var{full name}, \var{email address})} pairs
(even if there was only one address in the header).  If there is no
header matching \var{name}, return an empty list.

If multiple headers exist that match the named header (e.g. if there
are several \code{Cc} headers), all are parsed for addresses.  Any
continuation lines the named headers contain are also parsed.
\end{methoddesc}

\begin{methoddesc}{getdate}{name}
Retrieve a header using \method{getheader()} and parse it into a 9-tuple
compatible with \function{time.mktime()}; note that fields 6, 7, and 8 
are not usable.  If there is no header matching
\var{name}, or it is unparsable, return \code{None}.

Date parsing appears to be a black art, and not all mailers adhere to
the standard.  While it has been tested and found correct on a large
collection of email from many sources, it is still possible that this
function may occasionally yield an incorrect result.
\end{methoddesc}

\begin{methoddesc}{getdate_tz}{name}
Retrieve a header using \method{getheader()} and parse it into a
10-tuple; the first 9 elements will make a tuple compatible with
\function{time.mktime()}, and the 10th is a number giving the offset
of the date's timezone from UTC.  Note that fields 6, 7, and 8 
are not usable.  Similarly to \method{getdate()}, if
there is no header matching \var{name}, or it is unparsable, return
\code{None}. 
\end{methoddesc}

\class{Message} instances also support a limited mapping interface.
In particular: \code{\var{m}[name]} is like
\code{\var{m}.getheader(name)} but raises \exception{KeyError} if
there is no matching header; and \code{len(\var{m})},
\code{\var{m}.get(name\optional{, deafult})},
\code{\var{m}.has_key(name)}, \code{\var{m}.keys()},
\code{\var{m}.values()} \code{\var{m}.items()}, and
\code{\var{m}.setdefault(name\optional{, default})} act as expected,
with the one difference that \method{get()} and \method{setdefault()}
use an empty string as the default value.  \class{Message} instances
also support the mapping writable interface \code{\var{m}[name] =
value} and \code{del \var{m}[name]}.  \class{Message} objects do not
support the \method{clear()}, \method{copy()}, \method{popitem()}, or
\method{update()} methods of the mapping interface.  (Support for
\method{get()} and \method{setdefault()} was only added in Python
2.2.)

Finally, \class{Message} instances have two public instance variables:

\begin{memberdesc}{headers}
A list containing the entire set of header lines, in the order in
which they were read (except that setitem calls may disturb this
order). Each line contains a trailing newline.  The
blank line terminating the headers is not contained in the list.
\end{memberdesc}

\begin{memberdesc}{fp}
The file or file-like object passed at instantiation time.  This can
be used to read the message content.
\end{memberdesc}


\subsection{AddressList Objects \label{addresslist-objects}}

An \class{AddressList} instance has the following methods:

\begin{methoddesc}{__len__}{}
Return the number of addresses in the address list.
\end{methoddesc}

\begin{methoddesc}{__str__}{}
Return a canonicalized string representation of the address list.
Addresses are rendered in "name" <host@domain> form, comma-separated.
\end{methoddesc}

\begin{methoddesc}{__add__}{alist}
Return a new \class{AddressList} instance that contains all addresses
in both \class{AddressList} operands, with duplicates removed (set
union).
\end{methoddesc}

\begin{methoddesc}{__iadd__}{alist}
In-place version of \method{__add__()}; turns this \class{AddressList}
instance into the union of itself and the right-hand instance,
\var{alist}.
\end{methoddesc}

\begin{methoddesc}{__sub__}{alist}
Return a new \class{AddressList} instance that contains every address
in the left-hand \class{AddressList} operand that is not present in
the right-hand address operand (set difference).
\end{methoddesc}

\begin{methoddesc}{__isub__}{alist}
In-place version of \method{__sub__()}, removing addresses in this
list which are also in \var{alist}.
\end{methoddesc}


Finally, \class{AddressList} instances have one public instance variable:

\begin{memberdesc}{addresslist}
A list of tuple string pairs, one per address.  In each member, the
first is the canonicalized name part, the second is the
actual route-address (\character{@}-separated username-host.domain
pair).
\end{memberdesc}

\section{\module{mimetools} ---
         Tools for parsing MIME messages}

\declaremodule{standard}{mimetools}
\modulesynopsis{Tools for parsing MIME-style message bodies.}

\deprecated{2.3}{The \refmodule{email} package should be used in
                 preference to the \module{mimetools} module.  This
                 module is present only to maintain backward
                 compatibility.}

This module defines a subclass of the
\refmodule{rfc822}\refstmodindex{rfc822} module's
\class{Message} class and a number of utility functions that are
useful for the manipulation for MIME multipart or encoded message.

It defines the following items:

\begin{classdesc}{Message}{fp\optional{, seekable}}
Return a new instance of the \class{Message} class.  This is a
subclass of the \class{rfc822.Message} class, with some additional
methods (see below).  The \var{seekable} argument has the same meaning
as for \class{rfc822.Message}.
\end{classdesc}

\begin{funcdesc}{choose_boundary}{}
Return a unique string that has a high likelihood of being usable as a
part boundary.  The string has the form
\code{'\var{hostipaddr}.\var{uid}.\var{pid}.\var{timestamp}.\var{random}'}.
\end{funcdesc}

\begin{funcdesc}{decode}{input, output, encoding}
Read data encoded using the allowed MIME \var{encoding} from open file
object \var{input} and write the decoded data to open file object
\var{output}.  Valid values for \var{encoding} include
\code{'base64'}, \code{'quoted-printable'}, \code{'uuencode'},
\code{'x-uuencode'}, \code{'uue'}, \code{'x-uue'}, \code{'7bit'}, and 
\code{'8bit'}.  Decoding messages encoded in \code{'7bit'} or \code{'8bit'}
has no effect.  The input is simply copied to the output.
\end{funcdesc}

\begin{funcdesc}{encode}{input, output, encoding}
Read data from open file object \var{input} and write it encoded using
the allowed MIME \var{encoding} to open file object \var{output}.
Valid values for \var{encoding} are the same as for \method{decode()}.
\end{funcdesc}

\begin{funcdesc}{copyliteral}{input, output}
Read lines from open file \var{input} until \EOF{} and write them to
open file \var{output}.
\end{funcdesc}

\begin{funcdesc}{copybinary}{input, output}
Read blocks until \EOF{} from open file \var{input} and write them to
open file \var{output}.  The block size is currently fixed at 8192.
\end{funcdesc}


\begin{seealso}
  \seemodule{email}{Comprehensive email handling package; supercedes
                    the \module{mimetools} module.}
  \seemodule{rfc822}{Provides the base class for
                     \class{mimetools.Message}.}
  \seemodule{multifile}{Support for reading files which contain
                        distinct parts, such as MIME data.}
  \seeurl{http://www.cs.uu.nl/wais/html/na-dir/mail/mime-faq/.html}{
          The MIME Frequently Asked Questions document.  For an
          overview of MIME, see the answer to question 1.1 in Part 1
          of this document.}
\end{seealso}


\subsection{Additional Methods of Message Objects
            \label{mimetools-message-objects}}

The \class{Message} class defines the following methods in
addition to the \class{rfc822.Message} methods:

\begin{methoddesc}{getplist}{}
Return the parameter list of the \mailheader{Content-Type} header.
This is a list of strings.  For parameters of the form
\samp{\var{key}=\var{value}}, \var{key} is converted to lower case but
\var{value} is not.  For example, if the message contains the header
\samp{Content-type: text/html; spam=1; Spam=2; Spam} then
\method{getplist()} will return the Python list \code{['spam=1',
'spam=2', 'Spam']}.
\end{methoddesc}

\begin{methoddesc}{getparam}{name}
Return the \var{value} of the first parameter (as returned by
\method{getplist()} of the form \samp{\var{name}=\var{value}} for the
given \var{name}.  If \var{value} is surrounded by quotes of the form
`\code{<}...\code{>}' or `\code{"}...\code{"}', these are removed.
\end{methoddesc}

\begin{methoddesc}{getencoding}{}
Return the encoding specified in the
\mailheader{Content-Transfer-Encoding} message header.  If no such
header exists, return \code{'7bit'}.  The encoding is converted to
lower case.
\end{methoddesc}

\begin{methoddesc}{gettype}{}
Return the message type (of the form \samp{\var{type}/\var{subtype}})
as specified in the \mailheader{Content-Type} header.  If no such
header exists, return \code{'text/plain'}.  The type is converted to
lower case.
\end{methoddesc}

\begin{methoddesc}{getmaintype}{}
Return the main type as specified in the \mailheader{Content-Type}
header.  If no such header exists, return \code{'text'}.  The main
type is converted to lower case.
\end{methoddesc}

\begin{methoddesc}{getsubtype}{}
Return the subtype as specified in the \mailheader{Content-Type}
header.  If no such header exists, return \code{'plain'}.  The subtype
is converted to lower case.
\end{methoddesc}

\section{\module{MimeWriter} ---
         Generic MIME file writer}

\declaremodule{standard}{MimeWriter}

\modulesynopsis{Generic MIME file writer.}
\sectionauthor{Christopher G. Petrilli}{petrilli@amber.org}

This module defines the class \class{MimeWriter}.  The
\class{MimeWriter} class implements a basic formatter for creating
MIME multi-part files.  It doesn't seek around the output file nor
does it use large amounts of buffer space. You must write the parts
out in the order that they should occur in the final
file. \class{MimeWriter} does buffer the headers you add, allowing you 
to rearrange their order.

\begin{classdesc}{MimeWriter}{fp}
Return a new instance of the \class{MimeWriter} class.  The only
argument passed, \var{fp}, is a file object to be used for
writing. Note that a \class{StringIO} object could also be used.
\end{classdesc}


\subsection{MimeWriter Objects \label{MimeWriter-objects}}


\class{MimeWriter} instances have the following methods:

\begin{methoddesc}{addheader}{key, value\optional{, prefix}}
Add a header line to the MIME message. The \var{key} is the name of
the header, where the \var{value} obviously provides the value of the
header. The optional argument \var{prefix} determines where the header 
is inserted; \samp{0} means append at the end, \samp{1} is insert at
the start. The default is to append.
\end{methoddesc}

\begin{methoddesc}{flushheaders}{}
Causes all headers accumulated so far to be written out (and
forgotten). This is useful if you don't need a body part at all,
e.g.\ for a subpart of type \mimetype{message/rfc822} that's (mis)used
to store some header-like information.
\end{methoddesc}

\begin{methoddesc}{startbody}{ctype\optional{, plist\optional{, prefix}}}
Returns a file-like object which can be used to write to the
body of the message.  The content-type is set to the provided
\var{ctype}, and the optional parameter \var{plist} provides
additional parameters for the content-type declaration. \var{prefix}
functions as in \method{addheader()} except that the default is to
insert at the start.
\end{methoddesc}

\begin{methoddesc}{startmultipartbody}{subtype\optional{,
                   boundary\optional{, plist\optional{, prefix}}}}
Returns a file-like object which can be used to write to the
body of the message.  Additionally, this method initializes the
multi-part code, where \var{subtype} provides the mutlipart subtype,
\var{boundary} may provide a user-defined boundary specification, and
\var{plist} provides optional parameters for the subtype.
\var{prefix} functions as in \method{startbody()}.  Subparts should be
created using \method{nextpart()}.
\end{methoddesc}

\begin{methoddesc}{nextpart}{}
Returns a new instance of \class{MimeWriter} which represents an
individual part in a multipart message.  This may be used to write the 
part as well as used for creating recursively complex multipart
messages. The message must first be initialized with
\method{startmultipartbody()} before using \method{nextpart()}.
\end{methoddesc}

\begin{methoddesc}{lastpart}{}
This is used to designate the last part of a multipart message, and
should \emph{always} be used when writing multipart messages.
\end{methoddesc}

\section{\module{multifile} ---
         Support for files containing distinct parts}

\declaremodule{standard}{multifile}
\modulesynopsis{Support for reading files which contain distinct
                parts, such as some MIME data.}
\sectionauthor{Eric S. Raymond}{esr@snark.thyrsus.com}


The \class{MultiFile} object enables you to treat sections of a text
file as file-like input objects, with \code{''} being returned by
\method{readline()} when a given delimiter pattern is encountered.  The
defaults of this class are designed to make it useful for parsing
MIME multipart messages, but by subclassing it and overriding methods 
it can be easily adapted for more general use.

\begin{classdesc}{MultiFile}{fp\optional{, seekable}}
Create a multi-file.  You must instantiate this class with an input
object argument for the \class{MultiFile} instance to get lines from,
such as as a file object returned by \function{open()}.

\class{MultiFile} only ever looks at the input object's
\method{readline()}, \method{seek()} and \method{tell()} methods, and
the latter two are only needed if you want random access to the
individual MIME parts. To use \class{MultiFile} on a non-seekable
stream object, set the optional \var{seekable} argument to false; this
will prevent using the input object's \method{seek()} and
\method{tell()} methods.
\end{classdesc}

It will be useful to know that in \class{MultiFile}'s view of the world, text
is composed of three kinds of lines: data, section-dividers, and
end-markers.  MultiFile is designed to support parsing of
messages that may have multiple nested message parts, each with its
own pattern for section-divider and end-marker lines.


\subsection{MultiFile Objects \label{MultiFile-objects}}

A \class{MultiFile} instance has the following methods:

\begin{methoddesc}{push}{str}
Push a boundary string.  When an appropriately decorated version of
this boundary is found as an input line, it will be interpreted as a
section-divider or end-marker.  All subsequent
reads will return the empty string to indicate end-of-file, until a
call to \method{pop()} removes the boundary a or \method{next()} call
reenables it.

It is possible to push more than one boundary.  Encountering the
most-recently-pushed boundary will return EOF; encountering any other
boundary will raise an error.
\end{methoddesc}

\begin{methoddesc}{readline}{str}
Read a line.  If the line is data (not a section-divider or end-marker
or real EOF) return it.  If the line matches the most-recently-stacked
boundary, return \code{''} and set \code{self.last} to 1 or 0 according as
the match is or is not an end-marker.  If the line matches any other
stacked boundary, raise an error.  On encountering end-of-file on the
underlying stream object, the method raises \exception{Error} unless
all boundaries have been popped.
\end{methoddesc}

\begin{methoddesc}{readlines}{str}
Return all lines remaining in this part as a list of strings.
\end{methoddesc}

\begin{methoddesc}{read}{}
Read all lines, up to the next section.  Return them as a single
(multiline) string.  Note that this doesn't take a size argument!
\end{methoddesc}

\begin{methoddesc}{next}{}
Skip lines to the next section (that is, read lines until a
section-divider or end-marker has been consumed).  Return true if
there is such a section, false if an end-marker is seen.  Re-enable
the most-recently-pushed boundary.
\end{methoddesc}

\begin{methoddesc}{pop}{}
Pop a section boundary.  This boundary will no longer be interpreted
as EOF.
\end{methoddesc}

\begin{methoddesc}{seek}{pos\optional{, whence}}
Seek.  Seek indices are relative to the start of the current section.
The \var{pos} and \var{whence} arguments are interpreted as for a file
seek.
\end{methoddesc}

\begin{methoddesc}{tell}{}
Return the file position relative to the start of the current section.
\end{methoddesc}

\begin{methoddesc}{is_data}{str}
Return true if \var{str} is data and false if it might be a section
boundary.  As written, it tests for a prefix other than \code{'-}\code{-'} at
start of line (which all MIME boundaries have) but it is declared so
it can be overridden in derived classes.

Note that this test is used intended as a fast guard for the real
boundary tests; if it always returns false it will merely slow
processing, not cause it to fail.
\end{methoddesc}

\begin{methoddesc}{section_divider}{str}
Turn a boundary into a section-divider line.  By default, this
method prepends \code{'-}\code{-'} (which MIME section boundaries have) but
it is declared so it can be overridden in derived classes.  This
method need not append LF or CR-LF, as comparison with the result
ignores trailing whitespace. 
\end{methoddesc}

\begin{methoddesc}{end_marker}{str}
Turn a boundary string into an end-marker line.  By default, this
method prepends \code{'-}\code{-'} and appends \code{'-}\code{-'} (like a
MIME-multipart end-of-message marker) but it is declared so it can be
be overridden in derived classes.  This method need not append LF or
CR-LF, as comparison with the result ignores trailing whitespace.
\end{methoddesc}

Finally, \class{MultiFile} instances have two public instance variables:

\begin{memberdesc}{level}
Nesting depth of the current part.
\end{memberdesc}

\begin{memberdesc}{last}
True if the last end-of-file was for an end-of-message marker. 
\end{memberdesc}


\subsection{\class{MultiFile} Example \label{multifile-example}}

% This is almost unreadable; should be re-written when someone gets time.

\begin{verbatim}
fp = MultiFile(sys.stdin, 0)
fp.push(outer_boundary)
message1 = fp.readlines()
# We should now be either at real EOF or stopped on a message
# boundary. Re-enable the outer boundary.
fp.next()
# Read another message with the same delimiter
message2 = fp.readlines()
# Re-enable that delimiter again
fp.next()
# Now look for a message subpart with a different boundary
fp.push(inner_boundary)
sub_header = fp.readlines()
# If no exception has been thrown, we're looking at the start of
# the message subpart.  Reset and grab the subpart
fp.next()
sub_body = fp.readlines()
# Got it.  Now pop the inner boundary to re-enable the outer one.
fp.pop()
# Read to next outer boundary
message3 = fp.readlines()
\end{verbatim}

\section{\module{binhex} ---
         Encode and decode binhex4 files}

\declaremodule{standard}{binhex}
\modulesynopsis{Encode and decode files in binhex4 format.}


This module encodes and decodes files in binhex4 format, a format
allowing representation of Macintosh files in \ASCII.  On the Macintosh,
both forks of a file and the finder information are encoded (or
decoded), on other platforms only the data fork is handled.

The \module{binhex} module defines the following functions:

\begin{funcdesc}{binhex}{input, output}
Convert a binary file with filename \var{input} to binhex file
\var{output}. The \var{output} parameter can either be a filename or a
file-like object (any object supporting a \method{write()} and
\method{close()} method).
\end{funcdesc}

\begin{funcdesc}{hexbin}{input\optional{, output}}
Decode a binhex file \var{input}. \var{input} may be a filename or a
file-like object supporting \method{read()} and \method{close()} methods.
The resulting file is written to a file named \var{output}, unless the
argument is omitted in which case the output filename is read from the
binhex file.
\end{funcdesc}

The following exception is also defined:

\begin{excdesc}{Error}
Exception raised when something can't be encoded using the binhex
format (for example, a filename is too long to fit in the filename
field), or when input is not properly encoded binhex data.
\end{excdesc}


\begin{seealso}
  \seemodule{binascii}{Support module containing \ASCII-to-binary
                       and binary-to-\ASCII{} conversions.}
\end{seealso}


\subsection{Notes \label{binhex-notes}}

There is an alternative, more powerful interface to the coder and
decoder, see the source for details.

If you code or decode textfiles on non-Macintosh platforms they will
still use the Macintosh newline convention (carriage-return as end of
line).

As of this writing, \function{hexbin()} appears to not work in all
cases.

\section{Standard Module \module{uu}}
\label{module-uu}
\stmodindex{uu}

This module encodes and decodes files in uuencode format, allowing
arbitrary binary data to be transferred over ascii-only connections.
Wherever a file argument is expected, the methods accept a file-like
object.  For backwards compatibility, a string containing a pathname
is also accepted, and the corresponding file will be opened for
reading and writing; the pathname \code{'-'} is understood to mean the
standard input or output.  However, this interface is deprecated; it's
better for the caller to open the file itself, and be sure that, when
required, the mode is \code{'rb'} or \code{'wb'} on Windows or DOS.

This code was contributed by Lance Ellinghouse, and modified by Jack
Jansen.
\index{Jansen, Jack}
\index{Ellinghouse, Lance}

The \module{uu} module defines the following functions:

\begin{funcdesc}{encode}{in_file, out_file\optional{, name\optional{, mode}}}
Uuencode file \var{in_file} into file \var{out_file}.  The uuencoded
file will have the header specifying \var{name} and \var{mode} as the
defaults for the results of decoding the file. The default defaults
are taken from \var{in_file}, or \code{'-'} and \code{0666}
respectively. 
\end{funcdesc}

\begin{funcdesc}{decode}{in_file\optional{, out_file\optional{, mode}}}
This call decodes uuencoded file \var{in_file} placing the result on
file \var{out_file}. If \var{out_file} is a pathname the \var{mode} is
also set. Defaults for \var{out_file} and \var{mode} are taken from
the uuencode header.
\end{funcdesc}

\section{Standard Module \sectcode{binhex}}
\label{module-binhex}
\stmodindex{binhex}

This module encodes and decodes files in binhex4 format, a format
allowing representation of Macintosh files in ASCII. On the macintosh,
both forks of a file and the finder information are encoded (or
decoded), on other platforms only the data fork is handled.

The \code{binhex} module defines the following functions:

\setindexsubitem{(in module binhex)}

\begin{funcdesc}{binhex}{input\, output}
Convert a binary file with filename \var{input} to binhex file
\var{output}. The \var{output} parameter can either be a filename or a
file-like object (any object supporting a \var{write} and \var{close}
method).
\end{funcdesc}

\begin{funcdesc}{hexbin}{input\optional{\, output}}
Decode a binhex file \var{input}. \var{input} may be a filename or a
file-like object supporting \var{read} and \var{close} methods.
The resulting file is written to a file named \var{output}, unless the
argument is empty in which case the output filename is read from the
binhex file.
\end{funcdesc}

\subsection{Notes}
There is an alternative, more powerful interface to the coder and
decoder, see the source for details.

If you code or decode textfiles on non-Macintosh platforms they will
still use the macintosh newline convention (carriage-return as end of
line).

As of this writing, \var{hexbin} appears to not work in all cases.

\section{Standard Module \sectcode{uu}}
\stmodindex{uu}

This module encodes and decodes files in uuencode format, allowing
arbitrary binary data to be transferred over ascii-only connections.
Wherever a file argument is expected, the methods accept a file-like
object.  For backwards compatibility, a string containing a pathname
is also accepted, and the corresponding file will be opened for
reading and writing; the pathname \code{'-'} is understood to mean the
standard input or output.  However, this interface is deprecated; it's
better for the caller to open the file itself, and be sure that, when
required, the mode is \code{'rb'} or \code{'wb'} on Windows or DOS.

This code was contributed by Lance Ellinghouse, and modified by Jack
Jansen.

The \code{uu} module defines the following functions:

\setindexsubitem{(in module uu)}

\begin{funcdesc}{encode}{in_file\, out_file\optional{\, name\, mode}}
Uuencode file \var{in_file} into file \var{out_file}.  The uuencoded
file will have the header specifying \var{name} and \var{mode} as the
defaults for the results of decoding the file. The default defaults
are taken from \var{in_file}, or \code{'-'} and \code{0666}
respectively. 
\end{funcdesc}

\begin{funcdesc}{decode}{in_file\optional{\, out_file\, mode}}
This call decodes uuencoded file \var{in_file} placing the result on
file \var{out_file}. If \var{out_file} is a pathname the \var{mode} is
also set. Defaults for \var{out_file} and \var{mode} are taken from
the uuencode header.
\end{funcdesc}

\section{Built-in Module \sectcode{binascii}}	% If implemented in C
\bimodindex{binascii}

The binascii module contains a number of methods to convert between
binary and various ascii-encoded binary representations. Normally, you
will not use these modules directly but use wrapper modules like
\var{uu} or \var{hexbin} in stead, this module solely exists because
bit-manipuation of large amounts of data is slow in python.

The \code{binascii} module defines the following functions:

\setindexsubitem{(in module binascii)}

\begin{funcdesc}{a2b_uu}{string}
Convert a single line of uuencoded data back to binary and return the
binary data. Lines normally contain 45 (binary) bytes, except for the
last line. Line data may be followed by whitespace.
\end{funcdesc}

\begin{funcdesc}{b2a_uu}{data}
Convert binary data to a line of ascii characters, the return value is
the converted line, including a newline char. The length of \var{data}
should be at most 45.
\end{funcdesc}

\begin{funcdesc}{a2b_base64}{string}
Convert a block of base64 data back to binary and return the
binary data. More than one line may be passed at a time.
\end{funcdesc}

\begin{funcdesc}{b2a_base64}{data}
Convert binary data to a line of ascii characters in base64 coding.
The return value is the converted line, including a newline char.
The length of \var{data} should be at most 57 to adhere to the base64
standard.
\end{funcdesc}

\begin{funcdesc}{a2b_hqx}{string}
Convert binhex4 formatted ascii data to binary, without doing
rle-decompression. The string should contain a complete number of
binary bytes, or (in case of the last portion of the binhex4 data)
have the remaining bits zero.
\end{funcdesc}

\begin{funcdesc}{rledecode_hqx}{data}
Perform RLE-decompression on the data, as per the binhex4
standard. The algorithm uses \code{0x90} after a byte as a repeat
indicator, followed by a count. A count of \code{0} specifies a byte
value of \code{0x90}. The routine returns the decompressed data,
unless data input data ends in an orphaned repeat indicator, in which
case the \var{Incomplete} exception is raised.
\end{funcdesc}

\begin{funcdesc}{rlecode_hqx}{data}
Perform binhex4 style RLE-compression on \var{data} and return the
result.
\end{funcdesc}

\begin{funcdesc}{b2a_hqx}{data}
Perform hexbin4 binary-to-ascii translation and return the resulting
string. The argument should already be rle-coded, and have a length
divisible by 3 (except possibly the last fragment).
\end{funcdesc}

\begin{funcdesc}{crc_hqx}{data, crc}
Compute the binhex4 crc value of \var{data}, starting with an initial
\var{crc} and returning the result.
\end{funcdesc}
 
\begin{excdesc}{Error}
Exception raised on errors. These are usually programming errors.
\end{excdesc}

\begin{excdesc}{Incomplete}
Exception raised on incomplete data. These are usually not programming
errors, but handled by reading a little more data and trying again.
\end{excdesc}

\section{Standard Module \module{xdrlib}}
\declaremodule{standard}{xdrlib}

\modulesynopsis{The External Data Representation Standard as described
in \rfc{1014}.}

\index{XDR}
\index{External Data Representation}


The \module{xdrlib} module supports the External Data Representation
Standard as described in \rfc{1014}, written by Sun Microsystems,
Inc. June 1987.  It supports most of the data types described in the
RFC.

The \module{xdrlib} module defines two classes, one for packing
variables into XDR representation, and another for unpacking from XDR
representation.  There are also two exception classes.

\begin{classdesc}{Packer}{}
\class{Packer} is the class for packing data into XDR representation.
The \class{Packer} class is instantiated with no arguments.
\end{classdesc}

\begin{classdesc}{Unpacker}{data}
\code{Unpacker} is the complementary class which unpacks XDR data
values from a string buffer.  The input buffer is given as
\var{data}.
\end{classdesc}


\subsection{Packer Objects}
\label{xdr-packer-objects}

\class{Packer} instances have the following methods:

\begin{methoddesc}[Packer]{get_buffer}{}
Returns the current pack buffer as a string.
\end{methoddesc}

\begin{methoddesc}[Packer]{reset}{}
Resets the pack buffer to the empty string.
\end{methoddesc}

In general, you can pack any of the most common XDR data types by
calling the appropriate \code{pack_\var{type}()} method.  Each method
takes a single argument, the value to pack.  The following simple data
type packing methods are supported: \method{pack_uint()},
\method{pack_int()}, \method{pack_enum()}, \method{pack_bool()},
\method{pack_uhyper()}, and \method{pack_hyper()}.

\begin{methoddesc}[Packer]{pack_float}{value}
Packs the single-precision floating point number \var{value}.
\end{methoddesc}

\begin{methoddesc}[Packer]{pack_double}{value}
Packs the double-precision floating point number \var{value}.
\end{methoddesc}

The following methods support packing strings, bytes, and opaque data:

\begin{methoddesc}[Packer]{pack_fstring}{n, s}
Packs a fixed length string, \var{s}.  \var{n} is the length of the
string but it is \emph{not} packed into the data buffer.  The string
is padded with null bytes if necessary to guaranteed 4 byte alignment.
\end{methoddesc}

\begin{methoddesc}[Packer]{pack_fopaque}{n, data}
Packs a fixed length opaque data stream, similarly to
\method{pack_fstring()}.
\end{methoddesc}

\begin{methoddesc}[Packer]{pack_string}{s}
Packs a variable length string, \var{s}.  The length of the string is
first packed as an unsigned integer, then the string data is packed
with \method{pack_fstring()}.
\end{methoddesc}

\begin{methoddesc}[Packer]{pack_opaque}{data}
Packs a variable length opaque data string, similarly to
\method{pack_string()}.
\end{methoddesc}

\begin{methoddesc}[Packer]{pack_bytes}{bytes}
Packs a variable length byte stream, similarly to \method{pack_string()}.
\end{methoddesc}

The following methods support packing arrays and lists:

\begin{methoddesc}[Packer]{pack_list}{list, pack_item}
Packs a \var{list} of homogeneous items.  This method is useful for
lists with an indeterminate size; i.e. the size is not available until
the entire list has been walked.  For each item in the list, an
unsigned integer \code{1} is packed first, followed by the data value
from the list.  \var{pack_item} is the function that is called to pack
the individual item.  At the end of the list, an unsigned integer
\code{0} is packed.
\end{methoddesc}

\begin{methoddesc}[Packer]{pack_farray}{n, array, pack_item}
Packs a fixed length list (\var{array}) of homogeneous items.  \var{n}
is the length of the list; it is \emph{not} packed into the buffer,
but a \exception{ValueError} exception is raised if
\code{len(\var{array})} is not equal to \var{n}.  As above,
\var{pack_item} is the function used to pack each element.
\end{methoddesc}

\begin{methoddesc}[Packer]{pack_array}{list, pack_item}
Packs a variable length \var{list} of homogeneous items.  First, the
length of the list is packed as an unsigned integer, then each element
is packed as in \method{pack_farray()} above.
\end{methoddesc}


\subsection{Unpacker Objects}
\label{xdr-unpacker-objects}

The \class{Unpacker} class offers the following methods:

\begin{methoddesc}[Unpacker]{reset}{data}
Resets the string buffer with the given \var{data}.
\end{methoddesc}

\begin{methoddesc}[Unpacker]{get_position}{}
Returns the current unpack position in the data buffer.
\end{methoddesc}

\begin{methoddesc}[Unpacker]{set_position}{position}
Sets the data buffer unpack position to \var{position}.  You should be
careful about using \method{get_position()} and \method{set_position()}.
\end{methoddesc}

\begin{methoddesc}[Unpacker]{get_buffer}{}
Returns the current unpack data buffer as a string.
\end{methoddesc}

\begin{methoddesc}[Unpacker]{done}{}
Indicates unpack completion.  Raises an \exception{Error} exception
if all of the data has not been unpacked.
\end{methoddesc}

In addition, every data type that can be packed with a \class{Packer},
can be unpacked with an \class{Unpacker}.  Unpacking methods are of the
form \code{unpack_\var{type}()}, and take no arguments.  They return the
unpacked object.

\begin{methoddesc}[Unpacker]{unpack_float}{}
Unpacks a single-precision floating point number.
\end{methoddesc}

\begin{methoddesc}[Unpacker]{unpack_double}{}
Unpacks a double-precision floating point number, similarly to
\method{unpack_float()}.
\end{methoddesc}

In addition, the following methods unpack strings, bytes, and opaque
data:

\begin{methoddesc}[Unpacker]{unpack_fstring}{n}
Unpacks and returns a fixed length string.  \var{n} is the number of
characters expected.  Padding with null bytes to guaranteed 4 byte
alignment is assumed.
\end{methoddesc}

\begin{methoddesc}[Unpacker]{unpack_fopaque}{n}
Unpacks and returns a fixed length opaque data stream, similarly to
\method{unpack_fstring()}.
\end{methoddesc}

\begin{methoddesc}[Unpacker]{unpack_string}{}
Unpacks and returns a variable length string.  The length of the
string is first unpacked as an unsigned integer, then the string data
is unpacked with \method{unpack_fstring()}.
\end{methoddesc}

\begin{methoddesc}[Unpacker]{unpack_opaque}{}
Unpacks and returns a variable length opaque data string, similarly to
\method{unpack_string()}.
\end{methoddesc}

\begin{methoddesc}[Unpacker]{unpack_bytes}{}
Unpacks and returns a variable length byte stream, similarly to
\method{unpack_string()}.
\end{methoddesc}

The following methods support unpacking arrays and lists:

\begin{methoddesc}[Unpacker]{unpack_list}{unpack_item}
Unpacks and returns a list of homogeneous items.  The list is unpacked
one element at a time
by first unpacking an unsigned integer flag.  If the flag is \code{1},
then the item is unpacked and appended to the list.  A flag of
\code{0} indicates the end of the list.  \var{unpack_item} is the
function that is called to unpack the items.
\end{methoddesc}

\begin{methoddesc}[Unpacker]{unpack_farray}{n, unpack_item}
Unpacks and returns (as a list) a fixed length array of homogeneous
items.  \var{n} is number of list elements to expect in the buffer.
As above, \var{unpack_item} is the function used to unpack each element.
\end{methoddesc}

\begin{methoddesc}[Unpacker]{unpack_array}{unpack_item}
Unpacks and returns a variable length \var{list} of homogeneous items.
First, the length of the list is unpacked as an unsigned integer, then
each element is unpacked as in \method{unpack_farray()} above.
\end{methoddesc}


\subsection{Exceptions}
\nodename{xdr-exceptions}

Exceptions in this module are coded as class instances:

\begin{excdesc}{Error}
The base exception class.  \exception{Error} has a single public data
member \member{msg} containing the description of the error.
\end{excdesc}

\begin{excdesc}{ConversionError}
Class derived from \exception{Error}.  Contains no additional instance
variables.
\end{excdesc}

Here is an example of how you would catch one of these exceptions:

\begin{verbatim}
import xdrlib
p = xdrlib.Packer()
try:
    p.pack_double(8.01)
except xdrlib.ConversionError, instance:
    print 'packing the double failed:', instance.msg
\end{verbatim}

\section{Standard Module \sectcode{mailcap}}
\label{module-mailcap}
\stmodindex{mailcap}
\renewcommand{\indexsubitem}{(in module mailcap)}

Mailcap files are used to configure how MIME-aware applications such
as mail readers and Web browsers react to files with different MIME
types. (The name ``mailcap'' is derived from the phrase ``mail
capability''.)  For example, a mailcap file might contain a line like
\samp{video/mpeg; xmpeg \%s}.  Then, if the user encounters an email
message or Web document with the MIME type video/mpeg, \code{\%s} will be
replaced by a filename (usually one belonging to a temporary file) and
the xmpeg program can be automatically started to view the file.

The mailcap format is documented in RFC 1524, ``A User Agent
Configuration Mechanism For Multimedia Mail Format Information'', but
is not an Internet standard.  However, mailcap files are supported on
most Unix systems.

\begin{funcdesc}{findmatch}{caps\, MIMEtype\, key\, filename\, plist}
Return a 2-tuple; the first element is a string containing the command
line to be executed
(which can be passed to \code{os.system()}), and the second element is
the mailcap entry for a given MIME type.  If no matching MIME
type can be found, \code{(None, None)} is returned.

\var{key} is the name of the field desired, which represents the type of
activity to be performed; the default value is 'view', since in the
most common case you simply want to view the body of the MIME-typed
data.  Other possible values might be 'compose' and 'edit', if you
wanted to create a new body of the given MIME type or alter the
existing body data.  See RFC1524 for a complete list of these fields.

\var{filename} is the filename to be substituted for \%s in the
command line; the default value is
\file{/dev/null} which is almost certainly not what you want, so
usually you'll override it by specifying a filename.

\var{plist} can be a list containing named parameters; the default
value is simply an empty list.  Each entry in the list must be a
string containing the parameter name, an equals sign (\code{=}), and the
parameter's value.  Mailcap entries can contain 
named parameters like \code{\%\{foo\}}, which will be replaced by the
value of the parameter named 'foo'.  For example, if the command line
\samp{showpartial \%\{id\} \%\{number\} \%\{total\}}
was in a mailcap file, and \var{plist} was set to \code{['id=1',
'number=2', 'total=3']}, the resulting command line would be 
\code{"showpartial 1 2 3"}.  

In a mailcap file, the "test" field can optionally be specified to
test some external condition (e.g., the machine architecture, or the
window system in use) to determine whether or not the mailcap line
applies.  \code{findmatch()} will automatically check such conditions
and skip the entry if the check fails.
\end{funcdesc}

\begin{funcdesc}{getcaps}{}
Returns a dictionary mapping MIME types to a list of mailcap file
entries. This dictionary must be passed to the \code{findmatch()}
function.  An entry is stored as a list of dictionaries, but it
shouldn't be necessary to know the details of this representation.

The information is derived from all of the mailcap files found on the
system. Settings in the user's mailcap file \file{\$HOME/.mailcap}
will override settings in the system mailcap files
\file{/etc/mailcap}, \file{/usr/etc/mailcap}, and
\file{/usr/local/etc/mailcap}.
\end{funcdesc}

An example usage:
\bcode\begin{verbatim}
>>> import mailcap
>>> d=mailcap.getcaps()
>>> mailcap.findmatch(d, 'video/mpeg', filename='/tmp/tmp1223')
('xmpeg /tmp/tmp1223', {'view': 'xmpeg %s'})
\end{verbatim}\ecode

\section{\module{mimetypes} ---
         Map filenames to MIME types}

\declaremodule{standard}{mimetypes}
\modulesynopsis{Mapping of filename extensions to MIME types.}
\sectionauthor{Fred L. Drake, Jr.}{fdrake@acm.org}


\indexii{MIME}{content type}

The \module{mimetypes} module converts between a filename or URL and
the MIME type associated with the filename extension.  Conversions are
provided from filename to MIME type and from MIME type to filename
extension; encodings are not supported for the latter conversion.

The module provides one class and a number of convenience functions.
The functions are the normal interface to this module, but some
applications may be interested in the class as well.

The functions described below provide the primary interface for this
module.  If the module has not been initialized, they will call
\function{init()} if they rely on the information \function{init()}
sets up.


\begin{funcdesc}{guess_type}{filename\optional{, strict}}
Guess the type of a file based on its filename or URL, given by
\var{filename}.  The return value is a tuple \code{(\var{type},
\var{encoding})} where \var{type} is \code{None} if the type can't be
guessed (missing or unknown suffix) or a string of the form
\code{'\var{type}/\var{subtype}'}, usable for a MIME
\mailheader{content-type} header\indexii{MIME}{headers}.

\var{encoding} is \code{None} for no encoding or the name of the
program used to encode (e.g. \program{compress} or \program{gzip}).
The encoding is suitable for use as a \mailheader{Content-Encoding}
header, \emph{not} as a \mailheader{Content-Transfer-Encoding} header.
The mappings are table driven.  Encoding suffixes are case sensitive;
type suffixes are first tried case sensitively, then case
insensitively.

Optional \var{strict} is a flag specifying whether the list of known
MIME types is limited to only the official types \ulink{registered
with IANA}{http://www.isi.edu/in-notes/iana/assignments/media-types}
are recognized.  When \var{strict} is true (the default), only the
IANA types are supported; when \var{strict} is false, some additional
non-standard but commonly used MIME types are also recognized.
\end{funcdesc}

\begin{funcdesc}{guess_all_extensions}{type\optional{, strict}}
Guess the extensions for a file based on its MIME type, given by
\var{type}.
The return value is a list of strings giving all possible filename extensions,
including the leading dot (\character{.}).  The extensions are not guaranteed
to have been associated with any particular data stream, but would be mapped
to the MIME type \var{type} by \function{guess_type()}.

Optional \var{strict} has the same meaning as with the
\function{guess_type()} function.
\end{funcdesc}


\begin{funcdesc}{guess_extension}{type\optional{, strict}}
Guess the extension for a file based on its MIME type, given by
\var{type}.
The return value is a string giving a filename extension, including the
leading dot (\character{.}).  The extension is not guaranteed to have been
associated with any particular data stream, but would be mapped to the 
MIME type \var{type} by \function{guess_type()}.  If no extension can
be guessed for \var{type}, \code{None} is returned.

Optional \var{strict} has the same meaning as with the
\function{guess_type()} function.
\end{funcdesc}


Some additional functions and data items are available for controlling
the behavior of the module.


\begin{funcdesc}{init}{\optional{files}}
Initialize the internal data structures.  If given, \var{files} must
be a sequence of file names which should be used to augment the
default type map.  If omitted, the file names to use are taken from
\constant{knownfiles}.  Each file named in \var{files} or
\constant{knownfiles} takes precedence over those named before it.
Calling \function{init()} repeatedly is allowed.
\end{funcdesc}

\begin{funcdesc}{read_mime_types}{filename}
Load the type map given in the file \var{filename}, if it exists.  The 
type map is returned as a dictionary mapping filename extensions,
including the leading dot (\character{.}), to strings of the form
\code{'\var{type}/\var{subtype}'}.  If the file \var{filename} does
not exist or cannot be read, \code{None} is returned.
\end{funcdesc}


\begin{funcdesc}{add_type}{type, ext\optional{, strict}}
Add a mapping from the mimetype \var{type} to the extension \var{ext}.
When the extension is already known, the new type will replace the old
one. When the type is already known the extension will be added
to the list of known extensions.

When \var{strict} is the mapping will added to the official
MIME types, otherwise to the non-standard ones.
\end{funcdesc}


\begin{datadesc}{inited}
Flag indicating whether or not the global data structures have been
initialized.  This is set to true by \function{init()}.
\end{datadesc}

\begin{datadesc}{knownfiles}
List of type map file names commonly installed.  These files are
typically named \file{mime.types} and are installed in different
locations by different packages.\index{file!mime.types}
\end{datadesc}

\begin{datadesc}{suffix_map}
Dictionary mapping suffixes to suffixes.  This is used to allow
recognition of encoded files for which the encoding and the type are
indicated by the same extension.  For example, the \file{.tgz}
extension is mapped to \file{.tar.gz} to allow the encoding and type
to be recognized separately.
\end{datadesc}

\begin{datadesc}{encodings_map}
Dictionary mapping filename extensions to encoding types.
\end{datadesc}

\begin{datadesc}{types_map}
Dictionary mapping filename extensions to MIME types.
\end{datadesc}

\begin{datadesc}{common_types}
Dictionary mapping filename extensions to non-standard, but commonly
found MIME types.
\end{datadesc}


The \class{MimeTypes} class may be useful for applications which may
want more than one MIME-type database:

\begin{classdesc}{MimeTypes}{\optional{filenames}}
  This class represents a MIME-types database.  By default, it
  provides access to the same database as the rest of this module.
  The initial database is a copy of that provided by the module, and
  may be extended by loading additional \file{mime.types}-style files
  into the database using the \method{read()} or \method{readfp()}
  methods.  The mapping dictionaries may also be cleared before
  loading additional data if the default data is not desired.

  The optional \var{filenames} parameter can be used to cause
  additional files to be loaded ``on top'' of the default database.

  \versionadded{2.2}
\end{classdesc}

An example usage of the module:

\begin{verbatim}
>>> import mimetypes
>>> mimetypes.init()
>>> mimetypes.knownfiles
['/etc/mime.types', '/etc/httpd/mime.types', ... ]
>>> mimetypes.suffix_map['.tgz']
'.tar.gz'
>>> mimetypes.encodings_map['.gz']
'gzip'
>>> mimetypes.types_map['.tgz']
'application/x-tar-gz'
\end{verbatim}

\subsection{MimeTypes Objects \label{mimetypes-objects}}

\class{MimeTypes} instances provide an interface which is very like
that of the \refmodule{mimetypes} module.

\begin{memberdesc}[MimeTypes]{suffix_map}
  Dictionary mapping suffixes to suffixes.  This is used to allow
  recognition of encoded files for which the encoding and the type are
  indicated by the same extension.  For example, the \file{.tgz}
  extension is mapped to \file{.tar.gz} to allow the encoding and type
  to be recognized separately.  This is initially a copy of the global
  \code{suffix_map} defined in the module.
\end{memberdesc}

\begin{memberdesc}[MimeTypes]{encodings_map}
  Dictionary mapping filename extensions to encoding types.  This is
  initially a copy of the global \code{encodings_map} defined in the
  module.
\end{memberdesc}

\begin{memberdesc}[MimeTypes]{types_map}
  Dictionary mapping filename extensions to MIME types.  This is
  initially a copy of the global \code{types_map} defined in the
  module.
\end{memberdesc}

\begin{memberdesc}[MimeTypes]{common_types}
  Dictionary mapping filename extensions to non-standard, but commonly
  found MIME types.  This is initially a copy of the global
  \code{common_types} defined in the module.
\end{memberdesc}

\begin{methoddesc}[MimeTypes]{guess_extension}{type\optional{, strict}}
  Similar to the \function{guess_extension()} function, using the
  tables stored as part of the object.
\end{methoddesc}

\begin{methoddesc}[MimeTypes]{guess_type}{url\optional{, strict}}
  Similar to the \function{guess_type()} function, using the tables
  stored as part of the object.
\end{methoddesc}

\begin{methoddesc}[MimeTypes]{read}{path}
  Load MIME information from a file named \var{path}.  This uses
  \method{readfp()} to parse the file.
\end{methoddesc}

\begin{methoddesc}[MimeTypes]{readfp}{file}
  Load MIME type information from an open file.  The file must have
  the format of the standard \file{mime.types} files.
\end{methoddesc}

\section{\module{base64} ---
         Encode and decode MIME base64 data}

\declaremodule{standard}{base64}
\modulesynopsis{Encode and decode files using the MIME base64 data.}


\indexii{base64}{encoding}
\index{MIME!base64 encoding}

This module performs base64 encoding and decoding of arbitrary binary
strings into text strings that can be safely emailed or posted.  The
encoding scheme is defined in \rfc{1421} (``Privacy Enhancement for
Internet Electronic Mail: Part I: Message Encryption and
Authentication Procedures'', section 4.3.2.4, ``Step 4: Printable
Encoding'') and is used for MIME email and
various other Internet-related applications; it is not the same as the
output produced by the \program{uuencode} program.  For example, the
string \code{'www.python.org'} is encoded as the string
\code{'d3d3LnB5dGhvbi5vcmc=\e n'}.  


\begin{funcdesc}{decode}{input, output}
Decode the contents of the \var{input} file and write the resulting
binary data to the \var{output} file.
\var{input} and \var{output} must either be file objects or objects that
mimic the file object interface. \var{input} will be read until
\code{\var{input}.read()} returns an empty string.
\end{funcdesc}

\begin{funcdesc}{decodestring}{s}
Decode the string \var{s}, which must contain one or more lines of
base64 encoded data, and return a string containing the resulting
binary data.
\end{funcdesc}

\begin{funcdesc}{encode}{input, output}
Encode the contents of the \var{input} file and write the resulting
base64 encoded data to the \var{output} file.
\var{input} and \var{output} must either be file objects or objects that
mimic the file object interface. \var{input} will be read until
\code{\var{input}.read()} returns an empty string.
\end{funcdesc}

\begin{funcdesc}{encodestring}{s}
Encode the string \var{s}, which can contain arbitrary binary data,
and return a string containing one or more lines of
base64 encoded data.
\end{funcdesc}


\begin{seealso}
  \seemodule{binascii}{support module containing \ASCII{}-to-binary
                       and binary-to-\ASCII{} conversions}
\end{seealso}

\section{Standard Module \sectcode{quopri}}
\label{module-quopri}
\stmodindex{quopri}

This module performs quoted-printable transport encoding and decoding,
as defined in \rfc{1521}: ``MIME (Multipurpose Internet Mail Extensions)
Part One''.  The quoted-printable encoding is designed for data where
there are relatively few nonprintable characters; the base-64 encoding
scheme available via the \code{base64} module is more compact if there
are many such characters, as when sending a graphics file.
\indexii{quoted printable}{encoding}
\index{MIME!quoted-printable encoding}

\setindexsubitem{(in module quopri)}

\begin{funcdesc}{decode}{input, output}
Decode the contents of the \var{input} file and write the resulting
decoded binary data to the \var{output} file.
\var{input} and \var{output} must either be file objects or objects that
mimic the file object interface. \var{input} will be read until
\code{\var{input}.read()} returns an empty string.
\end{funcdesc}

\begin{funcdesc}{encode}{input, output, quotetabs}
Encode the contents of the \var{input} file and write the resulting
quoted-printable data to the \var{output} file.
\var{input} and \var{output} must either be file objects or objects that
mimic the file object interface. \var{input} will be read until
\code{\var{input}.read()} returns an empty string.
\end{funcdesc}




\section{\module{mailbox} ---
          Manipulate mailboxes in various formats}

\declaremodule{}{mailbox}
\moduleauthor{Gregory K.~Johnson}{gkj@gregorykjohnson.com}
\sectionauthor{Gregory K.~Johnson}{gkj@gregorykjohnson.com}
\modulesynopsis{Manipulate mailboxes in various formats}


This module defines two classes, \class{Mailbox} and \class{Message}, for
accessing and manipulating on-disk mailboxes and the messages they contain.
\class{Mailbox} offers a dictionary-like mapping from keys to messages.
\class{Message} extends the \module{email.Message} module's \class{Message}
class with format-specific state and behavior. Supported mailbox formats are
Maildir, mbox, MH, Babyl, and MMDF.

\begin{seealso}
    \seemodule{email}{Represent and manipulate messages.}
\end{seealso}

\subsection{\class{Mailbox} objects}
\label{mailbox-objects}

\begin{classdesc*}{Mailbox}
A mailbox, which may be inspected and modified.
\end{classdesc*}

The \class{Mailbox} interface is dictionary-like, with small keys
corresponding to messages. Keys are issued by the \class{Mailbox} instance
with which they will be used and are only meaningful to that \class{Mailbox}
instance. A key continues to identify a message even if the corresponding
message is modified, such as by replacing it with another message. Messages may
be added to a \class{Mailbox} instance using the set-like method
\method{add()} and removed using a \code{del} statement or the set-like methods
\method{remove()} and \method{discard()}.

\class{Mailbox} interface semantics differ from dictionary semantics in some
noteworthy ways. Each time a message is requested, a new representation
(typically a \class{Message} instance) is generated, based upon the current
state of the mailbox. Similarly, when a message is added to a \class{Mailbox}
instance, the provided message representation's contents are copied. In neither
case is a reference to the message representation kept by the \class{Mailbox}
instance.

The default \class{Mailbox} iterator iterates over message representations, not
keys as the default dictionary iterator does. Moreover, modification of a
mailbox during iteration is safe and well-defined. Messages added to the
mailbox after an iterator is created will not be seen by the iterator. Messages
removed from the mailbox before the iterator yields them will be silently
skipped, though using a key from an iterator may result in a
\exception{KeyError} exception if the corresponding message is subsequently
removed.

\class{Mailbox} itself is intended to define an interface and to be inherited
from by format-specific subclasses but is not intended to be instantiated.
Instead, you should instantiate a subclass.

\class{Mailbox} instances have the following methods:

\begin{methoddesc}{add}{message}
Add \var{message} to the mailbox and return the key that has been assigned to
it.

Parameter \var{message} may be a \class{Message} instance, an
\class{email.Message.Message} instance, a string, or a file-like object (which
should be open in text mode). If \var{message} is an instance of the
appropriate format-specific \class{Message} subclass (e.g., if it's an
\class{mboxMessage} instance and this is an \class{mbox} instance), its
format-specific information is used. Otherwise, reasonable defaults for
format-specific information are used.
\end{methoddesc}

\begin{methoddesc}{remove}{key}
\methodline{__delitem__}{key}
\methodline{discard}{key}
Delete the message corresponding to \var{key} from the mailbox.

If no such message exists, a \exception{KeyError} exception is raised if the
method was called as \method{remove()} or \method{__delitem__()} but no
exception is raised if the method was called as \method{discard()}. The
behavior of \method{discard()} may be preferred if the underlying mailbox
format supports concurrent modification by other processes.
\end{methoddesc}

\begin{methoddesc}{__setitem__}{key, message}
Replace the message corresponding to \var{key} with \var{message}. Raise a
\exception{KeyError} exception if no message already corresponds to \var{key}.

As with \method{add()}, parameter \var{message} may be a \class{Message}
instance, an \class{email.Message.Message} instance, a string, or a file-like
object (which should be open in text mode). If \var{message} is an instance of
the appropriate format-specific \class{Message} subclass (e.g., if it's an
\class{mboxMessage} instance and this is an \class{mbox} instance), its
format-specific information is used. Otherwise, the format-specific information
of the message that currently corresponds to \var{key} is left unchanged. 
\end{methoddesc}

\begin{methoddesc}{iterkeys}{}
\methodline{keys}{}
Return an iterator over all keys if called as \method{iterkeys()} or return a
list of keys if called as \method{keys()}.
\end{methoddesc}

\begin{methoddesc}{itervalues}{}
\methodline{__iter__}{}
\methodline{values}{}
Return an iterator over representations of all messages if called as
\method{itervalues()} or \method{__iter__()} or return a list of such
representations if called as \method{values()}. The messages are represented as
instances of the appropriate format-specific \class{Message} subclass unless a
custom message factory was specified when the \class{Mailbox} instance was
initialized. \note{The behavior of \method{__iter__()} is unlike that of
dictionaries, which iterate over keys.}
\end{methoddesc}

\begin{methoddesc}{iteritems}{}
\methodline{items}{}
Return an iterator over (\var{key}, \var{message}) pairs, where \var{key} is a
key and \var{message} is a message representation, if called as
\method{iteritems()} or return a list of such pairs if called as
\method{items()}. The messages are represented as instances of the appropriate
format-specific \class{Message} subclass unless a custom message factory was
specified when the \class{Mailbox} instance was initialized.
\end{methoddesc}

\begin{methoddesc}{get}{key\optional{, default=None}}
\methodline{__getitem__}{key}
Return a representation of the message corresponding to \var{key}. If no such
message exists, \var{default} is returned if the method was called as
\method{get()} and a \exception{KeyError} exception is raised if the method was
called as \method{__getitem__()}. The message is represented as an instance of
the appropriate format-specific \class{Message} subclass unless a custom
message factory was specified when the \class{Mailbox} instance was
initialized.
\end{methoddesc}

\begin{methoddesc}{get_message}{key}
Return a representation of the message corresponding to \var{key} as an
instance of the appropriate format-specific \class{Message} subclass, or raise
a \exception{KeyError} exception if no such message exists.
\end{methoddesc}

\begin{methoddesc}{get_string}{key}
Return a string representation of the message corresponding to \var{key}, or
raise a \exception{KeyError} exception if no such message exists.
\end{methoddesc}

\begin{methoddesc}{get_file}{key}
Return a file-like representation of the message corresponding to \var{key},
or raise a \exception{KeyError} exception if no such message exists. The
file-like object behaves as if open in binary mode. This file should be closed
once it is no longer needed.

\note{Unlike other representations of messages, file-like representations are
not necessarily independent of the \class{Mailbox} instance that created them
or of the underlying mailbox. More specific documentation is provided by each
subclass.}
\end{methoddesc}

\begin{methoddesc}{has_key}{key}
\methodline{__contains__}{key}
Return \code{True} if \var{key} corresponds to a message, \code{False}
otherwise.
\end{methoddesc}

\begin{methoddesc}{__len__}{}
Return a count of messages in the mailbox.
\end{methoddesc}

\begin{methoddesc}{clear}{}
Delete all messages from the mailbox.
\end{methoddesc}

\begin{methoddesc}{pop}{key\optional{, default}}
Return a representation of the message corresponding to \var{key} and delete
the message. If no such message exists, return \var{default} if it was supplied
or else raise a \exception{KeyError} exception. The message is represented as
an instance of the appropriate format-specific \class{Message} subclass unless
a custom message factory was specified when the \class{Mailbox} instance was
initialized.
\end{methoddesc}

\begin{methoddesc}{popitem}{}
Return an arbitrary (\var{key}, \var{message}) pair, where \var{key} is a key
and \var{message} is a message representation, and delete the corresponding
message. If the mailbox is empty, raise a \exception{KeyError} exception. The
message is represented as an instance of the appropriate format-specific
\class{Message} subclass unless a custom message factory was specified when the
\class{Mailbox} instance was initialized.
\end{methoddesc}

\begin{methoddesc}{update}{arg}
Parameter \var{arg} should be a \var{key}-to-\var{message} mapping or an
iterable of (\var{key}, \var{message}) pairs. Updates the mailbox so that, for
each given \var{key} and \var{message}, the message corresponding to \var{key}
is set to \var{message} as if by using \method{__setitem__()}. As with
\method{__setitem__()}, each \var{key} must already correspond to a message in
the mailbox or else a \exception{KeyError} exception will be raised, so in
general it is incorrect for \var{arg} to be a \class{Mailbox} instance.
\note{Unlike with dictionaries, keyword arguments are not supported.}
\end{methoddesc}

\begin{methoddesc}{flush}{}
Write any pending changes to the filesystem. For some \class{Mailbox}
subclasses, changes are always written immediately and this method does
nothing.
\end{methoddesc}

\begin{methoddesc}{lock}{}
Acquire an exclusive advisory lock on the mailbox so that other processes know
not to modify it. An \exception{ExternalClashError} is raised if the lock is
not available. The particular locking mechanisms used depend upon the mailbox
format.
\end{methoddesc}

\begin{methoddesc}{unlock}{}
Release the lock on the mailbox, if any.
\end{methoddesc}

\begin{methoddesc}{close}{}
Flush the mailbox, unlock it if necessary, and close any open files. For some
\class{Mailbox} subclasses, this method does nothing.
\end{methoddesc}


\subsubsection{\class{Maildir}}
\label{mailbox-maildir}

\begin{classdesc}{Maildir}{dirname\optional{, factory=rfc822.Message\optional{,
create=True}}}
A subclass of \class{Mailbox} for mailboxes in Maildir format. Parameter
\var{factory} is a callable object that accepts a file-like message
representation (which behaves as if opened in binary mode) and returns a custom
representation. If \var{factory} is \code{None}, \class{MaildirMessage} is used
as the default message representation. If \var{create} is \code{True}, the
mailbox is created if it does not exist.

It is for historical reasons that \var{factory} defaults to
\class{rfc822.Message} and that \var{dirname} is named as such rather than
\var{path}. For a \class{Maildir} instance that behaves like instances of other
\class{Mailbox} subclasses, set \var{factory} to \code{None}.
\end{classdesc}

Maildir is a directory-based mailbox format invented for the qmail mail
transfer agent and now widely supported by other programs. Messages in a
Maildir mailbox are stored in separate files within a common directory
structure. This design allows Maildir mailboxes to be accessed and modified by
multiple unrelated programs without data corruption, so file locking is
unnecessary.

Maildir mailboxes contain three subdirectories, namely: \file{tmp}, \file{new},
and \file{cur}. Messages are created momentarily in the \file{tmp} subdirectory
and then moved to the \file{new} subdirectory to finalize delivery. A mail user
agent may subsequently move the message to the \file{cur} subdirectory and
store information about the state of the message in a special "info" section
appended to its file name.

Folders of the style introduced by the Courier mail transfer agent are also
supported. Any subdirectory of the main mailbox is considered a folder if
\character{.} is the first character in its name. Folder names are represented
by \class{Maildir} without the leading \character{.}. Each folder is itself a
Maildir mailbox but should not contain other folders. Instead, a logical
nesting is indicated using \character{.} to delimit levels, e.g.,
"Archived.2005.07".

\begin{notice}
The Maildir specification requires the use of a colon (\character{:}) in
certain message file names. However, some operating systems do not permit this
character in file names, If you wish to use a Maildir-like format on such an
operating system, you should specify another character to use instead. The
exclamation point (\character{!}) is a popular choice. For example:
\begin{verbatim}
import mailbox
mailbox.Maildir.colon = '!'
\end{verbatim}
The \member{colon} attribute may also be set on a per-instance basis.
\end{notice}

\class{Maildir} instances have all of the methods of \class{Mailbox} in
addition to the following:

\begin{methoddesc}{list_folders}{}
Return a list of the names of all folders.
\end{methoddesc}

\begin{methoddesc}{get_folder}{folder}
Return a \class{Maildir} instance representing the folder whose name is
\var{folder}. A \exception{NoSuchMailboxError} exception is raised if the
folder does not exist.
\end{methoddesc}

\begin{methoddesc}{add_folder}{folder}
Create a folder whose name is \var{folder} and return a \class{Maildir}
instance representing it.
\end{methoddesc}

\begin{methoddesc}{remove_folder}{folder}
Delete the folder whose name is \var{folder}. If the folder contains any
messages, a \exception{NotEmptyError} exception will be raised and the folder
will not be deleted.
\end{methoddesc}

\begin{methoddesc}{clean}{}
Delete temporary files from the mailbox that have not been accessed in the
last 36 hours. The Maildir specification says that mail-reading programs
should do this occasionally.
\end{methoddesc}

Some \class{Mailbox} methods implemented by \class{Maildir} deserve special
remarks:

\begin{methoddesc}{add}{message}
\methodline[Maildir]{__setitem__}{key, message}
\methodline[Maildir]{update}{arg}
\warning{These methods generate unique file names based upon the current
process ID. When using multiple threads, undetected name clashes may occur and
cause corruption of the mailbox unless threads are coordinated to avoid using
these methods to manipulate the same mailbox simultaneously.}
\end{methoddesc}

\begin{methoddesc}{flush}{}
All changes to Maildir mailboxes are immediately applied, so this method does
nothing.
\end{methoddesc}

\begin{methoddesc}{lock}{}
\methodline{unlock}{}
Maildir mailboxes do not support (or require) locking, so these methods do
nothing. 
\end{methoddesc}

\begin{methoddesc}{close}{}
\class{Maildir} instances do not keep any open files and the underlying
mailboxes do not support locking, so this method does nothing.
\end{methoddesc}

\begin{methoddesc}{get_file}{key}
Depending upon the host platform, it may not be possible to modify or remove
the underlying message while the returned file remains open.
\end{methoddesc}

\begin{seealso}
    \seelink{http://www.qmail.org/man/man5/maildir.html}{maildir man page from
    qmail}{The original specification of the format.}
    \seelink{http://cr.yp.to/proto/maildir.html}{Using maildir format}{Notes
    on Maildir by its inventor. Includes an updated name-creation scheme and
    details on "info" semantics.}
    \seelink{http://www.courier-mta.org/?maildir.html}{maildir man page from
    Courier}{Another specification of the format. Describes a common extension
    for supporting folders.}
\end{seealso}

\subsubsection{\class{mbox}}
\label{mailbox-mbox}

\begin{classdesc}{mbox}{path\optional{, factory=None\optional{, create=True}}}
A subclass of \class{Mailbox} for mailboxes in mbox format. Parameter
\var{factory} is a callable object that accepts a file-like message
representation (which behaves as if opened in binary mode) and returns a custom
representation. If \var{factory} is \code{None}, \class{mboxMessage} is used as
the default message representation. If \var{create} is \code{True}, the mailbox
is created if it does not exist.
\end{classdesc}

The mbox format is the classic format for storing mail on \UNIX{} systems. All
messages in an mbox mailbox are stored in a single file with the beginning of
each message indicated by a line whose first five characters are "From~".

Several variations of the mbox format exist to address perceived shortcomings
in the original. In the interest of compatibility, \class{mbox} implements the
original format, which is sometimes referred to as \dfn{mboxo}. This means that
the \mailheader{Content-Length} header, if present, is ignored and that any
occurrences of "From~" at the beginning of a line in a message body are
transformed to ">From~" when storing the message, although occurences of
">From~" are not transformed to "From~" when reading the message.

Some \class{Mailbox} methods implemented by \class{mbox} deserve special
remarks:

\begin{methoddesc}{get_file}{key}
Using the file after calling \method{flush()} or \method{close()} on the
\class{mbox} instance may yield unpredictable results or raise an exception.
\end{methoddesc}

\begin{methoddesc}{lock}{}
\methodline{unlock}{}
Three locking mechanisms are used---dot locking and, if available, the
\cfunction{flock()} and \cfunction{lockf()} system calls.
\end{methoddesc}

\begin{seealso}
    \seelink{http://www.qmail.org/man/man5/mbox.html}{mbox man page from
    qmail}{A specification of the format and its variations.}
    \seelink{http://www.tin.org/bin/man.cgi?section=5\&topic=mbox}{mbox man
    page from tin}{Another specification of the format, with details on
    locking.}
    \seelink{http://home.netscape.com/eng/mozilla/2.0/relnotes/demo/content-length.html}
    {Configuring Netscape Mail on \UNIX{}: Why The Content-Length Format is
    Bad}{An argument for using the original mbox format rather than a
    variation.}
    \seelink{http://homepages.tesco.net./\tilde{}J.deBoynePollard/FGA/mail-mbox-formats.html}
    {"mbox" is a family of several mutually incompatible mailbox formats}{A
    history of mbox variations.}
\end{seealso}

\subsubsection{\class{MH}}
\label{mailbox-mh}

\begin{classdesc}{MH}{path\optional{, factory=None\optional{, create=True}}}
A subclass of \class{Mailbox} for mailboxes in MH format. Parameter
\var{factory} is a callable object that accepts a file-like message
representation (which behaves as if opened in binary mode) and returns a custom
representation. If \var{factory} is \code{None}, \class{MHMessage} is used as
the default message representation. If \var{create} is \code{True}, the mailbox
is created if it does not exist.
\end{classdesc}

MH is a directory-based mailbox format invented for the MH Message Handling
System, a mail user agent. Each message in an MH mailbox resides in its own
file. An MH mailbox may contain other MH mailboxes (called \dfn{folders}) in
addition to messages. Folders may be nested indefinitely. MH mailboxes also
support \dfn{sequences}, which are named lists used to logically group messages
without moving them to sub-folders. Sequences are defined in a file called
\file{.mh_sequences} in each folder.

The \class{MH} class manipulates MH mailboxes, but it does not attempt to
emulate all of \program{mh}'s behaviors. In particular, it does not modify and
is not affected by the \file{context} or \file{.mh_profile} files that are used
by \program{mh} to store its state and configuration.

\class{MH} instances have all of the methods of \class{Mailbox} in addition to
the following:

\begin{methoddesc}{list_folders}{}
Return a list of the names of all folders.
\end{methoddesc}

\begin{methoddesc}{get_folder}{folder}
Return an \class{MH} instance representing the folder whose name is
\var{folder}. A \exception{NoSuchMailboxError} exception is raised if the
folder does not exist.
\end{methoddesc}

\begin{methoddesc}{add_folder}{folder}
Create a folder whose name is \var{folder} and return an \class{MH} instance
representing it.
\end{methoddesc}

\begin{methoddesc}{remove_folder}{folder}
Delete the folder whose name is \var{folder}. If the folder contains any
messages, a \exception{NotEmptyError} exception will be raised and the folder
will not be deleted.
\end{methoddesc}

\begin{methoddesc}{get_sequences}{}
Return a dictionary of sequence names mapped to key lists. If there are no
sequences, the empty dictionary is returned.
\end{methoddesc}

\begin{methoddesc}{set_sequences}{sequences}
Re-define the sequences that exist in the mailbox based upon \var{sequences}, a
dictionary of names mapped to key lists, like returned by
\method{get_sequences()}.
\end{methoddesc}

\begin{methoddesc}{pack}{}
Rename messages in the mailbox as necessary to eliminate gaps in numbering.
Entries in the sequences list are updated correspondingly. \note{Already-issued
keys are invalidated by this operation and should not be subsequently used.}
\end{methoddesc}

Some \class{Mailbox} methods implemented by \class{MH} deserve special remarks:

\begin{methoddesc}{remove}{key}
\methodline{__delitem__}{key}
\methodline{discard}{key}
These methods immediately delete the message. The MH convention of marking a
message for deletion by prepending a comma to its name is not used.
\end{methoddesc}

\begin{methoddesc}{lock}{}
\methodline{unlock}{}
Three locking mechanisms are used---dot locking and, if available, the
\cfunction{flock()} and \cfunction{lockf()} system calls. For MH mailboxes,
locking the mailbox means locking the \file{.mh_sequences} file and, only for
the duration of any operations that affect them, locking individual message
files.
\end{methoddesc}

\begin{methoddesc}{get_file}{key}
Depending upon the host platform, it may not be possible to remove the
underlying message while the returned file remains open.
\end{methoddesc}

\begin{methoddesc}{flush}{}
All changes to MH mailboxes are immediately applied, so this method does
nothing.
\end{methoddesc}

\begin{methoddesc}{close}{}
\class{MH} instances do not keep any open files, so this method is equivelant
to \method{unlock()}.
\end{methoddesc}

\begin{seealso}
\seelink{http://www.nongnu.org/nmh/}{nmh - Message Handling System}{Home page
of \program{nmh}, an updated version of the original \program{mh}.}
\seelink{http://www.ics.uci.edu/\tilde{}mh/book/}{MH \& nmh: Email for Users \&
Programmers}{A GPL-licensed book on \program{mh} and \program{nmh}, with some
information on the mailbox format.}
\end{seealso}

\subsubsection{\class{Babyl}}
\label{mailbox-babyl}

\begin{classdesc}{Babyl}{path\optional{, factory=None\optional{, create=True}}}
A subclass of \class{Mailbox} for mailboxes in Babyl format. Parameter
\var{factory} is a callable object that accepts a file-like message
representation (which behaves as if opened in binary mode) and returns a custom
representation. If \var{factory} is \code{None}, \class{BabylMessage} is used
as the default message representation. If \var{create} is \code{True}, the
mailbox is created if it does not exist.
\end{classdesc}

Babyl is a single-file mailbox format used by the Rmail mail user agent
included with Emacs. The beginning of a message is indicated by a line
containing the two characters Control-Underscore
(\character{\textbackslash037}) and Control-L (\character{\textbackslash014}).
The end of a message is indicated by the start of the next message or, in the
case of the last message, a line containing a Control-Underscore
(\character{\textbackslash037}) character.

Messages in a Babyl mailbox have two sets of headers, original headers and
so-called visible headers. Visible headers are typically a subset of the
original headers that have been reformatted or abridged to be more attractive.
Each message in a Babyl mailbox also has an accompanying list of \dfn{labels},
or short strings that record extra information about the message, and a list of
all user-defined labels found in the mailbox is kept in the Babyl options
section.

\class{Babyl} instances have all of the methods of \class{Mailbox} in addition
to the following:

\begin{methoddesc}{get_labels}{}
Return a list of the names of all user-defined labels used in the mailbox.
\note{The actual messages are inspected to determine which labels exist in the
mailbox rather than consulting the list of labels in the Babyl options section,
but the Babyl section is updated whenever the mailbox is modified.}
\end{methoddesc}

Some \class{Mailbox} methods implemented by \class{Babyl} deserve special
remarks:

\begin{methoddesc}{get_file}{key}
In Babyl mailboxes, the headers of a message are not stored contiguously with
the body of the message. To generate a file-like representation, the headers
and body are copied together into a \class{StringIO} instance (from the
\module{StringIO} module), which has an API identical to that of a file. As a
result, the file-like object is truly independent of the underlying mailbox but
does not save memory compared to a string representation.
\end{methoddesc}

\begin{methoddesc}{lock}{}
\methodline{unlock}{}
Three locking mechanisms are used---dot locking and, if available, the
\cfunction{flock()} and \cfunction{lockf()} system calls.
\end{methoddesc}

\begin{seealso}
\seelink{http://quimby.gnus.org/notes/BABYL}{Format of Version 5 Babyl Files}{A
specification of the Babyl format.}
\seelink{http://www.gnu.org/software/emacs/manual/html_node/Rmail.html}{Reading
Mail with Rmail}{The Rmail manual, with some information on Babyl semantics.}
\end{seealso}

\subsubsection{\class{MMDF}}
\label{mailbox-mmdf}

\begin{classdesc}{MMDF}{path\optional{, factory=None\optional{, create=True}}}
A subclass of \class{Mailbox} for mailboxes in MMDF format. Parameter
\var{factory} is a callable object that accepts a file-like message
representation (which behaves as if opened in binary mode) and returns a custom
representation. If \var{factory} is \code{None}, \class{MMDFMessage} is used as
the default message representation. If \var{create} is \code{True}, the mailbox
is created if it does not exist.
\end{classdesc}

MMDF is a single-file mailbox format invented for the Multichannel Memorandum
Distribution Facility, a mail transfer agent. Each message is in the same form
as an mbox message but is bracketed before and after by lines containing four
Control-A (\character{\textbackslash001}) characters. As with the mbox format,
the beginning of each message is indicated by a line whose first five
characters are "From~", but additional occurrences of "From~" are not
transformed to ">From~" when storing messages because the extra message
separator lines prevent mistaking such occurrences for the starts of subsequent
messages.

Some \class{Mailbox} methods implemented by \class{MMDF} deserve special
remarks:

\begin{methoddesc}{get_file}{key}
Using the file after calling \method{flush()} or \method{close()} on the
\class{MMDF} instance may yield unpredictable results or raise an exception.
\end{methoddesc}

\begin{methoddesc}{lock}{}
\methodline{unlock}{}
Three locking mechanisms are used---dot locking and, if available, the
\cfunction{flock()} and \cfunction{lockf()} system calls.
\end{methoddesc}

\begin{seealso}
\seelink{http://www.tin.org/bin/man.cgi?section=5\&topic=mmdf}{mmdf man page
from tin}{A specification of MMDF format from the documentation of tin, a
newsreader.}
\seelink{http://en.wikipedia.org/wiki/MMDF}{MMDF}{A Wikipedia article
describing the Multichannel Memorandum Distribution Facility.}
\end{seealso}

\subsection{\class{Message} objects}
\label{mailbox-message-objects}

\begin{classdesc}{Message}{\optional{message}}
A subclass of the \module{email.Message} module's \class{Message}. Subclasses
of \class{mailbox.Message} add mailbox-format-specific state and behavior.

If \var{message} is omitted, the new instance is created in a default, empty
state. If \var{message} is an \class{email.Message.Message} instance, its
contents are copied; furthermore, any format-specific information is converted
insofar as possible if \var{message} is a \class{Message} instance. If
\var{message} is a string or a file, it should contain an \rfc{2822}-compliant
message, which is read and parsed.
\end{classdesc}

The format-specific state and behaviors offered by subclasses vary, but in
general it is only the properties that are not specific to a particular mailbox
that are supported (although presumably the properties are specific to a
particular mailbox format). For example, file offsets for single-file mailbox
formats and file names for directory-based mailbox formats are not retained,
because they are only applicable to the original mailbox. But state such as
whether a message has been read by the user or marked as important is retained,
because it applies to the message itself.

There is no requirement that \class{Message} instances be used to represent
messages retrieved using \class{Mailbox} instances. In some situations, the
time and memory required to generate \class{Message} representations might not
not acceptable. For such situations, \class{Mailbox} instances also offer
string and file-like representations, and a custom message factory may be
specified when a \class{Mailbox} instance is initialized. 

\subsubsection{\class{MaildirMessage}}
\label{mailbox-maildirmessage}

\begin{classdesc}{MaildirMessage}{\optional{message}}
A message with Maildir-specific behaviors. Parameter \var{message}
has the same meaning as with the \class{Message} constructor.
\end{classdesc}

Typically, a mail user agent application moves all of the messages in the
\file{new} subdirectory to the \file{cur} subdirectory after the first time the
user opens and closes the mailbox, recording that the messages are old whether
or not they've actually been read. Each message in \file{cur} has an "info"
section added to its file name to store information about its state. (Some mail
readers may also add an "info" section to messages in \file{new}.) The "info"
section may take one of two forms: it may contain "2," followed by a list of
standardized flags (e.g., "2,FR") or it may contain "1," followed by so-called
experimental information. Standard flags for Maildir messages are as follows:

\begin{tableiii}{l|l|l}{textrm}{Flag}{Meaning}{Explanation}
\lineiii{D}{Draft}{Under composition}
\lineiii{F}{Flagged}{Marked as important}
\lineiii{P}{Passed}{Forwarded, resent, or bounced}
\lineiii{R}{Replied}{Replied to}
\lineiii{S}{Seen}{Read}
\lineiii{T}{Trashed}{Marked for subsequent deletion}
\end{tableiii}

\class{MaildirMessage} instances offer the following methods:

\begin{methoddesc}{get_subdir}{}
Return either "new" (if the message should be stored in the \file{new}
subdirectory) or "cur" (if the message should be stored in the \file{cur}
subdirectory). \note{A message is typically moved from \file{new} to \file{cur}
after its mailbox has been accessed, whether or not the message is has been
read. A message \code{msg} has been read if \code{"S" not in msg.get_flags()}
is \code{True}.}
\end{methoddesc}

\begin{methoddesc}{set_subdir}{subdir}
Set the subdirectory the message should be stored in. Parameter \var{subdir}
must be either "new" or "cur".
\end{methoddesc}

\begin{methoddesc}{get_flags}{}
Return a string specifying the flags that are currently set. If the message
complies with the standard Maildir format, the result is the concatenation in
alphabetical order of zero or one occurrence of each of \character{D},
\character{F}, \character{P}, \character{R}, \character{S}, and \character{T}.
The empty string is returned if no flags are set or if "info" contains
experimental semantics.
\end{methoddesc}

\begin{methoddesc}{set_flags}{flags}
Set the flags specified by \var{flags} and unset all others.
\end{methoddesc}

\begin{methoddesc}{add_flag}{flag}
Set the flag(s) specified by \var{flag} without changing other flags. To add
more than one flag at a time, \var{flag} may be a string of more than one
character. The current "info" is overwritten whether or not it contains
experimental information rather than
flags.
\end{methoddesc}

\begin{methoddesc}{remove_flag}{flag}
Unset the flag(s) specified by \var{flag} without changing other flags. To
remove more than one flag at a time, \var{flag} maybe a string of more than one
character. If "info" contains experimental information rather than flags, the
current "info" is not modified.
\end{methoddesc}

\begin{methoddesc}{get_date}{}
Return the delivery date of the message as a floating-point number representing
seconds since the epoch.
\end{methoddesc}

\begin{methoddesc}{set_date}{date}
Set the delivery date of the message to \var{date}, a floating-point number
representing seconds since the epoch.
\end{methoddesc}

\begin{methoddesc}{get_info}{}
Return a string containing the "info" for a message. This is useful for
accessing and modifying "info" that is experimental (i.e., not a list of
flags).
\end{methoddesc}

\begin{methoddesc}{set_info}{info}
Set "info" to \var{info}, which should be a string.
\end{methoddesc}

When a \class{MaildirMessage} instance is created based upon an
\class{mboxMessage} or \class{MMDFMessage} instance, the \mailheader{Status}
and \mailheader{X-Status} headers are omitted and the following conversions
take place:

\begin{tableii}{l|l}{textrm}
    {Resulting state}{\class{mboxMessage} or \class{MMDFMessage} state}
\lineii{"cur" subdirectory}{O flag}
\lineii{F flag}{F flag}
\lineii{R flag}{A flag}
\lineii{S flag}{R flag}
\lineii{T flag}{D flag}
\end{tableii}

When a \class{MaildirMessage} instance is created based upon an
\class{MHMessage} instance, the following conversions take place:

\begin{tableii}{l|l}{textrm}
    {Resulting state}{\class{MHMessage} state}
\lineii{"cur" subdirectory}{"unseen" sequence}
\lineii{"cur" subdirectory and S flag}{no "unseen" sequence}
\lineii{F flag}{"flagged" sequence}
\lineii{R flag}{"replied" sequence}
\end{tableii}

When a \class{MaildirMessage} instance is created based upon a
\class{BabylMessage} instance, the following conversions take place:

\begin{tableii}{l|l}{textrm}
    {Resulting state}{\class{BabylMessage} state}
\lineii{"cur" subdirectory}{"unseen" label}
\lineii{"cur" subdirectory and S flag}{no "unseen" label}
\lineii{P flag}{"forwarded" or "resent" label}
\lineii{R flag}{"answered" label}
\lineii{T flag}{"deleted" label}
\end{tableii}

\subsubsection{\class{mboxMessage}}
\label{mailbox-mboxmessage}

\begin{classdesc}{mboxMessage}{\optional{message}}
A message with mbox-specific behaviors. Parameter \var{message} has the same
meaning as with the \class{Message} constructor.
\end{classdesc}

Messages in an mbox mailbox are stored together in a single file. The sender's
envelope address and the time of delivery are typically stored in a line
beginning with "From~" that is used to indicate the start of a message, though
there is considerable variation in the exact format of this data among mbox
implementations. Flags that indicate the state of the message, such as whether
it has been read or marked as important, are typically stored in
\mailheader{Status} and \mailheader{X-Status} headers.

Conventional flags for mbox messages are as follows:

\begin{tableiii}{l|l|l}{textrm}{Flag}{Meaning}{Explanation}
\lineiii{R}{Read}{Read}
\lineiii{O}{Old}{Previously detected by MUA}
\lineiii{D}{Deleted}{Marked for subsequent deletion}
\lineiii{F}{Flagged}{Marked as important}
\lineiii{A}{Answered}{Replied to}
\end{tableiii}

The "R" and "O" flags are stored in the \mailheader{Status} header, and the
"D", "F", and "A" flags are stored in the \mailheader{X-Status} header. The
flags and headers typically appear in the order mentioned.

\class{mboxMessage} instances offer the following methods:

\begin{methoddesc}{get_from}{}
Return a string representing the "From~" line that marks the start of the
message in an mbox mailbox. The leading "From~" and the trailing newline are
excluded.
\end{methoddesc}

\begin{methoddesc}{set_from}{from_\optional{, time_=None}}
Set the "From~" line to \var{from_}, which should be specified without a
leading "From~" or trailing newline. For convenience, \var{time_} may be
specified and will be formatted appropriately and appended to \var{from_}. If
\var{time_} is specified, it should be a \class{struct_time} instance, a tuple
suitable for passing to \method{time.strftime()}, or \code{True} (to use
\method{time.gmtime()}).
\end{methoddesc}

\begin{methoddesc}{get_flags}{}
Return a string specifying the flags that are currently set. If the message
complies with the conventional format, the result is the concatenation in the
following order of zero or one occurrence of each of \character{R},
\character{O}, \character{D}, \character{F}, and \character{A}.
\end{methoddesc}

\begin{methoddesc}{set_flags}{flags}
Set the flags specified by \var{flags} and unset all others. Parameter
\var{flags} should be the concatenation in any order of zero or more
occurrences of each of \character{R}, \character{O}, \character{D},
\character{F}, and \character{A}.
\end{methoddesc}

\begin{methoddesc}{add_flag}{flag}
Set the flag(s) specified by \var{flag} without changing other flags. To add
more than one flag at a time, \var{flag} may be a string of more than one
character.
\end{methoddesc}

\begin{methoddesc}{remove_flag}{flag}
Unset the flag(s) specified by \var{flag} without changing other flags. To
remove more than one flag at a time, \var{flag} maybe a string of more than one
character.
\end{methoddesc}

When an \class{mboxMessage} instance is created based upon a
\class{MaildirMessage} instance, a "From~" line is generated based upon the
\class{MaildirMessage} instance's delivery date, and the following conversions
take place:

\begin{tableii}{l|l}{textrm}
    {Resulting state}{\class{MaildirMessage} state}
\lineii{R flag}{S flag}
\lineii{O flag}{"cur" subdirectory}
\lineii{D flag}{T flag}
\lineii{F flag}{F flag}
\lineii{A flag}{R flag}
\end{tableii}

When an \class{mboxMessage} instance is created based upon an \class{MHMessage}
instance, the following conversions take place:

\begin{tableii}{l|l}{textrm}
    {Resulting state}{\class{MHMessage} state}
\lineii{R flag and O flag}{no "unseen" sequence}
\lineii{O flag}{"unseen" sequence}
\lineii{F flag}{"flagged" sequence}
\lineii{A flag}{"replied" sequence}
\end{tableii}

When an \class{mboxMessage} instance is created based upon a
\class{BabylMessage} instance, the following conversions take place:

\begin{tableii}{l|l}{textrm}
    {Resulting state}{\class{BabylMessage} state}
\lineii{R flag and O flag}{no "unseen" label}
\lineii{O flag}{"unseen" label}
\lineii{D flag}{"deleted" label}
\lineii{A flag}{"answered" label}
\end{tableii}

When a \class{Message} instance is created based upon an \class{MMDFMessage}
instance, the "From~" line is copied and all flags directly correspond:

\begin{tableii}{l|l}{textrm}
    {Resulting state}{\class{MMDFMessage} state}
\lineii{R flag}{R flag}
\lineii{O flag}{O flag}
\lineii{D flag}{D flag}
\lineii{F flag}{F flag}
\lineii{A flag}{A flag}
\end{tableii}

\subsubsection{\class{MHMessage}}
\label{mailbox-mhmessage}

\begin{classdesc}{MHMessage}{\optional{message}}
A message with MH-specific behaviors. Parameter \var{message} has the same
meaning as with the \class{Message} constructor.
\end{classdesc}

MH messages do not support marks or flags in the traditional sense, but they do
support sequences, which are logical groupings of arbitrary messages. Some mail
reading programs (although not the standard \program{mh} and \program{nmh}) use
sequences in much the same way flags are used with other formats, as follows:

\begin{tableii}{l|l}{textrm}{Sequence}{Explanation}
\lineii{unseen}{Not read, but previously detected by MUA}
\lineii{replied}{Replied to}
\lineii{flagged}{Marked as important}
\end{tableii}

\class{MHMessage} instances offer the following methods:

\begin{methoddesc}{get_sequences}{}
Return a list of the names of sequences that include this message.
\end{methoddesc}

\begin{methoddesc}{set_sequences}{sequences}
Set the list of sequences that include this message.
\end{methoddesc}

\begin{methoddesc}{add_sequence}{sequence}
Add \var{sequence} to the list of sequences that include this message.
\end{methoddesc}

\begin{methoddesc}{remove_sequence}{sequence}
Remove \var{sequence} from the list of sequences that include this message.
\end{methoddesc}

When an \class{MHMessage} instance is created based upon a
\class{MaildirMessage} instance, the following conversions take place:

\begin{tableii}{l|l}{textrm}
    {Resulting state}{\class{MaildirMessage} state}
\lineii{"unseen" sequence}{no S flag}
\lineii{"replied" sequence}{R flag}
\lineii{"flagged" sequence}{F flag}
\end{tableii}

When an \class{MHMessage} instance is created based upon an \class{mboxMessage}
or \class{MMDFMessage} instance, the \mailheader{Status} and
\mailheader{X-Status} headers are omitted and the following conversions take
place:

\begin{tableii}{l|l}{textrm}
    {Resulting state}{\class{mboxMessage} or \class{MMDFMessage} state}
\lineii{"unseen" sequence}{no R flag}
\lineii{"replied" sequence}{A flag}
\lineii{"flagged" sequence}{F flag}
\end{tableii}

When an \class{MHMessage} instance is created based upon a \class{BabylMessage}
instance, the following conversions take place:

\begin{tableii}{l|l}{textrm}
    {Resulting state}{\class{BabylMessage} state}
\lineii{"unseen" sequence}{"unseen" label}
\lineii{"replied" sequence}{"answered" label}
\end{tableii}

\subsubsection{\class{BabylMessage}}
\label{mailbox-babylmessage}

\begin{classdesc}{BabylMessage}{\optional{message}}
A message with Babyl-specific behaviors. Parameter \var{message} has the same
meaning as with the \class{Message} constructor.
\end{classdesc}

Certain message labels, called \dfn{attributes}, are defined by convention to
have special meanings. The attributes are as follows:

\begin{tableii}{l|l}{textrm}{Label}{Explanation}
\lineii{unseen}{Not read, but previously detected by MUA}
\lineii{deleted}{Marked for subsequent deletion}
\lineii{filed}{Copied to another file or mailbox}
\lineii{answered}{Replied to}
\lineii{forwarded}{Forwarded}
\lineii{edited}{Modified by the user}
\lineii{resent}{Resent}
\end{tableii}

By default, Rmail displays only
visible headers. The \class{BabylMessage} class, though, uses the original
headers because they are more complete. Visible headers may be accessed
explicitly if desired.

\class{BabylMessage} instances offer the following methods:

\begin{methoddesc}{get_labels}{}
Return a list of labels on the message.
\end{methoddesc}

\begin{methoddesc}{set_labels}{labels}
Set the list of labels on the message to \var{labels}.
\end{methoddesc}

\begin{methoddesc}{add_label}{label}
Add \var{label} to the list of labels on the message.
\end{methoddesc}

\begin{methoddesc}{remove_label}{label}
Remove \var{label} from the list of labels on the message.
\end{methoddesc}

\begin{methoddesc}{get_visible}{}
Return an \class{Message} instance whose headers are the message's visible
headers and whose body is empty.
\end{methoddesc}

\begin{methoddesc}{set_visible}{visible}
Set the message's visible headers to be the same as the headers in
\var{message}. Parameter \var{visible} should be a \class{Message} instance, an
\class{email.Message.Message} instance, a string, or a file-like object (which
should be open in text mode).
\end{methoddesc}

\begin{methoddesc}{update_visible}{}
When a \class{BabylMessage} instance's original headers are modified, the
visible headers are not automatically modified to correspond. This method
updates the visible headers as follows: each visible header with a
corresponding original header is set to the value of the original header, each
visible header without a corresponding original header is removed, and any of
\mailheader{Date}, \mailheader{From}, \mailheader{Reply-To}, \mailheader{To},
\mailheader{CC}, and \mailheader{Subject} that are present in the original
headers but not the visible headers are added to the visible headers.
\end{methoddesc}

When a \class{BabylMessage} instance is created based upon a
\class{MaildirMessage} instance, the following conversions take place:

\begin{tableii}{l|l}{textrm}
    {Resulting state}{\class{MaildirMessage} state}
\lineii{"unseen" label}{no S flag}
\lineii{"deleted" label}{T flag}
\lineii{"answered" label}{R flag}
\lineii{"forwarded" label}{P flag}
\end{tableii}

When a \class{BabylMessage} instance is created based upon an
\class{mboxMessage} or \class{MMDFMessage} instance, the \mailheader{Status}
and \mailheader{X-Status} headers are omitted and the following conversions
take place:

\begin{tableii}{l|l}{textrm}
    {Resulting state}{\class{mboxMessage} or \class{MMDFMessage} state}
\lineii{"unseen" label}{no R flag}
\lineii{"deleted" label}{D flag}
\lineii{"answered" label}{A flag}
\end{tableii}

When a \class{BabylMessage} instance is created based upon an \class{MHMessage}
instance, the following conversions take place:

\begin{tableii}{l|l}{textrm}
    {Resulting state}{\class{MHMessage} state}
\lineii{"unseen" label}{"unseen" sequence}
\lineii{"answered" label}{"replied" sequence}
\end{tableii}

\subsubsection{\class{MMDFMessage}}
\label{mailbox-mmdfmessage}

\begin{classdesc}{MMDFMessage}{\optional{message}}
A message with MMDF-specific behaviors. Parameter \var{message} has the same
meaning as with the \class{Message} constructor.
\end{classdesc}

As with message in an mbox mailbox, MMDF messages are stored with the sender's
address and the delivery date in an initial line beginning with "From ".
Likewise, flags that indicate the state of the message are typically stored in
\mailheader{Status} and \mailheader{X-Status} headers.

Conventional flags for MMDF messages are identical to those of mbox message and
are as follows:

\begin{tableiii}{l|l|l}{textrm}{Flag}{Meaning}{Explanation}
\lineiii{R}{Read}{Read}
\lineiii{O}{Old}{Previously detected by MUA}
\lineiii{D}{Deleted}{Marked for subsequent deletion}
\lineiii{F}{Flagged}{Marked as important}
\lineiii{A}{Answered}{Replied to}
\end{tableiii}

The "R" and "O" flags are stored in the \mailheader{Status} header, and the
"D", "F", and "A" flags are stored in the \mailheader{X-Status} header. The
flags and headers typically appear in the order mentioned.

\class{MMDFMessage} instances offer the following methods, which are identical
to those offered by \class{mboxMessage}:

\begin{methoddesc}{get_from}{}
Return a string representing the "From~" line that marks the start of the
message in an mbox mailbox. The leading "From~" and the trailing newline are
excluded.
\end{methoddesc}

\begin{methoddesc}{set_from}{from_\optional{, time_=None}}
Set the "From~" line to \var{from_}, which should be specified without a
leading "From~" or trailing newline. For convenience, \var{time_} may be
specified and will be formatted appropriately and appended to \var{from_}. If
\var{time_} is specified, it should be a \class{struct_time} instance, a tuple
suitable for passing to \method{time.strftime()}, or \code{True} (to use
\method{time.gmtime()}).
\end{methoddesc}

\begin{methoddesc}{get_flags}{}
Return a string specifying the flags that are currently set. If the message
complies with the conventional format, the result is the concatenation in the
following order of zero or one occurrence of each of \character{R},
\character{O}, \character{D}, \character{F}, and \character{A}.
\end{methoddesc}

\begin{methoddesc}{set_flags}{flags}
Set the flags specified by \var{flags} and unset all others. Parameter
\var{flags} should be the concatenation in any order of zero or more
occurrences of each of \character{R}, \character{O}, \character{D},
\character{F}, and \character{A}.
\end{methoddesc}

\begin{methoddesc}{add_flag}{flag}
Set the flag(s) specified by \var{flag} without changing other flags. To add
more than one flag at a time, \var{flag} may be a string of more than one
character.
\end{methoddesc}

\begin{methoddesc}{remove_flag}{flag}
Unset the flag(s) specified by \var{flag} without changing other flags. To
remove more than one flag at a time, \var{flag} maybe a string of more than one
character.
\end{methoddesc}

When an \class{MMDFMessage} instance is created based upon a
\class{MaildirMessage} instance, a "From~" line is generated based upon the
\class{MaildirMessage} instance's delivery date, and the following conversions
take place:

\begin{tableii}{l|l}{textrm}
    {Resulting state}{\class{MaildirMessage} state}
\lineii{R flag}{S flag}
\lineii{O flag}{"cur" subdirectory}
\lineii{D flag}{T flag}
\lineii{F flag}{F flag}
\lineii{A flag}{R flag}
\end{tableii}

When an \class{MMDFMessage} instance is created based upon an \class{MHMessage}
instance, the following conversions take place:

\begin{tableii}{l|l}{textrm}
    {Resulting state}{\class{MHMessage} state}
\lineii{R flag and O flag}{no "unseen" sequence}
\lineii{O flag}{"unseen" sequence}
\lineii{F flag}{"flagged" sequence}
\lineii{A flag}{"replied" sequence}
\end{tableii}

When an \class{MMDFMessage} instance is created based upon a
\class{BabylMessage} instance, the following conversions take place:

\begin{tableii}{l|l}{textrm}
    {Resulting state}{\class{BabylMessage} state}
\lineii{R flag and O flag}{no "unseen" label}
\lineii{O flag}{"unseen" label}
\lineii{D flag}{"deleted" label}
\lineii{A flag}{"answered" label}
\end{tableii}

When an \class{MMDFMessage} instance is created based upon an
\class{mboxMessage} instance, the "From~" line is copied and all flags directly
correspond:

\begin{tableii}{l|l}{textrm}
    {Resulting state}{\class{mboxMessage} state}
\lineii{R flag}{R flag}
\lineii{O flag}{O flag}
\lineii{D flag}{D flag}
\lineii{F flag}{F flag}
\lineii{A flag}{A flag}
\end{tableii}

\subsection{Exceptions}
\label{mailbox-deprecated}

The following exception classes are defined in the \module{mailbox} module:

\begin{classdesc}{Error}{}
The based class for all other module-specific exceptions.
\end{classdesc}

\begin{classdesc}{NoSuchMailboxError}{}
Raised when a mailbox is expected but is not found, such as when instantiating
a \class{Mailbox} subclass with a path that does not exist (and with the
\var{create} parameter set to \code{False}), or when opening a folder that does
not exist.
\end{classdesc}

\begin{classdesc}{NotEmptyErrorError}{}
Raised when a mailbox is not empty but is expected to be, such as when deleting
a folder that contains messages.
\end{classdesc}

\begin{classdesc}{ExternalClashError}{}
Raised when some mailbox-related condition beyond the control of the program
causes it to be unable to proceed, such as when failing to acquire a lock that
another program already holds a lock, or when a uniquely-generated file name
already exists.
\end{classdesc}

\begin{classdesc}{FormatError}{}
Raised when the data in a file cannot be parsed, such as when an \class{MH}
instance attempts to read a corrupted \file{.mh_sequences} file.
\end{classdesc}

\subsection{Deprecated classes and methods}
\label{mailbox-deprecated}

Older versions of the \module{mailbox} module do not support modification of
mailboxes, such as adding or removing message, and do not provide classes to
represent format-specific message properties. For backward compatibility, the
older mailbox classes are still available, but the newer classes should be used
in preference to them.

Older mailbox objects support only iteration and provide a single public
method:

\begin{methoddesc}{next}{}
Return the next message in the mailbox, created with the optional \var{factory}
argument passed into the mailbox object's constructor. By default this is an
\class{rfc822.Message} object (see the \refmodule{rfc822} module).  Depending
on the mailbox implementation the \var{fp} attribute of this object may be a
true file object or a class instance simulating a file object, taking care of
things like message boundaries if multiple mail messages are contained in a
single file, etc.  If no more messages are available, this method returns
\code{None}.
\end{methoddesc}

Most of the older mailbox classes have names that differ from the current
mailbox class names, except for \class{Maildir}. For this reason, the new
\class{Maildir} class defines a \method{next()} method and its constructor
differs slightly from those of the other new mailbox classes.

The older mailbox classes whose names are not the same as their newer
counterparts are as follows:

\begin{classdesc}{UnixMailbox}{fp\optional{, factory}}
Access to a classic \UNIX-style mailbox, where all messages are
contained in a single file and separated by \samp{From }
(a.k.a.\ \samp{From_}) lines.  The file object \var{fp} points to the
mailbox file.  The optional \var{factory} parameter is a callable that
should create new message objects.  \var{factory} is called with one
argument, \var{fp} by the \method{next()} method of the mailbox
object.  The default is the \class{rfc822.Message} class (see the
\refmodule{rfc822} module -- and the note below).

\begin{notice}
  For reasons of this module's internal implementation, you will
  probably want to open the \var{fp} object in binary mode.  This is
  especially important on Windows.
\end{notice}

For maximum portability, messages in a \UNIX-style mailbox are
separated by any line that begins exactly with the string \code{'From
'} (note the trailing space) if preceded by exactly two newlines.
Because of the wide-range of variations in practice, nothing else on
the From_ line should be considered.  However, the current
implementation doesn't check for the leading two newlines.  This is
usually fine for most applications.

The \class{UnixMailbox} class implements a more strict version of
From_ line checking, using a regular expression that usually correctly
matched From_ delimiters.  It considers delimiter line to be separated
by \samp{From \var{name} \var{time}} lines.  For maximum portability,
use the \class{PortableUnixMailbox} class instead.  This class is
identical to \class{UnixMailbox} except that individual messages are
separated by only \samp{From } lines.

For more information, see
\citetitle[http://home.netscape.com/eng/mozilla/2.0/relnotes/demo/content-length.html]{Configuring
Netscape Mail on \UNIX: Why the Content-Length Format is Bad}.
\end{classdesc}

\begin{classdesc}{PortableUnixMailbox}{fp\optional{, factory}}
A less-strict version of \class{UnixMailbox}, which considers only the
\samp{From } at the beginning of the line separating messages.  The
``\var{name} \var{time}'' portion of the From line is ignored, to
protect against some variations that are observed in practice.  This
works since lines in the message which begin with \code{'From '} are
quoted by mail handling software at delivery-time.
\end{classdesc}

\begin{classdesc}{MmdfMailbox}{fp\optional{, factory}}
Access an MMDF-style mailbox, where all messages are contained
in a single file and separated by lines consisting of 4 control-A
characters.  The file object \var{fp} points to the mailbox file.
Optional \var{factory} is as with the \class{UnixMailbox} class.
\end{classdesc}

\begin{classdesc}{MHMailbox}{dirname\optional{, factory}}
Access an MH mailbox, a directory with each message in a separate
file with a numeric name.
The name of the mailbox directory is passed in \var{dirname}.
\var{factory} is as with the \class{UnixMailbox} class.
\end{classdesc}

\begin{classdesc}{BabylMailbox}{fp\optional{, factory}}
Access a Babyl mailbox, which is similar to an MMDF mailbox.  In
Babyl format, each message has two sets of headers, the
\emph{original} headers and the \emph{visible} headers.  The original
headers appear before a line containing only \code{'*** EOOH ***'}
(End-Of-Original-Headers) and the visible headers appear after the
\code{EOOH} line.  Babyl-compliant mail readers will show you only the
visible headers, and \class{BabylMailbox} objects will return messages
containing only the visible headers.  You'll have to do your own
parsing of the mailbox file to get at the original headers.  Mail
messages start with the EOOH line and end with a line containing only
\code{'\e{}037\e{}014'}.  \var{factory} is as with the
\class{UnixMailbox} class.
\end{classdesc}

If you wish to use the older mailbox classes with the \module{email} module
rather than the deprecated \module{rfc822} module, you can do so as follows:

\begin{verbatim}
import email
import email.Errors
import mailbox

def msgfactory(fp):
    try:
        return email.message_from_file(fp)
    except email.Errors.MessageParseError:
        # Don't return None since that will
        # stop the mailbox iterator
        return ''

mbox = mailbox.UnixMailbox(fp, msgfactory)
\end{verbatim}

Alternatively, if you know your mailbox contains only well-formed MIME
messages, you can simplify this to:

\begin{verbatim}
import email
import mailbox

mbox = mailbox.UnixMailbox(fp, email.message_from_file)
\end{verbatim}

\subsection{Examples}
\label{mailbox-examples}

A simple example of printing the subjects of all messages in a mailbox that
seem interesting:

\begin{verbatim}
import mailbox
for message in mailbox.mbox('~/mbox'):
    subject = message['subject']       # Could possibly be None.
    if subject and 'python' in subject.lower():
        print subject
\end{verbatim}

To copy all mail from a Babyl mailbox to an MH mailbox, converting all
of the format-specific information that can be converted:

\begin{verbatim}
import mailbox
destination = mailbox.MH('~/Mail')
for message in mailbox.Babyl('~/RMAIL'):
    destination.add(MHMessage(message))
\end{verbatim}

An example of sorting mail from numerous mailing lists, being careful to avoid
mail corruption due to concurrent modification by other programs, mail loss due
to interruption of the program, or premature termination due to malformed
messages in the mailbox:

\begin{verbatim}
import mailbox
import email.Errors
list_names = ('python-list', 'python-dev', 'python-bugs')
boxes = dict((name, mailbox.mbox('~/email/%s' % name)) for name in list_names)
inbox = mailbox.Maildir('~/Maildir', None)
for key in inbox.iterkeys():
    try:
        message = inbox[key]
    except email.Errors.MessageParseError:
        continue                # The message is malformed. Just leave it.
    for name in list_names:
        list_id = message['list-id']
        if list_id and name in list_id:
            box = boxes[name]
            box.lock()
            box.add(message)
            box.flush()         # Write copy to disk before removing original.
            box.unlock()
            inbox.discard(key)
            break               # Found destination, so stop looking.
for box in boxes.itervalues():
    box.close()
\end{verbatim}

% LaTeX'ized from the comments in the module by Skip Montanaro
% <skip@mojam.com>.

\section{\module{mhlib} ---
         Object-oriented access to MH mailboxes}

\declaremodule{standard}{mhlib}
\modulesynopsis{Manipulate MH mailboxes from Python.}


The \module{mhlib} module provides a Python interface to MH folders and
their contents.

The module contains three basic classes, \class{MH}, which represents a
particular collection of folders, \class{Folder}, which represents a single
folder, and \class{Message}, which represents a single message.


\begin{classdesc}{MH}{\optional{path\optional{, profile}}}
\class{MH} represents a collection of MH folders.
\end{classdesc}

\begin{classdesc}{Folder}{mh, name}
The \class{Folder} class represents a single folder and its messages.
\end{classdesc}

\begin{classdesc}{Message}{folder, number\optional{, name}}
\class{Message} objects represent individual messages in a folder.  The
Message class is derived from \class{mimetools.Message}.
\end{classdesc}


\subsection{MH Objects \label{mh-objects}}

\class{MH} instances have the following methods:


\begin{methoddesc}[MH]{error}{format\optional{, ...}}
Print an error message -- can be overridden.
\end{methoddesc}

\begin{methoddesc}[MH]{getprofile}{key}
Return a profile entry (\code{None} if not set).
\end{methoddesc}

\begin{methoddesc}[MH]{getpath}{}
Return the mailbox pathname.
\end{methoddesc}

\begin{methoddesc}[MH]{getcontext}{}
Return the current folder name.
\end{methoddesc}

\begin{methoddesc}[MH]{setcontext}{name}
Set the current folder name.
\end{methoddesc}

\begin{methoddesc}[MH]{listfolders}{}
Return a list of top-level folders.
\end{methoddesc}

\begin{methoddesc}[MH]{listallfolders}{}
Return a list of all folders.
\end{methoddesc}

\begin{methoddesc}[MH]{listsubfolders}{name}
Return a list of direct subfolders of the given folder.
\end{methoddesc}

\begin{methoddesc}[MH]{listallsubfolders}{name}
Return a list of all subfolders of the given folder.
\end{methoddesc}

\begin{methoddesc}[MH]{makefolder}{name}
Create a new folder.
\end{methoddesc}

\begin{methoddesc}[MH]{deletefolder}{name}
Delete a folder -- must have no subfolders.
\end{methoddesc}

\begin{methoddesc}[MH]{openfolder}{name}
Return a new open folder object.
\end{methoddesc}



\subsection{Folder Objects \label{mh-folder-objects}}

\class{Folder} instances represent open folders and have the following
methods:


\begin{methoddesc}[Folder]{error}{format\optional{, ...}}
Print an error message -- can be overridden.
\end{methoddesc}

\begin{methoddesc}[Folder]{getfullname}{}
Return the folder's full pathname.
\end{methoddesc}

\begin{methoddesc}[Folder]{getsequencesfilename}{}
Return the full pathname of the folder's sequences file.
\end{methoddesc}

\begin{methoddesc}[Folder]{getmessagefilename}{n}
Return the full pathname of message \var{n} of the folder.
\end{methoddesc}

\begin{methoddesc}[Folder]{listmessages}{}
Return a list of messages in the folder (as numbers).
\end{methoddesc}

\begin{methoddesc}[Folder]{getcurrent}{}
Return the current message number.
\end{methoddesc}

\begin{methoddesc}[Folder]{setcurrent}{n}
Set the current message number to \var{n}.
\end{methoddesc}

\begin{methoddesc}[Folder]{parsesequence}{seq}
Parse msgs syntax into list of messages.
\end{methoddesc}

\begin{methoddesc}[Folder]{getlast}{}
Get last message, or \code{0} if no messages are in the folder.
\end{methoddesc}

\begin{methoddesc}[Folder]{setlast}{n}
Set last message (internal use only).
\end{methoddesc}

\begin{methoddesc}[Folder]{getsequences}{}
Return dictionary of sequences in folder.  The sequence names are used 
as keys, and the values are the lists of message numbers in the
sequences.
\end{methoddesc}

\begin{methoddesc}[Folder]{putsequences}{dict}
Return dictionary of sequences in folder {name: list}.
\end{methoddesc}

\begin{methoddesc}[Folder]{removemessages}{list}
Remove messages in list from folder.
\end{methoddesc}

\begin{methoddesc}[Folder]{refilemessages}{list, tofolder}
Move messages in list to other folder.
\end{methoddesc}

\begin{methoddesc}[Folder]{movemessage}{n, tofolder, ton}
Move one message to a given destination in another folder.
\end{methoddesc}

\begin{methoddesc}[Folder]{copymessage}{n, tofolder, ton}
Copy one message to a given destination in another folder.
\end{methoddesc}


\subsection{Message Objects \label{mh-message-objects}}

\begin{methoddesc}[Message]{openmessage}{n}
Return a new open message object (costs a file descriptor).
\end{methoddesc}

\section{Standard Module \sectcode{mimify}}
\label{module-mimify}
\stmodindex{mimify}

The mimify module defines two functions to convert mail messages to
and from MIME format.  The mail message can be either a simple message
or a so-called multipart message.  Each part is treated separately.
Mimifying (a part of) a message entails encoding the message as
quoted-printable if it contains any characters that cannot be
represented using 7-bit ASCII.  Unmimifying (a part of) a message
entails undoing the quoted-printable encoding.  Mimify and unmimify
are especially useful when a message has to be edited before being
sent.  Typical use would be:

\begin{verbatim}
unmimify message
edit message
mimify message
send message
\end{verbatim}

The modules defines the following user-callable functions and
user-settable variables:

\begin{funcdesc}{mimify}{infile, outfile}
Copy the message in \var{infile} to \var{outfile}, converting parts to
quoted-printable and adding MIME mail headers when necessary.
\var{infile} and \var{outfile} can be file objects (actually, any
object that has a \code{readline} method (for \var{infile}) or a
\code{write} method (for \var{outfile})) or strings naming the files.
If \var{infile} and \var{outfile} are both strings, they may have the
same value.
\end{funcdesc}

\begin{funcdesc}{unmimify}{infile, outfile, decode_base64 = 0} 
Copy the message in \var{infile} to \var{outfile}, decoding all
quoted-printable parts.  \var{infile} and \var{outfile} can be file
objects (actually, any object that has a \code{readline} method (for
\var{infile}) or a \code{write} method (for \var{outfile})) or strings
naming the files.  If \var{infile} and \var{outfile} are both strings,
they may have the same value.
If the \var{decode_base64} argument is provided and tests true, any
parts that are coded in the base64 encoding are decoded as well.
\end{funcdesc}

\begin{funcdesc}{mime_decode_header}{line}
Return a decoded version of the encoded header line in \var{line}.
\end{funcdesc}

\begin{funcdesc}{mime_encode_header}{line}
Return a MIME-encoded version of the header line in \var{line}.
\end{funcdesc}

\begin{datadesc}{MAXLEN}
By default, a part will be encoded as quoted-printable when it
contains any non-ASCII characters (i.e., characters with the 8th bit
set), or if there are any lines longer than \code{MAXLEN} characters
(default value 200).  
\end{datadesc}

\begin{datadesc}{CHARSET}
When not specified in the mail headers, a character set must be filled
in.  The string used is stored in \code{CHARSET}, and the default
value is ISO-8859-1 (also known as Latin1 (latin-one)).
\end{datadesc}

This module can also be used from the command line.  Usage is as
follows:
\begin{verbatim}
mimify.py -e [-l length] [infile [outfile]]
mimify.py -d [-b] [infile [outfile]]
\end{verbatim}
to encode (mimify) and decode (unmimify) respectively.  \var{infile}
defaults to standard input, \var{outfile} defaults to standard output.
The same file can be specified for input and output.

If the \code{-l} option is given when encoding, if there are any lines
longer than the specified \var{length}, the containing part will be
encoded.

If the \code{-b} option is given when decoding, any base64 parts will
be decoded as well.


% Module and documentation by Eric S. Raymond, 21 Dec 1998 
\section{Standard Module \module{netrc}}
\stmodindex{netrc}
\label{module-netrc}

The \code{netrc} class parses and encapsulates the netrc file format
used by Unix's ftp(1) and other FTP clientd

\begin{classdesc}{netrc}{\optional{file}}
A \class{netrc} instance or subclass instance enapsulates data from 
a netrc file.  The initialization argument, if present, specifies the file
to parse.  If no argument is given, the file .netrc in the user's home
directory will be read.  Parse errors will throw a SyntaxError
exception with associated diagnostic information including the file
name, line number, and terminating token.
\end{classdesc}

\subsection{netrc Objects}
\label{netrc-objects}

A \class{netrc} instance has the following methods:

\begin{methoddesc}{authenticators}{}
Return a 3-tuple (login, account, password) of authenticators for the
given host.  If the netrc file did not contain an entry for the given
host, return the tuple associated with the `default' entry.  If
neither matching host nor default entry is available, return None.
\end{methoddesc}

\begin{methoddesc}{__repr__}{host}
Dump the class data as a string in the format of a netrc file.
(This discards comments and may reorder the entries.)
\end{methoddesc}

Instances of \class{netrc} have public instance variables:

\begin{memberdesc}{hosts}
Dictionmary mapping host names to login/account/password tuples.  The
`default' entry, if any, is represented as a pseudo-host by that name.
\end{memberdesc}

\begin{memberdesc}{macros}
Dictionary mapping macro names to string lists.
\end{memberdesc}





\chapter{Restricted Execution}
\label{restricted}

In general, Python programs have complete access to the underlying
operating system throug the various functions and classes, For
example, a Python program can open any file for reading and writing by
using the \code{open()} built-in function (provided the underlying OS
gives you permission!).  This is exactly what you want for most
applications.

There exists a class of applications for which this ``openness'' is
inappropriate.  Take Grail: a web browser that accepts ``applets'',
snippets of Python code, from anywhere on the Internet for execution
on the local system.  This can be used to improve the user interface
of forms, for instance.  Since the originator of the code is unknown,
it is obvious that it cannot be trusted with the full resources of the
local machine.

\emph{Restricted execution} is the basic framework in Python that allows
for the segregation of trusted and untrusted code.  It is based on the
notion that trusted Python code (a \emph{supervisor}) can create a
``padded cell' (or environment) with limited permissions, and run the
untrusted code within this cell.  The untrusted code cannot break out
of its cell, and can only interact with sensitive system resources
through interfaces defined and managed by the trusted code.  The term
``restricted execution'' is favored over ``safe-Python''
since true safety is hard to define, and is determined by the way the
restricted environment is created.  Note that the restricted
environments can be nested, with inner cells creating subcells of
lesser, but never greater, privilege.

An interesting aspect of Python's restricted execution model is that
the interfaces presented to untrusted code usually have the same names
as those presented to trusted code.  Therefore no special interfaces
need to be learned to write code designed to run in a restricted
environment.  And because the exact nature of the padded cell is
determined by the supervisor, different restrictions can be imposed,
depending on the application.  For example, it might be deemed
``safe'' for untrusted code to read any file within a specified
directory, but never to write a file.  In this case, the supervisor
may redefine the built-in
\code{open()} function so that it raises an exception whenever the
\var{mode} parameter is \code{'w'}.  It might also perform a
\code{chroot()}-like operation on the \var{filename} parameter, such
that root is always relative to some safe ``sandbox'' area of the
filesystem.  In this case, the untrusted code would still see an
built-in \code{open()} function in its environment, with the same
calling interface.  The semantics would be identical too, with
\code{IOError}s being raised when the supervisor determined that an
unallowable parameter is being used.

The Python run-time determines whether a particular code block is
executing in restricted execution mode based on the identity of the
\code{__builtins__} object in its global variables: if this is (the
dictionary of) the standard \code{__builtin__} module, the code is
deemed to be unrestricted, else it is deemed to be restricted.

Python code executing in restricted mode faces a number of limitations
that are designed to prevent it from escaping from the padded cell.
For instance, the function object attribute \code{func_globals} and the
class and instance object attribute \code{__dict__} are unavailable.

Two modules provide the framework for setting up restricted execution
environments:

\begin{description}

\item[rexec]
--- Basic restricted execution framework.

\item[Bastion]
--- Providing restricted access to objects.

\end{description}

\section{\module{rexec} ---
         Restricted execution framework}

\declaremodule{standard}{rexec}
\modulesynopsis{Basic restricted execution framework.}
\versionchanged[Disabled module]{2.3}

\begin{notice}[warning]
  The documentation has been left in place to help in reading old code
  that uses the module.
\end{notice}

This module contains the \class{RExec} class, which supports
\method{r_eval()}, \method{r_execfile()}, \method{r_exec()}, and
\method{r_import()} methods, which are restricted versions of the standard
Python functions \method{eval()}, \method{execfile()} and
the \keyword{exec} and \keyword{import} statements.
Code executed in this restricted environment will
only have access to modules and functions that are deemed safe; you
can subclass \class{RExec} to add or remove capabilities as desired.

\begin{notice}[warning]
  While the \module{rexec} module is designed to perform as described
  below, it does have a few known vulnerabilities which could be
  exploited by carefully written code.  Thus it should not be relied
  upon in situations requiring ``production ready'' security.  In such
  situations, execution via sub-processes or very careful
  ``cleansing'' of both code and data to be processed may be
  necessary.  Alternatively, help in patching known \module{rexec}
  vulnerabilities would be welcomed.
\end{notice}

\begin{notice}
  The \class{RExec} class can prevent code from performing unsafe
  operations like reading or writing disk files, or using TCP/IP
  sockets.  However, it does not protect against code using extremely
  large amounts of memory or processor time.
\end{notice}

\begin{classdesc}{RExec}{\optional{hooks\optional{, verbose}}}
Returns an instance of the \class{RExec} class.  

\var{hooks} is an instance of the \class{RHooks} class or a subclass of it.
If it is omitted or \code{None}, the default \class{RHooks} class is
instantiated.
Whenever the \module{rexec} module searches for a module (even a
built-in one) or reads a module's code, it doesn't actually go out to
the file system itself.  Rather, it calls methods of an \class{RHooks}
instance that was passed to or created by its constructor.  (Actually,
the \class{RExec} object doesn't make these calls --- they are made by
a module loader object that's part of the \class{RExec} object.  This
allows another level of flexibility, which can be useful when changing
the mechanics of \keyword{import} within the restricted environment.)

By providing an alternate \class{RHooks} object, we can control the
file system accesses made to import a module, without changing the
actual algorithm that controls the order in which those accesses are
made.  For instance, we could substitute an \class{RHooks} object that
passes all filesystem requests to a file server elsewhere, via some
RPC mechanism such as ILU.  Grail's applet loader uses this to support
importing applets from a URL for a directory.

If \var{verbose} is true, additional debugging output may be sent to
standard output.
\end{classdesc}

It is important to be aware that code running in a restricted
environment can still call the \function{sys.exit()} function.  To
disallow restricted code from exiting the interpreter, always protect
calls that cause restricted code to run with a
\keyword{try}/\keyword{except} statement that catches the
\exception{SystemExit} exception.  Removing the \function{sys.exit()}
function from the restricted environment is not sufficient --- the
restricted code could still use \code{raise SystemExit}.  Removing
\exception{SystemExit} is not a reasonable option; some library code
makes use of this and would break were it not available.


\begin{seealso}
  \seetitle[http://grail.sourceforge.net/]{Grail Home Page}{Grail is a
            Web browser written entirely in Python.  It uses the
            \module{rexec} module as a foundation for supporting
            Python applets, and can be used as an example usage of
            this module.}
\end{seealso}


\subsection{RExec Objects \label{rexec-objects}}

\class{RExec} instances support the following methods:

\begin{methoddesc}{r_eval}{code}
\var{code} must either be a string containing a Python expression, or
a compiled code object, which will be evaluated in the restricted
environment's \module{__main__} module.  The value of the expression or
code object will be returned.
\end{methoddesc}

\begin{methoddesc}{r_exec}{code}
\var{code} must either be a string containing one or more lines of
Python code, or a compiled code object, which will be executed in the
restricted environment's \module{__main__} module.
\end{methoddesc}

\begin{methoddesc}{r_execfile}{filename}
Execute the Python code contained in the file \var{filename} in the
restricted environment's \module{__main__} module.
\end{methoddesc}

Methods whose names begin with \samp{s_} are similar to the functions
beginning with \samp{r_}, but the code will be granted access to
restricted versions of the standard I/O streams \code{sys.stdin},
\code{sys.stderr}, and \code{sys.stdout}.

\begin{methoddesc}{s_eval}{code}
\var{code} must be a string containing a Python expression, which will
be evaluated in the restricted environment.  
\end{methoddesc}

\begin{methoddesc}{s_exec}{code}
\var{code} must be a string containing one or more lines of Python code,
which will be executed in the restricted environment.  
\end{methoddesc}

\begin{methoddesc}{s_execfile}{code}
Execute the Python code contained in the file \var{filename} in the
restricted environment.
\end{methoddesc}

\class{RExec} objects must also support various methods which will be
implicitly called by code executing in the restricted environment.
Overriding these methods in a subclass is used to change the policies
enforced by a restricted environment.

\begin{methoddesc}{r_import}{modulename\optional{, globals\optional{,
                             locals\optional{, fromlist}}}}
Import the module \var{modulename}, raising an \exception{ImportError}
exception if the module is considered unsafe.
\end{methoddesc}

\begin{methoddesc}{r_open}{filename\optional{, mode\optional{, bufsize}}}
Method called when \function{open()} is called in the restricted
environment.  The arguments are identical to those of \function{open()},
and a file object (or a class instance compatible with file objects)
should be returned.  \class{RExec}'s default behaviour is allow opening
any file for reading, but forbidding any attempt to write a file.  See
the example below for an implementation of a less restrictive
\method{r_open()}.
\end{methoddesc}

\begin{methoddesc}{r_reload}{module}
Reload the module object \var{module}, re-parsing and re-initializing it.  
\end{methoddesc}

\begin{methoddesc}{r_unload}{module}
Unload the module object \var{module} (remove it from the
restricted environment's \code{sys.modules} dictionary).
\end{methoddesc}

And their equivalents with access to restricted standard I/O streams:

\begin{methoddesc}{s_import}{modulename\optional{, globals\optional{,
                             locals\optional{, fromlist}}}}
Import the module \var{modulename}, raising an \exception{ImportError}
exception if the module is considered unsafe.
\end{methoddesc}

\begin{methoddesc}{s_reload}{module}
Reload the module object \var{module}, re-parsing and re-initializing it.  
\end{methoddesc}

\begin{methoddesc}{s_unload}{module}
Unload the module object \var{module}.   
% XXX what are the semantics of this?  
\end{methoddesc}


\subsection{Defining restricted environments \label{rexec-extension}}

The \class{RExec} class has the following class attributes, which are
used by the \method{__init__()} method.  Changing them on an existing
instance won't have any effect; instead, create a subclass of
\class{RExec} and assign them new values in the class definition.
Instances of the new class will then use those new values.  All these
attributes are tuples of strings.

\begin{memberdesc}{nok_builtin_names}
Contains the names of built-in functions which will \emph{not} be
available to programs running in the restricted environment.  The
value for \class{RExec} is \code{('open', 'reload', '__import__')}.
(This gives the exceptions, because by far the majority of built-in
functions are harmless.  A subclass that wants to override this
variable should probably start with the value from the base class and
concatenate additional forbidden functions --- when new dangerous
built-in functions are added to Python, they will also be added to
this module.)
\end{memberdesc}

\begin{memberdesc}{ok_builtin_modules}
Contains the names of built-in modules which can be safely imported.
The value for \class{RExec} is \code{('audioop', 'array', 'binascii',
'cmath', 'errno', 'imageop', 'marshal', 'math', 'md5', 'operator',
'parser', 'regex', 'rotor', 'select', 'sha', '_sre', 'strop',
'struct', 'time')}.  A similar remark about overriding this variable
applies --- use the value from the base class as a starting point.
\end{memberdesc}

\begin{memberdesc}{ok_path}
Contains the directories which will be searched when an \keyword{import}
is performed in the restricted environment.  
The value for \class{RExec} is the same as \code{sys.path} (at the time
the module is loaded) for unrestricted code.
\end{memberdesc}

\begin{memberdesc}{ok_posix_names}
% Should this be called ok_os_names?
Contains the names of the functions in the \refmodule{os} module which will be
available to programs running in the restricted environment.  The
value for \class{RExec} is \code{('error', 'fstat', 'listdir',
'lstat', 'readlink', 'stat', 'times', 'uname', 'getpid', 'getppid',
'getcwd', 'getuid', 'getgid', 'geteuid', 'getegid')}.
\end{memberdesc}

\begin{memberdesc}{ok_sys_names}
Contains the names of the functions and variables in the \refmodule{sys}
module which will be available to programs running in the restricted
environment.  The value for \class{RExec} is \code{('ps1', 'ps2',
'copyright', 'version', 'platform', 'exit', 'maxint')}.
\end{memberdesc}

\begin{memberdesc}{ok_file_types}
Contains the file types from which modules are allowed to be loaded.
Each file type is an integer constant defined in the \refmodule{imp} module.
The meaningful values are \constant{PY_SOURCE}, \constant{PY_COMPILED}, and
\constant{C_EXTENSION}.  The value for \class{RExec} is \code{(C_EXTENSION,
PY_SOURCE)}.  Adding \constant{PY_COMPILED} in subclasses is not recommended;
an attacker could exit the restricted execution mode by putting a forged
byte-compiled file (\file{.pyc}) anywhere in your file system, for example
by writing it to \file{/tmp} or uploading it to the \file{/incoming}
directory of your public FTP server.
\end{memberdesc}


\subsection{An example}

Let us say that we want a slightly more relaxed policy than the
standard \class{RExec} class.  For example, if we're willing to allow
files in \file{/tmp} to be written, we can subclass the \class{RExec}
class:

\begin{verbatim}
class TmpWriterRExec(rexec.RExec):
    def r_open(self, file, mode='r', buf=-1):
        if mode in ('r', 'rb'):
            pass
        elif mode in ('w', 'wb', 'a', 'ab'):
            # check filename : must begin with /tmp/
            if file[:5]!='/tmp/': 
                raise IOError, "can't write outside /tmp"
            elif (string.find(file, '/../') >= 0 or
                 file[:3] == '../' or file[-3:] == '/..'):
                raise IOError, "'..' in filename forbidden"
        else: raise IOError, "Illegal open() mode"
        return open(file, mode, buf)
\end{verbatim}
%
Notice that the above code will occasionally forbid a perfectly valid
filename; for example, code in the restricted environment won't be
able to open a file called \file{/tmp/foo/../bar}.  To fix this, the
\method{r_open()} method would have to simplify the filename to
\file{/tmp/bar}, which would require splitting apart the filename and
performing various operations on it.  In cases where security is at
stake, it may be preferable to write simple code which is sometimes
overly restrictive, instead of more general code that is also more
complex and may harbor a subtle security hole.

\section{Standard Module \sectcode{Bastion}}
\stmodindex{Bastion}
\renewcommand{\indexsubitem}{(in module Bastion)}

% I'm concerned that the word 'bastion' won't be understood by people
% for whom English is a second language, making the module name
% somewhat mysterious.  Thus, the brief definition... --amk

According to the dictionary, a bastion is ``a fortified area or
position'', or ``something that is considered a stronghold.''  It's a
suitable name for this module, which provides a way to forbid access
to certain attributes of an object.  It must always be used with the
\code{rexec} module, in order to allow restricted-mode programs access
to certain safe attributes of an object, while denying access to
other, unsafe attributes.

% I've punted on the issue of documenting keyword arguments for now.

\begin{funcdesc}{Bastion}{object\optional{\, filter\, name\, class}}
Protect the class instance \var{object}, returning a bastion for the
object.  Any attempt to access one of the object's attributes will
have to be approved by the \var{filter} function; if the access is
denied an AttributeError exception will be raised.

If present, \var{filter} must be a function that accepts a string
containing an attribute name, and returns true if access to that
attribute will be permitted; if \var{filter} returns false, the access
is denied.  The default filter denies access to any function beginning
with an underscore (\code{_}).  The bastion's string representation
will be \code{<Bastion for \var{name}>} if a value for
\var{name} is provided; otherwise, \code{repr(\var{object})} will be used.

\var{class}, if present, would be a subclass of \code{BastionClass};
see the code in \file{bastion.py} for the details.  Overriding the
default \code{BastionClass} will rarely be required.  

\end{funcdesc}


\chapter{Multimedia Services}
\label{mmedia}

The modules described in this chapter implement various algorithms or
interfaces that are mainly useful for multimedia applications.  They
are available at the discretion of the installation.  Here's an overview:

\begin{description}

\item[audioop]
--- Manipulate raw audio data.

\item[imageop]
--- Manipulate raw image data.

\item[aifc]
--- Read and write audio files in AIFF or AIFC format.

\item[jpeg]
--- Read and write image files in compressed JPEG format.

\item[rgbimg]
--- Read and write image files in ``SGI RGB'' format (the module is
\emph{not} SGI specific though)!

\item[imghdr]
--- Determine the type of image contained in a file or byte stream.

\end{description}
			% Multimedia Services
\section{Built-in Module \sectcode{audioop}}
\label{module-audioop}
\bimodindex{audioop}

The \code{audioop} module contains some useful operations on sound fragments.
It operates on sound fragments consisting of signed integer samples
8, 16 or 32 bits wide, stored in Python strings.  This is the same
format as used by the \code{al} and \code{sunaudiodev} modules.  All
scalar items are integers, unless specified otherwise.

A few of the more complicated operations only take 16-bit samples,
otherwise the sample size (in bytes) is always a parameter of the operation.

The module defines the following variables and functions:

\renewcommand{\indexsubitem}{(in module audioop)}
\begin{excdesc}{error}
This exception is raised on all errors, such as unknown number of bytes
per sample, etc.
\end{excdesc}

\begin{funcdesc}{add}{fragment1\, fragment2\, width}
Return a fragment which is the addition of the two samples passed as
parameters.  \var{width} is the sample width in bytes, either
\code{1}, \code{2} or \code{4}.  Both fragments should have the same
length.
\end{funcdesc}

\begin{funcdesc}{adpcm2lin}{adpcmfragment\, width\, state}
Decode an Intel/DVI ADPCM coded fragment to a linear fragment.  See
the description of \code{lin2adpcm} for details on ADPCM coding.
Return a tuple \code{(\var{sample}, \var{newstate})} where the sample
has the width specified in \var{width}.
\end{funcdesc}

\begin{funcdesc}{adpcm32lin}{adpcmfragment\, width\, state}
Decode an alternative 3-bit ADPCM code.  See \code{lin2adpcm3} for
details.
\end{funcdesc}

\begin{funcdesc}{avg}{fragment\, width}
Return the average over all samples in the fragment.
\end{funcdesc}

\begin{funcdesc}{avgpp}{fragment\, width}
Return the average peak-peak value over all samples in the fragment.
No filtering is done, so the usefulness of this routine is
questionable.
\end{funcdesc}

\begin{funcdesc}{bias}{fragment\, width\, bias}
Return a fragment that is the original fragment with a bias added to
each sample.
\end{funcdesc}

\begin{funcdesc}{cross}{fragment\, width}
Return the number of zero crossings in the fragment passed as an
argument.
\end{funcdesc}

\begin{funcdesc}{findfactor}{fragment\, reference}
Return a factor \var{F} such that
\code{rms(add(fragment, mul(reference, -F)))} is minimal, i.e.,
return the factor with which you should multiply \var{reference} to
make it match as well as possible to \var{fragment}.  The fragments
should both contain 2-byte samples.

The time taken by this routine is proportional to \code{len(fragment)}. 
\end{funcdesc}

\begin{funcdesc}{findfit}{fragment\, reference}
This routine (which only accepts 2-byte sample fragments)

Try to match \var{reference} as well as possible to a portion of
\var{fragment} (which should be the longer fragment).  This is
(conceptually) done by taking slices out of \var{fragment}, using
\code{findfactor} to compute the best match, and minimizing the
result.  The fragments should both contain 2-byte samples.  Return a
tuple \code{(\var{offset}, \var{factor})} where \var{offset} is the
(integer) offset into \var{fragment} where the optimal match started
and \var{factor} is the (floating-point) factor as per
\code{findfactor}.
\end{funcdesc}

\begin{funcdesc}{findmax}{fragment\, length}
Search \var{fragment} for a slice of length \var{length} samples (not
bytes!)\ with maximum energy, i.e., return \var{i} for which
\code{rms(fragment[i*2:(i+length)*2])} is maximal.  The fragments
should both contain 2-byte samples.

The routine takes time proportional to \code{len(fragment)}.
\end{funcdesc}

\begin{funcdesc}{getsample}{fragment\, width\, index}
Return the value of sample \var{index} from the fragment.
\end{funcdesc}

\begin{funcdesc}{lin2lin}{fragment\, width\, newwidth}
Convert samples between 1-, 2- and 4-byte formats.
\end{funcdesc}

\begin{funcdesc}{lin2adpcm}{fragment\, width\, state}
Convert samples to 4 bit Intel/DVI ADPCM encoding.  ADPCM coding is an
adaptive coding scheme, whereby each 4 bit number is the difference
between one sample and the next, divided by a (varying) step.  The
Intel/DVI ADPCM algorithm has been selected for use by the IMA, so it
may well become a standard.

\code{State} is a tuple containing the state of the coder.  The coder
returns a tuple \code{(\var{adpcmfrag}, \var{newstate})}, and the
\var{newstate} should be passed to the next call of lin2adpcm.  In the
initial call \code{None} can be passed as the state.  \var{adpcmfrag}
is the ADPCM coded fragment packed 2 4-bit values per byte.
\end{funcdesc}

\begin{funcdesc}{lin2adpcm3}{fragment\, width\, state}
This is an alternative ADPCM coder that uses only 3 bits per sample.
It is not compatible with the Intel/DVI ADPCM coder and its output is
not packed (due to laziness on the side of the author).  Its use is
discouraged.
\end{funcdesc}

\begin{funcdesc}{lin2ulaw}{fragment\, width}
Convert samples in the audio fragment to U-LAW encoding and return
this as a Python string.  U-LAW is an audio encoding format whereby
you get a dynamic range of about 14 bits using only 8 bit samples.  It
is used by the Sun audio hardware, among others.
\end{funcdesc}

\begin{funcdesc}{minmax}{fragment\, width}
Return a tuple consisting of the minimum and maximum values of all
samples in the sound fragment.
\end{funcdesc}

\begin{funcdesc}{max}{fragment\, width}
Return the maximum of the {\em absolute value} of all samples in a
fragment.
\end{funcdesc}

\begin{funcdesc}{maxpp}{fragment\, width}
Return the maximum peak-peak value in the sound fragment.
\end{funcdesc}

\begin{funcdesc}{mul}{fragment\, width\, factor}
Return a fragment that has all samples in the original framgent
multiplied by the floating-point value \var{factor}.  Overflow is
silently ignored.
\end{funcdesc}

\begin{funcdesc}{ratecv}{fragment\, width\, nchannels\, inrate\, outrate\, state\optional{\, weightA\, weightB}}
Convert the frame rate of the input fragment.

\code{State} is a tuple containing the state of the converter.  The
converter returns a tupl \code{(\var{newfragment}, \var{newstate})},
and \var{newstate} should be passed to the next call of ratecv.

The \code{weightA} and \code{weightB} arguments are parameters for a
simple digital filter and default to 1 and 0 respectively.
\end{funcdesc}

\begin{funcdesc}{reverse}{fragment\, width}
Reverse the samples in a fragment and returns the modified fragment.
\end{funcdesc}

\begin{funcdesc}{rms}{fragment\, width}
Return the root-mean-square of the fragment, i.e.
\iftexi
the square root of the quotient of the sum of all squared sample value,
divided by the sumber of samples.
\else
% in eqn: sqrt { sum S sub i sup 2  over n }
\begin{displaymath}
\catcode`_=8
\sqrt{\frac{\sum{{S_{i}}^{2}}}{n}}
\end{displaymath}
\fi
This is a measure of the power in an audio signal.
\end{funcdesc}

\begin{funcdesc}{tomono}{fragment\, width\, lfactor\, rfactor} 
Convert a stereo fragment to a mono fragment.  The left channel is
multiplied by \var{lfactor} and the right channel by \var{rfactor}
before adding the two channels to give a mono signal.
\end{funcdesc}

\begin{funcdesc}{tostereo}{fragment\, width\, lfactor\, rfactor}
Generate a stereo fragment from a mono fragment.  Each pair of samples
in the stereo fragment are computed from the mono sample, whereby left
channel samples are multiplied by \var{lfactor} and right channel
samples by \var{rfactor}.
\end{funcdesc}

\begin{funcdesc}{ulaw2lin}{fragment\, width}
Convert sound fragments in ULAW encoding to linearly encoded sound
fragments.  ULAW encoding always uses 8 bits samples, so \var{width}
refers only to the sample width of the output fragment here.
\end{funcdesc}

Note that operations such as \code{mul} or \code{max} make no
distinction between mono and stereo fragments, i.e.\ all samples are
treated equal.  If this is a problem the stereo fragment should be split
into two mono fragments first and recombined later.  Here is an example
of how to do that:
\bcode\begin{verbatim}
def mul_stereo(sample, width, lfactor, rfactor):
    lsample = audioop.tomono(sample, width, 1, 0)
    rsample = audioop.tomono(sample, width, 0, 1)
    lsample = audioop.mul(sample, width, lfactor)
    rsample = audioop.mul(sample, width, rfactor)
    lsample = audioop.tostereo(lsample, width, 1, 0)
    rsample = audioop.tostereo(rsample, width, 0, 1)
    return audioop.add(lsample, rsample, width)
\end{verbatim}\ecode
%
If you use the ADPCM coder to build network packets and you want your
protocol to be stateless (i.e.\ to be able to tolerate packet loss)
you should not only transmit the data but also the state.  Note that
you should send the \var{initial} state (the one you passed to
\code{lin2adpcm}) along to the decoder, not the final state (as returned by
the coder).  If you want to use \code{struct} to store the state in
binary you can code the first element (the predicted value) in 16 bits
and the second (the delta index) in 8.

The ADPCM coders have never been tried against other ADPCM coders,
only against themselves.  It could well be that I misinterpreted the
standards in which case they will not be interoperable with the
respective standards.

The \code{find...} routines might look a bit funny at first sight.
They are primarily meant to do echo cancellation.  A reasonably
fast way to do this is to pick the most energetic piece of the output
sample, locate that in the input sample and subtract the whole output
sample from the input sample:
\bcode\begin{verbatim}
def echocancel(outputdata, inputdata):
    pos = audioop.findmax(outputdata, 800)    # one tenth second
    out_test = outputdata[pos*2:]
    in_test = inputdata[pos*2:]
    ipos, factor = audioop.findfit(in_test, out_test)
    # Optional (for better cancellation):
    # factor = audioop.findfactor(in_test[ipos*2:ipos*2+len(out_test)], 
    #              out_test)
    prefill = '\0'*(pos+ipos)*2
    postfill = '\0'*(len(inputdata)-len(prefill)-len(outputdata))
    outputdata = prefill + audioop.mul(outputdata,2,-factor) + postfill
    return audioop.add(inputdata, outputdata, 2)
\end{verbatim}\ecode

\section{\module{imageop} ---
         Manipulate raw image data}

\declaremodule{builtin}{imageop}
\modulesynopsis{Manipulate raw image data.}


The \module{imageop} module contains some useful operations on images.
It operates on images consisting of 8 or 32 bit pixels stored in
Python strings.  This is the same format as used by
\function{gl.lrectwrite()} and the \refmodule{imgfile} module.

The module defines the following variables and functions:

\begin{excdesc}{error}
This exception is raised on all errors, such as unknown number of bits
per pixel, etc.
\end{excdesc}


\begin{funcdesc}{crop}{image, psize, width, height, x0, y0, x1, y1}
Return the selected part of \var{image}, which should by
\var{width} by \var{height} in size and consist of pixels of
\var{psize} bytes. \var{x0}, \var{y0}, \var{x1} and \var{y1} are like
the \function{gl.lrectread()} parameters, i.e.\ the boundary is
included in the new image.  The new boundaries need not be inside the
picture.  Pixels that fall outside the old image will have their value
set to zero.  If \var{x0} is bigger than \var{x1} the new image is
mirrored.  The same holds for the y coordinates.
\end{funcdesc}

\begin{funcdesc}{scale}{image, psize, width, height, newwidth, newheight}
Return \var{image} scaled to size \var{newwidth} by \var{newheight}.
No interpolation is done, scaling is done by simple-minded pixel
duplication or removal.  Therefore, computer-generated images or
dithered images will not look nice after scaling.
\end{funcdesc}

\begin{funcdesc}{tovideo}{image, psize, width, height}
Run a vertical low-pass filter over an image.  It does so by computing
each destination pixel as the average of two vertically-aligned source
pixels.  The main use of this routine is to forestall excessive
flicker if the image is displayed on a video device that uses
interlacing, hence the name.
\end{funcdesc}

\begin{funcdesc}{grey2mono}{image, width, height, threshold}
Convert a 8-bit deep greyscale image to a 1-bit deep image by
thresholding all the pixels.  The resulting image is tightly packed and
is probably only useful as an argument to \function{mono2grey()}.
\end{funcdesc}

\begin{funcdesc}{dither2mono}{image, width, height}
Convert an 8-bit greyscale image to a 1-bit monochrome image using a
(simple-minded) dithering algorithm.
\end{funcdesc}

\begin{funcdesc}{mono2grey}{image, width, height, p0, p1}
Convert a 1-bit monochrome image to an 8 bit greyscale or color image.
All pixels that are zero-valued on input get value \var{p0} on output
and all one-value input pixels get value \var{p1} on output.  To
convert a monochrome black-and-white image to greyscale pass the
values \code{0} and \code{255} respectively.
\end{funcdesc}

\begin{funcdesc}{grey2grey4}{image, width, height}
Convert an 8-bit greyscale image to a 4-bit greyscale image without
dithering.
\end{funcdesc}

\begin{funcdesc}{grey2grey2}{image, width, height}
Convert an 8-bit greyscale image to a 2-bit greyscale image without
dithering.
\end{funcdesc}

\begin{funcdesc}{dither2grey2}{image, width, height}
Convert an 8-bit greyscale image to a 2-bit greyscale image with
dithering.  As for \function{dither2mono()}, the dithering algorithm
is currently very simple.
\end{funcdesc}

\begin{funcdesc}{grey42grey}{image, width, height}
Convert a 4-bit greyscale image to an 8-bit greyscale image.
\end{funcdesc}

\begin{funcdesc}{grey22grey}{image, width, height}
Convert a 2-bit greyscale image to an 8-bit greyscale image.
\end{funcdesc}

\section{Standard Module \sectcode{aifc}}
\label{module-aifc}
\stmodindex{aifc}

This module provides support for reading and writing AIFF and AIFF-C
files.  AIFF is Audio Interchange File Format, a format for storing
digital audio samples in a file.  AIFF-C is a newer version of the
format that includes the ability to compress the audio data.

Audio files have a number of parameters that describe the audio data.
The sampling rate or frame rate is the number of times per second the
sound is sampled.  The number of channels indicate if the audio is
mono, stereo, or quadro.  Each frame consists of one sample per
channel.  The sample size is the size in bytes of each sample.  Thus a
frame consists of \var{nchannels}*\var{samplesize} bytes, and a
second's worth of audio consists of
\var{nchannels}*\var{samplesize}*\var{framerate} bytes.

For example, CD quality audio has a sample size of two bytes (16
bits), uses two channels (stereo) and has a frame rate of 44,100
frames/second.  This gives a frame size of 4 bytes (2*2), and a
second's worth occupies 2*2*44100 bytes, i.e.\ 176,400 bytes.

Module \code{aifc} defines the following function:

\setindexsubitem{(in module aifc)}
\begin{funcdesc}{open}{file, mode}
Open an AIFF or AIFF-C file and return an object instance with
methods that are described below.  The argument file is either a
string naming a file or a file object.  The mode is either the string
\code{'r'} when the file must be opened for reading, or \code{'w'}
when the file must be opened for writing.  When used for writing, the
file object should be seekable, unless you know ahead of time how many
samples you are going to write in total and use
\code{writeframesraw()} and \code{setnframes()}.
\end{funcdesc}

Objects returned by \code{aifc.open()} when a file is opened for
reading have the following methods:

\setindexsubitem{(aifc object method)}
\begin{funcdesc}{getnchannels}{}
Return the number of audio channels (1 for mono, 2 for stereo).
\end{funcdesc}

\begin{funcdesc}{getsampwidth}{}
Return the size in bytes of individual samples.
\end{funcdesc}

\begin{funcdesc}{getframerate}{}
Return the sampling rate (number of audio frames per second).
\end{funcdesc}

\begin{funcdesc}{getnframes}{}
Return the number of audio frames in the file.
\end{funcdesc}

\begin{funcdesc}{getcomptype}{}
Return a four-character string describing the type of compression used
in the audio file.  For AIFF files, the returned value is
\code{'NONE'}.
\end{funcdesc}

\begin{funcdesc}{getcompname}{}
Return a human-readable description of the type of compression used in
the audio file.  For AIFF files, the returned value is \code{'not
compressed'}.
\end{funcdesc}

\begin{funcdesc}{getparams}{}
Return a tuple consisting of all of the above values in the above
order.
\end{funcdesc}

\begin{funcdesc}{getmarkers}{}
Return a list of markers in the audio file.  A marker consists of a
tuple of three elements.  The first is the mark ID (an integer), the
second is the mark position in frames from the beginning of the data
(an integer), the third is the name of the mark (a string).
\end{funcdesc}

\begin{funcdesc}{getmark}{id}
Return the tuple as described in \code{getmarkers} for the mark with
the given id.
\end{funcdesc}

\begin{funcdesc}{readframes}{nframes}
Read and return the next \var{nframes} frames from the audio file.  The
returned data is a string containing for each frame the uncompressed
samples of all channels.
\end{funcdesc}

\begin{funcdesc}{rewind}{}
Rewind the read pointer.  The next \code{readframes} will start from
the beginning.
\end{funcdesc}

\begin{funcdesc}{setpos}{pos}
Seek to the specified frame number.
\end{funcdesc}

\begin{funcdesc}{tell}{}
Return the current frame number.
\end{funcdesc}

\begin{funcdesc}{close}{}
Close the AIFF file.  After calling this method, the object can no
longer be used.
\end{funcdesc}

Objects returned by \code{aifc.open()} when a file is opened for
writing have all the above methods, except for \code{readframes} and
\code{setpos}.  In addition the following methods exist.  The
\code{get} methods can only be called after the corresponding
\code{set} methods have been called.  Before the first
\code{writeframes()} or \code{writeframesraw()}, all parameters except
for the number of frames must be filled in.

\begin{funcdesc}{aiff}{}
Create an AIFF file.  The default is that an AIFF-C file is created,
unless the name of the file ends in \code{'.aiff'} in which case the
default is an AIFF file.
\end{funcdesc}

\begin{funcdesc}{aifc}{}
Create an AIFF-C file.  The default is that an AIFF-C file is created,
unless the name of the file ends in \code{'.aiff'} in which case the
default is an AIFF file.
\end{funcdesc}

\begin{funcdesc}{setnchannels}{nchannels}
Specify the number of channels in the audio file.
\end{funcdesc}

\begin{funcdesc}{setsampwidth}{width}
Specify the size in bytes of audio samples.
\end{funcdesc}

\begin{funcdesc}{setframerate}{rate}
Specify the sampling frequency in frames per second.
\end{funcdesc}

\begin{funcdesc}{setnframes}{nframes}
Specify the number of frames that are to be written to the audio file.
If this parameter is not set, or not set correctly, the file needs to
support seeking.
\end{funcdesc}

\begin{funcdesc}{setcomptype}{type, name}
Specify the compression type.  If not specified, the audio data will
not be compressed.  In AIFF files, compression is not possible.  The
name parameter should be a human-readable description of the
compression type, the type parameter should be a four-character
string.  Currently the following compression types are supported:
NONE, ULAW, ALAW, G722.
\end{funcdesc}

\begin{funcdesc}{setparams}{nchannels, sampwidth, framerate, comptype, compname}
Set all the above parameters at once.  The argument is a tuple
consisting of the various parameters.  This means that it is possible
to use the result of a \code{getparams()} call as argument to
\code{setparams()}.
\end{funcdesc}

\begin{funcdesc}{setmark}{id, pos, name}
Add a mark with the given id (larger than 0), and the given name at
the given position.  This method can be called at any time before
\code{close()}.
\end{funcdesc}

\begin{funcdesc}{tell}{}
Return the current write position in the output file.  Useful in
combination with \code{setmark()}.
\end{funcdesc}

\begin{funcdesc}{writeframes}{data}
Write data to the output file.  This method can only be called after
the audio file parameters have been set.
\end{funcdesc}

\begin{funcdesc}{writeframesraw}{data}
Like \code{writeframes()}, except that the header of the audio file is
not updated.
\end{funcdesc}

\begin{funcdesc}{close}{}
Close the AIFF file.  The header of the file is updated to reflect the
actual size of the audio data. After calling this method, the object
can no longer be used.
\end{funcdesc}

\section{\module{sunau} ---
         Read and write Sun AU files}

\declaremodule{standard}{sunau}
\sectionauthor{Moshe Zadka}{mzadka@geocities.com}
\modulesynopsis{Provide an interface to the Sun AU sound format.}

The \module{sunau} module provides a convenient interface to the Sun
AU sound format.  Note that this module is interface-compatible with
the modules \refmodule{aifc} and \refmodule{wave}.

An audio file consists of a header followed by the data.  The fields
of the header are:

\begin{tableii}{l|l}{textrm}{Field}{Contents}
  \lineii{magic word}{The four bytes \samp{.snd}.}
  \lineii{header size}{Size of the header, including info, in bytes.}
  \lineii{data size}{Physical size of the data, in bytes.}
  \lineii{encoding}{Indicates how the audio samples are encoded.}
  \lineii{sample rate}{The sampling rate.}
  \lineii{\# of channels}{The number of channels in the samples.}
  \lineii{info}{\ASCII{} string giving a description of the audio
                file (padded with null bytes).}
\end{tableii}

Apart from the info field, all header fields are 4 bytes in size.
They are all 32-bit unsigned integers encoded in big-endian byte
order.


The \module{sunau} module defines the following functions:

\begin{funcdesc}{open}{file, mode}
If \var{file} is a string, open the file by that name, otherwise treat it
as a seekable file-like object. \var{mode} can be any of
\begin{description}
	\item[\code{'r'}] Read only mode.
	\item[\code{'w'}] Write only mode.
\end{description}
Note that it does not allow read/write files.

A \var{mode} of \code{'r'} returns a \class{AU_read}
object, while a \var{mode} of \code{'w'} or \code{'wb'} returns
a \class{AU_write} object.
\end{funcdesc}

\begin{funcdesc}{openfp}{file, mode}
A synonym for \function{open}, maintained for backwards compatibility.
\end{funcdesc}

The \module{sunau} module defines the following exception:

\begin{excdesc}{Error}
An error raised when something is impossible because of Sun AU specs or 
implementation deficiency.
\end{excdesc}

The \module{sunau} module defines the following data items:

\begin{datadesc}{AUDIO_FILE_MAGIC}
An integer every valid Sun AU file begins with, stored in big-endian
form.  This is the string \samp{.snd} interpreted as an integer.
\end{datadesc}

\begin{datadesc}{AUDIO_FILE_ENCODING_MULAW_8}
\dataline{AUDIO_FILE_ENCODING_LINEAR_8}
\dataline{AUDIO_FILE_ENCODING_LINEAR_16}
\dataline{AUDIO_FILE_ENCODING_LINEAR_24}
\dataline{AUDIO_FILE_ENCODING_LINEAR_32}
\dataline{AUDIO_FILE_ENCODING_ALAW_8}
Values of the encoding field from the AU header which are supported by
this module.
\end{datadesc}

\begin{datadesc}{AUDIO_FILE_ENCODING_FLOAT}
\dataline{AUDIO_FILE_ENCODING_DOUBLE}
\dataline{AUDIO_FILE_ENCODING_ADPCM_G721}
\dataline{AUDIO_FILE_ENCODING_ADPCM_G722}
\dataline{AUDIO_FILE_ENCODING_ADPCM_G723_3}
\dataline{AUDIO_FILE_ENCODING_ADPCM_G723_5}
Additional known values of the encoding field from the AU header, but
which are not supported by this module.
\end{datadesc}


\subsection{AU_read Objects \label{au-read-objects}}

AU_read objects, as returned by \function{open()} above, have the
following methods:

\begin{methoddesc}[AU_read]{close}{}
Close the stream, and make the instance unusable. (This is 
called automatically on deletion.)
\end{methoddesc}

\begin{methoddesc}[AU_read]{getnchannels}{}
Returns number of audio channels (1 for mone, 2 for stereo).
\end{methoddesc}

\begin{methoddesc}[AU_read]{getsampwidth}{}
Returns sample width in bytes.
\end{methoddesc}

\begin{methoddesc}[AU_read]{getframerate}{}
Returns sampling frequency.
\end{methoddesc}

\begin{methoddesc}[AU_read]{getnframes}{}
Returns number of audio frames.
\end{methoddesc}

\begin{methoddesc}[AU_read]{getcomptype}{}
Returns compression type.
Supported compression types are \code{'ULAW'}, \code{'ALAW'} and \code{'NONE'}.
\end{methoddesc}

\begin{methoddesc}[AU_read]{getcompname}{}
Human-readable version of \method{getcomptype()}. 
The supported types have the respective names \code{'CCITT G.711
u-law'}, \code{'CCITT G.711 A-law'} and \code{'not compressed'}.
\end{methoddesc}

\begin{methoddesc}[AU_read]{getparams}{}
Returns a tuple \code{(\var{nchannels}, \var{sampwidth},
\var{framerate}, \var{nframes}, \var{comptype}, \var{compname})},
equivalent to output of the \method{get*()} methods.
\end{methoddesc}

\begin{methoddesc}[AU_read]{readframes}{n}
Reads and returns at most \var{n} frames of audio, as a string of bytes.
\end{methoddesc}

\begin{methoddesc}[AU_read]{rewind}{}
Rewind the file pointer to the beginning of the audio stream.
\end{methoddesc}

The following two methods define a term ``position'' which is compatible
between them, and is otherwise implementation dependent.

\begin{methoddesc}[AU_read]{setpos}{pos}
Set the file pointer to the specified position.  Only values returned
from \method{tell()} should be used for \var{pos}.
\end{methoddesc}

\begin{methoddesc}[AU_read]{tell}{}
Return current file pointer position.  Note that the returned value
has nothing to do with the actual position in the file.
\end{methoddesc}

The following two functions are defined for compatibility with the 
\refmodule{aifc}, and don't do anything interesting.

\begin{methoddesc}[AU_read]{getmarkers}{}
Returns \code{None}.
\end{methoddesc}

\begin{methoddesc}[AU_read]{getmark}{id}
Raise an error.
\end{methoddesc}


\subsection{AU_write Objects \label{au-write-objects}}

AU_write objects, as returned by \function{open()} above, have the
following methods:

\begin{methoddesc}[AU_write]{setnchannels}{n}
Set the number of channels.
\end{methoddesc}

\begin{methoddesc}[AU_write]{setsampwidth}{n}
Set the sample width (in bytes.)
\end{methoddesc}

\begin{methoddesc}[AU_write]{setframerate}{n}
Set the frame rate.
\end{methoddesc}

\begin{methoddesc}[AU_write]{setnframes}{n}
Set the number of frames. This can be later changed, when and if more 
frames are written.
\end{methoddesc}


\begin{methoddesc}[AU_write]{setcomptype}{type, name}
Set the compression type and description.
Only \code{'NONE'} and \code{'ULAW'} are supported on output.
\end{methoddesc}

\begin{methoddesc}[AU_write]{setparams}{tuple}
The \var{tuple} should be \code{(\var{nchannels}, \var{sampwidth},
\var{framerate}, \var{nframes}, \var{comptype}, \var{compname})}, with
values valid for the \method{set*()} methods.  Set all parameters.
\end{methoddesc}

\begin{methoddesc}[AU_write]{tell}{}
Return current position in the file, with the same disclaimer for
the \method{AU_read.tell()} and \method{AU_read.setpos()} methods.
\end{methoddesc}

\begin{methoddesc}[AU_write]{writeframesraw}{data}
Write audio frames, without correcting \var{nframes}.
\end{methoddesc}

\begin{methoddesc}[AU_write]{writeframes}{data}
Write audio frames and make sure \var{nframes} is correct.
\end{methoddesc}

\begin{methoddesc}[AU_write]{close}{}
Make sure \var{nframes} is correct, and close the file.

This method is called upon deletion.
\end{methoddesc}

Note that it is invalid to set any parameters after calling 
\method{writeframes()} or \method{writeframesraw()}. 

% Documentations stolen and LaTeX'ed from comments in file.
\section{\module{wave} ---
         Read and write WAV files}

\declaremodule{standard}{wave}
\sectionauthor{Moshe Zadka}{moshez@zadka.site.co.il}
\modulesynopsis{Provide an interface to the WAV sound format.}

The \module{wave} module provides a convenient interface to the WAV sound
format. It does not support compression/decompression, but it does support
mono/stereo.

The \module{wave} module defines the following function and exception:

\begin{funcdesc}{open}{file\optional{, mode}}
If \var{file} is a string, open the file by that name, other treat it
as a seekable file-like object. \var{mode} can be any of
\begin{description}
        \item[\code{'r'}, \code{'rb'}] Read only mode.
        \item[\code{'w'}, \code{'wb'}] Write only mode.
\end{description}
Note that it does not allow read/write WAV files.

A \var{mode} of \code{'r'} or \code{'rb'} returns a \class{Wave_read}
object, while a \var{mode} of \code{'w'} or \code{'wb'} returns
a \class{Wave_write} object.  If \var{mode} is omitted and a file-like 
object is passed as \var{file}, \code{\var{file}.mode} is used as the
default value for \var{mode} (the \character{b} flag is still added if 
necessary).
\end{funcdesc}

\begin{funcdesc}{openfp}{file, mode}
A synonym for \function{open()}, maintained for backwards compatibility.
\end{funcdesc}

\begin{excdesc}{Error}
An error raised when something is impossible because it violates the
WAV specification or hits an implementation deficiency.
\end{excdesc}


\subsection{Wave_read Objects \label{Wave-read-objects}}

Wave_read objects, as returned by \function{open()}, have the
following methods:

\begin{methoddesc}[Wave_read]{close}{}
Close the stream, and make the instance unusable. This is
called automatically on object collection.
\end{methoddesc}

\begin{methoddesc}[Wave_read]{getnchannels}{}
Returns number of audio channels (\code{1} for mono, \code{2} for
stereo).
\end{methoddesc}

\begin{methoddesc}[Wave_read]{getsampwidth}{}
Returns sample width in bytes.
\end{methoddesc}

\begin{methoddesc}[Wave_read]{getframerate}{}
Returns sampling frequency.
\end{methoddesc}

\begin{methoddesc}[Wave_read]{getnframes}{}
Returns number of audio frames.
\end{methoddesc}

\begin{methoddesc}[Wave_read]{getcomptype}{}
Returns compression type (\code{'NONE'} is the only supported type).
\end{methoddesc}

\begin{methoddesc}[Wave_read]{getcompname}{}
Human-readable version of \method{getcomptype()}.
Usually \code{'not compressed'} parallels \code{'NONE'}.
\end{methoddesc}

\begin{methoddesc}[Wave_read]{getparams}{}
Returns a tuple
\code{(\var{nchannels}, \var{sampwidth}, \var{framerate},
\var{nframes}, \var{comptype}, \var{compname})}, equivalent to output
of the \method{get*()} methods.
\end{methoddesc}

\begin{methoddesc}[Wave_read]{readframes}{n}
Reads and returns at most \var{n} frames of audio, as a string of bytes.
\end{methoddesc}

\begin{methoddesc}[Wave_read]{rewind}{}
Rewind the file pointer to the beginning of the audio stream.
\end{methoddesc}

The following two methods are defined for compatibility with the
\refmodule{aifc} module, and don't do anything interesting.

\begin{methoddesc}[Wave_read]{getmarkers}{}
Returns \code{None}.
\end{methoddesc}

\begin{methoddesc}[Wave_read]{getmark}{id}
Raise an error.
\end{methoddesc}

The following two methods define a term ``position'' which is compatible
between them, and is otherwise implementation dependent.

\begin{methoddesc}[Wave_read]{setpos}{pos}
Set the file pointer to the specified position.
\end{methoddesc}

\begin{methoddesc}[Wave_read]{tell}{}
Return current file pointer position.
\end{methoddesc}


\subsection{Wave_write Objects \label{Wave-write-objects}}

Wave_write objects, as returned by \function{open()}, have the
following methods:

\begin{methoddesc}[Wave_write]{close}{}
Make sure \var{nframes} is correct, and close the file.
This method is called upon deletion.
\end{methoddesc}

\begin{methoddesc}[Wave_write]{setnchannels}{n}
Set the number of channels.
\end{methoddesc}

\begin{methoddesc}[Wave_write]{setsampwidth}{n}
Set the sample width to \var{n} bytes.
\end{methoddesc}

\begin{methoddesc}[Wave_write]{setframerate}{n}
Set the frame rate to \var{n}.
\end{methoddesc}

\begin{methoddesc}[Wave_write]{setnframes}{n}
Set the number of frames to \var{n}. This will be changed later if
more frames are written.
\end{methoddesc}

\begin{methoddesc}[Wave_write]{setcomptype}{type, name}
Set the compression type and description.
\end{methoddesc}

\begin{methoddesc}[Wave_write]{setparams}{tuple}
The \var{tuple} should be \code{(\var{nchannels}, \var{sampwidth},
\var{framerate}, \var{nframes}, \var{comptype}, \var{compname})}, with
values valid for the \method{set*()} methods.  Sets all parameters.
\end{methoddesc}

\begin{methoddesc}[Wave_write]{tell}{}
Return current position in the file, with the same disclaimer for
the \method{Wave_read.tell()} and \method{Wave_read.setpos()}
methods.
\end{methoddesc}

\begin{methoddesc}[Wave_write]{writeframesraw}{data}
Write audio frames, without correcting \var{nframes}.
\end{methoddesc}

\begin{methoddesc}[Wave_write]{writeframes}{data}
Write audio frames and make sure \var{nframes} is correct.
\end{methoddesc}

Note that it is invalid to set any parameters after calling
\method{writeframes()} or \method{writeframesraw()}, and any attempt
to do so will raise \exception{wave.Error}.

\section{\module{chunk} ---
	 Read IFF chunked data}

\declaremodule{standard}{chunk}
\modulesynopsis{Module to read IFF chunks.}
\moduleauthor{Sjoerd Mullender}{sjoerd@acm.org}
\sectionauthor{Sjoerd Mullender}{sjoerd@acm.org}



This module provides an interface for reading files that use EA IFF 85
chunks.\footnote{``EA IFF 85'' Standard for Interchange Format Files,
Jerry Morrison, Electronic Arts, January 1985.}  This format is used
in at least the Audio\index{Audio Interchange File
Format}\index{AIFF}\index{AIFF-C} Interchange File Format
(AIFF/AIFF-C), the Real\index{Real Media File Format} Media File
Format\index{RMFF} (RMFF), and the
Tagged\index{Tagged Image File Format} Image File Format\index{TIFF}
(TIFF).

A chunk has the following structure:

\begin{tableiii}{c|c|l}{textrm}{Offset}{Length}{Contents}
  \lineiii{0}{4}{Chunk ID}
  \lineiii{4}{4}{Size of chunk in big-endian byte order, including the 
                 header}
  \lineiii{8}{\var{n}}{Data bytes, where \var{n} is the size given in
                       the preceeding field}
  \lineiii{8 + \var{n}}{0 or 1}{Pad byte needed if \var{n} is odd and
                                chunk alignment is used}
\end{tableiii}

The ID is a 4-byte string which identifies the type of chunk.

The size field (a 32-bit value, encoded using big-endian byte order)
gives the size of the whole chunk, including the 8-byte header.

Usually an IFF-type file consists of one or more chunks.  The proposed
usage of the \class{Chunk} class defined here is to instantiate an
instance at the start of each chunk and read from the instance until
it reaches the end, after which a new instance can be instantiated.
At the end of the file, creating a new instance will fail with a
\exception{EOFError} exception.

\begin{classdesc}{Chunk}{file\optional{, align}}
Class which represents a chunk.  The \var{file} argument is expected
to be a file-like object.  An instance of this class is specifically
allowed.  The only method that is needed is \method{read()}.  If the
methods \method{seek()} and \method{tell()} are present and don't
raise an exception, they are also used.  If these methods are present
and raise an exception, they are expected to not have altered the
object.  If the optional argument \var{align} is true, chunks are
assumed to be aligned on 2-byte boundaries.  If \var{align} is
false, no alignment is assumed.  The default value is true.
\end{classdesc}

A \class{Chunk} object supports the following methods:

\begin{methoddesc}{getname}{}
Returns the name (ID) of the chunk.  This is the first 4 bytes of the
chunk.
\end{methoddesc}

\begin{methoddesc}{close}{}
Close and skip to the end of the chunk.  This does not close the
underlying file.
\end{methoddesc}

The remaining methods will raise \exception{IOError} if called after
the \method{close()} method has been called.

\begin{methoddesc}{isatty}{}
Returns \code{0}.
\end{methoddesc}

\begin{methoddesc}{seek}{pos\optional{, whence}}
Set the chunk's current position.  The \var{whence} argument is
optional and defaults to \code{0} (absolute file positioning); other
values are \code{1} (seek relative to the current position) and
\code{2} (seek relative to the file's end).  There is no return value.
If the underlying file does not allow seek, only forward seeks are
allowed.
\end{methoddesc}

\begin{methoddesc}{tell}{}
Return the current position into the chunk.
\end{methoddesc}

\begin{methoddesc}{read}{\optional{size}}
Read at most \var{size} bytes from the chunk (less if the read hits
the end of the chunk before obtaining \var{size} bytes).  If the
\var{size} argument is negative or omitted, read all data until the
end of the chunk.  The bytes are returned as a string object.  An
empty string is returned when the end of the chunk is encountered
immediately.
\end{methoddesc}

\begin{methoddesc}{skip}{}
Skip to the end of the chunk.  All further calls to \method{read()}
for the chunk will return \code{''}.  If you are not interested in the
contents of the chunk, this method should be called so that the file
points to the start of the next chunk.
\end{methoddesc}

\section{\module{colorsys} ---
         Conversions between color systems}

\declaremodule{standard}{colorsys}
\modulesynopsis{Conversion functions between RGB and other color systems.}
\sectionauthor{David Ascher}{da@python.net}

The \module{colorsys} module defines bidirectional conversions of
color values between colors expressed in the RGB (Red Green Blue)
color space used in computer monitors and three other coordinate
systems: YIQ, HLS (Hue Lightness Saturation) and HSV (Hue Saturation
Value).  Coordinates in all of these color spaces are floating point
values.  In the YIQ space, the Y coordinate is between 0 and 1, but
the I and Q coordinates can be positive or negative.  In all other
spaces, the coordinates are all between 0 and 1.

More information about color spaces can be found at 
\url{http://www.poynton.com/ColorFAQ.html}.

The \module{colorsys} module defines the following functions:

\begin{funcdesc}{rgb_to_yiq}{r, g, b}
Convert the color from RGB coordinates to YIQ coordinates.
\end{funcdesc}

\begin{funcdesc}{yiq_to_rgb}{y, i, q}
Convert the color from YIQ coordinates to RGB coordinates.
\end{funcdesc}

\begin{funcdesc}{rgb_to_hls}{r, g, b}
Convert the color from RGB coordinates to HLS coordinates.
\end{funcdesc}

\begin{funcdesc}{hls_to_rgb}{h, l, s}
Convert the color from HLS coordinates to RGB coordinates.
\end{funcdesc}

\begin{funcdesc}{rgb_to_hsv}{r, g, b}
Convert the color from RGB coordinates to HSV coordinates.
\end{funcdesc}

\begin{funcdesc}{hsv_to_rgb}{h, s, v}
Convert the color from HSV coordinates to RGB coordinates.
\end{funcdesc}

Example:

\begin{verbatim}
>>> import colorsys
>>> colorsys.rgb_to_hsv(.3, .4, .2)
(0.25, 0.5, 0.4)
>>> colorsys.hsv_to_rgb(0.25, 0.5, 0.4)
(0.3, 0.4, 0.2)
\end{verbatim}

\section{\module{rgbimg} ---
         Read and write image files in ``SGI RGB'' format.}
\declaremodule{builtin}{rgbimg}

\modulesynopsis{Read and write image files in ``SGI RGB'' format (the module is
\emph{not} SGI specific though!).}


The \module{rgbimg} module allows Python programs to access SGI imglib image
files (also known as \file{.rgb} files).  The module is far from
complete, but is provided anyway since the functionality that there is
enough in some cases.  Currently, colormap files are not supported.

The module defines the following variables and functions:

\begin{excdesc}{error}
This exception is raised on all errors, such as unsupported file type, etc.
\end{excdesc}

\begin{funcdesc}{sizeofimage}{file}
This function returns a tuple \code{(\var{x}, \var{y})} where
\var{x} and \var{y} are the size of the image in pixels.
Only 4 byte RGBA pixels, 3 byte RGB pixels, and 1 byte greyscale pixels
are currently supported.
\end{funcdesc}

\begin{funcdesc}{longimagedata}{file}
This function reads and decodes the image on the specified file, and
returns it as a Python string. The string has 4 byte RGBA pixels.
The bottom left pixel is the first in
the string. This format is suitable to pass to \function{gl.lrectwrite()},
for instance.
\end{funcdesc}

\begin{funcdesc}{longstoimage}{data, x, y, z, file}
This function writes the RGBA data in \var{data} to image
file \var{file}. \var{x} and \var{y} give the size of the image.
\var{z} is 1 if the saved image should be 1 byte greyscale, 3 if the
saved image should be 3 byte RGB data, or 4 if the saved images should
be 4 byte RGBA data.  The input data always contains 4 bytes per pixel.
These are the formats returned by \function{gl.lrectread()}.
\end{funcdesc}

\begin{funcdesc}{ttob}{flag}
This function sets a global flag which defines whether the scan lines
of the image are read or written from bottom to top (flag is zero,
compatible with SGI GL) or from top to bottom(flag is one,
compatible with X).  The default is zero.
\end{funcdesc}

\section{Standard module \sectcode{imghdr}}
\label{module-imghdr}
\stmodindex{imghdr}

The \code{imghdr} module determines the type of image contained in a
file or byte stream.

The \code{imghdr} module defines the following function:

\renewcommand{\indexsubitem}{(in module imghdr)}

\begin{funcdesc}{what}{filename\optional{\, h}}
Tests the image data contained in the file named by \var{filename},
and returns a string describing the image type.  If optional \var{h}
is provided, the \var{filename} is ignored and \var{h} is assumed to
contain the byte stream to test.
\end{funcdesc}

The following image types are recognized, as listed below with the
return value from \code{what}:

\begin{enumerate}
\item[``rgb''] SGI ImgLib Files

\item[``gif''] GIF 87a and 89a Files

\item[``pbm''] Portable Bitmap Files

\item[``pgm''] Portable Graymap Files

\item[``ppm''] Portable Pixmap Files

\item[``tiff''] TIFF Files

\item[``rast''] Sun Raster Files

\item[``xbm''] X Bitmap Files

\item[``jpeg''] JPEG data in JIFF format
\end{enumerate}

You can extend the list of file types \code{imghdr} can recognize by
appending to this variable:

\begin{datadesc}{tests}
A list of functions performing the individual tests.  Each function
takes two arguments: the byte-stream and an open file-like object.
When \code{what()} is called with a byte-stream, the file-like
object will be \code{None}.

The test function should return a string describing the image type if
the test succeeded, or \code{None} if it failed.
\end{datadesc}

Example:

\bcode\begin{verbatim}
>>> import imghdr
>>> imghdr.what('/tmp/bass.gif')
'gif'
\end{verbatim}\ecode

\section{\module{sndhdr} ---
         Determine type of sound file.}

\declaremodule{standard}{sndhdr}
\modulesynopsis{Determine type of a sound file.}
\sectionauthor{Fred L. Drake, Jr.}{fdrake@acm.org}
% Based on comments in the module source file.


The \module{sndhdr} provides utility functions which attempt to
determine the type of sound data which is in a file.  When these
functions are able to determine what type of sound data is stored in a
file, they return a tuple \code{(\var{type}, \var{sampling_rate},
\var{channels}, \var{frames}, \var{bits_per_sample})}.  The value for
\var{type} indicates the data type and will be one of the strings
\code{'aifc'}, \code{'aiff'}, \code{'au'}, \code{'hcom'},
\code{'sndr'}, \code{'sndt'}, \code{'voc'}, \code{'wav'},
\code{'8svx'}, \code{'sb'}, \code{'ub'}, or \code{'ul'}.  The
\var{sampling_rate} will be either the actual value or \code{0} if
unknown or difficult to decode.  Similarly, \var{channels} will be
either the number of channels or \code{0} if it cannot be determined
or if the value is difficult to decode.  The value for \var{frames}
will be either the number of frames or \code{-1}.  The last item in
the tuple, \var{bits_per_sample}, will either be the sample size in
bits or \code{'A'} for A-LAW\index{A-LAW} or \code{'U'} for
u-LAW\index{u-LAW}.


\begin{funcdesc}{what}{filename}
  Determines the type of sound data stored in the file \var{filename}
  using \function{whathdr()}.  If not successful, \function{whatraw()} 
  is used.  If neither attempt succeeds, returns \code{None},
  otherwise it returns a tuple as described above.
\end{funcdesc}


\begin{funcdesc}{whathdr}{filename}
  Determines the type of sound data stored in a file based on the file 
  header.  The name of the file is given by \var{filename}.  This
  function returns a tuple as described above on success, or
  \code{None}.
\end{funcdesc}


\begin{funcdesc}{whatraw}{filename}
  Determines the type of raw sound data stored in a file without a
  header.  The name of the file is given by \var{filename}.  This
  function returns a tuple as described above on success, or
  \code{None}.

  This requires the \program{whatsound} program to work.
\end{funcdesc}


\chapter{Cryptographic Services}
\label{crypto}
\index{cryptography}

The modules described in this chapter implement various algorithms of
a cryptographic nature.  They are available at the discretion of the
installation.  Here's an overview:

\localmoduletable

Hardcore cypherpunks will probably find the cryptographic modules
written by Andrew Kuchling of further interest; the package adds
built-in modules for DES and IDEA encryption, provides a Python module
for reading and decrypting PGP files, and then some.  These modules
are not distributed with Python but available separately.  See the URL
\url{http://starship.python.net/crew/amk/python/crypto.html} or
send email to \email{amk1@bigfoot.com} for more information.
\index{PGP}
\index{Pretty Good Privacy}
\indexii{DES}{cipher}
\indexii{IDEA}{cipher}
\index{cryptography}
\index{Kuchling, Andrew}
		% Cryptographic Services
\section{Built-in Module \sectcode{md5}}
\bimodindex{md5}

This module implements the interface to RSA's MD5 message digest
algorithm (see also Internet RFC 1321).  Its use is quite
straightforward:\ use the \code{md5.new()} to create an md5 object.
You can now feed this object with arbitrary strings using the
\code{update()} method, and at any point you can ask it for the
\dfn{digest} (a strong kind of 128-bit checksum,
a.k.a. ``fingerprint'') of the contatenation of the strings fed to it
so far using the \code{digest()} method.

For example, to obtain the digest of the string {\tt"Nobody inspects
the spammish repetition"}:

\bcode\begin{verbatim}
>>> import md5
>>> m = md5.new()
>>> m.update("Nobody inspects")
>>> m.update(" the spammish repetition")
>>> m.digest()
'\273d\234\203\335\036\245\311\331\336\311\241\215\360\377\351'
\end{verbatim}\ecode

More condensed:

\bcode\begin{verbatim}
>>> md5.new("Nobody inspects the spammish repetition").digest()
'\273d\234\203\335\036\245\311\331\336\311\241\215\360\377\351'
\end{verbatim}\ecode

\renewcommand{\indexsubitem}{(in module md5)}

\begin{funcdesc}{new}{\optional{arg}}
Return a new md5 object.  If \var{arg} is present, the method call
\code{update(\var{arg})} is made.
\end{funcdesc}

\begin{funcdesc}{md5}{\optional{arg}}
For backward compatibility reasons, this is an alternative name for the
\code{new()} function.
\end{funcdesc}

An md5 object has the following methods:

\renewcommand{\indexsubitem}{(md5 method)}
\begin{funcdesc}{update}{arg}
Update the md5 object with the string \var{arg}.  Repeated calls are
equivalent to a single call with the concatenation of all the
arguments, i.e.\ \code{m.update(a); m.update(b)} is equivalent to
\code{m.update(a+b)}.
\end{funcdesc}

\begin{funcdesc}{digest}{}
Return the digest of the strings passed to the \code{update()}
method so far.  This is an 16-byte string which may contain
non-\ASCII{} characters, including null bytes.
\end{funcdesc}

\begin{funcdesc}{copy}{}
Return a copy (``clone'') of the md5 object.  This can be used to
efficiently compute the digests of strings that share a common initial
substring.
\end{funcdesc}

\section{\module{sha} ---
         SHA message digest algorithm}

\declaremodule{builtin}{sha}
\modulesynopsis{NIST's secure hash algorithm, SHA.}
\sectionauthor{Fred L. Drake, Jr.}{fdrake@acm.org}


This module implements the interface to NIST's\index{NIST} secure hash 
algorithm,\index{Secure Hash Algorithm} known as SHA.  It is used in
the same way as the \refmodule{md5} module:\ use \function{new()}
to create an sha object, then feed this object with arbitrary strings
using the \method{update()} method, and at any point you can ask it
for the \dfn{digest} of the concatenation of the strings fed to it
so far.\index{checksum!SHA}  SHA digests are 160 bits instead of
MD5's 128 bits.


\begin{funcdesc}{new}{\optional{string}}
  Return a new sha object.  If \var{string} is present, the method
  call \code{update(\var{string})} is made.
\end{funcdesc}


The following values are provided as constants in the module and as
attributes of the sha objects returned by \function{new()}:

\begin{datadesc}{blocksize}
  Size of the blocks fed into the hash function; this is always
  \code{1}.  This size is used to allow an arbitrary string to be
  hashed.
\end{datadesc}

\begin{datadesc}{digestsize}
  The size of the resulting digest in bytes.  This is always
  \code{20}.
\end{datadesc}


An sha object has the same methods as md5 objects:

\begin{methoddesc}[sha]{update}{arg}
Update the sha object with the string \var{arg}.  Repeated calls are
equivalent to a single call with the concatenation of all the
arguments: \code{m.update(a); m.update(b)} is equivalent to
\code{m.update(a+b)}.
\end{methoddesc}

\begin{methoddesc}[sha]{digest}{}
Return the digest of the strings passed to the \method{update()}
method so far.  This is a 20-byte string which may contain
non-\ASCII{} characters, including null bytes.
\end{methoddesc}

\begin{methoddesc}[sha]{hexdigest}{}
Like \method{digest()} except the digest is returned as a string of
length 40, containing only hexadecimal digits.  This may 
be used to exchange the value safely in email or other non-binary
environments.
\end{methoddesc}

\begin{methoddesc}[sha]{copy}{}
Return a copy (``clone'') of the sha object.  This can be used to
efficiently compute the digests of strings that share a common initial
substring.
\end{methoddesc}

\begin{seealso}
  \seetitle[http://csrc.nist.gov/publications/fips/fips180-1/fip180-1.txt]
    {Secure Hash Standard}
    {The Secure Hash Algorithm is defined by NIST document FIPS
     PUB 180-1:
     \citetitle[http://csrc.nist.gov/publications/fips/fips180-1/fip180-1.txt]
        {Secure Hash Standard}, published in April of 1995.  It is
     available online as plain text (at least one diagram was
     omitted) and as PDF at
     \url{http://csrc.nist.gov/publications/fips/fips180-1/fip180-1.pdf}.}

  \seetitle[http://csrc.nist.gov/encryption/tkhash.html]
           {Cryptographic Toolkit (Secure Hashing)}
           {Links from NIST to various information on secure hashing.}
\end{seealso}

\section{Built-in Module \sectcode{mpz}}
\label{module-mpz}
\bimodindex{mpz}

This is an optional module.  It is only available when Python is
configured to include it, which requires that the GNU MP software is
installed.

This module implements the interface to part of the GNU MP library,
which defines arbitrary precision integer and rational number
arithmetic routines.  Only the interfaces to the \emph{integer}
(\samp{mpz_{\rm \ldots}}) routines are provided. If not stated
otherwise, the description in the GNU MP documentation can be applied.

In general, \dfn{mpz}-numbers can be used just like other standard
Python numbers, e.g.\ you can use the built-in operators like \code{+},
\code{*}, etc., as well as the standard built-in functions like
\code{abs}, \code{int}, \ldots, \code{divmod}, \code{pow}.
\strong{Please note:} the \emph{bitwise-xor} operation has been implemented as
a bunch of \emph{and}s, \emph{invert}s and \emph{or}s, because the library
lacks an \code{mpz_xor} function, and I didn't need one.

You create an mpz-number by calling the function called \code{mpz} (see
below for an exact description). An mpz-number is printed like this:
\code{mpz(\var{value})}.

\setindexsubitem{(in module mpz)}
\begin{funcdesc}{mpz}{value}
  Create a new mpz-number. \var{value} can be an integer, a long,
  another mpz-number, or even a string. If it is a string, it is
  interpreted as an array of radix-256 digits, least significant digit
  first, resulting in a positive number. See also the \code{binary}
  method, described below.
\end{funcdesc}

A number of \emph{extra} functions are defined in this module. Non
mpz-arguments are converted to mpz-values first, and the functions
return mpz-numbers.

\begin{funcdesc}{powm}{base, exponent, modulus}
  Return \code{pow(\var{base}, \var{exponent}) \%{} \var{modulus}}. If
  \code{\var{exponent} == 0}, return \code{mpz(1)}. In contrast to the
  \C-library function, this version can handle negative exponents.
\end{funcdesc}

\begin{funcdesc}{gcd}{op1, op2}
  Return the greatest common divisor of \var{op1} and \var{op2}.
\end{funcdesc}

\begin{funcdesc}{gcdext}{a, b}
  Return a tuple \code{(\var{g}, \var{s}, \var{t})}, such that
  \code{\var{a}*\var{s} + \var{b}*\var{t} == \var{g} == gcd(\var{a}, \var{b})}.
\end{funcdesc}

\begin{funcdesc}{sqrt}{op}
  Return the square root of \var{op}. The result is rounded towards zero.
\end{funcdesc}

\begin{funcdesc}{sqrtrem}{op}
  Return a tuple \code{(\var{root}, \var{remainder})}, such that
  \code{\var{root}*\var{root} + \var{remainder} == \var{op}}.
\end{funcdesc}

\begin{funcdesc}{divm}{numerator, denominator, modulus}
  Returns a number \var{q}. such that
  \code{\var{q} * \var{denominator} \%{} \var{modulus} == \var{numerator}}.
  One could also implement this function in Python, using \code{gcdext}.
\end{funcdesc}

An mpz-number has one method:

\setindexsubitem{(mpz method)}
\begin{funcdesc}{binary}{}
  Convert this mpz-number to a binary string, where the number has been
  stored as an array of radix-256 digits, least significant digit first.

  The mpz-number must have a value greater than or equal to zero,
  otherwise a \code{ValueError}-exception will be raised.
\end{funcdesc}

\section{Built-in Module \sectcode{rotor}}
\bimodindex{rotor}

This module implements a rotor-based encryption algorithm, contributed by
Lance Ellinghouse.  The design is derived from the Enigma device, a machine
used during World War II to encipher messages.  A rotor is simply a
permutation.  For example, if the character `A' is the origin of the rotor,
then a given rotor might map `A' to `L', `B' to `Z', `C' to `G', and so on.
To encrypt, we choose several different rotors, and set the origins of the
rotors to known positions; their initial position is the ciphering key.  To
encipher a character, we permute the original character by the first rotor,
and then apply the second rotor's permutation to the result. We continue
until we've applied all the rotors; the resulting character is our
ciphertext.  We then change the origin of the final rotor by one position,
from `A' to `B'; if the final rotor has made a complete revolution, then we
rotate the next-to-last rotor by one position, and apply the same procedure
recursively.  In other words, after enciphering one character, we advance
the rotors in the same fashion as a car's odometer. Decoding works in the
same way, except we reverse the permutations and apply them in the opposite
order.
\index{Ellinghouse, Lance}
\indexii{Enigma}{cipher}

The available functions in this module are:

\renewcommand{\indexsubitem}{(in module rotor)}
\begin{funcdesc}{newrotor}{key\optional{\, numrotors}}
Return a rotor object. \var{key} is a string containing the encryption key
for the object; it can contain arbitrary binary data. The key will be used
to randomly generate the rotor permutations and their initial positions.
\var{numrotors} is the number of rotor permutations in the returned object;
if it is omitted, a default value of 6 will be used.
\end{funcdesc}

Rotor objects have the following methods:

\renewcommand{\indexsubitem}{(rotor method)}
\begin{funcdesc}{setkey}{key}
Sets the rotor's key to \var{key}.
\end{funcdesc}

\begin{funcdesc}{encrypt}{plaintext}
Reset the rotor object to its initial state and encrypt \var{plaintext},
returning a string containing the ciphertext.  The ciphertext is always the
same length as the original plaintext.
\end{funcdesc}

\begin{funcdesc}{encryptmore}{plaintext}
Encrypt \var{plaintext} without resetting the rotor object, and return a
string containing the ciphertext.
\end{funcdesc}

\begin{funcdesc}{decrypt}{ciphertext}
Reset the rotor object to its initial state and decrypt \var{ciphertext},
returning a string containing the ciphertext.  The plaintext string will
always be the same length as the ciphertext.
\end{funcdesc}

\begin{funcdesc}{decryptmore}{ciphertext}
Decrypt \var{ciphertext} without resetting the rotor object, and return a
string containing the ciphertext.
\end{funcdesc}

An example usage:
\bcode\begin{verbatim}
>>> import rotor
>>> rt = rotor.newrotor('key', 12)
>>> rt.encrypt('bar')
'\2534\363'
>>> rt.encryptmore('bar')
'\357\375$'
>>> rt.encrypt('bar')
'\2534\363'
>>> rt.decrypt('\2534\363')
'bar'
>>> rt.decryptmore('\357\375$')
'bar'
>>> rt.decrypt('\357\375$')
'l(\315'
>>> del rt
\end{verbatim}\ecode

The module's code is not an exact simulation of the original Enigma device;
it implements the rotor encryption scheme differently from the original. The
most important difference is that in the original Enigma, there were only 5
or 6 different rotors in existence, and they were applied twice to each
character; the cipher key was the order in which they were placed in the
machine.  The Python rotor module uses the supplied key to initialize a
random number generator; the rotor permutations and their initial positions
are then randomly generated.  The original device only enciphered the
letters of the alphabet, while this module can handle any 8-bit binary data;
it also produces binary output.  This module can also operate with an
arbitrary number of rotors.

The original Enigma cipher was broken in 1944. % XXX: Is this right?
The version implemented here is probably a good deal more difficult to crack
(especially if you use many rotors), but it won't be impossible for
a truly skilful and determined attacker to break the cipher.  So if you want
to keep the NSA out of your files, this rotor cipher may well be unsafe, but
for discouraging casual snooping through your files, it will probably be
just fine, and may be somewhat safer than using the Unix \file{crypt}
command.
\index{National Security Agency}\index{crypt(1)}
% XXX How were Unix commands represented in the docs?



%\chapter{Amoeba Specific Services}

\section{Built-in Module \sectcode{amoeba}}

\bimodindex{amoeba}
This module provides some object types and operations useful for
Amoeba applications.  It is only available on systems that support
Amoeba operations.  RPC errors and other Amoeba errors are reported as
the exception \code{amoeba.error = 'amoeba.error'}.

The module \code{amoeba} defines the following items:

\renewcommand{\indexsubitem}{(in module amoeba)}
\begin{funcdesc}{name_append}{path\, cap}
Stores a capability in the Amoeba directory tree.
Arguments are the pathname (a string) and the capability (a capability
object as returned by
\code{name_lookup()}).
\end{funcdesc}

\begin{funcdesc}{name_delete}{path}
Deletes a capability from the Amoeba directory tree.
Argument is the pathname.
\end{funcdesc}

\begin{funcdesc}{name_lookup}{path}
Looks up a capability.
Argument is the pathname.
Returns a
\dfn{capability}
object, to which various interesting operations apply, described below.
\end{funcdesc}

\begin{funcdesc}{name_replace}{path\, cap}
Replaces a capability in the Amoeba directory tree.
Arguments are the pathname and the new capability.
(This differs from
\code{name_append()}
in the behavior when the pathname already exists:
\code{name_append()}
finds this an error while
\code{name_replace()}
allows it, as its name suggests.)
\end{funcdesc}

\begin{datadesc}{capv}
A table representing the capability environment at the time the
interpreter was started.
(Alas, modifying this table does not affect the capability environment
of the interpreter.)
For example,
\code{amoeba.capv['ROOT']}
is the capability of your root directory, similar to
\code{getcap("ROOT")}
in C.
\end{datadesc}

\begin{excdesc}{error}
The exception raised when an Amoeba function returns an error.
The value accompanying this exception is a pair containing the numeric
error code and the corresponding string, as returned by the C function
\code{err_why()}.
\end{excdesc}

\begin{funcdesc}{timeout}{msecs}
Sets the transaction timeout, in milliseconds.
Returns the previous timeout.
Initially, the timeout is set to 2 seconds by the Python interpreter.
\end{funcdesc}

\subsection{Capability Operations}

Capabilities are written in a convenient \ASCII{} format, also used by the
Amoeba utilities
{\it c2a}(U)
and
{\it a2c}(U).
For example:

\bcode\begin{verbatim}
>>> amoeba.name_lookup('/profile/cap')
aa:1c:95:52:6a:fa/14(ff)/8e:ba:5b:8:11:1a
>>> 
\end{verbatim}\ecode
%
The following methods are defined for capability objects.

\renewcommand{\indexsubitem}{(capability method)}
\begin{funcdesc}{dir_list}{}
Returns a list of the names of the entries in an Amoeba directory.
\end{funcdesc}

\begin{funcdesc}{b_read}{offset\, maxsize}
Reads (at most)
\var{maxsize}
bytes from a bullet file at offset
\var{offset.}
The data is returned as a string.
EOF is reported as an empty string.
\end{funcdesc}

\begin{funcdesc}{b_size}{}
Returns the size of a bullet file.
\end{funcdesc}

\begin{funcdesc}{dir_append}{}
\funcline{dir_delete}{}\ 
\funcline{dir_lookup}{}\ 
\funcline{dir_replace}{}
Like the corresponding
\samp{name_}*
functions, but with a path relative to the capability.
(For paths beginning with a slash the capability is ignored, since this
is the defined semantics for Amoeba.)
\end{funcdesc}

\begin{funcdesc}{std_info}{}
Returns the standard info string of the object.
\end{funcdesc}

\begin{funcdesc}{tod_gettime}{}
Returns the time (in seconds since the Epoch, in UCT, as for POSIX) from
a time server.
\end{funcdesc}

\begin{funcdesc}{tod_settime}{t}
Sets the time kept by a time server.
\end{funcdesc}
		% AMOEBA ONLY

%\chapter{Standard Windowing Interface}

The modules in this chapter are available only on those systems where
the STDWIN library is available.  STDWIN runs on \UNIX{} under X11 and
on the Macintosh.  See CWI report CS-R8817.

\strong{Warning:} Using STDWIN is not recommended for new
applications.  It has never been ported to Microsoft Windows or
Windows NT, and for X11 or the Macintosh it lacks important
functionality --- in particular, it has no tools for the construction
of dialogs.  For most platforms, alternative, native solutions exist
(though none are currently documented in this manual): Tkinter for
\UNIX{} under X11, native Xt with Motif or Athena widgets for \UNIX{}
under X11, Win32 for Windows and Windows NT, and a collection of
native toolkit interfaces for the Macintosh.

\section{\module{stdwin} ---
         Platform-independent GUI System}
\declaremodule{builtin}{stdwin}

\modulesynopsis{Older GUI system for X11 and Macintosh}


This module defines several new object types and functions that
provide access to the functionality of STDWIN.

On \UNIX{} running X11, it can only be used if the \envvar{DISPLAY}
environment variable is set or an explicit \samp{-display
\var{displayname}} argument is passed to the Python interpreter.

Functions have names that usually resemble their C STDWIN counterparts
with the initial `w' dropped.
Points are represented by pairs of integers; rectangles
by pairs of points.
For a complete description of STDWIN please refer to the documentation
of STDWIN for C programmers (aforementioned CWI report).

\subsection{Functions Defined in Module \module{stdwin}}
\nodename{STDWIN Functions}

The following functions are defined in the \module{stdwin} module:

\begin{funcdesc}{open}{title}
Open a new window whose initial title is given by the string argument.
Return a window object; window object methods are described below.%
\footnote{The Python version of STDWIN does not support draw procedures; all
	drawing requests are reported as draw events.}
\end{funcdesc}

\begin{funcdesc}{getevent}{}
Wait for and return the next event.
An event is returned as a triple: the first element is the event
type, a small integer; the second element is the window object to which
the event applies, or
\code{None}
if it applies to no window in particular;
the third element is type-dependent.
Names for event types and command codes are defined in the standard
module \module{stdwinevent}.
\end{funcdesc}

\begin{funcdesc}{pollevent}{}
Return the next event, if one is immediately available.
If no event is available, return \code{()}.
\end{funcdesc}

\begin{funcdesc}{getactive}{}
Return the window that is currently active, or \code{None} if no
window is currently active.  (This can be emulated by monitoring
WE_ACTIVATE and WE_DEACTIVATE events.)
\end{funcdesc}

\begin{funcdesc}{listfontnames}{pattern}
Return the list of font names in the system that match the pattern (a
string).  The pattern should normally be \code{'*'}; returns all
available fonts.  If the underlying window system is X11, other
patterns follow the standard X11 font selection syntax (as used e.g.
in resource definitions), i.e. the wildcard character \code{'*'}
matches any sequence of characters (including none) and \code{'?'}
matches any single character.
On the Macintosh this function currently returns an empty list.
\end{funcdesc}

\begin{funcdesc}{setdefscrollbars}{hflag, vflag}
Set the flags controlling whether subsequently opened windows will
have horizontal and/or vertical scroll bars.
\end{funcdesc}

\begin{funcdesc}{setdefwinpos}{h, v}
Set the default window position for windows opened subsequently.
\end{funcdesc}

\begin{funcdesc}{setdefwinsize}{width, height}
Set the default window size for windows opened subsequently.
\end{funcdesc}

\begin{funcdesc}{getdefscrollbars}{}
Return the flags controlling whether subsequently opened windows will
have horizontal and/or vertical scroll bars.
\end{funcdesc}

\begin{funcdesc}{getdefwinpos}{}
Return the default window position for windows opened subsequently.
\end{funcdesc}

\begin{funcdesc}{getdefwinsize}{}
Return the default window size for windows opened subsequently.
\end{funcdesc}

\begin{funcdesc}{getscrsize}{}
Return the screen size in pixels.
\end{funcdesc}

\begin{funcdesc}{getscrmm}{}
Return the screen size in millimeters.
\end{funcdesc}

\begin{funcdesc}{fetchcolor}{colorname}
Return the pixel value corresponding to the given color name.
Return the default foreground color for unknown color names.
Hint: the following code tests whether you are on a machine that
supports more than two colors:
\begin{verbatim}
if stdwin.fetchcolor('black') <> \
          stdwin.fetchcolor('red') <> \
          stdwin.fetchcolor('white'):
    print 'color machine'
else:
    print 'monochrome machine'
\end{verbatim}
\end{funcdesc}

\begin{funcdesc}{setfgcolor}{pixel}
Set the default foreground color.
This will become the default foreground color of windows opened
subsequently, including dialogs.
\end{funcdesc}

\begin{funcdesc}{setbgcolor}{pixel}
Set the default background color.
This will become the default background color of windows opened
subsequently, including dialogs.
\end{funcdesc}

\begin{funcdesc}{getfgcolor}{}
Return the pixel value of the current default foreground color.
\end{funcdesc}

\begin{funcdesc}{getbgcolor}{}
Return the pixel value of the current default background color.
\end{funcdesc}

\begin{funcdesc}{setfont}{fontname}
Set the current default font.
This will become the default font for windows opened subsequently,
and is also used by the text measuring functions \function{textwidth()},
\function{textbreak()}, \function{lineheight()} and
\function{baseline()} below.  This accepts two more optional
parameters, size and style:  Size is the font size (in `points').
Style is a single character specifying the style, as follows:
\code{'b'} = bold,
\code{'i'} = italic,
\code{'o'} = bold + italic,
\code{'u'} = underline;
default style is roman.
Size and style are ignored under X11 but used on the Macintosh.
(Sorry for all this complexity --- a more uniform interface is being designed.)
\end{funcdesc}

\begin{funcdesc}{menucreate}{title}
Create a menu object referring to a global menu (a menu that appears in
all windows).
Methods of menu objects are described below.
Note: normally, menus are created locally; see the window method
\method{menucreate()} below.
\strong{Warning:} the menu only appears in a window as long as the object
returned by this call exists.
\end{funcdesc}

\begin{funcdesc}{newbitmap}{width, height}
Create a new bitmap object of the given dimensions.
Methods of bitmap objects are described below.
Not available on the Macintosh.
\end{funcdesc}

\begin{funcdesc}{fleep}{}
Cause a beep or bell (or perhaps a `visual bell' or flash, hence the
name).
\end{funcdesc}

\begin{funcdesc}{message}{string}
Display a dialog box containing the string.
The user must click OK before the function returns.
\end{funcdesc}

\begin{funcdesc}{askync}{prompt, default}
Display a dialog that prompts the user to answer a question with yes or
no.  Return 0 for no, 1 for yes.  If the user hits the Return key, the
default (which must be 0 or 1) is returned.  If the user cancels the
dialog, \exception{KeyboardInterrupt} is raised.
\end{funcdesc}

\begin{funcdesc}{askstr}{prompt, default}
Display a dialog that prompts the user for a string.
If the user hits the Return key, the default string is returned.
If the user cancels the dialog, \exception{KeyboardInterrupt} is
raised.
\end{funcdesc}

\begin{funcdesc}{askfile}{prompt, default, new}
Ask the user to specify a filename.  If \var{new} is zero it must be
an existing file; otherwise, it must be a new file.  If the user
cancels the dialog, \exception{KeyboardInterrupt} is raised.
\end{funcdesc}

\begin{funcdesc}{setcutbuffer}{i, string}
Store the string in the system's cut buffer number \var{i}, where it
can be found (for pasting) by other applications.  On X11, there are 8
cut buffers (numbered 0..7).  Cut buffer number 0 is the `clipboard'
on the Macintosh.
\end{funcdesc}

\begin{funcdesc}{getcutbuffer}{i}
Return the contents of the system's cut buffer number \var{i}.
\end{funcdesc}

\begin{funcdesc}{rotatecutbuffers}{n}
On X11, rotate the 8 cut buffers by \var{n}.  Ignored on the
Macintosh.
\end{funcdesc}

\begin{funcdesc}{getselection}{i}
Return X11 selection number \var{i.}  Selections are not cut buffers.
Selection numbers are defined in module \module{stdwinevents}.
Selection \constant{WS_PRIMARY} is the \dfn{primary} selection (used
by \program{xterm}, for instance); selection \constant{WS_SECONDARY}
is the \dfn{secondary} selection; selection \constant{WS_CLIPBOARD} is
the \dfn{clipboard} selection (used by \program{xclipboard}).  On the
Macintosh, this always returns an empty string.
\end{funcdesc}

\begin{funcdesc}{resetselection}{i}
Reset selection number \var{i}, if this process owns it.  (See window
method \method{setselection()}).
\end{funcdesc}

\begin{funcdesc}{baseline}{}
Return the baseline of the current font (defined by STDWIN as the
vertical distance between the baseline and the top of the
characters).
\end{funcdesc}

\begin{funcdesc}{lineheight}{}
Return the total line height of the current font.
\end{funcdesc}

\begin{funcdesc}{textbreak}{str, width}
Return the number of characters of the string that fit into a space of
\var{width}
bits wide when drawn in the curent font.
\end{funcdesc}

\begin{funcdesc}{textwidth}{str}
Return the width in bits of the string when drawn in the current font.
\end{funcdesc}

\begin{funcdesc}{connectionnumber}{}
\funcline{fileno}{}
(X11 under \UNIX{} only) Return the ``connection number'' used by the
underlying X11 implementation.  (This is normally the file number of
the socket.)  Both functions return the same value;
\method{connectionnumber()} is named after the corresponding function in
X11 and STDWIN, while \method{fileno()} makes it possible to use the
\module{stdwin} module as a ``file'' object parameter to
\function{select.select()}.  Note that if \constant{select()} implies that
input is possible on \module{stdwin}, this does not guarantee that an
event is ready --- it may be some internal communication going on
between the X server and the client library.  Thus, you should call
\function{stdwin.pollevent()} until it returns \code{None} to check for
events if you don't want your program to block.  Because of internal
buffering in X11, it is also possible that \function{stdwin.pollevent()}
returns an event while \function{select()} does not find \module{stdwin} to
be ready, so you should read any pending events with
\function{stdwin.pollevent()} until it returns \code{None} before entering
a blocking \function{select()} call.
\withsubitem{(in module select)}{\ttindex{select()}}
\end{funcdesc}

\subsection{Window Objects}
\nodename{STDWIN Window Objects}

Window objects are created by \function{stdwin.open()}.  They are closed
by their \method{close()} method or when they are garbage-collected.
Window objects have the following methods:

\begin{methoddesc}[window]{begindrawing}{}
Return a drawing object, whose methods (described below) allow drawing
in the window.
\end{methoddesc}

\begin{methoddesc}[window]{change}{rect}
Invalidate the given rectangle; this may cause a draw event.
\end{methoddesc}

\begin{methoddesc}[window]{gettitle}{}
Returns the window's title string.
\end{methoddesc}

\begin{methoddesc}[window]{getdocsize}{}
\begin{sloppypar}
Return a pair of integers giving the size of the document as set by
\method{setdocsize()}.
\end{sloppypar}
\end{methoddesc}

\begin{methoddesc}[window]{getorigin}{}
Return a pair of integers giving the origin of the window with respect
to the document.
\end{methoddesc}

\begin{methoddesc}[window]{gettitle}{}
Return the window's title string.
\end{methoddesc}

\begin{methoddesc}[window]{getwinsize}{}
Return a pair of integers giving the size of the window.
\end{methoddesc}

\begin{methoddesc}[window]{getwinpos}{}
Return a pair of integers giving the position of the window's upper
left corner (relative to the upper left corner of the screen).
\end{methoddesc}

\begin{methoddesc}[window]{menucreate}{title}
Create a menu object referring to a local menu (a menu that appears
only in this window).
Methods of menu objects are described below.
\strong{Warning:} the menu only appears as long as the object
returned by this call exists.
\end{methoddesc}

\begin{methoddesc}[window]{scroll}{rect, point}
Scroll the given rectangle by the vector given by the point.
\end{methoddesc}

\begin{methoddesc}[window]{setdocsize}{point}
Set the size of the drawing document.
\end{methoddesc}

\begin{methoddesc}[window]{setorigin}{point}
Move the origin of the window (its upper left corner)
to the given point in the document.
\end{methoddesc}

\begin{methoddesc}[window]{setselection}{i, str}
Attempt to set X11 selection number \var{i} to the string \var{str}.
(See \module{stdwin} function \function{getselection()} for the
meaning of \var{i}.)  Return true if it succeeds.
If  succeeds, the window ``owns'' the selection until
(a) another application takes ownership of the selection; or
(b) the window is deleted; or
(c) the application clears ownership by calling
\function{stdwin.resetselection(\var{i})}.  When another application
takes ownership of the selection, a \constant{WE_LOST_SEL} event is
received for no particular window and with the selection number as
detail.  Ignored on the Macintosh.
\end{methoddesc}

\begin{methoddesc}[window]{settimer}{dsecs}
Schedule a timer event for the window in \code{\var{dsecs}/10}
seconds.
\end{methoddesc}

\begin{methoddesc}[window]{settitle}{title}
Set the window's title string.
\end{methoddesc}

\begin{methoddesc}[window]{setwincursor}{name}
\begin{sloppypar}
Set the window cursor to a cursor of the given name.  It raises
\exception{RuntimeError} if no cursor of the given name exists.
Suitable names include
\code{'ibeam'},
\code{'arrow'},
\code{'cross'},
\code{'watch'}
and
\code{'plus'}.
On X11, there are many more (see \code{<X11/cursorfont.h>}).
\end{sloppypar}
\end{methoddesc}

\begin{methoddesc}[window]{setwinpos}{h, v}
Set the the position of the window's upper left corner (relative to
the upper left corner of the screen).
\end{methoddesc}

\begin{methoddesc}[window]{setwinsize}{width, height}
Set the window's size.
\end{methoddesc}

\begin{methoddesc}[window]{show}{rect}
Try to ensure that the given rectangle of the document is visible in
the window.
\end{methoddesc}

\begin{methoddesc}[window]{textcreate}{rect}
Create a text-edit object in the document at the given rectangle.
Methods of text-edit objects are described below.
\end{methoddesc}

\begin{methoddesc}[window]{setactive}{}
Attempt to make this window the active window.  If successful, this
will generate a WE_ACTIVATE event (and a WE_DEACTIVATE event in case
another window in this application became inactive).
\end{methoddesc}

\begin{methoddesc}[window]{close}{}
Discard the window object.  It should not be used again.
\end{methoddesc}

\subsection{Drawing Objects}

Drawing objects are created exclusively by the window method
\method{begindrawing()}.  Only one drawing object can exist at any
given time; the drawing object must be deleted to finish drawing.  No
drawing object may exist when \function{stdwin.getevent()} is called.
Drawing objects have the following methods:

\begin{methoddesc}[drawing]{box}{rect}
Draw a box just inside a rectangle.
\end{methoddesc}

\begin{methoddesc}[drawing]{circle}{center, radius}
Draw a circle with given center point and radius.
\end{methoddesc}

\begin{methoddesc}[drawing]{elarc}{center, (rh, rv), (a1, a2)}
Draw an elliptical arc with given center point.
\code{(\var{rh}, \var{rv})}
gives the half sizes of the horizontal and vertical radii.
\code{(\var{a1}, \var{a2})}
gives the angles (in degrees) of the begin and end points.
0 degrees is at 3 o'clock, 90 degrees is at 12 o'clock.
\end{methoddesc}

\begin{methoddesc}[drawing]{erase}{rect}
Erase a rectangle.
\end{methoddesc}

\begin{methoddesc}[drawing]{fillcircle}{center, radius}
Draw a filled circle with given center point and radius.
\end{methoddesc}

\begin{methoddesc}[drawing]{fillelarc}{center, (rh, rv), (a1, a2)}
Draw a filled elliptical arc; arguments as for \method{elarc()}.
\end{methoddesc}

\begin{methoddesc}[drawing]{fillpoly}{points}
Draw a filled polygon given by a list (or tuple) of points.
\end{methoddesc}

\begin{methoddesc}[drawing]{invert}{rect}
Invert a rectangle.
\end{methoddesc}

\begin{methoddesc}[drawing]{line}{p1, p2}
Draw a line from point
\var{p1}
to
\var{p2}.
\end{methoddesc}

\begin{methoddesc}[drawing]{paint}{rect}
Fill a rectangle.
\end{methoddesc}

\begin{methoddesc}[drawing]{poly}{points}
Draw the lines connecting the given list (or tuple) of points.
\end{methoddesc}

\begin{methoddesc}[drawing]{shade}{rect, percent}
Fill a rectangle with a shading pattern that is about
\var{percent}
percent filled.
\end{methoddesc}

\begin{methoddesc}[drawing]{text}{p, str}
Draw a string starting at point p (the point specifies the
top left coordinate of the string).
\end{methoddesc}

\begin{methoddesc}[drawing]{xorcircle}{center, radius}
\funcline{xorelarc}{center, (rh, rv), (a1, a2)}
\funcline{xorline}{p1, p2}
\funcline{xorpoly}{points}
Draw a circle, an elliptical arc, a line or a polygon, respectively,
in XOR mode.
\end{methoddesc}

\begin{methoddesc}[drawing]{setfgcolor}{}
\funcline{setbgcolor}{}
\funcline{getfgcolor}{}
\funcline{getbgcolor}{}
These functions are similar to the corresponding functions described
above for the \module{stdwin}
module, but affect or return the colors currently used for drawing
instead of the global default colors.
When a drawing object is created, its colors are set to the window's
default colors, which are in turn initialized from the global default
colors when the window is created.
\end{methoddesc}

\begin{methoddesc}[drawing]{setfont}{}
\funcline{baseline}{}
\funcline{lineheight}{}
\funcline{textbreak}{}
\funcline{textwidth}{}
These functions are similar to the corresponding functions described
above for the \module{stdwin}
module, but affect or use the current drawing font instead of
the global default font.
When a drawing object is created, its font is set to the window's
default font, which is in turn initialized from the global default
font when the window is created.
\end{methoddesc}

\begin{methoddesc}[drawing]{bitmap}{point, bitmap, mask}
Draw the \var{bitmap} with its top left corner at \var{point}.
If the optional \var{mask} argument is present, it should be either
the same object as \var{bitmap}, to draw only those bits that are set
in the bitmap, in the foreground color, or \code{None}, to draw all
bits (ones are drawn in the foreground color, zeros in the background
color).
Not available on the Macintosh.
\end{methoddesc}

\begin{methoddesc}[drawing]{cliprect}{rect}
Set the ``clipping region'' to a rectangle.
The clipping region limits the effect of all drawing operations, until
it is changed again or until the drawing object is closed.  When a
drawing object is created the clipping region is set to the entire
window.  When an object to be drawn falls partly outside the clipping
region, the set of pixels drawn is the intersection of the clipping
region and the set of pixels that would be drawn by the same operation
in the absence of a clipping region.
\end{methoddesc}

\begin{methoddesc}[drawing]{noclip}{}
Reset the clipping region to the entire window.
\end{methoddesc}

\begin{methoddesc}[drawing]{close}{}
\funcline{enddrawing}{}
Discard the drawing object.  It should not be used again.
\end{methoddesc}

\subsection{Menu Objects}

A menu object represents a menu.
The menu is destroyed when the menu object is deleted.
The following methods are defined:


\begin{methoddesc}[menu]{additem}{text, shortcut}
Add a menu item with given text.
The shortcut must be a string of length 1, or omitted (to specify no
shortcut).
\end{methoddesc}

\begin{methoddesc}[menu]{setitem}{i, text}
Set the text of item number \var{i}.
\end{methoddesc}

\begin{methoddesc}[menu]{enable}{i, flag}
Enable or disables item \var{i}.
\end{methoddesc}

\begin{methoddesc}[menu]{check}{i, flag}
Set or clear the \dfn{check mark} for item \var{i}.
\end{methoddesc}

\begin{methoddesc}[menu]{close}{}
Discard the menu object.  It should not be used again.
\end{methoddesc}

\subsection{Bitmap Objects}

A bitmap represents a rectangular array of bits.
The top left bit has coordinate (0, 0).
A bitmap can be drawn with the \method{bitmap()} method of a drawing object.
Bitmaps are currently not available on the Macintosh.

The following methods are defined:


\begin{methoddesc}[bitmap]{getsize}{}
Return a tuple representing the width and height of the bitmap.
(This returns the values that have been passed to the
\function{newbitmap()} function.)
\end{methoddesc}

\begin{methoddesc}[bitmap]{setbit}{point, bit}
Set the value of the bit indicated by \var{point} to \var{bit}.
\end{methoddesc}

\begin{methoddesc}[bitmap]{getbit}{point}
Return the value of the bit indicated by \var{point}.
\end{methoddesc}

\begin{methoddesc}[bitmap]{close}{}
Discard the bitmap object.  It should not be used again.
\end{methoddesc}

\subsection{Text-edit Objects}

A text-edit object represents a text-edit block.
For semantics, see the STDWIN documentation for \C{} programmers.
The following methods exist:


\begin{methoddesc}[text-edit]{arrow}{code}
Pass an arrow event to the text-edit block.
The \var{code} must be one of \constant{WC_LEFT}, \constant{WC_RIGHT}, 
\constant{WC_UP} or \constant{WC_DOWN} (see module
\module{stdwinevents}).
\end{methoddesc}

\begin{methoddesc}[text-edit]{draw}{rect}
Pass a draw event to the text-edit block.
The rectangle specifies the redraw area.
\end{methoddesc}

\begin{methoddesc}[text-edit]{event}{type, window, detail}
Pass an event gotten from
\function{stdwin.getevent()}
to the text-edit block.
Return true if the event was handled.
\end{methoddesc}

\begin{methoddesc}[text-edit]{getfocus}{}
Return 2 integers representing the start and end positions of the
focus, usable as slice indices on the string returned by
\method{gettext()}.
\end{methoddesc}

\begin{methoddesc}[text-edit]{getfocustext}{}
Return the text in the focus.
\end{methoddesc}

\begin{methoddesc}[text-edit]{getrect}{}
Return a rectangle giving the actual position of the text-edit block.
(The bottom coordinate may differ from the initial position because
the block automatically shrinks or grows to fit.)
\end{methoddesc}

\begin{methoddesc}[text-edit]{gettext}{}
Return the entire text buffer.
\end{methoddesc}

\begin{methoddesc}[text-edit]{move}{rect}
Specify a new position for the text-edit block in the document.
\end{methoddesc}

\begin{methoddesc}[text-edit]{replace}{str}
Replace the text in the focus by the given string.
The new focus is an insert point at the end of the string.
\end{methoddesc}

\begin{methoddesc}[text-edit]{setfocus}{i, j}
Specify the new focus.
Out-of-bounds values are silently clipped.
\end{methoddesc}

\begin{methoddesc}[text-edit]{settext}{str}
Replace the entire text buffer by the given string and set the focus
to \code{(0, 0)}.
\end{methoddesc}

\begin{methoddesc}[text-edit]{setview}{rect}
Set the view rectangle to \var{rect}.  If \var{rect} is \code{None},
viewing mode is reset.  In viewing mode, all output from the text-edit
object is clipped to the viewing rectangle.  This may be useful to
implement your own scrolling text subwindow.
\end{methoddesc}

\begin{methoddesc}[text-edit]{close}{}
Discard the text-edit object.  It should not be used again.
\end{methoddesc}

\subsection{Example}
\nodename{STDWIN Example}

Here is a minimal example of using STDWIN in Python.
It creates a window and draws the string ``Hello world'' in the top
left corner of the window.
The window will be correctly redrawn when covered and re-exposed.
The program quits when the close icon or menu item is requested.

\begin{verbatim}
import stdwin
from stdwinevents import *

def main():
    mywin = stdwin.open('Hello')
    #
    while 1:
        (type, win, detail) = stdwin.getevent()
        if type == WE_DRAW:
            draw = win.begindrawing()
            draw.text((0, 0), 'Hello, world')
            del draw
        elif type == WE_CLOSE:
            break

main()
\end{verbatim}

\section{\module{stdwinevents} ---
         Constants for use with \module{stdwin}}
\declaremodule{standard}{stdwinevents}

\modulesynopsis{Constant definitions for use with \module{stdwin}}


This module defines constants used by STDWIN for event types
(\constant{WE_ACTIVATE} etc.), command codes (\constant{WC_LEFT} etc.)
and selection types (\constant{WS_PRIMARY} etc.).
Read the file for details.
Suggested usage is

\begin{verbatim}
>>> from stdwinevents import *
>>> 
\end{verbatim}

\section{\module{rect} ---
         Functions for use with \module{stdwin}}
\declaremodule{standard}{rect}

\modulesynopsis{Geometry-related utility function for use with \module{stdwin}}


This module contains useful operations on rectangles.
A rectangle is defined as in module \module{stdwin}:
a pair of points, where a point is a pair of integers.
For example, the rectangle

\begin{verbatim}
(10, 20), (90, 80)
\end{verbatim}

is a rectangle whose left, top, right and bottom edges are 10, 20, 90
and 80, respectively.  Note that the positive vertical axis points
down (as in \module{stdwin}).

The module defines the following objects:

\begin{excdesc}{error}
The exception raised by functions in this module when they detect an
error.  The exception argument is a string describing the problem in
more detail.
\end{excdesc}

\begin{datadesc}{empty}
The rectangle returned when some operations return an empty result.
This makes it possible to quickly check whether a result is empty:

\begin{verbatim}
>>> import rect
>>> r1 = (10, 20), (90, 80)
>>> r2 = (0, 0), (10, 20)
>>> r3 = rect.intersect([r1, r2])
>>> if r3 is rect.empty: print 'Empty intersection'
Empty intersection
>>> 
\end{verbatim}
\end{datadesc}

\begin{funcdesc}{is_empty}{r}
Returns true if the given rectangle is empty.
A rectangle
\code{(\var{left}, \var{top}), (\var{right}, \var{bottom})}
is empty if
\begin{math}\var{left} \geq \var{right}\end{math} or
\begin{math}\var{top} \geq \var{bottom}\end{math}.
\end{funcdesc}

\begin{funcdesc}{intersect}{list}
Returns the intersection of all rectangles in the list argument.
It may also be called with a tuple argument.  Raises
\exception{rect.error} if the list is empty.  Returns
\constant{rect.empty} if the intersection of the rectangles is empty.
\end{funcdesc}

\begin{funcdesc}{union}{list}
Returns the smallest rectangle that contains all non-empty rectangles in
the list argument.  It may also be called with a tuple argument or
with two or more rectangles as arguments.  Returns
\constant{rect.empty} if the list is empty or all its rectangles are
empty.
\end{funcdesc}

\begin{funcdesc}{pointinrect}{point, rect}
Returns true if the point is inside the rectangle.  By definition, a
point \code{(\var{h}, \var{v})} is inside a rectangle
\code{(\var{left}, \var{top}), (\var{right}, \var{bottom})} if
\begin{math}\var{left} \leq \var{h} < \var{right}\end{math} and
\begin{math}\var{top} \leq \var{v} < \var{bottom}\end{math}.
\end{funcdesc}

\begin{funcdesc}{inset}{rect, (dh, dv)}
Returns a rectangle that lies inside the \var{rect} argument by
\var{dh} pixels horizontally and \var{dv} pixels vertically.  If
\var{dh} or \var{dv} is negative, the result lies outside \var{rect}.
\end{funcdesc}

\begin{funcdesc}{rect2geom}{rect}
Converts a rectangle to geometry representation:
\code{(\var{left}, \var{top}), (\var{width}, \var{height})}.
\end{funcdesc}

\begin{funcdesc}{geom2rect}{geom}
Converts a rectangle given in geometry representation back to the
standard rectangle representation
\code{(\var{left}, \var{top}), (\var{right}, \var{bottom})}.
\end{funcdesc}
		% STDWIN ONLY

\chapter{SGI IRIX Specific Services}
\label{sgi}

The modules described in this chapter provide interfaces to features
that are unique to SGI's IRIX operating system (versions 4 and 5).

\localmoduletable
			% SGI IRIX ONLY
\section{\module{al} ---
         Audio functions on the SGI}

\declaremodule{builtin}{al}
  \platform{IRIX}
\modulesynopsis{Audio functions on the SGI.}


This module provides access to the audio facilities of the SGI Indy
and Indigo workstations.  See section 3A of the IRIX man pages for
details.  You'll need to read those man pages to understand what these
functions do!  Some of the functions are not available in IRIX
releases before 4.0.5.  Again, see the manual to check whether a
specific function is available on your platform.

All functions and methods defined in this module are equivalent to
the C functions with \samp{AL} prefixed to their name.

Symbolic constants from the C header file \code{<audio.h>} are
defined in the standard module \module{AL}\refstmodindex{AL}, see
below.

\strong{Warning:} the current version of the audio library may dump core
when bad argument values are passed rather than returning an error
status.  Unfortunately, since the precise circumstances under which
this may happen are undocumented and hard to check, the Python
interface can provide no protection against this kind of problems.
(One example is specifying an excessive queue size --- there is no
documented upper limit.)

The module defines the following functions:


\begin{funcdesc}{openport}{name, direction\optional{, config}}
The name and direction arguments are strings.  The optional
\var{config} argument is a configuration object as returned by
\function{newconfig()}.  The return value is an \dfn{audio port
object}; methods of audio port objects are described below.
\end{funcdesc}

\begin{funcdesc}{newconfig}{}
The return value is a new \dfn{audio configuration object}; methods of
audio configuration objects are described below.
\end{funcdesc}

\begin{funcdesc}{queryparams}{device}
The device argument is an integer.  The return value is a list of
integers containing the data returned by \cfunction{ALqueryparams()}.
\end{funcdesc}

\begin{funcdesc}{getparams}{device, list}
The \var{device} argument is an integer.  The list argument is a list
such as returned by \function{queryparams()}; it is modified in place
(!).
\end{funcdesc}

\begin{funcdesc}{setparams}{device, list}
The \var{device} argument is an integer.  The \var{list} argument is a
list such as returned by \function{queryparams()}.
\end{funcdesc}


\subsection{Configuration Objects \label{al-config-objects}}

Configuration objects (returned by \function{newconfig()} have the
following methods:

\begin{methoddesc}[audio configuration]{getqueuesize}{}
Return the queue size.
\end{methoddesc}

\begin{methoddesc}[audio configuration]{setqueuesize}{size}
Set the queue size.
\end{methoddesc}

\begin{methoddesc}[audio configuration]{getwidth}{}
Get the sample width.
\end{methoddesc}

\begin{methoddesc}[audio configuration]{setwidth}{width}
Set the sample width.
\end{methoddesc}

\begin{methoddesc}[audio configuration]{getchannels}{}
Get the channel count.
\end{methoddesc}

\begin{methoddesc}[audio configuration]{setchannels}{nchannels}
Set the channel count.
\end{methoddesc}

\begin{methoddesc}[audio configuration]{getsampfmt}{}
Get the sample format.
\end{methoddesc}

\begin{methoddesc}[audio configuration]{setsampfmt}{sampfmt}
Set the sample format.
\end{methoddesc}

\begin{methoddesc}[audio configuration]{getfloatmax}{}
Get the maximum value for floating sample formats.
\end{methoddesc}

\begin{methoddesc}[audio configuration]{setfloatmax}{floatmax}
Set the maximum value for floating sample formats.
\end{methoddesc}


\subsection{Port Objects \label{al-port-objects}}

Port objects, as returned by \function{openport()}, have the following
methods:

\begin{methoddesc}[audio port]{closeport}{}
Close the port.
\end{methoddesc}

\begin{methoddesc}[audio port]{getfd}{}
Return the file descriptor as an int.
\end{methoddesc}

\begin{methoddesc}[audio port]{getfilled}{}
Return the number of filled samples.
\end{methoddesc}

\begin{methoddesc}[audio port]{getfillable}{}
Return the number of fillable samples.
\end{methoddesc}

\begin{methoddesc}[audio port]{readsamps}{nsamples}
Read a number of samples from the queue, blocking if necessary.
Return the data as a string containing the raw data, (e.g., 2 bytes per
sample in big-endian byte order (high byte, low byte) if you have set
the sample width to 2 bytes).
\end{methoddesc}

\begin{methoddesc}[audio port]{writesamps}{samples}
Write samples into the queue, blocking if necessary.  The samples are
encoded as described for the \method{readsamps()} return value.
\end{methoddesc}

\begin{methoddesc}[audio port]{getfillpoint}{}
Return the `fill point'.
\end{methoddesc}

\begin{methoddesc}[audio port]{setfillpoint}{fillpoint}
Set the `fill point'.
\end{methoddesc}

\begin{methoddesc}[audio port]{getconfig}{}
Return a configuration object containing the current configuration of
the port.
\end{methoddesc}

\begin{methoddesc}[audio port]{setconfig}{config}
Set the configuration from the argument, a configuration object.
\end{methoddesc}

\begin{methoddesc}[audio port]{getstatus}{list}
Get status information on last error.
\end{methoddesc}


\section{\module{AL} ---
         Constants used with the \module{al} module}

\declaremodule[al-constants]{standard}{AL}
  \platform{IRIX}
\modulesynopsis{Constants used with the \module{al} module.}


This module defines symbolic constants needed to use the built-in
module \module{al} (see above); they are equivalent to those defined
in the C header file \code{<audio.h>} except that the name prefix
\samp{AL_} is omitted.  Read the module source for a complete list of
the defined names.  Suggested use:

\begin{verbatim}
import al
from AL import *
\end{verbatim}

\section{\module{cd} ---
         CD-ROM access on SGI systems}

\declaremodule{builtin}{cd}
  \platform{IRIX}
\modulesynopsis{Interface to the CD-ROM on Silicon Graphics systems.}


This module provides an interface to the Silicon Graphics CD library.
It is available only on Silicon Graphics systems.

The way the library works is as follows.  A program opens the CD-ROM
device with \function{open()} and creates a parser to parse the data
from the CD with \function{createparser()}.  The object returned by
\function{open()} can be used to read data from the CD, but also to get
status information for the CD-ROM device, and to get information about
the CD, such as the table of contents.  Data from the CD is passed to
the parser, which parses the frames, and calls any callback
functions that have previously been added.

An audio CD is divided into \dfn{tracks} or \dfn{programs} (the terms
are used interchangeably).  Tracks can be subdivided into
\dfn{indices}.  An audio CD contains a \dfn{table of contents} which
gives the starts of the tracks on the CD.  Index 0 is usually the
pause before the start of a track.  The start of the track as given by
the table of contents is normally the start of index 1.

Positions on a CD can be represented in two ways.  Either a frame
number or a tuple of three values, minutes, seconds and frames.  Most
functions use the latter representation.  Positions can be both
relative to the beginning of the CD, and to the beginning of the
track.

Module \module{cd} defines the following functions and constants:


\begin{funcdesc}{createparser}{}
Create and return an opaque parser object.  The methods of the parser
object are described below.
\end{funcdesc}

\begin{funcdesc}{msftoframe}{minutes, seconds, frames}
Converts a \code{(\var{minutes}, \var{seconds}, \var{frames})} triple
representing time in absolute time code into the corresponding CD
frame number.
\end{funcdesc}

\begin{funcdesc}{open}{\optional{device\optional{, mode}}}
Open the CD-ROM device.  The return value is an opaque player object;
methods of the player object are described below.  The device is the
name of the SCSI device file, e.g. \code{'/dev/scsi/sc0d4l0'}, or
\code{None}.  If omitted or \code{None}, the hardware inventory is
consulted to locate a CD-ROM drive.  The \var{mode}, if not omitted,
should be the string \code{'r'}.
\end{funcdesc}

The module defines the following variables:

\begin{excdesc}{error}
Exception raised on various errors.
\end{excdesc}

\begin{datadesc}{DATASIZE}
The size of one frame's worth of audio data.  This is the size of the
audio data as passed to the callback of type \code{audio}.
\end{datadesc}

\begin{datadesc}{BLOCKSIZE}
The size of one uninterpreted frame of audio data.
\end{datadesc}

The following variables are states as returned by
\function{getstatus()}:

\begin{datadesc}{READY}
The drive is ready for operation loaded with an audio CD.
\end{datadesc}

\begin{datadesc}{NODISC}
The drive does not have a CD loaded.
\end{datadesc}

\begin{datadesc}{CDROM}
The drive is loaded with a CD-ROM.  Subsequent play or read operations
will return I/O errors.
\end{datadesc}

\begin{datadesc}{ERROR}
An error occurred while trying to read the disc or its table of
contents.
\end{datadesc}

\begin{datadesc}{PLAYING}
The drive is in CD player mode playing an audio CD through its audio
jacks.
\end{datadesc}

\begin{datadesc}{PAUSED}
The drive is in CD layer mode with play paused.
\end{datadesc}

\begin{datadesc}{STILL}
The equivalent of \constant{PAUSED} on older (non 3301) model Toshiba
CD-ROM drives.  Such drives have never been shipped by SGI.
\end{datadesc}

\begin{datadesc}{audio}
\dataline{pnum}
\dataline{index}
\dataline{ptime}
\dataline{atime}
\dataline{catalog}
\dataline{ident}
\dataline{control}
Integer constants describing the various types of parser callbacks
that can be set by the \method{addcallback()} method of CD parser
objects (see below).
\end{datadesc}


\subsection{Player Objects}
\label{player-objects}

Player objects (returned by \function{open()}) have the following
methods:

\begin{methoddesc}[CD player]{allowremoval}{}
Unlocks the eject button on the CD-ROM drive permitting the user to
eject the caddy if desired.
\end{methoddesc}

\begin{methoddesc}[CD player]{bestreadsize}{}
Returns the best value to use for the \var{num_frames} parameter of
the \method{readda()} method.  Best is defined as the value that
permits a continuous flow of data from the CD-ROM drive.
\end{methoddesc}

\begin{methoddesc}[CD player]{close}{}
Frees the resources associated with the player object.  After calling
\method{close()}, the methods of the object should no longer be used.
\end{methoddesc}

\begin{methoddesc}[CD player]{eject}{}
Ejects the caddy from the CD-ROM drive.
\end{methoddesc}

\begin{methoddesc}[CD player]{getstatus}{}
Returns information pertaining to the current state of the CD-ROM
drive.  The returned information is a tuple with the following values:
\var{state}, \var{track}, \var{rtime}, \var{atime}, \var{ttime},
\var{first}, \var{last}, \var{scsi_audio}, \var{cur_block}.
\var{rtime} is the time relative to the start of the current track;
\var{atime} is the time relative to the beginning of the disc;
\var{ttime} is the total time on the disc.  For more information on
the meaning of the values, see the man page \manpage{CDgetstatus}{3dm}.
The value of \var{state} is one of the following: \constant{ERROR},
\constant{NODISC}, \constant{READY}, \constant{PLAYING},
\constant{PAUSED}, \constant{STILL}, or \constant{CDROM}.
\end{methoddesc}

\begin{methoddesc}[CD player]{gettrackinfo}{track}
Returns information about the specified track.  The returned
information is a tuple consisting of two elements, the start time of
the track and the duration of the track.
\end{methoddesc}

\begin{methoddesc}[CD player]{msftoblock}{min, sec, frame}
Converts a minutes, seconds, frames triple representing a time in
absolute time code into the corresponding logical block number for the
given CD-ROM drive.  You should use \function{msftoframe()} rather than
\method{msftoblock()} for comparing times.  The logical block number
differs from the frame number by an offset required by certain CD-ROM
drives.
\end{methoddesc}

\begin{methoddesc}[CD player]{play}{start, play}
Starts playback of an audio CD in the CD-ROM drive at the specified
track.  The audio output appears on the CD-ROM drive's headphone and
audio jacks (if fitted).  Play stops at the end of the disc.
\var{start} is the number of the track at which to start playing the
CD; if \var{play} is 0, the CD will be set to an initial paused
state.  The method \method{togglepause()} can then be used to commence
play.
\end{methoddesc}

\begin{methoddesc}[CD player]{playabs}{minutes, seconds, frames, play}
Like \method{play()}, except that the start is given in minutes,
seconds, and frames instead of a track number.
\end{methoddesc}

\begin{methoddesc}[CD player]{playtrack}{start, play}
Like \method{play()}, except that playing stops at the end of the
track.
\end{methoddesc}

\begin{methoddesc}[CD player]{playtrackabs}{track, minutes, seconds, frames, play}
Like \method{play()}, except that playing begins at the specified
absolute time and ends at the end of the specified track.
\end{methoddesc}

\begin{methoddesc}[CD player]{preventremoval}{}
Locks the eject button on the CD-ROM drive thus preventing the user
from arbitrarily ejecting the caddy.
\end{methoddesc}

\begin{methoddesc}[CD player]{readda}{num_frames}
Reads the specified number of frames from an audio CD mounted in the
CD-ROM drive.  The return value is a string representing the audio
frames.  This string can be passed unaltered to the
\method{parseframe()} method of the parser object.
\end{methoddesc}

\begin{methoddesc}[CD player]{seek}{minutes, seconds, frames}
Sets the pointer that indicates the starting point of the next read of
digital audio data from a CD-ROM.  The pointer is set to an absolute
time code location specified in \var{minutes}, \var{seconds}, and
\var{frames}.  The return value is the logical block number to which
the pointer has been set.
\end{methoddesc}

\begin{methoddesc}[CD player]{seekblock}{block}
Sets the pointer that indicates the starting point of the next read of
digital audio data from a CD-ROM.  The pointer is set to the specified
logical block number.  The return value is the logical block number to
which the pointer has been set.
\end{methoddesc}

\begin{methoddesc}[CD player]{seektrack}{track}
Sets the pointer that indicates the starting point of the next read of
digital audio data from a CD-ROM.  The pointer is set to the specified
track.  The return value is the logical block number to which the
pointer has been set.
\end{methoddesc}

\begin{methoddesc}[CD player]{stop}{}
Stops the current playing operation.
\end{methoddesc}

\begin{methoddesc}[CD player]{togglepause}{}
Pauses the CD if it is playing, and makes it play if it is paused.
\end{methoddesc}


\subsection{Parser Objects}
\label{cd-parser-objects}

Parser objects (returned by \function{createparser()}) have the
following methods:

\begin{methoddesc}[CD parser]{addcallback}{type, func, arg}
Adds a callback for the parser.  The parser has callbacks for eight
different types of data in the digital audio data stream.  Constants
for these types are defined at the \module{cd} module level (see above).
The callback is called as follows: \code{\var{func}(\var{arg}, type,
data)}, where \var{arg} is the user supplied argument, \var{type} is
the particular type of callback, and \var{data} is the data returned
for this \var{type} of callback.  The type of the data depends on the
\var{type} of callback as follows:

\begin{tableii}{l|p{4in}}{code}{Type}{Value}
  \lineii{audio}{String which can be passed unmodified to
\function{al.writesamps()}.}
  \lineii{pnum}{Integer giving the program (track) number.}
  \lineii{index}{Integer giving the index number.}
  \lineii{ptime}{Tuple consisting of the program time in minutes,
seconds, and frames.}
  \lineii{atime}{Tuple consisting of the absolute time in minutes,
seconds, and frames.}
  \lineii{catalog}{String of 13 characters, giving the catalog number
of the CD.}
  \lineii{ident}{String of 12 characters, giving the ISRC
identification number of the recording.  The string consists of two
characters country code, three characters owner code, two characters
giving the year, and five characters giving a serial number.}
  \lineii{control}{Integer giving the control bits from the CD
subcode data}
\end{tableii}
\end{methoddesc}

\begin{methoddesc}[CD parser]{deleteparser}{}
Deletes the parser and frees the memory it was using.  The object
should not be used after this call.  This call is done automatically
when the last reference to the object is removed.
\end{methoddesc}

\begin{methoddesc}[CD parser]{parseframe}{frame}
Parses one or more frames of digital audio data from a CD such as
returned by \method{readda()}.  It determines which subcodes are
present in the data.  If these subcodes have changed since the last
frame, then \method{parseframe()} executes a callback of the
appropriate type passing to it the subcode data found in the frame.
Unlike the \C{} function, more than one frame of digital audio data
can be passed to this method.
\end{methoddesc}

\begin{methoddesc}[CD parser]{removecallback}{type}
Removes the callback for the given \var{type}.
\end{methoddesc}

\begin{methoddesc}[CD parser]{resetparser}{}
Resets the fields of the parser used for tracking subcodes to an
initial state.  \method{resetparser()} should be called after the disc
has been changed.
\end{methoddesc}

\section{Built-in Module \sectcode{fl}}
\label{module-fl}
\bimodindex{fl}

This module provides an interface to the FORMS Library by Mark
Overmars.  The source for the library can be retrieved by anonymous
ftp from host \samp{ftp.cs.ruu.nl}, directory \file{SGI/FORMS}.  It
was last tested with version 2.0b.

Most functions are literal translations of their C equivalents,
dropping the initial \samp{fl_} from their name.  Constants used by
the library are defined in module \code{FL} described below.

The creation of objects is a little different in Python than in C:
instead of the `current form' maintained by the library to which new
FORMS objects are added, all functions that add a FORMS object to a
form are methods of the Python object representing the form.
Consequently, there are no Python equivalents for the C functions
\code{fl_addto_form} and \code{fl_end_form}, and the equivalent of
\code{fl_bgn_form} is called \code{fl.make_form}.

Watch out for the somewhat confusing terminology: FORMS uses the word
\dfn{object} for the buttons, sliders etc. that you can place in a form.
In Python, `object' means any value.  The Python interface to FORMS
introduces two new Python object types: form objects (representing an
entire form) and FORMS objects (representing one button, slider etc.).
Hopefully this isn't too confusing...

There are no `free objects' in the Python interface to FORMS, nor is
there an easy way to add object classes written in Python.  The FORMS
interface to GL event handling is available, though, so you can mix
FORMS with pure GL windows.

\strong{Please note:} importing \code{fl} implies a call to the GL function
\code{foreground()} and to the FORMS routine \code{fl_init()}.

\subsection{Functions Defined in Module \sectcode{fl}}
\nodename{FL Functions}

Module \code{fl} defines the following functions.  For more information
about what they do, see the description of the equivalent C function
in the FORMS documentation:

\setindexsubitem{(in module fl)}
\begin{funcdesc}{make_form}{type\, width\, height}
Create a form with given type, width and height.  This returns a
\dfn{form} object, whose methods are described below.
\end{funcdesc}

\begin{funcdesc}{do_forms}{}
The standard FORMS main loop.  Returns a Python object representing
the FORMS object needing interaction, or the special value
\code{FL.EVENT}.
\end{funcdesc}

\begin{funcdesc}{check_forms}{}
Check for FORMS events.  Returns what \code{do_forms} above returns,
or \code{None} if there is no event that immediately needs
interaction.
\end{funcdesc}

\begin{funcdesc}{set_event_call_back}{function}
Set the event callback function.
\end{funcdesc}

\begin{funcdesc}{set_graphics_mode}{rgbmode\, doublebuffering}
Set the graphics modes.
\end{funcdesc}

\begin{funcdesc}{get_rgbmode}{}
Return the current rgb mode.  This is the value of the C global
variable \code{fl_rgbmode}.
\end{funcdesc}

\begin{funcdesc}{show_message}{str1\, str2\, str3}
Show a dialog box with a three-line message and an OK button.
\end{funcdesc}

\begin{funcdesc}{show_question}{str1\, str2\, str3}
Show a dialog box with a three-line message and YES and NO buttons.
It returns \code{1} if the user pressed YES, \code{0} if NO.
\end{funcdesc}

\begin{funcdesc}{show_choice}{str1, str2, str3, but1\optional{, but2, but3}}
Show a dialog box with a three-line message and up to three buttons.
It returns the number of the button clicked by the user
(\code{1}, \code{2} or \code{3}).
\end{funcdesc}

\begin{funcdesc}{show_input}{prompt\, default}
Show a dialog box with a one-line prompt message and text field in
which the user can enter a string.  The second argument is the default
input string.  It returns the string value as edited by the user.
\end{funcdesc}

\begin{funcdesc}{show_file_selector}{message\, directory\, pattern\, default}
Show a dialog box in which the user can select a file.  It returns
the absolute filename selected by the user, or \code{None} if the user
presses Cancel.
\end{funcdesc}

\begin{funcdesc}{get_directory}{}
\funcline{get_pattern}{}
\funcline{get_filename}{}
These functions return the directory, pattern and filename (the tail
part only) selected by the user in the last \code{show_file_selector}
call.
\end{funcdesc}

\begin{funcdesc}{qdevice}{dev}
\funcline{unqdevice}{dev}
\funcline{isqueued}{dev}
\funcline{qtest}{}
\funcline{qread}{}
%\funcline{blkqread}{?}
\funcline{qreset}{}
\funcline{qenter}{dev\, val}
\funcline{get_mouse}{}
\funcline{tie}{button\, valuator1\, valuator2}
These functions are the FORMS interfaces to the corresponding GL
functions.  Use these if you want to handle some GL events yourself
when using \code{fl.do_events}.  When a GL event is detected that
FORMS cannot handle, \code{fl.do_forms()} returns the special value
\code{FL.EVENT} and you should call \code{fl.qread()} to read the
event from the queue.  Don't use the equivalent GL functions!
\end{funcdesc}

\begin{funcdesc}{color}{}
\funcline{mapcolor}{}
\funcline{getmcolor}{}
See the description in the FORMS documentation of \code{fl_color},
\code{fl_mapcolor} and \code{fl_getmcolor}.
\end{funcdesc}

\subsection{Form Objects}

Form objects (returned by \code{fl.make_form()} above) have the
following methods.  Each method corresponds to a C function whose name
is prefixed with \samp{fl_}; and whose first argument is a form
pointer; please refer to the official FORMS documentation for
descriptions.

All the \samp{add_{\rm \ldots}} functions return a Python object representing
the FORMS object.  Methods of FORMS objects are described below.  Most
kinds of FORMS object also have some methods specific to that kind;
these methods are listed here.

\begin{flushleft}
\setindexsubitem{(form object method)}
\begin{funcdesc}{show_form}{placement\, bordertype\, name}
  Show the form.
\end{funcdesc}

\begin{funcdesc}{hide_form}{}
  Hide the form.
\end{funcdesc}

\begin{funcdesc}{redraw_form}{}
  Redraw the form.
\end{funcdesc}

\begin{funcdesc}{set_form_position}{x\, y}
Set the form's position.
\end{funcdesc}

\begin{funcdesc}{freeze_form}{}
Freeze the form.
\end{funcdesc}

\begin{funcdesc}{unfreeze_form}{}
  Unfreeze the form.
\end{funcdesc}

\begin{funcdesc}{activate_form}{}
  Activate the form.
\end{funcdesc}

\begin{funcdesc}{deactivate_form}{}
  Deactivate the form.
\end{funcdesc}

\begin{funcdesc}{bgn_group}{}
  Begin a new group of objects; return a group object.
\end{funcdesc}

\begin{funcdesc}{end_group}{}
  End the current group of objects.
\end{funcdesc}

\begin{funcdesc}{find_first}{}
  Find the first object in the form.
\end{funcdesc}

\begin{funcdesc}{find_last}{}
  Find the last object in the form.
\end{funcdesc}

%---

\begin{funcdesc}{add_box}{type\, x\, y\, w\, h\, name}
Add a box object to the form.
No extra methods.
\end{funcdesc}

\begin{funcdesc}{add_text}{type\, x\, y\, w\, h\, name}
Add a text object to the form.
No extra methods.
\end{funcdesc}

%\begin{funcdesc}{add_bitmap}{type\, x\, y\, w\, h\, name}
%Add a bitmap object to the form.
%\end{funcdesc}

\begin{funcdesc}{add_clock}{type\, x\, y\, w\, h\, name}
Add a clock object to the form. \\
Method:
\code{get_clock}.
\end{funcdesc}

%---

\begin{funcdesc}{add_button}{type\, x\, y\, w\, h\,  name}
Add a button object to the form. \\
Methods:
\code{get_button},
\code{set_button}.
\end{funcdesc}

\begin{funcdesc}{add_lightbutton}{type\, x\, y\, w\, h\, name}
Add a lightbutton object to the form. \\
Methods:
\code{get_button},
\code{set_button}.
\end{funcdesc}

\begin{funcdesc}{add_roundbutton}{type\, x\, y\, w\, h\, name}
Add a roundbutton object to the form. \\
Methods:
\code{get_button},
\code{set_button}.
\end{funcdesc}

%---

\begin{funcdesc}{add_slider}{type\, x\, y\, w\, h\, name}
Add a slider object to the form. \\
Methods:
\code{set_slider_value},
\code{get_slider_value},
\code{set_slider_bounds},
\code{get_slider_bounds},
\code{set_slider_return},
\code{set_slider_size},
\code{set_slider_precision},
\code{set_slider_step}.
\end{funcdesc}

\begin{funcdesc}{add_valslider}{type\, x\, y\, w\, h\, name}
Add a valslider object to the form. \\
Methods:
\code{set_slider_value},
\code{get_slider_value},
\code{set_slider_bounds},
\code{get_slider_bounds},
\code{set_slider_return},
\code{set_slider_size},
\code{set_slider_precision},
\code{set_slider_step}.
\end{funcdesc}

\begin{funcdesc}{add_dial}{type\, x\, y\, w\, h\, name}
Add a dial object to the form. \\
Methods:
\code{set_dial_value},
\code{get_dial_value},
\code{set_dial_bounds},
\code{get_dial_bounds}.
\end{funcdesc}

\begin{funcdesc}{add_positioner}{type\, x\, y\, w\, h\, name}
Add a positioner object to the form. \\
Methods:
\code{set_positioner_xvalue},
\code{set_positioner_yvalue},
\code{set_positioner_xbounds},
\code{set_positioner_ybounds},
\code{get_positioner_xvalue},
\code{get_positioner_yvalue},
\code{get_positioner_xbounds},
\code{get_positioner_ybounds}.
\end{funcdesc}

\begin{funcdesc}{add_counter}{type\, x\, y\, w\, h\, name}
Add a counter object to the form. \\
Methods:
\code{set_counter_value},
\code{get_counter_value},
\code{set_counter_bounds},
\code{set_counter_step},
\code{set_counter_precision},
\code{set_counter_return}.
\end{funcdesc}

%---

\begin{funcdesc}{add_input}{type\, x\, y\, w\, h\, name}
Add a input object to the form. \\
Methods:
\code{set_input},
\code{get_input},
\code{set_input_color},
\code{set_input_return}.
\end{funcdesc}

%---

\begin{funcdesc}{add_menu}{type\, x\, y\, w\, h\, name}
Add a menu object to the form. \\
Methods:
\code{set_menu},
\code{get_menu},
\code{addto_menu}.
\end{funcdesc}

\begin{funcdesc}{add_choice}{type\, x\, y\, w\, h\, name}
Add a choice object to the form. \\
Methods:
\code{set_choice},
\code{get_choice},
\code{clear_choice},
\code{addto_choice},
\code{replace_choice},
\code{delete_choice},
\code{get_choice_text},
\code{set_choice_fontsize},
\code{set_choice_fontstyle}.
\end{funcdesc}

\begin{funcdesc}{add_browser}{type\, x\, y\, w\, h\, name}
Add a browser object to the form. \\
Methods:
\code{set_browser_topline},
\code{clear_browser},
\code{add_browser_line},
\code{addto_browser},
\code{insert_browser_line},
\code{delete_browser_line},
\code{replace_browser_line},
\code{get_browser_line},
\code{load_browser},
\code{get_browser_maxline},
\code{select_browser_line},
\code{deselect_browser_line},
\code{deselect_browser},
\code{isselected_browser_line},
\code{get_browser},
\code{set_browser_fontsize},
\code{set_browser_fontstyle},
\code{set_browser_specialkey}.
\end{funcdesc}

%---

\begin{funcdesc}{add_timer}{type\, x\, y\, w\, h\, name}
Add a timer object to the form. \\
Methods:
\code{set_timer},
\code{get_timer}.
\end{funcdesc}
\end{flushleft}

Form objects have the following data attributes; see the FORMS
documentation:

\begin{tableiii}{|l|c|l|}{code}{Name}{Type}{Meaning}
  \lineiii{window}{int (read-only)}{GL window id}
  \lineiii{w}{float}{form width}
  \lineiii{h}{float}{form height}
  \lineiii{x}{float}{form x origin}
  \lineiii{y}{float}{form y origin}
  \lineiii{deactivated}{int}{nonzero if form is deactivated}
  \lineiii{visible}{int}{nonzero if form is visible}
  \lineiii{frozen}{int}{nonzero if form is frozen}
  \lineiii{doublebuf}{int}{nonzero if double buffering on}
\end{tableiii}

\subsection{FORMS Objects}

Besides methods specific to particular kinds of FORMS objects, all
FORMS objects also have the following methods:

\setindexsubitem{(FORMS object method)}
\begin{funcdesc}{set_call_back}{function\, argument}
Set the object's callback function and argument.  When the object
needs interaction, the callback function will be called with two
arguments: the object, and the callback argument.  (FORMS objects
without a callback function are returned by \code{fl.do_forms()} or
\code{fl.check_forms()} when they need interaction.)  Call this method
without arguments to remove the callback function.
\end{funcdesc}

\begin{funcdesc}{delete_object}{}
  Delete the object.
\end{funcdesc}

\begin{funcdesc}{show_object}{}
  Show the object.
\end{funcdesc}

\begin{funcdesc}{hide_object}{}
  Hide the object.
\end{funcdesc}

\begin{funcdesc}{redraw_object}{}
  Redraw the object.
\end{funcdesc}

\begin{funcdesc}{freeze_object}{}
  Freeze the object.
\end{funcdesc}

\begin{funcdesc}{unfreeze_object}{}
  Unfreeze the object.
\end{funcdesc}

%\begin{funcdesc}{handle_object}{} XXX
%\end{funcdesc}

%\begin{funcdesc}{handle_object_direct}{} XXX
%\end{funcdesc}

FORMS objects have these data attributes; see the FORMS documentation:

\begin{tableiii}{|l|c|l|}{code}{Name}{Type}{Meaning}
  \lineiii{objclass}{int (read-only)}{object class}
  \lineiii{type}{int (read-only)}{object type}
  \lineiii{boxtype}{int}{box type}
  \lineiii{x}{float}{x origin}
  \lineiii{y}{float}{y origin}
  \lineiii{w}{float}{width}
  \lineiii{h}{float}{height}
  \lineiii{col1}{int}{primary color}
  \lineiii{col2}{int}{secondary color}
  \lineiii{align}{int}{alignment}
  \lineiii{lcol}{int}{label color}
  \lineiii{lsize}{float}{label font size}
  \lineiii{label}{string}{label string}
  \lineiii{lstyle}{int}{label style}
  \lineiii{pushed}{int (read-only)}{(see FORMS docs)}
  \lineiii{focus}{int (read-only)}{(see FORMS docs)}
  \lineiii{belowmouse}{int (read-only)}{(see FORMS docs)}
  \lineiii{frozen}{int (read-only)}{(see FORMS docs)}
  \lineiii{active}{int (read-only)}{(see FORMS docs)}
  \lineiii{input}{int (read-only)}{(see FORMS docs)}
  \lineiii{visible}{int (read-only)}{(see FORMS docs)}
  \lineiii{radio}{int (read-only)}{(see FORMS docs)}
  \lineiii{automatic}{int (read-only)}{(see FORMS docs)}
\end{tableiii}

\section{Standard Module \sectcode{FL}}
\nodename{FL (uppercase)}
\label{module-FL}
\stmodindex{FL}

This module defines symbolic constants needed to use the built-in
module \code{fl} (see above); they are equivalent to those defined in
the C header file \file{<forms.h>} except that the name prefix
\samp{FL_} is omitted.  Read the module source for a complete list of
the defined names.  Suggested use:

\begin{verbatim}
import fl
from FL import *
\end{verbatim}
%
\section{Standard Module \sectcode{flp}}
\label{module-flp}
\stmodindex{flp}

This module defines functions that can read form definitions created
by the `form designer' (\code{fdesign}) program that comes with the
FORMS library (see module \code{fl} above).

For now, see the file \file{flp.doc} in the Python library source
directory for a description.

XXX A complete description should be inserted here!

\section{\module{fm} ---
         \emph{Font Manager} interface}

\declaremodule{builtin}{fm}
  \platform{IRIX}
\modulesynopsis{\emph{Font Manager} interface for SGI workstations.}


This module provides access to the IRIS \emph{Font Manager} library.
\index{Font Manager, IRIS}
\index{IRIS Font Manager}
It is available only on Silicon Graphics machines.
See also: \emph{4Sight User's Guide}, section 1, chapter 5: ``Using
the IRIS Font Manager.''

This is not yet a full interface to the IRIS Font Manager.
Among the unsupported features are: matrix operations; cache
operations; character operations (use string operations instead); some
details of font info; individual glyph metrics; and printer matching.

It supports the following operations:

\begin{funcdesc}{init}{}
Initialization function.
Calls \cfunction{fminit()}.
It is normally not necessary to call this function, since it is called
automatically the first time the \module{fm} module is imported.
\end{funcdesc}

\begin{funcdesc}{findfont}{fontname}
Return a font handle object.
Calls \code{fmfindfont(\var{fontname})}.
\end{funcdesc}

\begin{funcdesc}{enumerate}{}
Returns a list of available font names.
This is an interface to \cfunction{fmenumerate()}.
\end{funcdesc}

\begin{funcdesc}{prstr}{string}
Render a string using the current font (see the \function{setfont()} font
handle method below).
Calls \code{fmprstr(\var{string})}.
\end{funcdesc}

\begin{funcdesc}{setpath}{string}
Sets the font search path.
Calls \code{fmsetpath(\var{string})}.
(XXX Does not work!?!)
\end{funcdesc}

\begin{funcdesc}{fontpath}{}
Returns the current font search path.
\end{funcdesc}

Font handle objects support the following operations:

\setindexsubitem{(font handle method)}
\begin{funcdesc}{scalefont}{factor}
Returns a handle for a scaled version of this font.
Calls \code{fmscalefont(\var{fh}, \var{factor})}.
\end{funcdesc}

\begin{funcdesc}{setfont}{}
Makes this font the current font.
Note: the effect is undone silently when the font handle object is
deleted.
Calls \code{fmsetfont(\var{fh})}.
\end{funcdesc}

\begin{funcdesc}{getfontname}{}
Returns this font's name.
Calls \code{fmgetfontname(\var{fh})}.
\end{funcdesc}

\begin{funcdesc}{getcomment}{}
Returns the comment string associated with this font.
Raises an exception if there is none.
Calls \code{fmgetcomment(\var{fh})}.
\end{funcdesc}

\begin{funcdesc}{getfontinfo}{}
Returns a tuple giving some pertinent data about this font.
This is an interface to \code{fmgetfontinfo()}.
The returned tuple contains the following numbers:
\code{(}\var{printermatched}, \var{fixed_width}, \var{xorig},
\var{yorig}, \var{xsize}, \var{ysize}, \var{height},
\var{nglyphs}\code{)}.
\end{funcdesc}

\begin{funcdesc}{getstrwidth}{string}
Returns the width, in pixels, of \var{string} when drawn in this font.
Calls \code{fmgetstrwidth(\var{fh}, \var{string})}.
\end{funcdesc}

\section{Built-in Module \sectcode{gl}}
\label{module-gl}
\bimodindex{gl}

This module provides access to the Silicon Graphics
\emph{Graphics Library}.
It is available only on Silicon Graphics machines.

\strong{Warning:}
Some illegal calls to the GL library cause the Python interpreter to dump
core.
In particular, the use of most GL calls is unsafe before the first
window is opened.

The module is too large to document here in its entirety, but the
following should help you to get started.
The parameter conventions for the C functions are translated to Python as
follows:

\begin{itemize}
\item
All (short, long, unsigned) int values are represented by Python
integers.
\item
All float and double values are represented by Python floating point
numbers.
In most cases, Python integers are also allowed.
\item
All arrays are represented by one-dimensional Python lists.
In most cases, tuples are also allowed.
\item
\begin{sloppypar}
All string and character arguments are represented by Python strings,
for instance,
\code{winopen('Hi There!')}
and
\code{rotate(900, 'z')}.
\end{sloppypar}
\item
All (short, long, unsigned) integer arguments or return values that are
only used to specify the length of an array argument are omitted.
For example, the C call

\bcode\begin{verbatim}
lmdef(deftype, index, np, props)
\end{verbatim}\ecode
%
is translated to Python as

\bcode\begin{verbatim}
lmdef(deftype, index, props)
\end{verbatim}\ecode
%
\item
Output arguments are omitted from the argument list; they are
transmitted as function return values instead.
If more than one value must be returned, the return value is a tuple.
If the C function has both a regular return value (that is not omitted
because of the previous rule) and an output argument, the return value
comes first in the tuple.
Examples: the C call

\bcode\begin{verbatim}
getmcolor(i, &red, &green, &blue)
\end{verbatim}\ecode
%
is translated to Python as

\bcode\begin{verbatim}
red, green, blue = getmcolor(i)
\end{verbatim}\ecode
%
\end{itemize}

The following functions are non-standard or have special argument
conventions:

\renewcommand{\indexsubitem}{(in module gl)}
\begin{funcdesc}{varray}{argument}
%JHXXX the argument-argument added
Equivalent to but faster than a number of
\code{v3d()}
calls.
The \var{argument} is a list (or tuple) of points.
Each point must be a tuple of coordinates
\code{(\var{x}, \var{y}, \var{z})} or \code{(\var{x}, \var{y})}.
The points may be 2- or 3-dimensional but must all have the
same dimension.
Float and int values may be mixed however.
The points are always converted to 3D double precision points
by assuming \code{\var{z} = 0.0} if necessary (as indicated in the man page),
and for each point
\code{v3d()}
is called.
\end{funcdesc}

\begin{funcdesc}{nvarray}{}
Equivalent to but faster than a number of
\code{n3f}
and
\code{v3f}
calls.
The argument is an array (list or tuple) of pairs of normals and points.
Each pair is a tuple of a point and a normal for that point.
Each point or normal must be a tuple of coordinates
\code{(\var{x}, \var{y}, \var{z})}.
Three coordinates must be given.
Float and int values may be mixed.
For each pair,
\code{n3f()}
is called for the normal, and then
\code{v3f()}
is called for the point.
\end{funcdesc}

\begin{funcdesc}{vnarray}{}
Similar to 
\code{nvarray()}
but the pairs have the point first and the normal second.
\end{funcdesc}

\begin{funcdesc}{nurbssurface}{s_k\, t_k\, ctl\, s_ord\, t_ord\, type}
% XXX s_k[], t_k[], ctl[][]
Defines a nurbs surface.
The dimensions of
\code{\var{ctl}[][]}
are computed as follows:
\code{[len(\var{s_k}) - \var{s_ord}]},
\code{[len(\var{t_k}) - \var{t_ord}]}.
\end{funcdesc}

\begin{funcdesc}{nurbscurve}{knots\, ctlpoints\, order\, type}
Defines a nurbs curve.
The length of ctlpoints is
\code{len(\var{knots}) - \var{order}}.
\end{funcdesc}

\begin{funcdesc}{pwlcurve}{points\, type}
Defines a piecewise-linear curve.
\var{points}
is a list of points.
\var{type}
must be
\code{N_ST}.
\end{funcdesc}

\begin{funcdesc}{pick}{n}
\funcline{select}{n}
The only argument to these functions specifies the desired size of the
pick or select buffer.
\end{funcdesc}

\begin{funcdesc}{endpick}{}
\funcline{endselect}{}
These functions have no arguments.
They return a list of integers representing the used part of the
pick/select buffer.
No method is provided to detect buffer overrun.
\end{funcdesc}

Here is a tiny but complete example GL program in Python:

\bcode\begin{verbatim}
import gl, GL, time

def main():
    gl.foreground()
    gl.prefposition(500, 900, 500, 900)
    w = gl.winopen('CrissCross')
    gl.ortho2(0.0, 400.0, 0.0, 400.0)
    gl.color(GL.WHITE)
    gl.clear()
    gl.color(GL.RED)
    gl.bgnline()
    gl.v2f(0.0, 0.0)
    gl.v2f(400.0, 400.0)
    gl.endline()
    gl.bgnline()
    gl.v2f(400.0, 0.0)
    gl.v2f(0.0, 400.0)
    gl.endline()
    time.sleep(5)

main()
\end{verbatim}\ecode
%
\section{Standard Modules \sectcode{GL} and \sectcode{DEVICE}}
\nodename{GL and DEVICE}
\stmodindex{GL}
\stmodindex{DEVICE}

These modules define the constants used by the Silicon Graphics
\emph{Graphics Library}
that C programmers find in the header files
\file{<gl/gl.h>}
and
\file{<gl/device.h>}.
Read the module source files for details.

\section{Built-in Module \module{imgfile}}
\label{module-imgfile}
\bimodindex{imgfile}

The \module{imgfile} module allows Python programs to access SGI imglib image
files (also known as \file{.rgb} files).  The module is far from
complete, but is provided anyway since the functionality that there is
is enough in some cases.  Currently, colormap files are not supported.

The module defines the following variables and functions:

\begin{excdesc}{error}
This exception is raised on all errors, such as unsupported file type, etc.
\end{excdesc}

\begin{funcdesc}{getsizes}{file}
This function returns a tuple \code{(\var{x}, \var{y}, \var{z})} where
\var{x} and \var{y} are the size of the image in pixels and
\var{z} is the number of
bytes per pixel. Only 3 byte RGB pixels and 1 byte greyscale pixels
are currently supported.
\end{funcdesc}

\begin{funcdesc}{read}{file}
This function reads and decodes the image on the specified file, and
returns it as a Python string. The string has either 1 byte greyscale
pixels or 4 byte RGBA pixels. The bottom left pixel is the first in
the string. This format is suitable to pass to \function{gl.lrectwrite()},
for instance.
\end{funcdesc}

\begin{funcdesc}{readscaled}{file, x, y, filter\optional{, blur}}
This function is identical to read but it returns an image that is
scaled to the given \var{x} and \var{y} sizes. If the \var{filter} and
\var{blur} parameters are omitted scaling is done by
simply dropping or duplicating pixels, so the result will be less than
perfect, especially for computer-generated images.

Alternatively, you can specify a filter to use to smoothen the image
after scaling. The filter forms supported are \code{'impulse'},
\code{'box'}, \code{'triangle'}, \code{'quadratic'} and
\code{'gaussian'}. If a filter is specified \var{blur} is an optional
parameter specifying the blurriness of the filter. It defaults to \code{1.0}.

\code{readscaled} makes no
attempt to keep the aspect ratio correct, so that is the users'
responsibility.
\end{funcdesc}

\begin{funcdesc}{ttob}{flag}
This function sets a global flag which defines whether the scan lines
of the image are read or written from bottom to top (flag is zero,
compatible with SGI GL) or from top to bottom(flag is one,
compatible with X).  The default is zero.
\end{funcdesc}

\begin{funcdesc}{write}{file, data, x, y, z}
This function writes the RGB or greyscale data in \var{data} to image
file \var{file}. \var{x} and \var{y} give the size of the image,
\var{z} is 1 for 1 byte greyscale images or 3 for RGB images (which are
stored as 4 byte values of which only the lower three bytes are used).
These are the formats returned by \code{gl.lrectread}.
\end{funcdesc}

\section{Built-in Module \module{jpeg}}
\label{module-jpeg}
\bimodindex{jpeg}

The module \module{jpeg} provides access to the jpeg compressor and
decompressor written by the Independent JPEG Group%
\index{Independent JPEG Group}%
. JPEG is a (draft?)
standard for compressing pictures.  For details on JPEG or the
Independent JPEG Group software refer to the JPEG standard or the
documentation provided with the software.

The \module{jpeg} module defines an exception and some functions.

\begin{excdesc}{error}
Exception raised by \function{compress()} and \function{decompress()}
in case of errors.
\end{excdesc}

\begin{funcdesc}{compress}{data, w, h, b}
Treat data as a pixmap of width \var{w} and height \var{h}, with
\var{b} bytes per pixel.  The data is in SGI GL order, so the first
pixel is in the lower-left corner. This means that \function{gl.lrectread()}
return data can immediately be passed to \function{compress()}.
Currently only 1 byte and 4 byte pixels are allowed, the former being
treated as greyscale and the latter as RGB color.
\function{compress()} returns a string that contains the compressed
picture, in JFIF\index{JFIF} format.
\end{funcdesc}

\begin{funcdesc}{decompress}{data}
Data is a string containing a picture in JFIF\index{JFIF} format. It
returns a tuple \code{(\var{data}, \var{width}, \var{height},
\var{bytesperpixel})}.  Again, the data is suitable to pass to
\function{gl.lrectwrite()}.
\end{funcdesc}

\begin{funcdesc}{setoption}{name, value}
Set various options.  Subsequent \function{compress()} and
\function{decompress()} calls will use these options.  The following
options are available:

\begin{tableii}{l|p{3in}}{code}{Option}{Effect}
  \lineii{'forcegray'}{%
    Force output to be grayscale, even if input is RGB.}
  \lineii{'quality'}{%
    Set the quality of the compressed image to a value between
    \code{0} and \code{100} (default is \code{75}).  This only affects
    compression.}
  \lineii{'optimize'}{%
    Perform Huffman table optimization.  Takes longer, but results in
    smaller compressed image.  This only affects compression.}
  \lineii{'smooth'}{%
    Perform inter-block smoothing on uncompressed image.  Only useful
    for low-quality images.  This only affects decompression.}
\end{tableii}
\end{funcdesc}

%\section{\module{panel} ---
         None}
\declaremodule{standard}{panel}

\modulesynopsis{None}


\strong{Please note:} The FORMS library, to which the
\code{fl}\refbimodindex{fl} module described above interfaces, is a
simpler and more accessible user interface library for use with GL
than the \code{panel} module (besides also being by a Dutch author).

This module should be used instead of the built-in module
\code{pnl}\refbimodindex{pnl}
to interface with the
\emph{Panel Library}.

The module is too large to document here in its entirety.
One interesting function:

\begin{funcdesc}{defpanellist}{filename}
Parses a panel description file containing S-expressions written by the
\emph{Panel Editor}
that accompanies the Panel Library and creates the described panels.
It returns a list of panel objects.
\end{funcdesc}

\strong{Warning:}
the Python interpreter will dump core if you don't create a GL window
before calling
\code{panel.mkpanel()}
or
\code{panel.defpanellist()}.

\section{\module{panelparser} ---
         None}
\declaremodule{standard}{panelparser}

\modulesynopsis{None}


This module defines a self-contained parser for S-expressions as output
by the Panel Editor (which is written in Scheme so it can't help writing
S-expressions).
The relevant function is
\code{panelparser.parse_file(\var{file})}
which has a file object (not a filename!) as argument and returns a list
of parsed S-expressions.
Each S-expression is converted into a Python list, with atoms converted
to Python strings and sub-expressions (recursively) to Python lists.
For more details, read the module file.
% XXXXJH should be funcdesc, I think

\section{\module{pnl} ---
         None}
\declaremodule{builtin}{pnl}

\modulesynopsis{None}


This module provides access to the
\emph{Panel Library}
built by NASA Ames\index{NASA} (to get it, send e-mail to
\code{panel-request@nas.nasa.gov}).
All access to it should be done through the standard module
\code{panel}\refstmodindex{panel},
which transparantly exports most functions from
\code{pnl}
but redefines
\code{pnl.dopanel()}.

\strong{Warning:}
the Python interpreter will dump core if you don't create a GL window
before calling
\code{pnl.mkpanel()}.

The module is too large to document here in its entirety.


\chapter{SunOS Specific Services}
\label{sunos}

The modules described in this chapter provide interfaces to features
that are unique to SunOS 5 (also known as Solaris version 2).
			% SUNOS ONLY
\section{\module{sunaudiodev} ---
         Access to Sun audio hardware.}
\declaremodule{builtin}{sunaudiodev}

\modulesynopsis{Access to Sun audio hardware.}


This module allows you to access the Sun audio interface. The Sun
audio hardware is capable of recording and playing back audio data
in u-LAW\index{u-LAW} format with a sample rate of 8K per second. A
full description can be found in the \manpage{audio}{7I} manual page.

The module defines the following variables and functions:

\begin{excdesc}{error}
This exception is raised on all errors. The argument is a string
describing what went wrong.
\end{excdesc}

\begin{funcdesc}{open}{mode}
This function opens the audio device and returns a Sun audio device
object. This object can then be used to do I/O on. The \var{mode} parameter
is one of \code{'r'} for record-only access, \code{'w'} for play-only
access, \code{'rw'} for both and \code{'control'} for access to the
control device. Since only one process is allowed to have the recorder
or player open at the same time it is a good idea to open the device
only for the activity needed. See \manpage{audio}{7I} for details.

As per the manpage, this module first looks in the environment
variable \code{AUDIODEV} for the base audio device filename.  If not
found, it falls back to \file{/dev/audio}.  The control device is
calculated by appending ``ctl'' to the base audio device.
\end{funcdesc}


\subsection{Audio Device Objects}
\label{audio-device-objects}

The audio device objects are returned by \function{open()} define the
following methods (except \code{control} objects which only provide
\method{getinfo()}, \method{setinfo()}, \method{fileno()}, and
\method{drain()}):

\begin{methoddesc}[audio device]{close}{}
This method explicitly closes the device. It is useful in situations
where deleting the object does not immediately close it since there
are other references to it. A closed device should not be used again.
\end{methoddesc}

\begin{methoddesc}[audio device]{fileno}{}
Returns the file descriptor associated with the device.  This can be
used to set up \code{SIGPOLL} notification, as described below.
\end{methoddesc}

\begin{methoddesc}[audio device]{drain}{}
This method waits until all pending output is processed and then returns.
Calling this method is often not necessary: destroying the object will
automatically close the audio device and this will do an implicit drain.
\end{methoddesc}

\begin{methoddesc}[audio device]{flush}{}
This method discards all pending output. It can be used avoid the
slow response to a user's stop request (due to buffering of up to one
second of sound).
\end{methoddesc}

\begin{methoddesc}[audio device]{getinfo}{}
This method retrieves status information like input and output volume,
etc. and returns it in the form of
an audio status object. This object has no methods but it contains a
number of attributes describing the current device status. The names
and meanings of the attributes are described in
\file{/usr/include/sun/audioio.h} and in the \manpage{audio}{7I}
manual page.  Member names
are slightly different from their \C{} counterparts: a status object is
only a single structure. Members of the \cdata{play} substructure have
\samp{o_} prepended to their name and members of the \cdata{record}
structure have \samp{i_}. So, the \C{} member \cdata{play.sample_rate} is
accessed as \member{o_sample_rate}, \cdata{record.gain} as \member{i_gain}
and \cdata{monitor_gain} plainly as \member{monitor_gain}.
\end{methoddesc}

\begin{methoddesc}[audio device]{ibufcount}{}
This method returns the number of samples that are buffered on the
recording side, i.e.\ the program will not block on a
\function{read()} call of so many samples.
\end{methoddesc}

\begin{methoddesc}[audio device]{obufcount}{}
This method returns the number of samples buffered on the playback
side. Unfortunately, this number cannot be used to determine a number
of samples that can be written without blocking since the kernel
output queue length seems to be variable.
\end{methoddesc}

\begin{methoddesc}[audio device]{read}{size}
This method reads \var{size} samples from the audio input and returns
them as a Python string. The function blocks until enough data is available.
\end{methoddesc}

\begin{methoddesc}[audio device]{setinfo}{status}
This method sets the audio device status parameters. The \var{status}
parameter is an device status object as returned by \function{getinfo()} and
possibly modified by the program.
\end{methoddesc}

\begin{methoddesc}[audio device]{write}{samples}
Write is passed a Python string containing audio samples to be played.
If there is enough buffer space free it will immediately return,
otherwise it will block.
\end{methoddesc}

There is a companion module,
\module{SUNAUDIODEV}\refstmodindex{SUNAUDIODEV}, which defines useful
symbolic constants like \constant{MIN_GAIN}, \constant{MAX_GAIN},
\constant{SPEAKER}, etc. The names of the constants are the same names
as used in the \C{} include file \code{<sun/audioio.h>}, with the
leading string \samp{AUDIO_} stripped.

The audio device supports asynchronous notification of various events,
through the SIGPOLL signal.  Here's an example of how you might enable 
this in Python:

\begin{verbatim}
def handle_sigpoll(signum, frame):
    print 'I got a SIGPOLL update'
pp
import fcntl, signal, STROPTS

signal.signal(signal.SIGPOLL, handle_sigpoll)
fcntl.ioctl(audio_obj.fileno(), STROPTS.I_SETSIG, STROPTS.S_MSG)
\end{verbatim}


\chapter{Building C and \Cpp{} Extensions on Windows%
     \label{building-on-windows}}


This chapter briefly explains how to create a Windows extension module
for Python using Microsoft Visual \Cpp, and follows with more
detailed background information on how it works.  The explanatory
material is useful for both the Windows programmer learning to build
Python extensions and the \UNIX{} programmer interested in producing
software which can be successfully built on both \UNIX{} and Windows.

Module authors are encouraged to use the distutils approach for
building extension modules, instead of the one described in this
section. You will still need the C compiler that was used to build
Python; typically Microsoft Visual \Cpp.

\begin{notice}
  This chapter mentions a number of filenames that include an encoded
  Python version number.  These filenames are represented with the
  version number shown as \samp{XY}; in practive, \character{X} will
  be the major version number and \character{Y} will be the minor
  version number of the Python release you're working with.  For
  example, if you are using Python 2.2.1, \samp{XY} will actually be
  \samp{22}.
\end{notice}


\section{A Cookbook Approach \label{win-cookbook}}

There are two approaches to building extension modules on Windows,
just as there are on \UNIX: use the
\ulink{\module{distutils}}{../lib/module-distutils.html} package to
control the build process, or do things manually.  The distutils
approach works well for most extensions; documentation on using
\ulink{\module{distutils}}{../lib/module-distutils.html} to build and
package extension modules is available in
\citetitle[../dist/dist.html]{Distributing Python Modules}.  This
section describes the manual approach to building Python extensions
written in C or \Cpp.

To build extensions using these instructions, you need to have a copy
of the Python sources of the same version as your installed Python.
You will need Microsoft Visual \Cpp{} ``Developer Studio''; project
files are supplied for V\Cpp{} version 7.1, but you can use older
versions of V\Cpp.  Notice that you should use the same version of
V\Cpp that was used to build Python itself. The example files
described here are distributed with the Python sources in the
\file{PC\textbackslash example_nt\textbackslash} directory.

\begin{enumerate}
  \item
  \strong{Copy the example files}\\
    The \file{example_nt} directory is a subdirectory of the \file{PC}
    directory, in order to keep all the PC-specific files under the
    same directory in the source distribution.  However, the
    \file{example_nt} directory can't actually be used from this
    location.  You first need to copy or move it up one level, so that
    \file{example_nt} is a sibling of the \file{PC} and \file{Include}
    directories.  Do all your work from within this new location.

  \item
  \strong{Open the project}\\
    From V\Cpp, use the \menuselection{File \sub Open Solution}
    dialog (not \menuselection{File \sub Open}!).  Navigate to and
    select the file \file{example.sln}, in the \emph{copy} of the
    \file{example_nt} directory you made above.  Click Open.

  \item
  \strong{Build the example DLL}\\
    In order to check that everything is set up right, try building:

    \begin{enumerate}
      \item
        Select a configuration.  This step is optional.  Choose
        \menuselection{Build \sub Configuration Manager \sub Active 
        Solution Configuration} and select either \guilabel{Release} 
        or\guilabel{Debug}.  If you skip this step,
        V\Cpp{} will use the Debug configuration by default.

      \item
        Build the DLL.  Choose \menuselection{Build \sub Build
        Solution}.  This creates all intermediate and result files in
        a subdirectory called either \file{Debug} or \file{Release},
        depending on which configuration you selected in the preceding
        step.
    \end{enumerate}

  \item
  \strong{Testing the debug-mode DLL}\\
    Once the Debug build has succeeded, bring up a DOS box, and change
    to the \file{example_nt\textbackslash Debug} directory.  You
    should now be able to repeat the following session (\code{C>} is
    the DOS prompt, \code{>\code{>}>} is the Python prompt; note that
    build information and various debug output from Python may not
    match this screen dump exactly):

\begin{verbatim}
C>..\..\PCbuild\python_d
Adding parser accelerators ...
Done.
Python 2.2 (#28, Dec 19 2001, 23:26:37) [MSC 32 bit (Intel)] on win32
Type "copyright", "credits" or "license" for more information.
>>> import example
[4897 refs]
>>> example.foo()
Hello, world
[4903 refs]
>>>
\end{verbatim}

    Congratulations!  You've successfully built your first Python
    extension module.

  \item
  \strong{Creating your own project}\\
    Choose a name and create a directory for it.  Copy your C sources
    into it.  Note that the module source file name does not
    necessarily have to match the module name, but the name of the
    initialization function should match the module name --- you can
    only import a module \module{spam} if its initialization function
    is called \cfunction{initspam()}, and it should call
    \cfunction{Py_InitModule()} with the string \code{"spam"} as its
    first argument (use the minimal \file{example.c} in this directory
    as a guide).  By convention, it lives in a file called
    \file{spam.c} or \file{spammodule.c}.  The output file should be
    called \file{spam.dll} or \file{spam.pyd} (the latter is supported
    to avoid confusion with a system library \file{spam.dll} to which
    your module could be a Python interface) in Release mode, or
    \file{spam_d.dll} or \file{spam_d.pyd} in Debug mode.

    Now your options are:

    \begin{enumerate}
      \item  Copy \file{example.sln} and \file{example.vcproj}, rename
             them to \file{spam.*}, and edit them by hand, or
      \item  Create a brand new project; instructions are below.
    \end{enumerate}

    In either case, copy \file{example_nt\textbackslash example.def}
    to \file{spam\textbackslash spam.def}, and edit the new
    \file{spam.def} so its second line contains the string
    `\code{initspam}'.  If you created a new project yourself, add the
    file \file{spam.def} to the project now.  (This is an annoying
    little file with only two lines.  An alternative approach is to
    forget about the \file{.def} file, and add the option
    \programopt{/export:initspam} somewhere to the Link settings, by
    manually editing the setting in Project Properties dialog).

  \item
  \strong{Creating a brand new project}\\
    Use the \menuselection{File \sub New \sub Project} dialog to
    create a new Project Workspace.  Select \guilabel{Visual C++
    Projects/Win32/ Win32 Project}, enter the name (\samp{spam}), and
    make sure the Location is set to parent of the \file{spam}
    directory you have created (which should be a direct subdirectory
    of the Python build tree, a sibling of \file{Include} and
    \file{PC}).  Select Win32 as the platform (in my version, this is
    the only choice).  Make sure the Create new workspace radio button
    is selected.  Click OK.

    You should now create the file \file{spam.def} as instructed in
    the previous section. Add the source files to the project, using
    \menuselection{Project \sub Add Existing Item}. Set the pattern to
    \code{*.*} and select both \file{spam.c} and \file{spam.def} and
    click OK.  (Inserting them one by one is fine too.)

    Now open the \menuselection{Project \sub spam properties} dialog.
    You only need to change a few settings.  Make sure \guilabel{All
    Configurations} is selected from the \guilabel{Settings for:}
    dropdown list.  Select the C/\Cpp{} tab.  Choose the General
    category in the popup menu at the top.  Type the following text in
    the entry box labeled \guilabel{Additional Include Directories}:

\begin{verbatim}
..\Include,..\PC
\end{verbatim}

    Then, choose the General category in the Linker tab, and enter

\begin{verbatim}
..\PCbuild
\end{verbatim}

    in the text box labelled \guilabel{Additional library Directories}.

    Now you need to add some mode-specific settings:

    Select \guilabel{Release} in the \guilabel{Configuration}
    dropdown list.  Choose the \guilabel{Link} tab, choose the
    \guilabel{Input} category, and append \code{pythonXY.lib} to the
    list in the \guilabel{Additional Dependencies} box.

    Select \guilabel{Debug} in the \guilabel{Configuration} dropdown
    list, and append \code{pythonXY_d.lib} to the list in the
    \guilabel{Additional Dependencies} box.  Then click the C/\Cpp{}
    tab, select \guilabel{Code Generation}, and select
    \guilabel{Multi-threaded Debug DLL} from the \guilabel{Runtime
    library} dropdown list.

    Select \guilabel{Release} again from the \guilabel{Configuration}
    dropdown list.  Select \guilabel{Multi-threaded DLL} from the
    \guilabel{Runtime library} dropdown list.
\end{enumerate}


If your module creates a new type, you may have trouble with this line:

\begin{verbatim}
    PyObject_HEAD_INIT(&PyType_Type)
\end{verbatim}

Change it to:

\begin{verbatim}
    PyObject_HEAD_INIT(NULL)
\end{verbatim}

and add the following to the module initialization function:

\begin{verbatim}
    MyObject_Type.ob_type = &PyType_Type;
\end{verbatim}

Refer to section~3 of the
\citetitle[http://www.python.org/doc/FAQ.html]{Python FAQ} for details
on why you must do this.


\section{Differences Between \UNIX{} and Windows
     \label{dynamic-linking}}
\sectionauthor{Chris Phoenix}{cphoenix@best.com}


\UNIX{} and Windows use completely different paradigms for run-time
loading of code.  Before you try to build a module that can be
dynamically loaded, be aware of how your system works.

In \UNIX, a shared object (\file{.so}) file contains code to be used by the
program, and also the names of functions and data that it expects to
find in the program.  When the file is joined to the program, all
references to those functions and data in the file's code are changed
to point to the actual locations in the program where the functions
and data are placed in memory.  This is basically a link operation.

In Windows, a dynamic-link library (\file{.dll}) file has no dangling
references.  Instead, an access to functions or data goes through a
lookup table.  So the DLL code does not have to be fixed up at runtime
to refer to the program's memory; instead, the code already uses the
DLL's lookup table, and the lookup table is modified at runtime to
point to the functions and data.

In \UNIX, there is only one type of library file (\file{.a}) which
contains code from several object files (\file{.o}).  During the link
step to create a shared object file (\file{.so}), the linker may find
that it doesn't know where an identifier is defined.  The linker will
look for it in the object files in the libraries; if it finds it, it
will include all the code from that object file.

In Windows, there are two types of library, a static library and an
import library (both called \file{.lib}).  A static library is like a
\UNIX{} \file{.a} file; it contains code to be included as necessary.
An import library is basically used only to reassure the linker that a
certain identifier is legal, and will be present in the program when
the DLL is loaded.  So the linker uses the information from the
import library to build the lookup table for using identifiers that
are not included in the DLL.  When an application or a DLL is linked,
an import library may be generated, which will need to be used for all
future DLLs that depend on the symbols in the application or DLL.

Suppose you are building two dynamic-load modules, B and C, which should
share another block of code A.  On \UNIX, you would \emph{not} pass
\file{A.a} to the linker for \file{B.so} and \file{C.so}; that would
cause it to be included twice, so that B and C would each have their
own copy.  In Windows, building \file{A.dll} will also build
\file{A.lib}.  You \emph{do} pass \file{A.lib} to the linker for B and
C.  \file{A.lib} does not contain code; it just contains information
which will be used at runtime to access A's code.  

In Windows, using an import library is sort of like using \samp{import
spam}; it gives you access to spam's names, but does not create a
separate copy.  On \UNIX, linking with a library is more like
\samp{from spam import *}; it does create a separate copy.


\section{Using DLLs in Practice \label{win-dlls}}
\sectionauthor{Chris Phoenix}{cphoenix@best.com}

Windows Python is built in Microsoft Visual \Cpp; using other
compilers may or may not work (though Borland seems to).  The rest of
this section is MSV\Cpp{} specific.

When creating DLLs in Windows, you must pass \file{pythonXY.lib} to
the linker.  To build two DLLs, spam and ni (which uses C functions
found in spam), you could use these commands:

\begin{verbatim}
cl /LD /I/python/include spam.c ../libs/pythonXY.lib
cl /LD /I/python/include ni.c spam.lib ../libs/pythonXY.lib
\end{verbatim}

The first command created three files: \file{spam.obj},
\file{spam.dll} and \file{spam.lib}.  \file{Spam.dll} does not contain
any Python functions (such as \cfunction{PyArg_ParseTuple()}), but it
does know how to find the Python code thanks to \file{pythonXY.lib}.

The second command created \file{ni.dll} (and \file{.obj} and
\file{.lib}), which knows how to find the necessary functions from
spam, and also from the Python executable.

Not every identifier is exported to the lookup table.  If you want any
other modules (including Python) to be able to see your identifiers,
you have to say \samp{_declspec(dllexport)}, as in \samp{void
_declspec(dllexport) initspam(void)} or \samp{PyObject
_declspec(dllexport) *NiGetSpamData(void)}.

Developer Studio will throw in a lot of import libraries that you do
not really need, adding about 100K to your executable.  To get rid of
them, use the Project Settings dialog, Link tab, to specify
\emph{ignore default libraries}.  Add the correct
\file{msvcrt\var{xx}.lib} to the list of libraries.
			% MS Windows ONLY
\section{\module{msvcrt} --
         Useful routines from the MS VC++ runtime}

\declaremodule{builtin}{msvcrt}
  \platform{Windows}
\modulesynopsis{Miscellaneous useful routines from the MS VC++ runtime.}
\sectionauthor{Fred L. Drake, Jr.}{fdrake@acm.org}


These functions provide access to some useful capabilities on Windows
platforms.  Some higher-level modules use these functions to build the 
Windows implementations of their services.  For example, the
\refmodule{getpass} module uses this in the implementation of the
\function{getpass()} function.

Further documentation on these functions can be found in the Platform
API documentation.


\subsection{File Operations \label{msvcrt-files}}

\begin{funcdesc}{locking}{fd, mode, nbytes}
  Lock part of a file based on a file descriptor from the C runtime.
  Raises \exception{IOError} on failure.
\end{funcdesc}

\begin{funcdesc}{setmode}{fd, flags}
  Set the line-end translation mode for the file descriptor \var{fd}.
  To set it to text mode, \var{flags} should be \constant{os.O_TEXT};
  for binary, it should be \constant{os.O_BINARY}.
\end{funcdesc}

\begin{funcdesc}{open_osfhandle}{handle, flags}
  Create a C runtime file descriptor from the file handle
  \var{handle}.  The \var{flags} parameter should be a bit-wise OR of
  \constant{os.O_APPEND}, \constant{os.O_RDONLY}, and
  \constant{os.O_TEXT}.  The returned file descriptor may be used as a
  parameter to \function{os.fdopen()} to create a file object.
\end{funcdesc}

\begin{funcdesc}{get_osfhandle}{fd}
  Return the file handle for the file descriptor \var{fd}.  Raises
  \exception{IOError} if \var{fd} is not recognized.
\end{funcdesc}


\subsection{Console I/O \label{msvcrt-console}}

\begin{funcdesc}{kbhit}{}
  Return true if a keypress is waiting to be read.
\end{funcdesc}

\begin{funcdesc}{getch}{}
  Read a keypress and return the resulting character.  Nothing is
  echoed to the console.  This call will block if a keypress is not
  already available, but will not wait for \kbd{Enter} to be pressed.
  If the pressed key was a special function key, this will return
  \code{'\e000'} or \code{'\e xe0'}; the next call will return the
  keycode.  The \kbd{Control-C} keypress cannot be read with this
  function.
\end{funcdesc}

\begin{funcdesc}{getche}{}
  Similar to \function{getch()}, but the keypress will be echoed if it 
  represents a printable character.
\end{funcdesc}

\begin{funcdesc}{putch}{char}
  Print the character \var{char} to the console without buffering.
\end{funcdesc}

\begin{funcdesc}{ungetch}{char}
  Cause the character \var{char} to be ``pushed back'' into the
  console buffer; it will be the next character read by
  \function{getch()} or \function{getche()}.
\end{funcdesc}


\subsection{Other Functions \label{msvcrt-other}}

\begin{funcdesc}{heapmin}{}
  Force the \cfunction{malloc()} heap to clean itself up and return
  unused blocks to the operating system.  This only works on Windows
  NT.  On failure, this raises \exception{IOError}.
\end{funcdesc}

\section{\module{winsound} ---
         Sound-playing interface for Windows}

\declaremodule{builtin}{winsound}
  \platform{Windows}
\modulesynopsis{Access to the sound-playing machinery for Windows.}
\moduleauthor{Toby Dickenson}{htrd90@zepler.org}
\sectionauthor{Fred L. Drake, Jr.}{fdrake@acm.org}

\versionadded{1.5.2}

The \module{winsound} module provides access to the basic
sound-playing machinery provided by Windows platforms.  It includes
two functions and several constants.


\begin{funcdesc}{Beep}{frequency, duration}
  Beep the PC's speaker.
  The \var{frequency} parameter specifies frequency, in hertz, of the
  sound, and must be in the range 37 through 32,767.
  The \var{duration} parameter specifies the number of milliseconds the
  sound should last.  If the system is not
  able to beep the speaker, \exception{RuntimeError} is raised.
  \note{Under Windows 95 and 98, the Windows \cfunction{Beep()}
  function exists but is useless (it ignores its arguments).  In that
  case Python simulates it via direct port manipulation (added in version
  2.1).  It's unknown whether that will work on all systems.}
  \versionadded{1.6}
\end{funcdesc}

\begin{funcdesc}{PlaySound}{sound, flags}
  Call the underlying \cfunction{PlaySound()} function from the
  Platform API.  The \var{sound} parameter may be a filename, audio
  data as a string, or \code{None}.  Its interpretation depends on the
  value of \var{flags}, which can be a bit-wise ORed combination of
  the constants described below.  If the system indicates an error,
  \exception{RuntimeError} is raised.
\end{funcdesc}


\begin{datadesc}{SND_FILENAME}
  The \var{sound} parameter is the name of a WAV file.
  Do not use with \constant{SND_ALIAS}.
\end{datadesc}

\begin{datadesc}{SND_ALIAS}
  The \var{sound} parameter is a sound association name from the
  registry.  If the registry contains no such name, play the system
  default sound unless \constant{SND_NODEFAULT} is also specified.
  If no default sound is registered, raise \exception{RuntimeError}.
  Do not use with \constant{SND_FILENAME}.

  All Win32 systems support at least the following; most systems support
  many more:

\begin{tableii}{l|l}{code}
               {\function{PlaySound()} \var{name}}
               {Corresponding Control Panel Sound name}
  \lineii{'SystemAsterisk'}   {Asterisk}
  \lineii{'SystemExclamation'}{Exclamation}
  \lineii{'SystemExit'}       {Exit Windows}
  \lineii{'SystemHand'}       {Critical Stop}
  \lineii{'SystemQuestion'}   {Question}
\end{tableii}

  For example:

\begin{verbatim}
import winsound
# Play Windows exit sound.
winsound.PlaySound("SystemExit", winsound.SND_ALIAS)

# Probably play Windows default sound, if any is registered (because
# "*" probably isn't the registered name of any sound).
winsound.PlaySound("*", winsound.SND_ALIAS)
\end{verbatim}
\end{datadesc}

\begin{datadesc}{SND_LOOP}
  Play the sound repeatedly.  The \constant{SND_ASYNC} flag must also
  be used to avoid blocking.  Cannot be used with \constant{SND_MEMORY}.
\end{datadesc}

\begin{datadesc}{SND_MEMORY}
  The \var{sound} parameter to \function{PlaySound()} is a memory
  image of a WAV file, as a string.

  \note{This module does not support playing from a memory
  image asynchronously, so a combination of this flag and
  \constant{SND_ASYNC} will raise \exception{RuntimeError}.}
\end{datadesc}

\begin{datadesc}{SND_PURGE}
  Stop playing all instances of the specified sound.
\end{datadesc}

\begin{datadesc}{SND_ASYNC}
  Return immediately, allowing sounds to play asynchronously.
\end{datadesc}

\begin{datadesc}{SND_NODEFAULT}
  If the specified sound cannot be found, do not play the system default
  sound.
\end{datadesc}

\begin{datadesc}{SND_NOSTOP}
  Do not interrupt sounds currently playing.
\end{datadesc}

\begin{datadesc}{SND_NOWAIT}
  Return immediately if the sound driver is busy.
\end{datadesc}


\chapter{Undocumented Modules \label{undoc}}

Here's a quick listing of modules that are currently undocumented, but
that should be documented.  Feel free to contribute documentation for
them!  (Send via email to \email{python-docs@python.org}.)

The idea and original contents for this chapter were taken
from a posting by Fredrik Lundh; the specific contents of this chapter
have been substantially revised.


\section{Frameworks}

Frameworks tend to be harder to document, but are well worth the
effort spent.

\begin{description}
\item[\module{Tkinter}]
--- Interface to Tcl/Tk for graphical user interfaces; Fredrik Lundh
is working on this one!  See
\citetitle[http://www.pythonware.com/library.htm]{An Introduction to
Tkinter} at \url{http://www.pythonware.com/library.htm} for on-line
reference material.

\item[\module{Tkdnd}]
--- Drag-and-drop support for \module{Tkinter}.

\item[\module{turtle}]
--- Turtle graphics in a Tk window.

\item[\module{test}]
--- Regression testing framework.  This is used for the Python
regression test, but is useful for other Python libraries as well.
This is a package rather than a single module.
\end{description}


\section{Miscellaneous useful utilities}

Some of these are very old and/or not very robust; marked with ``hmm.''

\begin{description}
\item[\module{bdb}]
--- A generic Python debugger base class (used by pdb).

\item[\module{ihooks}]
--- Import hook support (for \refmodule{rexec}; may become obsolete).

\item[\module{smtpd}]
--- An SMTP daemon implementation which meets the minimum requirements
for \rfc{821} conformance.
\end{description}


\section{Platform specific modules}

These modules are used to implement the \refmodule{os.path} module,
and are not documented beyond this mention.  There's little need to
document these.

\begin{description}
\item[\module{dospath}]
--- Implementation of \module{os.path} on MS-DOS.

\item[\module{ntpath}]
--- Implementation on \module{os.path} on Win32, Win64, WinCE, and
OS/2 platforms.

\item[\module{posixpath}]
--- Implementation on \module{os.path} on \POSIX.
\end{description}


\section{Multimedia}

\begin{description}
\item[\module{audiodev}]
--- Platform-independent API for playing audio data.

\item[\module{sunaudio}]
--- Interpret Sun audio headers (may become obsolete or a tool/demo).

\item[\module{toaiff}]
--- Convert "arbitrary" sound files to AIFF files; should probably
become a tool or demo.  Requires the external program \program{sox}.
\end{description}


\section{Obsolete \label{obsolete-modules}}

These modules are not normally available for import; additional work
must be done to make them available.

Those which are written in Python will be installed into the directory 
\file{lib-old/} installed as part of the standard library.  To use
these, the directory must be added to \code{sys.path}, possibly using
\envvar{PYTHONPATH}.

Obsolete extension modules written in C are not built by default.
Under \UNIX, these must be enabled by uncommenting the appropriate
lines in \file{Modules/Setup} in the build tree and either rebuilding
Python if the modules are statically linked, or building and
installing the shared object if using dynamically-loaded extensions.

% XXX need Windows instructions!

\begin{description}
\item[\module{addpack}]
--- Alternate approach to packages.  Use the built-in package support
instead.

\item[\module{cmp}]
--- File comparison function.  Use the newer \refmodule{filecmp} instead.

\item[\module{cmpcache}]
--- Caching version of the obsolete \module{cmp} module.  Use the
newer \refmodule{filecmp} instead.

\item[\module{codehack}]
--- Extract function name or line number from a function
code object (these are now accessible as attributes:
\member{co.co_name}, \member{func.func_name},
\member{co.co_firstlineno}).

\item[\module{dircmp}]
--- Class to build directory diff tools on (may become a demo or tool).
\deprecated{2.0}{The \refmodule{filecmp} module replaces
\module{dircmp}.}

\item[\module{dump}]
--- Print python code that reconstructs a variable.

\item[\module{fmt}]
--- Text formatting abstractions (too slow).

\item[\module{lockfile}]
--- Wrapper around FCNTL file locking (use
\function{fcntl.lockf()}/\function{flock()} instead; see \refmodule{fcntl}).

\item[\module{newdir}]
--- New \function{dir()} function (the standard \function{dir()} is
now just as good).

\item[\module{Para}]
--- Helper for \module{fmt}.

\item[\module{poly}]
--- Polynomials.

\item[\module{regex}]
--- Emacs-style regular expression support; may still be used in some
old code (extension module).  Refer to the
\citetitle[http://www.python.org/doc/1.6/lib/module-regex.html]{Python
1.6 Documentation} for documentation.

\item[\module{regsub}]
--- Regular expression based string replacement utilities, for use
with \module{regex} (extension module).  Refer to the
\citetitle[http://www.python.org/doc/1.6/lib/module-regsub.html]{Python
1.6 Documentation} for documentation.

\item[\module{tb}]
--- Print tracebacks, with a dump of local variables (use
\function{pdb.pm()} or \refmodule{traceback} instead).

\item[\module{timing}]
--- Measure time intervals to high resolution (use
\function{time.clock()} instead).  (This is an extension module.)

\item[\module{tzparse}]
--- Parse a timezone specification (unfinished; may disappear in the
future, and does not work when the \envvar{TZ} environment variable is
not set).

\item[\module{util}]
--- Useful functions that don't fit elsewhere.

\item[\module{whatsound}]
--- Recognize sound files; use \refmodule{sndhdr} instead.

\item[\module{zmod}]
--- Compute properties of mathematical ``fields.''
\end{description}


The following modules are obsolete, but are likely to re-surface as
tools or scripts:

\begin{description}
\item[\module{find}]
--- Find files matching pattern in directory tree.

\item[\module{grep}]
--- \program{grep} implementation in Python.

\item[\module{packmail}]
--- Create a self-unpacking \UNIX{} shell archive.
\end{description}


The following modules were documented in previous versions of this
manual, but are now considered obsolete.  The source for the
documentation is still available as part of the documentation source
archive.

\begin{description}
\item[\module{ni}]
--- Import modules in ``packages.''  Basic package support is now
built in.  The built-in support is very similar to what is provided in
this module.

\item[\module{rand}]
--- Old interface to the random number generator.

\item[\module{soundex}]
--- Algorithm for collapsing names which sound similar to a shared
key.  The specific algorithm doesn't seem to match any published
algorithm.  (This is an extension module.)
\end{description}


\section{SGI-specific Extension modules}

The following are SGI specific, and may be out of touch with the
current version of reality.

\begin{description}
\item[\module{cl}]
--- Interface to the SGI compression library.

\item[\module{sv}]
--- Interface to the ``simple video'' board on SGI Indigo
(obsolete hardware).
\end{description}


%
%  The ugly "%begin{latexonly}" pseudo-environments are really just to
%  keep LaTeX2HTML quiet during the \renewcommand{} macros; they're
%  not really valuable.
%

%begin{latexonly}
\renewcommand{\indexname}{Module Index}
%end{latexonly}
\input{modlib.ind}		% Module Index

%begin{latexonly}
\renewcommand{\indexname}{Index}
%end{latexonly}
\documentclass{manual}

% NOTE: this file controls which chapters/sections of the library
% manual are actually printed.  It is easy to customize your manual
% by commenting out sections that you're not interested in.

\title{Python Library Reference}

\author{Guido van Rossum\\
	Fred L. Drake, Jr., editor}
\authoraddress{
	PythonLabs\\
	E-mail: \email{python-docs@python.org}
}

\date{\today}		% XXX update before release!
\release{2.1 alpha}		% software release, not documentation
\setshortversion{2.1}		% major.minor only for software


\makeindex			% tell \index to actually write the
				% .idx file
\makemodindex			% ... and the module index as well.


\begin{document}

\maketitle

\ifhtml
\chapter*{Front Matter\label{front}}
\fi

Copyright \copyright{} 2001-2006 Python Software Foundation.
All rights reserved.

Copyright \copyright{} 2000 BeOpen.com.
All rights reserved.

Copyright \copyright{} 1995-2000 Corporation for National Research Initiatives.
All rights reserved.

Copyright \copyright{} 1991-1995 Stichting Mathematisch Centrum.
All rights reserved.

See the end of this document for complete license and permissions
information.


\begin{abstract}

\noindent
Python is an extensible, interpreted, object-oriented programming
language.  It supports a wide range of applications, from simple text
processing scripts to interactive WWW browsers.

While the \emph{Python Reference Manual} describes the exact syntax and
semantics of the language, it does not describe the standard library
that is distributed with the language, and which greatly enhances its
immediate usability.  This library contains built-in modules (written
in C) that provide access to system functionality such as file I/O
that would otherwise be inaccessible to Python programmers, as well as
modules written in Python that provide standardized solutions for many
problems that occur in everyday programming.  Some of these modules
are explicitly designed to encourage and enhance the portability of
Python programs.

This library reference manual documents Python's standard library, as
well as many optional library modules (which may or may not be
available, depending on whether the underlying platform supports them
and on the configuration choices made at compile time).  It also
documents the standard types of the language and its built-in
functions and exceptions, many of which are not or incompletely
documented in the Reference Manual.

This manual assumes basic knowledge about the Python language.  For an
informal introduction to Python, see the \emph{Python Tutorial}; the
\emph{Python Reference Manual} remains the highest authority on
syntactic and semantic questions.  Finally, the manual entitled
\emph{Extending and Embedding the Python Interpreter} describes how to
add new extensions to Python and how to embed it in other applications.

\end{abstract}

\tableofcontents

				% Chapter title:

\chapter{Introduction}
\label{intro}

The ``Python library'' contains several different kinds of components.

It contains data types that would normally be considered part of the
``core'' of a language, such as numbers and lists.  For these types,
the Python language core defines the form of literals and places some
constraints on their semantics, but does not fully define the
semantics.  (On the other hand, the language core does define
syntactic properties like the spelling and priorities of operators.)

The library also contains built-in functions and exceptions ---
objects that can be used by all Python code without the need of an
\keyword{import} statement.  Some of these are defined by the core
language, but many are not essential for the core semantics and are
only described here.

The bulk of the library, however, consists of a collection of modules.
There are many ways to dissect this collection.  Some modules are
written in C and built in to the Python interpreter; others are
written in Python and imported in source form.  Some modules provide
interfaces that are highly specific to Python, like printing a stack
trace; some provide interfaces that are specific to particular
operating systems, like socket I/O; others provide interfaces that are
specific to a particular application domain, like the World-Wide Web.
Some modules are avaiable in all versions and ports of Python; others
are only available when the underlying system supports or requires
them; yet others are available only when a particular configuration
option was chosen at the time when Python was compiled and installed.

This manual is organized ``from the inside out'': it first describes
the built-in data types, then the built-in functions and exceptions,
and finally the modules, grouped in chapters of related modules.  The
ordering of the chapters as well as the ordering of the modules within
each chapter is roughly from most relevant to least important.

This means that if you start reading this manual from the start, and
skip to the next chapter when you get bored, you will get a reasonable
overview of the available modules and application areas that are
supported by the Python library.  Of course, you don't \emph{have} to
read it like a novel --- you can also browse the table of contents (in
front of the manual), or look for a specific function, module or term
in the index (in the back).  And finally, if you enjoy learning about
random subjects, you choose a random page number (see module
\module{rand}) and read a section or two.

Let the show begin!
		% Introduction

\chapter{Built-in Types, Exceptions and Functions}
\nodename{Built-in Objects}
\label{builtin}

Names for built-in exceptions and functions are found in a separate
symbol table.  This table is searched last when the interpreter looks
up the meaning of a name, so local and global
user-defined names can override built-in names.  Built-in types are
described together here for easy reference.\footnote{
	Most descriptions sorely lack explanations of the exceptions
	that may be raised --- this will be fixed in a future version of
	this manual.}
\indexii{built-in}{types}
\indexii{built-in}{exceptions}
\indexii{built-in}{functions}
\index{symbol table}

The tables in this chapter document the priorities of operators by
listing them in order of ascending priority (within a table) and
grouping operators that have the same priority in the same box.
Binary operators of the same priority group from left to right.
(Unary operators group from right to left, but there you have no real
choice.)  See Chapter 5 of the \emph{Python Reference Manual} for the
complete picture on operator priorities.
			% Built-in Types, Exceptions and Functions
\section{Built-in Types \label{types}}

The following sections describe the standard types that are built into
the interpreter.  These are the numeric types, sequence types, and
several others, including types themselves.  There is no explicit
Boolean type; use integers instead.
\indexii{built-in}{types}
\indexii{Boolean}{type}

Some operations are supported by several object types; in particular,
all objects can be compared, tested for truth value, and converted to
a string (with the \code{`\textrm{\ldots}`} notation).  The latter
conversion is implicitly used when an object is written by the
\keyword{print}\stindex{print} statement.


\subsection{Truth Value Testing \label{truth}}

Any object can be tested for truth value, for use in an \keyword{if} or
\keyword{while} condition or as operand of the Boolean operations below.
The following values are considered false:
\stindex{if}
\stindex{while}
\indexii{truth}{value}
\indexii{Boolean}{operations}
\index{false}

\begin{itemize}

\item	\code{None}
	\withsubitem{(Built-in object)}{\ttindex{None}}

\item	zero of any numeric type, for example, \code{0}, \code{0L},
        \code{0.0}, \code{0j}.

\item	any empty sequence, for example, \code{''}, \code{()}, \code{[]}.

\item	any empty mapping, for example, \code{\{\}}.

\item	instances of user-defined classes, if the class defines a
	\method{__nonzero__()} or \method{__len__()} method, when that
	method returns zero.\footnote{Additional information on these
special methods may be found in the \citetitle[../ref/ref.html]{Python
Reference Manual}.}

\end{itemize}

All other values are considered true --- so objects of many types are
always true.
\index{true}

Operations and built-in functions that have a Boolean result always
return \code{0} for false and \code{1} for true, unless otherwise
stated.  (Important exception: the Boolean operations
\samp{or}\opindex{or} and \samp{and}\opindex{and} always return one of
their operands.)


\subsection{Boolean Operations \label{boolean}}

These are the Boolean operations, ordered by ascending priority:
\indexii{Boolean}{operations}

\begin{tableiii}{c|l|c}{code}{Operation}{Result}{Notes}
  \lineiii{\var{x} or \var{y}}
          {if \var{x} is false, then \var{y}, else \var{x}}{(1)}
  \lineiii{\var{x} and \var{y}}
          {if \var{x} is false, then \var{x}, else \var{y}}{(1)}
  \hline
  \lineiii{not \var{x}}
          {if \var{x} is false, then \code{1}, else \code{0}}{(2)}
\end{tableiii}
\opindex{and}
\opindex{or}
\opindex{not}

\noindent
Notes:

\begin{description}

\item[(1)]
These only evaluate their second argument if needed for their outcome.

\item[(2)]
\samp{not} has a lower priority than non-Boolean operators, so
\code{not \var{a} == \var{b}} is interpreted as \code{not (\var{a} ==
\var{b})}, and \code{\var{a} == not \var{b}} is a syntax error.

\end{description}


\subsection{Comparisons \label{comparisons}}

Comparison operations are supported by all objects.  They all have the
same priority (which is higher than that of the Boolean operations).
Comparisons can be chained arbitrarily; for example, \code{\var{x} <
\var{y} <= \var{z}} is equivalent to \code{\var{x} < \var{y} and
\var{y} <= \var{z}}, except that \var{y} is evaluated only once (but
in both cases \var{z} is not evaluated at all when \code{\var{x} <
\var{y}} is found to be false).
\indexii{chaining}{comparisons}

This table summarizes the comparison operations:

\begin{tableiii}{c|l|c}{code}{Operation}{Meaning}{Notes}
  \lineiii{<}{strictly less than}{}
  \lineiii{<=}{less than or equal}{}
  \lineiii{>}{strictly greater than}{}
  \lineiii{>=}{greater than or equal}{}
  \lineiii{==}{equal}{}
  \lineiii{!=}{not equal}{(1)}
  \lineiii{<>}{not equal}{(1)}
  \lineiii{is}{object identity}{}
  \lineiii{is not}{negated object identity}{}
\end{tableiii}
\indexii{operator}{comparison}
\opindex{==} % XXX *All* others have funny characters < ! >
\opindex{is}
\opindex{is not}

\noindent
Notes:

\begin{description}

\item[(1)]
\code{<>} and \code{!=} are alternate spellings for the same operator.
(I couldn't choose between \ABC{} and C! :-)
\index{ABC language@\ABC{} language}
\index{language!ABC@\ABC{}}
\indexii{C}{language}
\code{!=} is the preferred spelling; \code{<>} is obsolescent.

\end{description}

Objects of different types, except different numeric types, never
compare equal; such objects are ordered consistently but arbitrarily
(so that sorting a heterogeneous array yields a consistent result).
Furthermore, some types (for example, file objects) support only a
degenerate notion of comparison where any two objects of that type are
unequal.  Again, such objects are ordered arbitrarily but
consistently.
\indexii{object}{numeric}
\indexii{objects}{comparing}

Instances of a class normally compare as non-equal unless the class
\withsubitem{(instance method)}{\ttindex{__cmp__()}}
defines the \method{__cmp__()} method.  Refer to the
\citetitle[../ref/customization.html]{Python Reference Manual} for
information on the use of this method to effect object comparisons.

\strong{Implementation note:} Objects of different types except
numbers are ordered by their type names; objects of the same types
that don't support proper comparison are ordered by their address.

Two more operations with the same syntactic priority,
\samp{in}\opindex{in} and \samp{not in}\opindex{not in}, are supported
only by sequence types (below).


\subsection{Numeric Types \label{typesnumeric}}

There are four numeric types: \dfn{plain integers}, \dfn{long integers}, 
\dfn{floating point numbers}, and \dfn{complex numbers}.
Plain integers (also just called \dfn{integers})
are implemented using \ctype{long} in C, which gives them at least 32
bits of precision.  Long integers have unlimited precision.  Floating
point numbers are implemented using \ctype{double} in C.  All bets on
their precision are off unless you happen to know the machine you are
working with.
\obindex{numeric}
\obindex{integer}
\obindex{long integer}
\obindex{floating point}
\obindex{complex number}
\indexii{C}{language}

Complex numbers have a real and imaginary part, which are both
implemented using \ctype{double} in C.  To extract these parts from
a complex number \var{z}, use \code{\var{z}.real} and \code{\var{z}.imag}.  

Numbers are created by numeric literals or as the result of built-in
functions and operators.  Unadorned integer literals (including hex
and octal numbers) yield plain integers.  Integer literals with an
\character{L} or \character{l} suffix yield long integers
(\character{L} is preferred because \samp{1l} looks too much like
eleven!).  Numeric literals containing a decimal point or an exponent
sign yield floating point numbers.  Appending \character{j} or
\character{J} to a numeric literal yields a complex number.
\indexii{numeric}{literals}
\indexii{integer}{literals}
\indexiii{long}{integer}{literals}
\indexii{floating point}{literals}
\indexii{complex number}{literals}
\indexii{hexadecimal}{literals}
\indexii{octal}{literals}

Python fully supports mixed arithmetic: when a binary arithmetic
operator has operands of different numeric types, the operand with the
``smaller'' type is converted to that of the other, where plain
integer is smaller than long integer is smaller than floating point is
smaller than complex.
Comparisons between numbers of mixed type use the same rule.\footnote{
	As a consequence, the list \code{[1, 2]} is considered equal
        to \code{[1.0, 2.0]}, and similar for tuples.
} The functions \function{int()}, \function{long()}, \function{float()},
and \function{complex()} can be used
to coerce numbers to a specific type.
\index{arithmetic}
\bifuncindex{int}
\bifuncindex{long}
\bifuncindex{float}
\bifuncindex{complex}

All numeric types support the following operations, sorted by
ascending priority (operations in the same box have the same
priority; all numeric operations have a higher priority than
comparison operations):

\begin{tableiii}{c|l|c}{code}{Operation}{Result}{Notes}
  \lineiii{\var{x} + \var{y}}{sum of \var{x} and \var{y}}{}
  \lineiii{\var{x} - \var{y}}{difference of \var{x} and \var{y}}{}
  \hline
  \lineiii{\var{x} * \var{y}}{product of \var{x} and \var{y}}{}
  \lineiii{\var{x} / \var{y}}{quotient of \var{x} and \var{y}}{(1)}
  \lineiii{\var{x} \%{} \var{y}}{remainder of \code{\var{x} / \var{y}}}{}
  \hline
  \lineiii{-\var{x}}{\var{x} negated}{}
  \lineiii{+\var{x}}{\var{x} unchanged}{}
  \hline
  \lineiii{abs(\var{x})}{absolute value or magnitude of \var{x}}{}
  \lineiii{int(\var{x})}{\var{x} converted to integer}{(2)}
  \lineiii{long(\var{x})}{\var{x} converted to long integer}{(2)}
  \lineiii{float(\var{x})}{\var{x} converted to floating point}{}
  \lineiii{complex(\var{re},\var{im})}{a complex number with real part \var{re}, imaginary part \var{im}.  \var{im} defaults to zero.}{}
  \lineiii{\var{c}.conjugate()}{conjugate of the complex number \var{c}}{}
  \lineiii{divmod(\var{x}, \var{y})}{the pair \code{(\var{x} / \var{y}, \var{x} \%{} \var{y})}}{(3)}
  \lineiii{pow(\var{x}, \var{y})}{\var{x} to the power \var{y}}{}
  \lineiii{\var{x} ** \var{y}}{\var{x} to the power \var{y}}{}
\end{tableiii}
\indexiii{operations on}{numeric}{types}
\withsubitem{(complex number method)}{\ttindex{conjugate()}}

\noindent
Notes:
\begin{description}

\item[(1)]
For (plain or long) integer division, the result is an integer.
The result is always rounded towards minus infinity: 1/2 is 0, 
(-1)/2 is -1, 1/(-2) is -1, and (-1)/(-2) is 0.  Note that the result
is a long integer if either operand is a long integer, regardless of
the numeric value.
\indexii{integer}{division}
\indexiii{long}{integer}{division}

\item[(2)]
Conversion from floating point to (long or plain) integer may round or
truncate as in C; see functions \function{floor()} and
\function{ceil()} in the \refmodule{math}\refbimodindex{math} module
for well-defined conversions.
\withsubitem{(in module math)}{\ttindex{floor()}\ttindex{ceil()}}
\indexii{numeric}{conversions}
\indexii{C}{language}

\item[(3)]
See section \ref{built-in-funcs}, ``Built-in Functions,'' for a full
description.

\end{description}
% XXXJH exceptions: overflow (when? what operations?) zerodivision

\subsubsection{Bit-string Operations on Integer Types \label{bitstring-ops}}
\nodename{Bit-string Operations}

Plain and long integer types support additional operations that make
sense only for bit-strings.  Negative numbers are treated as their 2's
complement value (for long integers, this assumes a sufficiently large
number of bits that no overflow occurs during the operation).

The priorities of the binary bit-wise operations are all lower than
the numeric operations and higher than the comparisons; the unary
operation \samp{\~} has the same priority as the other unary numeric
operations (\samp{+} and \samp{-}).

This table lists the bit-string operations sorted in ascending
priority (operations in the same box have the same priority):

\begin{tableiii}{c|l|c}{code}{Operation}{Result}{Notes}
  \lineiii{\var{x} | \var{y}}{bitwise \dfn{or} of \var{x} and \var{y}}{}
  \lineiii{\var{x} \^{} \var{y}}{bitwise \dfn{exclusive or} of \var{x} and \var{y}}{}
  \lineiii{\var{x} \&{} \var{y}}{bitwise \dfn{and} of \var{x} and \var{y}}{}
  \lineiii{\var{x} << \var{n}}{\var{x} shifted left by \var{n} bits}{(1), (2)}
  \lineiii{\var{x} >> \var{n}}{\var{x} shifted right by \var{n} bits}{(1), (3)}
  \hline
  \lineiii{\~\var{x}}{the bits of \var{x} inverted}{}
\end{tableiii}
\indexiii{operations on}{integer}{types}
\indexii{bit-string}{operations}
\indexii{shifting}{operations}
\indexii{masking}{operations}

\noindent
Notes:
\begin{description}
\item[(1)] Negative shift counts are illegal and cause a
\exception{ValueError} to be raised.
\item[(2)] A left shift by \var{n} bits is equivalent to
multiplication by \code{pow(2, \var{n})} without overflow check.
\item[(3)] A right shift by \var{n} bits is equivalent to
division by \code{pow(2, \var{n})} without overflow check.
\end{description}


\subsection{Sequence Types \label{typesseq}}

There are six sequence types: strings, Unicode strings, lists,
tuples, buffers, and xrange objects.

Strings literals are written in single or double quotes:
\code{'xyzzy'}, \code{"frobozz"}.  See chapter 2 of the
\citetitle[../ref/strings.html]{Python Reference Manual} for more about
string literals.  Unicode strings are much like strings, but are
specified in the syntax using a preceeding \character{u} character:
\code{u'abc'}, \code{u"def"}.  Lists are constructed with square brackets,
separating items with commas: \code{[a, b, c]}.  Tuples are
constructed by the comma operator (not within square brackets), with
or without enclosing parentheses, but an empty tuple must have the
enclosing parentheses, e.g., \code{a, b, c} or \code{()}.  A single
item tuple must have a trailing comma, e.g., \code{(d,)}.  Buffers are
not directly supported by Python syntax, but can be created by calling the
builtin function \function{buffer()}.\bifuncindex{buffer}  XRanges
objects are similar to buffers in that there is no specific syntax to
create them, but they are created using the \function{xrange()}
function.\bifuncindex{xrange}
\obindex{sequence}
\obindex{string}
\obindex{Unicode}
\obindex{buffer}
\obindex{tuple}
\obindex{list}
\obindex{xrange}

Sequence types support the following operations.  The \samp{in} and
\samp{not in} operations have the same priorities as the comparison
operations.  The \samp{+} and \samp{*} operations have the same
priority as the corresponding numeric operations.\footnote{They must
have since the parser can't tell the type of the operands.}

This table lists the sequence operations sorted in ascending priority
(operations in the same box have the same priority).  In the table,
\var{s} and \var{t} are sequences of the same type; \var{n}, \var{i}
and \var{j} are integers:

\begin{tableiii}{c|l|c}{code}{Operation}{Result}{Notes}
  \lineiii{\var{x} in \var{s}}{\code{1} if an item of \var{s} is equal to \var{x}, else \code{0}}{}
  \lineiii{\var{x} not in \var{s}}{\code{0} if an item of \var{s} is
equal to \var{x}, else \code{1}}{}
  \hline
  \lineiii{\var{s} + \var{t}}{the concatenation of \var{s} and \var{t}}{}
  \lineiii{\var{s} * \var{n}\textrm{,} \var{n} * \var{s}}{\var{n} copies of \var{s} concatenated}{(1)}
  \hline
  \lineiii{\var{s}[\var{i}]}{\var{i}'th item of \var{s}, origin 0}{(2)}
  \lineiii{\var{s}[\var{i}:\var{j}]}{slice of \var{s} from \var{i} to \var{j}}{(2), (3)}
  \hline
  \lineiii{len(\var{s})}{length of \var{s}}{}
  \lineiii{min(\var{s})}{smallest item of \var{s}}{}
  \lineiii{max(\var{s})}{largest item of \var{s}}{}
\end{tableiii}
\indexiii{operations on}{sequence}{types}
\bifuncindex{len}
\bifuncindex{min}
\bifuncindex{max}
\indexii{concatenation}{operation}
\indexii{repetition}{operation}
\indexii{subscript}{operation}
\indexii{slice}{operation}
\opindex{in}
\opindex{not in}

\noindent
Notes:

\begin{description}
\item[(1)] Values of \var{n} less than \code{0} are treated as
  \code{0} (which yields an empty sequence of the same type as
  \var{s}).

\item[(2)] If \var{i} or \var{j} is negative, the index is relative to
  the end of the string, i.e., \code{len(\var{s}) + \var{i}} or
  \code{len(\var{s}) + \var{j}} is substituted.  But note that \code{-0} is
  still \code{0}.
  
\item[(3)] The slice of \var{s} from \var{i} to \var{j} is defined as
  the sequence of items with index \var{k} such that \code{\var{i} <=
  \var{k} < \var{j}}.  If \var{i} or \var{j} is greater than
  \code{len(\var{s})}, use \code{len(\var{s})}.  If \var{i} is omitted,
  use \code{0}.  If \var{j} is omitted, use \code{len(\var{s})}.  If
  \var{i} is greater than or equal to \var{j}, the slice is empty.
\end{description}


\subsubsection{String Methods \label{string-methods}}

These are the string methods which both 8-bit strings and Unicode
objects support:

\begin{methoddesc}[string]{capitalize}{}
Return a copy of the string with only its first character capitalized.
\end{methoddesc}

\begin{methoddesc}[string]{center}{width}
Return centered in a string of length \var{width}. Padding is done
using spaces.
\end{methoddesc}

\begin{methoddesc}[string]{count}{sub\optional{, start\optional{, end}}}
Return the number of occurrences of substring \var{sub} in string
S\code{[\var{start}:\var{end}]}.  Optional arguments \var{start} and
\var{end} are interpreted as in slice notation.
\end{methoddesc}

\begin{methoddesc}[string]{encode}{\optional{encoding\optional{,errors}}}
Return an encoded version of the string.  Default encoding is the current
default string encoding.  \var{errors} may be given to set a different
error handling scheme.  The default for \var{errors} is
\code{'strict'}, meaning that encoding errors raise a
\exception{ValueError}.  Other possible values are \code{'ignore'} and
\code{'replace'}.
\versionadded{2.0}
\end{methoddesc}

\begin{methoddesc}[string]{endswith}{suffix\optional{, start\optional{, end}}}
Return true if the string ends with the specified \var{suffix},
otherwise return false.  With optional \var{start}, test beginning at
that position.  With optional \var{end}, stop comparing at that position.
\end{methoddesc}

\begin{methoddesc}[string]{expandtabs}{\optional{tabsize}}
Return a copy of the string where all tab characters are expanded
using spaces.  If \var{tabsize} is not given, a tab size of \code{8}
characters is assumed.
\end{methoddesc}

\begin{methoddesc}[string]{find}{sub\optional{, start\optional{, end}}}
Return the lowest index in the string where substring \var{sub} is
found, such that \var{sub} is contained in the range [\var{start},
\var{end}).  Optional arguments \var{start} and \var{end} are
interpreted as in slice notation.  Return \code{-1} if \var{sub} is
not found.
\end{methoddesc}

\begin{methoddesc}[string]{index}{sub\optional{, start\optional{, end}}}
Like \method{find()}, but raise \exception{ValueError} when the
substring is not found.
\end{methoddesc}

\begin{methoddesc}[string]{isalnum}{}
Return true if all characters in the string are alphanumeric and there
is at least one character, false otherwise.
\end{methoddesc}

\begin{methoddesc}[string]{isalpha}{}
Return true if all characters in the string are alphabetic and there
is at least one character, false otherwise.
\end{methoddesc}

\begin{methoddesc}[string]{isdigit}{}
Return true if there are only digit characters, false otherwise.
\end{methoddesc}

\begin{methoddesc}[string]{islower}{}
Return true if all cased characters in the string are lowercase and
there is at least one cased character, false otherwise.
\end{methoddesc}

\begin{methoddesc}[string]{isspace}{}
Return true if there are only whitespace characters in the string and
the string is not empty, false otherwise.
\end{methoddesc}

\begin{methoddesc}[string]{istitle}{}
Return true if the string is a titlecased string, i.e.\ uppercase
characters may only follow uncased characters and lowercase characters
only cased ones.  Return false otherwise.
\end{methoddesc}

\begin{methoddesc}[string]{isupper}{}
Return true if all cased characters in the string are uppercase and
there is at least one cased character, false otherwise.
\end{methoddesc}

\begin{methoddesc}[string]{join}{seq}
Return a string which is the concatenation of the strings in the
sequence \var{seq}.  The separator between elements is the string
providing this method.
\end{methoddesc}

\begin{methoddesc}[string]{ljust}{width}
Return the string left justified in a string of length \var{width}.
Padding is done using spaces.  The original string is returned if
\var{width} is less than \code{len(\var{s})}.
\end{methoddesc}

\begin{methoddesc}[string]{lower}{}
Return a copy of the string converted to lowercase.
\end{methoddesc}

\begin{methoddesc}[string]{lstrip}{}
Return a copy of the string with leading whitespace removed.
\end{methoddesc}

\begin{methoddesc}[string]{replace}{old, new\optional{, maxsplit}}
Return a copy of the string with all occurrences of substring
\var{old} replaced by \var{new}.  If the optional argument
\var{maxsplit} is given, only the first \var{maxsplit} occurrences are
replaced.
\end{methoddesc}

\begin{methoddesc}[string]{rfind}{sub \optional{,start \optional{,end}}}
Return the highest index in the string where substring \var{sub} is
found, such that \var{sub} is contained within s[start,end].  Optional
arguments \var{start} and \var{end} are interpreted as in slice
notation.  Return \code{-1} on failure.
\end{methoddesc}

\begin{methoddesc}[string]{rindex}{sub\optional{, start\optional{, end}}}
Like \method{rfind()} but raises \exception{ValueError} when the
substring \var{sub} is not found.
\end{methoddesc}

\begin{methoddesc}[string]{rjust}{width}
Return the string right justified in a string of length \var{width}.
Padding is done using spaces.  The original string is returned if
\var{width} is less than \code{len(\var{s})}.
\end{methoddesc}

\begin{methoddesc}[string]{rstrip}{}
Return a copy of the string with trailing whitespace removed.
\end{methoddesc}

\begin{methoddesc}[string]{split}{\optional{sep \optional{,maxsplit}}}
Return a list of the words in the string, using \var{sep} as the
delimiter string.  If \var{maxsplit} is given, at most \var{maxsplit}
splits are done.  If \var{sep} is not specified or \code{None}, any
whitespace string is a separator.
\end{methoddesc}

\begin{methoddesc}[string]{splitlines}{\optional{keepends}}
Return a list of the lines in the string, breaking at line
boundaries.  Line breaks are not included in the resulting list unless
\var{keepends} is given and true.
\end{methoddesc}

\begin{methoddesc}[string]{startswith}{prefix\optional{, start\optional{, end}}}
Return true if string starts with the \var{prefix}, otherwise
return false.  With optional \var{start}, test string beginning at
that position.  With optional \var{end}, stop comparing string at that
position.
\end{methoddesc}

\begin{methoddesc}[string]{strip}{}
Return a copy of the string with leading and trailing whitespace
removed.
\end{methoddesc}

\begin{methoddesc}[string]{swapcase}{}
Return a copy of the string with uppercase characters converted to
lowercase and vice versa.
\end{methoddesc}

\begin{methoddesc}[string]{title}{}
Return a titlecased version of, i.e.\ words start with uppercase
characters, all remaining cased characters are lowercase.
\end{methoddesc}

\begin{methoddesc}[string]{translate}{table\optional{, deletechars}}
Return a copy of the string where all characters occurring in the
optional argument \var{deletechars} are removed, and the remaining
characters have been mapped through the given translation table, which
must be a string of length 256.
\end{methoddesc}

\begin{methoddesc}[string]{upper}{}
Return a copy of the string converted to uppercase.
\end{methoddesc}


\subsubsection{String Formatting Operations \label{typesseq-strings}}

\index{formatting, string}
\index{string!formatting}
\index{printf-style formatting}
\index{sprintf-style formatting}

String and Unicode objects have one unique built-in operation: the
\code{\%} operator (modulo).  Given \code{\var{format} \%
\var{values}} (where \var{format} is a string or Unicode object),
\code{\%} conversion specifications in \var{format} are replaced with
zero or more elements of \var{values}.  The effect is similar to the
using \cfunction{sprintf()} in the C language.  If \var{format} is a
Unicode object, or if any of the objects being converted using the
\code{\%s} conversion are Unicode objects, the result will be a
Unicode object as well.

If \var{format} requires a single argument, \var{values} may be a
single non-tuple object. \footnote{A tuple object in this case should
  be a singleton.}  Otherwise, \var{values} must be a tuple with
exactly the number of items specified by the format string, or a
single mapping object (for example, a dictionary).

A conversion specifier contains two or more characters and has the
following components, which must occur in this order:

\begin{enumerate}
  \item  The \character{\%} character, which marks the start of the
         specifier.
  \item  Mapping key value (optional), consisting of an identifier in
         parentheses (for example, \code{(somename)}).
  \item  Conversion flags (optional), which affect the result of some
         conversion types.
  \item  Minimum field width (optional).  If specified as an
         \character{*} (asterisk), the actual width is read from the
         next element of the tuple in \var{values}, and the object to
         convert comes after the minimum field width and optional
         precision.
  \item  Precision (optional), given as a \character{.} (dot) followed
         by the precision.  If specified as \character{*} (an
         asterisk), the actual width is read from the next element of
         the tuple in \var{values}, and the value to convert comes after
         the precision.
  \item  Length modifier (optional).
  \item  Conversion type.
\end{enumerate}

If the right argument is a dictionary (or any kind of mapping), then
the formats in the string \emph{must} have a parenthesized key into
that dictionary inserted immediately after the \character{\%}
character, and each format formats the corresponding entry from the
mapping.  For example:

\begin{verbatim}
>>> count = 2
>>> language = 'Python'
>>> print '%(language)s has %(count)03d quote types.' % vars()
Python has 002 quote types.
\end{verbatim}

In this case no \code{*} specifiers may occur in a format (since they
require a sequential parameter list).

The conversion flag characters are:

\begin{tableii}{c|l}{character}{Flag}{Meaning}
  \lineii{\#}{The value conversion will use the ``alternate form''
              (where defined below).}
  \lineii{0}{The conversion will be zero padded.}
  \lineii{-}{The converted value is left adjusted (overrides
             \character{-}).}
  \lineii{{~}}{(a space) A blank should be left before a positive number
             (or empty string) produced by a signed conversion.}
  \lineii{+}{A sign character (\character{+} or \character{-}) will
             precede the conversion (overrides a "space" flag).}
\end{tableii}

The length modifier may be \code{h}, \code{l}, and \code{L} may be
present, but are ignored as they are not necessary for Python.

The conversion types are:

\begin{tableii}{c|l}{character}{Conversion}{Meaning}
  \lineii{d}{Signed integer decimal.}
  \lineii{i}{Signed integer decimal.}
  \lineii{o}{Unsigned octal.}
  \lineii{u}{Unsigned decimal.}
  \lineii{x}{Unsigned hexidecimal (lowercase).}
  \lineii{X}{Unsigned hexidecimal (uppercase).}
  \lineii{e}{Floating point exponential format (lowercase).}
  \lineii{E}{Floating point exponential format (uppercase).}
  \lineii{f}{Floating point decimal format.}
  \lineii{F}{Floating point decimal format.}
  \lineii{g}{Same as \character{e} if exponent is greater than -4 or
             less than precision, \character{f} otherwise.}
  \lineii{G}{Same as \character{E} if exponent is greater than -4 or
             less than precision, \character{F} otherwise.}
  \lineii{c}{Single character (accepts integer or single character
             string).}
  \lineii{r}{String (converts any python object using
             \function{repr()}).}
  \lineii{s}{String (converts any python object using
             \function{str()}).}
  \lineii{\%}{No argument is converted, results in a \character{\%}
              character in the result.  (The complete specification is
              \code{\%\%}.)}
\end{tableii}

% XXX Examples?


Since Python strings have an explicit length, \code{\%s} conversions
do not assume that \code{'\e0'} is the end of the string.

For safety reasons, floating point precisions are clipped to 50;
\code{\%f} conversions for numbers whose absolute value is over 1e25
are replaced by \code{\%g} conversions.\footnote{
  These numbers are fairly arbitrary.  They are intended to
  avoid printing endless strings of meaningless digits without hampering
  correct use and without having to know the exact precision of floating
  point values on a particular machine.
}  All other errors raise exceptions.

Additional string operations are defined in standard module
\refmodule{string} and in built-in module \refmodule{re}.
\refstmodindex{string}
\refstmodindex{re}


\subsubsection{XRange Type \label{typesseq-xrange}}

The xrange\obindex{xrange} type is an immutable sequence which is
commonly used for looping.  The advantage of the xrange type is that an
xrange object will always take the same amount of memory, no matter the
size of the range it represents.  There are no consistent performance
advantages.

XRange objects behave like tuples, and offer a single method:

\begin{methoddesc}[xrange]{tolist}{}
  Return a list object which represents the same values as the xrange
  object.
\end{methoddesc}


\subsubsection{Mutable Sequence Types \label{typesseq-mutable}}

List objects support additional operations that allow in-place
modification of the object.
These operations would be supported by other mutable sequence types
(when added to the language) as well.
Strings and tuples are immutable sequence types and such objects cannot
be modified once created.
The following operations are defined on mutable sequence types (where
\var{x} is an arbitrary object):
\indexiii{mutable}{sequence}{types}
\obindex{list}

\begin{tableiii}{c|l|c}{code}{Operation}{Result}{Notes}
  \lineiii{\var{s}[\var{i}] = \var{x}}
	{item \var{i} of \var{s} is replaced by \var{x}}{}
  \lineiii{\var{s}[\var{i}:\var{j}] = \var{t}}
  	{slice of \var{s} from \var{i} to \var{j} is replaced by \var{t}}{}
  \lineiii{del \var{s}[\var{i}:\var{j}]}
	{same as \code{\var{s}[\var{i}:\var{j}] = []}}{}
  \lineiii{\var{s}.append(\var{x})}
	{same as \code{\var{s}[len(\var{s}):len(\var{s})] = [\var{x}]}}{(1)}
  \lineiii{\var{s}.extend(\var{x})}
        {same as \code{\var{s}[len(\var{s}):len(\var{s})] = \var{x}}}{(2)}
  \lineiii{\var{s}.count(\var{x})}
    {return number of \var{i}'s for which \code{\var{s}[\var{i}] == \var{x}}}{}
  \lineiii{\var{s}.index(\var{x})}
    {return smallest \var{i} such that \code{\var{s}[\var{i}] == \var{x}}}{(3)}
  \lineiii{\var{s}.insert(\var{i}, \var{x})}
	{same as \code{\var{s}[\var{i}:\var{i}] = [\var{x}]}
	  if \code{\var{i} >= 0}}{}
  \lineiii{\var{s}.pop(\optional{\var{i}})}
    {same as \code{\var{x} = \var{s}[\var{i}]; del \var{s}[\var{i}]; return \var{x}}}{(4)}
  \lineiii{\var{s}.remove(\var{x})}
	{same as \code{del \var{s}[\var{s}.index(\var{x})]}}{(3)}
  \lineiii{\var{s}.reverse()}
	{reverses the items of \var{s} in place}{(5)}
  \lineiii{\var{s}.sort(\optional{\var{cmpfunc}})}
	{sort the items of \var{s} in place}{(5), (6)}
\end{tableiii}
\indexiv{operations on}{mutable}{sequence}{types}
\indexiii{operations on}{sequence}{types}
\indexiii{operations on}{list}{type}
\indexii{subscript}{assignment}
\indexii{slice}{assignment}
\stindex{del}
\withsubitem{(list method)}{
  \ttindex{append()}\ttindex{extend()}\ttindex{count()}\ttindex{index()}
  \ttindex{insert()}\ttindex{pop()}\ttindex{remove()}\ttindex{reverse()}
  \ttindex{sort()}}
\noindent
Notes:
\begin{description}
\item[(1)] The C implementation of Python has historically accepted
  multiple parameters and implicitly joined them into a tuple; this
  no longer works in Python 2.0.  Use of this misfeature has been
  deprecated since Python 1.4.

\item[(2)] Raises an exception when \var{x} is not a list object.  The 
  \method{extend()} method is experimental and not supported by
  mutable sequence types other than lists.

\item[(3)] Raises \exception{ValueError} when \var{x} is not found in
  \var{s}.

\item[(4)] The \method{pop()} method is only supported by the list and
  array types.  The optional argument \var{i} defaults to \code{-1},
  so that by default the last item is removed and returned.

\item[(5)] The \method{sort()} and \method{reverse()} methods modify the
  list in place for economy of space when sorting or reversing a large
  list.  They don't return the sorted or reversed list to remind you
  of this side effect.

\item[(6)] The \method{sort()} method takes an optional argument
  specifying a comparison function of two arguments (list items) which
  should return \code{-1}, \code{0} or \code{1} depending on whether
  the first argument is considered smaller than, equal to, or larger
  than the second argument.  Note that this slows the sorting process
  down considerably; e.g. to sort a list in reverse order it is much
  faster to use calls to the methods \method{sort()} and
  \method{reverse()} than to use the built-in function
  \function{sort()} with a comparison function that reverses the
  ordering of the elements.
\end{description}


\subsection{Mapping Types \label{typesmapping}}
\obindex{mapping}
\obindex{dictionary}

A \dfn{mapping} object maps values of one type (the key type) to
arbitrary objects.  Mappings are mutable objects.  There is currently
only one standard mapping type, the \dfn{dictionary}.  A dictionary's keys are
almost arbitrary values.  The only types of values not acceptable as
keys are values containing lists or dictionaries or other mutable
types that are compared by value rather than by object identity.
Numeric types used for keys obey the normal rules for numeric
comparison: if two numbers compare equal (e.g. \code{1} and
\code{1.0}) then they can be used interchangeably to index the same
dictionary entry.

Dictionaries are created by placing a comma-separated list of
\code{\var{key}: \var{value}} pairs within braces, for example:
\code{\{'jack': 4098, 'sjoerd': 4127\}} or
\code{\{4098: 'jack', 4127: 'sjoerd'\}}.

The following operations are defined on mappings (where \var{a} and
\var{b} are mappings, \var{k} is a key, and \var{v} and \var{x} are
arbitrary objects):
\indexiii{operations on}{mapping}{types}
\indexiii{operations on}{dictionary}{type}
\stindex{del}
\bifuncindex{len}
\withsubitem{(dictionary method)}{
  \ttindex{clear()}
  \ttindex{copy()}
  \ttindex{has_key()}
  \ttindex{items()}
  \ttindex{keys()}
  \ttindex{update()}
  \ttindex{values()}
  \ttindex{get()}}

\begin{tableiii}{c|l|c}{code}{Operation}{Result}{Notes}
  \lineiii{len(\var{a})}{the number of items in \var{a}}{}
  \lineiii{\var{a}[\var{k}]}{the item of \var{a} with key \var{k}}{(1)}
  \lineiii{\var{a}[\var{k}] = \var{v}}
          {set \code{\var{a}[\var{k}]} to \var{v}}
          {}
  \lineiii{del \var{a}[\var{k}]}
          {remove \code{\var{a}[\var{k}]} from \var{a}}
          {(1)}
  \lineiii{\var{a}.clear()}{remove all items from \code{a}}{}
  \lineiii{\var{a}.copy()}{a (shallow) copy of \code{a}}{}
  \lineiii{\var{k} \code{in} \var{a}}
          {\code{1} if \var{a} has a key \var{k}, else \code{0}}
          {}
  \lineiii{\var{k} not in \var{a}}
          {\code{0} if \var{a} has a key \var{k}, else \code{1}}
          {}
  \lineiii{\var{a}.has_key(\var{k})}
          {Equivalent to \var{k} \code{in} \var{a}}
          {}
  \lineiii{\var{a}.items()}
          {a copy of \var{a}'s list of (\var{key}, \var{value}) pairs}
          {(2)}
  \lineiii{\var{a}.keys()}{a copy of \var{a}'s list of keys}{(2)}
  \lineiii{\var{a}.update(\var{b})}
          {\code{for k in \var{b}.keys(): \var{a}[k] = \var{b}[k]}}
          {(3)}
  \lineiii{\var{a}.values()}{a copy of \var{a}'s list of values}{(2)}
  \lineiii{\var{a}.get(\var{k}\optional{, \var{x}})}
          {\code{\var{a}[\var{k}]} if \code{\var{k} in \var{a}}},
           else \var{x}}
          {(4)}
  \lineiii{\var{a}.setdefault(\var{k}\optional{, \var{x}})}
          {\code{\var{a}[\var{k}]} if \code{\var{k} in \var{a}}},
           else \var{x} (also setting it)}
          {(5)}
  \lineiii{\var{a}.popitem()}
          {remove and return an arbitrary (\var{key}, \var{value}) pair}
          {(6)}
\end{tableiii}

\noindent
Notes:
\begin{description}
\item[(1)] Raises a \exception{KeyError} exception if \var{k} is not
in the map.

\item[(2)] Keys and values are listed in random order.  If
\method{keys()} and \method{values()} are called with no intervening
modifications to the dictionary, the two lists will directly
correspond.  This allows the creation of \code{(\var{value},
\var{key})} pairs using \function{map()}: \samp{pairs = map(None,
\var{a}.values(), \var{a}.keys())}.

\item[(3)] \var{b} must be of the same type as \var{a}.

\item[(4)] Never raises an exception if \var{k} is not in the map,
instead it returns \var{x}.  \var{x} is optional; when \var{x} is not
provided and \var{k} is not in the map, \code{None} is returned.

\item[(5)] \function{setdefault()} is like \function{get()}, except
that if \var{k} is missing, \var{x} is both returned and inserted into
the dictionary as the value of \var{k}.

\item[(6)] \function{popitem()} is useful to destructively iterate
over a dictionary, as often used in set algorithms.
\end{description}


\subsection{Other Built-in Types \label{typesother}}

The interpreter supports several other kinds of objects.
Most of these support only one or two operations.


\subsubsection{Modules \label{typesmodules}}

The only special operation on a module is attribute access:
\code{\var{m}.\var{name}}, where \var{m} is a module and \var{name}
accesses a name defined in \var{m}'s symbol table.  Module attributes
can be assigned to.  (Note that the \keyword{import} statement is not,
strictly speaking, an operation on a module object; \code{import
\var{foo}} does not require a module object named \var{foo} to exist,
rather it requires an (external) \emph{definition} for a module named
\var{foo} somewhere.)

A special member of every module is \member{__dict__}.
This is the dictionary containing the module's symbol table.
Modifying this dictionary will actually change the module's symbol
table, but direct assignment to the \member{__dict__} attribute is not
possible (i.e., you can write \code{\var{m}.__dict__['a'] = 1}, which
defines \code{\var{m}.a} to be \code{1}, but you can't write
\code{\var{m}.__dict__ = \{\}}.

Modules built into the interpreter are written like this:
\code{<module 'sys' (built-in)>}.  If loaded from a file, they are
written as \code{<module 'os' from
'/usr/local/lib/python\shortversion/os.pyc'>}.


\subsubsection{Classes and Class Instances \label{typesobjects}}
\nodename{Classes and Instances}

See chapters 3 and 7 of the \citetitle[../ref/ref.html]{Python
Reference Manual} for these.


\subsubsection{Functions \label{typesfunctions}}

Function objects are created by function definitions.  The only
operation on a function object is to call it:
\code{\var{func}(\var{argument-list})}.

There are really two flavors of function objects: built-in functions
and user-defined functions.  Both support the same operation (to call
the function), but the implementation is different, hence the
different object types.

The implementation adds two special read-only attributes:
\code{\var{f}.func_code} is a function's \dfn{code
object}\obindex{code} (see below) and \code{\var{f}.func_globals} is
the dictionary used as the function's global namespace (this is the
same as \code{\var{m}.__dict__} where \var{m} is the module in which
the function \var{f} was defined).

Function objects also support getting and setting arbitrary
attributes, which can be used to, e.g. attach metadata to functions.
Regular attribute dot-notation is used to get and set such
attributes. \emph{Note that the current implementation only supports
function attributes on functions written in Python.  Function
attributes on built-ins may be supported in the future.}

Functions have another special attribute \code{\var{f}.__dict__}
(a.k.a. \code{\var{f}.func_dict}) which contains the namespace used to
support function attributes.  \code{__dict__} can be accessed
directly, set to a dictionary object, or \code{None}.  It can also be
deleted (but the following two lines are equivalent):

\begin{verbatim}
del func.__dict__
func.__dict__ = None
\end{verbatim}

\subsubsection{Methods \label{typesmethods}}
\obindex{method}

Methods are functions that are called using the attribute notation.
There are two flavors: built-in methods (such as \method{append()} on
lists) and class instance methods.  Built-in methods are described
with the types that support them.

The implementation adds two special read-only attributes to class
instance methods: \code{\var{m}.im_self} is the object on which the
method operates, and \code{\var{m}.im_func} is the function
implementing the method.  Calling \code{\var{m}(\var{arg-1},
\var{arg-2}, \textrm{\ldots}, \var{arg-n})} is completely equivalent to
calling \code{\var{m}.im_func(\var{m}.im_self, \var{arg-1},
\var{arg-2}, \textrm{\ldots}, \var{arg-n})}.

Class instance methods are either \emph{bound} or \emph{unbound},
referring to whether the method was accessed through an instance or a
class, respectively.  When a method is unbound, its \code{im_self}
attribute will be \code{None} and if called, an explicit \code{self}
object must be passed as the first argument.  In this case,
\code{self} must be an instance of the unbound method's class (or a
subclass of that class), otherwise a \code{TypeError} is raised.

Like function objects, methods objects support getting
arbitrary attributes.  However, since method attributes are actually
stored on the underlying function object (i.e. \code{meth.im_func}),
setting method attributes on either bound or unbound methods is
disallowed.  Attempting to set a method attribute results in a
\code{TypeError} being raised.  In order to set a method attribute,
you need to explicitly set it on the underlying function object:

\begin{verbatim}
class C:
    def method(self):
        pass

c = C()
c.method.im_func.whoami = 'my name is c'
\end{verbatim}

See the \citetitle[../ref/ref.html]{Python Reference Manual} for more
information.


\subsubsection{Code Objects \label{bltin-code-objects}}
\obindex{code}

Code objects are used by the implementation to represent
``pseudo-compiled'' executable Python code such as a function body.
They differ from function objects because they don't contain a
reference to their global execution environment.  Code objects are
returned by the built-in \function{compile()} function and can be
extracted from function objects through their \member{func_code}
attribute.
\bifuncindex{compile}
\withsubitem{(function object attribute)}{\ttindex{func_code}}

A code object can be executed or evaluated by passing it (instead of a
source string) to the \keyword{exec} statement or the built-in
\function{eval()} function.
\stindex{exec}
\bifuncindex{eval}

See the \citetitle[../ref/ref.html]{Python Reference Manual} for more
information.


\subsubsection{Type Objects \label{bltin-type-objects}}

Type objects represent the various object types.  An object's type is
accessed by the built-in function \function{type()}.  There are no special
operations on types.  The standard module \module{types} defines names
for all standard built-in types.
\bifuncindex{type}
\refstmodindex{types}

Types are written like this: \code{<type 'int'>}.


\subsubsection{The Null Object \label{bltin-null-object}}

This object is returned by functions that don't explicitly return a
value.  It supports no special operations.  There is exactly one null
object, named \code{None} (a built-in name).

It is written as \code{None}.


\subsubsection{The Ellipsis Object \label{bltin-ellipsis-object}}

This object is used by extended slice notation (see the
\citetitle[../ref/ref.html]{Python Reference Manual}).  It supports no
special operations.  There is exactly one ellipsis object, named
\constant{Ellipsis} (a built-in name).

It is written as \code{Ellipsis}.


\subsubsection{File Objects\obindex{file}
               \label{bltin-file-objects}}

File objects are implemented using C's \code{stdio} package and can be
created with the built-in function
\function{open()}\bifuncindex{open} described in section 
\ref{built-in-funcs}, ``Built-in Functions.''  They are also returned
by some other built-in functions and methods, e.g.,
\function{os.popen()} and \function{os.fdopen()} and the
\method{makefile()} method of socket objects.
\refstmodindex{os}
\refbimodindex{socket}

When a file operation fails for an I/O-related reason, the exception
\exception{IOError} is raised.  This includes situations where the
operation is not defined for some reason, like \method{seek()} on a tty
device or writing a file opened for reading.

Files have the following methods:


\begin{methoddesc}[file]{close}{}
  Close the file.  A closed file cannot be read or written anymore.
  Any operation which requires that the file be open will raise a
  \exception{ValueError} after the file has been closed.  Calling
  \method{close()} more than once is allowed.
\end{methoddesc}

\begin{methoddesc}[file]{flush}{}
  Flush the internal buffer, like \code{stdio}'s
  \cfunction{fflush()}.  This may be a no-op on some file-like
  objects.
\end{methoddesc}

\begin{methoddesc}[file]{isatty}{}
  Return true if the file is connected to a tty(-like) device, else
  false.  \strong{Note:} If a file-like object is not associated
  with a real file, this method should \emph{not} be implemented.
\end{methoddesc}

\begin{methoddesc}[file]{fileno}{}
  \index{file descriptor}
  \index{descriptor, file}
  Return the integer ``file descriptor'' that is used by the
  underlying implementation to request I/O operations from the
  operating system.  This can be useful for other, lower level
  interfaces that use file descriptors, e.g.\ module
  \refmodule{fcntl}\refbimodindex{fcntl} or \function{os.read()} and
  friends.  \strong{Note:} File-like objects which do not have a real
  file descriptor should \emph{not} provide this method!
\end{methoddesc}

\begin{methoddesc}[file]{read}{\optional{size}}
  Read at most \var{size} bytes from the file (less if the read hits
  \EOF{} before obtaining \var{size} bytes).  If the \var{size}
  argument is negative or omitted, read all data until \EOF{} is
  reached.  The bytes are returned as a string object.  An empty
  string is returned when \EOF{} is encountered immediately.  (For
  certain files, like ttys, it makes sense to continue reading after
  an \EOF{} is hit.)  Note that this method may call the underlying
  C function \cfunction{fread()} more than once in an effort to
  acquire as close to \var{size} bytes as possible.
\end{methoddesc}

\begin{methoddesc}[file]{readline}{\optional{size}}
  Read one entire line from the file.  A trailing newline character is
  kept in the string\footnote{
	The advantage of leaving the newline on is that an empty string 
	can be returned to mean \EOF{} without being ambiguous.  Another 
	advantage is that (in cases where it might matter, e.g. if you 
	want to make an exact copy of a file while scanning its lines) 
	you can tell whether the last line of a file ended in a newline
	or not (yes this happens!).
  } (but may be absent when a file ends with an
  incomplete line).  If the \var{size} argument is present and
  non-negative, it is a maximum byte count (including the trailing
  newline) and an incomplete line may be returned.
  An empty string is returned when \EOF{} is hit
  immediately.  Note: Unlike \code{stdio}'s \cfunction{fgets()}, the
  returned string contains null characters (\code{'\e 0'}) if they
  occurred in the input.
\end{methoddesc}

\begin{methoddesc}[file]{readlines}{\optional{sizehint}}
  Read until \EOF{} using \method{readline()} and return a list containing
  the lines thus read.  If the optional \var{sizehint} argument is
  present, instead of reading up to \EOF{}, whole lines totalling
  approximately \var{sizehint} bytes (possibly after rounding up to an
  internal buffer size) are read.  Objects implementing a file-like
  interface may choose to ignore \var{sizehint} if it cannot be
  implemented, or cannot be implemented efficiently.
\end{methoddesc}

\begin{methoddesc}[file]{xreadlines}{}
  Equivalent to \function{xreadlines.xreadlines(file)}.\refstmodindex{xreadlines}
\end{methoddesc}

\begin{methoddesc}[file]{seek}{offset\optional{, whence}}
  Set the file's current position, like \code{stdio}'s \cfunction{fseek()}.
  The \var{whence} argument is optional and defaults to \code{0}
  (absolute file positioning); other values are \code{1} (seek
  relative to the current position) and \code{2} (seek relative to the
  file's end).  There is no return value.  Note that if the file is
  opened for appending (mode \code{'a'} or \code{'a+'}), any
  \method{seek()} operations will be undone at the next write.  If the
  file is only opened for writing in append mode (mode \code{'a'}),
  this method is essentially a no-op, but it remains useful for files
  opened in append mode with reading enabled (mode \code{'a+'}).
\end{methoddesc}

\begin{methoddesc}[file]{tell}{}
  Return the file's current position, like \code{stdio}'s
  \cfunction{ftell()}.
\end{methoddesc}

\begin{methoddesc}[file]{truncate}{\optional{size}}
  Truncate the file's size.  If the optional \var{size} argument
  present, the file is truncated to (at most) that size.  The size
  defaults to the current position.  Availability of this function
  depends on the operating system version (for example, not all
  \UNIX{} versions support this operation).
\end{methoddesc}

\begin{methoddesc}[file]{write}{str}
  Write a string to the file.  There is no return value.  Note: Due to
  buffering, the string may not actually show up in the file until
  the \method{flush()} or \method{close()} method is called.
\end{methoddesc}

\begin{methoddesc}[file]{writelines}{list}
  Write a list of strings to the file.  There is no return value.
  (The name is intended to match \method{readlines()};
  \method{writelines()} does not add line separators.)
\end{methoddesc}

\begin{methoddesc}[file]{xreadlines}{}
  Equivalent to
  \function{xreadlines.xreadlines(\var{file})}.\refstmodindex{xreadlines}
  (See the \refmodule{xreadlines} module for more information.)
\end{methoddesc}


File objects also offer a number of other interesting attributes.
These are not required for file-like objects, but should be
implemented if they make sense for the particular object.

\begin{memberdesc}[file]{closed}
Boolean indicating the current state of the file object.  This is a
read-only attribute; the \method{close()} method changes the value.
It may not be available on all file-like objects.
\end{memberdesc}

\begin{memberdesc}[file]{mode}
The I/O mode for the file.  If the file was created using the
\function{open()} built-in function, this will be the value of the
\var{mode} parameter.  This is a read-only attribute and may not be
present on all file-like objects.
\end{memberdesc}

\begin{memberdesc}[file]{name}
If the file object was created using \function{open()}, the name of
the file.  Otherwise, some string that indicates the source of the
file object, of the form \samp{<\mbox{\ldots}>}.  This is a read-only
attribute and may not be present on all file-like objects.
\end{memberdesc}

\begin{memberdesc}[file]{softspace}
Boolean that indicates whether a space character needs to be printed
before another value when using the \keyword{print} statement.
Classes that are trying to simulate a file object should also have a
writable \member{softspace} attribute, which should be initialized to
zero.  This will be automatic for most classes implemented in Python
(care may be needed for objects that override attribute access); types
implemented in C will have to provide a writable
\member{softspace} attribute.
\strong{Note:}  This attribute is not used to control the
\keyword{print} statement, but to allow the implementation of
\keyword{print} to keep track of its internal state.
\end{memberdesc}


\subsubsection{Internal Objects \label{typesinternal}}

See the \citetitle[../ref/ref.html]{Python Reference Manual} for this
information.  It describes stack frame objects, traceback objects, and
slice objects.


\subsection{Special Attributes \label{specialattrs}}

The implementation adds a few special read-only attributes to several
object types, where they are relevant:

\begin{memberdesc}[object]{__dict__}
A dictionary or other mapping object used to store an
object's (writable) attributes.
\end{memberdesc}

\begin{memberdesc}[object]{__methods__}
List of the methods of many built-in object types,
e.g., \code{[].__methods__} yields
\code{['append', 'count', 'index', 'insert', 'pop', 'remove',
'reverse', 'sort']}.  This usually does not need to be explicitly
provided by the object.
\end{memberdesc}

\begin{memberdesc}[object]{__members__}
Similar to \member{__methods__}, but lists data attributes.  This
usually does not need to be explicitly provided by the object.
\end{memberdesc}

\begin{memberdesc}[instance]{__class__}
The class to which a class instance belongs.
\end{memberdesc}

\begin{memberdesc}[class]{__bases__}
The tuple of base classes of a class object.
\end{memberdesc}

\section{Built-in Exceptions}

\declaremodule{standard}{exceptions}
\modulesynopsis{Standard exceptions classes.}


Exceptions can be class objects or string objects.  Though most
exceptions have been string objects in past versions of Python, in
Python 1.5 and newer versions, all standard exceptions have been
converted to class objects, and users are encouraged to do the same.
The exceptions are defined in the module \module{exceptions}.  This
module never needs to be imported explicitly: the exceptions are
provided in the built-in namespace.

Two distinct string objects with the same value are considered different
exceptions.  This is done to force programmers to use exception names
rather than their string value when specifying exception handlers.
The string value of all built-in exceptions is their name, but this is
not a requirement for user-defined exceptions or exceptions defined by
library modules.

For class exceptions, in a \keyword{try}\stindex{try} statement with
an \keyword{except}\stindex{except} clause that mentions a particular
class, that clause also handles any exception classes derived from
that class (but not exception classes from which \emph{it} is
derived).  Two exception classes that are not related via subclassing
are never equivalent, even if they have the same name.

The built-in exceptions listed below can be generated by the
interpreter or built-in functions.  Except where mentioned, they have
an ``associated value'' indicating the detailed cause of the error.
This may be a string or a tuple containing several items of
information (e.g., an error code and a string explaining the code).
The associated value is the second argument to the
\keyword{raise}\stindex{raise} statement.  For string exceptions, the
associated value itself will be stored in the variable named as the
second argument of the \keyword{except} clause (if any).  For class
exceptions, that variable receives the exception instance.  If the
exception class is derived from the standard root class
\exception{Exception}, the associated value is present as the
exception instance's \member{args} attribute, and possibly on other
attributes as well.

User code can raise built-in exceptions.  This can be used to test an
exception handler or to report an error condition ``just like'' the
situation in which the interpreter raises the same exception; but
beware that there is nothing to prevent user code from raising an
inappropriate error.

\setindexsubitem{(built-in exception base class)}

The following exceptions are only used as base classes for other
exceptions.

\begin{excdesc}{Exception}
The root class for exceptions.  All built-in exceptions are derived
from this class.  All user-defined exceptions should also be derived
from this class, but this is not (yet) enforced.  The \function{str()}
function, when applied to an instance of this class (or most derived
classes) returns the string value of the argument or arguments, or an
empty string if no arguments were given to the constructor.  When used
as a sequence, this accesses the arguments given to the constructor
(handy for backward compatibility with old code).  The arguments are
also available on the instance's \member{args} attribute, as a tuple.
\end{excdesc}

\begin{excdesc}{StandardError}
The base class for all built-in exceptions except
\exception{StopIteration} and \exception{SystemExit}.
\exception{StandardError} itself is derived from the root class
\exception{Exception}.
\end{excdesc}

\begin{excdesc}{ArithmeticError}
The base class for those built-in exceptions that are raised for
various arithmetic errors: \exception{OverflowError},
\exception{ZeroDivisionError}, \exception{FloatingPointError}.
\end{excdesc}

\begin{excdesc}{LookupError}
The base class for the exceptions that are raised when a key or
index used on a mapping or sequence is invalid: \exception{IndexError},
\exception{KeyError}.  This can be raised directly by
\function{sys.setdefaultencoding()}.
\end{excdesc}

\begin{excdesc}{EnvironmentError}
The base class for exceptions that
can occur outside the Python system: \exception{IOError},
\exception{OSError}.  When exceptions of this type are created with a
2-tuple, the first item is available on the instance's \member{errno}
attribute (it is assumed to be an error number), and the second item
is available on the \member{strerror} attribute (it is usually the
associated error message).  The tuple itself is also available on the
\member{args} attribute.
\versionadded{1.5.2}

When an \exception{EnvironmentError} exception is instantiated with a
3-tuple, the first two items are available as above, while the third
item is available on the \member{filename} attribute.  However, for
backwards compatibility, the \member{args} attribute contains only a
2-tuple of the first two constructor arguments.

The \member{filename} attribute is \code{None} when this exception is
created with other than 3 arguments.  The \member{errno} and
\member{strerror} attributes are also \code{None} when the instance was
created with other than 2 or 3 arguments.  In this last case,
\member{args} contains the verbatim constructor arguments as a tuple.
\end{excdesc}


\setindexsubitem{(built-in exception)}

The following exceptions are the exceptions that are actually raised.

\begin{excdesc}{AssertionError}
\stindex{assert}
Raised when an \keyword{assert} statement fails.
\end{excdesc}

\begin{excdesc}{AttributeError}
% xref to attribute reference?
  Raised when an attribute reference or assignment fails.  (When an
  object does not support attribute references or attribute assignments
  at all, \exception{TypeError} is raised.)
\end{excdesc}

\begin{excdesc}{EOFError}
% XXXJH xrefs here
  Raised when one of the built-in functions (\function{input()} or
  \function{raw_input()}) hits an end-of-file condition (\EOF{}) without
  reading any data.
% XXXJH xrefs here
  (N.B.: the \method{read()} and \method{readline()} methods of file
  objects return an empty string when they hit \EOF{}.)
\end{excdesc}

\begin{excdesc}{FloatingPointError}
  Raised when a floating point operation fails.  This exception is
  always defined, but can only be raised when Python is configured
  with the \longprogramopt{with-fpectl} option, or the
  \constant{WANT_SIGFPE_HANDLER} symbol is defined in the
  \file{config.h} file.
\end{excdesc}

\begin{excdesc}{IOError}
% XXXJH xrefs here
  Raised when an I/O operation (such as a \keyword{print} statement,
  the built-in \function{open()} function or a method of a file
  object) fails for an I/O-related reason, e.g., ``file not found'' or
  ``disk full''.

  This class is derived from \exception{EnvironmentError}.  See the
  discussion above for more information on exception instance
  attributes.
\end{excdesc}

\begin{excdesc}{ImportError}
% XXXJH xref to import statement?
  Raised when an \keyword{import} statement fails to find the module
  definition or when a \code{from \textrm{\ldots} import} fails to find a
  name that is to be imported.
\end{excdesc}

\begin{excdesc}{IndexError}
% XXXJH xref to sequences
  Raised when a sequence subscript is out of range.  (Slice indices are
  silently truncated to fall in the allowed range; if an index is not a
  plain integer, \exception{TypeError} is raised.)
\end{excdesc}

\begin{excdesc}{KeyError}
% XXXJH xref to mapping objects?
  Raised when a mapping (dictionary) key is not found in the set of
  existing keys.
\end{excdesc}

\begin{excdesc}{KeyboardInterrupt}
  Raised when the user hits the interrupt key (normally
  \kbd{Control-C} or \kbd{Delete}).  During execution, a check for
  interrupts is made regularly.
% XXXJH xrefs here
  Interrupts typed when a built-in function \function{input()} or
  \function{raw_input()}) is waiting for input also raise this
  exception.
\end{excdesc}

\begin{excdesc}{MemoryError}
  Raised when an operation runs out of memory but the situation may
  still be rescued (by deleting some objects).  The associated value is
  a string indicating what kind of (internal) operation ran out of memory.
  Note that because of the underlying memory management architecture
  (C's \cfunction{malloc()} function), the interpreter may not
  always be able to completely recover from this situation; it
  nevertheless raises an exception so that a stack traceback can be
  printed, in case a run-away program was the cause.
\end{excdesc}

\begin{excdesc}{NameError}
  Raised when a local or global name is not found.  This applies only
  to unqualified names.  The associated value is the name that could
  not be found.
\end{excdesc}

\begin{excdesc}{NotImplementedError}
  This exception is derived from \exception{RuntimeError}.  In user
  defined base classes, abstract methods should raise this exception
  when they require derived classes to override the method.
  \versionadded{1.5.2}
\end{excdesc}

\begin{excdesc}{OSError}
  %xref for os module
  This class is derived from \exception{EnvironmentError} and is used
  primarily as the \refmodule{os} module's \code{os.error} exception.
  See \exception{EnvironmentError} above for a description of the
  possible associated values.
  \versionadded{1.5.2}
\end{excdesc}

\begin{excdesc}{OverflowError}
% XXXJH reference to long's and/or int's?
  Raised when the result of an arithmetic operation is too large to be
  represented.  This cannot occur for long integers (which would rather
  raise \exception{MemoryError} than give up).  Because of the lack of
  standardization of floating point exception handling in C, most
  floating point operations also aren't checked.  For plain integers,
  all operations that can overflow are checked except left shift, where
  typical applications prefer to drop bits than raise an exception.
\end{excdesc}

\begin{excdesc}{RuntimeError}
  Raised when an error is detected that doesn't fall in any of the
  other categories.  The associated value is a string indicating what
  precisely went wrong.  (This exception is mostly a relic from a
  previous version of the interpreter; it is not used very much any
  more.)
\end{excdesc}

\begin{excdesc}{StopIteration}
  Raised by an iterator's \method{next()} method to signal that there
  are no further values.
  This is derived from \exception{Exception} rather than
  \exception{StandardError}, since this is not considered an error in
  its normal application.
  \versionadded{2.2}
\end{excdesc}

\begin{excdesc}{SyntaxError}
% XXXJH xref to these functions?
  Raised when the parser encounters a syntax error.  This may occur in
  an \keyword{import} statement, in an \keyword{exec} statement, in a call
  to the built-in function \function{eval()} or \function{input()}, or
  when reading the initial script or standard input (also
  interactively).

When class exceptions are used, instances of this class have
atttributes \member{filename}, \member{lineno}, \member{offset} and
\member{text} for easier access to the details; for string exceptions,
the associated value is usually a tuple of the form
\code{(message, (filename, lineno, offset, text))}.
For class exceptions, \function{str()} returns only the message.
\end{excdesc}

\begin{excdesc}{SystemError}
  Raised when the interpreter finds an internal error, but the
  situation does not look so serious to cause it to abandon all hope.
  The associated value is a string indicating what went wrong (in
  low-level terms).
  
  You should report this to the author or maintainer of your Python
  interpreter.  Be sure to report the version string of the Python
  interpreter (\code{sys.version}; it is also printed at the start of an
  interactive Python session), the exact error message (the exception's
  associated value) and if possible the source of the program that
  triggered the error.
\end{excdesc}

\begin{excdesc}{SystemExit}
% XXXJH xref to module sys?
  This exception is raised by the \function{sys.exit()} function.  When it
  is not handled, the Python interpreter exits; no stack traceback is
  printed.  If the associated value is a plain integer, it specifies the
  system exit status (passed to C's \cfunction{exit()} function); if it is
  \code{None}, the exit status is zero; if it has another type (such as
  a string), the object's value is printed and the exit status is one.

  Instances have an attribute \member{code} which is set to the
  proposed exit status or error message (defaulting to \code{None}).
  Also, this exception derives directly from \exception{Exception} and
  not \exception{StandardError}, since it is not technically an error.

  A call to \function{sys.exit()} is translated into an exception so that
  clean-up handlers (\keyword{finally} clauses of \keyword{try} statements)
  can be executed, and so that a debugger can execute a script without
  running the risk of losing control.  The \function{os._exit()} function
  can be used if it is absolutely positively necessary to exit
  immediately (for example, in the child process after a call to
  \function{fork()}).
\end{excdesc}

\begin{excdesc}{TypeError}
  Raised when a built-in operation or function is applied to an object
  of inappropriate type.  The associated value is a string giving
  details about the type mismatch.
\end{excdesc}

\begin{excdesc}{UnboundLocalError}
  Raised when a reference is made to a local variable in a function or
  method, but no value has been bound to that variable.  This is a
  subclass of \exception{NameError}.
\versionadded{2.0}
\end{excdesc}

\begin{excdesc}{UnicodeError}
  Raised when a Unicode-related encoding or decoding error occurs.  It
  is a subclass of \exception{ValueError}.
\versionadded{2.0}
\end{excdesc}

\begin{excdesc}{ValueError}
  Raised when a built-in operation or function receives an argument
  that has the right type but an inappropriate value, and the
  situation is not described by a more precise exception such as
  \exception{IndexError}.
\end{excdesc}

\begin{excdesc}{WindowsError}
  Raised when a Windows-specific error occurs or when the error number
  does not correspond to an \cdata{errno} value.  The
  \member{errno} and \member{strerror} values are created from the
  return values of the \cfunction{GetLastError()} and
  \cfunction{FormatMessage()} functions from the Windows Platform API.
  This is a subclass of \exception{OSError}.
\versionadded{2.0}
\end{excdesc}

\begin{excdesc}{ZeroDivisionError}
  Raised when the second argument of a division or modulo operation is
  zero.  The associated value is a string indicating the type of the
  operands and the operation.
\end{excdesc}


\setindexsubitem{(built-in warning category)}

The following exceptions are used as warning categories; see the
\module{warnings} module for more information.

\begin{excdesc}{Warning}
Base class for warning categories.
\end{excdesc}

\begin{excdesc}{UserWarning}
Base class for warnings generated by user code.
\end{excdesc}

\begin{excdesc}{DeprecationWarning}
Base class for warnings about deprecated features.
\end{excdesc}

\begin{excdesc}{SyntaxWarning}
Base class for warnings about dubious syntax
\end{excdesc}

\begin{excdesc}{RuntimeWarning}
Base class for warnings about dubious runtime behavior.
\end{excdesc}

\section{Built-in Functions \label{built-in-funcs}}

The Python interpreter has a number of functions built into it that
are always available.  They are listed here in alphabetical order.


\setindexsubitem{(built-in function)}

\begin{funcdesc}{__import__}{name\optional{, globals\optional{, locals\optional{, fromlist}}}}
  This function is invoked by the \keyword{import}\stindex{import}
  statement.  It mainly exists so that you can replace it with another
  function that has a compatible interface, in order to change the
  semantics of the \keyword{import} statement.  For examples of why
  and how you would do this, see the standard library modules
  \module{ihooks}\refstmodindex{ihooks} and
  \refmodule{rexec}\refstmodindex{rexec}.  See also the built-in
  module \refmodule{imp}\refbimodindex{imp}, which defines some useful
  operations out of which you can build your own
  \function{__import__()} function.

  For example, the statement \samp{import spam} results in the
  following call: \code{__import__('spam',} \code{globals(),}
  \code{locals(), [])}; the statement \samp{from spam.ham import eggs}
  results in \samp{__import__('spam.ham', globals(), locals(),
  ['eggs'])}.  Note that even though \code{locals()} and
  \code{['eggs']} are passed in as arguments, the
  \function{__import__()} function does not set the local variable
  named \code{eggs}; this is done by subsequent code that is generated
  for the import statement.  (In fact, the standard implementation
  does not use its \var{locals} argument at all, and uses its
  \var{globals} only to determine the package context of the
  \keyword{import} statement.)

  When the \var{name} variable is of the form \code{package.module},
  normally, the top-level package (the name up till the first dot) is
  returned, \emph{not} the module named by \var{name}.  However, when
  a non-empty \var{fromlist} argument is given, the module named by
  \var{name} is returned.  This is done for compatibility with the
  bytecode generated for the different kinds of import statement; when
  using \samp{import spam.ham.eggs}, the top-level package \module{spam}
  must be placed in the importing namespace, but when using \samp{from
  spam.ham import eggs}, the \code{spam.ham} subpackage must be used
  to find the \code{eggs} variable.  As a workaround for this
  behavior, use \function{getattr()} to extract the desired
  components.  For example, you could define the following helper:

\begin{verbatim}
def my_import(name):
    mod = __import__(name)
    components = name.split('.')
    for comp in components[1:]:
        mod = getattr(mod, comp)
    return mod
\end{verbatim}
\end{funcdesc}

\begin{funcdesc}{abs}{x}
  Return the absolute value of a number.  The argument may be a plain
  or long integer or a floating point number.  If the argument is a
  complex number, its magnitude is returned.
\end{funcdesc}

\begin{funcdesc}{basestring}{}
  This abstract type is the superclass for \class{str} and \class{unicode}.
  It cannot be called or instantiated, but it can be used to test whether
  an object is an instance of \class{str} or \class{unicode}.
  \code{isinstance(obj, basestring)} is equivalent to
  \code{isinstance(obj, (str, unicode))}.
  \versionadded{2.3}
\end{funcdesc}

\begin{funcdesc}{bool}{\optional{x}}
  Convert a value to a Boolean, using the standard truth testing
  procedure.  If \var{x} is false or omitted, this returns
  \constant{False}; otherwise it returns \constant{True}.
  \class{bool} is also a class, which is a subclass of \class{int}.
  Class \class{bool} cannot be subclassed further.  Its only instances
  are \constant{False} and \constant{True}.

  \indexii{Boolean}{type}
  \versionadded{2.2.1}
  \versionchanged[If no argument is given, this function returns 
                  \constant{False}]{2.3}
\end{funcdesc}

\begin{funcdesc}{callable}{object}
  Return true if the \var{object} argument appears callable, false if
  not.  If this returns true, it is still possible that a call fails,
  but if it is false, calling \var{object} will never succeed.  Note
  that classes are callable (calling a class returns a new instance);
  class instances are callable if they have a \method{__call__()}
  method.
\end{funcdesc}

\begin{funcdesc}{chr}{i}
  Return a string of one character whose \ASCII{} code is the integer
  \var{i}.  For example, \code{chr(97)} returns the string \code{'a'}.
  This is the inverse of \function{ord()}.  The argument must be in
  the range [0..255], inclusive; \exception{ValueError} will be raised
  if \var{i} is outside that range.
\end{funcdesc}

\begin{funcdesc}{classmethod}{function}
  Return a class method for \var{function}.

  A class method receives the class as implicit first argument,
  just like an instance method receives the instance.
  To declare a class method, use this idiom:

\begin{verbatim}
class C:
    def f(cls, arg1, arg2, ...): ...
    f = classmethod(f)
\end{verbatim}

  It can be called either on the class (such as \code{C.f()}) or on an
  instance (such as \code{C().f()}).  The instance is ignored except for
  its class.
  If a class method is called for a derived class, the derived class
  object is passed as the implied first argument.

  Class methods are different than \Cpp{} or Java static methods.
  If you want those, see \function{staticmethod()} in this section.
  \versionadded{2.2}
\end{funcdesc}

\begin{funcdesc}{cmp}{x, y}
  Compare the two objects \var{x} and \var{y} and return an integer
  according to the outcome.  The return value is negative if \code{\var{x}
  < \var{y}}, zero if \code{\var{x} == \var{y}} and strictly positive if
  \code{\var{x} > \var{y}}.
\end{funcdesc}

\begin{funcdesc}{compile}{string, filename, kind\optional{,
                          flags\optional{, dont_inherit}}}
  Compile the \var{string} into a code object.  Code objects can be
  executed by an \keyword{exec} statement or evaluated by a call to
  \function{eval()}.  The \var{filename} argument should
  give the file from which the code was read; pass some recognizable value
  if it wasn't read from a file (\code{'<string>'} is commonly used).
  The \var{kind} argument specifies what kind of code must be
  compiled; it can be \code{'exec'} if \var{string} consists of a
  sequence of statements, \code{'eval'} if it consists of a single
  expression, or \code{'single'} if it consists of a single
  interactive statement (in the latter case, expression statements
  that evaluate to something else than \code{None} will printed).

  When compiling multi-line statements, two caveats apply: line
  endings must be represented by a single newline character
  (\code{'\e n'}), and the input must be terminated by at least one
  newline character.  If line endings are represented by
  \code{'\e r\e n'}, use the string \method{replace()} method to
  change them into \code{'\e n'}.

  The optional arguments \var{flags} and \var{dont_inherit}
  (which are new in Python 2.2) control which future statements (see
  \pep{236}) affect the compilation of \var{string}.  If neither is
  present (or both are zero) the code is compiled with those future
  statements that are in effect in the code that is calling compile.
  If the \var{flags} argument is given and \var{dont_inherit} is not
  (or is zero) then the future statements specified by the \var{flags}
  argument are used in addition to those that would be used anyway.
  If \var{dont_inherit} is a non-zero integer then the \var{flags}
  argument is it -- the future statements in effect around the call to
  compile are ignored.

  Future statemants are specified by bits which can be bitwise or-ed
  together to specify multiple statements.  The bitfield required to
  specify a given feature can be found as the \member{compiler_flag}
  attribute on the \class{_Feature} instance in the
  \module{__future__} module.
\end{funcdesc}

\begin{funcdesc}{complex}{\optional{real\optional{, imag}}}
  Create a complex number with the value \var{real} + \var{imag}*j or
  convert a string or number to a complex number.  If the first
  parameter is a string, it will be interpreted as a complex number
  and the function must be called without a second parameter.  The
  second parameter can never be a string.
  Each argument may be any numeric type (including complex).
  If \var{imag} is omitted, it defaults to zero and the function
  serves as a numeric conversion function like \function{int()},
  \function{long()} and \function{float()}.  If both arguments
  are omitted, returns \code{0j}.
\end{funcdesc}

\begin{funcdesc}{delattr}{object, name}
  This is a relative of \function{setattr()}.  The arguments are an
  object and a string.  The string must be the name
  of one of the object's attributes.  The function deletes
  the named attribute, provided the object allows it.  For example,
  \code{delattr(\var{x}, '\var{foobar}')} is equivalent to
  \code{del \var{x}.\var{foobar}}.
\end{funcdesc}

\begin{funcdesc}{dict}{\optional{mapping-or-sequence}}
  Return a new dictionary initialized from an optional positional
  argument or from a set of keyword arguments.
  If no arguments are given, return a new empty dictionary.
  If the positional argument is a mapping object, return a dictionary
  mapping the same keys to the same values as does the mapping object.
  Otherwise the positional argument must be a sequence, a container that
  supports iteration, or an iterator object.  The elements of the argument
  must each also be of one of those kinds, and each must in turn contain
  exactly two objects.  The first is used as a key in the new dictionary,
  and the second as the key's value.  If a given key is seen more than
  once, the last value associated with it is retained in the new
  dictionary.

  If keyword arguments are given, the keywords themselves with their
  associated values are added as items to the dictionary. If a key
  is specified both in the positional argument and as a keyword argument,
  the value associated with the keyword is retained in the dictionary.
  For example, these all return a dictionary equal to
  \code{\{"one": 2, "two": 3\}}:

  \begin{itemize}
    \item \code{dict(\{'one': 2, 'two': 3\})}
    \item \code{dict(\{'one': 2, 'two': 3\}.items())}
    \item \code{dict(\{'one': 2, 'two': 3\}.iteritems())}
    \item \code{dict(zip(('one', 'two'), (2, 3)))}
    \item \code{dict([['two', 3], ['one', 2]])}
    \item \code{dict(one=2, two=3)}
    \item \code{dict([(['one', 'two'][i-2], i) for i in (2, 3)])}
  \end{itemize}

  \versionadded{2.2}
  \versionchanged[Support for building a dictionary from keyword
                  arguments added]{2.3}
\end{funcdesc}

\begin{funcdesc}{dir}{\optional{object}}
  Without arguments, return the list of names in the current local
  symbol table.  With an argument, attempts to return a list of valid
  attributes for that object.  This information is gleaned from the
  object's \member{__dict__} attribute, if defined, and from the class
  or type object.  The list is not necessarily complete.
  If the object is a module object, the list contains the names of the
  module's attributes.
  If the object is a type or class object,
  the list contains the names of its attributes,
  and recursively of the attributes of its bases.
  Otherwise, the list contains the object's attributes' names,
  the names of its class's attributes,
  and recursively of the attributes of its class's base classes.
  The resulting list is sorted alphabetically.
  For example:

\begin{verbatim}
>>> import struct
>>> dir()
['__builtins__', '__doc__', '__name__', 'struct']
>>> dir(struct)
['__doc__', '__name__', 'calcsize', 'error', 'pack', 'unpack']
\end{verbatim}

  \note{Because \function{dir()} is supplied primarily as a convenience
  for use at an interactive prompt,
  it tries to supply an interesting set of names more than it tries to
  supply a rigorously or consistently defined set of names,
  and its detailed behavior may change across releases.}
\end{funcdesc}

\begin{funcdesc}{divmod}{a, b}
  Take two (non complex) numbers as arguments and return a pair of numbers
  consisting of their quotient and remainder when using long division.  With
  mixed operand types, the rules for binary arithmetic operators apply.  For
  plain and long integers, the result is the same as
  \code{(\var{a} / \var{b}, \var{a} \%{} \var{b})}.
  For floating point numbers the result is \code{(\var{q}, \var{a} \%{}
  \var{b})}, where \var{q} is usually \code{math.floor(\var{a} /
  \var{b})} but may be 1 less than that.  In any case \code{\var{q} *
  \var{b} + \var{a} \%{} \var{b}} is very close to \var{a}, if
  \code{\var{a} \%{} \var{b}} is non-zero it has the same sign as
  \var{b}, and \code{0 <= abs(\var{a} \%{} \var{b}) < abs(\var{b})}.

  \versionchanged[Using \function{divmod()} with complex numbers is
                  deprecated]{2.3}
\end{funcdesc}

\begin{funcdesc}{enumerate}{iterable}
  Return an enumerate object. \var{iterable} must be a sequence, an
  iterator, or some other object which supports iteration.  The
  \method{next()} method of the iterator returned by
  \function{enumerate()} returns a tuple containing a count (from
  zero) and the corresponding value obtained from iterating over
  \var{iterable}.  \function{enumerate()} is useful for obtaining an
  indexed series: \code{(0, seq[0])}, \code{(1, seq[1])}, \code{(2,
  seq[2])}, \ldots.
  \versionadded{2.3}
\end{funcdesc}

\begin{funcdesc}{eval}{expression\optional{, globals\optional{, locals}}}
  The arguments are a string and two optional dictionaries.  The
  \var{expression} argument is parsed and evaluated as a Python
  expression (technically speaking, a condition list) using the
  \var{globals} and \var{locals} dictionaries as global and local name
  space.  If the \var{globals} dictionary is present and lacks
  '__builtins__', the current globals are copied into \var{globals} before
  \var{expression} is parsed.  This means that \var{expression}
  normally has full access to the standard
  \refmodule[builtin]{__builtin__} module and restricted environments
  are propagated.  If the \var{locals} dictionary is omitted it defaults to
  the \var{globals} dictionary.  If both dictionaries are omitted, the
  expression is executed in the environment where \keyword{eval} is
  called.  The return value is the result of the evaluated expression.
  Syntax errors are reported as exceptions.  Example:

\begin{verbatim}
>>> x = 1
>>> print eval('x+1')
2
\end{verbatim}

  This function can also be used to execute arbitrary code objects
  (such as those created by \function{compile()}).  In this case pass
  a code object instead of a string.  The code object must have been
  compiled passing \code{'eval'} as the \var{kind} argument.

  Hints: dynamic execution of statements is supported by the
  \keyword{exec} statement.  Execution of statements from a file is
  supported by the \function{execfile()} function.  The
  \function{globals()} and \function{locals()} functions returns the
  current global and local dictionary, respectively, which may be
  useful to pass around for use by \function{eval()} or
  \function{execfile()}.
\end{funcdesc}

\begin{funcdesc}{execfile}{filename\optional{, globals\optional{, locals}}}
  This function is similar to the
  \keyword{exec} statement, but parses a file instead of a string.  It
  is different from the \keyword{import} statement in that it does not
  use the module administration --- it reads the file unconditionally
  and does not create a new module.\footnote{It is used relatively
  rarely so does not warrant being made into a statement.}

  The arguments are a file name and two optional dictionaries.  The
  file is parsed and evaluated as a sequence of Python statements
  (similarly to a module) using the \var{globals} and \var{locals}
  dictionaries as global and local namespace.  If the \var{locals}
  dictionary is omitted it defaults to the \var{globals} dictionary.
  If both dictionaries are omitted, the expression is executed in the
  environment where \function{execfile()} is called.  The return value is
  \code{None}.

  \warning{The default \var{locals} act as described for function
  \function{locals()} below:  modifications to the default \var{locals}
  dictionary should not be attempted.  Pass an explicit \var{locals}
  dictionary if you need to see effects of the code on \var{locals} after
  function \function{execfile()} returns.  \function{execfile()} cannot
  be used reliably to modify a function's locals.}
\end{funcdesc}

\begin{funcdesc}{file}{filename\optional{, mode\optional{, bufsize}}}
  Return a new file object (described in
  section~\ref{bltin-file-objects}, ``\ulink{File
  Objects}{bltin-file-objects.html}'').
  The first two arguments are the same as for \code{stdio}'s
  \cfunction{fopen()}: \var{filename} is the file name to be opened,
  \var{mode} indicates how the file is to be opened: \code{'r'} for
  reading, \code{'w'} for writing (truncating an existing file), and
  \code{'a'} opens it for appending (which on \emph{some} \UNIX{}
  systems means that \emph{all} writes append to the end of the file,
  regardless of the current seek position).

  Modes \code{'r+'}, \code{'w+'} and \code{'a+'} open the file for
  updating (note that \code{'w+'} truncates the file).  Append
  \code{'b'} to the mode to open the file in binary mode, on systems
  that differentiate between binary and text files (else it is
  ignored).  If the file cannot be opened, \exception{IOError} is
  raised.
  
  In addition to the standard \cfunction{fopen()} values \var{mode}
  may be \code{'U'} or \code{'rU'}. If Python is built with universal
  newline support (the default) the file is opened as a text file, but
  lines may be terminated by any of \code{'\e n'}, the Unix end-of-line
  convention,
  \code{'\e r'}, the Macintosh convention or \code{'\e r\e n'}, the Windows
  convention. All of these external representations are seen as
  \code{'\e n'}
  by the Python program. If Python is built without universal newline support
  \var{mode} \code{'U'} is the same as normal text mode.  Note that
  file objects so opened also have an attribute called
  \member{newlines} which has a value of \code{None} (if no newlines
  have yet been seen), \code{'\e n'}, \code{'\e r'}, \code{'\e r\e n'}, 
  or a tuple containing all the newline types seen.

  If \var{mode} is omitted, it defaults to \code{'r'}.  When opening a
  binary file, you should append \code{'b'} to the \var{mode} value
  for improved portability.  (It's useful even on systems which don't
  treat binary and text files differently, where it serves as
  documentation.)
  \index{line-buffered I/O}\index{unbuffered I/O}\index{buffer size, I/O}
  \index{I/O control!buffering}
  The optional \var{bufsize} argument specifies the
  file's desired buffer size: 0 means unbuffered, 1 means line
  buffered, any other positive value means use a buffer of
  (approximately) that size.  A negative \var{bufsize} means to use
  the system default, which is usually line buffered for tty
  devices and fully buffered for other files.  If omitted, the system
  default is used.\footnote{
    Specifying a buffer size currently has no effect on systems that
    don't have \cfunction{setvbuf()}.  The interface to specify the
    buffer size is not done using a method that calls
    \cfunction{setvbuf()}, because that may dump core when called
    after any I/O has been performed, and there's no reliable way to
    determine whether this is the case.}

  The \function{file()} constructor is new in Python 2.2.  The previous
  spelling, \function{open()}, is retained for compatibility, and is an
  alias for \function{file()}.
\end{funcdesc}

\begin{funcdesc}{filter}{function, list}
  Construct a list from those elements of \var{list} for which
  \var{function} returns true.  \var{list} may be either a sequence, a
  container which supports iteration, or an iterator,  If \var{list}
  is a string or a tuple, the result also has that type; otherwise it
  is always a list.  If \var{function} is \code{None}, the identity
  function is assumed, that is, all elements of \var{list} that are false
  (zero or empty) are removed.

  Note that \code{filter(function, \var{list})} is equivalent to
  \code{[item for item in \var{list} if function(item)]} if function is
  not \code{None} and \code{[item for item in \var{list} if item]} if
  function is \code{None}.
\end{funcdesc}

\begin{funcdesc}{float}{\optional{x}}
  Convert a string or a number to floating point.  If the argument is a
  string, it must contain a possibly signed decimal or floating point
  number, possibly embedded in whitespace. Otherwise, the argument may be a plain
  or long integer or a floating point number, and a floating point
  number with the same value (within Python's floating point
  precision) is returned.  If no argument is given, returns \code{0.0}.

  \note{When passing in a string, values for NaN\index{NaN}
  and Infinity\index{Infinity} may be returned, depending on the
  underlying C library.  The specific set of strings accepted which
  cause these values to be returned depends entirely on the C library
  and is known to vary.}
\end{funcdesc}

\begin{funcdesc}{frozenset}{\optional{iterable}}
  Return a frozenset object whose elements are taken from \var{iterable}.
  Frozensets are sets that have no update methods but can be hashed and
  used as members of other sets or as dictionary keys.  The elements of
  a frozenset must be immutable themselves.  To represent sets of sets,
  the inner sets should also be \class{frozenset} objects.  If
  \var{iterable} is not specified, returns a new empty set,
  \code{frozenset([])}.
  \versionadded{2.4}  
\end{funcdesc}

\begin{funcdesc}{getattr}{object, name\optional{, default}}
  Return the value of the named attributed of \var{object}.  \var{name}
  must be a string.  If the string is the name of one of the object's
  attributes, the result is the value of that attribute.  For example,
  \code{getattr(x, 'foobar')} is equivalent to \code{x.foobar}.  If the
  named attribute does not exist, \var{default} is returned if provided,
  otherwise \exception{AttributeError} is raised.
\end{funcdesc}

\begin{funcdesc}{globals}{}
  Return a dictionary representing the current global symbol table.
  This is always the dictionary of the current module (inside a
  function or method, this is the module where it is defined, not the
  module from which it is called).
\end{funcdesc}

\begin{funcdesc}{hasattr}{object, name}
  The arguments are an object and a string.  The result is \code{True} if the
  string is the name of one of the object's attributes, \code{False} if not.
  (This is implemented by calling \code{getattr(\var{object},
  \var{name})} and seeing whether it raises an exception or not.)
\end{funcdesc}

\begin{funcdesc}{hash}{object}
  Return the hash value of the object (if it has one).  Hash values
  are integers.  They are used to quickly compare dictionary
  keys during a dictionary lookup.  Numeric values that compare equal
  have the same hash value (even if they are of different types, as is
  the case for 1 and 1.0).
\end{funcdesc}

\begin{funcdesc}{help}{\optional{object}}
  Invoke the built-in help system.  (This function is intended for
  interactive use.)  If no argument is given, the interactive help
  system starts on the interpreter console.  If the argument is a
  string, then the string is looked up as the name of a module,
  function, class, method, keyword, or documentation topic, and a
  help page is printed on the console.  If the argument is any other
  kind of object, a help page on the object is generated.
  \versionadded{2.2}
\end{funcdesc}

\begin{funcdesc}{hex}{x}
  Convert an integer number (of any size) to a hexadecimal string.
  The result is a valid Python expression.  Note: this always yields
  an unsigned literal.  For example, on a 32-bit machine,
  \code{hex(-1)} yields \code{'0xffffffff'}.  When evaluated on a
  machine with the same word size, this literal is evaluated as -1; at
  a different word size, it may turn up as a large positive number or
  raise an \exception{OverflowError} exception.
\end{funcdesc}

\begin{funcdesc}{id}{object}
  Return the `identity' of an object.  This is an integer (or long
  integer) which is guaranteed to be unique and constant for this
  object during its lifetime.  Two objects whose lifetimes are
  disjunct may have the same \function{id()} value.  (Implementation
  note: this is the address of the object.)
\end{funcdesc}

\begin{funcdesc}{input}{\optional{prompt}}
  Equivalent to \code{eval(raw_input(\var{prompt}))}.
  \warning{This function is not safe from user errors!  It
  expects a valid Python expression as input; if the input is not
  syntactically valid, a \exception{SyntaxError} will be raised.
  Other exceptions may be raised if there is an error during
  evaluation.  (On the other hand, sometimes this is exactly what you
  need when writing a quick script for expert use.)}

  If the \refmodule{readline} module was loaded, then
  \function{input()} will use it to provide elaborate line editing and
  history features.

  Consider using the \function{raw_input()} function for general input
  from users.
\end{funcdesc}

\begin{funcdesc}{int}{\optional{x\optional{, radix}}}
  Convert a string or number to a plain integer.  If the argument is a
  string, it must contain a possibly signed decimal number
  representable as a Python integer, possibly embedded in whitespace.
  The \var{radix} parameter gives the base for the
  conversion and may be any integer in the range [2, 36], or zero.  If
  \var{radix} is zero, the proper radix is guessed based on the
  contents of string; the interpretation is the same as for integer
  literals.  If \var{radix} is specified and \var{x} is not a string,
  \exception{TypeError} is raised.
  Otherwise, the argument may be a plain or
  long integer or a floating point number.  Conversion of floating
  point numbers to integers truncates (towards zero).
  If the argument is outside the integer range a long object will
  be returned instead.  If no arguments are given, returns \code{0}.
\end{funcdesc}

\begin{funcdesc}{isinstance}{object, classinfo}
  Return true if the \var{object} argument is an instance of the
  \var{classinfo} argument, or of a (direct or indirect) subclass
  thereof.  Also return true if \var{classinfo} is a type object and
  \var{object} is an object of that type.  If \var{object} is not a
  class instance or an object of the given type, the function always
  returns false.  If \var{classinfo} is neither a class object nor a
  type object, it may be a tuple of class or type objects, or may
  recursively contain other such tuples (other sequence types are not
  accepted).  If \var{classinfo} is not a class, type, or tuple of
  classes, types, and such tuples, a \exception{TypeError} exception
  is raised.
  \versionchanged[Support for a tuple of type information was added]{2.2}
\end{funcdesc}

\begin{funcdesc}{issubclass}{class, classinfo}
  Return true if \var{class} is a subclass (direct or indirect) of
  \var{classinfo}.  A class is considered a subclass of itself.
  \var{classinfo} may be a tuple of class objects, in which case every
  entry in \var{classinfo} will be checked. In any other case, a
  \exception{TypeError} exception is raised.
  \versionchanged[Support for a tuple of type information was added]{2.3}
\end{funcdesc}

\begin{funcdesc}{iter}{o\optional{, sentinel}}
  Return an iterator object.  The first argument is interpreted very
  differently depending on the presence of the second argument.
  Without a second argument, \var{o} must be a collection object which
  supports the iteration protocol (the \method{__iter__()} method), or
  it must support the sequence protocol (the \method{__getitem__()}
  method with integer arguments starting at \code{0}).  If it does not
  support either of those protocols, \exception{TypeError} is raised.
  If the second argument, \var{sentinel}, is given, then \var{o} must
  be a callable object.  The iterator created in this case will call
  \var{o} with no arguments for each call to its \method{next()}
  method; if the value returned is equal to \var{sentinel},
  \exception{StopIteration} will be raised, otherwise the value will
  be returned.
  \versionadded{2.2}
\end{funcdesc}

\begin{funcdesc}{len}{s}
  Return the length (the number of items) of an object.  The argument
  may be a sequence (string, tuple or list) or a mapping (dictionary).
\end{funcdesc}

\begin{funcdesc}{list}{\optional{sequence}}
  Return a list whose items are the same and in the same order as
  \var{sequence}'s items.  \var{sequence} may be either a sequence, a
  container that supports iteration, or an iterator object.  If
  \var{sequence} is already a list, a copy is made and returned,
  similar to \code{\var{sequence}[:]}.  For instance,
  \code{list('abc')} returns \code{['a', 'b', 'c']} and \code{list(
  (1, 2, 3) )} returns \code{[1, 2, 3]}.  If no argument is given,
  returns a new empty list, \code{[]}.
\end{funcdesc}

\begin{funcdesc}{locals}{}
  Update and return a dictionary representing the current local symbol table.
  \warning{The contents of this dictionary should not be modified;
  changes may not affect the values of local variables used by the
  interpreter.}
\end{funcdesc}

\begin{funcdesc}{long}{\optional{x\optional{, radix}}}
  Convert a string or number to a long integer.  If the argument is a
  string, it must contain a possibly signed number of
  arbitrary size, possibly embedded in whitespace. The
  \var{radix} argument is interpreted in the same way as for
  \function{int()}, and may only be given when \var{x} is a string.
  Otherwise, the argument may be a plain or
  long integer or a floating point number, and a long integer with
  the same value is returned.    Conversion of floating
  point numbers to integers truncates (towards zero).  If no arguments
  are given, returns \code{0L}.
\end{funcdesc}

\begin{funcdesc}{map}{function, list, ...}
  Apply \var{function} to every item of \var{list} and return a list
  of the results.  If additional \var{list} arguments are passed,
  \var{function} must take that many arguments and is applied to the
  items of all lists in parallel; if a list is shorter than another it
  is assumed to be extended with \code{None} items.  If \var{function}
  is \code{None}, the identity function is assumed; if there are
  multiple list arguments, \function{map()} returns a list consisting
  of tuples containing the corresponding items from all lists (a kind
  of transpose operation).  The \var{list} arguments may be any kind
  of sequence; the result is always a list.
\end{funcdesc}

\begin{funcdesc}{max}{s\optional{, args...}}
  With a single argument \var{s}, return the largest item of a
  non-empty sequence (such as a string, tuple or list).  With more
  than one argument, return the largest of the arguments.
\end{funcdesc}

\begin{funcdesc}{min}{s\optional{, args...}}
  With a single argument \var{s}, return the smallest item of a
  non-empty sequence (such as a string, tuple or list).  With more
  than one argument, return the smallest of the arguments.
\end{funcdesc}

\begin{funcdesc}{object}{}
  Return a new featureless object.  \function{object()} is a base 
  for all new style classes.  It has the methods that are common
  to all instances of new style classes.
  \versionadded{2.2}

  \versionchanged[This function does not accept any arguments.
  Formerly, it accepted arguments but ignored them]{2.3}
\end{funcdesc}

\begin{funcdesc}{oct}{x}
  Convert an integer number (of any size) to an octal string.  The
  result is a valid Python expression.  Note: this always yields an
  unsigned literal.  For example, on a 32-bit machine, \code{oct(-1)}
  yields \code{'037777777777'}.  When evaluated on a machine with the
  same word size, this literal is evaluated as -1; at a different word
  size, it may turn up as a large positive number or raise an
  \exception{OverflowError} exception.
\end{funcdesc}

\begin{funcdesc}{open}{filename\optional{, mode\optional{, bufsize}}}
  An alias for the \function{file()} function above.
\end{funcdesc}

\begin{funcdesc}{ord}{c}
  Return the \ASCII{} value of a string of one character or a Unicode
  character.  E.g., \code{ord('a')} returns the integer \code{97},
  \code{ord(u'\e u2020')} returns \code{8224}.  This is the inverse of
  \function{chr()} for strings and of \function{unichr()} for Unicode
  characters.
\end{funcdesc}

\begin{funcdesc}{pow}{x, y\optional{, z}}
  Return \var{x} to the power \var{y}; if \var{z} is present, return
  \var{x} to the power \var{y}, modulo \var{z} (computed more
  efficiently than \code{pow(\var{x}, \var{y}) \%\ \var{z}}).  The
  arguments must have numeric types.  With mixed operand types, the
  coercion rules for binary arithmetic operators apply.  For int and
  long int operands, the result has the same type as the operands
  (after coercion) unless the second argument is negative; in that
  case, all arguments are converted to float and a float result is
  delivered.  For example, \code{10**2} returns \code{100}, but
  \code{10**-2} returns \code{0.01}.  (This last feature was added in
  Python 2.2.  In Python 2.1 and before, if both arguments were of integer
  types and the second argument was negative, an exception was raised.)
  If the second argument is negative, the third argument must be omitted.
  If \var{z} is present, \var{x} and \var{y} must be of integer types,
  and \var{y} must be non-negative.  (This restriction was added in
  Python 2.2.  In Python 2.1 and before, floating 3-argument \code{pow()}
  returned platform-dependent results depending on floating-point
  rounding accidents.)
\end{funcdesc}

\begin{funcdesc}{property}{\optional{fget\optional{, fset\optional{,
                           fdel\optional{, doc}}}}}
  Return a property attribute for new-style classes (classes that
  derive from \class{object}).

  \var{fget} is a function for getting an attribute value, likewise
  \var{fset} is a function for setting, and \var{fdel} a function
  for del'ing, an attribute.  Typical use is to define a managed attribute x:

\begin{verbatim}
class C(object):
    def getx(self): return self.__x
    def setx(self, value): self.__x = value
    def delx(self): del self.__x
    x = property(getx, setx, delx, "I'm the 'x' property.")
\end{verbatim}

  \versionadded{2.2}
\end{funcdesc}

\begin{funcdesc}{range}{\optional{start,} stop\optional{, step}}
  This is a versatile function to create lists containing arithmetic
  progressions.  It is most often used in \keyword{for} loops.  The
  arguments must be plain integers.  If the \var{step} argument is
  omitted, it defaults to \code{1}.  If the \var{start} argument is
  omitted, it defaults to \code{0}.  The full form returns a list of
  plain integers \code{[\var{start}, \var{start} + \var{step},
  \var{start} + 2 * \var{step}, \ldots]}.  If \var{step} is positive,
  the last element is the largest \code{\var{start} + \var{i} *
  \var{step}} less than \var{stop}; if \var{step} is negative, the last
  element is the largest \code{\var{start} + \var{i} * \var{step}}
  greater than \var{stop}.  \var{step} must not be zero (or else
  \exception{ValueError} is raised).  Example:

\begin{verbatim}
>>> range(10)
[0, 1, 2, 3, 4, 5, 6, 7, 8, 9]
>>> range(1, 11)
[1, 2, 3, 4, 5, 6, 7, 8, 9, 10]
>>> range(0, 30, 5)
[0, 5, 10, 15, 20, 25]
>>> range(0, 10, 3)
[0, 3, 6, 9]
>>> range(0, -10, -1)
[0, -1, -2, -3, -4, -5, -6, -7, -8, -9]
>>> range(0)
[]
>>> range(1, 0)
[]
\end{verbatim}
\end{funcdesc}

\begin{funcdesc}{raw_input}{\optional{prompt}}
  If the \var{prompt} argument is present, it is written to standard output
  without a trailing newline.  The function then reads a line from input,
  converts it to a string (stripping a trailing newline), and returns that.
  When \EOF{} is read, \exception{EOFError} is raised. Example:

\begin{verbatim}
>>> s = raw_input('--> ')
--> Monty Python's Flying Circus
>>> s
"Monty Python's Flying Circus"
\end{verbatim}

  If the \refmodule{readline} module was loaded, then
  \function{raw_input()} will use it to provide elaborate
  line editing and history features.
\end{funcdesc}

\begin{funcdesc}{reduce}{function, sequence\optional{, initializer}}
  Apply \var{function} of two arguments cumulatively to the items of
  \var{sequence}, from left to right, so as to reduce the sequence to
  a single value.  For example, \code{reduce(lambda x, y: x+y, [1, 2,
  3, 4, 5])} calculates \code{((((1+2)+3)+4)+5)}.  The left argument,
  \var{x}, is the accumulated value and the right argument, \var{y},
  is the update value from the \var{sequence}.  If the optional
  \var{initializer} is present, it is placed before the items of the
  sequence in the calculation, and serves as a default when the
  sequence is empty.  If \var{initializer} is not given and
  \var{sequence} contains only one item, the first item is returned.
\end{funcdesc}

\begin{funcdesc}{reload}{module}
  Reload a previously imported \var{module}.  The
  argument must be a module object, so it must have been successfully
  imported before.  This is useful if you have edited the module
  source file using an external editor and want to try out the new
  version without leaving the Python interpreter.  The return value is
  the module object (the same as the \var{module} argument).

  When \code{reload(module)} is executed:

\begin{itemize}

    \item{Python modules' code is recompiled and the module-level code
    reexecuted, defining a new set of objects which are bound to names in
    the module's dictionary.  The \code{init} function of extension
    modules is not called a second time.}

    \item{As with all other objects in Python the old objects are only
    reclaimed after their reference counts drop to zero.}

    \item{The names in the module namespace are updated to point to
    any new or changed objects.}

    \item{Other references to the old objects (such as names external
    to the module) are not rebound to refer to the new objects and
    must be updated in each namespace where they occur if that is
    desired.}

\end{itemize}

  There are a number of other caveats:

  If a module is syntactically correct but its initialization fails,
  the first \keyword{import} statement for it does not bind its name
  locally, but does store a (partially initialized) module object in
  \code{sys.modules}.  To reload the module you must first
  \keyword{import} it again (this will bind the name to the partially
  initialized module object) before you can \function{reload()} it.

  When a module is reloaded, its dictionary (containing the module's
  global variables) is retained.  Redefinitions of names will override
  the old definitions, so this is generally not a problem.  If the new
  version of a module does not define a name that was defined by the
  old version, the old definition remains.  This feature can be used
  to the module's advantage if it maintains a global table or cache of
  objects --- with a \keyword{try} statement it can test for the
  table's presence and skip its initialization if desired.

  It is legal though generally not very useful to reload built-in or
  dynamically loaded modules, except for \refmodule{sys},
  \refmodule[main]{__main__} and \refmodule[builtin]{__builtin__}.  In
  many cases, however, extension modules are not designed to be
  initialized more than once, and may fail in arbitrary ways when
  reloaded.

  If a module imports objects from another module using \keyword{from}
  \ldots{} \keyword{import} \ldots{}, calling \function{reload()} for
  the other module does not redefine the objects imported from it ---
  one way around this is to re-execute the \keyword{from} statement,
  another is to use \keyword{import} and qualified names
  (\var{module}.\var{name}) instead.

  If a module instantiates instances of a class, reloading the module
  that defines the class does not affect the method definitions of the
  instances --- they continue to use the old class definition.  The
  same is true for derived classes.
\end{funcdesc}

\begin{funcdesc}{repr}{object}
  Return a string containing a printable representation of an object.
  This is the same value yielded by conversions (reverse quotes).
  It is sometimes useful to be able to access this operation as an
  ordinary function.  For many types, this function makes an attempt
  to return a string that would yield an object with the same value
  when passed to \function{eval()}.
\end{funcdesc}

\begin{funcdesc}{reversed}{seq}
  Return a reverse iterator.  \var{seq} must be an object which
  supports the sequence protocol (the __len__() method and the
  \method{__getitem__()} method with integer arguments starting at
  \code{0}).
  \versionadded{2.4}
\end{funcdesc}

\begin{funcdesc}{round}{x\optional{, n}}
  Return the floating point value \var{x} rounded to \var{n} digits
  after the decimal point.  If \var{n} is omitted, it defaults to zero.
  The result is a floating point number.  Values are rounded to the
  closest multiple of 10 to the power minus \var{n}; if two multiples
  are equally close, rounding is done away from 0 (so. for example,
  \code{round(0.5)} is \code{1.0} and \code{round(-0.5)} is \code{-1.0}).
\end{funcdesc}

\begin{funcdesc}{set}{\optional{iterable}}
  Return a set whose elements are taken from \var{iterable}.  The elements
  must be immutable.  To represent sets of sets, the inner sets should
  be \class{frozenset} objects.  If \var{iterable} is not specified,
  returns a new empty set, \code{set([])}.
  \versionadded{2.4}  
\end{funcdesc}

\begin{funcdesc}{setattr}{object, name, value}
  This is the counterpart of \function{getattr()}.  The arguments are an
  object, a string and an arbitrary value.  The string may name an
  existing attribute or a new attribute.  The function assigns the
  value to the attribute, provided the object allows it.  For example,
  \code{setattr(\var{x}, '\var{foobar}', 123)} is equivalent to
  \code{\var{x}.\var{foobar} = 123}.
\end{funcdesc}

\begin{funcdesc}{slice}{\optional{start,} stop\optional{, step}}
  Return a slice object representing the set of indices specified by
  \code{range(\var{start}, \var{stop}, \var{step})}.  The \var{start}
  and \var{step} arguments default to \code{None}.  Slice objects have
  read-only data attributes \member{start}, \member{stop} and
  \member{step} which merely return the argument values (or their
  default).  They have no other explicit functionality; however they
  are used by Numerical Python\index{Numerical Python} and other third
  party extensions.  Slice objects are also generated when extended
  indexing syntax is used.  For example: \samp{a[start:stop:step]} or
  \samp{a[start:stop, i]}.
\end{funcdesc}

\begin{funcdesc}{sorted}{iterable\optional{, cmp\optional{,
                         key\optional{, reverse}}}}
  Return a new sorted list from the items in \var{iterable}.
  The optional arguments \var{cmp}, \var{key}, and \var{reverse}
  have the same meaning as those for the \method{list.sort()} method.
  \versionadded{2.4}    
\end{funcdesc}

\begin{funcdesc}{staticmethod}{function}
  Return a static method for \var{function}.

  A static method does not receive an implicit first argument.
  To declare a static method, use this idiom:

\begin{verbatim}
class C:
    def f(arg1, arg2, ...): ...
    f = staticmethod(f)
\end{verbatim}

  It can be called either on the class (such as \code{C.f()}) or on an
  instance (such as \code{C().f()}).  The instance is ignored except
  for its class.

  Static methods in Python are similar to those found in Java or \Cpp.
  For a more advanced concept, see \function{classmethod()} in this
  section.
  \versionadded{2.2}
\end{funcdesc}

\begin{funcdesc}{str}{\optional{object}}
  Return a string containing a nicely printable representation of an
  object.  For strings, this returns the string itself.  The
  difference with \code{repr(\var{object})} is that
  \code{str(\var{object})} does not always attempt to return a string
  that is acceptable to \function{eval()}; its goal is to return a
  printable string.  If no argument is given, returns the empty
  string, \code{''}.
\end{funcdesc}

\begin{funcdesc}{sum}{sequence\optional{, start}}
  Sums \var{start} and the items of a \var{sequence}, from left to
  right, and returns the total.  \var{start} defaults to \code{0}.
  The \var{sequence}'s items are normally numbers, and are not allowed
  to be strings.  The fast, correct way to concatenate sequence of
  strings is by calling \code{''.join(\var{sequence})}.
  Note that \code{sum(range(\var{n}), \var{m})} is equivalent to
  \code{reduce(operator.add, range(\var{n}), \var{m})}
  \versionadded{2.3}
\end{funcdesc}

\begin{funcdesc}{super}{type\optional{, object-or-type}}
  Return the superclass of \var{type}.  If the second argument is omitted
  the super object returned is unbound.  If the second argument is an
  object, \code{isinstance(\var{obj}, \var{type})} must be true.  If
  the second argument is a type, \code{issubclass(\var{type2},
  \var{type})} must be true.
  \function{super()} only works for new-style classes.

  A typical use for calling a cooperative superclass method is:
\begin{verbatim}
class C(B):
    def meth(self, arg):
        super(C, self).meth(arg)
\end{verbatim}
\versionadded{2.2}
\end{funcdesc}

\begin{funcdesc}{tuple}{\optional{sequence}}
  Return a tuple whose items are the same and in the same order as
  \var{sequence}'s items.  \var{sequence} may be a sequence, a
  container that supports iteration, or an iterator object.
  If \var{sequence} is already a tuple, it
  is returned unchanged.  For instance, \code{tuple('abc')} returns
  \code{('a', 'b', 'c')} and \code{tuple([1, 2, 3])} returns
  \code{(1, 2, 3)}.  If no argument is given, returns a new empty
  tuple, \code{()}.
\end{funcdesc}

\begin{funcdesc}{type}{object}
  Return the type of an \var{object}.  The return value is a
  type\obindex{type} object.  The standard module
  \module{types}\refstmodindex{types} defines names for all built-in
  types that don't already have built-in names.
  For instance:

\begin{verbatim}
>>> import types
>>> x = 'abc'
>>> if type(x) is str: print "It's a string"
...
It's a string
>>> def f(): pass
...
>>> if type(f) is types.FunctionType: print "It's a function"
...
It's a function
\end{verbatim}

  The \function{isinstance()} built-in function is recommended for
  testing the type of an object.
\end{funcdesc}

\begin{funcdesc}{unichr}{i}
  Return the Unicode string of one character whose Unicode code is the
  integer \var{i}.  For example, \code{unichr(97)} returns the string
  \code{u'a'}.  This is the inverse of \function{ord()} for Unicode
  strings.  The argument must be in the range [0..65535], inclusive.
  \exception{ValueError} is raised otherwise.
  \versionadded{2.0}
\end{funcdesc}

\begin{funcdesc}{unicode}{\optional{object\optional{, encoding
				    \optional{, errors}}}}
  Return the Unicode string version of \var{object} using one of the
  following modes:

  If \var{encoding} and/or \var{errors} are given, \code{unicode()}
  will decode the object which can either be an 8-bit string or a
  character buffer using the codec for \var{encoding}. The
  \var{encoding} parameter is a string giving the name of an encoding;
  if the encoding is not known, \exception{LookupError} is raised.
  Error handling is done according to \var{errors}; this specifies the
  treatment of characters which are invalid in the input encoding.  If
  \var{errors} is \code{'strict'} (the default), a
  \exception{ValueError} is raised on errors, while a value of
  \code{'ignore'} causes errors to be silently ignored, and a value of
  \code{'replace'} causes the official Unicode replacement character,
  \code{U+FFFD}, to be used to replace input characters which cannot
  be decoded.  See also the \refmodule{codecs} module.

  If no optional parameters are given, \code{unicode()} will mimic the
  behaviour of \code{str()} except that it returns Unicode strings
  instead of 8-bit strings. More precisely, if \var{object} is a
  Unicode string or subclass it will return that Unicode string without
  any additional decoding applied.

  For objects which provide a \method{__unicode__()} method, it will
  call this method without arguments to create a Unicode string. For
  all other objects, the 8-bit string version or representation is
  requested and then converted to a Unicode string using the codec for
  the default encoding in \code{'strict'} mode.

  \versionadded{2.0}
  \versionchanged[Support for \method{__unicode__()} added]{2.2}
\end{funcdesc}

\begin{funcdesc}{vars}{\optional{object}}
  Without arguments, return a dictionary corresponding to the current
  local symbol table.  With a module, class or class instance object
  as argument (or anything else that has a \member{__dict__}
  attribute), returns a dictionary corresponding to the object's
  symbol table.  The returned dictionary should not be modified: the
  effects on the corresponding symbol table are undefined.\footnote{
    In the current implementation, local variable bindings cannot
    normally be affected this way, but variables retrieved from
    other scopes (such as modules) can be.  This may change.}
\end{funcdesc}

\begin{funcdesc}{xrange}{\optional{start,} stop\optional{, step}}
  This function is very similar to \function{range()}, but returns an
  ``xrange object'' instead of a list.  This is an opaque sequence
  type which yields the same values as the corresponding list, without
  actually storing them all simultaneously.  The advantage of
  \function{xrange()} over \function{range()} is minimal (since
  \function{xrange()} still has to create the values when asked for
  them) except when a very large range is used on a memory-starved
  machine or when all of the range's elements are never used (such as
  when the loop is usually terminated with \keyword{break}).
\end{funcdesc}

\begin{funcdesc}{zip}{\optional{seq1, \moreargs}}
  This function returns a list of tuples, where the \var{i}-th tuple contains
  the \var{i}-th element from each of the argument sequences.
  The returned list is truncated in length to the length of
  the shortest argument sequence.  When there are multiple argument
  sequences which are all of the same length, \function{zip()} is
  similar to \function{map()} with an initial argument of \code{None}.
  With a single sequence argument, it returns a list of 1-tuples.
  With no arguments, it returns an empty list.
  \versionadded{2.0}

  \versionchanged[Formerly, \function{zip()} required at least one argument
  and \code{zip()} raised a \exception{TypeError} instead of returning
  an empty list.]{2.4} 
\end{funcdesc}


% ---------------------------------------------------------------------------	


\section{Non-essential Built-in Functions \label{non-essential-built-in-funcs}}

There are several built-in functions that are no longer essential to learn,
know or use in modern Python programming.  They have been kept here to
maintain backwards compatability with programs written for older versions
of Python.

Python programmers, trainers, students and bookwriters should feel free to
bypass these functions without concerns about missing something important.


\setindexsubitem{(non-essential built-in functions)}

\begin{funcdesc}{apply}{function, args\optional{, keywords}}
  The \var{function} argument must be a callable object (a
  user-defined or built-in function or method, or a class object) and
  the \var{args} argument must be a sequence.  The \var{function} is
  called with \var{args} as the argument list; the number of arguments
  is the length of the tuple.
  If the optional \var{keywords} argument is present, it must be a
  dictionary whose keys are strings.  It specifies keyword arguments
  to be added to the end of the argument list.
  Calling \function{apply()} is different from just calling
  \code{\var{function}(\var{args})}, since in that case there is always
  exactly one argument.  The use of \function{apply()} is equivalent
  to \code{\var{function}(*\var{args}, **\var{keywords})}.
  Use of \function{apply()} is not necessary since the ``extended call
  syntax,'' as used in the last example, is completely equivalent.

  \deprecated{2.3}{Use the extended call syntax instead, as described
                   above.}
\end{funcdesc}

\begin{funcdesc}{buffer}{object\optional{, offset\optional{, size}}}
  The \var{object} argument must be an object that supports the buffer
  call interface (such as strings, arrays, and buffers).  A new buffer
  object will be created which references the \var{object} argument.
  The buffer object will be a slice from the beginning of \var{object}
  (or from the specified \var{offset}). The slice will extend to the
  end of \var{object} (or will have a length given by the \var{size}
  argument).
\end{funcdesc}

\begin{funcdesc}{coerce}{x, y}
  Return a tuple consisting of the two numeric arguments converted to
  a common type, using the same rules as used by arithmetic
  operations.
\end{funcdesc}

\begin{funcdesc}{intern}{string}
  Enter \var{string} in the table of ``interned'' strings and return
  the interned string -- which is \var{string} itself or a copy.
  Interning strings is useful to gain a little performance on
  dictionary lookup -- if the keys in a dictionary are interned, and
  the lookup key is interned, the key comparisons (after hashing) can
  be done by a pointer compare instead of a string compare.  Normally,
  the names used in Python programs are automatically interned, and
  the dictionaries used to hold module, class or instance attributes
  have interned keys.  \versionchanged[Interned strings are not
  immortal (like they used to be in Python 2.2 and before);
  you must keep a reference to the return value of \function{intern()}
  around to benefit from it]{2.3}
\end{funcdesc}


\chapter{Python Services}

The modules described in this chapter provide a wide range of services
related to the Python interpreter and its interaction with its
environment.  Here's an overview:

\begin{description}

\item[sys]
--- Access system specific parameters and functions.

\item[types]
--- Names for all built-in types.

\item[traceback]
--- Print or retrieve a stack traceback.

\item[pickle]
--- Convert Python objects to streams of bytes and back.

\item[shelve]
--- Python object persistency.

\item[copy]
--- Shallow and deep copy operations.

\item[marshal]
--- Convert Python objects to streams of bytes and back (with
different constraints).

\item[imp]
--- Access the implementation of the \code{import} statement.

\item[parser]
--- Retrieve and submit parse trees from and to the runtime support
environment.

\item[__builtin__]
--- The set of built-in functions.

\item[__main__]
--- The environment where the top-level script is run.

\end{description}
		% Python Services
\section{\module{sys} ---
         System-specific parameters and functions}

\declaremodule{builtin}{sys}
\modulesynopsis{Access system-specific parameters and functions.}

This module provides access to some variables used or maintained by the
interpreter and to functions that interact strongly with the interpreter.
It is always available.


\begin{datadesc}{argv}
  The list of command line arguments passed to a Python script.
  \code{argv[0]} is the script name (it is operating system dependent
  whether this is a full pathname or not).  If the command was
  executed using the \programopt{-c} command line option to the
  interpreter, \code{argv[0]} is set to the string \code{'-c'}.  If no
  script name was passed to the Python interpreter, \code{argv} has
  zero length.
\end{datadesc}

\begin{datadesc}{byteorder}
  An indicator of the native byte order.  This will have the value
  \code{'big'} on big-endian (most-signigicant byte first) platforms,
  and \code{'little'} on little-endian (least-significant byte first)
  platforms.
  \versionadded{2.0}
\end{datadesc}

\begin{datadesc}{builtin_module_names}
  A tuple of strings giving the names of all modules that are compiled
  into this Python interpreter.  (This information is not available in
  any other way --- \code{modules.keys()} only lists the imported
  modules.)
\end{datadesc}

\begin{datadesc}{copyright}
  A string containing the copyright pertaining to the Python
  interpreter.
\end{datadesc}

\begin{datadesc}{dllhandle}
  Integer specifying the handle of the Python DLL.
  Availability: Windows.
\end{datadesc}

\begin{funcdesc}{displayhook}{\var{value}}
  If \var{value} is not \code{None}, this function prints it to
  \code{sys.stdout}, and saves it in \code{__builtin__._}.

  \code{sys.displayhook} is called on the result of evaluating an
  expression entered in an interactive Python session.  The display of
  these values can be customized by assigning another one-argument
  function to \code{sys.displayhook}.
\end{funcdesc}

\begin{funcdesc}{excepthook}{\var{type}, \var{value}, \var{traceback}}
  This function prints out a given traceback and exception to
  \code{sys.stderr}.

  When an exception is raised and uncaught, the interpreter calls
  \code{sys.excepthook} with three arguments, the exception class,
  exception instance, and a traceback object.  In an interactive
  session this happens just before control is returned to the prompt;
  in a Python program this happens just before the program exits.  The
  handling of such top-level exceptions can be customized by assigning
  another three-argument function to \code{sys.excepthook}.
\end{funcdesc}

\begin{datadesc}{__displayhook__}
\dataline{__excepthook__}
  These objects contain the original values of \code{displayhook} and
  \code{excepthook} at the start of the program.  They are saved so
  that \code{displayhook} and \code{excepthook} can be restored in
  case they happen to get replaced with broken objects.
\end{datadesc}

\begin{funcdesc}{exc_info}{}
  This function returns a tuple of three values that give information
  about the exception that is currently being handled.  The
  information returned is specific both to the current thread and to
  the current stack frame.  If the current stack frame is not handling
  an exception, the information is taken from the calling stack frame,
  or its caller, and so on until a stack frame is found that is
  handling an exception.  Here, ``handling an exception'' is defined
  as ``executing or having executed an except clause.''  For any stack
  frame, only information about the most recently handled exception is
  accessible.

  If no exception is being handled anywhere on the stack, a tuple
  containing three \code{None} values is returned.  Otherwise, the
  values returned are \code{(\var{type}, \var{value},
  \var{traceback})}.  Their meaning is: \var{type} gets the exception
  type of the exception being handled (a class object);
  \var{value} gets the exception parameter (its \dfn{associated value}
  or the second argument to \keyword{raise}, which is always a class
  instance if the exception type is a class object); \var{traceback}
  gets a traceback object (see the Reference Manual) which
  encapsulates the call stack at the point where the exception
  originally occurred.  \obindex{traceback}

  If \function{exc_clear()} is called, this function will return three
  \code{None} values until either another exception is raised in the
  current thread or the execution stack returns to a frame where
  another exception is being handled.

  \warning{Assigning the \var{traceback} return value to a
  local variable in a function that is handling an exception will
  cause a circular reference.  This will prevent anything referenced
  by a local variable in the same function or by the traceback from
  being garbage collected.  Since most functions don't need access to
  the traceback, the best solution is to use something like
  \code{exctype, value = sys.exc_info()[:2]} to extract only the
  exception type and value.  If you do need the traceback, make sure
  to delete it after use (best done with a \keyword{try}
  ... \keyword{finally} statement) or to call \function{exc_info()} in
  a function that does not itself handle an exception.} \note{Beginning
  with Python 2.2, such cycles are automatically reclaimed when garbage
  collection is enabled and they become unreachable, but it remains more
  efficient to avoid creating cycles.}
\end{funcdesc}

\begin{funcdesc}{exc_clear}{}
  This function clears all information relating to the current or last
  exception that occured in the current thread.  After calling this
  function, \function{exc_info()} will return three \code{None} values until
  another exception is raised in the current thread or the execution stack
  returns to a frame where another exception is being handled.
  
  This function is only needed in only a few obscure situations.  These
  include logging and error handling systems that report information on the
  last or current exception.  This function can also be used to try to free
  resources and trigger object finalization, though no guarantee is made as
  to what objects will be freed, if any.
\versionadded{2.3}
\end{funcdesc}

\begin{datadesc}{exc_type}
\dataline{exc_value}
\dataline{exc_traceback}
\deprecated {1.5}
            {Use \function{exc_info()} instead.}
  Since they are global variables, they are not specific to the
  current thread, so their use is not safe in a multi-threaded
  program.  When no exception is being handled, \code{exc_type} is set
  to \code{None} and the other two are undefined.
\end{datadesc}

\begin{datadesc}{exec_prefix}
  A string giving the site-specific directory prefix where the
  platform-dependent Python files are installed; by default, this is
  also \code{'/usr/local'}.  This can be set at build time with the
  \longprogramopt{exec-prefix} argument to the \program{configure}
  script.  Specifically, all configuration files (e.g. the
  \file{pyconfig.h} header file) are installed in the directory
  \code{exec_prefix + '/lib/python\var{version}/config'}, and shared
  library modules are installed in \code{exec_prefix +
  '/lib/python\var{version}/lib-dynload'}, where \var{version} is
  equal to \code{version[:3]}.
\end{datadesc}

\begin{datadesc}{executable}
  A string giving the name of the executable binary for the Python
  interpreter, on systems where this makes sense.
\end{datadesc}

\begin{funcdesc}{exit}{\optional{arg}}
  Exit from Python.  This is implemented by raising the
  \exception{SystemExit} exception, so cleanup actions specified by
  finally clauses of \keyword{try} statements are honored, and it is
  possible to intercept the exit attempt at an outer level.  The
  optional argument \var{arg} can be an integer giving the exit status
  (defaulting to zero), or another type of object.  If it is an
  integer, zero is considered ``successful termination'' and any
  nonzero value is considered ``abnormal termination'' by shells and
  the like.  Most systems require it to be in the range 0-127, and
  produce undefined results otherwise.  Some systems have a convention
  for assigning specific meanings to specific exit codes, but these
  are generally underdeveloped; \UNIX{} programs generally use 2 for
  command line syntax errors and 1 for all other kind of errors.  If
  another type of object is passed, \code{None} is equivalent to
  passing zero, and any other object is printed to \code{sys.stderr}
  and results in an exit code of 1.  In particular,
  \code{sys.exit("some error message")} is a quick way to exit a
  program when an error occurs.
\end{funcdesc}

\begin{datadesc}{exitfunc}
  This value is not actually defined by the module, but can be set by
  the user (or by a program) to specify a clean-up action at program
  exit.  When set, it should be a parameterless function.  This
  function will be called when the interpreter exits.  Only one
  function may be installed in this way; to allow multiple functions
  which will be called at termination, use the \refmodule{atexit}
  module.  \note{The exit function is not called when the program is
  killed by a signal, when a Python fatal internal error is detected,
  or when \code{os._exit()} is called.}
  \deprecated{2.4}{Use \refmodule{atexit} instead.}
\end{datadesc}

\begin{funcdesc}{getcheckinterval}{}
  Return the interpreter's ``check interval'';
  see \function{setcheckinterval()}.
  \versionadded{2.3}
\end{funcdesc}

\begin{funcdesc}{getdefaultencoding}{}
  Return the name of the current default string encoding used by the
  Unicode implementation.
  \versionadded{2.0}
\end{funcdesc}

\begin{funcdesc}{getdlopenflags}{}
  Return the current value of the flags that are used for
  \cfunction{dlopen()} calls. The flag constants are defined in the
  \refmodule{dl} and \module{DLFCN} modules.
  Availability: \UNIX.
  \versionadded{2.2}
\end{funcdesc}

\begin{funcdesc}{getfilesystemencoding}{}
  Return the name of the encoding used to convert Unicode filenames
  into system file names, or \code{None} if the system default encoding
  is used. The result value depends on the operating system:
\begin{itemize}
\item On Windows 9x, the encoding is ``mbcs''.
\item On Mac OS X, the encoding is ``utf-8''.
\item On Unix, the encoding is the user's preference 
      according to the result of nl_langinfo(CODESET), or None if
      the nl_langinfo(CODESET) failed.
\item On Windows NT+, file names are Unicode natively, so no conversion
      is performed. \code{getfilesystemencoding} still returns ``mbcs'',
      as this is the encoding that applications should use when they
      explicitly want to convert Unicode strings to byte strings that
      are equivalent when used as file names.
\end{itemize}
  \versionadded{2.3}
\end{funcdesc}

\begin{funcdesc}{getrefcount}{object}
  Return the reference count of the \var{object}.  The count returned
  is generally one higher than you might expect, because it includes
  the (temporary) reference as an argument to
  \function{getrefcount()}.
\end{funcdesc}

\begin{funcdesc}{getrecursionlimit}{}
  Return the current value of the recursion limit, the maximum depth
  of the Python interpreter stack.  This limit prevents infinite
  recursion from causing an overflow of the C stack and crashing
  Python.  It can be set by \function{setrecursionlimit()}.
\end{funcdesc}

\begin{funcdesc}{_getframe}{\optional{depth}}
  Return a frame object from the call stack.  If optional integer
  \var{depth} is given, return the frame object that many calls below
  the top of the stack.  If that is deeper than the call stack,
  \exception{ValueError} is raised.  The default for \var{depth} is
  zero, returning the frame at the top of the call stack.

  This function should be used for internal and specialized purposes
  only.
\end{funcdesc}

\begin{funcdesc}{getwindowsversion}{}
  Return a tuple containing five components, describing the Windows 
  version currently running.  The elements are \var{major}, \var{minor}, 
  \var{build}, \var{platform}, and \var{text}.  \var{text} contains
  a string while all other values are integers.

  \var{platform} may be one of the following values:
  \begin{list}{}{\leftmargin 0.7in \labelwidth 0.65in}
    \item[0 (\constant{VER_PLATFORM_WIN32s})]
      Win32s on Windows 3.1.
    \item[1 (\constant{VER_PLATFORM_WIN32_WINDOWS})] 
      Windows 95/98/ME
    \item[2 (\constant{VER_PLATFORM_WIN32_NT})] 
      Windows NT/2000/XP
    \item[3 (\constant{VER_PLATFORM_WIN32_CE})] 
      Windows CE.
  \end{list}
  
  This function wraps the Win32 \function{GetVersionEx()} function;
  see the Microsoft Documentation for more information about these
  fields.

  Availability: Windows.
  \versionadded{2.3}
\end{funcdesc}

\begin{datadesc}{hexversion}
  The version number encoded as a single integer.  This is guaranteed
  to increase with each version, including proper support for
  non-production releases.  For example, to test that the Python
  interpreter is at least version 1.5.2, use:

\begin{verbatim}
if sys.hexversion >= 0x010502F0:
    # use some advanced feature
    ...
else:
    # use an alternative implementation or warn the user
    ...
\end{verbatim}

  This is called \samp{hexversion} since it only really looks
  meaningful when viewed as the result of passing it to the built-in
  \function{hex()} function.  The \code{version_info} value may be
  used for a more human-friendly encoding of the same information.
  \versionadded{1.5.2}
\end{datadesc}

\begin{datadesc}{last_type}
\dataline{last_value}
\dataline{last_traceback}
  These three variables are not always defined; they are set when an
  exception is not handled and the interpreter prints an error message
  and a stack traceback.  Their intended use is to allow an
  interactive user to import a debugger module and engage in
  post-mortem debugging without having to re-execute the command that
  caused the error.  (Typical use is \samp{import pdb; pdb.pm()} to
  enter the post-mortem debugger; see chapter \ref{debugger}, ``The
  Python Debugger,'' for more information.)

  The meaning of the variables is the same as that of the return
  values from \function{exc_info()} above.  (Since there is only one
  interactive thread, thread-safety is not a concern for these
  variables, unlike for \code{exc_type} etc.)
\end{datadesc}

\begin{datadesc}{maxint}
  The largest positive integer supported by Python's regular integer
  type.  This is at least 2**31-1.  The largest negative integer is
  \code{-maxint-1} --- the asymmetry results from the use of 2's
  complement binary arithmetic.
\end{datadesc}

\begin{datadesc}{maxunicode}
  An integer giving the largest supported code point for a Unicode
  character.  The value of this depends on the configuration option
  that specifies whether Unicode characters are stored as UCS-2 or
  UCS-4.
\end{datadesc}

\begin{datadesc}{modules}
  This is a dictionary that maps module names to modules which have
  already been loaded.  This can be manipulated to force reloading of
  modules and other tricks.  Note that removing a module from this
  dictionary is \emph{not} the same as calling
  \function{reload()}\bifuncindex{reload} on the corresponding module
  object.
\end{datadesc}

\begin{datadesc}{path}
\indexiii{module}{search}{path}
  A list of strings that specifies the search path for modules.
  Initialized from the environment variable \envvar{PYTHONPATH}, plus an
  installation-dependent default.

  As initialized upon program startup,
  the first item of this list, \code{path[0]}, is the directory
  containing the script that was used to invoke the Python
  interpreter.  If the script directory is not available (e.g.  if the
  interpreter is invoked interactively or if the script is read from
  standard input), \code{path[0]} is the empty string, which directs
  Python to search modules in the current directory first.  Notice
  that the script directory is inserted \emph{before} the entries
  inserted as a result of \envvar{PYTHONPATH}.

  A program is free to modify this list for its own purposes.

  \versionchanged[Unicode strings are no longer ignored]{2.3}
\end{datadesc}

\begin{datadesc}{platform}
  This string contains a platform identifier, e.g. \code{'sunos5'} or
  \code{'linux1'}.  This can be used to append platform-specific
  components to \code{path}, for instance.
\end{datadesc}

\begin{datadesc}{prefix}
  A string giving the site-specific directory prefix where the
  platform independent Python files are installed; by default, this is
  the string \code{'/usr/local'}.  This can be set at build time with
  the \longprogramopt{prefix} argument to the \program{configure}
  script.  The main collection of Python library modules is installed
  in the directory \code{prefix + '/lib/python\var{version}'} while
  the platform independent header files (all except \file{pyconfig.h})
  are stored in \code{prefix + '/include/python\var{version}'}, where
  \var{version} is equal to \code{version[:3]}.
\end{datadesc}

\begin{datadesc}{ps1}
\dataline{ps2}
\index{interpreter prompts}
\index{prompts, interpreter}
  Strings specifying the primary and secondary prompt of the
  interpreter.  These are only defined if the interpreter is in
  interactive mode.  Their initial values in this case are
  \code{'>\code{>}> '} and \code{'... '}.  If a non-string object is
  assigned to either variable, its \function{str()} is re-evaluated
  each time the interpreter prepares to read a new interactive
  command; this can be used to implement a dynamic prompt.
\end{datadesc}

\begin{funcdesc}{setcheckinterval}{interval}
  Set the interpreter's ``check interval''.  This integer value
  determines how often the interpreter checks for periodic things such
  as thread switches and signal handlers.  The default is \code{100},
  meaning the check is performed every 100 Python virtual instructions.
  Setting it to a larger value may increase performance for programs
  using threads.  Setting it to a value \code{<=} 0 checks every
  virtual instruction, maximizing responsiveness as well as overhead.
\end{funcdesc}

\begin{funcdesc}{setdefaultencoding}{name}
  Set the current default string encoding used by the Unicode
  implementation.  If \var{name} does not match any available
  encoding, \exception{LookupError} is raised.  This function is only
  intended to be used by the \refmodule{site} module implementation
  and, where needed, by \module{sitecustomize}.  Once used by the
  \refmodule{site} module, it is removed from the \module{sys}
  module's namespace.
%  Note that \refmodule{site} is not imported if
%  the \programopt{-S} option is passed to the interpreter, in which
%  case this function will remain available.
  \versionadded{2.0}
\end{funcdesc}

\begin{funcdesc}{setdlopenflags}{n}
  Set the flags used by the interpreter for \cfunction{dlopen()}
  calls, such as when the interpreter loads extension modules.  Among
  other things, this will enable a lazy resolving of symbols when
  importing a module, if called as \code{sys.setdlopenflags(0)}.  To
  share symbols across extension modules, call as
  \code{sys.setdlopenflags(dl.RTLD_NOW | dl.RTLD_GLOBAL)}.  Symbolic
  names for the flag modules can be either found in the \refmodule{dl}
  module, or in the \module{DLFCN} module. If \module{DLFCN} is not
  available, it can be generated from \file{/usr/include/dlfcn.h}
  using the \program{h2py} script.
  Availability: \UNIX.
  \versionadded{2.2}
\end{funcdesc}

\begin{funcdesc}{setprofile}{profilefunc}
  Set the system's profile function,\index{profile function} which
  allows you to implement a Python source code profiler in
  Python.\index{profiler}  See chapter \ref{profile} for more
  information on the Python profiler.  The system's profile function
  is called similarly to the system's trace function (see
  \function{settrace()}), but it isn't called for each executed line
  of code (only on call and return, but the return event is reported
  even when an exception has been set).  The function is
  thread-specific, but there is no way for the profiler to know about
  context switches between threads, so it does not make sense to use
  this in the presence of multiple threads.
  Also, its return value is not used, so it can simply return
  \code{None}.
\end{funcdesc}

\begin{funcdesc}{setrecursionlimit}{limit}
  Set the maximum depth of the Python interpreter stack to
  \var{limit}.  This limit prevents infinite recursion from causing an
  overflow of the C stack and crashing Python.

  The highest possible limit is platform-dependent.  A user may need
  to set the limit higher when she has a program that requires deep
  recursion and a platform that supports a higher limit.  This should
  be done with care, because a too-high limit can lead to a crash.
\end{funcdesc}

\begin{funcdesc}{settrace}{tracefunc}
  Set the system's trace function,\index{trace function} which allows
  you to implement a Python source code debugger in Python.  See
  section \ref{debugger-hooks}, ``How It Works,'' in the chapter on
  the Python debugger.\index{debugger}  The function is
  thread-specific; for a debugger to support multiple threads, it must
  be registered using \function{settrace()} for each thread being
  debugged.  \note{The \function{settrace()} function is intended only
  for implementing debuggers, profilers, coverage tools and the like.
  Its behavior is part of the implementation platform, rather than 
  part of the language definition, and thus may not be available in
  all Python implementations.}
\end{funcdesc}

\begin{funcdesc}{settscdump}{on_flag}
  Activate dumping of VM measurements using the Pentium timestamp
  counter, if \var{on_flag} is true. Deactivate these dumps if
  \var{on_flag} is off. The function is available only if Python
  was compiled with \longprogramopt{with-tsc}. To understand the
  output of this dump, read \file{Python/ceval.c} in the Python
  sources.
  \versionadded{2.4}
\end{funcdesc}

\begin{datadesc}{stdin}
\dataline{stdout}
\dataline{stderr}
  File objects corresponding to the interpreter's standard input,
  output and error streams.  \code{stdin} is used for all interpreter
  input except for scripts but including calls to
  \function{input()}\bifuncindex{input} and
  \function{raw_input()}\bifuncindex{raw_input}.  \code{stdout} is
  used for the output of \keyword{print} and expression statements and
  for the prompts of \function{input()} and \function{raw_input()}.
  The interpreter's own prompts and (almost all of) its error messages
  go to \code{stderr}.  \code{stdout} and \code{stderr} needn't be
  built-in file objects: any object is acceptable as long as it has a
  \method{write()} method that takes a string argument.  (Changing
  these objects doesn't affect the standard I/O streams of processes
  executed by \function{os.popen()}, \function{os.system()} or the
  \function{exec*()} family of functions in the \refmodule{os}
  module.)
\end{datadesc}

\begin{datadesc}{__stdin__}
\dataline{__stdout__}
\dataline{__stderr__}
  These objects contain the original values of \code{stdin},
  \code{stderr} and \code{stdout} at the start of the program.  They
  are used during finalization, and could be useful to restore the
  actual files to known working file objects in case they have been
  overwritten with a broken object.
\end{datadesc}

\begin{datadesc}{tracebacklimit}
  When this variable is set to an integer value, it determines the
  maximum number of levels of traceback information printed when an
  unhandled exception occurs.  The default is \code{1000}.  When set
  to \code{0} or less, all traceback information is suppressed and
  only the exception type and value are printed.
\end{datadesc}

\begin{datadesc}{version}
  A string containing the version number of the Python interpreter
  plus additional information on the build number and compiler used.
  It has a value of the form \code{'\var{version}
  (\#\var{build_number}, \var{build_date}, \var{build_time})
  [\var{compiler}]'}.  The first three characters are used to identify
  the version in the installation directories (where appropriate on
  each platform).  An example:

\begin{verbatim}
>>> import sys
>>> sys.version
'1.5.2 (#0 Apr 13 1999, 10:51:12) [MSC 32 bit (Intel)]'
\end{verbatim}
\end{datadesc}

\begin{datadesc}{api_version}
  The C API version for this interpreter.  Programmers may find this useful
  when debugging version conflicts between Python and extension
  modules. \versionadded{2.3}
\end{datadesc}

\begin{datadesc}{version_info}
  A tuple containing the five components of the version number:
  \var{major}, \var{minor}, \var{micro}, \var{releaselevel}, and
  \var{serial}.  All values except \var{releaselevel} are integers;
  the release level is \code{'alpha'}, \code{'beta'},
  \code{'candidate'}, or \code{'final'}.  The \code{version_info}
  value corresponding to the Python version 2.0 is \code{(2, 0, 0,
  'final', 0)}.
  \versionadded{2.0}
\end{datadesc}

\begin{datadesc}{warnoptions}
  This is an implementation detail of the warnings framework; do not
  modify this value.  Refer to the \refmodule{warnings} module for
  more information on the warnings framework.
\end{datadesc}

\begin{datadesc}{winver}
  The version number used to form registry keys on Windows platforms.
  This is stored as string resource 1000 in the Python DLL.  The value
  is normally the first three characters of \constant{version}.  It is
  provided in the \module{sys} module for informational purposes;
  modifying this value has no effect on the registry keys used by
  Python.
  Availability: Windows.
\end{datadesc}


\begin{seealso}
  \seemodule{site}
    {This describes how to use .pth files to extend \code{sys.path}.}
\end{seealso}

\section{Built-in Types}
\label{types}

The following sections describe the standard types that are built into
the interpreter.  These are the numeric types, sequence types, and
several others, including types themselves.  There is no explicit
Boolean type; use integers instead.
\indexii{built-in}{types}
\indexii{Boolean}{type}

Some operations are supported by several object types; in particular,
all objects can be compared, tested for truth value, and converted to
a string (with the \code{`{\rm \ldots}`} notation).  The latter conversion is
implicitly used when an object is written by the \code{print} statement.
\stindex{print}


\subsection{Truth Value Testing}
\label{truth}

Any object can be tested for truth value, for use in an \code{if} or
\code{while} condition or as operand of the Boolean operations below.
The following values are considered false:
\stindex{if}
\stindex{while}
\indexii{truth}{value}
\indexii{Boolean}{operations}
\index{false}

\setindexsubitem{(Built-in object)}
\begin{itemize}

\item	\code{None}
	\ttindex{None}

\item	zero of any numeric type, e.g., \code{0}, \code{0L}, \code{0.0}.

\item	any empty sequence, e.g., \code{''}, \code{()}, \code{[]}.

\item	any empty mapping, e.g., \code{\{\}}.

\item	instances of user-defined classes, if the class defines a
	\code{__nonzero__()} or \code{__len__()} method, when that
	method returns zero.

\end{itemize}

All other values are considered true --- so objects of many types are
always true.
\index{true}

Operations and built-in functions that have a Boolean result always
return \code{0} for false and \code{1} for true, unless otherwise
stated.  (Important exception: the Boolean operations
\samp{or}\opindex{or} and \samp{and}\opindex{and} always return one of
their operands.)


\subsection{Boolean Operations}
\label{boolean}

These are the Boolean operations, ordered by ascending priority:
\indexii{Boolean}{operations}

\begin{tableiii}{|c|l|c|}{code}{Operation}{Result}{Notes}
  \lineiii{\var{x} or \var{y}}{if \var{x} is false, then \var{y}, else \var{x}}{(1)}
  \hline
  \lineiii{\var{x} and \var{y}}{if \var{x} is false, then \var{x}, else \var{y}}{(1)}
  \hline
  \lineiii{not \var{x}}{if \var{x} is false, then \code{1}, else \code{0}}{(2)}
\end{tableiii}
\opindex{and}
\opindex{or}
\opindex{not}

\noindent
Notes:

\begin{description}

\item[(1)]
These only evaluate their second argument if needed for their outcome.

\item[(2)]
\samp{not} has a lower priority than non-Boolean operators, so e.g.
\code{not a == b} is interpreted as \code{not(a == b)}, and
\code{a == not b} is a syntax error.

\end{description}


\subsection{Comparisons}
\label{comparisons}

Comparison operations are supported by all objects.  They all have the
same priority (which is higher than that of the Boolean operations).
Comparisons can be chained arbitrarily, e.g. \code{x < y <= z} is
equivalent to \code{x < y and y <= z}, except that \code{y} is
evaluated only once (but in both cases \code{z} is not evaluated at
all when \code{x < y} is found to be false).
\indexii{chaining}{comparisons}

This table summarizes the comparison operations:

\begin{tableiii}{|c|l|c|}{code}{Operation}{Meaning}{Notes}
  \lineiii{<}{strictly less than}{}
  \lineiii{<=}{less than or equal}{}
  \lineiii{>}{strictly greater than}{}
  \lineiii{>=}{greater than or equal}{}
  \lineiii{==}{equal}{}
  \lineiii{<>}{not equal}{(1)}
  \lineiii{!=}{not equal}{(1)}
  \lineiii{is}{object identity}{}
  \lineiii{is not}{negated object identity}{}
\end{tableiii}
\indexii{operator}{comparison}
\opindex{==} % XXX *All* others have funny characters < ! >
\opindex{is}
\opindex{is not}

\noindent
Notes:

\begin{description}

\item[(1)]
\code{<>} and \code{!=} are alternate spellings for the same operator.
(I couldn't choose between \ABC{} and \C{}! :-)
\indexii{ABC@\ABC{}}{language}
\indexii{\C{}}{language}

\end{description}

Objects of different types, except different numeric types, never
compare equal; such objects are ordered consistently but arbitrarily
(so that sorting a heterogeneous array yields a consistent result).
Furthermore, some types (e.g., windows) support only a degenerate
notion of comparison where any two objects of that type are unequal.
Again, such objects are ordered arbitrarily but consistently.
\indexii{types}{numeric}
\indexii{objects}{comparing}

(Implementation note: objects of different types except numbers are
ordered by their type names; objects of the same types that don't
support proper comparison are ordered by their address.)

Two more operations with the same syntactic priority, \code{in} and
\code{not in}, are supported only by sequence types (below).
\opindex{in}
\opindex{not in}


\subsection{Numeric Types}
\label{typesnumeric}

There are four numeric types: \dfn{plain integers}, \dfn{long integers}, 
\dfn{floating point numbers}, and \dfn{complex numbers}.
Plain integers (also just called \dfn{integers})
are implemented using \code{long} in \C{}, which gives them at least 32
bits of precision.  Long integers have unlimited precision.  Floating
point numbers are implemented using \code{double} in \C{}.  All bets on
their precision are off unless you happen to know the machine you are
working with.
\indexii{numeric}{types}
\indexii{integer}{types}
\indexii{integer}{type}
\indexiii{long}{integer}{type}
\indexii{floating point}{type}
\indexii{complex number}{type}
\indexii{\C{}}{language}

Complex numbers have a real and imaginary part, which are both
implemented using \code{double} in \C{}.  To extract these parts from
a complex number \code{z}, use \code{z.real} and \code{z.imag}.  

Numbers are created by numeric literals or as the result of built-in
functions and operators.  Unadorned integer literals (including hex
and octal numbers) yield plain integers.  Integer literals with an \samp{L}
or \samp{l} suffix yield long integers
(\samp{L} is preferred because \samp{1l} looks too much like eleven!).
Numeric literals containing a decimal point or an exponent sign yield
floating point numbers.  Appending \samp{j} or \samp{J} to a numeric
literal yields a complex number.
\indexii{numeric}{literals}
\indexii{integer}{literals}
\indexiii{long}{integer}{literals}
\indexii{floating point}{literals}
\indexii{complex number}{literals}
\indexii{hexadecimal}{literals}
\indexii{octal}{literals}

Python fully supports mixed arithmetic: when a binary arithmetic
operator has operands of different numeric types, the operand with the
``smaller'' type is converted to that of the other, where plain
integer is smaller than long integer is smaller than floating point is
smaller than complex.
Comparisons between numbers of mixed type use the same rule.%
\footnote{As a consequence, the list \code{[1, 2]} is considered equal
	to \code{[1.0, 2.0]}, and similar for tuples.}
The functions \code{int()}, \code{long()}, \code{float()},
and \code{complex()} can be used
to coerce numbers to a specific type.
\index{arithmetic}
\bifuncindex{int}
\bifuncindex{long}
\bifuncindex{float}
\bifuncindex{complex}

All numeric types support the following operations, sorted by
ascending priority (operations in the same box have the same
priority; all numeric operations have a higher priority than
comparison operations):

\begin{tableiii}{|c|l|c|}{code}{Operation}{Result}{Notes}
  \lineiii{\var{x} + \var{y}}{sum of \var{x} and \var{y}}{}
  \lineiii{\var{x} - \var{y}}{difference of \var{x} and \var{y}}{}
  \hline
  \lineiii{\var{x} * \var{y}}{product of \var{x} and \var{y}}{}
  \lineiii{\var{x} / \var{y}}{quotient of \var{x} and \var{y}}{(1)}
  \lineiii{\var{x} \%{} \var{y}}{remainder of \code{\var{x} / \var{y}}}{}
  \hline
  \lineiii{-\var{x}}{\var{x} negated}{}
  \lineiii{+\var{x}}{\var{x} unchanged}{}
  \hline
  \lineiii{abs(\var{x})}{absolute value or magnitude of \var{x}}{}
  \lineiii{int(\var{x})}{\var{x} converted to integer}{(2)}
  \lineiii{long(\var{x})}{\var{x} converted to long integer}{(2)}
  \lineiii{float(\var{x})}{\var{x} converted to floating point}{}
  \lineiii{complex(\var{re},\var{im})}{a complex number with real part \var{re}, imaginary part \var{im}.  \var{im} defaults to zero.}{}
  \lineiii{divmod(\var{x}, \var{y})}{the pair \code{(\var{x} / \var{y}, \var{x} \%{} \var{y})}}{(3)}
  \lineiii{pow(\var{x}, \var{y})}{\var{x} to the power \var{y}}{}
  \lineiii{\var{x}**\var{y}}{\var{x} to the power \var{y}}{}
\end{tableiii}
\indexiii{operations on}{numeric}{types}

\noindent
Notes:
\begin{description}

\item[(1)]
For (plain or long) integer division, the result is an integer.
The result is always rounded towards minus infinity: 1/2 is 0, 
(-1)/2 is -1, 1/(-2) is -1, and (-1)/(-2) is 0.
\indexii{integer}{division}
\indexiii{long}{integer}{division}

\item[(2)]
Conversion from floating point to (long or plain) integer may round or
truncate as in \C{}; see functions \code{floor()} and \code{ceil()} in
module \code{math} for well-defined conversions.
\bifuncindex{floor}
\bifuncindex{ceil}
\indexii{numeric}{conversions}
\refbimodindex{math}
\indexii{\C{}}{language}

\item[(3)]
See the section on built-in functions for an exact definition.

\end{description}
% XXXJH exceptions: overflow (when? what operations?) zerodivision

\subsubsection{Bit-string Operations on Integer Types}
\nodename{Bit-string Operations}

Plain and long integer types support additional operations that make
sense only for bit-strings.  Negative numbers are treated as their 2's
complement value (for long integers, this assumes a sufficiently large
number of bits that no overflow occurs during the operation).

The priorities of the binary bit-wise operations are all lower than
the numeric operations and higher than the comparisons; the unary
operation \samp{\~} has the same priority as the other unary numeric
operations (\samp{+} and \samp{-}).

This table lists the bit-string operations sorted in ascending
priority (operations in the same box have the same priority):

\begin{tableiii}{|c|l|c|}{code}{Operation}{Result}{Notes}
  \lineiii{\var{x} | \var{y}}{bitwise \dfn{or} of \var{x} and \var{y}}{}
  \hline
  \lineiii{\var{x} \^{} \var{y}}{bitwise \dfn{exclusive or} of \var{x} and \var{y}}{}
  \hline
  \lineiii{\var{x} \&{} \var{y}}{bitwise \dfn{and} of \var{x} and \var{y}}{}
  \hline
  \lineiii{\var{x} << \var{n}}{\var{x} shifted left by \var{n} bits}{(1), (2)}
  \lineiii{\var{x} >> \var{n}}{\var{x} shifted right by \var{n} bits}{(1), (3)}
  \hline
  \hline
  \lineiii{\~\var{x}}{the bits of \var{x} inverted}{}
\end{tableiii}
\indexiii{operations on}{integer}{types}
\indexii{bit-string}{operations}
\indexii{shifting}{operations}
\indexii{masking}{operations}

\noindent
Notes:
\begin{description}
\item[(1)] Negative shift counts are illegal and cause a
\exception{ValueError} to be raised.
\item[(2)] A left shift by \var{n} bits is equivalent to
multiplication by \code{pow(2, \var{n})} without overflow check.
\item[(3)] A right shift by \var{n} bits is equivalent to
division by \code{pow(2, \var{n})} without overflow check.
\end{description}


\subsection{Sequence Types}
\label{typesseq}

There are three sequence types: strings, lists and tuples.

Strings literals are written in single or double quotes:
\code{'xyzzy'}, \code{"frobozz"}.  See Chapter 2 of the \emph{Python
Reference Manual} for more about string literals.  Lists are
constructed with square brackets, separating items with commas:
\code{[a, b, c]}.  Tuples are constructed by the comma operator (not
within square brackets), with or without enclosing parentheses, but an
empty tuple must have the enclosing parentheses, e.g.,
\code{a, b, c} or \code{()}.  A single item tuple must have a trailing
comma, e.g., \code{(d,)}.
\indexii{sequence}{types}
\indexii{string}{type}
\indexii{tuple}{type}
\indexii{list}{type}

Sequence types support the following operations.  The \samp{in} and
\samp{not in} operations have the same priorities as the comparison
operations.  The \samp{+} and \samp{*} operations have the same
priority as the corresponding numeric operations.\footnote{They must
have since the parser can't tell the type of the operands.}

This table lists the sequence operations sorted in ascending priority
(operations in the same box have the same priority).  In the table,
\var{s} and \var{t} are sequences of the same type; \var{n}, \var{i}
and \var{j} are integers:

\begin{tableiii}{|c|l|c|}{code}{Operation}{Result}{Notes}
  \lineiii{\var{x} in \var{s}}{\code{1} if an item of \var{s} is equal to \var{x}, else \code{0}}{}
  \lineiii{\var{x} not in \var{s}}{\code{0} if an item of \var{s} is
equal to \var{x}, else \code{1}}{}
  \hline
  \lineiii{\var{s} + \var{t}}{the concatenation of \var{s} and \var{t}}{}
  \hline
  \lineiii{\var{s} * \var{n}{\rm ,} \var{n} * \var{s}}{\var{n} copies of \var{s} concatenated}{(3)}
  \hline
  \lineiii{\var{s}[\var{i}]}{\var{i}'th item of \var{s}, origin 0}{(1)}
  \lineiii{\var{s}[\var{i}:\var{j}]}{slice of \var{s} from \var{i} to \var{j}}{(1), (2)}
  \hline
  \lineiii{len(\var{s})}{length of \var{s}}{}
  \lineiii{min(\var{s})}{smallest item of \var{s}}{}
  \lineiii{max(\var{s})}{largest item of \var{s}}{}
\end{tableiii}
\indexiii{operations on}{sequence}{types}
\bifuncindex{len}
\bifuncindex{min}
\bifuncindex{max}
\indexii{concatenation}{operation}
\indexii{repetition}{operation}
\indexii{subscript}{operation}
\indexii{slice}{operation}
\opindex{in}
\opindex{not in}

\noindent
Notes:

\begin{description}
  
\item[(1)] If \var{i} or \var{j} is negative, the index is relative to
  the end of the string, i.e., \code{len(\var{s}) + \var{i}} or
  \code{len(\var{s}) + \var{j}} is substituted.  But note that \code{-0} is
  still \code{0}.
  
\item[(2)] The slice of \var{s} from \var{i} to \var{j} is defined as
  the sequence of items with index \var{k} such that \code{\var{i} <=
  \var{k} < \var{j}}.  If \var{i} or \var{j} is greater than
  \code{len(\var{s})}, use \code{len(\var{s})}.  If \var{i} is omitted,
  use \code{0}.  If \var{j} is omitted, use \code{len(\var{s})}.  If
  \var{i} is greater than or equal to \var{j}, the slice is empty.

\item[(3)] Values of \var{n} less than \code{0} are treated as
  \code{0} (which yields an empty sequence of the same type as
  \var{s}).

\end{description}

\subsubsection{More String Operations}

String objects have one unique built-in operation: the \code{\%}
operator (modulo) with a string left argument interprets this string
as a \C{} \cfunction{sprintf()} format string to be applied to the
right argument, and returns the string resulting from this formatting
operation.

The right argument should be a tuple with one item for each argument
required by the format string; if the string requires a single
argument, the right argument may also be a single non-tuple object.%
\footnote{A tuple object in this case should be a singleton.}
The following format characters are understood:
\%, c, s, i, d, u, o, x, X, e, E, f, g, G.
Width and precision may be a * to specify that an integer argument
specifies the actual width or precision.  The flag characters -, +,
blank, \# and 0 are understood.  The size specifiers h, l or L may be
present but are ignored.  The \code{\%s} conversion takes any Python
object and converts it to a string using \code{str()} before
formatting it.  The ANSI features \code{\%p} and \code{\%n}
are not supported.  Since Python strings have an explicit length,
\code{\%s} conversions don't assume that \code{'\e0'} is the end of
the string.

For safety reasons, floating point precisions are clipped to 50;
\code{\%f} conversions for numbers whose absolute value is over 1e25
are replaced by \code{\%g} conversions.%
\footnote{These numbers are fairly arbitrary.  They are intended to
avoid printing endless strings of meaningless digits without hampering
correct use and without having to know the exact precision of floating
point values on a particular machine.}
All other errors raise exceptions.

If the right argument is a dictionary (or any kind of mapping), then
the formats in the string must have a parenthesized key into that
dictionary inserted immediately after the \code{\%} character, and
each format formats the corresponding entry from the mapping.  E.g.
\begin{verbatim}
>>> count = 2
>>> language = 'Python'
>>> print '%(language)s has %(count)03d quote types.' % vars()
Python has 002 quote types.
>>> 
\end{verbatim}
In this case no * specifiers may occur in a format (since they
require a sequential parameter list).

Additional string operations are defined in standard module
\code{string} and in built-in module \code{re}.
\refstmodindex{string}
\refbimodindex{re}

\subsubsection{Mutable Sequence Types}

List objects support additional operations that allow in-place
modification of the object.
These operations would be supported by other mutable sequence types
(when added to the language) as well.
Strings and tuples are immutable sequence types and such objects cannot
be modified once created.
The following operations are defined on mutable sequence types (where
\var{x} is an arbitrary object):
\indexiii{mutable}{sequence}{types}
\indexii{list}{type}

\begin{tableiii}{|c|l|c|}{code}{Operation}{Result}{Notes}
  \lineiii{\var{s}[\var{i}] = \var{x}}
	{item \var{i} of \var{s} is replaced by \var{x}}{}
  \lineiii{\var{s}[\var{i}:\var{j}] = \var{t}}
  	{slice of \var{s} from \var{i} to \var{j} is replaced by \var{t}}{}
  \lineiii{del \var{s}[\var{i}:\var{j}]}
	{same as \code{\var{s}[\var{i}:\var{j}] = []}}{}
  \lineiii{\var{s}.append(\var{x})}
	{same as \code{\var{s}[len(\var{s}):len(\var{s})] = [\var{x}]}}{}
  \lineiii{\var{s}.count(\var{x})}
	{return number of \var{i}'s for which \code{\var{s}[\var{i}] == \var{x}}}{}
  \lineiii{\var{s}.index(\var{x})}
	{return smallest \var{i} such that \code{\var{s}[\var{i}] == \var{x}}}{(1)}
  \lineiii{\var{s}.insert(\var{i}, \var{x})}
	{same as \code{\var{s}[\var{i}:\var{i}] = [\var{x}]}
	  if \code{\var{i} >= 0}}{}
  \lineiii{\var{s}.remove(\var{x})}
	{same as \code{del \var{s}[\var{s}.index(\var{x})]}}{(1)}
  \lineiii{\var{s}.reverse()}
	{reverses the items of \var{s} in place}{(3)}
  \lineiii{\var{s}.sort()}
	{sort the items of \var{s} in place}{(2), (3)}
\end{tableiii}
\indexiv{operations on}{mutable}{sequence}{types}
\indexiii{operations on}{sequence}{types}
\indexiii{operations on}{list}{type}
\indexii{subscript}{assignment}
\indexii{slice}{assignment}
\stindex{del}
\setindexsubitem{(list method)}
\ttindex{append}
\ttindex{count}
\ttindex{index}
\ttindex{insert}
\ttindex{remove}
\ttindex{reverse}
\ttindex{sort}

\noindent
Notes:
\begin{description}
\item[(1)] Raises an exception when \var{x} is not found in \var{s}.
  
\item[(2)] The \code{sort()} method takes an optional argument
  specifying a comparison function of two arguments (list items) which
  should return \code{-1}, \code{0} or \code{1} depending on whether the
  first argument is considered smaller than, equal to, or larger than the
  second argument.  Note that this slows the sorting process down
  considerably; e.g. to sort a list in reverse order it is much faster
  to use calls to \code{sort()} and \code{reverse()} than to use
  \code{sort()} with a comparison function that reverses the ordering of
  the elements.

\item[(3)] The \code{sort()} and \code{reverse()} methods modify the
list in place for economy of space when sorting or reversing a large
list.  They don't return the sorted or reversed list to remind you of
this side effect.

\end{description}


\subsection{Mapping Types}
\label{typesmapping}

A \dfn{mapping} object maps values of one type (the key type) to
arbitrary objects.  Mappings are mutable objects.  There is currently
only one standard mapping type, the \dfn{dictionary}.  A dictionary's keys are
almost arbitrary values.  The only types of values not acceptable as
keys are values containing lists or dictionaries or other mutable
types that are compared by value rather than by object identity.
Numeric types used for keys obey the normal rules for numeric
comparison: if two numbers compare equal (e.g. \code{1} and
\code{1.0}) then they can be used interchangeably to index the same
dictionary entry.

\indexii{mapping}{types}
\indexii{dictionary}{type}

Dictionaries are created by placing a comma-separated list of
\code{\var{key}:\,\var{value}} pairs within braces, for example:
\code{\{'jack':\,4098, 'sjoerd':\,4127\}} or
\code{\{4098:\,'jack', 4127:\,'sjoerd'\}}.

The following operations are defined on mappings (where \var{a} is a
mapping, \var{k} is a key and \var{x} is an arbitrary object):

\begin{tableiii}{|c|l|c|}{code}{Operation}{Result}{Notes}
  \lineiii{len(\var{a})}{the number of items in \var{a}}{}
  \lineiii{\var{a}[\var{k}]}{the item of \var{a} with key \var{k}}{(1)}
  \lineiii{\var{a}[\var{k}] = \var{x}}{set \code{\var{a}[\var{k}]} to \var{x}}{}
  \lineiii{del \var{a}[\var{k}]}{remove \code{\var{a}[\var{k}]} from \var{a}}{(1)}
  \lineiii{\var{a}.clear()}{remove all items from \code{a}}{}
  \lineiii{\var{a}.copy()}{a (shallow) copy of \code{a}}{}
  \lineiii{\var{a}.has_key(\var{k})}{\code{1} if \var{a} has a key \var{k}, else \code{0}}{}
  \lineiii{\var{a}.items()}{a copy of \var{a}'s list of (key, item) pairs}{(2)}
  \lineiii{\var{a}.keys()}{a copy of \var{a}'s list of keys}{(2)}
  \lineiii{\var{a}.update(\var{b})}{\code{for k, v in \var{b}.items(): \var{a}[k] = v}}{(3)}
  \lineiii{\var{a}.values()}{a copy of \var{a}'s list of values}{(2)}
  \lineiii{\var{a}.get(\var{k}, \var{f})}{the item of \var{a} with key \var{k}}{(4)}
\end{tableiii}
\indexiii{operations on}{mapping}{types}
\indexiii{operations on}{dictionary}{type}
\stindex{del}
\bifuncindex{len}
\setindexsubitem{(dictionary method)}
\ttindex{keys}
\ttindex{has_key}

\noindent
Notes:
\begin{description}
\item[(1)] Raises an exception if \var{k} is not in the map.

\item[(2)] Keys and values are listed in random order.

\item[(3)] \var{b} must be of the same type as \var{a}.

\item[(4)] Never raises an exception if \var{k} is not in the map,
instead it returns \var{f}.  \var{f} is optional, when not provided
and \var{k} is not in the map, \code{None} is returned.
\end{description}


\subsection{Other Built-in Types}
\label{typesother}

The interpreter supports several other kinds of objects.
Most of these support only one or two operations.

\subsubsection{Modules}

The only special operation on a module is attribute access:
\code{\var{m}.\var{name}}, where \var{m} is a module and \var{name} accesses
a name defined in \var{m}'s symbol table.  Module attributes can be
assigned to.  (Note that the \code{import} statement is not, strictly
spoken, an operation on a module object; \code{import \var{foo}} does not
require a module object named \var{foo} to exist, rather it requires
an (external) \emph{definition} for a module named \var{foo}
somewhere.)

A special member of every module is \code{__dict__}.
This is the dictionary containing the module's symbol table.
Modifying this dictionary will actually change the module's symbol
table, but direct assignment to the \code{__dict__} attribute is not
possible (i.e., you can write \code{\var{m}.__dict__['a'] = 1}, which
defines \code{\var{m}.a} to be \code{1}, but you can't write \code{\var{m}.__dict__ = \{\}}.

Modules are written like this: \code{<module 'sys'>}.

\subsubsection{Classes and Class Instances}
\nodename{Classes and Instances}

See Chapters 3 and 7 of the \emph{Python Reference Manual} for these.

\subsubsection{Functions}

Function objects are created by function definitions.  The only
operation on a function object is to call it:
\code{\var{func}(\var{argument-list})}.

There are really two flavors of function objects: built-in functions
and user-defined functions.  Both support the same operation (to call
the function), but the implementation is different, hence the
different object types.

The implementation adds two special read-only attributes:
\code{\var{f}.func_code} is a function's \dfn{code object} (see below) and
\code{\var{f}.func_globals} is the dictionary used as the function's
global name space (this is the same as \code{\var{m}.__dict__} where
\var{m} is the module in which the function \var{f} was defined).

\subsubsection{Methods}
\obindex{method}

Methods are functions that are called using the attribute notation.
There are two flavors: built-in methods (such as \code{append()} on
lists) and class instance methods.  Built-in methods are described
with the types that support them.

The implementation adds two special read-only attributes to class
instance methods: \code{\var{m}.im_self} is the object whose method this
is, and \code{\var{m}.im_func} is the function implementing the method.
Calling \code{\var{m}(\var{arg-1}, \var{arg-2}, {\rm \ldots},
\var{arg-n})} is completely equivalent to calling
\code{\var{m}.im_func(\var{m}.im_self, \var{arg-1}, \var{arg-2}, {\rm
\ldots}, \var{arg-n})}.

See the \emph{Python Reference Manual} for more information.

\subsubsection{Code Objects}
\obindex{code}

Code objects are used by the implementation to represent
``pseudo-compiled'' executable Python code such as a function body.
They differ from function objects because they don't contain a
reference to their global execution environment.  Code objects are
returned by the built-in \code{compile()} function and can be
extracted from function objects through their \code{func_code}
attribute.
\bifuncindex{compile}
\ttindex{func_code}

A code object can be executed or evaluated by passing it (instead of a
source string) to the \code{exec} statement or the built-in
\code{eval()} function.
\stindex{exec}
\bifuncindex{eval}

See the \emph{Python Reference Manual} for more information.

\subsubsection{Type Objects}

Type objects represent the various object types.  An object's type is
accessed by the built-in function \code{type()}.  There are no special
operations on types.  The standard module \code{types} defines names
for all standard built-in types.
\bifuncindex{type}
\refstmodindex{types}

Types are written like this: \code{<type 'int'>}.

\subsubsection{The Null Object}

This object is returned by functions that don't explicitly return a
value.  It supports no special operations.  There is exactly one null
object, named \code{None} (a built-in name).

It is written as \code{None}.

\subsubsection{File Objects}

File objects are implemented using \C{}'s \code{stdio} package and can be
created with the built-in function \code{open()} described under
Built-in Functions below.  They are also returned by some other
built-in functions and methods, e.g.\ \code{posix.popen()} and
\code{posix.fdopen()} and the \code{makefile()} method of socket
objects.
\bifuncindex{open}
\refbimodindex{posix}
\refbimodindex{socket}

When a file operation fails for an I/O-related reason, the exception
\code{IOError} is raised.  This includes situations where the
operation is not defined for some reason, like \code{seek()} on a tty
device or writing a file opened for reading.

Files have the following methods:


\setindexsubitem{(file method)}

\begin{funcdesc}{close}{}
  Close the file.  A closed file cannot be read or written anymore.
\end{funcdesc}

\begin{funcdesc}{flush}{}
  Flush the internal buffer, like \code{stdio}'s \code{fflush()}.
\end{funcdesc}

\begin{funcdesc}{isatty}{}
  Return \code{1} if the file is connected to a tty(-like) device, else
  \code{0}.
\end{funcdesc}

\begin{funcdesc}{fileno}{}
Return the integer ``file descriptor'' that is used by the underlying
implementation to request I/O operations from the operating system.
This can be useful for other, lower level interfaces that use file
descriptors, e.g. module \code{fcntl} or \code{os.read()} and friends.
\refbimodindex{fcntl}
\end{funcdesc}

\begin{funcdesc}{read}{\optional{size}}
  Read at most \var{size} bytes from the file (less if the read hits
  \EOF{} or no more data is immediately available on a pipe, tty or
  similar device).  If the \var{size} argument is negative or omitted,
  read all data until \EOF{} is reached.  The bytes are returned as a string
  object.  An empty string is returned when \EOF{} is encountered
  immediately.  (For certain files, like ttys, it makes sense to
  continue reading after an \EOF{} is hit.)
\end{funcdesc}

\begin{funcdesc}{readline}{\optional{size}}
  Read one entire line from the file.  A trailing newline character is
  kept in the string%
\footnote{The advantage of leaving the newline on is that an empty string 
	can be returned to mean \EOF{} without being ambiguous.  Another 
	advantage is that (in cases where it might matter, e.g. if you 
	want to make an exact copy of a file while scanning its lines) 
	you can tell whether the last line of a file ended in a newline
	or not (yes this happens!).}
  (but may be absent when a file ends with an
  incomplete line).  If the \var{size} argument is present and
  non-negative, it is a maximum byte count (including the trailing
  newline) and an incomplete line may be returned.
  An empty string is returned when \EOF{} is hit
  immediately.  Note: unlike \code{stdio}'s \code{fgets()}, the returned
  string contains null characters (\code{'\e 0'}) if they occurred in the
  input.
\end{funcdesc}

\begin{funcdesc}{readlines}{\optional{sizehint}}
  Read until \EOF{} using \code{readline()} and return a list containing
  the lines thus read.  If the optional \var{sizehint} argument is
  present, instead of reading up to \EOF{}, whole lines totalling
  approximately \var{sizehint} bytes (possibly after rounding up to an
  internal buffer size) are read.
\end{funcdesc}

\begin{funcdesc}{seek}{offset\, whence}
  Set the file's current position, like \code{stdio}'s \code{fseek()}.
  The \var{whence} argument is optional and defaults to \code{0}
  (absolute file positioning); other values are \code{1} (seek
  relative to the current position) and \code{2} (seek relative to the
  file's end).  There is no return value.
\end{funcdesc}

\begin{funcdesc}{tell}{}
  Return the file's current position, like \code{stdio}'s \code{ftell()}.
\end{funcdesc}

\begin{funcdesc}{truncate}{\optional{size}}
Truncate the file's size.  If the optional size argument present, the
file is truncated to (at most) that size.  The size defaults to the
current position.  Availability of this function depends on the
operating system version (e.g., not all \UNIX{} versions support this
operation).
\end{funcdesc}

\begin{funcdesc}{write}{str}
Write a string to the file.  There is no return value.  Note: due to
buffering, the string may not actually show up in the file until
the \code{flush()} or \code{close()} method is called.
\end{funcdesc}

\begin{funcdesc}{writelines}{list}
Write a list of strings to the file.  There is no return value.
(The name is intended to match \code{readlines}; \code{writelines}
does not add line separators.)
\end{funcdesc}

File objects also offer the following attributes:

\setindexsubitem{(file attribute)}

\begin{datadesc}{closed}
Boolean indicating the current state of the file object.  This is a
read-only attribute; the \method{close()} method changes the value.
\end{datadesc}

\begin{datadesc}{mode}
The I/O mode for the file.  If the file was created using the
\function{open()} built-in function, this will be the value of the
\var{mode} parameter.  This is a read-only attribute.
\end{datadesc}

\begin{datadesc}{name}
If the file object was created using \function{open()}, the name of
the file.  Otherwise, some string that indicates the source of the
file object, of the form \samp{<\mbox{\ldots}>}.  This is a read-only
attribute.
\end{datadesc}

\begin{datadesc}{softspace}
Boolean that indicates whether a space character needs to be printed
before another value when using the \keyword{print} statement.
Classes that are trying to simulate a file object should also have a
writable \code{softspace} attribute, which should be initialized to
zero.  This will be automatic for classes implemented in Python; types
implemented in \C{} will have to provide a writable \code{softspace}
attribute.
\end{datadesc}

\subsubsection{Internal Objects}

See the \emph{Python Reference Manual} for this information.  It
describes code objects, stack frame objects, traceback objects, and
slice objects.


\subsection{Special Attributes}
\label{specialattrs}

The implementation adds a few special read-only attributes to several
object types, where they are relevant:

\begin{itemize}

\item
\code{\var{x}.__dict__} is a dictionary of some sort used to store an
object's (writable) attributes;

\item
\code{\var{x}.__methods__} lists the methods of many built-in object types,
e.g., \code{[].__methods__} yields
\code{['append', 'count', 'index', 'insert', 'remove', 'reverse', 'sort']};

\item
\code{\var{x}.__members__} lists data attributes;

\item
\code{\var{x}.__class__} is the class to which a class instance belongs;

\item
\code{\var{x}.__bases__} is the tuple of base classes of a class object.

\end{itemize}

\section{\module{UserDict} ---
         Class wrapper for dictionary objects}

\declaremodule{standard}{UserDict}
\modulesynopsis{Class wrapper for dictionary objects.}

This module defines a class that acts as a wrapper around
dictionary objects.  It is a useful base class for
your own dictionary-like classes, which can inherit from
them and override existing methods or add new ones.  In this way one
can add new behaviors to dictionaries.

The \module{UserDict} module defines the \class{UserDict} class:

\begin{classdesc}{UserDict}{\optional{initialdata}}
Return a class instance that simulates a dictionary.  The instance's
contents are kept in a regular dictionary, which is accessible via the
\member{data} attribute of \class{UserDict} instances.  If
\var{initialdata} is provided, \member{data} is initialized with its
contents; note that a reference to \var{initialdata} will not be kept, 
allowing it be used used for other purposes.
\end{classdesc}

In addition to supporting the methods and operations of mappings (see
section \ref{typesmapping}), \class{UserDict} instances provide the
following attribute:

\begin{memberdesc}{data}
A real dictionary used to store the contents of the \class{UserDict}
class.
\end{memberdesc}


\section{\module{UserList} ---
         Class wrapper for list objects}

\declaremodule{standard}{UserList}
\modulesynopsis{Class wrapper for list objects.}


This module defines a class that acts as a wrapper around
list objects.  It is a useful base class for
your own list-like classes, which can inherit from
them and override existing methods or add new ones.  In this way one
can add new behaviors to lists.

The \module{UserList} module defines the \class{UserList} class:

\begin{classdesc}{UserList}{\optional{list}}
Return a class instance that simulates a list.  The instance's
contents are kept in a regular list, which is accessible via the
\member{data} attribute of \class{UserList} instances.  The instance's
contents are initially set to a copy of \var{list}, defaulting to the
empty list \code{[]}.  \var{list} can be either a regular Python list,
or an instance of \class{UserList} (or a subclass).
\end{classdesc}

In addition to supporting the methods and operations of mutable
sequences (see section \ref{typesseq}), \class{UserList} instances
provide the following attribute:

\begin{memberdesc}{data}
A real Python list object used to store the contents of the
\class{UserList} class.
\end{memberdesc}

\strong{Subclassing requirements:}
Subclasses of \class{UserList} are expect to offer a constructor which
can be called with either no arguments or one argument.  List
operations which return a new sequence attempt to create an instance
of the actual implementation class.  To do so, it assumes that the
constructor can be called with a single parameter, which is a sequence
object used as a data source.

If a derived class does not wish to comply with this requirement, all
of the special methods supported by this class will need to be
overridden; please consult the sources for information about the
methods which need to be provided in that case.

\versionchanged[Python versions 1.5.2 and 1.6 also required that the
                constructor be callable with no parameters, and offer
                a mutable \member{data} attribute.  Earlier versions
                of Python did not attempt to create instances of the
                derived class]{2.0}


\section{\module{UserString} ---
         Class wrapper for string objects}

\declaremodule{standard}{UserString}
\modulesynopsis{Class wrapper for string objects.}
\moduleauthor{Peter Funk}{pf@artcom-gmbh.de}
\sectionauthor{Peter Funk}{pf@artcom-gmbh.de}

This module defines a class that acts as a wrapper around string
objects.  It is a useful base class for your own string-like classes,
which can inherit from them and override existing methods or add new
ones.  In this way one can add new behaviors to strings.

It should be noted that these classes are highly inefficient compared
to real string or Unicode objects; this is especially the case for
\class{MutableString}.

The \module{UserString} module defines the following classes:

\begin{classdesc}{UserString}{\optional{sequence}}
Return a class instance that simulates a string or a Unicode string
object.  The instance's content is kept in a regular string or Unicode
string object, which is accessible via the \member{data} attribute of
\class{UserString} instances.  The instance's contents are initially
set to a copy of \var{sequence}.  \var{sequence} can be either a
regular Python string or Unicode string, an instance of
\class{UserString} (or a subclass) or an arbitrary sequence which can
be converted into a string using the built-in \function{str()} function.
\end{classdesc}

\begin{classdesc}{MutableString}{\optional{sequence}}
This class is derived from the \class{UserString} above and redefines
strings to be \emph{mutable}.  Mutable strings can't be used as
dictionary keys, because dictionaries require \emph{immutable} objects as
keys.  The main intention of this class is to serve as an educational
example for inheritance and necessity to remove (override) the
\method{__hash__()} method in order to trap attempts to use a
mutable object as dictionary key, which would be otherwise very
error prone and hard to track down.
\end{classdesc}

In addition to supporting the methods and operations of string and
Unicode objects (see section \ref{string-methods}, ``String
Methods''), \class{UserString} instances provide the following
attribute:

\begin{memberdesc}{data}
A real Python string or Unicode object used to store the content of the
\class{UserString} class.
\end{memberdesc}

\section{\module{operator} ---
         Standard operators as functions.}
\declaremodule{builtin}{operator}
\sectionauthor{Skip Montanaro}{skip@automatrix.com}

\modulesynopsis{All Python's standard operators as built-in functions.}


The \module{operator} module exports a set of functions implemented in C
corresponding to the intrinsic operators of Python.  For example,
\code{operator.add(x, y)} is equivalent to the expression \code{x+y}.  The
function names are those used for special class methods; variants without
leading and trailing \samp{__} are also provided for convenience.

The functions fall into categories that perform object comparisons,
logical operations, mathematical operations, sequence operations, and
abstract type tests.

The object comparison functions are useful for all objects, and are
named after the rich comparison operators they support:

\begin{funcdesc}{lt}{a, b}
\funcline{le}{a, b}
\funcline{eq}{a, b}
\funcline{ne}{a, b}
\funcline{ge}{a, b}
\funcline{gt}{a, b}
\funcline{__lt__}{a, b}
\funcline{__le__}{a, b}
\funcline{__eq__}{a, b}
\funcline{__ne__}{a, b}
\funcline{__ge__}{a, b}
\funcline{__gt__}{a, b}
Perform ``rich comparisons'' between \var{a} and \var{b}. Specifically,
\code{lt(\var{a}, \var{b})} is equivalent to \code{\var{a} < \var{b}},
\code{le(\var{a}, \var{b})} is equivalent to \code{\var{a} <= \var{b}},
\code{eq(\var{a}, \var{b})} is equivalent to \code{\var{a} == \var{b}},
\code{ne(\var{a}, \var{b})} is equivalent to \code{\var{a} != \var{b}},
\code{gt(\var{a}, \var{b})} is equivalent to \code{\var{a} > \var{b}}
and
\code{ge(\var{a}, \var{b})} is equivalent to \code{\var{a} >= \var{b}}.
Note that unlike the built-in \function{cmp()}, these functions can
return any value, which may or may not be interpretable as a Boolean
value.  See the \citetitle[../ref/ref.html]{Python Reference Manual}
for more information about rich comparisons.
\versionadded{2.2}
\end{funcdesc}


The logical operations are also generally applicable to all objects,
and support truth tests, identity tests, and boolean operations:

\begin{funcdesc}{not_}{o}
\funcline{__not__}{o}
Return the outcome of \keyword{not} \var{o}.  (Note that there is no
\method{__not__()} method for object instances; only the interpreter
core defines this operation.  The result is affected by the
\method{__nonzero__()} and \method{__len__()} methods.)
\end{funcdesc}

\begin{funcdesc}{truth}{o}
Return \constant{True} if \var{o} is true, and \constant{False}
otherwise.  This is equivalent to using the \class{bool}
constructor.
\end{funcdesc}

\begin{funcdesc}{is_}{a, b}
Return \code{\var{a} is \var{b}}.  Tests object identity.
\versionadded{2.3}
\end{funcdesc}

\begin{funcdesc}{is_not}{a, b}
Return \code{\var{a} is not \var{b}}.  Tests object identity.
\versionadded{2.3}
\end{funcdesc}


The mathematical and bitwise operations are the most numerous:

\begin{funcdesc}{abs}{o}
\funcline{__abs__}{o}
Return the absolute value of \var{o}.
\end{funcdesc}

\begin{funcdesc}{add}{a, b}
\funcline{__add__}{a, b}
Return \var{a} \code{+} \var{b}, for \var{a} and \var{b} numbers.
\end{funcdesc}

\begin{funcdesc}{and_}{a, b}
\funcline{__and__}{a, b}
Return the bitwise and of \var{a} and \var{b}.
\end{funcdesc}

\begin{funcdesc}{div}{a, b}
\funcline{__div__}{a, b}
Return \var{a} \code{/} \var{b} when \code{__future__.division} is not
in effect.  This is also known as ``classic'' division.
\end{funcdesc}

\begin{funcdesc}{floordiv}{a, b}
\funcline{__floordiv__}{a, b}
Return \var{a} \code{//} \var{b}.
\versionadded{2.2}
\end{funcdesc}

\begin{funcdesc}{inv}{o}
\funcline{invert}{o}
\funcline{__inv__}{o}
\funcline{__invert__}{o}
Return the bitwise inverse of the number \var{o}.  This is equivalent
to \code{\textasciitilde}\var{o}.  The names \function{invert()} and
\function{__invert__()} were added in Python 2.0.
\end{funcdesc}

\begin{funcdesc}{lshift}{a, b}
\funcline{__lshift__}{a, b}
Return \var{a} shifted left by \var{b}.
\end{funcdesc}

\begin{funcdesc}{mod}{a, b}
\funcline{__mod__}{a, b}
Return \var{a} \code{\%} \var{b}.
\end{funcdesc}

\begin{funcdesc}{mul}{a, b}
\funcline{__mul__}{a, b}
Return \var{a} \code{*} \var{b}, for \var{a} and \var{b} numbers.
\end{funcdesc}

\begin{funcdesc}{neg}{o}
\funcline{__neg__}{o}
Return \var{o} negated.
\end{funcdesc}

\begin{funcdesc}{or_}{a, b}
\funcline{__or__}{a, b}
Return the bitwise or of \var{a} and \var{b}.
\end{funcdesc}

\begin{funcdesc}{pos}{o}
\funcline{__pos__}{o}
Return \var{o} positive.
\end{funcdesc}

\begin{funcdesc}{pow}{a, b}
\funcline{__pow__}{a, b}
Return \var{a} \code{**} \var{b}, for \var{a} and \var{b} numbers.
\versionadded{2.3}
\end{funcdesc}

\begin{funcdesc}{rshift}{a, b}
\funcline{__rshift__}{a, b}
Return \var{a} shifted right by \var{b}.
\end{funcdesc}

\begin{funcdesc}{sub}{a, b}
\funcline{__sub__}{a, b}
Return \var{a} \code{-} \var{b}.
\end{funcdesc}

\begin{funcdesc}{truediv}{a, b}
\funcline{__truediv__}{a, b}
Return \var{a} \code{/} \var{b} when \code{__future__.division} is in
effect.  This is also known as division.
\versionadded{2.2}
\end{funcdesc}

\begin{funcdesc}{xor}{a, b}
\funcline{__xor__}{a, b}
Return the bitwise exclusive or of \var{a} and \var{b}.
\end{funcdesc}


Operations which work with sequences include:

\begin{funcdesc}{concat}{a, b}
\funcline{__concat__}{a, b}
Return \var{a} \code{+} \var{b} for \var{a} and \var{b} sequences.
\end{funcdesc}

\begin{funcdesc}{contains}{a, b}
\funcline{__contains__}{a, b}
Return the outcome of the test \var{b} \code{in} \var{a}.
Note the reversed operands.  The name \function{__contains__()} was
added in Python 2.0.
\end{funcdesc}

\begin{funcdesc}{countOf}{a, b}
Return the number of occurrences of \var{b} in \var{a}.
\end{funcdesc}

\begin{funcdesc}{delitem}{a, b}
\funcline{__delitem__}{a, b}
Remove the value of \var{a} at index \var{b}.
\end{funcdesc}

\begin{funcdesc}{delslice}{a, b, c}
\funcline{__delslice__}{a, b, c}
Delete the slice of \var{a} from index \var{b} to index \var{c}\code{-1}.
\end{funcdesc}

\begin{funcdesc}{getitem}{a, b}
\funcline{__getitem__}{a, b}
Return the value of \var{a} at index \var{b}.
\end{funcdesc}

\begin{funcdesc}{getslice}{a, b, c}
\funcline{__getslice__}{a, b, c}
Return the slice of \var{a} from index \var{b} to index \var{c}\code{-1}.
\end{funcdesc}

\begin{funcdesc}{indexOf}{a, b}
Return the index of the first of occurrence of \var{b} in \var{a}.
\end{funcdesc}

\begin{funcdesc}{repeat}{a, b}
\funcline{__repeat__}{a, b}
Return \var{a} \code{*} \var{b} where \var{a} is a sequence and
\var{b} is an integer.
\end{funcdesc}

\begin{funcdesc}{sequenceIncludes}{\unspecified}
\deprecated{2.0}{Use \function{contains()} instead.}
Alias for \function{contains()}.
\end{funcdesc}

\begin{funcdesc}{setitem}{a, b, c}
\funcline{__setitem__}{a, b, c}
Set the value of \var{a} at index \var{b} to \var{c}.
\end{funcdesc}

\begin{funcdesc}{setslice}{a, b, c, v}
\funcline{__setslice__}{a, b, c, v}
Set the slice of \var{a} from index \var{b} to index \var{c}\code{-1} to the
sequence \var{v}.
\end{funcdesc}


The \module{operator} module also defines a few predicates to test the
type of objects.  \note{Be careful not to misinterpret the
results of these functions; only \function{isCallable()} has any
measure of reliability with instance objects.  For example:}

\begin{verbatim}
>>> class C:
...     pass
... 
>>> import operator
>>> o = C()
>>> operator.isMappingType(o)
True
\end{verbatim}

\begin{funcdesc}{isCallable}{o}
\deprecated{2.0}{Use the \function{callable()} built-in function instead.}
Returns true if the object \var{o} can be called like a function,
otherwise it returns false.  True is returned for functions, bound and
unbound methods, class objects, and instance objects which support the
\method{__call__()} method.
\end{funcdesc}

\begin{funcdesc}{isMappingType}{o}
Returns true if the object \var{o} supports the mapping interface.
This is true for dictionaries and all instance objects defining
\method{__getitem__}.
\warning{There is no reliable way to test if an instance
supports the complete mapping protocol since the interface itself is
ill-defined.  This makes this test less useful than it otherwise might
be.}
\end{funcdesc}

\begin{funcdesc}{isNumberType}{o}
Returns true if the object \var{o} represents a number.  This is true
for all numeric types implemented in C.
\warning{There is no reliable way to test if an instance
supports the complete numeric interface since the interface itself is
ill-defined.  This makes this test less useful than it otherwise might
be.}
\end{funcdesc}

\begin{funcdesc}{isSequenceType}{o}
Returns true if the object \var{o} supports the sequence protocol.
This returns true for all objects which define sequence methods in C,
and for all instance objects defining \method{__getitem__}.
\warning{There is no reliable
way to test if an instance supports the complete sequence interface
since the interface itself is ill-defined.  This makes this test less
useful than it otherwise might be.}
\end{funcdesc}


Example: Build a dictionary that maps the ordinals from \code{0} to
\code{255} to their character equivalents.

\begin{verbatim}
>>> import operator
>>> d = {}
>>> keys = range(256)
>>> vals = map(chr, keys)
>>> map(operator.setitem, [d]*len(keys), keys, vals)
\end{verbatim}


The \module{operator} module also defines tools for generalized attribute
and item lookups.  These are useful for making fast field extractors
as arguments for \function{map()}, \function{sorted()},
\method{itertools.groupby()}, or other functions that expect a
function argument.

\begin{funcdesc}{attrgetter}{attr\optional{, args...}}
Return a callable object that fetches \var{attr} from its operand.
If more than one attribute is requested, returns a tuple of attributes.
After, \samp{f=attrgetter('name')}, the call \samp{f(b)} returns
\samp{b.name}.  After, \samp{f=attrgetter('name', 'date')}, the call
\samp{f(b)} returns \samp{(b.name, b.date)}. 
\versionadded{2.4}
\versionchanged[Added support for multiple attributes]{2.5}
\end{funcdesc}
    
\begin{funcdesc}{itemgetter}{item\optional{, args...}}
Return a callable object that fetches \var{item} from its operand.
If more than one item is requested, returns a tuple of items.
After, \samp{f=itemgetter(2)}, the call \samp{f(b)} returns
\samp{b[2]}.
After, \samp{f=itemgetter(2,5,3)}, the call \samp{f(b)} returns
\samp{(b[2], b[5], b[3])}.		
\versionadded{2.4}
\versionchanged[Added support for multiple item extraction]{2.5}		
\end{funcdesc}

Examples:
                
\begin{verbatim}
>>> from operator import itemgetter
>>> inventory = [('apple', 3), ('banana', 2), ('pear', 5), ('orange', 1)]
>>> getcount = itemgetter(1)
>>> map(getcount, inventory)
[3, 2, 5, 1]
>>> sorted(inventory, key=getcount)
[('orange', 1), ('banana', 2), ('apple', 3), ('pear', 5)]
\end{verbatim}
                

\subsection{Mapping Operators to Functions \label{operator-map}}

This table shows how abstract operations correspond to operator
symbols in the Python syntax and the functions in the
\refmodule{operator} module.


\begin{tableiii}{l|c|l}{textrm}{Operation}{Syntax}{Function}
  \lineiii{Addition}{\code{\var{a} + \var{b}}}
          {\code{add(\var{a}, \var{b})}}
  \lineiii{Concatenation}{\code{\var{seq1} + \var{seq2}}}
          {\code{concat(\var{seq1}, \var{seq2})}}
  \lineiii{Containment Test}{\code{\var{o} in \var{seq}}}
          {\code{contains(\var{seq}, \var{o})}}
  \lineiii{Division}{\code{\var{a} / \var{b}}}
          {\code{div(\var{a}, \var{b}) \#} without \code{__future__.division}}
  \lineiii{Division}{\code{\var{a} / \var{b}}}
          {\code{truediv(\var{a}, \var{b}) \#} with \code{__future__.division}}
  \lineiii{Division}{\code{\var{a} // \var{b}}}
          {\code{floordiv(\var{a}, \var{b})}}
  \lineiii{Bitwise And}{\code{\var{a} \&\ \var{b}}}
          {\code{and_(\var{a}, \var{b})}}
  \lineiii{Bitwise Exclusive Or}{\code{\var{a} \^\ \var{b}}}
          {\code{xor(\var{a}, \var{b})}}
  \lineiii{Bitwise Inversion}{\code{\~{} \var{a}}}
          {\code{invert(\var{a})}}
  \lineiii{Bitwise Or}{\code{\var{a} | \var{b}}}
          {\code{or_(\var{a}, \var{b})}}
  \lineiii{Exponentiation}{\code{\var{a} ** \var{b}}}
          {\code{pow(\var{a}, \var{b})}}
  \lineiii{Identity}{\code{\var{a} is \var{b}}}
          {\code{is_(\var{a}, \var{b})}}
  \lineiii{Identity}{\code{\var{a} is not \var{b}}}
          {\code{is_not(\var{a}, \var{b})}}
  \lineiii{Indexed Assignment}{\code{\var{o}[\var{k}] = \var{v}}}
          {\code{setitem(\var{o}, \var{k}, \var{v})}}
  \lineiii{Indexed Deletion}{\code{del \var{o}[\var{k}]}}
          {\code{delitem(\var{o}, \var{k})}}
  \lineiii{Indexing}{\code{\var{o}[\var{k}]}}
          {\code{getitem(\var{o}, \var{k})}}
  \lineiii{Left Shift}{\code{\var{a} <\code{<} \var{b}}}
          {\code{lshift(\var{a}, \var{b})}}
  \lineiii{Modulo}{\code{\var{a} \%\ \var{b}}}
          {\code{mod(\var{a}, \var{b})}}
  \lineiii{Multiplication}{\code{\var{a} * \var{b}}}
          {\code{mul(\var{a}, \var{b})}}
  \lineiii{Negation (Arithmetic)}{\code{- \var{a}}}
          {\code{neg(\var{a})}}
  \lineiii{Negation (Logical)}{\code{not \var{a}}}
          {\code{not_(\var{a})}}
  \lineiii{Right Shift}{\code{\var{a} >\code{>} \var{b}}}
          {\code{rshift(\var{a}, \var{b})}}
  \lineiii{Sequence Repitition}{\code{\var{seq} * \var{i}}}
          {\code{repeat(\var{seq}, \var{i})}}
  \lineiii{Slice Assignment}{\code{\var{seq}[\var{i}:\var{j}]} = \var{values}}
          {\code{setslice(\var{seq}, \var{i}, \var{j}, \var{values})}}
  \lineiii{Slice Deletion}{\code{del \var{seq}[\var{i}:\var{j}]}}
          {\code{delslice(\var{seq}, \var{i}, \var{j})}}
  \lineiii{Slicing}{\code{\var{seq}[\var{i}:\var{j}]}}
          {\code{getslice(\var{seq}, \var{i}, \var{j})}}
  \lineiii{String Formatting}{\code{\var{s} \%\ \var{o}}}
          {\code{mod(\var{s}, \var{o})}}
  \lineiii{Subtraction}{\code{\var{a} - \var{b}}}
          {\code{sub(\var{a}, \var{b})}}
  \lineiii{Truth Test}{\code{\var{o}}}
          {\code{truth(\var{o})}}
  \lineiii{Ordering}{\code{\var{a} < \var{b}}}
          {\code{lt(\var{a}, \var{b})}}
  \lineiii{Ordering}{\code{\var{a} <= \var{b}}}
          {\code{le(\var{a}, \var{b})}}
  \lineiii{Equality}{\code{\var{a} == \var{b}}}
          {\code{eq(\var{a}, \var{b})}}
  \lineiii{Difference}{\code{\var{a} != \var{b}}}
          {\code{ne(\var{a}, \var{b})}}
  \lineiii{Ordering}{\code{\var{a} >= \var{b}}}
          {\code{ge(\var{a}, \var{b})}}
  \lineiii{Ordering}{\code{\var{a} > \var{b}}}
          {\code{gt(\var{a}, \var{b})}}
\end{tableiii}

\section{Standard Module \module{traceback}}
\declaremodule{standard}{traceback}

\modulesynopsis{Print or retrieve a stack traceback.}



This module provides a standard interface to extract, format and print
stack traces of Python programs.  It exactly mimics the behavior of
the Python interpreter when it prints a stack trace.  This is useful
when you want to print stack traces under program control, e.g. in a
``wrapper'' around the interpreter.

The module uses traceback objects --- this is the object type
that is stored in the variables \code{sys.exc_traceback} and
\code{sys.last_traceback} and returned as the third item from
\function{sys.exc_info()}.
\obindex{traceback}

The module defines the following functions:

\begin{funcdesc}{print_tb}{traceback\optional{, limit\optional{, file}}}
Print up to \var{limit} stack trace entries from \var{traceback}.  If
\var{limit} is omitted or \code{None}, all entries are printed.
If \var{file} is omitted or \code{None}, the output goes to
\code{sys.stderr}; otherwise it should be an open file or file-like
object to receive the output.
\end{funcdesc}

\begin{funcdesc}{extract_tb}{traceback\optional{, limit}}
Return a list of up to \var{limit} ``pre-processed'' stack trace
entries extracted from \var{traceback}.  It is useful for alternate
formatting of stack traces.  If \var{limit} is omitted or \code{None},
all entries are extracted.  A ``pre-processed'' stack trace entry is a
quadruple (\var{filename}, \var{line number}, \var{function name},
\var{line text}) representing the information that is usually printed
for a stack trace.  The \var{line text} is a string with leading and
trailing whitespace stripped; if the source is not available it is
\code{None}.
\end{funcdesc}

\begin{funcdesc}{print_exception}{type, value,
traceback\optional{, limit\optional{, file}}}
Print exception information and up to \var{limit} stack trace entries
from \var{traceback} to \var{file}.
This differs from \function{print_tb()} in the
following ways: (1) if \var{traceback} is not \code{None}, it prints a
header \samp{Traceback (innermost last):}; (2) it prints the
exception \var{type} and \var{value} after the stack trace; (3) if
\var{type} is \exception{SyntaxError} and \var{value} has the appropriate
format, it prints the line where the syntax error occurred with a
caret indicating the approximate position of the error.
\end{funcdesc}

\begin{funcdesc}{print_exc}{\optional{limit\optional{, file}}}
This is a shorthand for `\code{print_exception(sys.exc_type,}
\code{sys.exc_value,} \code{sys.exc_traceback,} \var{limit}\code{,}
\var{file}\code{)}'.  (In fact, it uses \code{sys.exc_info()} to
retrieve the same information in a thread-safe way.)
\end{funcdesc}

\begin{funcdesc}{print_last}{\optional{limit\optional{, file}}}
This is a shorthand for `\code{print_exception(sys.last_type,}
\code{sys.last_value,} \code{sys.last_traceback,} \var{limit}\code{,}
\var{file}\code{)}'.
\end{funcdesc}

\begin{funcdesc}{print_stack}{\optional{f\optional{, limit\optional{, file}}}}
This function prints a stack trace from its invocation point.  The
optional \var{f} argument can be used to specify an alternate stack
frame to start.  The optional \var{limit} and \var{file} arguments have the
same meaning as for \function{print_exception()}.
\end{funcdesc}

\begin{funcdesc}{extract_tb}{tb\optional{, limit}}
Return a list containing the raw (unformatted) traceback information
extracted from the traceback object \var{tb}.  The optional
\var{limit} argument has the same meaning as for
\function{print_exception()}.  The items in the returned list are
4-tuples containing the following values: filename, line number,
function name, and source text line.  The source text line is stripped 
of leading and trailing whitespace; it is \code{None} when the source
text file is unavailable.
\end{funcdesc}

\begin{funcdesc}{extract_stack}{\optional{f\optional{, limit}}}
Extract the raw traceback from the current stack frame.  The return
value has the same format as for \function{extract_tb()}.  The
optional \var{f} and \var{limit} arguments have the same meaning as
for \function{print_stack()}.
\end{funcdesc}

\begin{funcdesc}{format_list}{list}
Given a list of tuples as returned by \function{extract_tb()} or
\function{extract_stack()}, return a list of strings ready for
printing.  Each string in the resulting list corresponds to the item
with the same index in the argument list.  Each string ends in a
newline; the strings may contain internal newlines as well, for those
items whose source text line is not \code{None}.
\end{funcdesc}

\begin{funcdesc}{format_exception_only}{type, value}
Format the exception part of a traceback.  The arguments are the
exception type and value such as given by \code{sys.last_type} and
\code{sys.last_value}.  The return value is a list of strings, each
ending in a newline.  Normally, the list contains a single string;
however, for \code{SyntaxError} exceptions, it contains several lines
that (when printed) display detailed information about where the
syntax error occurred.  The message indicating which exception
occurred is the always last string in the list.
\end{funcdesc}

\begin{funcdesc}{format_exception}{type, value, tb\optional{, limit}}
Format a stack trace and the exception information.  The arguments 
have the same meaning as the corresponding arguments to
\function{print_exception()}.  The return value is a list of strings,
each ending in a newline and some containing internal newlines.  When
these lines are contatenated and printed, exactly the same text is
printed as does \function{print_exception()}.
\end{funcdesc}

\begin{funcdesc}{format_tb}{tb\optional{, limit}}
A shorthand for \code{format_list(extract_tb(\var{tb}, \var{limit}))}.
\end{funcdesc}

\begin{funcdesc}{format_stack}{\optional{f\optional{, limit}}}
A shorthand for \code{format_list(extract_stack(\var{f}, \var{limit}))}.
\end{funcdesc}

\begin{funcdesc}{tb_lineno}{tb}
This function returns the current line number set in the traceback
object.  This is normally the same as the \code{\var{tb}.tb_lineno}
field of the object, but when optimization is used (the -O flag) this
field is not updated correctly; this function calculates the correct
value.
\end{funcdesc}

A simple example follows:

\begin{verbatim}
import sys, traceback

def run_user_code(envdir):
    source = raw_input(">>> ")
    try:
        exec source in envdir
    except:
        print "Exception in user code:"
        print '-'*60
        traceback.print_exc(file=sys.stdout)
        print '-'*60

envdir = {}
while 1:
    run_user_code(envdir)
\end{verbatim}

\section{\module{pickle} ---
         Python object serialization}

\declaremodule{standard}{pickle}
\modulesynopsis{Convert Python objects to streams of bytes and back.}

\index{persistency}
\indexii{persistent}{objects}
\indexii{serializing}{objects}
\indexii{marshalling}{objects}
\indexii{flattening}{objects}
\indexii{pickling}{objects}


The \module{pickle} module implements a basic but powerful algorithm
for ``pickling'' (a.k.a.\ serializing, marshalling or flattening)
nearly arbitrary Python objects.  This is the act of converting
objects to a stream of bytes (and back: ``unpickling'').  This is a
more primitive notion than persistency --- although \module{pickle}
reads and writes file objects, it does not handle the issue of naming
persistent objects, nor the (even more complicated) area of concurrent
access to persistent objects.  The \module{pickle} module can
transform a complex object into a byte stream and it can transform the
byte stream into an object with the same internal structure.  The most
obvious thing to do with these byte streams is to write them onto a
file, but it is also conceivable to send them across a network or
store them in a database.  The module
\refmodule{shelve}\refstmodindex{shelve} provides a simple interface
to pickle and unpickle objects on DBM-style database files.


\strong{Note:} The \module{pickle} module is rather slow.  A
reimplementation of the same algorithm in C, which is up to 1000 times
faster, is available as the
\refmodule{cPickle}\refbimodindex{cPickle} module.  This has the same
interface except that \class{Pickler} and \class{Unpickler} are
factory functions, not classes (so they cannot be used as base classes
for inheritance).

Unlike the built-in module \refmodule{marshal}\refbimodindex{marshal},
\module{pickle} handles the following correctly:


\begin{itemize}

\item recursive objects (objects containing references to themselves)

\item object sharing (references to the same object in different places)

\item user-defined classes and their instances

\end{itemize}

The data format used by \module{pickle} is Python-specific.  This has
the advantage that there are no restrictions imposed by external
standards such as
XDR\index{XDR}\index{External Data Representation} (which can't
represent pointer sharing); however it means that non-Python programs
may not be able to reconstruct pickled Python objects.

By default, the \module{pickle} data format uses a printable \ASCII{}
representation.  This is slightly more voluminous than a binary
representation.  The big advantage of using printable \ASCII{} (and of
some other characteristics of \module{pickle}'s representation) is that
for debugging or recovery purposes it is possible for a human to read
the pickled file with a standard text editor.

A binary format, which is slightly more efficient, can be chosen by
specifying a nonzero (true) value for the \var{bin} argument to the
\class{Pickler} constructor or the \function{dump()} and \function{dumps()}
functions.  The binary format is not the default because of backwards
compatibility with the Python 1.4 pickle module.  In a future version,
the default may change to binary.

The \module{pickle} module doesn't handle code objects, which the
\refmodule{marshal}\refbimodindex{marshal} module does.  I suppose
\module{pickle} could, and maybe it should, but there's probably no
great need for it right now (as long as \refmodule{marshal} continues
to be used for reading and writing code objects), and at least this
avoids the possibility of smuggling Trojan horses into a program.

For the benefit of persistency modules written using \module{pickle}, it
supports the notion of a reference to an object outside the pickled
data stream.  Such objects are referenced by a name, which is an
arbitrary string of printable \ASCII{} characters.  The resolution of
such names is not defined by the \module{pickle} module --- the
persistent object module will have to implement a method
\method{persistent_load()}.  To write references to persistent objects,
the persistent module must define a method \method{persistent_id()} which
returns either \code{None} or the persistent ID of the object.

There are some restrictions on the pickling of class instances.

First of all, the class must be defined at the top level in a module.
Furthermore, all its instance variables must be picklable.

\setindexsubitem{(pickle protocol)}

When a pickled class instance is unpickled, its \method{__init__()} method
is normally \emph{not} invoked.  \strong{Note:} This is a deviation
from previous versions of this module; the change was introduced in
Python 1.5b2.  The reason for the change is that in many cases it is
desirable to have a constructor that requires arguments; it is a
(minor) nuisance to have to provide a \method{__getinitargs__()} method.

If it is desirable that the \method{__init__()} method be called on
unpickling, a class can define a method \method{__getinitargs__()},
which should return a \emph{tuple} containing the arguments to be
passed to the class constructor (\method{__init__()}).  This method is
called at pickle time; the tuple it returns is incorporated in the
pickle for the instance.
\withsubitem{(copy protocol)}{\ttindex{__getinitargs__()}}
\withsubitem{(instance constructor)}{\ttindex{__init__()}}

Classes can further influence how their instances are pickled --- if
the class
\withsubitem{(copy protocol)}{
  \ttindex{__getstate__()}\ttindex{__setstate__()}}
\withsubitem{(instance attribute)}{
  \ttindex{__dict__}}
defines the method \method{__getstate__()}, it is called and the return
state is pickled as the contents for the instance, and if the class
defines the method \method{__setstate__()}, it is called with the
unpickled state.  (Note that these methods can also be used to
implement copying class instances.)  If there is no
\method{__getstate__()} method, the instance's \member{__dict__} is
pickled.  If there is no \method{__setstate__()} method, the pickled
object must be a dictionary and its items are assigned to the new
instance's dictionary.  (If a class defines both \method{__getstate__()}
and \method{__setstate__()}, the state object needn't be a dictionary
--- these methods can do what they want.)  This protocol is also used
by the shallow and deep copying operations defined in the
\refmodule{copy}\refstmodindex{copy} module.

Note that when class instances are pickled, their class's code and
data are not pickled along with them.  Only the instance data are
pickled.  This is done on purpose, so you can fix bugs in a class or
add methods and still load objects that were created with an earlier
version of the class.  If you plan to have long-lived objects that
will see many versions of a class, it may be worthwhile to put a version
number in the objects so that suitable conversions can be made by the
class's \method{__setstate__()} method.

When a class itself is pickled, only its name is pickled --- the class
definition is not pickled, but re-imported by the unpickling process.
Therefore, the restriction that the class must be defined at the top
level in a module applies to pickled classes as well.

\setindexsubitem{(in module pickle)}

The interface can be summarized as follows.

To pickle an object \code{x} onto a file \code{f}, open for writing:

\begin{verbatim}
p = pickle.Pickler(f)
p.dump(x)
\end{verbatim}

A shorthand for this is:

\begin{verbatim}
pickle.dump(x, f)
\end{verbatim}

To unpickle an object \code{x} from a file \code{f}, open for reading:

\begin{verbatim}
u = pickle.Unpickler(f)
x = u.load()
\end{verbatim}

A shorthand is:

\begin{verbatim}
x = pickle.load(f)
\end{verbatim}

The \class{Pickler} class only calls the method \code{f.write()} with a
\withsubitem{(class in pickle)}{
  \ttindex{Unpickler}\ttindex{Pickler}}
string argument.  The \class{Unpickler} calls the methods \code{f.read()}
(with an integer argument) and \code{f.readline()} (without argument),
both returning a string.  It is explicitly allowed to pass non-file
objects here, as long as they have the right methods.

The constructor for the \class{Pickler} class has an optional second
argument, \var{bin}.  If this is present and nonzero, the binary
pickle format is used; if it is zero or absent, the (less efficient,
but backwards compatible) text pickle format is used.  The
\class{Unpickler} class does not have an argument to distinguish
between binary and text pickle formats; it accepts either format.

The following types can be pickled:

\begin{itemize}

\item \code{None}

\item integers, long integers, floating point numbers

\item strings

\item tuples, lists and dictionaries containing only picklable objects

\item classes that are defined at the top level in a module

\item instances of such classes whose \member{__dict__} or
\method{__setstate__()} is picklable

\end{itemize}

Attempts to pickle unpicklable objects will raise the
\exception{PicklingError} exception; when this happens, an unspecified
number of bytes may have been written to the file.

It is possible to make multiple calls to the \method{dump()} method of
the same \class{Pickler} instance.  These must then be matched to the
same number of calls to the \method{load()} method of the
corresponding \class{Unpickler} instance.  If the same object is
pickled by multiple \method{dump()} calls, the \method{load()} will all
yield references to the same object.  \emph{Warning}: this is intended
for pickling multiple objects without intervening modifications to the
objects or their parts.  If you modify an object and then pickle it
again using the same \class{Pickler} instance, the object is not
pickled again --- a reference to it is pickled and the
\class{Unpickler} will return the old value, not the modified one.
(There are two problems here: (a) detecting changes, and (b)
marshalling a minimal set of changes.  I have no answers.  Garbage
Collection may also become a problem here.)

Apart from the \class{Pickler} and \class{Unpickler} classes, the
module defines the following functions, and an exception:

\begin{funcdesc}{dump}{object, file\optional{, bin}}
Write a pickled representation of \var{obect} to the open file object
\var{file}.  This is equivalent to
\samp{Pickler(\var{file}, \var{bin}).dump(\var{object})}.
If the optional \var{bin} argument is present and nonzero, the binary
pickle format is used; if it is zero or absent, the (less efficient)
text pickle format is used.
\end{funcdesc}

\begin{funcdesc}{load}{file}
Read a pickled object from the open file object \var{file}.  This is
equivalent to \samp{Unpickler(\var{file}).load()}.
\end{funcdesc}

\begin{funcdesc}{dumps}{object\optional{, bin}}
Return the pickled representation of the object as a string, instead
of writing it to a file.  If the optional \var{bin} argument is
present and nonzero, the binary pickle format is used; if it is zero
or absent, the (less efficient) text pickle format is used.
\end{funcdesc}

\begin{funcdesc}{loads}{string}
Read a pickled object from a string instead of a file.  Characters in
the string past the pickled object's representation are ignored.
\end{funcdesc}

\begin{excdesc}{PicklingError}
This exception is raised when an unpicklable object is passed to
\method{Pickler.dump()}.
\end{excdesc}


\begin{seealso}
  \seemodule[copyreg]{copy_reg}{pickle interface constructor
                                registration}

  \seemodule{shelve}{indexed databases of objects; uses \module{pickle}}

  \seemodule{copy}{shallow and deep object copying}

  \seemodule{marshal}{high-performance serialization of built-in types}
\end{seealso}


\section{\module{cPickle} ---
         Alternate implementation of \module{pickle}}

\declaremodule{builtin}{cPickle}
\modulesynopsis{Faster version of \module{pickle}, but not subclassable.}
\moduleauthor{Jim Fulton}{jfulton@digicool.com}
\sectionauthor{Fred L. Drake, Jr.}{fdrake@acm.org}


The \module{cPickle} module provides a similar interface and identical
functionality as the \refmodule{pickle}\refstmodindex{pickle} module,
but can be up to 1000 times faster since it is implemented in C.  The
only other important difference to note is that \function{Pickler()}
and \function{Unpickler()} are functions and not classes, and so
cannot be subclassed.  This should not be an issue in most cases.

The format of the pickle data is identical to that produced using the
\refmodule{pickle} module, so it is possible to use \refmodule{pickle} and
\module{cPickle} interchangably with existing pickles.

(Since the pickle data format is actually a tiny stack-oriented
programming language, and there are some freedoms in the encodings of
certain objects, it's possible that the two modules produce different
pickled data for the same input objects; however they will always be
able to read each others pickles back in.)

\section{\module{copy_reg} ---
         Register \module{pickle} support functions}

\declaremodule[copyreg]{standard}{copy_reg}
\modulesynopsis{Register \module{pickle} support functions.}


The \module{copy_reg} module provides support for the
\refmodule{pickle}\refstmodindex{pickle}\ and
\refmodule{cPickle}\refbimodindex{cPickle}\ modules.  The
\refmodule{copy}\refstmodindex{copy}\ module is likely to use this in the
future as well.  It provides configuration information about object
constructors which are not classes.  Such constructors may be factory
functions or class instances.


\begin{funcdesc}{constructor}{object}
  Declares \var{object} to be a valid constructor.  If \var{object} is
  not callable (and hence not valid as a constructor), raises
  \exception{TypeError}.
\end{funcdesc}

\begin{funcdesc}{pickle}{type, function\optional{, constructor}}
  Declares that \var{function} should be used as a ``reduction''
  function for objects of type \var{type}; \var{type} must not be a
  ``classic'' class object.  (Classic classes are handled differently;
  see the documentation for the \refmodule{pickle} module for
  details.)  \var{function} should return either a string or a tuple
  containing two or three elements.

  The optional \var{constructor} parameter, if provided, is a
  callable object which can be used to reconstruct the object when
  called with the tuple of arguments returned by \var{function} at
  pickling time.  \exception{TypeError} will be raised if
  \var{object} is a class or \var{constructor} is not callable.

  See the \refmodule{pickle} module for more
  details on the interface expected of \var{function} and
  \var{constructor}.
\end{funcdesc}
		% really copy_reg
\section{Standard Module \module{shelve}}
\label{module-shelve}
\stmodindex{shelve}

A ``shelf'' is a persistent, dictionary-like object.  The difference
with ``dbm'' databases is that the values (not the keys!) in a shelf
can be essentially arbitrary Python objects --- anything that the
\code{pickle} module can handle.  This includes most class instances,
recursive data types, and objects containing lots of shared
sub-objects.  The keys are ordinary strings.
\refstmodindex{pickle}

To summarize the interface (\code{key} is a string, \code{data} is an
arbitrary object):

\begin{verbatim}
import shelve

d = shelve.open(filename) # open, with (g)dbm filename -- no suffix

d[key] = data   # store data at key (overwrites old data if
                # using an existing key)
data = d[key]   # retrieve data at key (raise KeyError if no
                # such key)
del d[key]      # delete data stored at key (raises KeyError
                # if no such key)
flag = d.has_key(key)   # true if the key exists
list = d.keys() # a list of all existing keys (slow!)

d.close()       # close it
\end{verbatim}
%
Restrictions:

\begin{itemize}

\item
The choice of which database package will be used (e.g. \code{dbm} or
\code{gdbm})
depends on which interface is available.  Therefore it isn't safe to
open the database directly using \code{dbm}.  The database is also
(unfortunately) subject to the limitations of \code{dbm}, if it is used ---
this means that (the pickled representation of) the objects stored in
the database should be fairly small, and in rare cases key collisions
may cause the database to refuse updates.
\refbimodindex{dbm}
\refbimodindex{gdbm}

\item
Dependent on the implementation, closing a persistent dictionary may
or may not be necessary to flush changes to disk.

\item
The \code{shelve} module does not support \emph{concurrent} read/write
access to shelved objects.  (Multiple simultaneous read accesses are
safe.)  When a program has a shelf open for writing, no other program
should have it open for reading or writing.  \UNIX{} file locking can
be used to solve this, but this differs across \UNIX{} versions and
requires knowledge about the database implementation used.

\end{itemize}

\section{\module{copy} ---
         Shallow and deep copy operations}
\declaremodule{standard}{copy}

\modulesynopsis{Shallow and deep copy operations.}


This module provides generic (shallow and deep) copying operations.
\withsubitem{(in copy)}{\ttindex{copy()}\ttindex{deepcopy()}}

Interface summary:

\begin{verbatim}
import copy

x = copy.copy(y)        # make a shallow copy of y
x = copy.deepcopy(y)    # make a deep copy of y
\end{verbatim}
%
For module specific errors, \exception{copy.error} is raised.

The difference between shallow and deep copying is only relevant for
compound objects (objects that contain other objects, like lists or
class instances):

\begin{itemize}

\item
A \emph{shallow copy} constructs a new compound object and then (to the
extent possible) inserts \emph{references} into it to the objects found
in the original.

\item
A \emph{deep copy} constructs a new compound object and then,
recursively, inserts \emph{copies} into it of the objects found in the
original.

\end{itemize}

Two problems often exist with deep copy operations that don't exist
with shallow copy operations:

\begin{itemize}

\item
Recursive objects (compound objects that, directly or indirectly,
contain a reference to themselves) may cause a recursive loop.

\item
Because deep copy copies \emph{everything} it may copy too much,
e.g., administrative data structures that should be shared even
between copies.

\end{itemize}

The \function{deepcopy()} function avoids these problems by:

\begin{itemize}

\item
keeping a ``memo'' dictionary of objects already copied during the current
copying pass; and

\item
letting user-defined classes override the copying operation or the
set of components copied.

\end{itemize}

This version does not copy types like module, class, function, method,
nor stack trace, stack frame, nor file, socket, window, nor array, nor
any similar types.

Classes can use the same interfaces to control copying that they use
to control pickling: they can define methods called
\method{__getinitargs__()}, \method{__getstate__()} and
\method{__setstate__()}.  See the description of module
\refmodule{pickle}\refstmodindex{pickle} for information on these
methods.  The \module{copy} module does not use the \module{copy_reg}
registration module.
\withsubitem{(copy protocol)}{\ttindex{__getinitargs__()}
  \ttindex{__getstate__()}\ttindex{__setstate__()}}

In order for a class to define its own copy implementation, it can
define special methods \method{__copy__()} and
\method{__deepcopy__()}.  The former is called to implement the
shallow copy operation; no additional arguments are passed.  The
latter is called to implement the deep copy operation; it is passed
one argument, the memo dictionary.  If the \method{__deepcopy__()}
implementation needs to make a deep copy of a component, it should
call the \function{deepcopy()} function with the component as first
argument and the memo dictionary as second argument.
\withsubitem{(copy protocol)}{\ttindex{__copy__()}\ttindex{__deepcopy__()}}

\begin{seealso}
\seemodule{pickle}{Discussion of the special disciplines used to
support object state retrieval and restoration.}
\end{seealso}

\section{\module{marshal} ---
         Internal Python object serialization}

\declaremodule{builtin}{marshal}
\modulesynopsis{Convert Python objects to streams of bytes and back
                (with different constraints).}


This module contains functions that can read and write Python
values in a binary format.  The format is specific to Python, but
independent of machine architecture issues (e.g., you can write a
Python value to a file on a PC, transport the file to a Sun, and read
it back there).  Details of the format are undocumented on purpose;
it may change between Python versions (although it rarely
does).\footnote{The name of this module stems from a bit of
  terminology used by the designers of Modula-3 (amongst others), who
  use the term ``marshalling'' for shipping of data around in a
  self-contained form. Strictly speaking, ``to marshal'' means to
  convert some data from internal to external form (in an RPC buffer for
  instance) and ``unmarshalling'' for the reverse process.}

This is not a general ``persistence'' module.  For general persistence
and transfer of Python objects through RPC calls, see the modules
\refmodule{pickle} and \refmodule{shelve}.  The \module{marshal} module exists
mainly to support reading and writing the ``pseudo-compiled'' code for
Python modules of \file{.pyc} files.  Therefore, the Python
maintainers reserve the right to modify the marshal format in backward
incompatible ways should the need arise.  If you're serializing and
de-serializing Python objects, use the \module{pickle} module instead.  
\refstmodindex{pickle}
\refstmodindex{shelve}
\obindex{code}

\begin{notice}[warning]
The \module{marshal} module is not intended to be secure against
erroneous or maliciously constructed data.  Never unmarshal data
received from an untrusted or unauthenticated source.
\end{notice}

Not all Python object types are supported; in general, only objects
whose value is independent from a particular invocation of Python can
be written and read by this module.  The following types are supported:
\code{None}, integers, long integers, floating point numbers,
strings, Unicode objects, tuples, lists, dictionaries, and code
objects, where it should be understood that tuples, lists and
dictionaries are only supported as long as the values contained
therein are themselves supported; and recursive lists and dictionaries
should not be written (they will cause infinite loops).

\strong{Caveat:} On machines where C's \code{long int} type has more than
32 bits (such as the DEC Alpha), it is possible to create plain Python
integers that are longer than 32 bits.
If such an integer is marshaled and read back in on a machine where
C's \code{long int} type has only 32 bits, a Python long integer object
is returned instead.  While of a different type, the numeric value is
the same.  (This behavior is new in Python 2.2.  In earlier versions,
all but the least-significant 32 bits of the value were lost, and a
warning message was printed.)

There are functions that read/write files as well as functions
operating on strings.

The module defines these functions:

\begin{funcdesc}{dump}{value, file\optional{, version}}
  Write the value on the open file.  The value must be a supported
  type.  The file must be an open file object such as
  \code{sys.stdout} or returned by \function{open()} or
  \function{posix.popen()}.  It must be opened in binary mode
  (\code{'wb'} or \code{'w+b'}).

  If the value has (or contains an object that has) an unsupported type,
  a \exception{ValueError} exception is raised --- but garbage data
  will also be written to the file.  The object will not be properly
  read back by \function{load()}.

  \versionadded[The \var{version} argument indicates the data
  format that \code{dump} should use (see below)]{2.4}
\end{funcdesc}

\begin{funcdesc}{load}{file}
  Read one value from the open file and return it.  If no valid value
  is read, raise \exception{EOFError}, \exception{ValueError} or
  \exception{TypeError}.  The file must be an open file object opened
  in binary mode (\code{'rb'} or \code{'r+b'}).

  \warning{If an object containing an unsupported type was
  marshalled with \function{dump()}, \function{load()} will substitute
  \code{None} for the unmarshallable type.}
\end{funcdesc}

\begin{funcdesc}{dumps}{value\optional{, version}}
  Return the string that would be written to a file by
  \code{dump(\var{value}, \var{file})}.  The value must be a supported
  type.  Raise a \exception{ValueError} exception if value has (or
  contains an object that has) an unsupported type.

  \versionadded[The \var{version} argument indicates the data
  format that \code{dumps} should use (see below)]{2.4}
\end{funcdesc}

\begin{funcdesc}{loads}{string}
  Convert the string to a value.  If no valid value is found, raise
  \exception{EOFError}, \exception{ValueError} or
  \exception{TypeError}.  Extra characters in the string are ignored.
\end{funcdesc}

In addition, the following constants are defined:

\begin{datadesc}{version}
  Indicates the format that the module uses. Version 0 is the
  historical format, version 1 (added in Python 2.4) shares
  interned strings. The current version is 1.

  \versionadded{2.4}
\end{datadesc}
\section{\module{imp} ---
         Access the \keyword{import} internals}

\declaremodule{builtin}{imp}
\modulesynopsis{Access the implementation of the \keyword{import} statement.}


This\stindex{import} module provides an interface to the mechanisms
used to implement the \keyword{import} statement.  It defines the
following constants and functions:


\begin{funcdesc}{get_magic}{}
\indexii{file}{byte-code}
Return the magic string value used to recognize byte-compiled code
files (\file{.pyc} files).  (This value may be different for each
Python version.)
\end{funcdesc}

\begin{funcdesc}{get_suffixes}{}
Return a list of triples, each describing a particular type of module.
Each triple has the form \code{(\var{suffix}, \var{mode},
\var{type})}, where \var{suffix} is a string to be appended to the
module name to form the filename to search for, \var{mode} is the mode
string to pass to the built-in \function{open()} function to open the
file (this can be \code{'r'} for text files or \code{'rb'} for binary
files), and \var{type} is the file type, which has one of the values
\constant{PY_SOURCE}, \constant{PY_COMPILED}, or
\constant{C_EXTENSION}, described below.
\end{funcdesc}

\begin{funcdesc}{find_module}{name\optional{, path}}
Try to find the module \var{name} on the search path \var{path}.  If
\var{path} is a list of directory names, each directory is searched
for files with any of the suffixes returned by \function{get_suffixes()}
above.  Invalid names in the list are silently ignored (but all list
items must be strings).  If \var{path} is omitted or \code{None}, the
list of directory names given by \code{sys.path} is searched, but
first it searches a few special places: it tries to find a built-in
module with the given name (\constant{C_BUILTIN}), then a frozen module
(\constant{PY_FROZEN}), and on some systems some other places are looked
in as well (on the Mac, it looks for a resource (\constant{PY_RESOURCE});
on Windows, it looks in the registry which may point to a specific
file).

If search is successful, the return value is a triple
\code{(\var{file}, \var{pathname}, \var{description})} where
\var{file} is an open file object positioned at the beginning,
\var{pathname} is the pathname of the
file found, and \var{description} is a triple as contained in the list
returned by \function{get_suffixes()} describing the kind of module found.
If the module does not live in a file, the returned \var{file} is
\code{None}, \var{filename} is the empty string, and the
\var{description} tuple contains empty strings for its suffix and
mode; the module type is as indicate in parentheses above.  If the
search is unsuccessful, \exception{ImportError} is raised.  Other
exceptions indicate problems with the arguments or environment.

This function does not handle hierarchical module names (names
containing dots).  In order to find \var{P}.\var{M}, that is, submodule
\var{M} of package \var{P}, use \function{find_module()} and
\function{load_module()} to find and load package \var{P}, and then use
\function{find_module()} with the \var{path} argument set to
\code{\var{P}.__path__}.  When \var{P} itself has a dotted name, apply
this recipe recursively.
\end{funcdesc}

\begin{funcdesc}{load_module}{name, file, filename, description}
Load a module that was previously found by \function{find_module()} (or by
an otherwise conducted search yielding compatible results).  This
function does more than importing the module: if the module was
already imported, it is equivalent to a
\function{reload()}\bifuncindex{reload}!  The \var{name} argument
indicates the full module name (including the package name, if this is
a submodule of a package).  The \var{file} argument is an open file,
and \var{filename} is the corresponding file name; these can be
\code{None} and \code{''}, respectively, when the module is not being
loaded from a file.  The \var{description} argument is a tuple, as
would be returned by \function{get_suffixes()}, describing what kind
of module must be loaded.

If the load is successful, the return value is the module object;
otherwise, an exception (usually \exception{ImportError}) is raised.

\strong{Important:} the caller is responsible for closing the
\var{file} argument, if it was not \code{None}, even when an exception
is raised.  This is best done using a \keyword{try}
... \keyword{finally} statement.
\end{funcdesc}

\begin{funcdesc}{new_module}{name}
Return a new empty module object called \var{name}.  This object is
\emph{not} inserted in \code{sys.modules}.
\end{funcdesc}

\begin{funcdesc}{lock_held}{}
Return \code{True} if the import lock is currently held, else \code{False}.
On platforms without threads, always return \code{False}.

On platforms with threads, a thread executing an import holds an internal
lock until the import is complete.
This lock blocks other threads from doing an import until the original
import completes, which in turn prevents other threads from seeing
incomplete module objects constructed by the original thread while in
the process of completing its import (and the imports, if any,
triggered by that).
\end{funcdesc}

\begin{funcdesc}{set_frozenmodules}{seq_of_tuples}
Set the global list of frozen modules.  \var{seq_of_tuples} is a sequence
of tuples of length 3: (\var{modulename}, \var{codedata}, \var{ispkg})
\var{modulename} is the name of the frozen module (may contain dots).
\var{codedata} is a marshalled code object. \var{ispkg} is a boolean
indicating whether the module is a package.
\versionadded{2.3}
\end{funcdesc}

\begin{funcdesc}{get_frozenmodules}{}
Return the global list of frozen modules as a tuple of tuples. See
\function{set_frozenmodules()} for a description of its contents.
\versionadded{2.3}
\end{funcdesc}

The following constants with integer values, defined in this module,
are used to indicate the search result of \function{find_module()}.

\begin{datadesc}{PY_SOURCE}
The module was found as a source file.
\end{datadesc}

\begin{datadesc}{PY_COMPILED}
The module was found as a compiled code object file.
\end{datadesc}

\begin{datadesc}{C_EXTENSION}
The module was found as dynamically loadable shared library.
\end{datadesc}

\begin{datadesc}{PY_RESOURCE}
The module was found as a Macintosh resource.  This value can only be
returned on a Macintosh.
\end{datadesc}

\begin{datadesc}{PKG_DIRECTORY}
The module was found as a package directory.
\end{datadesc}

\begin{datadesc}{C_BUILTIN}
The module was found as a built-in module.
\end{datadesc}

\begin{datadesc}{PY_FROZEN}
The module was found as a frozen module (see \function{init_frozen()}).
\end{datadesc}

The following constant and functions are obsolete; their functionality
is available through \function{find_module()} or \function{load_module()}.
They are kept around for backward compatibility:

\begin{datadesc}{SEARCH_ERROR}
Unused.
\end{datadesc}

\begin{funcdesc}{init_builtin}{name}
Initialize the built-in module called \var{name} and return its module
object.  If the module was already initialized, it will be initialized
\emph{again}.  A few modules cannot be initialized twice --- attempting
to initialize these again will raise an \exception{ImportError}
exception.  If there is no
built-in module called \var{name}, \code{None} is returned.
\end{funcdesc}

\begin{funcdesc}{init_frozen}{name}
Initialize the frozen module called \var{name} and return its module
object.  If the module was already initialized, it will be initialized
\emph{again}.  If there is no frozen module called \var{name},
\code{None} is returned.  (Frozen modules are modules written in
Python whose compiled byte-code object is incorporated into a
custom-built Python interpreter by Python's \program{freeze} utility.
See \file{Tools/freeze/} for now.)
\end{funcdesc}

\begin{funcdesc}{is_builtin}{name}
Return \code{1} if there is a built-in module called \var{name} which
can be initialized again.  Return \code{-1} if there is a built-in
module called \var{name} which cannot be initialized again (see
\function{init_builtin()}).  Return \code{0} if there is no built-in
module called \var{name}.
\end{funcdesc}

\begin{funcdesc}{is_frozen}{name}
Return \code{True} if there is a frozen module (see
\function{init_frozen()}) called \var{name}, or \code{False} if there is
no such module.
\end{funcdesc}

\begin{funcdesc}{load_compiled}{name, pathname, file}
\indexii{file}{byte-code}
Load and initialize a module implemented as a byte-compiled code file
and return its module object.  If the module was already initialized,
it will be initialized \emph{again}.  The \var{name} argument is used
to create or access a module object.  The \var{pathname} argument
points to the byte-compiled code file.  The \var{file}
argument is the byte-compiled code file, open for reading in binary
mode, from the beginning.
It must currently be a real file object, not a
user-defined class emulating a file.
\end{funcdesc}

\begin{funcdesc}{load_dynamic}{name, pathname\optional{, file}}
Load and initialize a module implemented as a dynamically loadable
shared library and return its module object.  If the module was
already initialized, it will be initialized \emph{again}.  Some modules
don't like that and may raise an exception.  The \var{pathname}
argument must point to the shared library.  The \var{name} argument is
used to construct the name of the initialization function: an external
C function called \samp{init\var{name}()} in the shared library is
called.  The optional \var{file} argument is ignored.  (Note: using
shared libraries is highly system dependent, and not all systems
support it.)
\end{funcdesc}

\begin{funcdesc}{load_source}{name, pathname, file}
Load and initialize a module implemented as a Python source file and
return its module object.  If the module was already initialized, it
will be initialized \emph{again}.  The \var{name} argument is used to
create or access a module object.  The \var{pathname} argument points
to the source file.  The \var{file} argument is the source
file, open for reading as text, from the beginning.
It must currently be a real file
object, not a user-defined class emulating a file.  Note that if a
properly matching byte-compiled file (with suffix \file{.pyc} or
\file{.pyo}) exists, it will be used instead of parsing the given
source file.
\end{funcdesc}


\subsection{Examples}
\label{examples-imp}

The following function emulates what was the standard import statement
up to Python 1.4 (no hierarchical module names).  (This
\emph{implementation} wouldn't work in that version, since
\function{find_module()} has been extended and
\function{load_module()} has been added in 1.4.)

\begin{verbatim}
import imp
import sys

def __import__(name, globals=None, locals=None, fromlist=None):
    # Fast path: see if the module has already been imported.
    try:
        return sys.modules[name]
    except KeyError:
        pass

    # If any of the following calls raises an exception,
    # there's a problem we can't handle -- let the caller handle it.

    fp, pathname, description = imp.find_module(name)
    
    try:
        return imp.load_module(name, fp, pathname, description)
    finally:
        # Since we may exit via an exception, close fp explicitly.
        if fp:
            fp.close()
\end{verbatim}

A more complete example that implements hierarchical module names and
includes a \function{reload()}\bifuncindex{reload} function can be
found in the module \module{knee}\refmodindex{knee}.  The
\module{knee} module can be found in \file{Demo/imputil/} in the
Python source distribution.

%\section{Built-in Module \sectcode{ni}}
\label{module-ni}
\bimodindex{ni}

\strong{Warning: This module is obsolete.}  As of Python 1.5a4,
package support (with different semantics for \code{__init__} and no
support for \code{__domain__} or\code{__}) is built in the
interpreter.  The ni module is retained only for backward
compatibility.

The \code{ni} module defines a new importing scheme, which supports
packages containing several Python modules.  To enable package
support, execute \code{import ni} before importing any packages.  Importing
this module automatically installs the relevant import hooks.  There
are no publicly-usable functions or variables in the \code{ni} module.

To create a package named \code{spam} containing sub-modules \code{ham}, \code{bacon} and
\code{eggs}, create a directory \file{spam} somewhere on Python's module search
path, as given in \code{sys.path}.  Then, create files called \file{ham.py}, \file{bacon.py} and
\file{eggs.py} inside \file{spam}.

To import module \code{ham} from package \code{spam} and use function
\code{hamneggs()} from that module, you can use any of the following
possibilities:

\bcode\begin{verbatim}
import spam.ham		# *not* "import spam" !!!
spam.ham.hamneggs()
\end{verbatim}\ecode
%
\bcode\begin{verbatim}
from spam import ham
ham.hamneggs()
\end{verbatim}\ecode
%
\bcode\begin{verbatim}
from spam.ham import hamneggs
hamneggs()
\end{verbatim}\ecode
%
\code{import spam} creates an
empty package named \code{spam} if one does not already exist, but it does
\emph{not} automatically import \code{spam}'s submodules.  
The only submodule that is guaranteed to be imported is
\code{spam.__init__}, if it exists; it would be in a file named
\file{__init__.py} in the \file{spam} directory.  Note that
\code{spam.__init__} is a submodule of package spam.  It can refer to
spam's namespace as \code{__} (two underscores):

\bcode\begin{verbatim}
__.spam_inited = 1		# Set a package-level variable
\end{verbatim}\ecode
%
Additional initialization code (setting up variables, importing other
submodules) can be performed in \file{spam/__init__.py}.

% libparser.tex
%
% Copyright 1995 Virginia Polytechnic Institute and State University
% and Fred L. Drake, Jr.  This copyright notice must be distributed on
% all copies, but this document otherwise may be distributed as part
% of the Python distribution.  No fee may be charged for this document
% in any representation, either on paper or electronically.  This
% restriction does not affect other elements in a distributed package
% in any way.
%

\section{Built-in Module \module{parser}}
\label{module-parser}
\bimodindex{parser}
\index{parsing!Python source code}

The \module{parser} module provides an interface to Python's internal
parser and byte-code compiler.  The primary purpose for this interface
is to allow Python code to edit the parse tree of a Python expression
and create executable code from this.  This is better than trying
to parse and modify an arbitrary Python code fragment as a string
because parsing is performed in a manner identical to the code
forming the application.  It is also faster.

The \module{parser} module was written and documented by Fred
L. Drake, Jr. (\email{fdrake@acm.org}).%
\index{Drake, Fred L., Jr.}

There are a few things to note about this module which are important
to making use of the data structures created.  This is not a tutorial
on editing the parse trees for Python code, but some examples of using
the \module{parser} module are presented.

Most importantly, a good understanding of the Python grammar processed
by the internal parser is required.  For full information on the
language syntax, refer to the \emph{Python Language Reference}.  The
parser itself is created from a grammar specification defined in the file
\file{Grammar/Grammar} in the standard Python distribution.  The parse
trees stored in the AST objects created by this module are the
actual output from the internal parser when created by the
\function{expr()} or \function{suite()} functions, described below.  The AST
objects created by \function{sequence2ast()} faithfully simulate those
structures.  Be aware that the values of the sequences which are
considered ``correct'' will vary from one version of Python to another
as the formal grammar for the language is revised.  However,
transporting code from one Python version to another as source text
will always allow correct parse trees to be created in the target
version, with the only restriction being that migrating to an older
version of the interpreter will not support more recent language
constructs.  The parse trees are not typically compatible from one
version to another, whereas source code has always been
forward-compatible.

Each element of the sequences returned by \function{ast2list()} or
\function{ast2tuple()} has a simple form.  Sequences representing
non-terminal elements in the grammar always have a length greater than
one.  The first element is an integer which identifies a production in
the grammar.  These integers are given symbolic names in the C header
file \file{Include/graminit.h} and the Python module
\module{symbol}.  Each additional element of the sequence represents
a component of the production as recognized in the input string: these
are always sequences which have the same form as the parent.  An
important aspect of this structure which should be noted is that
keywords used to identify the parent node type, such as the keyword
\keyword{if} in an \constant{if_stmt}, are included in the node tree without
any special treatment.  For example, the \keyword{if} keyword is
represented by the tuple \code{(1, 'if')}, where \code{1} is the
numeric value associated with all \code{NAME} tokens, including
variable and function names defined by the user.  In an alternate form
returned when line number information is requested, the same token
might be represented as \code{(1, 'if', 12)}, where the \code{12}
represents the line number at which the terminal symbol was found.

Terminal elements are represented in much the same way, but without
any child elements and the addition of the source text which was
identified.  The example of the \keyword{if} keyword above is
representative.  The various types of terminal symbols are defined in
the C header file \file{Include/token.h} and the Python module
\module{token}.

The AST objects are not required to support the functionality of this
module, but are provided for three purposes: to allow an application
to amortize the cost of processing complex parse trees, to provide a
parse tree representation which conserves memory space when compared
to the Python list or tuple representation, and to ease the creation
of additional modules in C which manipulate parse trees.  A simple
``wrapper'' class may be created in Python to hide the use of AST
objects.

The \module{parser} module defines functions for a few distinct
purposes.  The most important purposes are to create AST objects and
to convert AST objects to other representations such as parse trees
and compiled code objects, but there are also functions which serve to
query the type of parse tree represented by an AST object.


\subsection{Creating AST Objects}
\label{Creating ASTs}

AST objects may be created from source code or from a parse tree.
When creating an AST object from source, different functions are used
to create the \code{'eval'} and \code{'exec'} forms.

\begin{funcdesc}{expr}{string}
The \function{expr()} function parses the parameter \code{\var{string}}
as if it were an input to \samp{compile(\var{string}, 'eval')}.  If
the parse succeeds, an AST object is created to hold the internal
parse tree representation, otherwise an appropriate exception is
thrown.
\end{funcdesc}

\begin{funcdesc}{suite}{string}
The \function{suite()} function parses the parameter \code{\var{string}}
as if it were an input to \samp{compile(\var{string}, 'exec')}.  If
the parse succeeds, an AST object is created to hold the internal
parse tree representation, otherwise an appropriate exception is
thrown.
\end{funcdesc}

\begin{funcdesc}{sequence2ast}{sequence}
This function accepts a parse tree represented as a sequence and
builds an internal representation if possible.  If it can validate
that the tree conforms to the Python grammar and all nodes are valid
node types in the host version of Python, an AST object is created
from the internal representation and returned to the called.  If there
is a problem creating the internal representation, or if the tree
cannot be validated, a \exception{ParserError} exception is thrown.  An AST
object created this way should not be assumed to compile correctly;
normal exceptions thrown by compilation may still be initiated when
the AST object is passed to \function{compileast()}.  This may indicate
problems not related to syntax (such as a \exception{MemoryError}
exception), but may also be due to constructs such as the result of
parsing \code{del f(0)}, which escapes the Python parser but is
checked by the bytecode compiler.

Sequences representing terminal tokens may be represented as either
two-element lists of the form \code{(1, 'name')} or as three-element
lists of the form \code{(1, 'name', 56)}.  If the third element is
present, it is assumed to be a valid line number.  The line number
may be specified for any subset of the terminal symbols in the input
tree.
\end{funcdesc}

\begin{funcdesc}{tuple2ast}{sequence}
This is the same function as \function{sequence2ast()}.  This entry point
is maintained for backward compatibility.
\end{funcdesc}


\subsection{Converting AST Objects}
\label{Converting ASTs}

AST objects, regardless of the input used to create them, may be
converted to parse trees represented as list- or tuple- trees, or may
be compiled into executable code objects.  Parse trees may be
extracted with or without line numbering information.

\begin{funcdesc}{ast2list}{ast\optional{, line_info}}
This function accepts an AST object from the caller in
\code{\var{ast}} and returns a Python list representing the
equivelent parse tree.  The resulting list representation can be used
for inspection or the creation of a new parse tree in list form.  This
function does not fail so long as memory is available to build the
list representation.  If the parse tree will only be used for
inspection, \function{ast2tuple()} should be used instead to reduce memory
consumption and fragmentation.  When the list representation is
required, this function is significantly faster than retrieving a
tuple representation and converting that to nested lists.

If \code{\var{line_info}} is true, line number information will be
included for all terminal tokens as a third element of the list
representing the token.  Note that the line number provided specifies
the line on which the token \emph{ends}.  This information is
omitted if the flag is false or omitted.
\end{funcdesc}

\begin{funcdesc}{ast2tuple}{ast\optional{, line_info}}
This function accepts an AST object from the caller in
\code{\var{ast}} and returns a Python tuple representing the
equivelent parse tree.  Other than returning a tuple instead of a
list, this function is identical to \function{ast2list()}.

If \code{\var{line_info}} is true, line number information will be
included for all terminal tokens as a third element of the list
representing the token.  This information is omitted if the flag is
false or omitted.
\end{funcdesc}

\begin{funcdesc}{compileast}{ast\optional{, filename\code{ = '<ast>'}}}
The Python byte compiler can be invoked on an AST object to produce
code objects which can be used as part of an \code{exec} statement or
a call to the built-in \function{eval()}\bifuncindex{eval} function.
This function provides the interface to the compiler, passing the
internal parse tree from \code{\var{ast}} to the parser, using the
source file name specified by the \code{\var{filename}} parameter.
The default value supplied for \code{\var{filename}} indicates that
the source was an AST object.

Compiling an AST object may result in exceptions related to
compilation; an example would be a \exception{SyntaxError} caused by the
parse tree for \code{del f(0)}: this statement is considered legal
within the formal grammar for Python but is not a legal language
construct.  The \exception{SyntaxError} raised for this condition is
actually generated by the Python byte-compiler normally, which is why
it can be raised at this point by the \module{parser} module.  Most
causes of compilation failure can be diagnosed programmatically by
inspection of the parse tree.
\end{funcdesc}


\subsection{Queries on AST Objects}
\label{Querying ASTs}

Two functions are provided which allow an application to determine if
an AST was created as an expression or a suite.  Neither of these
functions can be used to determine if an AST was created from source
code via \function{expr()} or \function{suite()} or from a parse tree
via \function{sequence2ast()}.

\begin{funcdesc}{isexpr}{ast}
When \code{\var{ast}} represents an \code{'eval'} form, this function
returns true, otherwise it returns false.  This is useful, since code
objects normally cannot be queried for this information using existing
built-in functions.  Note that the code objects created by
\function{compileast()} cannot be queried like this either, and are
identical to those created by the built-in
\function{compile()}\bifuncindex{compile} function.
\end{funcdesc}


\begin{funcdesc}{issuite}{ast}
This function mirrors \function{isexpr()} in that it reports whether an
AST object represents an \code{'exec'} form, commonly known as a
``suite.''  It is not safe to assume that this function is equivelent
to \samp{not isexpr(\var{ast})}, as additional syntactic fragments may
be supported in the future.
\end{funcdesc}


\subsection{Exceptions and Error Handling}
\label{AST Errors}

The parser module defines a single exception, but may also pass other
built-in exceptions from other portions of the Python runtime
environment.  See each function for information about the exceptions
it can raise.

\begin{excdesc}{ParserError}
Exception raised when a failure occurs within the parser module.  This
is generally produced for validation failures rather than the built in
\exception{SyntaxError} thrown during normal parsing.
The exception argument is either a string describing the reason of the
failure or a tuple containing a sequence causing the failure from a parse
tree passed to \function{sequence2ast()} and an explanatory string.  Calls to
\function{sequence2ast()} need to be able to handle either type of exception,
while calls to other functions in the module will only need to be
aware of the simple string values.
\end{excdesc}

Note that the functions \function{compileast()}, \function{expr()}, and
\function{suite()} may throw exceptions which are normally thrown by the
parsing and compilation process.  These include the built in
exceptions \exception{MemoryError}, \exception{OverflowError},
\exception{SyntaxError}, and \exception{SystemError}.  In these cases, these
exceptions carry all the meaning normally associated with them.  Refer
to the descriptions of each function for detailed information.


\subsection{AST Objects}
\label{AST Objects}

AST objects returned by \function{expr()}, \function{suite()} and
\function{sequence2ast()} have no methods of their own.
Some of the functions defined which accept an AST object as their
first argument may change to object methods in the future.

Ordered and equality comparisons are supported between AST objects.
Pickling of AST objects (using the \module{pickle} module) is also
supported.

\begin{datadesc}{ASTType}
The type of the objects returned by \function{expr()},
\function{suite()} and \function{sequence2ast()}.
\end{datadesc}


\subsection{Examples}
\nodename{AST Examples}

The parser modules allows operations to be performed on the parse tree
of Python source code before the bytecode is generated, and provides
for inspection of the parse tree for information gathering purposes.
Two examples are presented.  The simple example demonstrates emulation
of the \function{compile()}\bifuncindex{compile} built-in function and
the complex example shows the use of a parse tree for information
discovery.

\subsubsection{Emulation of \function{compile()}}

While many useful operations may take place between parsing and
bytecode generation, the simplest operation is to do nothing.  For
this purpose, using the \module{parser} module to produce an
intermediate data structure is equivelent to the code

\begin{verbatim}
>>> code = compile('a + 5', 'eval')
>>> a = 5
>>> eval(code)
10
\end{verbatim}

The equivelent operation using the \module{parser} module is somewhat
longer, and allows the intermediate internal parse tree to be retained
as an AST object:

\begin{verbatim}
>>> import parser
>>> ast = parser.expr('a + 5')
>>> code = parser.compileast(ast)
>>> a = 5
>>> eval(code)
10
\end{verbatim}

An application which needs both AST and code objects can package this
code into readily available functions:

\begin{verbatim}
import parser

def load_suite(source_string):
    ast = parser.suite(source_string)
    code = parser.compileast(ast)
    return ast, code

def load_expression(source_string):
    ast = parser.expr(source_string)
    code = parser.compileast(ast)
    return ast, code
\end{verbatim}

\subsubsection{Information Discovery}

Some applications benefit from direct access to the parse tree.  The
remainder of this section demonstrates how the parse tree provides
access to module documentation defined in docstrings without requiring
that the code being examined be loaded into a running interpreter via
\keyword{import}.  This can be very useful for performing analyses of
untrusted code.

Generally, the example will demonstrate how the parse tree may be
traversed to distill interesting information.  Two functions and a set
of classes are developed which provide programmatic access to high
level function and class definitions provided by a module.  The
classes extract information from the parse tree and provide access to
the information at a useful semantic level, one function provides a
simple low-level pattern matching capability, and the other function
defines a high-level interface to the classes by handling file
operations on behalf of the caller.  All source files mentioned here
which are not part of the Python installation are located in the
\file{Demo/parser/} directory of the distribution.

The dynamic nature of Python allows the programmer a great deal of
flexibility, but most modules need only a limited measure of this when
defining classes, functions, and methods.  In this example, the only
definitions that will be considered are those which are defined in the
top level of their context, e.g., a function defined by a \keyword{def}
statement at column zero of a module, but not a function defined
within a branch of an \code{if} ... \code{else} construct, though
there are some good reasons for doing so in some situations.  Nesting
of definitions will be handled by the code developed in the example.

To construct the upper-level extraction methods, we need to know what
the parse tree structure looks like and how much of it we actually
need to be concerned about.  Python uses a moderately deep parse tree
so there are a large number of intermediate nodes.  It is important to
read and understand the formal grammar used by Python.  This is
specified in the file \file{Grammar/Grammar} in the distribution.
Consider the simplest case of interest when searching for docstrings:
a module consisting of a docstring and nothing else.  (See file
\file{docstring.py}.)

\begin{verbatim}
"""Some documentation.
"""
\end{verbatim}

Using the interpreter to take a look at the parse tree, we find a
bewildering mass of numbers and parentheses, with the documentation
buried deep in nested tuples.

\begin{verbatim}
>>> import parser
>>> import pprint
>>> ast = parser.suite(open('docstring.py').read())
>>> tup = parser.ast2tuple(ast)
>>> pprint.pprint(tup)
(257,
 (264,
  (265,
   (266,
    (267,
     (307,
      (287,
       (288,
        (289,
         (290,
          (292,
           (293,
            (294,
             (295,
              (296,
               (297,
                (298,
                 (299,
                  (300, (3, '"""Some documentation.\012"""'))))))))))))))))),
   (4, ''))),
 (4, ''),
 (0, ''))
\end{verbatim}

The numbers at the first element of each node in the tree are the node
types; they map directly to terminal and non-terminal symbols in the
grammar.  Unfortunately, they are represented as integers in the
internal representation, and the Python structures generated do not
change that.  However, the \module{symbol} and \module{token} modules
provide symbolic names for the node types and dictionaries which map
from the integers to the symbolic names for the node types.

In the output presented above, the outermost tuple contains four
elements: the integer \code{257} and three additional tuples.  Node
type \code{257} has the symbolic name \constant{file_input}.  Each of
these inner tuples contains an integer as the first element; these
integers, \code{264}, \code{4}, and \code{0}, represent the node types
\constant{stmt}, \constant{NEWLINE}, and \constant{ENDMARKER},
respectively.
Note that these values may change depending on the version of Python
you are using; consult \file{symbol.py} and \file{token.py} for
details of the mapping.  It should be fairly clear that the outermost
node is related primarily to the input source rather than the contents
of the file, and may be disregarded for the moment.  The \constant{stmt}
node is much more interesting.  In particular, all docstrings are
found in subtrees which are formed exactly as this node is formed,
with the only difference being the string itself.  The association
between the docstring in a similar tree and the defined entity (class,
function, or module) which it describes is given by the position of
the docstring subtree within the tree defining the described
structure.

By replacing the actual docstring with something to signify a variable
component of the tree, we allow a simple pattern matching approach to
check any given subtree for equivelence to the general pattern for
docstrings.  Since the example demonstrates information extraction, we
can safely require that the tree be in tuple form rather than list
form, allowing a simple variable representation to be
\code{['variable_name']}.  A simple recursive function can implement
the pattern matching, returning a boolean and a dictionary of variable
name to value mappings.  (See file \file{example.py}.)

\begin{verbatim}
from types import ListType, TupleType

def match(pattern, data, vars=None):
    if vars is None:
        vars = {}
    if type(pattern) is ListType:
        vars[pattern[0]] = data
        return 1, vars
    if type(pattern) is not TupleType:
        return (pattern == data), vars
    if len(data) != len(pattern):
        return 0, vars
    for pattern, data in map(None, pattern, data):
        same, vars = match(pattern, data, vars)
        if not same:
            break
    return same, vars
\end{verbatim}

Using this simple representation for syntactic variables and the symbolic
node types, the pattern for the candidate docstring subtrees becomes
fairly readable.  (See file \file{example.py}.)

\begin{verbatim}
import symbol
import token

DOCSTRING_STMT_PATTERN = (
    symbol.stmt,
    (symbol.simple_stmt,
     (symbol.small_stmt,
      (symbol.expr_stmt,
       (symbol.testlist,
        (symbol.test,
         (symbol.and_test,
          (symbol.not_test,
           (symbol.comparison,
            (symbol.expr,
             (symbol.xor_expr,
              (symbol.and_expr,
               (symbol.shift_expr,
                (symbol.arith_expr,
                 (symbol.term,
                  (symbol.factor,
                   (symbol.power,
                    (symbol.atom,
                     (token.STRING, ['docstring'])
                     )))))))))))))))),
     (token.NEWLINE, '')
     ))
\end{verbatim}

Using the \function{match()} function with this pattern, extracting the
module docstring from the parse tree created previously is easy:

\begin{verbatim}
>>> found, vars = match(DOCSTRING_STMT_PATTERN, tup[1])
>>> found
1
>>> vars
{'docstring': '"""Some documentation.\012"""'}
\end{verbatim}

Once specific data can be extracted from a location where it is
expected, the question of where information can be expected
needs to be answered.  When dealing with docstrings, the answer is
fairly simple: the docstring is the first \constant{stmt} node in a code
block (\constant{file_input} or \constant{suite} node types).  A module
consists of a single \constant{file_input} node, and class and function
definitions each contain exactly one \constant{suite} node.  Classes and
functions are readily identified as subtrees of code block nodes which
start with \code{(stmt, (compound_stmt, (classdef, ...} or
\code{(stmt, (compound_stmt, (funcdef, ...}.  Note that these subtrees
cannot be matched by \function{match()} since it does not support multiple
sibling nodes to match without regard to number.  A more elaborate
matching function could be used to overcome this limitation, but this
is sufficient for the example.

Given the ability to determine whether a statement might be a
docstring and extract the actual string from the statement, some work
needs to be performed to walk the parse tree for an entire module and
extract information about the names defined in each context of the
module and associate any docstrings with the names.  The code to
perform this work is not complicated, but bears some explanation.

The public interface to the classes is straightforward and should
probably be somewhat more flexible.  Each ``major'' block of the
module is described by an object providing several methods for inquiry
and a constructor which accepts at least the subtree of the complete
parse tree which it represents.  The \class{ModuleInfo} constructor
accepts an optional \var{name} parameter since it cannot
otherwise determine the name of the module.

The public classes include \class{ClassInfo}, \class{FunctionInfo},
and \class{ModuleInfo}.  All objects provide the
methods \method{get_name()}, \method{get_docstring()},
\method{get_class_names()}, and \method{get_class_info()}.  The
\class{ClassInfo} objects support \method{get_method_names()} and
\method{get_method_info()} while the other classes provide
\method{get_function_names()} and \method{get_function_info()}.

Within each of the forms of code block that the public classes
represent, most of the required information is in the same form and is
accessed in the same way, with classes having the distinction that
functions defined at the top level are referred to as ``methods.''
Since the difference in nomenclature reflects a real semantic
distinction from functions defined outside of a class, the
implementation needs to maintain the distinction.
Hence, most of the functionality of the public classes can be
implemented in a common base class, \class{SuiteInfoBase}, with the
accessors for function and method information provided elsewhere.
Note that there is only one class which represents function and method
information; this parallels the use of the \keyword{def} statement to
define both types of elements.

Most of the accessor functions are declared in \class{SuiteInfoBase}
and do not need to be overriden by subclasses.  More importantly, the
extraction of most information from a parse tree is handled through a
method called by the \class{SuiteInfoBase} constructor.  The example
code for most of the classes is clear when read alongside the formal
grammar, but the method which recursively creates new information
objects requires further examination.  Here is the relevant part of
the \class{SuiteInfoBase} definition from \file{example.py}:

\begin{verbatim}
class SuiteInfoBase:
    _docstring = ''
    _name = ''

    def __init__(self, tree = None):
        self._class_info = {}
        self._function_info = {}
        if tree:
            self._extract_info(tree)

    def _extract_info(self, tree):
        # extract docstring
        if len(tree) == 2:
            found, vars = match(DOCSTRING_STMT_PATTERN[1], tree[1])
        else:
            found, vars = match(DOCSTRING_STMT_PATTERN, tree[3])
        if found:
            self._docstring = eval(vars['docstring'])
        # discover inner definitions
        for node in tree[1:]:
            found, vars = match(COMPOUND_STMT_PATTERN, node)
            if found:
                cstmt = vars['compound']
                if cstmt[0] == symbol.funcdef:
                    name = cstmt[2][1]
                    self._function_info[name] = FunctionInfo(cstmt)
                elif cstmt[0] == symbol.classdef:
                    name = cstmt[2][1]
                    self._class_info[name] = ClassInfo(cstmt)
\end{verbatim}

After initializing some internal state, the constructor calls the
\method{_extract_info()} method.  This method performs the bulk of the
information extraction which takes place in the entire example.  The
extraction has two distinct phases: the location of the docstring for
the parse tree passed in, and the discovery of additional definitions
within the code block represented by the parse tree.

The initial \keyword{if} test determines whether the nested suite is of
the ``short form'' or the ``long form.''  The short form is used when
the code block is on the same line as the definition of the code
block, as in

\begin{verbatim}
def square(x): "Square an argument."; return x ** 2
\end{verbatim}

while the long form uses an indented block and allows nested
definitions:

\begin{verbatim}
def make_power(exp):
    "Make a function that raises an argument to the exponent `exp'."
    def raiser(x, y=exp):
        return x ** y
    return raiser
\end{verbatim}

When the short form is used, the code block may contain a docstring as
the first, and possibly only, \constant{small_stmt} element.  The
extraction of such a docstring is slightly different and requires only
a portion of the complete pattern used in the more common case.  As
implemented, the docstring will only be found if there is only
one \constant{small_stmt} node in the \constant{simple_stmt} node.
Since most functions and methods which use the short form do not
provide a docstring, this may be considered sufficient.  The
extraction of the docstring proceeds using the \function{match()} function
as described above, and the value of the docstring is stored as an
attribute of the \class{SuiteInfoBase} object.

After docstring extraction, a simple definition discovery
algorithm operates on the \constant{stmt} nodes of the
\constant{suite} node.  The special case of the short form is not
tested; since there are no \constant{stmt} nodes in the short form,
the algorithm will silently skip the single \constant{simple_stmt}
node and correctly not discover any nested definitions.

Each statement in the code block is categorized as
a class definition, function or method definition, or
something else.  For the definition statements, the name of the
element defined is extracted and a representation object
appropriate to the definition is created with the defining subtree
passed as an argument to the constructor.  The repesentation objects
are stored in instance variables and may be retrieved by name using
the appropriate accessor methods.

The public classes provide any accessors required which are more
specific than those provided by the \class{SuiteInfoBase} class, but
the real extraction algorithm remains common to all forms of code
blocks.  A high-level function can be used to extract the complete set
of information from a source file.  (See file \file{example.py}.)

\begin{verbatim}
def get_docs(fileName):
    source = open(fileName).read()
    import os
    basename = os.path.basename(os.path.splitext(fileName)[0])
    import parser
    ast = parser.suite(source)
    tup = parser.ast2tuple(ast)
    return ModuleInfo(tup, basename)
\end{verbatim}

This provides an easy-to-use interface to the documentation of a
module.  If information is required which is not extracted by the code
of this example, the code may be extended at clearly defined points to
provide additional capabilities.

\begin{seealso}

\seemodule{symbol}{useful constants representing internal nodes of the
parse tree}

\seemodule{token}{useful constants representing leaf nodes of the
parse tree and functions for testing node values}

\end{seealso}

\section{\module{symbol} ---
         Constants used with Python parse trees}

\declaremodule{standard}{symbol}
\modulesynopsis{Constants representing internal nodes of the parse tree.}
\sectionauthor{Fred L. Drake, Jr.}{fdrake@acm.org}


This module provides constants which represent the numeric values of
internal nodes of the parse tree.  Unlike most Python constants, these
use lower-case names.  Refer to the file \file{Grammar/Grammar} in the
Python distribution for the definitions of the names in the context of
the language grammar.  The specific numeric values which the names map
to may change between Python versions.

This module also provides one additional data object:


\begin{datadesc}{sym_name}
  Dictionary mapping the numeric values of the constants defined in
  this module back to name strings, allowing more human-readable
  representation of parse trees to be generated.
\end{datadesc}


\begin{seealso}
  \seemodule{parser}{The second example for the \refmodule{parser}
                     module shows how to use the \module{symbol}
                     module.}
\end{seealso}

\section{\module{token} ---
         Constants used with Python parse trees}

\declaremodule{standard}{token}
\modulesynopsis{Constants representing terminal nodes of the parse tree.}
\sectionauthor{Fred L. Drake, Jr.}{fdrake@acm.org}


This module provides constants which represent the numeric values of
leaf nodes of the parse tree (terminal tokens).  Refer to the file
\file{Grammar/Grammar} in the Python distribution for the definitions
of the names in the context of the language grammar.  The specific
numeric values which the names map to may change between Python
versions.

This module also provides one data object and some functions.  The
functions mirror definitions in the Python C header files.



\begin{datadesc}{tok_name}
Dictionary mapping the numeric values of the constants defined in this
module back to name strings, allowing more human-readable
representation of parse trees to be generated.
\end{datadesc}

\begin{funcdesc}{ISTERMINAL}{x}
Return true for terminal token values.
\end{funcdesc}

\begin{funcdesc}{ISNONTERMINAL}{x}
Return true for non-terminal token values.
\end{funcdesc}

\begin{funcdesc}{ISEOF}{x}
Return true if \var{x} is the marker indicating the end of input.
\end{funcdesc}


\begin{seealso}
  \seemodule{parser}{The second example for the \refmodule{parser}
                     module shows how to use the \module{symbol}
                     module.}
\end{seealso}

\section{Standard Module \module{keyword}}
\label{module-keyword}
\stmodindex{keyword}

This module allows a Python program to determine if a string is a
keyword.  A single function is provided:

\begin{funcdesc}{iskeyword}{s}
Return true if \var{s} is a Python keyword.
\end{funcdesc}

\section{Standard Module \sectcode{code}}
\label{module-code}
\stmodindex{code}

The \code{code} module defines operations pertaining to Python code
objects.

The \code{code} module defines the following functions:

\renewcommand{\indexsubitem}{(in module code)}

\begin{funcdesc}{compile_command}{source\,
\optional{filename\optional{\, symbol}}}
This function is useful for programs that want to emulate Python's
interpreter main loop (a.k.a. the read-eval-print loop).  The tricky
part is to determine when the user has entered an incomplete command
that can be completed by entering more text (as opposed to a complete
command or a syntax error).  This function \emph{almost} always makes
the same decision as the real interpreter main loop.

Arguments: \var{source} is the source string; \var{filename} is the
optional filename from which source was read, defaulting to
\code{"<input>"}; and \var{symbol} is the optional grammar start
symbol, which should be either \code{"single"} (the default) or
\code{"eval"}.

Return a code object (the same as \code{compile(\var{source},
\var{filename}, \var{symbol})}) if the command is complete and valid;
return \code{None} if the command is incomplete; raise
\code{SyntaxError} if the command is a syntax error.


\end{funcdesc}

\section{\module{pprint} ---
         Data pretty printer}

\declaremodule{standard}{pprint}
\modulesynopsis{Data pretty printer.}
\moduleauthor{Fred L. Drake, Jr.}{fdrake@acm.org}
\sectionauthor{Fred L. Drake, Jr.}{fdrake@acm.org}


The \module{pprint} module provides a capability to ``pretty-print''
arbitrary Python data structures in a form which can be used as input
to the interpreter.  If the formatted structures include objects which
are not fundamental Python types, the representation may not be
loadable.  This may be the case if objects such as files, sockets,
classes, or instances are included, as well as many other builtin
objects which are not representable as Python constants.

The formatted representation keeps objects on a single line if it can,
and breaks them onto multiple lines if they don't fit within the
allowed width.  Construct \class{PrettyPrinter} objects explicitly if
you need to adjust the width constraint.

The \module{pprint} module defines one class:


% First the implementation class:

\begin{classdesc}{PrettyPrinter}{...}
Construct a \class{PrettyPrinter} instance.  This constructor
understands several keyword parameters.  An output stream may be set
using the \var{stream} keyword; the only method used on the stream
object is the file protocol's \method{write()} method.  If not
specified, the \class{PrettyPrinter} adopts \code{sys.stdout}.  Three
additional parameters may be used to control the formatted
representation.  The keywords are \var{indent}, \var{depth}, and
\var{width}.  The amount of indentation added for each recursive level
is specified by \var{indent}; the default is one.  Other values can
cause output to look a little odd, but can make nesting easier to
spot.  The number of levels which may be printed is controlled by
\var{depth}; if the data structure being printed is too deep, the next
contained level is replaced by \samp{...}.  By default, there is no
constraint on the depth of the objects being formatted.  The desired
output width is constrained using the \var{width} parameter; the
default is eighty characters.  If a structure cannot be formatted
within the constrained width, a best effort will be made.

\begin{verbatim}
>>> import pprint, sys
>>> stuff = sys.path[:]
>>> stuff.insert(0, stuff[:])
>>> pp = pprint.PrettyPrinter(indent=4)
>>> pp.pprint(stuff)
[   [   '',
        '/usr/local/lib/python1.5',
        '/usr/local/lib/python1.5/test',
        '/usr/local/lib/python1.5/sunos5',
        '/usr/local/lib/python1.5/sharedmodules',
        '/usr/local/lib/python1.5/tkinter'],
    '',
    '/usr/local/lib/python1.5',
    '/usr/local/lib/python1.5/test',
    '/usr/local/lib/python1.5/sunos5',
    '/usr/local/lib/python1.5/sharedmodules',
    '/usr/local/lib/python1.5/tkinter']
>>>
>>> import parser
>>> tup = parser.ast2tuple(
...     parser.suite(open('pprint.py').read()))[1][1][1]
>>> pp = pprint.PrettyPrinter(depth=6)
>>> pp.pprint(tup)
(266, (267, (307, (287, (288, (...))))))
\end{verbatim}
\end{classdesc}


% Now the derivative functions:

The \class{PrettyPrinter} class supports several derivative functions:

\begin{funcdesc}{pformat}{object\optional{, indent\optional{,
width\optional{, depth}}}}
Return the formatted representation of \var{object} as a string.  \var{indent},
\var{width} and \var{depth} will be passed to the \class{PrettyPrinter}
constructor as formatting parameters.
\versionchanged[The parameters \var{indent}, \var{width} and \var{depth}
were added]{2.4}
\end{funcdesc}

\begin{funcdesc}{pprint}{object\optional{, stream\optional{,
indent\optional{, width\optional{, depth}}}}}
Prints the formatted representation of \var{object} on \var{stream},
followed by a newline.  If \var{stream} is omitted, \code{sys.stdout}
is used.  This may be used in the interactive interpreter instead of a
\keyword{print} statement for inspecting values.    \var{indent},
\var{width} and \var{depth} will be passed to the \class{PrettyPrinter}
constructor as formatting parameters.

\begin{verbatim}
>>> stuff = sys.path[:]
>>> stuff.insert(0, stuff)
>>> pprint.pprint(stuff)
[<Recursion on list with id=869440>,
 '',
 '/usr/local/lib/python1.5',
 '/usr/local/lib/python1.5/test',
 '/usr/local/lib/python1.5/sunos5',
 '/usr/local/lib/python1.5/sharedmodules',
 '/usr/local/lib/python1.5/tkinter']
\end{verbatim}
\versionchanged[The parameters \var{indent}, \var{width} and \var{depth}
were added]{2.4}
\end{funcdesc}

\begin{funcdesc}{isreadable}{object}
Determine if the formatted representation of \var{object} is
``readable,'' or can be used to reconstruct the value using
\function{eval()}\bifuncindex{eval}.  This always returns false for
recursive objects.

\begin{verbatim}
>>> pprint.isreadable(stuff)
False
\end{verbatim}
\end{funcdesc}

\begin{funcdesc}{isrecursive}{object}
Determine if \var{object} requires a recursive representation.
\end{funcdesc}


One more support function is also defined:

\begin{funcdesc}{saferepr}{object}
Return a string representation of \var{object}, protected against
recursive data structures.  If the representation of \var{object}
exposes a recursive entry, the recursive reference will be represented
as \samp{<Recursion on \var{typename} with id=\var{number}>}.  The
representation is not otherwise formatted.
\end{funcdesc}

% This example is outside the {funcdesc} to keep it from running over
% the right margin.
\begin{verbatim}
>>> pprint.saferepr(stuff)
"[<Recursion on list with id=682968>, '', '/usr/local/lib/python1.5', '/usr/loca
l/lib/python1.5/test', '/usr/local/lib/python1.5/sunos5', '/usr/local/lib/python
1.5/sharedmodules', '/usr/local/lib/python1.5/tkinter']"
\end{verbatim}


\subsection{PrettyPrinter Objects}
\label{PrettyPrinter Objects}

\class{PrettyPrinter} instances have the following methods:


\begin{methoddesc}{pformat}{object}
Return the formatted representation of \var{object}.  This takes into
account the options passed to the \class{PrettyPrinter} constructor.
\end{methoddesc}

\begin{methoddesc}{pprint}{object}
Print the formatted representation of \var{object} on the configured
stream, followed by a newline.
\end{methoddesc}

The following methods provide the implementations for the
corresponding functions of the same names.  Using these methods on an
instance is slightly more efficient since new \class{PrettyPrinter}
objects don't need to be created.

\begin{methoddesc}{isreadable}{object}
Determine if the formatted representation of the object is
``readable,'' or can be used to reconstruct the value using
\function{eval()}\bifuncindex{eval}.  Note that this returns false for
recursive objects.  If the \var{depth} parameter of the
\class{PrettyPrinter} is set and the object is deeper than allowed,
this returns false.
\end{methoddesc}

\begin{methoddesc}{isrecursive}{object}
Determine if the object requires a recursive representation.
\end{methoddesc}

This method is provided as a hook to allow subclasses to modify the
way objects are converted to strings.  The default implementation uses
the internals of the \function{saferepr()} implementation.

\begin{methoddesc}{format}{object, context, maxlevels, level}
Returns three values: the formatted version of \var{object} as a
string, a flag indicating whether the result is readable, and a flag
indicating whether recursion was detected.  The first argument is the
object to be presented.  The second is a dictionary which contains the
\function{id()} of objects that are part of the current presentation
context (direct and indirect containers for \var{object} that are
affecting the presentation) as the keys; if an object needs to be
presented which is already represented in \var{context}, the third
return value should be true.  Recursive calls to the \method{format()}
method should add additional entries for containers to this
dictionary.  The third argument, \var{maxlevels}, gives the requested
limit to recursion; this will be \code{0} if there is no requested
limit.  This argument should be passed unmodified to recursive calls.
The fourth argument, \var{level}, gives the current level; recursive
calls should be passed a value less than that of the current call.
\versionadded{2.3}
\end{methoddesc}

\section{\module{py_compile} ---
         Compile Python source files}

% Documentation based on module docstrings, by Fred L. Drake, Jr.
% <fdrake@acm.org>

\declaremodule[pycompile]{standard}{py_compile}

\modulesynopsis{Compile Python source files to byte-code files.}


\indexii{file}{byte-code}
The \module{py_compile} module provides a single function to generate
a byte-code file from a source file.

Though not often needed, this function can be useful when installing
modules for shared use, especially if some of the users may not have
permission to write the byte-code cache files in the directory
containing the source code.


\begin{funcdesc}{compile}{file\optional{, cfile\optional{, dfile}}}
  Compile a source file to byte-code and write out the byte-code cache 
  file.  The source code is loaded from the file name \var{file}.  The 
  byte-code is written to \var{cfile}, which defaults to \var{file}
  \code{+} \code{'c'} (\code{'o'} if optimization is enabled in the
  current interpreter).  If \var{dfile} is specified, it is used as
  the name of the source file in error messages instead of \var{file}. 
\end{funcdesc}


\begin{seealso}
  \seemodule{compileall}{Utilities to compile all Python source files
                         in a directory tree.}
\end{seealso}
		% really py_compile
\section{\module{compileall} ---
         Byte-compile Python libraries}

\declaremodule{standard}{compileall}
\modulesynopsis{Tools for byte-compiling all Python source files in a
                directory tree.}


This module provides some utility functions to support installing
Python libraries.  These functions compile Python source files in a
directory tree, allowing users without permission to write to the
libraries to take advantage of cached byte-code files.

The source file for this module may also be used as a script to
compile Python sources in directories named on the command line or in
\code{sys.path}.


\begin{funcdesc}{compile_dir}{dir\optional{, maxlevels\optional{,
                              ddir\optional{, force\optional{, 
                              rx\optional{, quiet}}}}}}
  Recursively descend the directory tree named by \var{dir}, compiling
  all \file{.py} files along the way.  The \var{maxlevels} parameter
  is used to limit the depth of the recursion; it defaults to
  \code{10}.  If \var{ddir} is given, it is used as the base path from 
  which the filenames used in error messages will be generated.  If
  \var{force} is true, modules are re-compiled even if the timestamps
  are up to date. 

  If \var{rx} is given, it specifies a regular expression of file
  names to exclude from the search; that expression is searched for in
  the full path.

  If \var{quiet} is true, nothing is printed to the standard output
  in normal operation.
\end{funcdesc}

\begin{funcdesc}{compile_path}{\optional{skip_curdir\optional{,
                               maxlevels\optional{, force}}}}
  Byte-compile all the \file{.py} files found along \code{sys.path}.
  If \var{skip_curdir} is true (the default), the current directory is
  not included in the search.  The \var{maxlevels} and
  \var{force} parameters default to \code{0} and are passed to the
  \function{compile_dir()} function.
\end{funcdesc}


\begin{seealso}
  \seemodule[pycompile]{py_compile}{Byte-compile a single source file.}
\end{seealso}

\section{\module{dis} ---
         Disassembler for Python byte code}

\declaremodule{standard}{dis}
\modulesynopsis{Disassembler for Python byte code.}


The \module{dis} module supports the analysis of Python byte code by
disassembling it.  Since there is no Python assembler, this module
defines the Python assembly language.  The Python byte code which
this module takes as an input is defined in the file 
\file{Include/opcode.h} and used by the compiler and the interpreter.

Example: Given the function \function{myfunc}:

\begin{verbatim}
def myfunc(alist):
    return len(alist)
\end{verbatim}

the following command can be used to get the disassembly of
\function{myfunc()}:

\begin{verbatim}
>>> dis.dis(myfunc)
          0 SET_LINENO          1

          3 SET_LINENO          2
          6 LOAD_GLOBAL         0 (len)
          9 LOAD_FAST           0 (alist)
         12 CALL_FUNCTION       1
         15 RETURN_VALUE   
         16 LOAD_CONST          0 (None)
         19 RETURN_VALUE   
\end{verbatim}

The \module{dis} module defines the following functions and constants:

\begin{funcdesc}{dis}{\optional{bytesource}}
Disassemble the \var{bytesource} object. \var{bytesource} can denote
either a class, a method, a function, or a code object.  For a class,
it disassembles all methods.  For a single code sequence, it prints
one line per byte code instruction.  If no object is provided, it
disassembles the last traceback.
\end{funcdesc}

\begin{funcdesc}{distb}{\optional{tb}}
Disassembles the top-of-stack function of a traceback, using the last
traceback if none was passed.  The instruction causing the exception
is indicated.
\end{funcdesc}

\begin{funcdesc}{disassemble}{code\optional{, lasti}}
Disassembles a code object, indicating the last instruction if \var{lasti}
was provided.  The output is divided in the following columns:

\begin{enumerate}
\item the current instruction, indicated as \samp{-->},
\item a labelled instruction, indicated with \samp{>>},
\item the address of the instruction,
\item the operation code name,
\item operation parameters, and
\item interpretation of the parameters in parentheses.
\end{enumerate}

The parameter interpretation recognizes local and global
variable names, constant values, branch targets, and compare
operators.
\end{funcdesc}

\begin{funcdesc}{disco}{code\optional{, lasti}}
A synonym for disassemble.  It is more convenient to type, and kept
for compatibility with earlier Python releases.
\end{funcdesc}

\begin{datadesc}{opname}
Sequence of operation names, indexable using the byte code.
\end{datadesc}

\begin{datadesc}{cmp_op}
Sequence of all compare operation names.
\end{datadesc}

\begin{datadesc}{hasconst}
Sequence of byte codes that have a constant parameter.
\end{datadesc}

\begin{datadesc}{hasname}
Sequence of byte codes that access an attribute by name.
\end{datadesc}

\begin{datadesc}{hasjrel}
Sequence of byte codes that have a relative jump target.
\end{datadesc}

\begin{datadesc}{hasjabs}
Sequence of byte codes that have an absolute jump target.
\end{datadesc}

\begin{datadesc}{haslocal}
Sequence of byte codes that access a local variable.
\end{datadesc}

\begin{datadesc}{hascompare}
Sequence of byte codes of boolean operations.
\end{datadesc}

\subsection{Python Byte Code Instructions}
\label{bytecodes}

The Python compiler currently generates the following byte code
instructions.

\setindexsubitem{(byte code insns)}

\begin{opcodedesc}{STOP_CODE}{}
Indicates end-of-code to the compiler, not used by the interpreter.
\end{opcodedesc}

\begin{opcodedesc}{POP_TOP}{}
Removes the top-of-stack (TOS) item.
\end{opcodedesc}

\begin{opcodedesc}{ROT_TWO}{}
Swaps the two top-most stack items.
\end{opcodedesc}

\begin{opcodedesc}{ROT_THREE}{}
Lifts second and third stack item one position up, moves top down
to position three.
\end{opcodedesc}

\begin{opcodedesc}{ROT_FOUR}{}
Lifts second, third and forth stack item one position up, moves top down to
position four.
\end{opcodedesc}

\begin{opcodedesc}{DUP_TOP}{}
Duplicates the reference on top of the stack.
\end{opcodedesc}

Unary Operations take the top of the stack, apply the operation, and
push the result back on the stack.

\begin{opcodedesc}{UNARY_POSITIVE}{}
Implements \code{TOS = +TOS}.
\end{opcodedesc}

\begin{opcodedesc}{UNARY_NEGATIVE}{}
Implements \code{TOS = -TOS}.
\end{opcodedesc}

\begin{opcodedesc}{UNARY_NOT}{}
Implements \code{TOS = not TOS}.
\end{opcodedesc}

\begin{opcodedesc}{UNARY_CONVERT}{}
Implements \code{TOS = `TOS`}.
\end{opcodedesc}

\begin{opcodedesc}{UNARY_INVERT}{}
Implements \code{TOS = \~{}TOS}.
\end{opcodedesc}

Binary operations remove the top of the stack (TOS) and the second top-most
stack item (TOS1) from the stack.  They perform the operation, and put the
result back on the stack.

\begin{opcodedesc}{BINARY_POWER}{}
Implements \code{TOS = TOS1 ** TOS}.
\end{opcodedesc}

\begin{opcodedesc}{BINARY_MULTIPLY}{}
Implements \code{TOS = TOS1 * TOS}.
\end{opcodedesc}

\begin{opcodedesc}{BINARY_DIVIDE}{}
Implements \code{TOS = TOS1 / TOS}.
\end{opcodedesc}

\begin{opcodedesc}{BINARY_MODULO}{}
Implements \code{TOS = TOS1 \%{} TOS}.
\end{opcodedesc}

\begin{opcodedesc}{BINARY_ADD}{}
Implements \code{TOS = TOS1 + TOS}.
\end{opcodedesc}

\begin{opcodedesc}{BINARY_SUBTRACT}{}
Implements \code{TOS = TOS1 - TOS}.
\end{opcodedesc}

\begin{opcodedesc}{BINARY_SUBSCR}{}
Implements \code{TOS = TOS1[TOS]}.
\end{opcodedesc}

\begin{opcodedesc}{BINARY_LSHIFT}{}
Implements \code{TOS = TOS1 <\code{}< TOS}.
\end{opcodedesc}

\begin{opcodedesc}{BINARY_RSHIFT}{}
Implements \code{TOS = TOS1 >\code{}> TOS}.
\end{opcodedesc}

\begin{opcodedesc}{BINARY_AND}{}
Implements \code{TOS = TOS1 \&\ TOS}.
\end{opcodedesc}

\begin{opcodedesc}{BINARY_XOR}{}
Implements \code{TOS = TOS1 \^\ TOS}.
\end{opcodedesc}

\begin{opcodedesc}{BINARY_OR}{}
Implements \code{TOS = TOS1 | TOS}.
\end{opcodedesc}

In-place operations are like binary operations, in that they remove TOS and
TOS1, and push the result back on the stack, but the operation is done
in-place when TOS1 supports it, and the resulting TOS may be (but does not
have to be) the original TOS1.

\begin{opcodedesc}{INPLACE_POWER}{}
Implements in-place \code{TOS = TOS1 ** TOS}.
\end{opcodedesc}

\begin{opcodedesc}{INPLACE_MULTIPLY}{}
Implements in-place \code{TOS = TOS1 * TOS}.
\end{opcodedesc}

\begin{opcodedesc}{INPLACE_DIVIDE}{}
Implements in-place \code{TOS = TOS1 / TOS}.
\end{opcodedesc}

\begin{opcodedesc}{INPLACE_MODULO}{}
Implements in-place \code{TOS = TOS1 \%{} TOS}.
\end{opcodedesc}

\begin{opcodedesc}{INPLACE_ADD}{}
Implements in-place \code{TOS = TOS1 + TOS}.
\end{opcodedesc}

\begin{opcodedesc}{INPLACE_SUBTRACT}{}
Implements in-place \code{TOS = TOS1 - TOS}.
\end{opcodedesc}

\begin{opcodedesc}{INPLACE_LSHIFT}{}
Implements in-place \code{TOS = TOS1 <\code{}< TOS}.
\end{opcodedesc}

\begin{opcodedesc}{INPLACE_RSHIFT}{}
Implements in-place \code{TOS = TOS1 >\code{}> TOS}.
\end{opcodedesc}

\begin{opcodedesc}{INPLACE_AND}{}
Implements in-place \code{TOS = TOS1 \&\ TOS}.
\end{opcodedesc}

\begin{opcodedesc}{INPLACE_XOR}{}
Implements in-place \code{TOS = TOS1 \^\ TOS}.
\end{opcodedesc}

\begin{opcodedesc}{INPLACE_OR}{}
Implements in-place \code{TOS = TOS1 | TOS}.
\end{opcodedesc}

The slice opcodes take up to three parameters.

\begin{opcodedesc}{SLICE+0}{}
Implements \code{TOS = TOS[:]}.
\end{opcodedesc}

\begin{opcodedesc}{SLICE+1}{}
Implements \code{TOS = TOS1[TOS:]}.
\end{opcodedesc}

\begin{opcodedesc}{SLICE+2}{}
Implements \code{TOS = TOS1[:TOS1]}.
\end{opcodedesc}

\begin{opcodedesc}{SLICE+3}{}
Implements \code{TOS = TOS2[TOS1:TOS]}.
\end{opcodedesc}

Slice assignment needs even an additional parameter.  As any statement,
they put nothing on the stack.

\begin{opcodedesc}{STORE_SLICE+0}{}
Implements \code{TOS[:] = TOS1}.
\end{opcodedesc}

\begin{opcodedesc}{STORE_SLICE+1}{}
Implements \code{TOS1[TOS:] = TOS2}.
\end{opcodedesc}

\begin{opcodedesc}{STORE_SLICE+2}{}
Implements \code{TOS1[:TOS] = TOS2}.
\end{opcodedesc}

\begin{opcodedesc}{STORE_SLICE+3}{}
Implements \code{TOS2[TOS1:TOS] = TOS3}.
\end{opcodedesc}

\begin{opcodedesc}{DELETE_SLICE+0}{}
Implements \code{del TOS[:]}.
\end{opcodedesc}

\begin{opcodedesc}{DELETE_SLICE+1}{}
Implements \code{del TOS1[TOS:]}.
\end{opcodedesc}

\begin{opcodedesc}{DELETE_SLICE+2}{}
Implements \code{del TOS1[:TOS]}.
\end{opcodedesc}

\begin{opcodedesc}{DELETE_SLICE+3}{}
Implements \code{del TOS2[TOS1:TOS]}.
\end{opcodedesc}

\begin{opcodedesc}{STORE_SUBSCR}{}
Implements \code{TOS1[TOS] = TOS2}.
\end{opcodedesc}

\begin{opcodedesc}{DELETE_SUBSCR}{}
Implements \code{del TOS1[TOS]}.
\end{opcodedesc}

\begin{opcodedesc}{PRINT_EXPR}{}
Implements the expression statement for the interactive mode.  TOS is
removed from the stack and printed.  In non-interactive mode, an
expression statement is terminated with \code{POP_STACK}.
\end{opcodedesc}

\begin{opcodedesc}{PRINT_ITEM}{}
Prints TOS to the file-like object bound to \code{sys.stdout}.  There
is one such instruction for each item in the \keyword{print} statement.
\end{opcodedesc}

\begin{opcodedesc}{PRINT_ITEM_TO}{}
Like \code{PRINT_ITEM}, but prints the item second from TOS to the
file-like object at TOS.  This is used by the extended print statement.
\end{opcodedesc}

\begin{opcodedesc}{PRINT_NEWLINE}{}
Prints a new line on \code{sys.stdout}.  This is generated as the
last operation of a \keyword{print} statement, unless the statement
ends with a comma.
\end{opcodedesc}

\begin{opcodedesc}{PRINT_NEWLINE_TO}{}
Like \code{PRINT_NEWLINE}, but prints the new line on the file-like
object on the TOS.  This is used by the extended print statement.
\end{opcodedesc}

\begin{opcodedesc}{BREAK_LOOP}{}
Terminates a loop due to a \keyword{break} statement.
\end{opcodedesc}

\begin{opcodedesc}{LOAD_LOCALS}{}
Pushes a reference to the locals of the current scope on the stack.
This is used in the code for a class definition: After the class body
is evaluated, the locals are passed to the class definition.
\end{opcodedesc}

\begin{opcodedesc}{RETURN_VALUE}{}
Returns with TOS to the caller of the function.
\end{opcodedesc}

\begin{opcodedesc}{IMPORT_STAR}{}
Loads all symbols not starting with \character{_} directly from the module TOS
to the local namespace. The module is popped after loading all names.
This opcode implements \code{from module import *}.
\end{opcodedesc}

\begin{opcodedesc}{EXEC_STMT}{}
Implements \code{exec TOS2,TOS1,TOS}.  The compiler fills
missing optional parameters with \code{None}.
\end{opcodedesc}

\begin{opcodedesc}{POP_BLOCK}{}
Removes one block from the block stack.  Per frame, there is a 
stack of blocks, denoting nested loops, try statements, and such.
\end{opcodedesc}

\begin{opcodedesc}{END_FINALLY}{}
Terminates a \keyword{finally} clause.  The interpreter recalls
whether the exception has to be re-raised, or whether the function
returns, and continues with the outer-next block.
\end{opcodedesc}

\begin{opcodedesc}{BUILD_CLASS}{}
Creates a new class object.  TOS is the methods dictionary, TOS1
the tuple of the names of the base classes, and TOS2 the class name.
\end{opcodedesc}

All of the following opcodes expect arguments.  An argument is two
bytes, with the more significant byte last.

\begin{opcodedesc}{STORE_NAME}{namei}
Implements \code{name = TOS}. \var{namei} is the index of \var{name}
in the attribute \member{co_names} of the code object.
The compiler tries to use \code{STORE_LOCAL} or \code{STORE_GLOBAL}
if possible.
\end{opcodedesc}

\begin{opcodedesc}{DELETE_NAME}{namei}
Implements \code{del name}, where \var{namei} is the index into
\member{co_names} attribute of the code object.
\end{opcodedesc}

\begin{opcodedesc}{UNPACK_SEQUENCE}{count}
Unpacks TOS into \var{count} individual values, which are put onto
the stack right-to-left.
\end{opcodedesc}

%\begin{opcodedesc}{UNPACK_LIST}{count}
%This opcode is obsolete.
%\end{opcodedesc}

%\begin{opcodedesc}{UNPACK_ARG}{count}
%This opcode is obsolete.
%\end{opcodedesc}

\begin{opcodedesc}{DUP_TOPX}{count}
Duplicate \var{count} items, keeping them in the same order. Due to
implementation limits, \var{count} should be between 1 and 5 inclusive.
\end{opcodedesc}

\begin{opcodedesc}{STORE_ATTR}{namei}
Implements \code{TOS.name = TOS1}, where \var{namei} is the index
of name in \member{co_names}.
\end{opcodedesc}

\begin{opcodedesc}{DELETE_ATTR}{namei}
Implements \code{del TOS.name}, using \var{namei} as index into
\member{co_names}.
\end{opcodedesc}

\begin{opcodedesc}{STORE_GLOBAL}{namei}
Works as \code{STORE_NAME}, but stores the name as a global.
\end{opcodedesc}

\begin{opcodedesc}{DELETE_GLOBAL}{namei}
Works as \code{DELETE_NAME}, but deletes a global name.
\end{opcodedesc}

%\begin{opcodedesc}{UNPACK_VARARG}{argc}
%This opcode is obsolete.
%\end{opcodedesc}

\begin{opcodedesc}{LOAD_CONST}{consti}
Pushes \samp{co_consts[\var{consti}]} onto the stack.
\end{opcodedesc}

\begin{opcodedesc}{LOAD_NAME}{namei}
Pushes the value associated with \samp{co_names[\var{namei}]} onto the stack.
\end{opcodedesc}

\begin{opcodedesc}{BUILD_TUPLE}{count}
Creates a tuple consuming \var{count} items from the stack, and pushes
the resulting tuple onto the stack.
\end{opcodedesc}

\begin{opcodedesc}{BUILD_LIST}{count}
Works as \code{BUILD_TUPLE}, but creates a list.
\end{opcodedesc}

\begin{opcodedesc}{BUILD_MAP}{zero}
Pushes a new empty dictionary object onto the stack.  The argument is
ignored and set to zero by the compiler.
\end{opcodedesc}

\begin{opcodedesc}{LOAD_ATTR}{namei}
Replaces TOS with \code{getattr(TOS, co_names[\var{namei}]}.
\end{opcodedesc}

\begin{opcodedesc}{COMPARE_OP}{opname}
Performs a boolean operation.  The operation name can be found
in \code{cmp_op[\var{opname}]}.
\end{opcodedesc}

\begin{opcodedesc}{IMPORT_NAME}{namei}
Imports the module \code{co_names[\var{namei}]}.  The module object is
pushed onto the stack.  The current namespace is not affected: for a
proper import statement, a subsequent \code{STORE_FAST} instruction
modifies the namespace.
\end{opcodedesc}

\begin{opcodedesc}{IMPORT_FROM}{namei}
Loads the attribute \code{co_names[\var{namei}]} from the module found in
TOS. The resulting object is pushed onto the stack, to be subsequently
stored by a \code{STORE_FAST} instruction.
\end{opcodedesc}

\begin{opcodedesc}{JUMP_FORWARD}{delta}
Increments byte code counter by \var{delta}.
\end{opcodedesc}

\begin{opcodedesc}{JUMP_IF_TRUE}{delta}
If TOS is true, increment the byte code counter by \var{delta}.  TOS is
left on the stack.
\end{opcodedesc}

\begin{opcodedesc}{JUMP_IF_FALSE}{delta}
If TOS is false, increment the byte code counter by \var{delta}.  TOS
is not changed. 
\end{opcodedesc}

\begin{opcodedesc}{JUMP_ABSOLUTE}{target}
Set byte code counter to \var{target}.
\end{opcodedesc}

\begin{opcodedesc}{FOR_LOOP}{delta}
Iterate over a sequence.  TOS is the current index, TOS1 the sequence.
First, the next element is computed.  If the sequence is exhausted,
increment byte code counter by \var{delta}.  Otherwise, push the
sequence, the incremented counter, and the current item onto the stack.
\end{opcodedesc}

%\begin{opcodedesc}{LOAD_LOCAL}{namei}
%This opcode is obsolete.
%\end{opcodedesc}

\begin{opcodedesc}{LOAD_GLOBAL}{namei}
Loads the global named \code{co_names[\var{namei}]} onto the stack.
\end{opcodedesc}

%\begin{opcodedesc}{SET_FUNC_ARGS}{argc}
%This opcode is obsolete.
%\end{opcodedesc}

\begin{opcodedesc}{SETUP_LOOP}{delta}
Pushes a block for a loop onto the block stack.  The block spans
from the current instruction with a size of \var{delta} bytes.
\end{opcodedesc}

\begin{opcodedesc}{SETUP_EXCEPT}{delta}
Pushes a try block from a try-except clause onto the block stack.
\var{delta} points to the first except block.
\end{opcodedesc}

\begin{opcodedesc}{SETUP_FINALLY}{delta}
Pushes a try block from a try-except clause onto the block stack.
\var{delta} points to the finally block.
\end{opcodedesc}

\begin{opcodedesc}{LOAD_FAST}{var_num}
Pushes a reference to the local \code{co_varnames[\var{var_num}]} onto
the stack.
\end{opcodedesc}

\begin{opcodedesc}{STORE_FAST}{var_num}
Stores TOS into the local \code{co_varnames[\var{var_num}]}.
\end{opcodedesc}

\begin{opcodedesc}{DELETE_FAST}{var_num}
Deletes local \code{co_varnames[\var{var_num}]}.
\end{opcodedesc}

\begin{opcodedesc}{LOAD_CLOSURE}{i}
Pushes a reference to the cell contained in slot \var{i} of the
cell and free variable storage.  The name of the variable is 
\code{co_cellvars[\var{i}]} if \var{i} is less than the length of
\var{co_cellvars}.  Otherwise it is 
\code{co_freevars[\var{i} - len(co_cellvars)]}.
\end{opcodedesc}

\begin{opcodedesc}{LOAD_DEREF}{i}
Loads the cell contained in slot \var{i} of the cell and free variable
storage.  Pushes a reference to the object the cell contains on the
stack. 
\end{opcodedesc}

\begin{opcodedesc}{STORE_DEREF}{i}
Stores TOS into the cell contained in slot \var{i} of the cell and
free variable storage.
\end{opcodedesc}

\begin{opcodedesc}{SET_LINENO}{lineno}
Sets the current line number to \var{lineno}.
\end{opcodedesc}

\begin{opcodedesc}{RAISE_VARARGS}{argc}
Raises an exception. \var{argc} indicates the number of parameters
to the raise statement, ranging from 0 to 3.  The handler will find
the traceback as TOS2, the parameter as TOS1, and the exception
as TOS.
\end{opcodedesc}

\begin{opcodedesc}{CALL_FUNCTION}{argc}
Calls a function.  The low byte of \var{argc} indicates the number of
positional parameters, the high byte the number of keyword parameters.
On the stack, the opcode finds the keyword parameters first.  For each
keyword argument, the value is on top of the key.  Below the keyword
parameters, the positional parameters are on the stack, with the
right-most parameter on top.  Below the parameters, the function object
to call is on the stack.
\end{opcodedesc}

\begin{opcodedesc}{MAKE_FUNCTION}{argc}
Pushes a new function object on the stack.  TOS is the code associated
with the function.  The function object is defined to have \var{argc}
default parameters, which are found below TOS.
\end{opcodedesc}

\begin{opcodedesc}{MAKE_CLOSURE}{argc}
Creates a new function object, sets its \var{func_closure} slot, and
pushes it on the stack.  TOS is the code associated with the function.
If the code object has N free variables, the next N items on the stack
are the cells for these variables.  The function also has \var{argc}
default parameters, where are found before the cells.
\end{opcodedesc}

\begin{opcodedesc}{BUILD_SLICE}{argc}
Pushes a slice object on the stack.  \var{argc} must be 2 or 3.  If it
is 2, \code{slice(TOS1, TOS)} is pushed; if it is 3,
\code{slice(TOS2, TOS1, TOS)} is pushed.
See the \code{slice()}\bifuncindex{slice} built-in function for more
information.
\end{opcodedesc}

\begin{opcodedesc}{EXTENDED_ARG}{ext}
Prefixes any opcode which has an argument too big to fit into the
default two bytes.  \var{ext} holds two additional bytes which, taken
together with the subsequent opcode's argument, comprise a four-byte
argument, \var{ext} being the two most-significant bytes.
\end{opcodedesc}

\begin{opcodedesc}{CALL_FUNCTION_VAR}{argc}
Calls a function. \var{argc} is interpreted as in \code{CALL_FUNCTION}.
The top element on the stack contains the variable argument list, followed
by keyword and positional arguments.
\end{opcodedesc}

\begin{opcodedesc}{CALL_FUNCTION_KW}{argc}
Calls a function. \var{argc} is interpreted as in \code{CALL_FUNCTION}.
The top element on the stack contains the keyword arguments dictionary, 
followed by explicit keyword and positional arguments.
\end{opcodedesc}

\begin{opcodedesc}{CALL_FUNCTION_VAR_KW}{argc}
Calls a function. \var{argc} is interpreted as in
\code{CALL_FUNCTION}.  The top element on the stack contains the
keyword arguments dictionary, followed by the variable-arguments
tuple, followed by explicit keyword and positional arguments.
\end{opcodedesc}

\section{\module{site} ---
         Site-specific configuration hook}

\declaremodule{standard}{site}
\modulesynopsis{A standard way to reference site-specific modules.}


\strong{This module is automatically imported during initialization.}
The automatic import can be suppressed using the interpreter's
\programopt{-S} option.

Importing this module will append site-specific paths to the module
search path.
\indexiii{module}{search}{path}

It starts by constructing up to four directories from a head and a
tail part.  For the head part, it uses \code{sys.prefix} and
\code{sys.exec_prefix}; empty heads are skipped.  For
the tail part, it uses the empty string (on Windows) or
\file{lib/python\shortversion/site-packages} (on \UNIX{} and Macintosh)
and then \file{lib/site-python}.  For each of the distinct
head-tail combinations, it sees if it refers to an existing directory,
and if so, adds it to \code{sys.path} and also inspects the newly added 
path for configuration files.
\indexii{site-python}{directory}
\indexii{site-packages}{directory}

A path configuration file is a file whose name has the form
\file{\var{package}.pth} and exists in one of the four directories
mentioned above; its contents are additional items (one
per line) to be added to \code{sys.path}.  Non-existing items are
never added to \code{sys.path}, but no check is made that the item
refers to a directory (rather than a file).  No item is added to
\code{sys.path} more than once.  Blank lines and lines beginning with
\code{\#} are skipped.  Lines starting with \code{import} are executed.
\index{package}
\indexiii{path}{configuration}{file}

For example, suppose \code{sys.prefix} and \code{sys.exec_prefix} are
set to \file{/usr/local}.  The Python \version\ library is then
installed in \file{/usr/local/lib/python\shortversion} (where only the
first three characters of \code{sys.version} are used to form the
installation path name).  Suppose this has a subdirectory
\file{/usr/local/lib/python\shortversion/site-packages} with three
subsubdirectories, \file{foo}, \file{bar} and \file{spam}, and two
path configuration files, \file{foo.pth} and \file{bar.pth}.  Assume
\file{foo.pth} contains the following:

\begin{verbatim}
# foo package configuration

foo
bar
bletch
\end{verbatim}

and \file{bar.pth} contains:

\begin{verbatim}
# bar package configuration

bar
\end{verbatim}

Then the following directories are added to \code{sys.path}, in this
order:

\begin{verbatim}
/usr/local/lib/python2.3/site-packages/bar
/usr/local/lib/python2.3/site-packages/foo
\end{verbatim}

Note that \file{bletch} is omitted because it doesn't exist; the
\file{bar} directory precedes the \file{foo} directory because
\file{bar.pth} comes alphabetically before \file{foo.pth}; and
\file{spam} is omitted because it is not mentioned in either path
configuration file.

After these path manipulations, an attempt is made to import a module
named \module{sitecustomize}\refmodindex{sitecustomize}, which can
perform arbitrary site-specific customizations.  If this import fails
with an \exception{ImportError} exception, it is silently ignored.

Note that for some non-\UNIX{} systems, \code{sys.prefix} and
\code{sys.exec_prefix} are empty, and the path manipulations are
skipped; however the import of
\module{sitecustomize}\refmodindex{sitecustomize} is still attempted.

\section{\module{user} ---
         User-specific configuration hook}

\declaremodule{standard}{user}
\modulesynopsis{A standard way to reference user-specific modules.}


\indexii{.pythonrc.py}{file}
\indexiii{user}{configuration}{file}

As a policy, Python doesn't run user-specified code on startup of
Python programs.  (Only interactive sessions execute the script
specified in the \envvar{PYTHONSTARTUP} environment variable if it
exists).

However, some programs or sites may find it convenient to allow users
to have a standard customization file, which gets run when a program
requests it.  This module implements such a mechanism.  A program
that wishes to use the mechanism must execute the statement

\begin{verbatim}
import user
\end{verbatim}

The \module{user} module looks for a file \file{.pythonrc.py} in the user's
home directory and if it can be opened, executes it (using
\function{exec()}\bifuncindex{exec}) in its own (the
module \module{user}'s) global namespace.  Errors during this phase
are not caught; that's up to the program that imports the
\module{user} module, if it wishes.  The home directory is assumed to
be named by the \envvar{HOME} environment variable; if this is not set,
the current directory is used.

The user's \file{.pythonrc.py} could conceivably test for
\code{sys.version} if it wishes to do different things depending on
the Python version.

A warning to users: be very conservative in what you place in your
\file{.pythonrc.py} file.  Since you don't know which programs will
use it, changing the behavior of standard modules or functions is
generally not a good idea.

A suggestion for programmers who wish to use this mechanism: a simple
way to let users specify options for your package is to have them
define variables in their \file{.pythonrc.py} file that you test in
your module.  For example, a module \module{spam} that has a verbosity
level can look for a variable \code{user.spam_verbose}, as follows:

\begin{verbatim}
import user

verbose = bool(getattr(user, "spam_verbose", 0))
\end{verbatim}

(The three-argument form of \function{getattr()} is used in case
the user has not defined \code{spam_verbose} in their
\file{.pythonrc.py} file.)

Programs with extensive customization needs are better off reading a
program-specific customization file.

Programs with security or privacy concerns should \emph{not} import
this module; a user can easily break into a program by placing
arbitrary code in the \file{.pythonrc.py} file.

Modules for general use should \emph{not} import this module; it may
interfere with the operation of the importing program.

\begin{seealso}
  \seemodule{site}{Site-wide customization mechanism.}
\end{seealso}

\section{\module{__builtin__} ---
         Built-in functions.}
\declaremodule[builtin]{builtin}{__builtin__}

\modulesynopsis{The set of built-in functions.}


This module provides direct access to all `built-in' identifiers of
Python; e.g. \code{__builtin__.open} is the full name for the built-in
function \code{open()}.  See section \ref{built-in-funcs}, ``Built-in
Functions.''
		% really __builtin__
\section{\module{__main__} ---
         Top-level script environment}

\declaremodule[main]{builtin}{__main__}
\modulesynopsis{The environment where the top-level script is run.}

This module represents the (otherwise anonymous) scope in which the
interpreter's main program executes --- commands read either from
standard input, from a script file, or from an interactive prompt.  It
is this environment in which the idiomatic ``conditional script''
stanza causes a script to run:

\begin{verbatim}
if __name__ == "__main__":
    main()
\end{verbatim}
			% really __main__

\chapter{String Services}

The modules described in this chapter provide a wide range of string
manipulation operations.  Here's an overview:

\begin{description}

\item[string]
--- Common string operations.

\item[re]
--- New Perl-style regular expression search and match operations.

\item[regex]
--- Regular expression search and match operations.

\item[regsub]
--- Substitution and splitting operations that use regular expressions.

\item[struct]
--- Interpret strings as packed binary data.

\item[StringIO]
--- Read and write strings as if they were files.

\item[soundex]
--- Compute hash values for English words.

\end{description}
		% String Services
\section{\module{string} ---
         Common string operations}

\declaremodule{standard}{string}
\modulesynopsis{Common string operations.}

The \module{string} module contains a number of useful constants and classes,
as well as some deprecated legacy functions that are also available as methods
on strings.  See the module \refmodule{re}\refstmodindex{re} for string
functions based on regular expressions.

\subsection{String constants}

The constants defined in this module are:

\begin{datadesc}{ascii_letters}
  The concatenation of the \constant{ascii_lowercase} and
  \constant{ascii_uppercase} constants described below.  This value is
  not locale-dependent.
\end{datadesc}

\begin{datadesc}{ascii_lowercase}
  The lowercase letters \code{'abcdefghijklmnopqrstuvwxyz'}.  This
  value is not locale-dependent and will not change.
\end{datadesc}

\begin{datadesc}{ascii_uppercase}
  The uppercase letters \code{'ABCDEFGHIJKLMNOPQRSTUVWXYZ'}.  This
  value is not locale-dependent and will not change.
\end{datadesc}

\begin{datadesc}{digits}
  The string \code{'0123456789'}.
\end{datadesc}

\begin{datadesc}{hexdigits}
  The string \code{'0123456789abcdefABCDEF'}.
\end{datadesc}

\begin{datadesc}{letters}
  The concatenation of the strings \constant{lowercase} and
  \constant{uppercase} described below.  The specific value is
  locale-dependent, and will be updated when
  \function{locale.setlocale()} is called.
\end{datadesc}

\begin{datadesc}{lowercase}
  A string containing all the characters that are considered lowercase
  letters.  On most systems this is the string
  \code{'abcdefghijklmnopqrstuvwxyz'}.  Do not change its definition ---
  the effect on the routines \function{upper()} and
  \function{swapcase()} is undefined.  The specific value is
  locale-dependent, and will be updated when
  \function{locale.setlocale()} is called.
\end{datadesc}

\begin{datadesc}{octdigits}
  The string \code{'01234567'}.
\end{datadesc}

\begin{datadesc}{punctuation}
  String of \ASCII{} characters which are considered punctuation
  characters in the \samp{C} locale.
\end{datadesc}

\begin{datadesc}{printable}
  String of characters which are considered printable.  This is a
  combination of \constant{digits}, \constant{letters},
  \constant{punctuation}, and \constant{whitespace}.
\end{datadesc}

\begin{datadesc}{uppercase}
  A string containing all the characters that are considered uppercase
  letters.  On most systems this is the string
  \code{'ABCDEFGHIJKLMNOPQRSTUVWXYZ'}.  Do not change its definition ---
  the effect on the routines \function{lower()} and
  \function{swapcase()} is undefined.  The specific value is
  locale-dependent, and will be updated when
  \function{locale.setlocale()} is called.
\end{datadesc}

\begin{datadesc}{whitespace}
  A string containing all characters that are considered whitespace.
  On most systems this includes the characters space, tab, linefeed,
  return, formfeed, and vertical tab.  Do not change its definition ---
  the effect on the routines \function{strip()} and \function{split()}
  is undefined.
\end{datadesc}

\subsection{Template strings}

Templates provide simpler string substitutions as described in \pep{292}.
Instead of the normal \samp{\%}-based substitutions, Templates support
\samp{\$}-based substitutions, using the following rules:

\begin{itemize}
\item \samp{\$\$} is an escape; it is replaced with a single \samp{\$}.

\item \samp{\$identifier} names a substitution placeholder matching a mapping
       key of "identifier".  By default, "identifier" must spell a Python
       identifier.  The first non-identifier character after the \samp{\$}
       character terminates this placeholder specification.

\item \samp{\$\{identifier\}} is equivalent to \samp{\$identifier}.  It is
      required when valid identifier characters follow the placeholder but are
      not part of the placeholder, such as "\$\{noun\}ification".
\end{itemize}

Any other appearance of \samp{\$} in the string will result in a
\exception{ValueError} being raised.

\versionadded{2.4}

The \module{string} module provides a \class{Template} class that implements
these rules.  The methods of \class{Template} are:

\begin{classdesc}{Template}{template}
The constructor takes a single argument which is the template string.
\end{classdesc}

\begin{methoddesc}[Template]{substitute}{mapping\optional{, **kws}}
Performs the template substitution, returning a new string.  \var{mapping} is
any dictionary-like object with keys that match the placeholders in the
template.  Alternatively, you can provide keyword arguments, where the
keywords are the placeholders.  When both \var{mapping} and \var{kws} are
given and there are duplicates, the placeholders from \var{kws} take
precedence.
\end{methoddesc}

\begin{methoddesc}[Template]{safe_substitute}{mapping\optional{, **kws}}
Like \method{substitute()}, except that if placeholders are missing from
\var{mapping} and \var{kws}, instead of raising a \exception{KeyError}
exception, the original placeholder will appear in the resulting string
intact.  Also, unlike with \method{substitute()}, any other appearances of the
\samp{\$} will simply return \samp{\$} instead of raising
\exception{ValueError}.

While other exceptions may still occur, this method is called ``safe'' because
substitutions always tries to return a usable string instead of raising an
exception.  In another sense, \method{safe_substitute()} may be anything other
than safe, since it will silently ignore malformed templates containing
dangling delimiters, unmatched braces, or placeholders that are not valid
Python identifiers.
\end{methoddesc}

\class{Template} instances also provide one public data attribute:

\begin{memberdesc}[string]{template}
This is the object passed to the constructor's \var{template} argument.  In
general, you shouldn't change it, but read-only access is not enforced.
\end{memberdesc}

Here is an example of how to use a Template:

\begin{verbatim}
>>> from string import Template
>>> s = Template('$who likes $what')
>>> s.substitute(who='tim', what='kung pao')
'tim likes kung pao'
>>> d = dict(who='tim')
>>> Template('Give $who $100').substitute(d)
Traceback (most recent call last):
[...]
ValueError: Invalid placeholder in string: line 1, col 10
>>> Template('$who likes $what').substitute(d)
Traceback (most recent call last):
[...]
KeyError: 'what'
>>> Template('$who likes $what').safe_substitute(d)
'tim likes $what'
\end{verbatim}

Advanced usage: you can derive subclasses of \class{Template} to customize the
placeholder syntax, delimiter character, or the entire regular expression used
to parse template strings.  To do this, you can override these class
attributes:

\begin{itemize}
\item \var{delimiter} -- This is the literal string describing a placeholder
      introducing delimiter.  The default value \samp{\$}.  Note that this
      should \emph{not} be a regular expression, as the implementation will
      call \method{re.escape()} on this string as needed.
\item \var{idpattern} -- This is the regular expression describing the pattern
      for non-braced placeholders (the braces will be added automatically as
      appropriate).  The default value is the regular expression
      \samp{[_a-z][_a-z0-9]*}.
\end{itemize}

Alternatively, you can provide the entire regular expression pattern by
overriding the class attribute \var{pattern}.  If you do this, the value must
be a regular expression object with four named capturing groups.  The
capturing groups correspond to the rules given above, along with the invalid
placeholder rule:

\begin{itemize}
\item \var{escaped} -- This group matches the escape sequence,
      e.g. \samp{\$\$}, in the default pattern.
\item \var{named} -- This group matches the unbraced placeholder name; it
      should not include the delimiter in capturing group.
\item \var{braced} -- This group matches the brace enclosed placeholder name;
      it should not include either the delimiter or braces in the capturing
      group.
\item \var{invalid} -- This group matches any other delimiter pattern (usually
      a single delimiter), and it should appear last in the regular
      expression.
\end{itemize}

\subsection{String functions}

The following functions are available to operate on string and Unicode
objects.  They are not available as string methods.

\begin{funcdesc}{capwords}{s}
  Split the argument into words using \function{split()}, capitalize
  each word using \function{capitalize()}, and join the capitalized
  words using \function{join()}.  Note that this replaces runs of
  whitespace characters by a single space, and removes leading and
  trailing whitespace.
\end{funcdesc}

\begin{funcdesc}{maketrans}{from, to}
  Return a translation table suitable for passing to
  \function{translate()}, that will map
  each character in \var{from} into the character at the same position
  in \var{to}; \var{from} and \var{to} must have the same length.

  \warning{Don't use strings derived from \constant{lowercase}
  and \constant{uppercase} as arguments; in some locales, these don't have
  the same length.  For case conversions, always use
  \function{lower()} and \function{upper()}.}
\end{funcdesc}

\subsection{Deprecated string functions}

The following list of functions are also defined as methods of string and
Unicode objects; see ``String Methods'' (section
\ref{string-methods}) for more information on those.  You should consider
these functions as deprecated, although they will not be removed until Python
3.0.  The functions defined in this module are:

\begin{funcdesc}{atof}{s}
  \deprecated{2.0}{Use the \function{float()} built-in function.}
  Convert a string to a floating point number.  The string must have
  the standard syntax for a floating point literal in Python,
  optionally preceded by a sign (\samp{+} or \samp{-}).  Note that
  this behaves identical to the built-in function
  \function{float()}\bifuncindex{float} when passed a string.

  \note{When passing in a string, values for NaN\index{NaN}
  and Infinity\index{Infinity} may be returned, depending on the
  underlying C library.  The specific set of strings accepted which
  cause these values to be returned depends entirely on the C library
  and is known to vary.}
\end{funcdesc}

\begin{funcdesc}{atoi}{s\optional{, base}}
  \deprecated{2.0}{Use the \function{int()} built-in function.}
  Convert string \var{s} to an integer in the given \var{base}.  The
  string must consist of one or more digits, optionally preceded by a
  sign (\samp{+} or \samp{-}).  The \var{base} defaults to 10.  If it
  is 0, a default base is chosen depending on the leading characters
  of the string (after stripping the sign): \samp{0x} or \samp{0X}
  means 16, \samp{0} means 8, anything else means 10.  If \var{base}
  is 16, a leading \samp{0x} or \samp{0X} is always accepted, though
  not required.  This behaves identically to the built-in function
  \function{int()} when passed a string.  (Also note: for a more
  flexible interpretation of numeric literals, use the built-in
  function \function{eval()}\bifuncindex{eval}.)
\end{funcdesc}

\begin{funcdesc}{atol}{s\optional{, base}}
  \deprecated{2.0}{Use the \function{long()} built-in function.}
  Convert string \var{s} to a long integer in the given \var{base}.
  The string must consist of one or more digits, optionally preceded
  by a sign (\samp{+} or \samp{-}).  The \var{base} argument has the
  same meaning as for \function{atoi()}.  A trailing \samp{l} or
  \samp{L} is not allowed, except if the base is 0.  Note that when
  invoked without \var{base} or with \var{base} set to 10, this
  behaves identical to the built-in function
  \function{long()}\bifuncindex{long} when passed a string.
\end{funcdesc}

\begin{funcdesc}{capitalize}{word}
  Return a copy of \var{word} with only its first character capitalized.
\end{funcdesc}

\begin{funcdesc}{expandtabs}{s\optional{, tabsize}}
  Expand tabs in a string replacing them by one or more spaces,
  depending on the current column and the given tab size.  The column
  number is reset to zero after each newline occurring in the string.
  This doesn't understand other non-printing characters or escape
  sequences.  The tab size defaults to 8.
\end{funcdesc}

\begin{funcdesc}{find}{s, sub\optional{, start\optional{,end}}}
  Return the lowest index in \var{s} where the substring \var{sub} is
  found such that \var{sub} is wholly contained in
  \code{\var{s}[\var{start}:\var{end}]}.  Return \code{-1} on failure.
  Defaults for \var{start} and \var{end} and interpretation of
  negative values is the same as for slices.
\end{funcdesc}

\begin{funcdesc}{rfind}{s, sub\optional{, start\optional{, end}}}
  Like \function{find()} but find the highest index.
\end{funcdesc}

\begin{funcdesc}{index}{s, sub\optional{, start\optional{, end}}}
  Like \function{find()} but raise \exception{ValueError} when the
  substring is not found.
\end{funcdesc}

\begin{funcdesc}{rindex}{s, sub\optional{, start\optional{, end}}}
  Like \function{rfind()} but raise \exception{ValueError} when the
  substring is not found.
\end{funcdesc}

\begin{funcdesc}{count}{s, sub\optional{, start\optional{, end}}}
  Return the number of (non-overlapping) occurrences of substring
  \var{sub} in string \code{\var{s}[\var{start}:\var{end}]}.
  Defaults for \var{start} and \var{end} and interpretation of
  negative values are the same as for slices.
\end{funcdesc}

\begin{funcdesc}{lower}{s}
  Return a copy of \var{s}, but with upper case letters converted to
  lower case.
\end{funcdesc}

\begin{funcdesc}{split}{s\optional{, sep\optional{, maxsplit}}}
  Return a list of the words of the string \var{s}.  If the optional
  second argument \var{sep} is absent or \code{None}, the words are
  separated by arbitrary strings of whitespace characters (space, tab, 
  newline, return, formfeed).  If the second argument \var{sep} is
  present and not \code{None}, it specifies a string to be used as the 
  word separator.  The returned list will then have one more item
  than the number of non-overlapping occurrences of the separator in
  the string.  The optional third argument \var{maxsplit} defaults to
  0.  If it is nonzero, at most \var{maxsplit} number of splits occur,
  and the remainder of the string is returned as the final element of
  the list (thus, the list will have at most \code{\var{maxsplit}+1}
  elements).

  The behavior of split on an empty string depends on the value of \var{sep}.
  If \var{sep} is not specified, or specified as \code{None}, the result will
  be an empty list.  If \var{sep} is specified as any string, the result will
  be a list containing one element which is an empty string.
\end{funcdesc}

\begin{funcdesc}{rsplit}{s\optional{, sep\optional{, maxsplit}}}
  Return a list of the words of the string \var{s}, scanning \var{s}
  from the end.  To all intents and purposes, the resulting list of
  words is the same as returned by \function{split()}, except when the
  optional third argument \var{maxsplit} is explicitly specified and
  nonzero.  When \var{maxsplit} is nonzero, at most \var{maxsplit}
  number of splits -- the \emph{rightmost} ones -- occur, and the remainder
  of the string is returned as the first element of the list (thus, the
  list will have at most \code{\var{maxsplit}+1} elements).
  \versionadded{2.4}
\end{funcdesc}

\begin{funcdesc}{splitfields}{s\optional{, sep\optional{, maxsplit}}}
  This function behaves identically to \function{split()}.  (In the
  past, \function{split()} was only used with one argument, while
  \function{splitfields()} was only used with two arguments.)
\end{funcdesc}

\begin{funcdesc}{join}{words\optional{, sep}}
  Concatenate a list or tuple of words with intervening occurrences of 
  \var{sep}.  The default value for \var{sep} is a single space
  character.  It is always true that
  \samp{string.join(string.split(\var{s}, \var{sep}), \var{sep})}
  equals \var{s}.
\end{funcdesc}

\begin{funcdesc}{joinfields}{words\optional{, sep}}
  This function behaves identically to \function{join()}.  (In the past, 
  \function{join()} was only used with one argument, while
  \function{joinfields()} was only used with two arguments.)
  Note that there is no \method{joinfields()} method on string
  objects; use the \method{join()} method instead.
\end{funcdesc}

\begin{funcdesc}{lstrip}{s\optional{, chars}}
Return a copy of the string with leading characters removed.  If
\var{chars} is omitted or \code{None}, whitespace characters are
removed.  If given and not \code{None}, \var{chars} must be a string;
the characters in the string will be stripped from the beginning of
the string this method is called on.
\versionchanged[The \var{chars} parameter was added.  The \var{chars}
parameter cannot be passed in earlier 2.2 versions]{2.2.3}
\end{funcdesc}

\begin{funcdesc}{rstrip}{s\optional{, chars}}
Return a copy of the string with trailing characters removed.  If
\var{chars} is omitted or \code{None}, whitespace characters are
removed.  If given and not \code{None}, \var{chars} must be a string;
the characters in the string will be stripped from the end of the
string this method is called on.
\versionchanged[The \var{chars} parameter was added.  The \var{chars}
parameter cannot be passed in earlier 2.2 versions]{2.2.3}
\end{funcdesc}

\begin{funcdesc}{strip}{s\optional{, chars}}
Return a copy of the string with leading and trailing characters
removed.  If \var{chars} is omitted or \code{None}, whitespace
characters are removed.  If given and not \code{None}, \var{chars}
must be a string; the characters in the string will be stripped from
the both ends of the string this method is called on.
\versionchanged[The \var{chars} parameter was added.  The \var{chars}
parameter cannot be passed in earlier 2.2 versions]{2.2.3}
\end{funcdesc}

\begin{funcdesc}{swapcase}{s}
  Return a copy of \var{s}, but with lower case letters
  converted to upper case and vice versa.
\end{funcdesc}

\begin{funcdesc}{translate}{s, table\optional{, deletechars}}
  Delete all characters from \var{s} that are in \var{deletechars} (if 
  present), and then translate the characters using \var{table}, which 
  must be a 256-character string giving the translation for each
  character value, indexed by its ordinal.
\end{funcdesc}

\begin{funcdesc}{upper}{s}
  Return a copy of \var{s}, but with lower case letters converted to
  upper case.
\end{funcdesc}

\begin{funcdesc}{ljust}{s, width}
\funcline{rjust}{s, width}
\funcline{center}{s, width}
  These functions respectively left-justify, right-justify and center
  a string in a field of given width.  They return a string that is at
  least \var{width} characters wide, created by padding the string
  \var{s} with spaces until the given width on the right, left or both
  sides.  The string is never truncated.
\end{funcdesc}

\begin{funcdesc}{zfill}{s, width}
  Pad a numeric string on the left with zero digits until the given
  width is reached.  Strings starting with a sign are handled
  correctly.
\end{funcdesc}

\begin{funcdesc}{replace}{str, old, new\optional{, maxreplace}}
  Return a copy of string \var{str} with all occurrences of substring
  \var{old} replaced by \var{new}.  If the optional argument
  \var{maxreplace} is given, the first \var{maxreplace} occurrences are
  replaced.
\end{funcdesc}

\section{\module{re} ---
         New Perl-style regular expression search and match operations.}
\declaremodule{standard}{re}
\moduleauthor{Andrew M. Kuchling}{akuchling@acm.org}
\sectionauthor{Andrew M. Kuchling}{akuchling@acm.org}


\modulesynopsis{New Perl-style regular expression search and match
operations.}


This module provides regular expression matching operations similar to
those found in Perl.  It's 8-bit clean: the strings being processed
may contain both null bytes and characters whose high bit is set.  Regular
expression patterns may not contain null bytes, but they may contain
characters with the high bit set.  The \module{re} module is always
available.

Regular expressions use the backslash character (\character{\e}) to
indicate special forms or to allow special characters to be used
without invoking their special meaning.  This collides with Python's
usage of the same character for the same purpose in string literals;
for example, to match a literal backslash, one might have to write
\code{'\e\e\e\e'} as the pattern string, because the regular expression
must be \samp{\e\e}, and each backslash must be expressed as
\samp{\e\e} inside a regular Python string literal. 

The solution is to use Python's raw string notation for regular
expression patterns; backslashes are not handled in any special way in
a string literal prefixed with \character{r}.  So \code{r"\e n"} is a
two-character string containing \character{\e} and \character{n},
while \code{"\e n"} is a one-character string containing a newline.
Usually patterns will be expressed in Python code using this raw
string notation.

\subsection{Regular Expression Syntax \label{re-syntax}}

A regular expression (or RE) specifies a set of strings that matches
it; the functions in this module let you check if a particular string
matches a given regular expression (or if a given regular expression
matches a particular string, which comes down to the same thing).

Regular expressions can be concatenated to form new regular
expressions; if \emph{A} and \emph{B} are both regular expressions,
then \emph{AB} is also an regular expression.  If a string \emph{p}
matches A and another string \emph{q} matches B, the string \emph{pq}
will match AB.  Thus, complex expressions can easily be constructed
from simpler primitive expressions like the ones described here.  For
details of the theory and implementation of regular expressions,
consult the Friedl book referenced below, or almost any textbook about
compiler construction.

A brief explanation of the format of regular expressions follows.  For
further information and a gentler presentation, consult the Regular
Expression HOWTO, accessible from \url{http://www.python.org/doc/howto/}.

Regular expressions can contain both special and ordinary characters.
Most ordinary characters, like \character{A}, \character{a}, or \character{0},
are the simplest regular expressions; they simply match themselves.  
You can concatenate ordinary characters, so \regexp{last} matches the
string \code{'last'}.  (In the rest of this section, we'll write RE's in
\regexp{this special style}, usually without quotes, and strings to be
matched \code{'in single quotes'}.)

Some characters, like \character{|} or \character{(}, are special.  Special
characters either stand for classes of ordinary characters, or affect
how the regular expressions around them are interpreted.

The special characters are:

\begin{list}{}{\leftmargin 0.7in \labelwidth 0.65in}

\item[\character{.}] (Dot.)  In the default mode, this matches any
character except a newline.  If the \constant{DOTALL} flag has been
specified, this matches any character including a newline.

\item[\character{\^}] (Caret.)  Matches the start of the string, and in
\constant{MULTILINE} mode also matches immediately after each newline.

\item[\character{\$}] Matches the end of the string, and in
\constant{MULTILINE} mode also matches before a newline.
\regexp{foo} matches both 'foo' and 'foobar', while the regular
expression \regexp{foo\$} matches only 'foo'.

\item[\character{*}] Causes the resulting RE to
match 0 or more repetitions of the preceding RE, as many repetitions
as are possible.  \regexp{ab*} will
match 'a', 'ab', or 'a' followed by any number of 'b's.

\item[\character{+}] Causes the
resulting RE to match 1 or more repetitions of the preceding RE.
\regexp{ab+} will match 'a' followed by any non-zero number of 'b's; it
will not match just 'a'.

\item[\character{?}] Causes the resulting RE to
match 0 or 1 repetitions of the preceding RE.  \regexp{ab?} will
match either 'a' or 'ab'.
\item[\code{*?}, \code{+?}, \code{??}] The \character{*}, \character{+}, and
\character{?} qualifiers are all \dfn{greedy}; they match as much text as
possible.  Sometimes this behaviour isn't desired; if the RE
\regexp{<.*>} is matched against \code{'<H1>title</H1>'}, it will match the
entire string, and not just \code{'<H1>'}.
Adding \character{?} after the qualifier makes it perform the match in
\dfn{non-greedy} or \dfn{minimal} fashion; as \emph{few} characters as
possible will be matched.  Using \regexp{.*?} in the previous
expression will match only \code{'<H1>'}.

\item[\code{\{\var{m},\var{n}\}}] Causes the resulting RE to match from
\var{m} to \var{n} repetitions of the preceding RE, attempting to
match as many repetitions as possible.  For example, \regexp{a\{3,5\}}
will match from 3 to 5 \character{a} characters.  Omitting \var{n}
specifies an infinite upper bound; you can't omit \var{m}.

\item[\code{\{\var{m},\var{n}\}?}] Causes the resulting RE to
match from \var{m} to \var{n} repetitions of the preceding RE,
attempting to match as \emph{few} repetitions as possible.  This is
the non-greedy version of the previous qualifier.  For example, on the
6-character string \code{'aaaaaa'}, \regexp{a\{3,5\}} will match 5
\character{a} characters, while \regexp{a\{3,5\}?} will only match 3
characters.

\item[\character{\e}] Either escapes special characters (permitting
you to match characters like \character{*}, \character{?}, and so
forth), or signals a special sequence; special sequences are discussed
below.

If you're not using a raw string to
express the pattern, remember that Python also uses the
backslash as an escape sequence in string literals; if the escape
sequence isn't recognized by Python's parser, the backslash and
subsequent character are included in the resulting string.  However,
if Python would recognize the resulting sequence, the backslash should
be repeated twice.  This is complicated and hard to understand, so
it's highly recommended that you use raw strings for all but the
simplest expressions.

\item[\code{[]}] Used to indicate a set of characters.  Characters can
be listed individually, or a range of characters can be indicated by
giving two characters and separating them by a \character{-}.  Special
characters are not active inside sets.  For example, \regexp{[akm\$]}
will match any of the characters \character{a}, \character{k},
\character{m}, or \character{\$}; \regexp{[a-z]}
will match any lowercase letter, and \code{[a-zA-Z0-9]} matches any
letter or digit.  Character classes such as \code{\e w} or \code{\e S}
(defined below) are also acceptable inside a range.  If you want to
include a \character{]} or a \character{-} inside a set, precede it with a
backslash, or place it as the first character.  The 
pattern \regexp{[]]} will match \code{']'}, for example.  

You can match the characters not within a range by \dfn{complementing}
the set.  This is indicated by including a
\character{\^} as the first character of the set; \character{\^} elsewhere will
simply match the \character{\^} character.  For example, \regexp{[{\^}5]}
will match any character except \character{5}.

\item[\character{|}]\code{A|B}, where A and B can be arbitrary REs,
creates a regular expression that will match either A or B.  This can
be used inside groups (see below) as well.  To match a literal \character{|},
use \regexp{\e|}, or enclose it inside a character class, as in  \regexp{[|]}.

\item[\code{(...)}] Matches whatever regular expression is inside the
parentheses, and indicates the start and end of a group; the contents
of a group can be retrieved after a match has been performed, and can
be matched later in the string with the \regexp{\e \var{number}} special
sequence, described below.  To match the literals \character{(} or
\character{')}, use \regexp{\e(} or \regexp{\e)}, or enclose them
inside a character class: \regexp{[(] [)]}.

\item[\code{(?...)}] This is an extension notation (a \character{?}
following a \character{(} is not meaningful otherwise).  The first
character after the \character{?} 
determines what the meaning and further syntax of the construct is.
Extensions usually do not create a new group;
\regexp{(?P<\var{name}>...)} is the only exception to this rule.
Following are the currently supported extensions.

\item[\code{(?iLmsx)}] (One or more letters from the set \character{i},
\character{L}, \character{m}, \character{s}, \character{x}.)  The group matches
the empty string; the letters set the corresponding flags
(\constant{re.I}, \constant{re.L}, \constant{re.M}, \constant{re.S},
\constant{re.X}) for the entire regular expression.  This is useful if
you wish to include the flags as part of the regular expression, instead
of passing a \var{flag} argument to the \function{compile()} function. 

\item[\code{(?:...)}] A non-grouping version of regular parentheses.
Matches whatever regular expression is inside the parentheses, but the
substring matched by the 
group \emph{cannot} be retrieved after performing a match or
referenced later in the pattern. 

\item[\code{(?P<\var{name}>...)}] Similar to regular parentheses, but
the substring matched by the group is accessible via the symbolic group
name \var{name}.  Group names must be valid Python identifiers.  A
symbolic group is also a numbered group, just as if the group were not
named.  So the group named 'id' in the example above can also be
referenced as the numbered group 1.

For example, if the pattern is
\regexp{(?P<id>[a-zA-Z_]\e w*)}, the group can be referenced by its
name in arguments to methods of match objects, such as \code{m.group('id')}
or \code{m.end('id')}, and also by name in pattern text
(e.g. \regexp{(?P=id)}) and replacement text (e.g. \code{\e g<id>}).

\item[\code{(?P=\var{name})}] Matches whatever text was matched by the
earlier group named \var{name}.

\item[\code{(?\#...)}] A comment; the contents of the parentheses are
simply ignored.

\item[\code{(?=...)}] Matches if \regexp{...} matches next, but doesn't
consume any of the string.  This is called a lookahead assertion.  For
example, \regexp{Isaac (?=Asimov)} will match \code{'Isaac~'} only if it's
followed by \code{'Asimov'}.

\item[\code{(?!...)}] Matches if \regexp{...} doesn't match next.  This
is a negative lookahead assertion.  For example,
\regexp{Isaac (?!Asimov)} will match \code{'Isaac~'} only if it's \emph{not}
followed by \code{'Asimov'}.

\end{list}

The special sequences consist of \character{\e} and a character from the
list below.  If the ordinary character is not on the list, then the
resulting RE will match the second character.  For example,
\regexp{\e\$} matches the character \character{\$}.

\begin{list}{}{\leftmargin 0.7in \labelwidth 0.65in}

%
\item[\code{\e \var{number}}] Matches the contents of the group of the
same number.  Groups are numbered starting from 1.  For example,
\regexp{(.+) \e 1} matches \code{'the the'} or \code{'55 55'}, but not
\code{'the end'} (note 
the space after the group).  This special sequence can only be used to
match one of the first 99 groups.  If the first digit of \var{number}
is 0, or \var{number} is 3 octal digits long, it will not be interpreted
as a group match, but as the character with octal value \var{number}.
Inside the \character{[} and \character{]} of a character class, all numeric
escapes are treated as characters. 
%
\item[\code{\e A}] Matches only at the start of the string.
%
\item[\code{\e b}] Matches the empty string, but only at the
beginning or end of a word.  A word is defined as a sequence of
alphanumeric characters, so the end of a word is indicated by
whitespace or a non-alphanumeric character.  Inside a character range,
\regexp{\e b} represents the backspace character, for compatibility with
Python's string literals.
%
\item[\code{\e B}] Matches the empty string, but only when it is
\emph{not} at the beginning or end of a word.
%
\item[\code{\e d}]Matches any decimal digit; this is
equivalent to the set \regexp{[0-9]}.
%
\item[\code{\e D}]Matches any non-digit character; this is
equivalent to the set \regexp{[{\^}0-9]}.
%
\item[\code{\e s}]Matches any whitespace character; this is
equivalent to the set \regexp{[ \e t\e n\e r\e f\e v]}.
%
\item[\code{\e S}]Matches any non-whitespace character; this is
equivalent to the set \regexp{[\^\ \e t\e n\e r\e f\e v]}.
%
\item[\code{\e w}]When the \constant{LOCALE} flag is not specified,
matches any alphanumeric character; this is equivalent to the set
\regexp{[a-zA-Z0-9_]}.  With \constant{LOCALE}, it will match the set
\regexp{[0-9_]} plus whatever characters are defined as letters for the
current locale.
%
\item[\code{\e W}]When the \constant{LOCALE} flag is not specified,
matches any non-alphanumeric character; this is equivalent to the set
\regexp{[{\^}a-zA-Z0-9_]}.   With \constant{LOCALE}, it will match any
character not in the set \regexp{[0-9_]}, and not defined as a letter
for the current locale.

\item[\code{\e Z}]Matches only at the end of the string.
%

\item[\code{\e \e}] Matches a literal backslash.

\end{list}


\subsection{Matching vs. Searching \label{matching-searching}}
\sectionauthor{Fred L. Drake, Jr.}{fdrake@acm.org}

\strong{XXX This section is still incomplete!}

Python offers two different primitive operations based on regular
expressions: match and search.  If you are accustomed to Perl's
semantics, the search operation is what you're looking for.  See the
\function{search()} function and corresponding method of compiled
regular expression objects.

Note that match may differ from search using a regular expression
beginning with \character{\^}:  \character{\^} matches only at the start
of the string, or in \constant{MULTILINE} mode also immediately
following a newline.  "match" succeeds only if the pattern matches at
the start of the string regardless of mode, or at the starting
position given by the optional \var{pos} argument regardless of
whether a newline precedes it.

% Examples from Tim Peters:
\begin{verbatim}
re.compile("a").match("ba", 1)           # succeeds
re.compile("^a").search("ba", 1)         # fails; 'a' not at start
re.compile("^a").search("\na", 1)        # fails; 'a' not at start
re.compile("^a", re.M).search("\na", 1)  # succeeds
re.compile("^a", re.M).search("ba", 1)   # fails; no preceding \n
\end{verbatim}


\subsection{Module Contents}
\nodename{Contents of Module re}

The module defines the following functions and constants, and an exception:


\begin{funcdesc}{compile}{pattern\optional{, flags}}
  Compile a regular expression pattern into a regular expression
  object, which can be used for matching using its \function{match()} and
  \function{search()} methods, described below.  

  The expression's behaviour can be modified by specifying a
  \var{flags} value.  Values can be any of the following variables,
  combined using bitwise OR (the \code{|} operator).

The sequence

\begin{verbatim}
prog = re.compile(pat)
result = prog.match(str)
\end{verbatim}

is equivalent to

\begin{verbatim}
result = re.match(pat, str)
\end{verbatim}

but the version using \function{compile()} is more efficient when the
expression will be used several times in a single program.
%(The compiled version of the last pattern passed to
%\function{regex.match()} or \function{regex.search()} is cached, so
%programs that use only a single regular expression at a time needn't
%worry about compiling regular expressions.)
\end{funcdesc}

\begin{datadesc}{I}
\dataline{IGNORECASE}
Perform case-insensitive matching; expressions like \regexp{[A-Z]} will match
lowercase letters, too.  This is not affected by the current locale.
\end{datadesc}

\begin{datadesc}{L}
\dataline{LOCALE}
Make \regexp{\e w}, \regexp{\e W}, \regexp{\e b},
\regexp{\e B}, dependent on the current locale. 
\end{datadesc}

\begin{datadesc}{M}
\dataline{MULTILINE}
When specified, the pattern character \character{\^} matches at the
beginning of the string and at the beginning of each line
(immediately following each newline); and the pattern character
\character{\$} matches at the end of the string and at the end of each line
(immediately preceding each newline).
By default, \character{\^} matches only at the beginning of the string, and
\character{\$} only at the end of the string and immediately before the
newline (if any) at the end of the string. 
\end{datadesc}

\begin{datadesc}{S}
\dataline{DOTALL}
Make the \character{.} special character match any character at all, including a
newline; without this flag, \character{.} will match anything \emph{except}
a newline.
\end{datadesc}

\begin{datadesc}{X}
\dataline{VERBOSE}
This flag allows you to write regular expressions that look nicer.
Whitespace within the pattern is ignored, 
except when in a character class or preceded by an unescaped
backslash, and, when a line contains a \character{\#} neither in a character
class or preceded by an unescaped backslash, all characters from the
leftmost such \character{\#} through the end of the line are ignored.
% XXX should add an example here
\end{datadesc}


\begin{funcdesc}{search}{pattern, string\optional{, flags}}
  Scan through \var{string} looking for a location where the regular
  expression \var{pattern} produces a match, and return a
  corresponding \class{MatchObject} instance.
  Return \code{None} if no
  position in the string matches the pattern; note that this is
  different from finding a zero-length match at some point in the string.
\end{funcdesc}

\begin{funcdesc}{match}{pattern, string\optional{, flags}}
  If zero or more characters at the beginning of \var{string} match
  the regular expression \var{pattern}, return a corresponding
  \class{MatchObject} instance.  Return \code{None} if the string does not
  match the pattern; note that this is different from a zero-length
  match.

  \strong{Note:}  If you want to locate a match anywhere in
  \var{string}, use \method{search()} instead.
\end{funcdesc}

\begin{funcdesc}{split}{pattern, string, \optional{, maxsplit\code{ = 0}}}
  Split \var{string} by the occurrences of \var{pattern}.  If
  capturing parentheses are used in \var{pattern}, then the text of all
  groups in the pattern are also returned as part of the resulting list.
  If \var{maxsplit} is nonzero, at most \var{maxsplit} splits
  occur, and the remainder of the string is returned as the final
  element of the list.  (Incompatibility note: in the original Python
  1.5 release, \var{maxsplit} was ignored.  This has been fixed in
  later releases.)

\begin{verbatim}
>>> re.split('\W+', 'Words, words, words.')
['Words', 'words', 'words', '']
>>> re.split('(\W+)', 'Words, words, words.')
['Words', ', ', 'words', ', ', 'words', '.', '']
>>> re.split('\W+', 'Words, words, words.', 1)
['Words', 'words, words.']
\end{verbatim}

  This function combines and extends the functionality of
  the old \function{regsub.split()} and \function{regsub.splitx()}.
\end{funcdesc}

\begin{funcdesc}{findall}{pattern, string}
\versionadded{1.5.2}
Return a list of all non-overlapping matches of \var{pattern} in
\var{string}.  If one or more groups are present in the pattern,
return a list of groups; this will be a list of tuples if the pattern
has more than one group.  Empty matches are included in the result.
\end{funcdesc}

\begin{funcdesc}{sub}{pattern, repl, string\optional{, count\code{ = 0}}}
Return the string obtained by replacing the leftmost non-overlapping
occurrences of \var{pattern} in \var{string} by the replacement
\var{repl}.  If the pattern isn't found, \var{string} is returned
unchanged.  \var{repl} can be a string or a function; if a function,
it is called for every non-overlapping occurance of \var{pattern}.
The function takes a single match object argument, and returns the
replacement string.  For example:

\begin{verbatim}
>>> def dashrepl(matchobj):
....    if matchobj.group(0) == '-': return ' '
....    else: return '-'
>>> re.sub('-{1,2}', dashrepl, 'pro----gram-files')
'pro--gram files'
\end{verbatim}

The pattern may be a string or a 
regex object; if you need to specify
regular expression flags, you must use a regex object, or use
embedded modifiers in a pattern; e.g.
\samp{sub("(?i)b+", "x", "bbbb BBBB")} returns \code{'x x'}.

The optional argument \var{count} is the maximum number of pattern
occurrences to be replaced; \var{count} must be a non-negative integer, and
the default value of 0 means to replace all occurrences.

Empty matches for the pattern are replaced only when not adjacent to a
previous match, so \samp{sub('x*', '-', 'abc')} returns \code{'-a-b-c-'}.

If \var{repl} is a string, any backslash escapes in it are processed.
That is, \samp{\e n} is converted to a single newline character,
\samp{\e r} is converted to a linefeed, and so forth.  Unknown escapes
such as \samp{\e j} are left alone.  Backreferences, such as \samp{\e 6}, are
replaced with the substring matched by group 6 in the pattern. 

In addition to character escapes and backreferences as described
above, \samp{\e g<name>} will use the substring matched by the group
named \samp{name}, as defined by the \regexp{(?P<name>...)} syntax.
\samp{\e g<number>} uses the corresponding group number; \samp{\e
g<2>} is therefore equivalent to \samp{\e 2}, but isn't ambiguous in a
replacement such as \samp{\e g<2>0}.  \samp{\e 20} would be
interpreted as a reference to group 20, not a reference to group 2
followed by the literal character \character{0}.  
\end{funcdesc}

\begin{funcdesc}{subn}{pattern, repl, string\optional{, count\code{ = 0}}}
Perform the same operation as \function{sub()}, but return a tuple
\code{(\var{new_string}, \var{number_of_subs_made})}.
\end{funcdesc}

\begin{funcdesc}{escape}{string}
  Return \var{string} with all non-alphanumerics backslashed; this is
  useful if you want to match an arbitrary literal string that may have
  regular expression metacharacters in it.
\end{funcdesc}

\begin{excdesc}{error}
  Exception raised when a string passed to one of the functions here
  is not a valid regular expression (e.g., unmatched parentheses) or
  when some other error occurs during compilation or matching.  It is
  never an error if a string contains no match for a pattern.
\end{excdesc}


\subsection{Regular Expression Objects \label{re-objects}}

Compiled regular expression objects support the following methods and
attributes:

\begin{methoddesc}[RegexObject]{search}{string\optional{, pos}\optional{,
                                        endpos}}
  Scan through \var{string} looking for a location where this regular
  expression produces a match, and return a
  corresponding \class{MatchObject} instance.  Return \code{None} if no
  position in the string matches the pattern; note that this is
  different from finding a zero-length match at some point in the string.
  
  The optional \var{pos} and \var{endpos} parameters have the same
  meaning as for the \method{match()} method.
\end{methoddesc}

\begin{methoddesc}[RegexObject]{match}{string\optional{, pos}\optional{,
                                       endpos}}
  If zero or more characters at the beginning of \var{string} match
  this regular expression, return a corresponding
  \class{MatchObject} instance.  Return \code{None} if the string does not
  match the pattern; note that this is different from a zero-length
  match.

  \strong{Note:}  If you want to locate a match anywhere in
  \var{string}, use \method{search()} instead.

  The optional second parameter \var{pos} gives an index in the string
  where the search is to start; it defaults to \code{0}.  This is not
  completely equivalent to slicing the string; the \code{'\^'} pattern
  character matches at the real beginning of the string and at positions
  just after a newline, but not necessarily at the index where the search
  is to start.

  The optional parameter \var{endpos} limits how far the string will
  be searched; it will be as if the string is \var{endpos} characters
  long, so only the characters from \var{pos} to \var{endpos} will be
  searched for a match.
\end{methoddesc}

\begin{methoddesc}[RegexObject]{split}{string, \optional{,
                                       maxsplit\code{ = 0}}}
Identical to the \function{split()} function, using the compiled pattern.
\end{methoddesc}

\begin{methoddesc}[RegexObject]{findall}{string}
Identical to the \function{findall()} function, using the compiled pattern.
\end{methoddesc}

\begin{methoddesc}[RegexObject]{sub}{repl, string\optional{, count\code{ = 0}}}
Identical to the \function{sub()} function, using the compiled pattern.
\end{methoddesc}

\begin{methoddesc}[RegexObject]{subn}{repl, string\optional{,
                                      count\code{ = 0}}}
Identical to the \function{subn()} function, using the compiled pattern.
\end{methoddesc}


\begin{memberdesc}[RegexObject]{flags}
The flags argument used when the regex object was compiled, or
\code{0} if no flags were provided.
\end{memberdesc}

\begin{memberdesc}[RegexObject]{groupindex}
A dictionary mapping any symbolic group names defined by 
\regexp{(?P<\var{id}>)} to group numbers.  The dictionary is empty if no
symbolic groups were used in the pattern.
\end{memberdesc}

\begin{memberdesc}[RegexObject]{pattern}
The pattern string from which the regex object was compiled.
\end{memberdesc}


\subsection{Match Objects \label{match-objects}}

\class{MatchObject} instances support the following methods and attributes:

\begin{methoddesc}[MatchObject]{group}{\optional{group1, group2, ...}}
Returns one or more subgroups of the match.  If there is a single
argument, the result is a single string; if there are
multiple arguments, the result is a tuple with one item per argument.
Without arguments, \var{group1} defaults to zero (i.e. the whole match
is returned).
If a \var{groupN} argument is zero, the corresponding return value is the
entire matching string; if it is in the inclusive range [1..99], it is
the string matching the the corresponding parenthesized group.  If a
group number is negative or larger than the number of groups defined
in the pattern, an \exception{IndexError} exception is raised.
If a group is contained in a part of the pattern that did not match,
the corresponding result is \code{None}.  If a group is contained in a 
part of the pattern that matched multiple times, the last match is
returned.

If the regular expression uses the \regexp{(?P<\var{name}>...)} syntax,
the \var{groupN} arguments may also be strings identifying groups by
their group name.  If a string argument is not used as a group name in 
the pattern, an \exception{IndexError} exception is raised.

A moderately complicated example:

\begin{verbatim}
m = re.match(r"(?P<int>\d+)\.(\d*)", '3.14')
\end{verbatim}

After performing this match, \code{m.group(1)} is \code{'3'}, as is
\code{m.group('int')}, and \code{m.group(2)} is \code{'14'}.
\end{methoddesc}

\begin{methoddesc}[MatchObject]{groups}{\optional{default}}
Return a tuple containing all the subgroups of the match, from 1 up to
however many groups are in the pattern.  The \var{default} argument is
used for groups that did not participate in the match; it defaults to
\code{None}.  (Incompatibility note: in the original Python 1.5
release, if the tuple was one element long, a string would be returned
instead.  In later versions (from 1.5.1 on), a singleton tuple is
returned in such cases.)
\end{methoddesc}

\begin{methoddesc}[MatchObject]{groupdict}{\optional{default}}
Return a dictionary containing all the \emph{named} subgroups of the
match, keyed by the subgroup name.  The \var{default} argument is
used for groups that did not participate in the match; it defaults to
\code{None}.
\end{methoddesc}

\begin{methoddesc}[MatchObject]{start}{\optional{group}}
\funcline{end}{\optional{group}}
Return the indices of the start and end of the substring
matched by \var{group}; \var{group} defaults to zero (meaning the whole
matched substring).
Return \code{None} if \var{group} exists but
did not contribute to the match.  For a match object
\var{m}, and a group \var{g} that did contribute to the match, the
substring matched by group \var{g} (equivalent to
\code{\var{m}.group(\var{g})}) is

\begin{verbatim}
m.string[m.start(g):m.end(g)]
\end{verbatim}

Note that
\code{m.start(\var{group})} will equal \code{m.end(\var{group})} if
\var{group} matched a null string.  For example, after \code{\var{m} =
re.search('b(c?)', 'cba')}, \code{\var{m}.start(0)} is 1,
\code{\var{m}.end(0)} is 2, \code{\var{m}.start(1)} and
\code{\var{m}.end(1)} are both 2, and \code{\var{m}.start(2)} raises
an \exception{IndexError} exception.
\end{methoddesc}

\begin{methoddesc}[MatchObject]{span}{\optional{group}}
For \class{MatchObject} \var{m}, return the 2-tuple
\code{(\var{m}.start(\var{group}), \var{m}.end(\var{group}))}.
Note that if \var{group} did not contribute to the match, this is
\code{(None, None)}.  Again, \var{group} defaults to zero.
\end{methoddesc}

\begin{memberdesc}[MatchObject]{pos}
The value of \var{pos} which was passed to the
\function{search()} or \function{match()} function.  This is the index into
the string at which the regex engine started looking for a match. 
\end{memberdesc}

\begin{memberdesc}[MatchObject]{endpos}
The value of \var{endpos} which was passed to the
\function{search()} or \function{match()} function.  This is the index into
the string beyond which the regex engine will not go.
\end{memberdesc}

\begin{memberdesc}[MatchObject]{re}
The regular expression object whose \method{match()} or
\method{search()} method produced this \class{MatchObject} instance.
\end{memberdesc}

\begin{memberdesc}[MatchObject]{string}
The string passed to \function{match()} or \function{search()}.
\end{memberdesc}

\begin{seealso}
\seetext{Jeffrey Friedl, \emph{Mastering Regular Expressions},
O'Reilly.  The Python material in this book dates from before the
\module{re} module, but it covers writing good regular expression
patterns in great detail.}
\end{seealso}


\section{Built-in Module \sectcode{regex}}
\label{module-regex}

\bimodindex{regex}
This module provides regular expression matching operations similar to
those found in Emacs.

\strong{Obsolescence note:}
This module is obsolete as of Python version 1.5; it is still being
maintained because much existing code still uses it.  All new code in
need of regular expressions should use the new \code{re} module, which
supports the more powerful and regular Perl-style regular expressions.
Existing code should be converted.  The standard library module
\code{reconvert} helps in converting \code{regex} style regular
expressions to \code{re} style regular expressions.  (For more
conversion help, see the URL
\file{http://starship.skyport.net/crew/amk/regex/regex-to-re.html}.)

By default the patterns are Emacs-style regular expressions
(with one exception).  There is
a way to change the syntax to match that of several well-known
\UNIX{} utilities.  The exception is that Emacs' \samp{\e s}
pattern is not supported, since the original implementation references
the Emacs syntax tables.

This module is 8-bit clean: both patterns and strings may contain null
bytes and characters whose high bit is set.

\strong{Please note:} There is a little-known fact about Python string
literals which means that you don't usually have to worry about
doubling backslashes, even though they are used to escape special
characters in string literals as well as in regular expressions.  This
is because Python doesn't remove backslashes from string literals if
they are followed by an unrecognized escape character.
\emph{However}, if you want to include a literal \dfn{backslash} in a
regular expression represented as a string literal, you have to
\emph{quadruple} it or enclose it in a singleton character class.
E.g.\  to extract \LaTeX\ \samp{\e section\{{\rm
\ldots}\}} headers from a document, you can use this pattern:
\code{'[\e ]section\{\e (.*\e )\}'}.  \emph{Another exception:}
the escape sequece \samp{\e b} is significant in string literals
(where it means the ASCII bell character) as well as in Emacs regular
expressions (where it stands for a word boundary), so in order to
search for a word boundary, you should use the pattern \code{'\e \e b'}.
Similarly, a backslash followed by a digit 0-7 should be doubled to
avoid interpretation as an octal escape.

\subsection{Regular Expressions}

A regular expression (or RE) specifies a set of strings that matches
it; the functions in this module let you check if a particular string
matches a given regular expression (or if a given regular expression
matches a particular string, which comes down to the same thing).

Regular expressions can be concatenated to form new regular
expressions; if \emph{A} and \emph{B} are both regular expressions,
then \emph{AB} is also an regular expression.  If a string \emph{p}
matches A and another string \emph{q} matches B, the string \emph{pq}
will match AB.  Thus, complex expressions can easily be constructed
from simpler ones like the primitives described here.  For details of
the theory and implementation of regular expressions, consult almost
any textbook about compiler construction.

% XXX The reference could be made more specific, say to 
% "Compilers: Principles, Techniques and Tools", by Alfred V. Aho, 
% Ravi Sethi, and Jeffrey D. Ullman, or some FA text.   

A brief explanation of the format of regular expressions follows.

Regular expressions can contain both special and ordinary characters.
Ordinary characters, like '\code{A}', '\code{a}', or '\code{0}', are
the simplest regular expressions; they simply match themselves.  You
can concatenate ordinary characters, so '\code{last}' matches the
characters 'last'.  (In the rest of this section, we'll write RE's in
\code{this special font}, usually without quotes, and strings to be
matched 'in single quotes'.)

Special characters either stand for classes of ordinary characters, or
affect how the regular expressions around them are interpreted.

The special characters are:
\begin{itemize}
\item[\code{.}] (Dot.)  Matches any character except a newline.
\item[\code{\^}] (Caret.)  Matches the start of the string.
\item[\code{\$}] Matches the end of the string.  
\code{foo} matches both 'foo' and 'foobar', while the regular
expression '\code{foo\$}' matches only 'foo'.
\item[\code{*}] Causes the resulting RE to
match 0 or more repetitions of the preceding RE.  \code{ab*} will
match 'a', 'ab', or 'a' followed by any number of 'b's.
\item[\code{+}] Causes the
resulting RE to match 1 or more repetitions of the preceding RE.
\code{ab+} will match 'a' followed by any non-zero number of 'b's; it
will not match just 'a'.
\item[\code{?}] Causes the resulting RE to
match 0 or 1 repetitions of the preceding RE.  \code{ab?} will
match either 'a' or 'ab'.

\item[\code{\e}] Either escapes special characters (permitting you to match
characters like '*?+\&\$'), or signals a special sequence; special
sequences are discussed below.  Remember that Python also uses the
backslash as an escape sequence in string literals; if the escape
sequence isn't recognized by Python's parser, the backslash and
subsequent character are included in the resulting string.  However,
if Python would recognize the resulting sequence, the backslash should
be repeated twice.  

\item[\code{[]}] Used to indicate a set of characters.  Characters can
be listed individually, or a range is indicated by giving two
characters and separating them by a '-'.  Special characters are
not active inside sets.  For example, \code{[akm\$]}
will match any of the characters 'a', 'k', 'm', or '\$'; \code{[a-z]} will
match any lowercase letter.  

If you want to include a \code{]} inside a
set, it must be the first character of the set; to include a \code{-},
place it as the first or last character. 

Characters \emph{not} within a range can be matched by including a
\code{\^} as the first character of the set; \code{\^} elsewhere will
simply match the '\code{\^}' character.  
\end{itemize}

The special sequences consist of '\code{\e}' and a character
from the list below.  If the ordinary character is not on the list,
then the resulting RE will match the second character.  For example,
\code{\e\$} matches the character '\$'.  Ones where the backslash
should be doubled in string literals are indicated.

\begin{itemize}
\item[\code{\e|}]\code{A\e|B}, where A and B can be arbitrary REs,
creates a regular expression that will match either A or B.  This can
be used inside groups (see below) as well.
%
\item[\code{\e( \e)}] Indicates the start and end of a group; the
contents of a group can be matched later in the string with the
\code{\e [1-9]} special sequence, described next.
%
{\fulllineitems\item[\code{\e \e 1, ... \e \e 7, \e 8, \e 9}]
Matches the contents of the group of the same
number.  For example, \code{\e (.+\e ) \e \e 1} matches 'the the' or
'55 55', but not 'the end' (note the space after the group).  This
special sequence can only be used to match one of the first 9 groups;
groups with higher numbers can be matched using the \code{\e v}
sequence.  (\code{\e 8} and \code{\e 9} don't need a double backslash
because they are not octal digits.)}
%
\item[\code{\e \e b}] Matches the empty string, but only at the
beginning or end of a word.  A word is defined as a sequence of
alphanumeric characters, so the end of a word is indicated by
whitespace or a non-alphanumeric character.
%
\item[\code{\e B}] Matches the empty string, but when it is \emph{not} at the
beginning or end of a word.
%
\item[\code{\e v}] Must be followed by a two digit decimal number, and
matches the contents of the group of the same number.  The group number must be between 1 and 99, inclusive.
%
\item[\code{\e w}]Matches any alphanumeric character; this is
equivalent to the set \code{[a-zA-Z0-9]}.
%
\item[\code{\e W}] Matches any non-alphanumeric character; this is
equivalent to the set \code{[\^a-zA-Z0-9]}.
\item[\code{\e <}] Matches the empty string, but only at the beginning of a
word.  A word is defined as a sequence of alphanumeric characters, so
the end of a word is indicated by whitespace or a non-alphanumeric 
character.
\item[\code{\e >}] Matches the empty string, but only at the end of a
word.

\item[\code{\e \e \e \e}] Matches a literal backslash.

% In Emacs, the following two are start of buffer/end of buffer.  In
% Python they seem to be synonyms for ^$.
\item[\code{\e `}] Like \code{\^}, this only matches at the start of the
string.
\item[\code{\e \e '}] Like \code{\$}, this only matches at the end of the
string.
% end of buffer
\end{itemize}

\subsection{Module Contents}

The module defines these functions, and an exception:

\renewcommand{\indexsubitem}{(in module regex)}

\begin{funcdesc}{match}{pattern\, string}
  Return how many characters at the beginning of \var{string} match
  the regular expression \var{pattern}.  Return \code{-1} if the
  string does not match the pattern (this is different from a
  zero-length match!).
\end{funcdesc}

\begin{funcdesc}{search}{pattern\, string}
  Return the first position in \var{string} that matches the regular
  expression \var{pattern}.  Return \code{-1} if no position in the string
  matches the pattern (this is different from a zero-length match
  anywhere!).
\end{funcdesc}

\begin{funcdesc}{compile}{pattern\optional{\, translate}}
  Compile a regular expression pattern into a regular expression
  object, which can be used for matching using its \code{match} and
  \code{search} methods, described below.  The optional argument
  \var{translate}, if present, must be a 256-character string
  indicating how characters (both of the pattern and of the strings to
  be matched) are translated before comparing them; the \code{i}-th
  element of the string gives the translation for the character with
  \ASCII{} code \code{i}.  This can be used to implement
  case-insensitive matching; see the \code{casefold} data item below.

  The sequence

\bcode\begin{verbatim}
prog = regex.compile(pat)
result = prog.match(str)
\end{verbatim}\ecode
%
is equivalent to

\bcode\begin{verbatim}
result = regex.match(pat, str)
\end{verbatim}\ecode
%
but the version using \code{compile()} is more efficient when multiple
regular expressions are used concurrently in a single program.  (The
compiled version of the last pattern passed to \code{regex.match()} or
\code{regex.search()} is cached, so programs that use only a single
regular expression at a time needn't worry about compiling regular
expressions.)
\end{funcdesc}

\begin{funcdesc}{set_syntax}{flags}
  Set the syntax to be used by future calls to \code{compile},
  \code{match} and \code{search}.  (Already compiled expression objects
  are not affected.)  The argument is an integer which is the OR of
  several flag bits.  The return value is the previous value of
  the syntax flags.  Names for the flags are defined in the standard
  module \code{regex_syntax}; read the file \file{regex_syntax.py} for
  more information.
\end{funcdesc}

\begin{funcdesc}{get_syntax}{}
  Returns the current value of the syntax flags as an integer.
\end{funcdesc}

\begin{funcdesc}{symcomp}{pattern\optional{\, translate}}
This is like \code{compile}, but supports symbolic group names: if a
parenthesis-enclosed group begins with a group name in angular
brackets, e.g. \code{'\e(<id>[a-z][a-z0-9]*\e)'}, the group can
be referenced by its name in arguments to the \code{group} method of
the resulting compiled regular expression object, like this:
\code{p.group('id')}.  Group names may contain alphanumeric characters
and \code{'_'} only.
\end{funcdesc}

\begin{excdesc}{error}
  Exception raised when a string passed to one of the functions here
  is not a valid regular expression (e.g., unmatched parentheses) or
  when some other error occurs during compilation or matching.  (It is
  never an error if a string contains no match for a pattern.)
\end{excdesc}

\begin{datadesc}{casefold}
A string suitable to pass as \var{translate} argument to
\code{compile} to map all upper case characters to their lowercase
equivalents.
\end{datadesc}

\noindent
Compiled regular expression objects support these methods:

\renewcommand{\indexsubitem}{(regex method)}
\begin{funcdesc}{match}{string\optional{\, pos}}
  Return how many characters at the beginning of \var{string} match
  the compiled regular expression.  Return \code{-1} if the string
  does not match the pattern (this is different from a zero-length
  match!).
  
  The optional second parameter \var{pos} gives an index in the string
  where the search is to start; it defaults to \code{0}.  This is not
  completely equivalent to slicing the string; the \code{'\^'} pattern
  character matches at the real begin of the string and at positions
  just after a newline, not necessarily at the index where the search
  is to start.
\end{funcdesc}

\begin{funcdesc}{search}{string\optional{\, pos}}
  Return the first position in \var{string} that matches the regular
  expression \code{pattern}.  Return \code{-1} if no position in the
  string matches the pattern (this is different from a zero-length
  match anywhere!).
  
  The optional second parameter has the same meaning as for the
  \code{match} method.
\end{funcdesc}

\begin{funcdesc}{group}{index\, index\, ...}
This method is only valid when the last call to the \code{match}
or \code{search} method found a match.  It returns one or more
groups of the match.  If there is a single \var{index} argument,
the result is a single string; if there are multiple arguments, the
result is a tuple with one item per argument.  If the \var{index} is
zero, the corresponding return value is the entire matching string; if
it is in the inclusive range [1..99], it is the string matching the
the corresponding parenthesized group (using the default syntax,
groups are parenthesized using \code{{\e}(} and \code{{\e})}).  If no
such group exists, the corresponding result is \code{None}.

If the regular expression was compiled by \code{symcomp} instead of
\code{compile}, the \var{index} arguments may also be strings
identifying groups by their group name.
\end{funcdesc}

\noindent
Compiled regular expressions support these data attributes:

\renewcommand{\indexsubitem}{(regex attribute)}

\begin{datadesc}{regs}
When the last call to the \code{match} or \code{search} method found a
match, this is a tuple of pairs of indices corresponding to the
beginning and end of all parenthesized groups in the pattern.  Indices
are relative to the string argument passed to \code{match} or
\code{search}.  The 0-th tuple gives the beginning and end or the
whole pattern.  When the last match or search failed, this is
\code{None}.
\end{datadesc}

\begin{datadesc}{last}
When the last call to the \code{match} or \code{search} method found a
match, this is the string argument passed to that method.  When the
last match or search failed, this is \code{None}.
\end{datadesc}

\begin{datadesc}{translate}
This is the value of the \var{translate} argument to
\code{regex.compile} that created this regular expression object.  If
the \var{translate} argument was omitted in the \code{regex.compile}
call, this is \code{None}.
\end{datadesc}

\begin{datadesc}{givenpat}
The regular expression pattern as passed to \code{compile} or
\code{symcomp}.
\end{datadesc}

\begin{datadesc}{realpat}
The regular expression after stripping the group names for regular
expressions compiled with \code{symcomp}.  Same as \code{givenpat}
otherwise.
\end{datadesc}

\begin{datadesc}{groupindex}
A dictionary giving the mapping from symbolic group names to numerical
group indices for regular expressions compiled with \code{symcomp}.
\code{None} otherwise.
\end{datadesc}

\section{\module{regsub} ---
         String operations using regular expressions}

\declaremodule{standard}{regsub}
\modulesynopsis{Substitution and splitting operations that use
                regular expressions.  \strong{Obsolete!}}


This module defines a number of functions useful for working with
regular expressions (see built-in module \refmodule{regex}).

Warning: these functions are not thread-safe.

\strong{Obsolescence note:}
This module is obsolete as of Python version 1.5; it is still being
maintained because much existing code still uses it.  All new code in
need of regular expressions should use the new \refmodule{re} module, which
supports the more powerful and regular Perl-style regular expressions.
Existing code should be converted.  The standard library module
\module{reconvert} helps in converting \refmodule{regex} style regular
expressions to \refmodule{re} style regular expressions.  (For more
conversion help, see Andrew Kuchling's\index{Kuchling, Andrew}
``regex-to-re HOWTO'' at
\url{http://www.python.org/doc/howto/regex-to-re/}.)


\begin{funcdesc}{sub}{pat, repl, str}
Replace the first occurrence of pattern \var{pat} in string
\var{str} by replacement \var{repl}.  If the pattern isn't found,
the string is returned unchanged.  The pattern may be a string or an
already compiled pattern.  The replacement may contain references
\samp{\e \var{digit}} to subpatterns and escaped backslashes.
\end{funcdesc}

\begin{funcdesc}{gsub}{pat, repl, str}
Replace all (non-overlapping) occurrences of pattern \var{pat} in
string \var{str} by replacement \var{repl}.  The same rules as for
\code{sub()} apply.  Empty matches for the pattern are replaced only
when not adjacent to a previous match, so e.g.
\code{gsub('', '-', 'abc')} returns \code{'-a-b-c-'}.
\end{funcdesc}

\begin{funcdesc}{split}{str, pat\optional{, maxsplit}}
Split the string \var{str} in fields separated by delimiters matching
the pattern \var{pat}, and return a list containing the fields.  Only
non-empty matches for the pattern are considered, so e.g.
\code{split('a:b', ':*')} returns \code{['a', 'b']} and
\code{split('abc', '')} returns \code{['abc']}.  The \var{maxsplit}
defaults to 0. If it is nonzero, only \var{maxsplit} number of splits
occur, and the remainder of the string is returned as the final
element of the list.
\end{funcdesc}

\begin{funcdesc}{splitx}{str, pat\optional{, maxsplit}}
Split the string \var{str} in fields separated by delimiters matching
the pattern \var{pat}, and return a list containing the fields as well
as the separators.  For example, \code{splitx('a:::b', ':*')} returns
\code{['a', ':::', 'b']}.  Otherwise, this function behaves the same
as \code{split}.
\end{funcdesc}

\begin{funcdesc}{capwords}{s\optional{, pat}}
Capitalize words separated by optional pattern \var{pat}.  The default
pattern uses any characters except letters, digits and underscores as
word delimiters.  Capitalization is done by changing the first
character of each word to upper case.
\end{funcdesc}

\begin{funcdesc}{clear_cache}{}
The regsub module maintains a cache of compiled regular expressions,
keyed on the regular expression string and the syntax of the regex
module at the time the expression was compiled.  This function clears
that cache.
\end{funcdesc}

\section{\module{struct} ---
         Interpret strings as packed binary data}
\declaremodule{builtin}{struct}

\modulesynopsis{Interpret strings as packed binary data.}

\indexii{C}{structures}
\indexiii{packing}{binary}{data}

This module performs conversions between Python values and C
structs represented as Python strings.  It uses \dfn{format strings}
(explained below) as compact descriptions of the lay-out of the C
structs and the intended conversion to/from Python values.  This can
be used in handling binary data stored in files or from network
connections, among other sources.

The module defines the following exception and functions:


\begin{excdesc}{error}
  Exception raised on various occasions; argument is a string
  describing what is wrong.
\end{excdesc}

\begin{funcdesc}{pack}{fmt, v1, v2, \textrm{\ldots}}
  Return a string containing the values
  \code{\var{v1}, \var{v2}, \textrm{\ldots}} packed according to the given
  format.  The arguments must match the values required by the format
  exactly.
\end{funcdesc}

\begin{funcdesc}{unpack}{fmt, string}
  Unpack the string (presumably packed by \code{pack(\var{fmt},
  \textrm{\ldots})}) according to the given format.  The result is a
  tuple even if it contains exactly one item.  The string must contain
  exactly the amount of data required by the format
  (\code{len(\var{string})} must equal \code{calcsize(\var{fmt})}).
\end{funcdesc}

\begin{funcdesc}{calcsize}{fmt}
  Return the size of the struct (and hence of the string)
  corresponding to the given format.
\end{funcdesc}

Format characters have the following meaning; the conversion between
C and Python values should be obvious given their types:

\begin{tableiv}{c|l|l|c}{samp}{Format}{C Type}{Python}{Notes}
  \lineiv{x}{pad byte}{no value}{}
  \lineiv{c}{\ctype{char}}{string of length 1}{}
  \lineiv{b}{\ctype{signed char}}{integer}{}
  \lineiv{B}{\ctype{unsigned char}}{integer}{}
  \lineiv{h}{\ctype{short}}{integer}{}
  \lineiv{H}{\ctype{unsigned short}}{integer}{}
  \lineiv{i}{\ctype{int}}{integer}{}
  \lineiv{I}{\ctype{unsigned int}}{long}{}
  \lineiv{l}{\ctype{long}}{integer}{}
  \lineiv{L}{\ctype{unsigned long}}{long}{}
  \lineiv{q}{\ctype{long long}}{long}{(1)}
  \lineiv{Q}{\ctype{unsigned long long}}{long}{(1)}
  \lineiv{f}{\ctype{float}}{float}{}
  \lineiv{d}{\ctype{double}}{float}{}
  \lineiv{s}{\ctype{char[]}}{string}{}
  \lineiv{p}{\ctype{char[]}}{string}{}
  \lineiv{P}{\ctype{void *}}{integer}{}
\end{tableiv}

\noindent
Notes:

\begin{description}
\item[(1)]
  The \character{q} and \character{Q} conversion codes are available in
  native mode only if the platform C compiler supports C \ctype{long long},
  or, on Windows, \ctype{__int64}.  They are always available in standard
  modes.
  \versionadded{2.2}
\end{description}


A format character may be preceded by an integral repeat count.  For
example, the format string \code{'4h'} means exactly the same as
\code{'hhhh'}.

Whitespace characters between formats are ignored; a count and its
format must not contain whitespace though.

For the \character{s} format character, the count is interpreted as the
size of the string, not a repeat count like for the other format
characters; for example, \code{'10s'} means a single 10-byte string, while
\code{'10c'} means 10 characters.  For packing, the string is
truncated or padded with null bytes as appropriate to make it fit.
For unpacking, the resulting string always has exactly the specified
number of bytes.  As a special case, \code{'0s'} means a single, empty
string (while \code{'0c'} means 0 characters).

The \character{p} format character can be used to encode a Pascal
string.  The first byte is the length of the stored string, with the
bytes of the string following.  If count is given, it is used as the
total number of bytes used, including the length byte.  If the string
passed in to \function{pack()} is too long, the stored representation
is truncated.  If the string is too short, padding is used to ensure
that exactly enough bytes are used to satisfy the count.

For the \character{I}, \character{L}, \character{q} and \character{Q}
format characters, the return value is a Python long integer.

For the \character{P} format character, the return value is a Python
integer or long integer, depending on the size needed to hold a
pointer when it has been cast to an integer type.  A \NULL{} pointer will
always be returned as the Python integer \code{0}. When packing pointer-sized
values, Python integer or long integer objects may be used.  For
example, the Alpha and Merced processors use 64-bit pointer values,
meaning a Python long integer will be used to hold the pointer; other
platforms use 32-bit pointers and will use a Python integer.

By default, C numbers are represented in the machine's native format
and byte order, and properly aligned by skipping pad bytes if
necessary (according to the rules used by the C compiler).

Alternatively, the first character of the format string can be used to
indicate the byte order, size and alignment of the packed data,
according to the following table:

\begin{tableiii}{c|l|l}{samp}{Character}{Byte order}{Size and alignment}
  \lineiii{@}{native}{native}
  \lineiii{=}{native}{standard}
  \lineiii{<}{little-endian}{standard}
  \lineiii{>}{big-endian}{standard}
  \lineiii{!}{network (= big-endian)}{standard}
\end{tableiii}

If the first character is not one of these, \character{@} is assumed.

Native byte order is big-endian or little-endian, depending on the
host system.  For example, Motorola and Sun processors are big-endian;
Intel and DEC processors are little-endian.

Native size and alignment are determined using the C compiler's
\keyword{sizeof} expression.  This is always combined with native byte
order.

Standard size and alignment are as follows: no alignment is required
for any type (so you have to use pad bytes);
\ctype{short} is 2 bytes;
\ctype{int} and \ctype{long} are 4 bytes;
\ctype{long long} (\ctype{__int64} on Windows) is 8 bytes;
\ctype{float} and \ctype{double} are 32-bit and 64-bit
IEEE floating point numbers, respectively.

Note the difference between \character{@} and \character{=}: both use
native byte order, but the size and alignment of the latter is
standardized.

The form \character{!} is available for those poor souls who claim they
can't remember whether network byte order is big-endian or
little-endian.

There is no way to indicate non-native byte order (force
byte-swapping); use the appropriate choice of \character{<} or
\character{>}.

The \character{P} format character is only available for the native
byte ordering (selected as the default or with the \character{@} byte
order character). The byte order character \character{=} chooses to
use little- or big-endian ordering based on the host system. The
struct module does not interpret this as native ordering, so the
\character{P} format is not available.

Examples (all using native byte order, size and alignment, on a
big-endian machine):

\begin{verbatim}
>>> from struct import *
>>> pack('hhl', 1, 2, 3)
'\x00\x01\x00\x02\x00\x00\x00\x03'
>>> unpack('hhl', '\x00\x01\x00\x02\x00\x00\x00\x03')
(1, 2, 3)
>>> calcsize('hhl')
8
\end{verbatim}

Hint: to align the end of a structure to the alignment requirement of
a particular type, end the format with the code for that type with a
repeat count of zero.  For example, the format \code{'llh0l'}
specifies two pad bytes at the end, assuming longs are aligned on
4-byte boundaries.  This only works when native size and alignment are
in effect; standard size and alignment does not enforce any alignment.

\begin{seealso}
  \seemodule{array}{Packed binary storage of homogeneous data.}
  \seemodule{xdrlib}{Packing and unpacking of XDR data.}
\end{seealso}

\section{\module{StringIO} ---
         Read and write strings as if they were files.}
\declaremodule{standard}{StringIO}


\modulesynopsis{Read and write strings as if they were files.}


This module implements a file-like class, \class{StringIO},
that reads and writes a string buffer (also known as \emph{memory
files}). See the description on file objects for operations.

\begin{classdesc}{StringIO}{\optional{buffer}}
When a \class{StringIO} object is created, it can be initialized
to an existing string by passing the string to the constructor.
If no string is given, the \class{StringIO} will start empty.
\end{classdesc}

The following methods of \class{StringIO} objects require special
mention:

\begin{methoddesc}{getvalue}{}
Retrieve the entire contents of the ``file'' at any time before the
\class{StringIO} object's \method{close()} method is called.
\end{methoddesc}

\begin{methoddesc}{close}{}
Free the memory buffer.
\end{methoddesc}


\section{\module{cStringIO} ---
         Faster version of \module{StringIO}, but not subclassable.}

\declaremodule{builtin}{cStringIO}
\modulesynopsis{Faster version of \module{StringIO}, but not subclassable.}
\sectionauthor{Fred L. Drake, Jr.}{fdrake@acm.org}

The module \module{cStringIO} provides an interface similar to that of
the \module{StringIO} module.  Heavy use of \class{StringIO.StringIO}
objects can be made more efficient by using the function
\function{StringIO()} from this module instead.

Since this module provides a factory function which returns objects of
built-in types, there's no way to build your own version using
subclassing.  Use the original \module{StringIO} module in that case.

%\section{Built-in Module \sectcode{soundex}}
\label{module-soundex}
\bimodindex{soundex}

\setindexsubitem{(in module soundex)}
The soundex algorithm takes an English word, and returns an
easily-computed hash of it; this hash is intended to be the same for
words that sound alike.  This module provides an interface to the
soundex algorithm.

Note that the soundex algorithm is quite simple-minded, and isn't
perfect by any measure.  Its main purpose is to help looking up names
in databases, when the name may be misspelled --- soundex hashes common
misspellings together.

\begin{funcdesc}{get_soundex}{string}
Return the soundex hash value for a word; it will always be a
6-character string.  \var{string} must contain the word to be hashed,
with no leading whitespace; the case of the word is ignored.
\end{funcdesc}

\begin{funcdesc}{sound_similar}{string1, string2}
Compare the word in \var{string1} with the word in \var{string2}; this
is equivalent to 
\code{get_soundex(\var{string1})} \code{==}
\code{get_soundex(\var{string2})}.
\end{funcdesc}


\chapter{Miscellaneous Services}
\label{misc}

The modules described in this chapter provide miscellaneous services
that are available in all Python versions.  Here's an overview:

\begin{description}

\item[math]
--- Mathematical functions (\function{sin()} etc.).

\item[cmath]
--- Mathematical functions for complex numbers.

\item[whrandom]
--- Floating point pseudo-random number generator.

\item[random]
--- Generate pseudo-random numbers with various common distributions.

\item[array]
--- Efficient arrays of uniformly typed numeric values.

\item[fileinput]
--- Perl-like iteration over lines from multiple input streams, with
``save in place'' capability.

\end{description}
			% Miscellaneous Services
\section{Built-in Module \sectcode{math}}
\label{module-math}

\bimodindex{math}
\renewcommand{\indexsubitem}{(in module math)}
This module is always available.
It provides access to the mathematical functions defined by the C
standard.
They are:

\begin{funcdesc}{acos}{x}
Return the arc cosine of \var{x}.
\end{funcdesc}

\begin{funcdesc}{asin}{x}
Return the arc sine of \var{x}.
\end{funcdesc}

\begin{funcdesc}{atan}{x}
Return the arc tangent of \var{x}.
\end{funcdesc}

\begin{funcdesc}{atan2}{x, y}
Return \code{atan(x / y)}.
\end{funcdesc}

\begin{funcdesc}{ceil}{x}
Return the ceiling of \var{x}.
\end{funcdesc}

\begin{funcdesc}{cos}{x}
Return the cosine of \var{x}.
\end{funcdesc}

\begin{funcdesc}{cosh}{x}
Return the hyperbolic cosine of \var{x}.
\end{funcdesc}

\begin{funcdesc}{exp}{x}
Return the exponential value $\mbox{e}^x$.
\end{funcdesc}

\begin{funcdesc}{fabs}{x}
Return the absolute value of the real \var{x}.
\end{funcdesc}

\begin{funcdesc}{floor}{x}
Return the floor of \var{x}.
\end{funcdesc}

\begin{funcdesc}{fmod}{x, y}
Return \code{x \% y}.
\end{funcdesc}

\begin{funcdesc}{frexp}{x}
Return the matissa and exponent for \var{x}.  The mantissa is
positive.
\end{funcdesc}

\begin{funcdesc}{hypot}{x, y}
Return the Euclidean distance, \code{sqrt(x*x + y*y)}.
\end{funcdesc}

\begin{funcdesc}{ldexp}{x, i}
Return $x {\times} 2^i$.
\end{funcdesc}

\begin{funcdesc}{modf}{x}
Return the fractional and integer parts of \var{x}.  Both results
carry the sign of \var{x}.
\end{funcdesc}

\begin{funcdesc}{pow}{x, y}
Return $x^y$.
\end{funcdesc}

\begin{funcdesc}{sin}{x}
Return the sine of \var{x}.
\end{funcdesc}

\begin{funcdesc}{sinh}{x}
Return the hyperbolic sine of \var{x}.
\end{funcdesc}

\begin{funcdesc}{sqrt}{x}
Return the square root of \var{x}.
\end{funcdesc}

\begin{funcdesc}{tan}{x}
Return the tangent of \var{x}.
\end{funcdesc}

\begin{funcdesc}{tanh}{x}
Return the hyperbolic tangent of \var{x}.
\end{funcdesc}

Note that \code{frexp} and \code{modf} have a different call/return
pattern than their C equivalents: they take a single argument and
return a pair of values, rather than returning their second return
value through an `output parameter' (there is no such thing in Python).

The module also defines two mathematical constants:

\begin{datadesc}{pi}
The mathematical constant \emph{pi}.
\end{datadesc}

\begin{datadesc}{e}
The mathematical constant \emph{e}.
\end{datadesc}

\begin{seealso}
  \seemodule{cmath}{Complex number versions of many of these functions.}
\end{seealso}

\section{Built-in Module \sectcode{cmath}}
\label{module-cmath}

\bimodindex{cmath}
This module is always available.
It provides access to mathematical functions for complex numbers.
The functions are:

\begin{funcdesc}{acos}{x}
Return the arc cosine of \var{x}.
\end{funcdesc}

\begin{funcdesc}{acosh}{x}
Return the hyperbolic arc cosine of \var{x}.
\end{funcdesc}

\begin{funcdesc}{asin}{x}
Return the arc sine of \var{x}.
\end{funcdesc}

\begin{funcdesc}{asinh}{x}
Return the hyperbolic arc sine of \var{x}.
\end{funcdesc}

\begin{funcdesc}{atan}{x}
Return the arc tangent of \var{x}.
\end{funcdesc}

\begin{funcdesc}{atanh}{x}
Return the hyperbolic arc tangent of \var{x}.
\end{funcdesc}

\begin{funcdesc}{cos}{x}
Return the cosine of \var{x}.
\end{funcdesc}

\begin{funcdesc}{cosh}{x}
Return the hyperbolic cosine of \var{x}.
\end{funcdesc}

\begin{funcdesc}{exp}{x}
Return the exponential value \code{e**\var{x}}.
\end{funcdesc}

\begin{funcdesc}{log}{x}
Return the natural logarithm of \var{x}.
\end{funcdesc}

\begin{funcdesc}{log10}{x}
Return the base-10 logarithm of \var{x}.
\end{funcdesc}

\begin{funcdesc}{sin}{x}
Return the sine of \var{x}.
\end{funcdesc}

\begin{funcdesc}{sinh}{x}
Return the hyperbolic sine of \var{x}.
\end{funcdesc}

\begin{funcdesc}{sqrt}{x}
Return the square root of \var{x}.
\end{funcdesc}

\begin{funcdesc}{tan}{x}
Return the tangent of \var{x}.
\end{funcdesc}

\begin{funcdesc}{tanh}{x}
Return the hyperbolic tangent of \var{x}.
\end{funcdesc}

The module also defines two mathematical constants:

\begin{datadesc}{pi}
The mathematical constant \emph{pi}, as a real.
\end{datadesc}

\begin{datadesc}{e}
The mathematical constant \emph{e}, as a real.
\end{datadesc}

Note that the selection of functions is similar, but not identical, to
that in module \code{math}\refbimodindex{math}.  The reason for having
two modules is, that some users aren't interested in complex numbers,
and perhaps don't even know what they are.  They would rather have
\code{math.sqrt(-1)} raise an exception than return a complex number.
Also note that the functions defined in \code{cmath} always return a
complex number, even if the answer can be expressed as a real number
(in which case the complex number has an imaginary part of zero).

\section{\module{whrandom} ---
         Pseudo-random number generator}

\declaremodule{standard}{whrandom}
\modulesynopsis{Floating point pseudo-random number generator.}

\deprecated{2.1}{Use \refmodule{random} instead.}

\note{This module was an implementation detail of the
\refmodule{random} module in releases of Python prior to 2.1.  It is
no longer used.  Please do not use this module directly; use
\refmodule{random} instead.}

This module implements a Wichmann-Hill pseudo-random number generator
class that is also named \class{whrandom}.  Instances of the
\class{whrandom} class conform to the Random Number Generator
interface described in the docs for the \refmodule{random} module.
They also offer the 
following method, specific to the Wichmann-Hill algorithm:

\begin{methoddesc}[whrandom]{seed}{\optional{x, y, z}}
  Initializes the random number generator from the integers \var{x},
  \var{y} and \var{z}.  When the module is first imported, the random
  number is initialized using values derived from the current time.
  If \var{x}, \var{y}, and \var{z} are either omitted or \code{0}, the 
  seed will be computed from the current system time.  If one or two
  of the parameters are \code{0}, but not all three, the zero values
  are replaced by ones.  This causes some apparently different seeds
  to be equal, with the corresponding result on the pseudo-random
  series produced by the generator.
\end{methoddesc}

Other supported methods include:

\begin{funcdesc}{choice}{seq}
Chooses a random element from the non-empty sequence \var{seq} and returns it.
\end{funcdesc}

\begin{funcdesc}{randint}{a, b}
Returns a random integer \var{N} such that \code{\var{a}<=\var{N}<=\var{b}}.
\end{funcdesc}

\begin{funcdesc}{random}{}
Returns the next random floating point number in the range [0.0 ... 1.0).
\end{funcdesc}

\begin{funcdesc}{seed}{x, y, z}
Initializes the random number generator from the integers \var{x},
\var{y} and \var{z}.  When the module is first imported, the random
number is initialized using values derived from the current time.
\end{funcdesc}

\begin{funcdesc}{uniform}{a, b}
Returns a random real number \var{N} such that \code{\var{a}<=\var{N}<\var{b}}.
\end{funcdesc}

When imported, the \module{whrandom} module also creates an instance of
the \class{whrandom} class, and makes the methods of that instance
available at the module level.  Therefore one can write either 
\code{N = whrandom.random()} or:

\begin{verbatim}
generator = whrandom.whrandom()
N = generator.random()
\end{verbatim}

Note that using separate instances of the generator leads to
independent sequences of pseudo-random numbers.

\begin{seealso}
  \seemodule{random}{Generators for various random distributions and
                     documentation for the Random Number Generator
                     interface.}
  \seetext{Wichmann, B. A. \& Hill, I. D., ``Algorithm AS 183: 
           An efficient and portable pseudo-random number generator'',
           \citetitle{Applied Statistics} 31 (1982) 188-190.}
\end{seealso}

\section{Standard Module \sectcode{random}}
\label{module-random}
\stmodindex{random}

This module implements pseudo-random number generators for various
distributions: on the real line, there are functions to compute normal
or Gaussian, lognormal, negative exponential, gamma, and beta
distributions.  For generating distribution of angles, the circular
uniform and von Mises distributions are available.

The module exports the following functions, which are exactly
equivalent to those in the \code{whrandom} module: \code{choice},
\code{randint}, \code{random}, \code{uniform}.  See the documentation
for the \code{whrandom} module for these functions.

The following functions specific to the \code{random} module are also
defined, and all return real values.  Function parameters are named
after the corresponding variables in the distribution's equation, as
used in common mathematical practice; most of these equations can be
found in any statistics text.

\renewcommand{\indexsubitem}{(in module random)}
\begin{funcdesc}{betavariate}{alpha\, beta}
Beta distribution.  Conditions on the parameters are \code{alpha>-1}
and \code{beta>-1}.
Returned values will range between 0 and 1.
\end{funcdesc}

\begin{funcdesc}{cunifvariate}{mean\, arc}
Circular uniform distribution.  \var{mean} is the mean angle, and
\var{arc} is the range of the distribution, centered around the mean
angle.  Both values must be expressed in radians, and can range
between 0 and \code{pi}.  Returned values will range between
\code{mean - arc/2} and \code{mean + arc/2}.
\end{funcdesc}

\begin{funcdesc}{expovariate}{lambd}
Exponential distribution.  \var{lambd} is 1.0 divided by the desired mean.
(The parameter would be called ``lambda'', but that's also a reserved
word in Python.)  Returned values will range from 0 to positive infinity.
\end{funcdesc}

\begin{funcdesc}{gamma}{alpha\, beta}
Gamma distribution.  (\emph{Not} the gamma function!) 
Conditions on the parameters are \code{alpha>-1} and \code{beta>0}.
\end{funcdesc}

\begin{funcdesc}{gauss}{mu\, sigma}
Gaussian distribution.  \var{mu} is the mean, and \var{sigma} is the
standard deviation.  This is slightly faster than the
\code{normalvariate} function defined below.
\end{funcdesc}

\begin{funcdesc}{lognormvariate}{mu\, sigma}
Log normal distribution.  If you take the natural logarithm of this
distribution, you'll get a normal distribution with mean \var{mu} and
standard deviation \var{sigma}  \var{mu} can have any value, and \var{sigma}
must be greater than zero.  
\end{funcdesc}

\begin{funcdesc}{normalvariate}{mu\, sigma}
Normal distribution.  \var{mu} is the mean, and \var{sigma} is the
standard deviation.
\end{funcdesc}

\begin{funcdesc}{vonmisesvariate}{mu\, kappa}
\var{mu} is the mean angle, expressed in radians between 0 and pi,
and \var{kappa} is the concentration parameter, which must be greater
then or equal to zero.  If \var{kappa} is equal to zero, this
distribution reduces to a uniform random angle over the range 0 to
\code{2*pi}.
\end{funcdesc}


\begin{seealso}
\seemodule{whrandom}{the standard Python random number generator}
\end{seealso}

%\section{Standard Module \sectcode{rand}}
\label{module-rand}
\stmodindex{rand}

The \code{rand} module simulates the C library's \code{rand()}
interface, though the results aren't necessarily compatible with any
given library's implementation.  While still supported for
compatibility, the \code{rand} module is now considered obsolete; if
possible, use the \code{whrandom} module instead.

\begin{funcdesc}{choice}{seq}
Returns a random element from the sequence \var{seq}.
\end{funcdesc}

\begin{funcdesc}{rand}{}
Return a random integer between 0 and 32767, inclusive.
\end{funcdesc}

\begin{funcdesc}{srand}{seed}
Set a starting seed value for the random number generator; \var{seed}
can be an arbitrary integer. 
\end{funcdesc}

\begin{seealso}
\seemodule{whrandom}{the standard Python random number generator}
\end{seealso}


\section{\module{bisect} ---
         Array bisection algorithm}

\declaremodule{standard}{bisect}
\modulesynopsis{Array bisection algorithms for binary searching.}
\sectionauthor{Fred L. Drake, Jr.}{fdrake@acm.org}
% LaTeX produced by Fred L. Drake, Jr. <fdrake@acm.org>, with an
% example based on the PyModules FAQ entry by Aaron Watters
% <arw@pythonpros.com>.


This module provides support for maintaining a list in sorted order
without having to sort the list after each insertion.  For long lists
of items with expensive comparison operations, this can be an
improvement over the more common approach.  The module is called
\module{bisect} because it uses a basic bisection algorithm to do its
work.  The source code may be most useful as a working example of the
algorithm (the boundary conditions are already right!).

The following functions are provided:

\begin{funcdesc}{bisect_left}{list, item\optional{, lo\optional{, hi}}}
  Locate the proper insertion point for \var{item} in \var{list} to
  maintain sorted order.  The parameters \var{lo} and \var{hi} may be
  used to specify a subset of the list which should be considered; by
  default the entire list is used.  If \var{item} is already present
  in \var{list}, the insertion point will be before (to the left of)
  any existing entries.  The return value is suitable for use as the
  first parameter to \code{\var{list}.insert()}.  This assumes that
  \var{list} is already sorted.
\versionadded{2.1}
\end{funcdesc}

\begin{funcdesc}{bisect_right}{list, item\optional{, lo\optional{, hi}}}
  Similar to \function{bisect_left()}, but returns an insertion point
  which comes after (to the right of) any existing entries of
  \var{item} in \var{list}.
\versionadded{2.1}
\end{funcdesc}

\begin{funcdesc}{bisect}{\unspecified}
  Alias for \function{bisect_right()}.
\end{funcdesc}

\begin{funcdesc}{insort_left}{list, item\optional{, lo\optional{, hi}}}
  Insert \var{item} in \var{list} in sorted order.  This is equivalent
  to \code{\var{list}.insert(bisect.bisect_left(\var{list}, \var{item},
  \var{lo}, \var{hi}), \var{item})}.  This assumes that \var{list} is
  already sorted.
\versionadded{2.1}
\end{funcdesc}

\begin{funcdesc}{insort_right}{list, item\optional{, lo\optional{, hi}}}
  Similar to \function{insort_left()}, but inserting \var{item} in
  \var{list} after any existing entries of \var{item}.
\versionadded{2.1}
\end{funcdesc}

\begin{funcdesc}{insort}{\unspecified}
  Alias for \function{insort_right()}.
\end{funcdesc}


\subsection{Examples}
\nodename{bisect-example}

The \function{bisect()} function is generally useful for categorizing
numeric data.  This example uses \function{bisect()} to look up a
letter grade for an exam total (say) based on a set of ordered numeric
breakpoints: 85 and up is an `A', 75..84 is a `B', etc.

\begin{verbatim}
>>> grades = "FEDCBA"
>>> breakpoints = [30, 44, 66, 75, 85]
>>> from bisect import bisect
>>> def grade(total):
...           return grades[bisect(breakpoints, total)]
...
>>> grade(66)
'C'
>>> map(grade, [33, 99, 77, 44, 12, 88])
['E', 'A', 'B', 'D', 'F', 'A']
\end{verbatim}

The bisect module can be used with the Queue module to implement a priority
queue (example courtesy of Fredrik Lundh): \index{Priority Queue}

\begin{verbatim}
import Queue, bisect

class PriorityQueue(Queue.Queue):
    def _put(self, item):
        bisect.insort(self.queue, item)

# usage
queue = PriorityQueue(0)
queue.put((2, "second"))
queue.put((1, "first"))
queue.put((3, "third"))
priority, value = queue.get()
\end{verbatim}

\section{\module{array} ---
         Efficient arrays of numeric values}

\declaremodule{builtin}{array}
\modulesynopsis{Efficient arrays of uniformly typed numeric values.}


This module defines a new object type which can efficiently represent
an array of basic values: characters, integers, floating point
numbers.  Arrays\index{arrays} are sequence types and behave very much
like lists, except that the type of objects stored in them is
constrained.  The type is specified at object creation time by using a
\dfn{type code}, which is a single character.  The following type
codes are defined:

\begin{tableiii}{c|l|c}{code}{Type code}{C Type}{Minimum size in bytes}
\lineiii{'c'}{character}{1}
\lineiii{'b'}{signed int}{1}
\lineiii{'B'}{unsigned int}{1}
\lineiii{'h'}{signed int}{2}
\lineiii{'H'}{unsigned int}{2}
\lineiii{'i'}{signed int}{2}
\lineiii{'I'}{unsigned int}{2}
\lineiii{'l'}{signed int}{4}
\lineiii{'L'}{unsigned int}{4}
\lineiii{'f'}{float}{4}
\lineiii{'d'}{double}{8}
\end{tableiii}

The actual representation of values is determined by the machine
architecture (strictly speaking, by the C implementation).  The actual
size can be accessed through the \member{itemsize} attribute.  The values
stored  for \code{'L'} and \code{'I'} items will be represented as
Python long integers when retrieved, because Python's plain integer
type cannot represent the full range of C's unsigned (long) integers.


The module defines the following function and type object:

\begin{funcdesc}{array}{typecode\optional{, initializer}}
Return a new array whose items are restricted by \var{typecode}, and
initialized from the optional \var{initializer} value, which must be a
list or a string.  The list or string is passed to the new array's
\method{fromlist()} or \method{fromstring()} method (see below) to add
initial items to the array.
\end{funcdesc}

\begin{datadesc}{ArrayType}
Type object corresponding to the objects returned by
\function{array()}.
\end{datadesc}


Array objects support the following data items and methods:

\begin{memberdesc}[array]{typecode}
The typecode character used to create the array.
\end{memberdesc}

\begin{memberdesc}[array]{itemsize}
The length in bytes of one array item in the internal representation.
\end{memberdesc}


\begin{methoddesc}[array]{append}{x}
Append a new item with value \var{x} to the end of the array.
\end{methoddesc}

\begin{methoddesc}[array]{buffer_info}{}
Return a tuple \code{(\var{address}, \var{length})} giving the current
memory address and the length in bytes of the buffer used to hold
array's contents.  This is occasionally useful when working with
low-level (and inherently unsafe) I/O interfaces that require memory
addresses, such as certain \cfunction{ioctl()} operations.  The returned
numbers are valid as long as the array exists and no length-changing
operations are applied to it.
\end{methoddesc}

\begin{methoddesc}[array]{byteswap}{x}
``Byteswap'' all items of the array.  This is only supported for
integer values.  It is useful when reading data from a file written
on a machine with a different byte order.
\end{methoddesc}

\begin{methoddesc}[array]{fromfile}{f, n}
Read \var{n} items (as machine values) from the file object \var{f}
and append them to the end of the array.  If less than \var{n} items
are available, \exception{EOFError} is raised, but the items that were
available are still inserted into the array.  \var{f} must be a real
built-in file object; something else with a \method{read()} method won't
do.
\end{methoddesc}

\begin{methoddesc}[array]{fromlist}{list}
Append items from the list.  This is equivalent to
\samp{for x in \var{list}:\ a.append(x)}
except that if there is a type error, the array is unchanged.
\end{methoddesc}

\begin{methoddesc}[array]{fromstring}{s}
Appends items from the string, interpreting the string as an
array of machine values (i.e. as if it had been read from a
file using the \method{fromfile()} method).
\end{methoddesc}

\begin{methoddesc}[array]{insert}{i, x}
Insert a new item with value \var{x} in the array before position
\var{i}.
\end{methoddesc}

\begin{methoddesc}[array]{read}{f, n}
\deprecated {1.5.1}
  {Use the \method{fromfile()} method.}
Read \var{n} items (as machine values) from the file object \var{f}
and append them to the end of the array.  If less than \var{n} items
are available, \exception{EOFError} is raised, but the items that were
available are still inserted into the array.  \var{f} must be a real
built-in file object; something else with a \method{read()} method won't
do.
\end{methoddesc}

\begin{methoddesc}[array]{reverse}{}
Reverse the order of the items in the array.
\end{methoddesc}

\begin{methoddesc}[array]{tofile}{f}
Write all items (as machine values) to the file object \var{f}.
\end{methoddesc}

\begin{methoddesc}[array]{tolist}{}
Convert the array to an ordinary list with the same items.
\end{methoddesc}

\begin{methoddesc}[array]{tostring}{}
Convert the array to an array of machine values and return the
string representation (the same sequence of bytes that would
be written to a file by the \method{tofile()} method.)
\end{methoddesc}

\begin{methoddesc}[array]{write}{f}
\deprecated {1.5.1}
  {Use the \method{tofile()} method.}
Write all items (as machine values) to the file object \var{f}.
\end{methoddesc}

When an array object is printed or converted to a string, it is
represented as \code{array(\var{typecode}, \var{initializer})}.  The
\var{initializer} is omitted if the array is empty, otherwise it is a
string if the \var{typecode} is \code{'c'}, otherwise it is a list of
numbers.  The string is guaranteed to be able to be converted back to
an array with the same type and value using reverse quotes
(\code{``}).  Examples:

\begin{verbatim}
array('l')
array('c', 'hello world')
array('l', [1, 2, 3, 4, 5])
array('d', [1.0, 2.0, 3.14])
\end{verbatim}


\begin{seealso}
  \seemodule{struct}{packing and unpacking of heterogeneous binary data}
  \seemodule{xdrlib{{packing and unpacking of XDR data}
\end{seealso}

\section{\module{fileinput} ---
         Iterate over lines from multiple input streams}
\declaremodule{standard}{fileinput}
\moduleauthor{Guido van Rossum}{guido@python.org}
\sectionauthor{Fred L. Drake, Jr.}{fdrake@acm.org}

\modulesynopsis{Perl-like iteration over lines from multiple input
streams, with ``save in place'' capability.}


This module implements a helper class and functions to quickly write a
loop over standard input or a list of files.

The typical use is:

\begin{verbatim}
import fileinput
for line in fileinput.input():
    process(line)
\end{verbatim}

This iterates over the lines of all files listed in
\code{sys.argv[1:]}, defaulting to \code{sys.stdin} if the list is
empty.  If a filename is \code{'-'}, it is also replaced by
\code{sys.stdin}.  To specify an alternative list of filenames, pass
it as the first argument to \function{input()}.  A single file name is
also allowed.

All files are opened in text mode.  If an I/O error occurs during
opening or reading a file, \exception{IOError} is raised.

If \code{sys.stdin} is used more than once, the second and further use
will return no lines, except perhaps for interactive use, or if it has
been explicitly reset (e.g. using \code{sys.stdin.seek(0)}).

Empty files are opened and immediately closed; the only time their
presence in the list of filenames is noticeable at all is when the
last file opened is empty.

It is possible that the last line of a file does not end in a newline
character; lines are returned including the trailing newline when it
is present.

The following function is the primary interface of this module:

\begin{funcdesc}{input}{\optional{files\optional{,
                       inplace\optional{, backup}}}}
  Create an instance of the \class{FileInput} class.  The instance
  will be used as global state for the functions of this module, and
  is also returned to use during iteration.
\end{funcdesc}


The following functions use the global state created by
\function{input()}; if there is no active state,
\exception{RuntimeError} is raised.

\begin{funcdesc}{filename}{}
  Return the name of the file currently being read.  Before the first
  line has been read, returns \code{None}.
\end{funcdesc}

\begin{funcdesc}{lineno}{}
  Return the cumulative line number of the line that has just been
  read.  Before the first line has been read, returns \code{0}.  After
  the last line of the last file has been read, returns the line
  number of that line.
\end{funcdesc}

\begin{funcdesc}{filelineno}{}
  Return the line number in the current file.  Before the first line
  has been read, returns \code{0}.  After the last line of the last
  file has been read, returns the line number of that line within the
  file.
\end{funcdesc}

\begin{funcdesc}{isfirstline}{}
  Returns true the line just read is the first line of its file,
  otherwise returns false.
\end{funcdesc}

\begin{funcdesc}{isstdin}{}
  Returns true if the last line was read from \code{sys.stdin},
  otherwise returns false.
\end{funcdesc}

\begin{funcdesc}{nextfile}{}
  Close the current file so that the next iteration will read the
  first line from the next file (if any); lines not read from the file
  will not count towards the cumulative line count.  The filename is
  not changed until after the first line of the next file has been
  read.  Before the first line has been read, this function has no
  effect; it cannot be used to skip the first file.  After the last
  line of the last file has been read, this function has no effect.
\end{funcdesc}

\begin{funcdesc}{close}{}
  Close the sequence.
\end{funcdesc}


The class which implements the sequence behavior provided by the
module is available for subclassing as well:

\begin{classdesc}{FileInput}{\optional{files\optional{,
                             inplace\optional{, backup}}}}
  Class \class{FileInput} is the implementation; its methods
  \method{filename()}, \method{lineno()}, \method{fileline()},
  \method{isfirstline()}, \method{isstdin()}, \method{nextfile()} and
  \method{close()} correspond to the functions of the same name in the
  module.  In addition it has a \method{readline()} method which
  returns the next input line, and a \method{__getitem__()} method
  which implements the sequence behavior.  The sequence must be
  accessed in strictly sequential order; random access and
  \method{readline()} cannot be mixed.
\end{classdesc}

\strong{Optional in-place filtering:} if the keyword argument
\code{\var{inplace}=1} is passed to \function{input()} or to the
\class{FileInput} constructor, the file is moved to a backup file and
standard output is directed to the input file.
This makes it possible to write a filter that rewrites its input file
in place.  If the keyword argument \code{\var{backup}='.<some
extension>'} is also given, it specifies the extension for the backup
file, and the backup file remains around; by default, the extension is
\code{'.bak'} and it is deleted when the output file is closed.  In-place
filtering is disabled when standard input is read.

\strong{Caveat:} The current implementation does not work for MS-DOS
8+3 filesystems.

% This section was contributed by Drew Csillag <drew_csillag@geocities.com>.

\section{\module{calendar} ---
         Functions that emulate the \UNIX{} \program{cal} program.}
\declaremodule{standard}{calendar}

\modulesynopsis{Functions that emulate the \UNIX{} \program{cal}
program.}


This module allows you to output calendars like the \UNIX{}
\manpage{cal}{1} program.

\begin{funcdesc}{isleap}{year}
Returns \code{1} if \var{year} is a leap year.
\end{funcdesc}

\begin{funcdesc}{leapdays}{year1, year2}
Return the number of leap years in the range
[\var{year1}\ldots\var{year2}].
\end{funcdesc}

\begin{funcdesc}{weekday}{year, month, day}
Returns the day of the week (\code{0} is Monday) for \var{year}
(\code{1970}--\ldots), \var{month} (\code{1}--\code{12}), \var{day}
(\code{1}--\code{31}).
\end{funcdesc}

\begin{funcdesc}{monthrange}{year, month}
Returns weekday of first day of the month and number of days in month, 
for the specified \var{year} and \var{month}.
\end{funcdesc}

\begin{funcdesc}{monthcalendar}{year, month}
Returns a matrix representing a month's calendar.  Each row represents
a week; days outside of the month a represented by zeros.
\end{funcdesc}

\begin{funcdesc}{prmonth}{year, month\optional{, width\optional{, length}}}
Prints a month's calendar.  If \var{width} is provided, it specifies
the width of the columns that the numbers are centered in.  If
\var{length} is given, it specifies the number of lines that each
week will use.
\end{funcdesc}

\begin{funcdesc}{prcal}{year}
Prints the calendar for the year \var{year}.
\end{funcdesc}

\section{\module{cmd} ---
         Support for line-oriented command interpreters}

\declaremodule{standard}{cmd}
\sectionauthor{Eric S. Raymond}{esr@snark.thyrsus.com}
\modulesynopsis{Build line-oriented command interpreters.}


The \class{Cmd} class provides a simple framework for writing
line-oriented command interpreters.  These are often useful for
test harnesses, administrative tools, and prototypes that will
later be wrapped in a more sophisticated interface.

\begin{classdesc}{Cmd}{\optional{completekey\optional{,
                       stdin\optional{, stdout}}}}
A \class{Cmd} instance or subclass instance is a line-oriented
interpreter framework.  There is no good reason to instantiate
\class{Cmd} itself; rather, it's useful as a superclass of an
interpreter class you define yourself in order to inherit
\class{Cmd}'s methods and encapsulate action methods.

The optional argument \var{completekey} is the \refmodule{readline} name
of a completion key; it defaults to \kbd{Tab}. If \var{completekey} is
not \constant{None} and \refmodule{readline} is available, command completion
is done automatically.

The optional arguments \var{stdin} and \var{stdout} specify the 
input and output file objects that the Cmd instance or subclass 
instance will use for input and output. If not specified, they
will default to \var{sys.stdin} and \var{sys.stdout}.

\versionchanged[The \var{stdin} and \var{stdout} parameters were added]{2.3}
\end{classdesc}

\subsection{Cmd Objects}
\label{Cmd-objects}

A \class{Cmd} instance has the following methods:

\begin{methoddesc}{cmdloop}{\optional{intro}}
Repeatedly issue a prompt, accept input, parse an initial prefix off
the received input, and dispatch to action methods, passing them the
remainder of the line as argument.

The optional argument is a banner or intro string to be issued before the
first prompt (this overrides the \member{intro} class member).

If the \refmodule{readline} module is loaded, input will automatically
inherit \program{bash}-like history-list editing (e.g. \kbd{Control-P}
scrolls back to the last command, \kbd{Control-N} forward to the next
one, \kbd{Control-F} moves the cursor to the right non-destructively,
\kbd{Control-B} moves the cursor to the left non-destructively, etc.).

An end-of-file on input is passed back as the string \code{'EOF'}.

An interpreter instance will recognize a command name \samp{foo} if
and only if it has a method \method{do_foo()}.  As a special case,
a line beginning with the character \character{?} is dispatched to
the method \method{do_help()}.  As another special case, a line
beginning with the character \character{!} is dispatched to the
method \method{do_shell()} (if such a method is defined).

This method will return when the \method{postcmd()} method returns a
true value.  The \var{stop} argument to \method{postcmd()} is the
return value from the command's corresponding \method{do_*()} method.

If completion is enabled, completing commands will be done
automatically, and completing of commands args is done by calling
\method{complete_foo()} with arguments \var{text}, \var{line},
\var{begidx}, and \var{endidx}.  \var{text} is the string prefix we
are attempting to match: all returned matches must begin with it.
\var{line} is the current input line with leading whitespace removed,
\var{begidx} and \var{endidx} are the beginning and ending indexes
of the prefix text, which could be used to provide different
completion depending upon which position the argument is in.

All subclasses of \class{Cmd} inherit a predefined \method{do_help()}.
This method, called with an argument \code{'bar'}, invokes the
corresponding method \method{help_bar()}.  With no argument,
\method{do_help()} lists all available help topics (that is, all
commands with corresponding \method{help_*()} methods), and also lists
any undocumented commands.
\end{methoddesc}

\begin{methoddesc}{onecmd}{str}
Interpret the argument as though it had been typed in response to the
prompt.  This may be overridden, but should not normally need to be;
see the \method{precmd()} and \method{postcmd()} methods for useful
execution hooks.  The return value is a flag indicating whether
interpretation of commands by the interpreter should stop.  If there
is a \method{do_*()} method for the command \var{str}, the return
value of that method is returned, otherwise the return value from the
\method{default()} method is returned.
\end{methoddesc}

\begin{methoddesc}{emptyline}{}
Method called when an empty line is entered in response to the prompt.
If this method is not overridden, it repeats the last nonempty command
entered.  
\end{methoddesc}

\begin{methoddesc}{default}{line}
Method called on an input line when the command prefix is not
recognized. If this method is not overridden, it prints an
error message and returns.
\end{methoddesc}

\begin{methoddesc}{completedefault}{text, line, begidx, endidx}
Method called to complete an input line when no command-specific
\method{complete_*()} method is available.  By default, it returns an
empty list.
\end{methoddesc}

\begin{methoddesc}{precmd}{line}
Hook method executed just before the command line \var{line} is
interpreted, but after the input prompt is generated and issued.  This
method is a stub in \class{Cmd}; it exists to be overridden by
subclasses.  The return value is used as the command which will be
executed by the \method{onecmd()} method; the \method{precmd()}
implementation may re-write the command or simply return \var{line}
unchanged.
\end{methoddesc}

\begin{methoddesc}{postcmd}{stop, line}
Hook method executed just after a command dispatch is finished.  This
method is a stub in \class{Cmd}; it exists to be overridden by
subclasses.  \var{line} is the command line which was executed, and
\var{stop} is a flag which indicates whether execution will be
terminated after the call to \method{postcmd()}; this will be the
return value of the \method{onecmd()} method.  The return value of
this method will be used as the new value for the internal flag which
corresponds to \var{stop}; returning false will cause interpretation
to continue.
\end{methoddesc}

\begin{methoddesc}{preloop}{}
Hook method executed once when \method{cmdloop()} is called.  This
method is a stub in \class{Cmd}; it exists to be overridden by
subclasses.
\end{methoddesc}

\begin{methoddesc}{postloop}{}
Hook method executed once when \method{cmdloop()} is about to return.
This method is a stub in \class{Cmd}; it exists to be overridden by
subclasses.
\end{methoddesc}

Instances of \class{Cmd} subclasses have some public instance variables:

\begin{memberdesc}{prompt}
The prompt issued to solicit input.
\end{memberdesc}

\begin{memberdesc}{identchars}
The string of characters accepted for the command prefix.
\end{memberdesc}

\begin{memberdesc}{lastcmd}
The last nonempty command prefix seen. 
\end{memberdesc}

\begin{memberdesc}{intro}
A string to issue as an intro or banner.  May be overridden by giving
the \method{cmdloop()} method an argument.
\end{memberdesc}

\begin{memberdesc}{doc_header}
The header to issue if the help output has a section for documented
commands.
\end{memberdesc}

\begin{memberdesc}{misc_header}
The header to issue if the help output has a section for miscellaneous 
help topics (that is, there are \method{help_*()} methods without
corresponding \method{do_*()} methods).
\end{memberdesc}

\begin{memberdesc}{undoc_header}
The header to issue if the help output has a section for undocumented 
commands (that is, there are \method{do_*()} methods without
corresponding \method{help_*()} methods).
\end{memberdesc}

\begin{memberdesc}{ruler}
The character used to draw separator lines under the help-message
headers.  If empty, no ruler line is drawn.  It defaults to
\character{=}.
\end{memberdesc}

\section{\module{shlex} ---
         Simple lexical analysis}

\declaremodule{standard}{shlex}
\modulesynopsis{Simple lexical analysis for \UNIX{} shell-like languages.}
\moduleauthor{Eric S. Raymond}{esr@snark.thyrsus.com}
\sectionauthor{Eric S. Raymond}{esr@snark.thyrsus.com}

\versionadded{1.5.2}

The \class{shlex} class makes it easy to write lexical analyzers for
simple syntaxes resembling that of the \UNIX{} shell.  This will often
be useful for writing minilanguages, e.g.\ in run control files for
Python applications.

\begin{classdesc}{shlex}{\optional{stream\optional{, file}}}
A \class{shlex} instance or subclass instance is a lexical analyzer
object.  The initialization argument, if present, specifies where to
read characters from. It must be a file- or stream-like object with
\method{read()} and \method{readline()} methods.  If no argument is given,
input will be taken from \code{sys.stdin}.  The second optional 
argument is a filename string, which sets the initial value of the
\member{infile} member.  If the stream argument is omitted or
equal to \code{sys.stdin}, this second argument defaults to ``stdin''.
\end{classdesc}


\begin{seealso}
  \seemodule{ConfigParser}{Parser for configuration files similar to the
                           Windows \file{.ini} files.}
\end{seealso}


\subsection{shlex Objects \label{shlex-objects}}

A \class{shlex} instance has the following methods:


\begin{methoddesc}{get_token}{}
Return a token.  If tokens have been stacked using
\method{push_token()}, pop a token off the stack.  Otherwise, read one
from the input stream.  If reading encounters an immediate
end-of-file, an empty string is returned. 
\end{methoddesc}

\begin{methoddesc}{push_token}{str}
Push the argument onto the token stack.
\end{methoddesc}

\begin{methoddesc}{read_token}{}
Read a raw token.  Ignore the pushback stack, and do not interpret source
requests.  (This is not ordinarily a useful entry point, and is
documented here only for the sake of completeness.)
\end{methoddesc}

\begin{methoddesc}{sourcehook}{filename}
When \class{shlex} detects a source request (see
\member{source} below) this method is given the following token as
argument, and expected to return a tuple consisting of a filename and
an open file-like object.

Normally, this method first strips any quotes off the argument.  If
the result is an absolute pathname, or there was no previous source
request in effect, or the previous source was a stream
(e.g. \code{sys.stdin}), the result is left alone.  Otherwise, if the
result is a relative pathname, the directory part of the name of the
file immediately before it on the source inclusion stack is prepended
(this behavior is like the way the C preprocessor handles
\code{\#include "file.h"}).

The result of the manipulations is treated as a filename, and returned
as the first component of the tuple, with
\function{open()} called on it to yield the second component. (Note:
this is the reverse of the order of arguments in instance initialization!)

This hook is exposed so that you can use it to implement directory
search paths, addition of file extensions, and other namespace hacks.
There is no corresponding `close' hook, but a shlex instance will call
the \method{close()} method of the sourced input stream when it
returns \EOF.

For more explicit control of source stacking, use the
\method{push_source()} and \method{pop_source()} methods. 
\end{methoddesc}

\begin{methoddesc}{push_source}{stream\optional{, filename}}
Push an input source stream onto the input stack.  If the filename
argument is specified it will later be available for use in error
messages.  This is the same method used internally by the
\method{sourcehook} method.
\versionadded{2.1}
\end{methoddesc}

\begin{methoddesc}{pop_source}{}
Pop the last-pushed input source from the input stack.
This is the same method used internally when the lexer reaches
\EOF on a stacked input stream.
\versionadded{2.1}
\end{methoddesc}

\begin{methoddesc}{error_leader}{\optional{file\optional{, line}}}
This method generates an error message leader in the format of a
\UNIX{} C compiler error label; the format is \code{'"\%s", line \%d: '},
where the \samp{\%s} is replaced with the name of the current source
file and the \samp{\%d} with the current input line number (the
optional arguments can be used to override these).

This convenience is provided to encourage \module{shlex} users to
generate error messages in the standard, parseable format understood
by Emacs and other \UNIX{} tools.
\end{methoddesc}

Instances of \class{shlex} subclasses have some public instance
variables which either control lexical analysis or can be used for
debugging:

\begin{memberdesc}{commenters}
The string of characters that are recognized as comment beginners.
All characters from the comment beginner to end of line are ignored.
Includes just \character{\#} by default.   
\end{memberdesc}

\begin{memberdesc}{wordchars}
The string of characters that will accumulate into multi-character
tokens.  By default, includes all \ASCII{} alphanumerics and
underscore.
\end{memberdesc}

\begin{memberdesc}{whitespace}
Characters that will be considered whitespace and skipped.  Whitespace
bounds tokens.  By default, includes space, tab, linefeed and
carriage-return.
\end{memberdesc}

\begin{memberdesc}{quotes}
Characters that will be considered string quotes.  The token
accumulates until the same quote is encountered again (thus, different
quote types protect each other as in the shell.)  By default, includes
\ASCII{} single and double quotes.
\end{memberdesc}

\begin{memberdesc}{infile}
The name of the current input file, as initially set at class
instantiation time or stacked by later source requests.  It may
be useful to examine this when constructing error messages.
\end{memberdesc}

\begin{memberdesc}{instream}
The input stream from which this \class{shlex} instance is reading
characters.
\end{memberdesc}

\begin{memberdesc}{source}
This member is \code{None} by default.  If you assign a string to it,
that string will be recognized as a lexical-level inclusion request
similar to the \samp{source} keyword in various shells.  That is, the
immediately following token will opened as a filename and input taken
from that stream until \EOF, at which point the \method{close()}
method of that stream will be called and the input source will again
become the original input stream. Source requests may be stacked any
number of levels deep.
\end{memberdesc}

\begin{memberdesc}{debug}
If this member is numeric and \code{1} or more, a \class{shlex}
instance will print verbose progress output on its behavior.  If you
need to use this, you can read the module source code to learn the
details.
\end{memberdesc}

Note that any character not declared to be a word character,
whitespace, or a quote will be returned as a single-character token.

Quote and comment characters are not recognized within words.  Thus,
the bare words \samp{ain't} and \samp{ain\#t} would be returned as single
tokens by the default parser.

\begin{memberdesc}{lineno}
Source line number (count of newlines seen so far plus one).
\end{memberdesc}

\begin{memberdesc}{token}
The token buffer.  It may be useful to examine this when catching
exceptions.
\end{memberdesc}


\chapter{Generic Operating System Services}
\label{allos}

The modules described in this chapter provide interfaces to operating
system features that are available on (almost) all operating systems,
such as files and a clock.  The interfaces are generally modelled
after the \UNIX{} or \C{} interfaces but they are available on most
other systems as well.  Here's an overview:

\begin{description}

\item[os]
--- Miscellaneous OS interfaces.

\item[time]
--- Time access and conversions.

\item[getopt]
--- Parser for command line options.

\item[tempfile]
--- Generate temporary file names.

\item[errno]
--- Standard errno system symbols.

\item[glob]
--- \UNIX{} shell style pathname pattern expansion.

\item[fnmatch]
--- \UNIX{} shell style pathname pattern matching.

\item[locale]
--- Internationalization services.

\end{description}
		% Generic Operating System Services
\section{\module{os} ---
         Miscellaneous operating system interfaces}

\declaremodule{standard}{os}
\modulesynopsis{Miscellaneous operating system interfaces.}


This module provides a more portable way of using operating system
dependent functionality than importing a operating system dependent
built-in module like \refmodule{posix} or \module{nt}.

This module searches for an operating system dependent built-in module like
\module{mac} or \refmodule{posix} and exports the same functions and data
as found there.  The design of all Python's built-in operating system dependent
modules is such that as long as the same functionality is available,
it uses the same interface; for example, the function
\code{os.stat(\var{path})} returns stat information about \var{path} in
the same format (which happens to have originated with the
\POSIX{} interface).

Extensions peculiar to a particular operating system are also
available through the \module{os} module, but using them is of course a
threat to portability!

Note that after the first time \module{os} is imported, there is
\emph{no} performance penalty in using functions from \module{os}
instead of directly from the operating system dependent built-in module,
so there should be \emph{no} reason not to use \module{os}!


% Frank Stajano <fstajano@uk.research.att.com> complained that it
% wasn't clear that the entries described in the subsections were all
% available at the module level (most uses of subsections are
% different); I think this is only a problem for the HTML version,
% where the relationship may not be as clear.
%
\ifhtml
The \module{os} module contains many functions and data values.
The items below and in the following sub-sections are all available
directly from the \module{os} module.
\fi


\begin{excdesc}{error}
This exception is raised when a function returns a system-related
error (not for illegal argument types or other incidental errors).
This is also known as the built-in exception \exception{OSError}.  The
accompanying value is a pair containing the numeric error code from
\cdata{errno} and the corresponding string, as would be printed by the
C function \cfunction{perror()}.  See the module
\refmodule{errno}\refbimodindex{errno}, which contains names for the
error codes defined by the underlying operating system.

When exceptions are classes, this exception carries two attributes,
\member{errno} and \member{strerror}.  The first holds the value of
the C \cdata{errno} variable, and the latter holds the corresponding
error message from \cfunction{strerror()}.  For exceptions that
involve a file system path (such as \function{chdir()} or
\function{unlink()}), the exception instance will contain a third
attribute, \member{filename}, which is the file name passed to the
function.
\end{excdesc}

\begin{datadesc}{name}
The name of the operating system dependent module imported.  The
following names have currently been registered: \code{'posix'},
\code{'nt'}, \code{'mac'}, \code{'os2'}, \code{'ce'},
\code{'java'}, \code{'riscos'}.
\end{datadesc}

\begin{datadesc}{path}
The corresponding operating system dependent standard module for pathname
operations, such as \module{posixpath} or \module{macpath}.  Thus,
given the proper imports, \code{os.path.split(\var{file})} is
equivalent to but more portable than
\code{posixpath.split(\var{file})}.  Note that this is also an
importable module: it may be imported directly as
\refmodule{os.path}.
\end{datadesc}



\subsection{Process Parameters \label{os-procinfo}}

These functions and data items provide information and operate on the
current process and user.

\begin{datadesc}{environ}
A mapping object representing the string environment. For example,
\code{environ['HOME']} is the pathname of your home directory (on some
platforms), and is equivalent to \code{getenv("HOME")} in C.

This mapping is captured the first time the \module{os} module is
imported, typically during Python startup as part of processing
\file{site.py}.  Changes to the environment made after this time are
not reflected in \code{os.environ}, except for changes made by modifying
\code{os.environ} directly.

If the platform supports the \function{putenv()} function, this
mapping may be used to modify the environment as well as query the
environment.  \function{putenv()} will be called automatically when
the mapping is modified.
\note{Calling \function{putenv()} directly does not change
\code{os.environ}, so it's better to modify \code{os.environ}.}
\note{On some platforms, including FreeBSD and Mac OS X, setting
\code{environ} may cause memory leaks.  Refer to the system documentation
for \cfunction{putenv()}.}

If \function{putenv()} is not provided, a modified copy of this mapping 
may be passed to the appropriate process-creation functions to cause 
child processes to use a modified environment.

If the platform supports the \function{unsetenv()} function, you can 
delete items in this mapping to unset environment variables.
\function{unsetenv()} will be called automatically when an item is
deleted from \code{os.environ}.

\end{datadesc}

\begin{funcdescni}{chdir}{path}
\funclineni{fchdir}{fd}
\funclineni{getcwd}{}
These functions are described in ``Files and Directories'' (section
\ref{os-file-dir}).
\end{funcdescni}

\begin{funcdesc}{ctermid}{}
Return the filename corresponding to the controlling terminal of the
process.
Availability: \UNIX.
\end{funcdesc}

\begin{funcdesc}{getegid}{}
Return the effective group id of the current process.  This
corresponds to the `set id' bit on the file being executed in the
current process.
Availability: \UNIX.
\end{funcdesc}

\begin{funcdesc}{geteuid}{}
\index{user!effective id}
Return the current process' effective user id.
Availability: \UNIX.
\end{funcdesc}

\begin{funcdesc}{getgid}{}
\index{process!group}
Return the real group id of the current process.
Availability: \UNIX.
\end{funcdesc}

\begin{funcdesc}{getgroups}{}
Return list of supplemental group ids associated with the current
process.
Availability: \UNIX.
\end{funcdesc}

\begin{funcdesc}{getlogin}{}
Return the name of the user logged in on the controlling terminal of
the process.  For most purposes, it is more useful to use the
environment variable \envvar{LOGNAME} to find out who the user is,
or \code{pwd.getpwuid(os.getuid())[0]} to get the login name
of the currently effective user ID.
Availability: \UNIX.
\end{funcdesc}

\begin{funcdesc}{getpgid}{pid}
Return the process group id of the process with process id \var{pid}.
If \var{pid} is 0, the process group id of the current process is
returned. Availability: \UNIX.
\versionadded{2.3}
\end{funcdesc}

\begin{funcdesc}{getpgrp}{}
\index{process!group}
Return the id of the current process group.
Availability: \UNIX.
\end{funcdesc}

\begin{funcdesc}{getpid}{}
\index{process!id}
Return the current process id.
Availability: \UNIX, Windows.
\end{funcdesc}

\begin{funcdesc}{getppid}{}
\index{process!id of parent}
Return the parent's process id.
Availability: \UNIX.
\end{funcdesc}

\begin{funcdesc}{getuid}{}
\index{user!id}
Return the current process' user id.
Availability: \UNIX.
\end{funcdesc}

\begin{funcdesc}{getenv}{varname\optional{, value}}
Return the value of the environment variable \var{varname} if it
exists, or \var{value} if it doesn't.  \var{value} defaults to
\code{None}.
Availability: most flavors of \UNIX, Windows.
\end{funcdesc}

\begin{funcdesc}{putenv}{varname, value}
\index{environment variables!setting}
Set the environment variable named \var{varname} to the string
\var{value}.  Such changes to the environment affect subprocesses
started with \function{os.system()}, \function{popen()} or
\function{fork()} and \function{execv()}.
Availability: most flavors of \UNIX, Windows.

\note{On some platforms, including FreeBSD and Mac OS X,
setting \code{environ} may cause memory leaks.
Refer to the system documentation for putenv.}

When \function{putenv()} is
supported, assignments to items in \code{os.environ} are automatically
translated into corresponding calls to \function{putenv()}; however,
calls to \function{putenv()} don't update \code{os.environ}, so it is
actually preferable to assign to items of \code{os.environ}.
\end{funcdesc}

\begin{funcdesc}{setegid}{egid}
Set the current process's effective group id.
Availability: \UNIX.
\end{funcdesc}

\begin{funcdesc}{seteuid}{euid}
Set the current process's effective user id.
Availability: \UNIX.
\end{funcdesc}

\begin{funcdesc}{setgid}{gid}
Set the current process' group id.
Availability: \UNIX.
\end{funcdesc}

\begin{funcdesc}{setgroups}{groups}
Set the list of supplemental group ids associated with the current
process to \var{groups}. \var{groups} must be a sequence, and each
element must be an integer identifying a group. This operation is
typical available only to the superuser.
Availability: \UNIX.
\versionadded{2.2}
\end{funcdesc}

\begin{funcdesc}{setpgrp}{}
Calls the system call \cfunction{setpgrp()} or \cfunction{setpgrp(0,
0)} depending on which version is implemented (if any).  See the
\UNIX{} manual for the semantics.
Availability: \UNIX.
\end{funcdesc}

\begin{funcdesc}{setpgid}{pid, pgrp} Calls the system call
\cfunction{setpgid()} to set the process group id of the process with
id \var{pid} to the process group with id \var{pgrp}.  See the \UNIX{}
manual for the semantics.
Availability: \UNIX.
\end{funcdesc}

\begin{funcdesc}{setreuid}{ruid, euid}
Set the current process's real and effective user ids.
Availability: \UNIX.
\end{funcdesc}

\begin{funcdesc}{setregid}{rgid, egid}
Set the current process's real and effective group ids.
Availability: \UNIX.
\end{funcdesc}

\begin{funcdesc}{getsid}{pid}
Calls the system call \cfunction{getsid()}.  See the \UNIX{} manual
for the semantics.
Availability: \UNIX. \versionadded{2.4}
\end{funcdesc}

\begin{funcdesc}{setsid}{}
Calls the system call \cfunction{setsid()}.  See the \UNIX{} manual
for the semantics.
Availability: \UNIX.
\end{funcdesc}

\begin{funcdesc}{setuid}{uid}
\index{user!id, setting}
Set the current process' user id.
Availability: \UNIX.
\end{funcdesc}

% placed in this section since it relates to errno.... a little weak
\begin{funcdesc}{strerror}{code}
Return the error message corresponding to the error code in
\var{code}.
Availability: \UNIX, Windows.
\end{funcdesc}

\begin{funcdesc}{umask}{mask}
Set the current numeric umask and returns the previous umask.
Availability: \UNIX, Windows.
\end{funcdesc}

\begin{funcdesc}{uname}{}
Return a 5-tuple containing information identifying the current
operating system.  The tuple contains 5 strings:
\code{(\var{sysname}, \var{nodename}, \var{release}, \var{version},
\var{machine})}.  Some systems truncate the nodename to 8
characters or to the leading component; a better way to get the
hostname is \function{socket.gethostname()}
\withsubitem{(in module socket)}{\ttindex{gethostname()}}
or even
\withsubitem{(in module socket)}{\ttindex{gethostbyaddr()}}
\code{socket.gethostbyaddr(socket.gethostname())}.
Availability: recent flavors of \UNIX.
\end{funcdesc}

\begin{funcdesc}{unsetenv}{varname}
\index{environment variables!deleting}
Unset (delete) the environment variable named \var{varname}. Such
changes to the environment affect subprocesses started with
\function{os.system()}, \function{popen()} or \function{fork()} and
\function{execv()}. Availability: most flavors of \UNIX, Windows.

When \function{unsetenv()} is
supported, deletion of items in \code{os.environ} is automatically
translated into a corresponding call to \function{unsetenv()}; however,
calls to \function{unsetenv()} don't update \code{os.environ}, so it is
actually preferable to delete items of \code{os.environ}.
\end{funcdesc}

\subsection{File Object Creation \label{os-newstreams}}

These functions create new file objects.


\begin{funcdesc}{fdopen}{fd\optional{, mode\optional{, bufsize}}}
Return an open file object connected to the file descriptor \var{fd}.
\index{I/O control!buffering}
The \var{mode} and \var{bufsize} arguments have the same meaning as
the corresponding arguments to the built-in \function{open()}
function.
Availability: Macintosh, \UNIX, Windows.

\versionchanged[When specified, the \var{mode} argument must now start
  with one of the letters \character{r}, \character{w}, or \character{a},
  otherwise a \exception{ValueError} is raised]{2.3}
\end{funcdesc}

\begin{funcdesc}{popen}{command\optional{, mode\optional{, bufsize}}}
Open a pipe to or from \var{command}.  The return value is an open
file object connected to the pipe, which can be read or written
depending on whether \var{mode} is \code{'r'} (default) or \code{'w'}.
The \var{bufsize} argument has the same meaning as the corresponding
argument to the built-in \function{open()} function.  The exit status of
the command (encoded in the format specified for \function{wait()}) is
available as the return value of the \method{close()} method of the file
object, except that when the exit status is zero (termination without
errors), \code{None} is returned.
Availability: Macintosh, \UNIX, Windows.

\versionchanged[This function worked unreliably under Windows in
  earlier versions of Python.  This was due to the use of the
  \cfunction{_popen()} function from the libraries provided with
  Windows.  Newer versions of Python do not use the broken
  implementation from the Windows libraries]{2.0}
\end{funcdesc}

\begin{funcdesc}{tmpfile}{}
Return a new file object opened in update mode (\samp{w+b}).  The file
has no directory entries associated with it and will be automatically
deleted once there are no file descriptors for the file.
Availability: Macintosh, \UNIX, Windows.
\end{funcdesc}


For each of the following \function{popen()} variants, if \var{bufsize} is
specified, it specifies the buffer size for the I/O pipes.
\var{mode}, if provided, should be the string \code{'b'} or
\code{'t'}; on Windows this is needed to determine whether the file
objects should be opened in binary or text mode.  The default value
for \var{mode} is \code{'t'}.

Also, for each of these variants, on \UNIX, \var{cmd} may be a sequence, in
which case arguments will be passed directly to the program without shell
intervention (as with \function{os.spawnv()}). If \var{cmd} is a string it will
be passed to the shell (as with \function{os.system()}).

These methods do not make it possible to retrieve the exit status from
the child processes.  The only way to control the input and output
streams and also retrieve the return codes is to use the
\class{Popen3} and \class{Popen4} classes from the \refmodule{popen2}
module; these are only available on \UNIX.

For a discussion of possible deadlock conditions related to the use
of these functions, see ``\ulink{Flow Control
Issues}{popen2-flow-control.html}''
(section~\ref{popen2-flow-control}).

\begin{funcdesc}{popen2}{cmd\optional{, mode\optional{, bufsize}}}
Executes \var{cmd} as a sub-process.  Returns the file objects
\code{(\var{child_stdin}, \var{child_stdout})}.
Availability: Macintosh, \UNIX, Windows.
\versionadded{2.0}
\end{funcdesc}

\begin{funcdesc}{popen3}{cmd\optional{, mode\optional{, bufsize}}}
Executes \var{cmd} as a sub-process.  Returns the file objects
\code{(\var{child_stdin}, \var{child_stdout}, \var{child_stderr})}.
Availability: Macintosh, \UNIX, Windows.
\versionadded{2.0}
\end{funcdesc}

\begin{funcdesc}{popen4}{cmd\optional{, mode\optional{, bufsize}}}
Executes \var{cmd} as a sub-process.  Returns the file objects
\code{(\var{child_stdin}, \var{child_stdout_and_stderr})}.
Availability: Macintosh, \UNIX, Windows.
\versionadded{2.0}
\end{funcdesc}

(Note that \code{\var{child_stdin}, \var{child_stdout}, and
\var{child_stderr}} are named from the point of view of the child
process, i.e. \var{child_stdin} is the child's standard input.)

This functionality is also available in the \refmodule{popen2} module
using functions of the same names, but the return values of those
functions have a different order.


\subsection{File Descriptor Operations \label{os-fd-ops}}

These functions operate on I/O streams referenced using file
descriptors.  

File descriptors are small integers corresponding to a file that has
been opened by the current process.  For example, standard input is
usually file descriptor 0, standard output is 1, and standard error is
2.  Further files opened by a process will then be assigned 3, 4, 5,
and so forth.  The name ``file descriptor'' is slightly deceptive; on
{\UNIX} platforms, sockets and pipes are also referenced by file descriptors.


\begin{funcdesc}{close}{fd}
Close file descriptor \var{fd}.
Availability: Macintosh, \UNIX, Windows.

\begin{notice}
This function is intended for low-level I/O and must be applied
to a file descriptor as returned by \function{open()} or
\function{pipe()}.  To close a ``file object'' returned by the
built-in function \function{open()} or by \function{popen()} or
\function{fdopen()}, use its \method{close()} method.
\end{notice}
\end{funcdesc}

\begin{funcdesc}{dup}{fd}
Return a duplicate of file descriptor \var{fd}.
Availability: Macintosh, \UNIX, Windows.
\end{funcdesc}

\begin{funcdesc}{dup2}{fd, fd2}
Duplicate file descriptor \var{fd} to \var{fd2}, closing the latter
first if necessary.
Availability: Macintosh, \UNIX, Windows.
\end{funcdesc}

\begin{funcdesc}{fdatasync}{fd}
Force write of file with filedescriptor \var{fd} to disk.
Does not force update of metadata.
Availability: \UNIX.
\end{funcdesc}

\begin{funcdesc}{fpathconf}{fd, name}
Return system configuration information relevant to an open file.
\var{name} specifies the configuration value to retrieve; it may be a
string which is the name of a defined system value; these names are
specified in a number of standards (\POSIX.1, \UNIX{} 95, \UNIX{} 98, and
others).  Some platforms define additional names as well.  The names
known to the host operating system are given in the
\code{pathconf_names} dictionary.  For configuration variables not
included in that mapping, passing an integer for \var{name} is also
accepted.
Availability: Macintosh, \UNIX.

If \var{name} is a string and is not known, \exception{ValueError} is
raised.  If a specific value for \var{name} is not supported by the
host system, even if it is included in \code{pathconf_names}, an
\exception{OSError} is raised with \constant{errno.EINVAL} for the
error number.
\end{funcdesc}

\begin{funcdesc}{fstat}{fd}
Return status for file descriptor \var{fd}, like \function{stat()}.
Availability: Macintosh, \UNIX, Windows.
\end{funcdesc}

\begin{funcdesc}{fstatvfs}{fd}
Return information about the filesystem containing the file associated
with file descriptor \var{fd}, like \function{statvfs()}.
Availability: \UNIX.
\end{funcdesc}

\begin{funcdesc}{fsync}{fd}
Force write of file with filedescriptor \var{fd} to disk.  On \UNIX,
this calls the native \cfunction{fsync()} function; on Windows, the
MS \cfunction{_commit()} function.

If you're starting with a Python file object \var{f}, first do
\code{\var{f}.flush()}, and then do \code{os.fsync(\var{f}.fileno())},
to ensure that all internal buffers associated with \var{f} are written
to disk.
Availability: Macintosh, \UNIX, and Windows starting in 2.2.3.
\end{funcdesc}

\begin{funcdesc}{ftruncate}{fd, length}
Truncate the file corresponding to file descriptor \var{fd},
so that it is at most \var{length} bytes in size.
Availability: Macintosh, \UNIX.
\end{funcdesc}

\begin{funcdesc}{isatty}{fd}
Return \code{True} if the file descriptor \var{fd} is open and
connected to a tty(-like) device, else \code{False}.
Availability: Macintosh, \UNIX.
\end{funcdesc}

\begin{funcdesc}{lseek}{fd, pos, how}
Set the current position of file descriptor \var{fd} to position
\var{pos}, modified by \var{how}: \code{0} to set the position
relative to the beginning of the file; \code{1} to set it relative to
the current position; \code{2} to set it relative to the end of the
file.
Availability: Macintosh, \UNIX, Windows.
\end{funcdesc}

\begin{funcdesc}{open}{file, flags\optional{, mode}}
Open the file \var{file} and set various flags according to
\var{flags} and possibly its mode according to \var{mode}.
The default \var{mode} is \code{0777} (octal), and the current umask
value is first masked out.  Return the file descriptor for the newly
opened file.
Availability: Macintosh, \UNIX, Windows.

For a description of the flag and mode values, see the C run-time
documentation; flag constants (like \constant{O_RDONLY} and
\constant{O_WRONLY}) are defined in this module too (see below).

\begin{notice}
This function is intended for low-level I/O.  For normal usage,
use the built-in function \function{open()}, which returns a ``file
object'' with \method{read()} and \method{write()} methods (and many
more).
\end{notice}
\end{funcdesc}

\begin{funcdesc}{openpty}{}
Open a new pseudo-terminal pair. Return a pair of file descriptors
\code{(\var{master}, \var{slave})} for the pty and the tty,
respectively. For a (slightly) more portable approach, use the
\refmodule{pty}\refstmodindex{pty} module.
Availability: Macintosh, Some flavors of \UNIX.
\end{funcdesc}

\begin{funcdesc}{pipe}{}
Create a pipe.  Return a pair of file descriptors \code{(\var{r},
\var{w})} usable for reading and writing, respectively.
Availability: Macintosh, \UNIX, Windows.
\end{funcdesc}

\begin{funcdesc}{read}{fd, n}
Read at most \var{n} bytes from file descriptor \var{fd}.
Return a string containing the bytes read.  If the end of the file
referred to by \var{fd} has been reached, an empty string is
returned.
Availability: Macintosh, \UNIX, Windows.

\begin{notice}
This function is intended for low-level I/O and must be applied
to a file descriptor as returned by \function{open()} or
\function{pipe()}.  To read a ``file object'' returned by the
built-in function \function{open()} or by \function{popen()} or
\function{fdopen()}, or \code{sys.stdin}, use its
\method{read()} or \method{readline()} methods.
\end{notice}
\end{funcdesc}

\begin{funcdesc}{tcgetpgrp}{fd}
Return the process group associated with the terminal given by
\var{fd} (an open file descriptor as returned by \function{open()}).
Availability: Macintosh, \UNIX.
\end{funcdesc}

\begin{funcdesc}{tcsetpgrp}{fd, pg}
Set the process group associated with the terminal given by
\var{fd} (an open file descriptor as returned by \function{open()})
to \var{pg}.
Availability: Macintosh, \UNIX.
\end{funcdesc}

\begin{funcdesc}{ttyname}{fd}
Return a string which specifies the terminal device associated with
file-descriptor \var{fd}.  If \var{fd} is not associated with a terminal
device, an exception is raised.
Availability:Macintosh,  \UNIX.
\end{funcdesc}

\begin{funcdesc}{write}{fd, str}
Write the string \var{str} to file descriptor \var{fd}.
Return the number of bytes actually written.
Availability: Macintosh, \UNIX, Windows.

\begin{notice}
This function is intended for low-level I/O and must be applied
to a file descriptor as returned by \function{open()} or
\function{pipe()}.  To write a ``file object'' returned by the
built-in function \function{open()} or by \function{popen()} or
\function{fdopen()}, or \code{sys.stdout} or \code{sys.stderr}, use
its \method{write()} method.
\end{notice}
\end{funcdesc}


The following data items are available for use in constructing the
\var{flags} parameter to the \function{open()} function.  Some items will
not be available on all platforms.  For descriptions of their availability
and use, consult \manpage{open}{2}.

\begin{datadesc}{O_RDONLY}
\dataline{O_WRONLY}
\dataline{O_RDWR}
\dataline{O_APPEND}
\dataline{O_CREAT}
\dataline{O_EXCL}
\dataline{O_TRUNC}
Options for the \var{flag} argument to the \function{open()} function.
These can be bit-wise OR'd together.
Availability: Macintosh, \UNIX, Windows.
\end{datadesc}

\begin{datadesc}{O_DSYNC}
\dataline{O_RSYNC}
\dataline{O_SYNC}
\dataline{O_NDELAY}
\dataline{O_NONBLOCK}
\dataline{O_NOCTTY}
\dataline{O_SHLOCK}
\dataline{O_EXLOCK}
More options for the \var{flag} argument to the \function{open()} function.
Availability: Macintosh, \UNIX.
\end{datadesc}

\begin{datadesc}{O_BINARY}
Option for the \var{flag} argument to the \function{open()} function.
This can be bit-wise OR'd together with those listed above.
Availability: Windows.
% XXX need to check on the availability of this one.
\end{datadesc}

\begin{datadesc}{O_NOINHERIT}
\dataline{O_SHORT_LIVED}
\dataline{O_TEMPORARY}
\dataline{O_RANDOM}
\dataline{O_SEQUENTIAL}
\dataline{O_TEXT}
Options for the \var{flag} argument to the \function{open()} function.
These can be bit-wise OR'd together.
Availability: Windows.
\end{datadesc}

\begin{datadesc}{SEEK_SET}
\dataline{SEEK_CUR}
\dataline{SEEK_END}
Parameters to the \function{lseek()} function.
Their values are 0, 1, and 2, respectively.
Availability: Windows, Macintosh, \UNIX.
\versionadded{2.5}
\end{datadesc}

\subsection{Files and Directories \label{os-file-dir}}

\begin{funcdesc}{access}{path, mode}
Use the real uid/gid to test for access to \var{path}.  Note that most
operations will use the effective uid/gid, therefore this routine can
be used in a suid/sgid environment to test if the invoking user has the
specified access to \var{path}.  \var{mode} should be \constant{F_OK}
to test the existence of \var{path}, or it can be the inclusive OR of
one or more of \constant{R_OK}, \constant{W_OK}, and \constant{X_OK} to
test permissions.  Return \constant{True} if access is allowed,
\constant{False} if not.
See the \UNIX{} man page \manpage{access}{2} for more information.
Availability: Macintosh, \UNIX, Windows.

\note{Using \function{access()} to check if a user is authorized to e.g.
open a file before actually doing so using \function{open()} creates a 
security hole, because the user might exploit the short time interval 
between checking and opening the file to manipulate it.}

\note{I/O operations may fail even when \function{access()}
indicates that they would succeed, particularly for operations
on network filesystems which may have permissions semantics
beyond the usual \POSIX{} permission-bit model.}
\end{funcdesc}

\begin{datadesc}{F_OK}
  Value to pass as the \var{mode} parameter of \function{access()} to
  test the existence of \var{path}.
\end{datadesc}

\begin{datadesc}{R_OK}
  Value to include in the \var{mode} parameter of \function{access()}
  to test the readability of \var{path}.
\end{datadesc}

\begin{datadesc}{W_OK}
  Value to include in the \var{mode} parameter of \function{access()}
  to test the writability of \var{path}.
\end{datadesc}

\begin{datadesc}{X_OK}
  Value to include in the \var{mode} parameter of \function{access()}
  to determine if \var{path} can be executed.
\end{datadesc}

\begin{funcdesc}{chdir}{path}
\index{directory!changing}
Change the current working directory to \var{path}.
Availability: Macintosh, \UNIX, Windows.
\end{funcdesc}

\begin{funcdesc}{fchdir}{fd}
Change the current working directory to the directory represented by
the file descriptor \var{fd}.  The descriptor must refer to an opened
directory, not an open file.
Availability: \UNIX.
\versionadded{2.3}
\end{funcdesc}

\begin{funcdesc}{getcwd}{}
Return a string representing the current working directory.
Availability: Macintosh, \UNIX, Windows.
\end{funcdesc}

\begin{funcdesc}{getcwdu}{}
Return a Unicode object representing the current working directory.
Availability: Macintosh, \UNIX, Windows.
\versionadded{2.3}
\end{funcdesc}

\begin{funcdesc}{chroot}{path}
Change the root directory of the current process to \var{path}.
Availability: Macintosh, \UNIX.
\versionadded{2.2}
\end{funcdesc}

\begin{funcdesc}{chmod}{path, mode}
Change the mode of \var{path} to the numeric \var{mode}.
\var{mode} may take one of the following values
(as defined in the \module{stat} module) or bitwise or-ed
combinations of them:
\begin{itemize}
  \item \code{S_ISUID}
  \item \code{S_ISGID}
  \item \code{S_ENFMT}
  \item \code{S_ISVTX}
  \item \code{S_IREAD}
  \item \code{S_IWRITE}
  \item \code{S_IEXEC}
  \item \code{S_IRWXU}
  \item \code{S_IRUSR}
  \item \code{S_IWUSR}
  \item \code{S_IXUSR}
  \item \code{S_IRWXG}
  \item \code{S_IRGRP}
  \item \code{S_IWGRP}
  \item \code{S_IXGRP}
  \item \code{S_IRWXO}
  \item \code{S_IROTH}
  \item \code{S_IWOTH}
  \item \code{S_IXOTH}
\end{itemize}
Availability: Macintosh, \UNIX, Windows.

\note{Although Windows supports \function{chmod()}, you can only 
set the file's read-only flag with it (via the \code{S_IWRITE} 
and \code{S_IREAD} constants or a corresponding integer value). 
All other bits are ignored.}
\end{funcdesc}

\begin{funcdesc}{chown}{path, uid, gid}
Change the owner and group id of \var{path} to the numeric \var{uid}
and \var{gid}. To leave one of the ids unchanged, set it to -1.
Availability: Macintosh, \UNIX.
\end{funcdesc}

\begin{funcdesc}{lchown}{path, uid, gid}
Change the owner and group id of \var{path} to the numeric \var{uid}
and gid. This function will not follow symbolic links.
Availability: Macintosh, \UNIX.
\versionadded{2.3}
\end{funcdesc}

\begin{funcdesc}{link}{src, dst}
Create a hard link pointing to \var{src} named \var{dst}.
Availability: Macintosh, \UNIX.
\end{funcdesc}

\begin{funcdesc}{listdir}{path}
Return a list containing the names of the entries in the directory.
The list is in arbitrary order.  It does not include the special
entries \code{'.'} and \code{'..'} even if they are present in the
directory.
Availability: Macintosh, \UNIX, Windows.

\versionchanged[On Windows NT/2k/XP and \UNIX, if \var{path} is a Unicode
object, the result will be a list of Unicode objects.]{2.3}
\end{funcdesc}

\begin{funcdesc}{lstat}{path}
Like \function{stat()}, but do not follow symbolic links.
Availability: Macintosh, \UNIX.
\end{funcdesc}

\begin{funcdesc}{mkfifo}{path\optional{, mode}}
Create a FIFO (a named pipe) named \var{path} with numeric mode
\var{mode}.  The default \var{mode} is \code{0666} (octal).  The current
umask value is first masked out from the mode.
Availability: Macintosh, \UNIX.

FIFOs are pipes that can be accessed like regular files.  FIFOs exist
until they are deleted (for example with \function{os.unlink()}).
Generally, FIFOs are used as rendezvous between ``client'' and
``server'' type processes: the server opens the FIFO for reading, and
the client opens it for writing.  Note that \function{mkfifo()}
doesn't open the FIFO --- it just creates the rendezvous point.
\end{funcdesc}

\begin{funcdesc}{mknod}{filename\optional{, mode=0600, device}}
Create a filesystem node (file, device special file or named pipe)
named \var{filename}. \var{mode} specifies both the permissions to use and
the type of node to be created, being combined (bitwise OR) with one
of S_IFREG, S_IFCHR, S_IFBLK, and S_IFIFO (those constants are
available in \module{stat}). For S_IFCHR and S_IFBLK, \var{device}
defines the newly created device special file (probably using
\function{os.makedev()}), otherwise it is ignored.
\versionadded{2.3}
\end{funcdesc}

\begin{funcdesc}{major}{device}
Extracts the device major number from a raw device number (usually
the \member{st_dev} or \member{st_rdev} field from \ctype{stat}).
\versionadded{2.3}
\end{funcdesc}

\begin{funcdesc}{minor}{device}
Extracts the device minor number from a raw device number (usually
the \member{st_dev} or \member{st_rdev} field from \ctype{stat}).
\versionadded{2.3}
\end{funcdesc}

\begin{funcdesc}{makedev}{major, minor}
Composes a raw device number from the major and minor device numbers.
\versionadded{2.3}
\end{funcdesc}

\begin{funcdesc}{mkdir}{path\optional{, mode}}
Create a directory named \var{path} with numeric mode \var{mode}.
The default \var{mode} is \code{0777} (octal).  On some systems,
\var{mode} is ignored.  Where it is used, the current umask value is
first masked out.
Availability: Macintosh, \UNIX, Windows.
\end{funcdesc}

\begin{funcdesc}{makedirs}{path\optional{, mode}}
Recursive directory creation function.\index{directory!creating}
\index{UNC paths!and \function{os.makedirs()}}
Like \function{mkdir()},
but makes all intermediate-level directories needed to contain the
leaf directory.  Throws an \exception{error} exception if the leaf
directory already exists or cannot be created.  The default \var{mode}
is \code{0777} (octal).  On some systems, \var{mode} is ignored.
Where it is used, the current umask value is first masked out.
\note{\function{makedirs()} will become confused if the path elements
to create include \var{os.pardir}.}
\versionadded{1.5.2}
\versionchanged[This function now handles UNC paths correctly]{2.3}
\end{funcdesc}

\begin{funcdesc}{pathconf}{path, name}
Return system configuration information relevant to a named file.
\var{name} specifies the configuration value to retrieve; it may be a
string which is the name of a defined system value; these names are
specified in a number of standards (\POSIX.1, \UNIX{} 95, \UNIX{} 98, and
others).  Some platforms define additional names as well.  The names
known to the host operating system are given in the
\code{pathconf_names} dictionary.  For configuration variables not
included in that mapping, passing an integer for \var{name} is also
accepted.
Availability: Macintosh, \UNIX.

If \var{name} is a string and is not known, \exception{ValueError} is
raised.  If a specific value for \var{name} is not supported by the
host system, even if it is included in \code{pathconf_names}, an
\exception{OSError} is raised with \constant{errno.EINVAL} for the
error number.
\end{funcdesc}

\begin{datadesc}{pathconf_names}
Dictionary mapping names accepted by \function{pathconf()} and
\function{fpathconf()} to the integer values defined for those names
by the host operating system.  This can be used to determine the set
of names known to the system.
Availability: Macintosh, \UNIX.
\end{datadesc}

\begin{funcdesc}{readlink}{path}
Return a string representing the path to which the symbolic link
points.  The result may be either an absolute or relative pathname; if
it is relative, it may be converted to an absolute pathname using
\code{os.path.join(os.path.dirname(\var{path}), \var{result})}.
Availability: Macintosh, \UNIX.
\end{funcdesc}

\begin{funcdesc}{remove}{path}
Remove the file \var{path}.  If \var{path} is a directory,
\exception{OSError} is raised; see \function{rmdir()} below to remove
a directory.  This is identical to the \function{unlink()} function
documented below.  On Windows, attempting to remove a file that is in
use causes an exception to be raised; on \UNIX, the directory entry is
removed but the storage allocated to the file is not made available
until the original file is no longer in use.
Availability: Macintosh, \UNIX, Windows.
\end{funcdesc}

\begin{funcdesc}{removedirs}{path}
\index{directory!deleting}
Removes directories recursively.  Works like
\function{rmdir()} except that, if the leaf directory is
successfully removed, \function{removedirs()} 
tries to successively remove every parent directory mentioned in 
\var{path} until an error is raised (which is ignored, because
it generally means that a parent directory is not empty).
For example, \samp{os.removedirs('foo/bar/baz')} will first remove
the directory \samp{'foo/bar/baz'}, and then remove \samp{'foo/bar'}
and \samp{'foo'} if they are empty.
Raises \exception{OSError} if the leaf directory could not be 
successfully removed.
\versionadded{1.5.2}
\end{funcdesc}

\begin{funcdesc}{rename}{src, dst}
Rename the file or directory \var{src} to \var{dst}.  If \var{dst} is
a directory, \exception{OSError} will be raised.  On \UNIX, if
\var{dst} exists and is a file, it will be removed silently if the
user has permission.  The operation may fail on some \UNIX{} flavors
if \var{src} and \var{dst} are on different filesystems.  If
successful, the renaming will be an atomic operation (this is a
\POSIX{} requirement).  On Windows, if \var{dst} already exists,
\exception{OSError} will be raised even if it is a file; there may be
no way to implement an atomic rename when \var{dst} names an existing
file.
Availability: Macintosh, \UNIX, Windows.
\end{funcdesc}

\begin{funcdesc}{renames}{old, new}
Recursive directory or file renaming function.
Works like \function{rename()}, except creation of any intermediate
directories needed to make the new pathname good is attempted first.
After the rename, directories corresponding to rightmost path segments
of the old name will be pruned away using \function{removedirs()}.
\versionadded{1.5.2}

\begin{notice}
This function can fail with the new directory structure made if
you lack permissions needed to remove the leaf directory or file.
\end{notice}
\end{funcdesc}

\begin{funcdesc}{rmdir}{path}
Remove the directory \var{path}.
Availability: Macintosh, \UNIX, Windows.
\end{funcdesc}

\begin{funcdesc}{stat}{path}
Perform a \cfunction{stat()} system call on the given path.  The
return value is an object whose attributes correspond to the members of
the \ctype{stat} structure, namely:
\member{st_mode} (protection bits),
\member{st_ino} (inode number),
\member{st_dev} (device),
\member{st_nlink} (number of hard links),
\member{st_uid} (user ID of owner),
\member{st_gid} (group ID of owner),
\member{st_size} (size of file, in bytes),
\member{st_atime} (time of most recent access),
\member{st_mtime} (time of most recent content modification),
\member{st_ctime}
(platform dependent; time of most recent metadata change on \UNIX, or
the time of creation on Windows):

\begin{verbatim}
>>> import os
>>> statinfo = os.stat('somefile.txt')
>>> statinfo
(33188, 422511L, 769L, 1, 1032, 100, 926L, 1105022698,1105022732, 1105022732)
>>> statinfo.st_size
926L
>>>
\end{verbatim}

\versionchanged [If \function{stat_float_times} returns true, the time
values are floats, measuring seconds. Fractions of a second may be
reported if the system supports that. On Mac OS, the times are always
floats. See \function{stat_float_times} for further discussion. ]{2.3}

On some \UNIX{} systems (such as Linux), the following attributes may
also be available:
\member{st_blocks} (number of blocks allocated for file),
\member{st_blksize} (filesystem blocksize),
\member{st_rdev} (type of device if an inode device).
\member{st_flags} (user defined flags for file).

On other \UNIX{} systems (such as FreeBSD), the following attributes
may be available (but may be only filled out of root tries to
use them:
\member{st_gen} (file generation number),
\member{st_birthtime} (time of file creation).

On Mac OS systems, the following attributes may also be available:
\member{st_rsize},
\member{st_creator},
\member{st_type}.

On RISCOS systems, the following attributes are also available:
\member{st_ftype} (file type),
\member{st_attrs} (attributes),
\member{st_obtype} (object type).

For backward compatibility, the return value of \function{stat()} is
also accessible as a tuple of at least 10 integers giving the most
important (and portable) members of the \ctype{stat} structure, in the
order
\member{st_mode},
\member{st_ino},
\member{st_dev},
\member{st_nlink},
\member{st_uid},
\member{st_gid},
\member{st_size},
\member{st_atime},
\member{st_mtime},
\member{st_ctime}.
More items may be added at the end by some implementations.
The standard module \refmodule{stat}\refstmodindex{stat} defines
functions and constants that are useful for extracting information
from a \ctype{stat} structure.
(On Windows, some items are filled with dummy values.)

\note{The exact meaning and resolution of the \member{st_atime},
 \member{st_mtime}, and \member{st_ctime} members depends on the
 operating system and the file system.  For example, on Windows systems
 using the FAT or FAT32 file systems, \member{st_mtime} has 2-second
 resolution, and \member{st_atime} has only 1-day resolution.  See
 your operating system documentation for details.}

Availability: Macintosh, \UNIX, Windows.

\versionchanged
[Added access to values as attributes of the returned object]{2.2}
\versionchanged[Added st_gen, st_birthtime]{2.5}
\end{funcdesc}

\begin{funcdesc}{stat_float_times}{\optional{newvalue}}
Determine whether \class{stat_result} represents time stamps as float
objects.  If newval is True, future calls to stat() return floats, if
it is False, future calls return ints.  If newval is omitted, return
the current setting.

For compatibility with older Python versions, accessing
\class{stat_result} as a tuple always returns integers.

\versionchanged[Python now returns float values by default. Applications
which do not work correctly with floating point time stamps can use
this function to restore the old behaviour]{2.5}

The resolution of the timestamps (i.e. the smallest possible fraction)
depends on the system. Some systems only support second resolution;
on these systems, the fraction will always be zero.

It is recommended that this setting is only changed at program startup
time in the \var{__main__} module; libraries should never change this
setting. If an application uses a library that works incorrectly if
floating point time stamps are processed, this application should turn
the feature off until the library has been corrected.

\end{funcdesc}

\begin{funcdesc}{statvfs}{path}
Perform a \cfunction{statvfs()} system call on the given path.  The
return value is an object whose attributes describe the filesystem on
the given path, and correspond to the members of the
\ctype{statvfs} structure, namely:
\member{f_frsize},
\member{f_blocks},
\member{f_bfree},
\member{f_bavail},
\member{f_files},
\member{f_ffree},
\member{f_favail},
\member{f_flag},
\member{f_namemax}.
Availability: \UNIX.

For backward compatibility, the return value is also accessible as a
tuple whose values correspond to the attributes, in the order given above.
The standard module \refmodule{statvfs}\refstmodindex{statvfs}
defines constants that are useful for extracting information
from a \ctype{statvfs} structure when accessing it as a sequence; this
remains useful when writing code that needs to work with versions of
Python that don't support accessing the fields as attributes.

\versionchanged
[Added access to values as attributes of the returned object]{2.2}
\end{funcdesc}

\begin{funcdesc}{symlink}{src, dst}
Create a symbolic link pointing to \var{src} named \var{dst}.
Availability: \UNIX.
\end{funcdesc}

\begin{funcdesc}{tempnam}{\optional{dir\optional{, prefix}}}
Return a unique path name that is reasonable for creating a temporary
file.  This will be an absolute path that names a potential directory
entry in the directory \var{dir} or a common location for temporary
files if \var{dir} is omitted or \code{None}.  If given and not
\code{None}, \var{prefix} is used to provide a short prefix to the
filename.  Applications are responsible for properly creating and
managing files created using paths returned by \function{tempnam()};
no automatic cleanup is provided.
On \UNIX, the environment variable \envvar{TMPDIR} overrides
\var{dir}, while on Windows the \envvar{TMP} is used.  The specific
behavior of this function depends on the C library implementation;
some aspects are underspecified in system documentation.
\warning{Use of \function{tempnam()} is vulnerable to symlink attacks;
consider using \function{tmpfile()} (section \ref{os-newstreams})
instead.}  Availability: Macintosh, \UNIX, Windows.
\end{funcdesc}

\begin{funcdesc}{tmpnam}{}
Return a unique path name that is reasonable for creating a temporary
file.  This will be an absolute path that names a potential directory
entry in a common location for temporary files.  Applications are
responsible for properly creating and managing files created using
paths returned by \function{tmpnam()}; no automatic cleanup is
provided.
\warning{Use of \function{tmpnam()} is vulnerable to symlink attacks;
consider using \function{tmpfile()} (section \ref{os-newstreams})
instead.}  Availability: \UNIX, Windows.  This function probably
shouldn't be used on Windows, though: Microsoft's implementation of
\function{tmpnam()} always creates a name in the root directory of the
current drive, and that's generally a poor location for a temp file
(depending on privileges, you may not even be able to open a file
using this name).
\end{funcdesc}

\begin{datadesc}{TMP_MAX}
The maximum number of unique names that \function{tmpnam()} will
generate before reusing names.
\end{datadesc}

\begin{funcdesc}{unlink}{path}
Remove the file \var{path}.  This is the same function as
\function{remove()}; the \function{unlink()} name is its traditional
\UNIX{} name.
Availability: Macintosh, \UNIX, Windows.
\end{funcdesc}

\begin{funcdesc}{utime}{path, times}
Set the access and modified times of the file specified by \var{path}.
If \var{times} is \code{None}, then the file's access and modified
times are set to the current time.  Otherwise, \var{times} must be a
2-tuple of numbers, of the form \code{(\var{atime}, \var{mtime})}
which is used to set the access and modified times, respectively.
Whether a directory can be given for \var{path} depends on whether the
operating system implements directories as files (for example, Windows
does not).  Note that the exact times you set here may not be returned
by a subsequent \function{stat()} call, depending on the resolution
with which your operating system records access and modification times;
see \function{stat()}.
\versionchanged[Added support for \code{None} for \var{times}]{2.0}
Availability: Macintosh, \UNIX, Windows.
\end{funcdesc}

\begin{funcdesc}{walk}{top\optional{, topdown\code{=True}
                       \optional{, onerror\code{=None}}}}
\index{directory!walking}
\index{directory!traversal}
\function{walk()} generates the file names in a directory tree, by
walking the tree either top down or bottom up.
For each directory in the tree rooted at directory \var{top} (including
\var{top} itself), it yields a 3-tuple
\code{(\var{dirpath}, \var{dirnames}, \var{filenames})}.

\var{dirpath} is a string, the path to the directory.  \var{dirnames} is
a list of the names of the subdirectories in \var{dirpath}
(excluding \code{'.'} and \code{'..'}).  \var{filenames} is a list of
the names of the non-directory files in \var{dirpath}.  Note that the
names in the lists contain no path components.  To get a full
path (which begins with \var{top}) to a file or directory in
\var{dirpath}, do \code{os.path.join(\var{dirpath}, \var{name})}.

If optional argument \var{topdown} is true or not specified, the triple
for a directory is generated before the triples for any of its
subdirectories (directories are generated top down).  If \var{topdown} is
false, the triple for a directory is generated after the triples for all
of its subdirectories (directories are generated bottom up).

When \var{topdown} is true, the caller can modify the \var{dirnames} list
in-place (perhaps using \keyword{del} or slice assignment), and
\function{walk()} will only recurse into the subdirectories whose names
remain in \var{dirnames}; this can be used to prune the search,
impose a specific order of visiting, or even to inform \function{walk()}
about directories the caller creates or renames before it resumes
\function{walk()} again.  Modifying \var{dirnames} when \var{topdown} is
false is ineffective, because in bottom-up mode the directories in
\var{dirnames} are generated before \var{dirnames} itself is generated.

By default errors from the \code{os.listdir()} call are ignored.  If
optional argument \var{onerror} is specified, it should be a function;
it will be called with one argument, an os.error instance.  It can
report the error to continue with the walk, or raise the exception
to abort the walk.  Note that the filename is available as the
\code{filename} attribute of the exception object.

\begin{notice}
If you pass a relative pathname, don't change the current working
directory between resumptions of \function{walk()}.  \function{walk()}
never changes the current directory, and assumes that its caller
doesn't either.
\end{notice}

\begin{notice}
On systems that support symbolic links, links to subdirectories appear
in \var{dirnames} lists, but \function{walk()} will not visit them
(infinite loops are hard to avoid when following symbolic links).
To visit linked directories, you can identify them with
\code{os.path.islink(\var{path})}, and invoke \code{walk(\var{path})}
on each directly.
\end{notice}

This example displays the number of bytes taken by non-directory files
in each directory under the starting directory, except that it doesn't
look under any CVS subdirectory:

\begin{verbatim}
import os
from os.path import join, getsize
for root, dirs, files in os.walk('python/Lib/email'):
    print root, "consumes",
    print sum(getsize(join(root, name)) for name in files),
    print "bytes in", len(files), "non-directory files"
    if 'CVS' in dirs:
        dirs.remove('CVS')  # don't visit CVS directories
\end{verbatim}

In the next example, walking the tree bottom up is essential:
\function{rmdir()} doesn't allow deleting a directory before the
directory is empty:

\begin{verbatim}
# Delete everything reachable from the directory named in 'top',
# assuming there are no symbolic links.
# CAUTION:  This is dangerous!  For example, if top == '/', it
# could delete all your disk files.
import os
for root, dirs, files in os.walk(top, topdown=False):
    for name in files:
        os.remove(os.path.join(root, name))
    for name in dirs:
        os.rmdir(os.path.join(root, name))
\end{verbatim}

\versionadded{2.3}
\end{funcdesc}

\subsection{Process Management \label{os-process}}

These functions may be used to create and manage processes.

The various \function{exec*()} functions take a list of arguments for
the new program loaded into the process.  In each case, the first of
these arguments is passed to the new program as its own name rather
than as an argument a user may have typed on a command line.  For the
C programmer, this is the \code{argv[0]} passed to a program's
\cfunction{main()}.  For example, \samp{os.execv('/bin/echo', ['foo',
'bar'])} will only print \samp{bar} on standard output; \samp{foo}
will seem to be ignored.


\begin{funcdesc}{abort}{}
Generate a \constant{SIGABRT} signal to the current process.  On
\UNIX, the default behavior is to produce a core dump; on Windows, the
process immediately returns an exit code of \code{3}.  Be aware that
programs which use \function{signal.signal()} to register a handler
for \constant{SIGABRT} will behave differently.
Availability: Macintosh, \UNIX, Windows.
\end{funcdesc}

\begin{funcdesc}{execl}{path, arg0, arg1, \moreargs}
\funcline{execle}{path, arg0, arg1, \moreargs, env}
\funcline{execlp}{file, arg0, arg1, \moreargs}
\funcline{execlpe}{file, arg0, arg1, \moreargs, env}
\funcline{execv}{path, args}
\funcline{execve}{path, args, env}
\funcline{execvp}{file, args}
\funcline{execvpe}{file, args, env}
These functions all execute a new program, replacing the current
process; they do not return.  On \UNIX, the new executable is loaded
into the current process, and will have the same process ID as the
caller.  Errors will be reported as \exception{OSError} exceptions.

The \character{l} and \character{v} variants of the
\function{exec*()} functions differ in how command-line arguments are
passed.  The \character{l} variants are perhaps the easiest to work
with if the number of parameters is fixed when the code is written;
the individual parameters simply become additional parameters to the
\function{execl*()} functions.  The \character{v} variants are good
when the number of parameters is variable, with the arguments being
passed in a list or tuple as the \var{args} parameter.  In either
case, the arguments to the child process should start with the name of
the command being run, but this is not enforced.

The variants which include a \character{p} near the end
(\function{execlp()}, \function{execlpe()}, \function{execvp()},
and \function{execvpe()}) will use the \envvar{PATH} environment
variable to locate the program \var{file}.  When the environment is
being replaced (using one of the \function{exec*e()} variants,
discussed in the next paragraph), the
new environment is used as the source of the \envvar{PATH} variable.
The other variants, \function{execl()}, \function{execle()},
\function{execv()}, and \function{execve()}, will not use the
\envvar{PATH} variable to locate the executable; \var{path} must
contain an appropriate absolute or relative path.

For \function{execle()}, \function{execlpe()}, \function{execve()},
and \function{execvpe()} (note that these all end in \character{e}),
the \var{env} parameter must be a mapping which is used to define the
environment variables for the new process; the \function{execl()},
\function{execlp()}, \function{execv()}, and \function{execvp()}
all cause the new process to inherit the environment of the current
process.
Availability: Macintosh, \UNIX, Windows.
\end{funcdesc}

\begin{funcdesc}{_exit}{n}
Exit to the system with status \var{n}, without calling cleanup
handlers, flushing stdio buffers, etc.
Availability: Macintosh, \UNIX, Windows.

\begin{notice}
The standard way to exit is \code{sys.exit(\var{n})}.
\function{_exit()} should normally only be used in the child process
after a \function{fork()}.
\end{notice}
\end{funcdesc}

The following exit codes are a defined, and can be used with
\function{_exit()}, although they are not required.  These are
typically used for system programs written in Python, such as a
mail server's external command delivery program.
\note{Some of these may not be available on all \UNIX{} platforms,
since there is some variation.  These constants are defined where they
are defined by the underlying platform.}

\begin{datadesc}{EX_OK}
Exit code that means no error occurred.
Availability: Macintosh, \UNIX.
\versionadded{2.3}
\end{datadesc}

\begin{datadesc}{EX_USAGE}
Exit code that means the command was used incorrectly, such as when
the wrong number of arguments are given.
Availability: Macintosh, \UNIX.
\versionadded{2.3}
\end{datadesc}

\begin{datadesc}{EX_DATAERR}
Exit code that means the input data was incorrect.
Availability: Macintosh, \UNIX.
\versionadded{2.3}
\end{datadesc}

\begin{datadesc}{EX_NOINPUT}
Exit code that means an input file did not exist or was not readable.
Availability: Macintosh, \UNIX.
\versionadded{2.3}
\end{datadesc}

\begin{datadesc}{EX_NOUSER}
Exit code that means a specified user did not exist.
Availability: Macintosh, \UNIX.
\versionadded{2.3}
\end{datadesc}

\begin{datadesc}{EX_NOHOST}
Exit code that means a specified host did not exist.
Availability: Macintosh, \UNIX.
\versionadded{2.3}
\end{datadesc}

\begin{datadesc}{EX_UNAVAILABLE}
Exit code that means that a required service is unavailable.
Availability: Macintosh, \UNIX.
\versionadded{2.3}
\end{datadesc}

\begin{datadesc}{EX_SOFTWARE}
Exit code that means an internal software error was detected.
Availability: Macintosh, \UNIX.
\versionadded{2.3}
\end{datadesc}

\begin{datadesc}{EX_OSERR}
Exit code that means an operating system error was detected, such as
the inability to fork or create a pipe.
Availability: Macintosh, \UNIX.
\versionadded{2.3}
\end{datadesc}

\begin{datadesc}{EX_OSFILE}
Exit code that means some system file did not exist, could not be
opened, or had some other kind of error.
Availability: Macintosh, \UNIX.
\versionadded{2.3}
\end{datadesc}

\begin{datadesc}{EX_CANTCREAT}
Exit code that means a user specified output file could not be created.
Availability: Macintosh, \UNIX.
\versionadded{2.3}
\end{datadesc}

\begin{datadesc}{EX_IOERR}
Exit code that means that an error occurred while doing I/O on some file.
Availability: Macintosh, \UNIX.
\versionadded{2.3}
\end{datadesc}

\begin{datadesc}{EX_TEMPFAIL}
Exit code that means a temporary failure occurred.  This indicates
something that may not really be an error, such as a network
connection that couldn't be made during a retryable operation.
Availability: Macintosh, \UNIX.
\versionadded{2.3}
\end{datadesc}

\begin{datadesc}{EX_PROTOCOL}
Exit code that means that a protocol exchange was illegal, invalid, or
not understood.
Availability: Macintosh, \UNIX.
\versionadded{2.3}
\end{datadesc}

\begin{datadesc}{EX_NOPERM}
Exit code that means that there were insufficient permissions to
perform the operation (but not intended for file system problems).
Availability: Macintosh, \UNIX.
\versionadded{2.3}
\end{datadesc}

\begin{datadesc}{EX_CONFIG}
Exit code that means that some kind of configuration error occurred.
Availability: Macintosh, \UNIX.
\versionadded{2.3}
\end{datadesc}

\begin{datadesc}{EX_NOTFOUND}
Exit code that means something like ``an entry was not found''.
Availability: Macintosh, \UNIX.
\versionadded{2.3}
\end{datadesc}

\begin{funcdesc}{fork}{}
Fork a child process.  Return \code{0} in the child, the child's
process id in the parent.
Availability: Macintosh, \UNIX.
\end{funcdesc}

\begin{funcdesc}{forkpty}{}
Fork a child process, using a new pseudo-terminal as the child's
controlling terminal. Return a pair of \code{(\var{pid}, \var{fd})},
where \var{pid} is \code{0} in the child, the new child's process id
in the parent, and \var{fd} is the file descriptor of the master end
of the pseudo-terminal.  For a more portable approach, use the
\refmodule{pty} module.
Availability: Macintosh, Some flavors of \UNIX.
\end{funcdesc}

\begin{funcdesc}{kill}{pid, sig}
\index{process!killing}
\index{process!signalling}
Send signal \var{sig} to the process \var{pid}.  Constants for the
specific signals available on the host platform are defined in the
\refmodule{signal} module.
Availability: Macintosh, \UNIX.
\end{funcdesc}

\begin{funcdesc}{killpg}{pgid, sig}
\index{process!killing}
\index{process!signalling}
Send the signal \var{sig} to the process group \var{pgid}.
Availability: Macintosh, \UNIX.
\versionadded{2.3}
\end{funcdesc}

\begin{funcdesc}{nice}{increment}
Add \var{increment} to the process's ``niceness''.  Return the new
niceness.
Availability: Macintosh, \UNIX.
\end{funcdesc}

\begin{funcdesc}{plock}{op}
Lock program segments into memory.  The value of \var{op}
(defined in \code{<sys/lock.h>}) determines which segments are locked.
Availability: Macintosh, \UNIX.
\end{funcdesc}

\begin{funcdescni}{popen}{\unspecified}
\funclineni{popen2}{\unspecified}
\funclineni{popen3}{\unspecified}
\funclineni{popen4}{\unspecified}
Run child processes, returning opened pipes for communications.  These
functions are described in section \ref{os-newstreams}.
\end{funcdescni}

\begin{funcdesc}{spawnl}{mode, path, \moreargs}
\funcline{spawnle}{mode, path, \moreargs, env}
\funcline{spawnlp}{mode, file, \moreargs}
\funcline{spawnlpe}{mode, file, \moreargs, env}
\funcline{spawnv}{mode, path, args}
\funcline{spawnve}{mode, path, args, env}
\funcline{spawnvp}{mode, file, args}
\funcline{spawnvpe}{mode, file, args, env}
Execute the program \var{path} in a new process.  If \var{mode} is
\constant{P_NOWAIT}, this function returns the process ID of the new
process; if \var{mode} is \constant{P_WAIT}, returns the process's
exit code if it exits normally, or \code{-\var{signal}}, where
\var{signal} is the signal that killed the process.  On Windows, the
process ID will actually be the process handle, so can be used with
the \function{waitpid()} function.

The \character{l} and \character{v} variants of the
\function{spawn*()} functions differ in how command-line arguments are
passed.  The \character{l} variants are perhaps the easiest to work
with if the number of parameters is fixed when the code is written;
the individual parameters simply become additional parameters to the
\function{spawnl*()} functions.  The \character{v} variants are good
when the number of parameters is variable, with the arguments being
passed in a list or tuple as the \var{args} parameter.  In either
case, the arguments to the child process must start with the name of
the command being run.

The variants which include a second \character{p} near the end
(\function{spawnlp()}, \function{spawnlpe()}, \function{spawnvp()},
and \function{spawnvpe()}) will use the \envvar{PATH} environment
variable to locate the program \var{file}.  When the environment is
being replaced (using one of the \function{spawn*e()} variants,
discussed in the next paragraph), the new environment is used as the
source of the \envvar{PATH} variable.  The other variants,
\function{spawnl()}, \function{spawnle()}, \function{spawnv()}, and
\function{spawnve()}, will not use the \envvar{PATH} variable to
locate the executable; \var{path} must contain an appropriate absolute
or relative path.

For \function{spawnle()}, \function{spawnlpe()}, \function{spawnve()},
and \function{spawnvpe()} (note that these all end in \character{e}),
the \var{env} parameter must be a mapping which is used to define the
environment variables for the new process; the \function{spawnl()},
\function{spawnlp()}, \function{spawnv()}, and \function{spawnvp()}
all cause the new process to inherit the environment of the current
process.

As an example, the following calls to \function{spawnlp()} and
\function{spawnvpe()} are equivalent:

\begin{verbatim}
import os
os.spawnlp(os.P_WAIT, 'cp', 'cp', 'index.html', '/dev/null')

L = ['cp', 'index.html', '/dev/null']
os.spawnvpe(os.P_WAIT, 'cp', L, os.environ)
\end{verbatim}

Availability: \UNIX, Windows.  \function{spawnlp()},
\function{spawnlpe()}, \function{spawnvp()} and \function{spawnvpe()}
are not available on Windows.
\versionadded{1.6}
\end{funcdesc}

\begin{datadesc}{P_NOWAIT}
\dataline{P_NOWAITO}
Possible values for the \var{mode} parameter to the \function{spawn*()}
family of functions.  If either of these values is given, the
\function{spawn*()} functions will return as soon as the new process
has been created, with the process ID as the return value.
Availability: Macintosh, \UNIX, Windows.
\versionadded{1.6}
\end{datadesc}

\begin{datadesc}{P_WAIT}
Possible value for the \var{mode} parameter to the \function{spawn*()}
family of functions.  If this is given as \var{mode}, the
\function{spawn*()} functions will not return until the new process
has run to completion and will return the exit code of the process the
run is successful, or \code{-\var{signal}} if a signal kills the
process.
Availability: Macintosh, \UNIX, Windows.
\versionadded{1.6}
\end{datadesc}

\begin{datadesc}{P_DETACH}
\dataline{P_OVERLAY}
Possible values for the \var{mode} parameter to the
\function{spawn*()} family of functions.  These are less portable than
those listed above.
\constant{P_DETACH} is similar to \constant{P_NOWAIT}, but the new
process is detached from the console of the calling process.
If \constant{P_OVERLAY} is used, the current process will be replaced;
the \function{spawn*()} function will not return.
Availability: Windows.
\versionadded{1.6}
\end{datadesc}

\begin{funcdesc}{startfile}{path}
Start a file with its associated application.  This acts like
double-clicking the file in Windows Explorer, or giving the file name
as an argument to the \program{start} command from the interactive
command shell: the file is opened with whatever application (if any)
its extension is associated.

\function{startfile()} returns as soon as the associated application
is launched.  There is no option to wait for the application to close,
and no way to retrieve the application's exit status.  The \var{path}
parameter is relative to the current directory.  If you want to use an
absolute path, make sure the first character is not a slash
(\character{/}); the underlying Win32 \cfunction{ShellExecute()}
function doesn't work if it is.  Use the \function{os.path.normpath()}
function to ensure that the path is properly encoded for Win32.
Availability: Windows.
\versionadded{2.0}
\end{funcdesc}

\begin{funcdesc}{system}{command}
Execute the command (a string) in a subshell.  This is implemented by
calling the Standard C function \cfunction{system()}, and has the
same limitations.  Changes to \code{posix.environ}, \code{sys.stdin},
etc.\ are not reflected in the environment of the executed command.

On \UNIX, the return value is the exit status of the process encoded in the
format specified for \function{wait()}.  Note that \POSIX{} does not
specify the meaning of the return value of the C \cfunction{system()}
function, so the return value of the Python function is system-dependent.

On Windows, the return value is that returned by the system shell after
running \var{command}, given by the Windows environment variable
\envvar{COMSPEC}: on \program{command.com} systems (Windows 95, 98 and ME)
this is always \code{0}; on \program{cmd.exe} systems (Windows NT, 2000
and XP) this is the exit status of the command run; on systems using
a non-native shell, consult your shell documentation.

Availability: Macintosh, \UNIX, Windows.
\end{funcdesc}

\begin{funcdesc}{times}{}
Return a 5-tuple of floating point numbers indicating accumulated
(processor or other)
times, in seconds.  The items are: user time, system time, children's
user time, children's system time, and elapsed real time since a fixed
point in the past, in that order.  See the \UNIX{} manual page
\manpage{times}{2} or the corresponding Windows Platform API
documentation.
Availability: Macintosh, \UNIX, Windows.
\end{funcdesc}

\begin{funcdesc}{wait}{}
Wait for completion of a child process, and return a tuple containing
its pid and exit status indication: a 16-bit number, whose low byte is
the signal number that killed the process, and whose high byte is the
exit status (if the signal number is zero); the high bit of the low
byte is set if a core file was produced.
Availability: Macintosh, \UNIX.
\end{funcdesc}

\begin{funcdesc}{waitpid}{pid, options}
The details of this function differ on \UNIX{} and Windows.

On \UNIX:
Wait for completion of a child process given by process id \var{pid},
and return a tuple containing its process id and exit status
indication (encoded as for \function{wait()}).  The semantics of the
call are affected by the value of the integer \var{options}, which
should be \code{0} for normal operation.

If \var{pid} is greater than \code{0}, \function{waitpid()} requests
status information for that specific process.  If \var{pid} is
\code{0}, the request is for the status of any child in the process
group of the current process.  If \var{pid} is \code{-1}, the request
pertains to any child of the current process.  If \var{pid} is less
than \code{-1}, status is requested for any process in the process
group \code{-\var{pid}} (the absolute value of \var{pid}).

On Windows:
Wait for completion of a process given by process handle \var{pid},
and return a tuple containing \var{pid},
and its exit status shifted left by 8 bits (shifting makes cross-platform
use of the function easier).
A \var{pid} less than or equal to \code{0} has no special meaning on
Windows, and raises an exception.
The value of integer \var{options} has no effect.
\var{pid} can refer to any process whose id is known, not necessarily a
child process.
The \function{spawn()} functions called with \constant{P_NOWAIT}
return suitable process handles.
\end{funcdesc}

\begin{datadesc}{WNOHANG}
The option for \function{waitpid()} to return immediately if no child
process status is available immediately. The function returns
\code{(0, 0)} in this case.
Availability: Macintosh, \UNIX.
\end{datadesc}

\begin{datadesc}{WCONTINUED}
This option causes child processes to be reported if they have been
continued from a job control stop since their status was last
reported.
Availability: Some \UNIX{} systems.
\versionadded{2.3}
\end{datadesc}

\begin{datadesc}{WUNTRACED}
This option causes child processes to be reported if they have been
stopped but their current state has not been reported since they were
stopped.
Availability: Macintosh, \UNIX.
\versionadded{2.3}
\end{datadesc}

The following functions take a process status code as returned by
\function{system()}, \function{wait()}, or \function{waitpid()} as a
parameter.  They may be used to determine the disposition of a
process.

\begin{funcdesc}{WCOREDUMP}{status}
Returns \code{True} if a core dump was generated for the process,
otherwise it returns \code{False}.
Availability: Macintosh, \UNIX.
\versionadded{2.3}
\end{funcdesc}

\begin{funcdesc}{WIFCONTINUED}{status}
Returns \code{True} if the process has been continued from a job
control stop, otherwise it returns \code{False}.
Availability: \UNIX.
\versionadded{2.3}
\end{funcdesc}

\begin{funcdesc}{WIFSTOPPED}{status}
Returns \code{True} if the process has been stopped, otherwise it
returns \code{False}.
Availability: \UNIX.
\end{funcdesc}

\begin{funcdesc}{WIFSIGNALED}{status}
Returns \code{True} if the process exited due to a signal, otherwise
it returns \code{False}.
Availability: Macintosh, \UNIX.
\end{funcdesc}

\begin{funcdesc}{WIFEXITED}{status}
Returns \code{True} if the process exited using the \manpage{exit}{2}
system call, otherwise it returns \code{False}.
Availability: Macintosh, \UNIX.
\end{funcdesc}

\begin{funcdesc}{WEXITSTATUS}{status}
If \code{WIFEXITED(\var{status})} is true, return the integer
parameter to the \manpage{exit}{2} system call.  Otherwise, the return
value is meaningless.
Availability: Macintosh, \UNIX.
\end{funcdesc}

\begin{funcdesc}{WSTOPSIG}{status}
Return the signal which caused the process to stop.
Availability: Macintosh, \UNIX.
\end{funcdesc}

\begin{funcdesc}{WTERMSIG}{status}
Return the signal which caused the process to exit.
Availability: Macintosh, \UNIX.
\end{funcdesc}


\subsection{Miscellaneous System Information \label{os-path}}


\begin{funcdesc}{confstr}{name}
Return string-valued system configuration values.
\var{name} specifies the configuration value to retrieve; it may be a
string which is the name of a defined system value; these names are
specified in a number of standards (\POSIX, \UNIX{} 95, \UNIX{} 98, and
others).  Some platforms define additional names as well.  The names
known to the host operating system are given in the
\code{confstr_names} dictionary.  For configuration variables not
included in that mapping, passing an integer for \var{name} is also
accepted.
Availability: Macintosh, \UNIX.

If the configuration value specified by \var{name} isn't defined, the
empty string is returned.

If \var{name} is a string and is not known, \exception{ValueError} is
raised.  If a specific value for \var{name} is not supported by the
host system, even if it is included in \code{confstr_names}, an
\exception{OSError} is raised with \constant{errno.EINVAL} for the
error number.
\end{funcdesc}

\begin{datadesc}{confstr_names}
Dictionary mapping names accepted by \function{confstr()} to the
integer values defined for those names by the host operating system.
This can be used to determine the set of names known to the system.
Availability: Macintosh, \UNIX.
\end{datadesc}

\begin{funcdesc}{getloadavg}{}
Return the number of processes in the system run queue averaged over
the last 1, 5, and 15 minutes or raises OSError if the load average
was unobtainable.

\versionadded{2.3}
\end{funcdesc}

\begin{funcdesc}{sysconf}{name}
Return integer-valued system configuration values.
If the configuration value specified by \var{name} isn't defined,
\code{-1} is returned.  The comments regarding the \var{name}
parameter for \function{confstr()} apply here as well; the dictionary
that provides information on the known names is given by
\code{sysconf_names}.
Availability: Macintosh, \UNIX.
\end{funcdesc}

\begin{datadesc}{sysconf_names}
Dictionary mapping names accepted by \function{sysconf()} to the
integer values defined for those names by the host operating system.
This can be used to determine the set of names known to the system.
Availability: Macintosh, \UNIX.
\end{datadesc}


The follow data values are used to support path manipulation
operations.  These are defined for all platforms.

Higher-level operations on pathnames are defined in the
\refmodule{os.path} module.


\begin{datadesc}{curdir}
The constant string used by the operating system to refer to the current
directory.
For example: \code{'.'} for \POSIX{} or \code{':'} for Mac OS 9.
Also available via \module{os.path}.
\end{datadesc}

\begin{datadesc}{pardir}
The constant string used by the operating system to refer to the parent
directory.
For example: \code{'..'} for \POSIX{} or \code{'::'} for Mac OS 9.
Also available via \module{os.path}.
\end{datadesc}

\begin{datadesc}{sep}
The character used by the operating system to separate pathname components,
for example, \character{/} for \POSIX{} or \character{:} for
Mac OS 9.  Note that knowing this is not sufficient to be able to
parse or concatenate pathnames --- use \function{os.path.split()} and
\function{os.path.join()} --- but it is occasionally useful.
Also available via \module{os.path}.
\end{datadesc}

\begin{datadesc}{altsep}
An alternative character used by the operating system to separate pathname
components, or \code{None} if only one separator character exists.  This is
set to \character{/} on Windows systems where \code{sep} is a
backslash.
Also available via \module{os.path}.
\end{datadesc}

\begin{datadesc}{extsep}
The character which separates the base filename from the extension;
for example, the \character{.} in \file{os.py}.
Also available via \module{os.path}.
\versionadded{2.2}
\end{datadesc}

\begin{datadesc}{pathsep}
The character conventionally used by the operating system to separate
search path components (as in \envvar{PATH}), such as \character{:} for
\POSIX{} or \character{;} for Windows.
Also available via \module{os.path}.
\end{datadesc}

\begin{datadesc}{defpath}
The default search path used by \function{exec*p*()} and
\function{spawn*p*()} if the environment doesn't have a \code{'PATH'}
key.
Also available via \module{os.path}.
\end{datadesc}

\begin{datadesc}{linesep}
The string used to separate (or, rather, terminate) lines on the
current platform.  This may be a single character, such as \code{'\e
n'} for \POSIX{} or \code{'\e r'} for Mac OS, or multiple characters,
for example, \code{'\e r\e n'} for Windows.
\end{datadesc}

\begin{datadesc}{devnull}
The file path of the null device.
For example: \code{'/dev/null'} for \POSIX{} or \code{'Dev:Nul'} for
Mac OS 9.
Also available via \module{os.path}.
\versionadded{2.4}
\end{datadesc}


\subsection{Miscellaneous Functions \label{os-miscfunc}}

\begin{funcdesc}{urandom}{n}
Return a string of \var{n} random bytes suitable for cryptographic use.

This function returns random bytes from an OS-specific
randomness source.  The returned data should be unpredictable enough for
cryptographic applications, though its exact quality depends on the OS
implementation.  On a UNIX-like system this will query /dev/urandom, and
on Windows it will use CryptGenRandom.  If a randomness source is not
found, \exception{NotImplementedError} will be raised.
\versionadded{2.4}
\end{funcdesc}





\section{\module{time} ---
         Time access and conversions}

\declaremodule{builtin}{time}
\modulesynopsis{Time access and conversions.}


This module provides various time-related functions.
It is always available, but not all functions are available
on all platforms.

An explanation of some terminology and conventions is in order.

\begin{itemize}

\item
The \dfn{epoch}\index{epoch} is the point where the time starts.  On
January 1st of that year, at 0 hours, the ``time since the epoch'' is
zero.  For \UNIX{}, the epoch is 1970.  To find out what the epoch is,
look at \code{gmtime(0)}.

\item
The functions in this module do not handle dates and times before the
epoch or far in the future.  The cut-off point in the future is
determined by the C library; for \UNIX{}, it is typically in
2038\index{Year 2038}.

\item
\strong{Year 2000 (Y2K) issues}:\index{Year 2000}\index{Y2K}  Python
depends on the platform's C library, which generally doesn't have year
2000 issues, since all dates and times are represented internally as
seconds since the epoch.  Functions accepting a time tuple (see below)
generally require a 4-digit year.  For backward compatibility, 2-digit
years are supported if the module variable \code{accept2dyear} is a
non-zero integer; this variable is initialized to \code{1} unless the
environment variable \envvar{PYTHONY2K} is set to a non-empty string,
in which case it is initialized to \code{0}.  Thus, you can set
\envvar{PYTHONY2K} to a non-empty string in the environment to require 4-digit
years for all year input.  When 2-digit years are accepted, they are
converted according to the \POSIX{} or X/Open standard: values 69-99
are mapped to 1969-1999, and values 0--68 are mapped to 2000--2068.
Values 100--1899 are always illegal.  Note that this is new as of
Python 1.5.2(a2); earlier versions, up to Python 1.5.1 and 1.5.2a1,
would add 1900 to year values below 1900.

\item
UTC\index{UTC} is Coordinated Universal Time\index{Coordinated
Universal Time} (formerly known as Greenwich Mean
Time,\index{Greenwich Mean Time} or GMT).  The acronym UTC is not a
mistake but a compromise between English and French.

\item
DST is Daylight Saving Time,\index{Daylight Saving Time} an adjustment
of the timezone by (usually) one hour during part of the year.  DST
rules are magic (determined by local law) and can change from year to
year.  The C library has a table containing the local rules (often it
is read from a system file for flexibility) and is the only source of
True Wisdom in this respect.

\item
The precision of the various real-time functions may be less than
suggested by the units in which their value or argument is expressed.
E.g.\ on most \UNIX{} systems, the clock ``ticks'' only 50 or 100 times a
second, and on the Mac, times are only accurate to whole seconds.

\item
On the other hand, the precision of \function{time()} and
\function{sleep()} is better than their \UNIX{} equivalents: times are
expressed as floating point numbers, \function{time()} returns the
most accurate time available (using \UNIX{} \cfunction{gettimeofday()}
where available), and \function{sleep()} will accept a time with a
nonzero fraction (\UNIX{} \cfunction{select()} is used to implement
this, where available).

\item

The time tuple as returned by \function{gmtime()},
\function{localtime()}, and \function{strptime()}, and accepted by
\function{asctime()}, \function{mktime()} and \function{strftime()},
is a tuple of 9 integers:

\begin{tableiii}{r|l|l}{textrm}{Index}{Field}{Values}
  \lineiii{0}{year}{(e.g.\ 1993)}
  \lineiii{1}{month}{range [1,12]}
  \lineiii{2}{day}{range [1,31]}
  \lineiii{3}{hour}{range [0,23]}
  \lineiii{4}{minute}{range [0,59]}
  \lineiii{5}{second}{range [0,61]; see \strong{(1)} in \function{strftime()} description}
  \lineiii{6}{weekday}{range [0,6], monday is 0}
  \lineiii{7}{Julian day}{range [1,366]}
  \lineiii{8}{daylight savings flag}{0, 1 or -1; see below}
\end{tableiii}

Note that unlike the C structure, the month value is a
range of 1-12, not 0-11.  A year value will be handled as described
under ``Year 2000 (Y2K) issues'' above.  A \code{-1} argument as
daylight savings flag, passed to \function{mktime()} will usually
result in the correct daylight savings state to be filled in.

\end{itemize}

The module defines the following functions and data items:


\begin{datadesc}{accept2dyear}
Boolean value indicating whether two-digit year values will be
accepted.  This is true by default, but will be set to false if the
environment variable \envvar{PYTHONY2K} has been set to a non-empty
string.  It may also be modified at run time.
\end{datadesc}

\begin{datadesc}{altzone}
The offset of the local DST timezone, in seconds west of UTC, if one
is defined.  This is negative if the local DST timezone is east of UTC
(as in Western Europe, including the UK).  Only use this if
\code{daylight} is nonzero.
\end{datadesc}

\begin{funcdesc}{asctime}{tuple}
Convert a tuple representing a time as returned by \function{gmtime()}
or \function{localtime()} to a 24-character string of the following form:
\code{'Sun Jun 20 23:21:05 1993'}.  Note: unlike the C function of
the same name, there is no trailing newline.
\end{funcdesc}

\begin{funcdesc}{clock}{}
Return the current CPU time as a floating point number expressed in
seconds.  The precision, and in fact the very definiton of the meaning
of ``CPU time''\index{CPU time}, depends on that of the C function
of the same name, but in any case, this is the function to use for
benchmarking\index{benchmarking} Python or timing algorithms.
\end{funcdesc}

\begin{funcdesc}{ctime}{secs}
Convert a time expressed in seconds since the epoch to a string
representing local time.  \code{ctime(\var{secs})} is equivalent to
\code{asctime(localtime(\var{secs}))}.
\end{funcdesc}

\begin{datadesc}{daylight}
Nonzero if a DST timezone is defined.
\end{datadesc}

\begin{funcdesc}{gmtime}{secs}
Convert a time expressed in seconds since the epoch to a time tuple
in UTC in which the dst flag is always zero.  Fractions of a second are
ignored.  See above for a description of the tuple lay-out.
\end{funcdesc}

\begin{funcdesc}{localtime}{secs}
Like \function{gmtime()} but converts to local time.  The dst flag is
set to \code{1} when DST applies to the given time.
\end{funcdesc}

\begin{funcdesc}{mktime}{tuple}
This is the inverse function of \function{localtime()}.  Its argument
is the full 9-tuple (since the dst flag is needed --- pass \code{-1}
as the dst flag if it is unknown) which expresses the time in
\emph{local} time, not UTC.  It returns a floating point number, for
compatibility with \function{time()}.  If the input value cannot be
represented as a valid time, \exception{OverflowError} is raised.
\end{funcdesc}

\begin{funcdesc}{sleep}{secs}
Suspend execution for the given number of seconds.  The argument may
be a floating point number to indicate a more precise sleep time.
The actual suspension time may be less than that requested because any
caught signal will terminate the \function{sleep()} following
execution of that signal's catching routine.  Also, the suspension
time may be longer than requested by an arbitrary amount because of
the scheduling of other activity in the system.
\end{funcdesc}

\begin{funcdesc}{strftime}{format, tuple}
Convert a tuple representing a time as returned by \function{gmtime()}
or \function{localtime()} to a string as specified by the \var{format}
argument.  \var{format} must be a string.

The following directives can be embedded in the \var{format} string.
They are shown without the optional field width and precision
specification, and are replaced by the indicated characters in the
\function{strftime()} result:

\begin{tableiii}{c|p{24em}|c}{code}{Directive}{Meaning}{Notes}
  \lineiii{\%a}{Locale's abbreviated weekday name.}{}
  \lineiii{\%A}{Locale's full weekday name.}{}
  \lineiii{\%b}{Locale's abbreviated month name.}{}
  \lineiii{\%B}{Locale's full month name.}{}
  \lineiii{\%c}{Locale's appropriate date and time representation.}{}
  \lineiii{\%d}{Day of the month as a decimal number [01,31].}{}
  \lineiii{\%H}{Hour (24-hour clock) as a decimal number [00,23].}{}
  \lineiii{\%I}{Hour (12-hour clock) as a decimal number [01,12].}{}
  \lineiii{\%j}{Day of the year as a decimal number [001,366].}{}
  \lineiii{\%m}{Month as a decimal number [01,12].}{}
  \lineiii{\%M}{Minute as a decimal number [00,59].}{}
  \lineiii{\%p}{Locale's equivalent of either AM or PM.}{}
  \lineiii{\%S}{Second as a decimal number [00,61].}{(1)}
  \lineiii{\%U}{Week number of the year (Sunday as the first day of the
                week) as a decimal number [00,53].  All days in a new year
                preceding the first Sunday are considered to be in week 0.}{}
  \lineiii{\%w}{Weekday as a decimal number [0(Sunday),6].}{}
  \lineiii{\%W}{Week number of the year (Monday as the first day of the
                week) as a decimal number [00,53].  All days in a new year
                preceding the first Sunday are considered to be in week 0.}{}
  \lineiii{\%x}{Locale's appropriate date representation.}{}
  \lineiii{\%X}{Locale's appropriate time representation.}{}
  \lineiii{\%y}{Year without century as a decimal number [00,99].}{}
  \lineiii{\%Y}{Year with century as a decimal number.}{}
  \lineiii{\%Z}{Time zone name (or by no characters if no time zone exists).}{}
  \lineiii{\%\%}{A literal \character{\%} character.}{}
\end{tableiii}

\noindent
Notes:

\begin{description}
  \item[(1)]
    The range really is \code{0} to \code{61}; this accounts for leap
    seconds and the (very rare) double leap seconds.
\end{description}

Additional directives may be supported on certain platforms, but
only the ones listed here have a meaning standardized by ANSI C.

On some platforms, an optional field width and precision
specification can immediately follow the initial \character{\%} of a
directive in the following order; this is also not portable.
The field width is normally 2 except for \code{\%j} where it is 3.
\end{funcdesc}

\begin{funcdesc}{strptime}{string\optional{, format}}
Parse a string representing a time according to a format.  The return 
value is a tuple as returned by \function{gmtime()} or
\function{localtime()}.  The \var{format} parameter uses the same
directives as those used by \function{strftime()}; it defaults to
\code{"\%a \%b \%d \%H:\%M:\%S \%Y"} which matches the formatting
returned by \function{ctime()}.  The same platform caveats apply; see
the local \UNIX{} documentation for restrictions or additional
supported directives.  If \var{string} cannot be parsed according to
\var{format}, \exception{ValueError} is raised.

Availability: Most modern \UNIX{} systems.
\end{funcdesc}

\begin{funcdesc}{time}{}
Return the time as a floating point number expressed in seconds since
the epoch, in UTC.  Note that even though the time is always returned
as a floating point number, not all systems provide time with a better
precision than 1 second.
\end{funcdesc}

\begin{datadesc}{timezone}
The offset of the local (non-DST) timezone, in seconds west of UTC
(i.e. negative in most of Western Europe, positive in the US, zero in
the UK).
\end{datadesc}

\begin{datadesc}{tzname}
A tuple of two strings: the first is the name of the local non-DST
timezone, the second is the name of the local DST timezone.  If no DST
timezone is defined, the second string should not be used.
\end{datadesc}


\begin{seealso}
  \seemodule{locale}{Internationalization services.  The locale
                     settings can affect the return values for some of 
                     the functions in the \module{time} module.}
\end{seealso}

\section{\module{getpass}
         --- Portable password input}

\declaremodule{standard}{getpass}
\modulesynopsis{Portable reading of passwords and retrieval of the userid.}
\moduleauthor{Piers Lauder}{piers@cs.su.oz.au}
% Windows (& Mac?) support by Guido van Rossum.
\sectionauthor{Fred L. Drake, Jr.}{fdrake@acm.org}


The \module{getpass} module provides two functions:


\begin{funcdesc}{getpass}{\optional{prompt}}
  Prompt the user for a password without echoing.  The user is
  prompted using the string \var{prompt}, which defaults to
  \code{'Password: '}.
  Availability: Macintosh, \UNIX, Windows.
\end{funcdesc}


\begin{funcdesc}{getuser}{}
  Return the ``login name'' of the user.
  Availability: \UNIX, Windows.

  This function checks the environment variables \envvar{LOGNAME},
  \envvar{USER}, \envvar{LNAME} and \envvar{USERNAME}, in order, and
  returns the value of the first one which is set to a non-empty
  string.  If none are set, the login name from the password database
  is returned on systems which support the \refmodule{pwd} module,
  otherwise, an exception is raised.
\end{funcdesc}

\section{\module{getopt} ---
         Parser for command line options}

\declaremodule{standard}{getopt}
\modulesynopsis{Portable parser for command line options; support both
                short and long option names.}


This module helps scripts to parse the command line arguments in
\code{sys.argv}.
It supports the same conventions as the \UNIX{} \cfunction{getopt()}
function (including the special meanings of arguments of the form
`\code{-}' and `\code{-}\code{-}').
% That's to fool latex2html into leaving the two hyphens alone!
Long options similar to those supported by
GNU software may be used as well via an optional third argument.
This module provides a single function and an exception:

\begin{funcdesc}{getopt}{args, options\optional{, long_options}}
Parses command line options and parameter list.  \var{args} is the
argument list to be parsed, without the leading reference to the
running program. Typically, this means \samp{sys.argv[1:]}.
\var{options} is the string of option letters that the script wants to
recognize, with options that require an argument followed by a colon
(\character{:}; i.e., the same format that \UNIX{}
\cfunction{getopt()} uses).

\note{Unlike GNU \cfunction{getopt()}, after a non-option
argument, all further arguments are considered also non-options.
This is similar to the way non-GNU \UNIX{} systems work.}

\var{long_options}, if specified, must be a list of strings with the
names of the long options which should be supported.  The leading
\code{'-}\code{-'} characters should not be included in the option
name.  Long options which require an argument should be followed by an
equal sign (\character{=}).  To accept only long options,
\var{options} should be an empty string.  Long options on the command
line can be recognized so long as they provide a prefix of the option
name that matches exactly one of the accepted options.  For example,
if \var{long_options} is \code{['foo', 'frob']}, the option
\longprogramopt{fo} will match as \longprogramopt{foo}, but
\longprogramopt{f} will not match uniquely, so \exception{GetoptError}
will be raised.

The return value consists of two elements: the first is a list of
\code{(\var{option}, \var{value})} pairs; the second is the list of
program arguments left after the option list was stripped (this is a
trailing slice of \var{args}).  Each option-and-value pair returned
has the option as its first element, prefixed with a hyphen for short
options (e.g., \code{'-x'}) or two hyphens for long options (e.g.,
\code{'-}\code{-long-option'}), and the option argument as its second
element, or an empty string if the option has no argument.  The
options occur in the list in the same order in which they were found,
thus allowing multiple occurrences.  Long and short options may be
mixed.
\end{funcdesc}

\begin{funcdesc}{gnu_getopt}{args, options\optional{, long_options}}
This function works like \function{getopt()}, except that GNU style
scanning mode is used by default. This means that option and
non-option arguments may be intermixed. The \function{getopt()}
function stops processing options as soon as a non-option argument is
encountered.

If the first character of the option string is `+', or if the
environment variable POSIXLY_CORRECT is set, then option processing
stops as soon as a non-option argument is encountered.

\versionadded{2.3}
\end{funcdesc}

\begin{excdesc}{GetoptError}
This is raised when an unrecognized option is found in the argument
list or when an option requiring an argument is given none.
The argument to the exception is a string indicating the cause of the
error.  For long options, an argument given to an option which does
not require one will also cause this exception to be raised.  The
attributes \member{msg} and \member{opt} give the error message and
related option; if there is no specific option to which the exception
relates, \member{opt} is an empty string.

\versionchanged[Introduced \exception{GetoptError} as a synonym for
                \exception{error}]{1.6}
\end{excdesc}

\begin{excdesc}{error}
Alias for \exception{GetoptError}; for backward compatibility.
\end{excdesc}


An example using only \UNIX{} style options:

\begin{verbatim}
>>> import getopt
>>> args = '-a -b -cfoo -d bar a1 a2'.split()
>>> args
['-a', '-b', '-cfoo', '-d', 'bar', 'a1', 'a2']
>>> optlist, args = getopt.getopt(args, 'abc:d:')
>>> optlist
[('-a', ''), ('-b', ''), ('-c', 'foo'), ('-d', 'bar')]
>>> args
['a1', 'a2']
\end{verbatim}

Using long option names is equally easy:

\begin{verbatim}
>>> s = '--condition=foo --testing --output-file abc.def -x a1 a2'
>>> args = s.split()
>>> args
['--condition=foo', '--testing', '--output-file', 'abc.def', '-x', 'a1', 'a2']
>>> optlist, args = getopt.getopt(args, 'x', [
...     'condition=', 'output-file=', 'testing'])
>>> optlist
[('--condition', 'foo'), ('--testing', ''), ('--output-file', 'abc.def'), ('-x',
 '')]
>>> args
['a1', 'a2']
\end{verbatim}

In a script, typical usage is something like this:

\begin{verbatim}
import getopt, sys

def main():
    try:
        opts, args = getopt.getopt(sys.argv[1:], "ho:v", ["help", "output="])
    except getopt.GetoptError:
        # print help information and exit:
        usage()
        sys.exit(2)
    output = None
    verbose = False
    for o, a in opts:
        if o == "-v":
            verbose = True
        if o in ("-h", "--help"):
            usage()
            sys.exit()
        if o in ("-o", "--output"):
            output = a
    # ...

if __name__ == "__main__":
    main()
\end{verbatim}

\begin{seealso}
  \seemodule{optparse}{More object-oriented command line option parsing.}
\end{seealso}

\section{\module{tempfile} ---
         Generate temporary file names.}
\declaremodule{standard}{tempfile}

\modulesynopsis{Generate temporary file names.}

\indexii{temporary}{file name}
\indexii{temporary}{file}


This module generates temporary file names.  It is not \UNIX{} specific,
but it may require some help on non-\UNIX{} systems.

Note: the modules does not create temporary files, nor does it
automatically remove them when the current process exits or dies.

The module defines a single user-callable function:

\begin{funcdesc}{mktemp}{}
Return a unique temporary filename.  This is an absolute pathname of a
file that does not exist at the time the call is made.  No two calls
will return the same filename.
\end{funcdesc}

The module uses two global variables that tell it how to construct a
temporary name.  The caller may assign values to them; by default they
are initialized at the first call to \function{mktemp()}.

\begin{datadesc}{tempdir}
When set to a value other than \code{None}, this variable defines the
directory in which filenames returned by \function{mktemp()} reside.
The default is taken from the environment variable \envvar{TMPDIR}; if
this is not set, either \file{/usr/tmp} is used (on \UNIX{}), or the
current working directory (all other systems).  No check is made to
see whether its value is valid.
\end{datadesc}

\begin{datadesc}{template}
When set to a value other than \code{None}, this variable defines the
prefix of the final component of the filenames returned by
\function{mktemp()}.  A string of decimal digits is added to generate
unique filenames.  The default is either \file{@\var{pid}.} where
\var{pid} is the current process ID (on \UNIX{}), or \file{tmp} (all
other systems).
\end{datadesc}

\strong{Warning:} if a \UNIX{} process uses \code{mktemp()}, then
calls \function{fork()} and both parent and child continue to use
\function{mktemp()}, the processes will generate conflicting temporary
names.  To resolve this, the child process should assign \code{None}
to \code{template}, to force recomputing the default on the next call
to \function{mktemp()}.

\section{Standard Module \sectcode{errno}}
\stmodindex{errno}

\renewcommand{\indexsubitem}{(in module errno)}

This module makes available standard errno system symbols.
The value of each symbol is the corresponding integer value.
The names and descriptions are borrowed from linux/include/errno.h,
which should be pretty all-inclusive.  Of the following list, symbols
that are not used on the current platform are not defined by the
module.

Symbols available can include:
\begin{datadesc}{EPERM} Operation not permitted \end{datadesc}
\begin{datadesc}{ENOENT} No such file or directory \end{datadesc}
\begin{datadesc}{ESRCH} No such process \end{datadesc}
\begin{datadesc}{EINTR} Interrupted system call \end{datadesc}
\begin{datadesc}{EIO} I/O error \end{datadesc}
\begin{datadesc}{ENXIO} No such device or address \end{datadesc}
\begin{datadesc}{E2BIG} Arg list too long \end{datadesc}
\begin{datadesc}{ENOEXEC} Exec format error \end{datadesc}
\begin{datadesc}{EBADF} Bad file number \end{datadesc}
\begin{datadesc}{ECHILD} No child processes \end{datadesc}
\begin{datadesc}{EAGAIN} Try again \end{datadesc}
\begin{datadesc}{ENOMEM} Out of memory \end{datadesc}
\begin{datadesc}{EACCES} Permission denied \end{datadesc}
\begin{datadesc}{EFAULT} Bad address \end{datadesc}
\begin{datadesc}{ENOTBLK} Block device required \end{datadesc}
\begin{datadesc}{EBUSY} Device or resource busy \end{datadesc}
\begin{datadesc}{EEXIST} File exists \end{datadesc}
\begin{datadesc}{EXDEV} Cross-device link \end{datadesc}
\begin{datadesc}{ENODEV} No such device \end{datadesc}
\begin{datadesc}{ENOTDIR} Not a directory \end{datadesc}
\begin{datadesc}{EISDIR} Is a directory \end{datadesc}
\begin{datadesc}{EINVAL} Invalid argument \end{datadesc}
\begin{datadesc}{ENFILE} File table overflow \end{datadesc}
\begin{datadesc}{EMFILE} Too many open files \end{datadesc}
\begin{datadesc}{ENOTTY} Not a typewriter \end{datadesc}
\begin{datadesc}{ETXTBSY} Text file busy \end{datadesc}
\begin{datadesc}{EFBIG} File too large \end{datadesc}
\begin{datadesc}{ENOSPC} No space left on device \end{datadesc}
\begin{datadesc}{ESPIPE} Illegal seek \end{datadesc}
\begin{datadesc}{EROFS} Read-only file system \end{datadesc}
\begin{datadesc}{EMLINK} Too many links \end{datadesc}
\begin{datadesc}{EPIPE} Broken pipe \end{datadesc}
\begin{datadesc}{EDOM} Math argument out of domain of func \end{datadesc}
\begin{datadesc}{ERANGE} Math result not representable \end{datadesc}
\begin{datadesc}{EDEADLK} Resource deadlock would occur \end{datadesc}
\begin{datadesc}{ENAMETOOLONG} File name too long \end{datadesc}
\begin{datadesc}{ENOLCK} No record locks available \end{datadesc}
\begin{datadesc}{ENOSYS} Function not implemented \end{datadesc}
\begin{datadesc}{ENOTEMPTY} Directory not empty \end{datadesc}
\begin{datadesc}{ELOOP} Too many symbolic links encountered \end{datadesc}
\begin{datadesc}{EWOULDBLOCK} Operation would block \end{datadesc}
\begin{datadesc}{ENOMSG} No message of desired type \end{datadesc}
\begin{datadesc}{EIDRM} Identifier removed \end{datadesc}
\begin{datadesc}{ECHRNG} Channel number out of range \end{datadesc}
\begin{datadesc}{EL2NSYNC} Level 2 not synchronized \end{datadesc}
\begin{datadesc}{EL3HLT} Level 3 halted \end{datadesc}
\begin{datadesc}{EL3RST} Level 3 reset \end{datadesc}
\begin{datadesc}{ELNRNG} Link number out of range \end{datadesc}
\begin{datadesc}{EUNATCH} Protocol driver not attached \end{datadesc}
\begin{datadesc}{ENOCSI} No CSI structure available \end{datadesc}
\begin{datadesc}{EL2HLT} Level 2 halted \end{datadesc}
\begin{datadesc}{EBADE} Invalid exchange \end{datadesc}
\begin{datadesc}{EBADR} Invalid request descriptor \end{datadesc}
\begin{datadesc}{EXFULL} Exchange full \end{datadesc}
\begin{datadesc}{ENOANO} No anode \end{datadesc}
\begin{datadesc}{EBADRQC} Invalid request code \end{datadesc}
\begin{datadesc}{EBADSLT} Invalid slot \end{datadesc}
\begin{datadesc}{EDEADLOCK} File locking deadlock error \end{datadesc}
\begin{datadesc}{EBFONT} Bad font file format \end{datadesc}
\begin{datadesc}{ENOSTR} Device not a stream \end{datadesc}
\begin{datadesc}{ENODATA} No data available \end{datadesc}
\begin{datadesc}{ETIME} Timer expired \end{datadesc}
\begin{datadesc}{ENOSR} Out of streams resources \end{datadesc}
\begin{datadesc}{ENONET} Machine is not on the network \end{datadesc}
\begin{datadesc}{ENOPKG} Package not installed \end{datadesc}
\begin{datadesc}{EREMOTE} Object is remote \end{datadesc}
\begin{datadesc}{ENOLINK} Link has been severed \end{datadesc}
\begin{datadesc}{EADV} Advertise error \end{datadesc}
\begin{datadesc}{ESRMNT} Srmount error \end{datadesc}
\begin{datadesc}{ECOMM} Communication error on send \end{datadesc}
\begin{datadesc}{EPROTO} Protocol error \end{datadesc}
\begin{datadesc}{EMULTIHOP} Multihop attempted \end{datadesc}
\begin{datadesc}{EDOTDOT} RFS specific error \end{datadesc}
\begin{datadesc}{EBADMSG} Not a data message \end{datadesc}
\begin{datadesc}{EOVERFLOW} Value too large for defined data type \end{datadesc}
\begin{datadesc}{ENOTUNIQ} Name not unique on network \end{datadesc}
\begin{datadesc}{EBADFD} File descriptor in bad state \end{datadesc}
\begin{datadesc}{EREMCHG} Remote address changed \end{datadesc}
\begin{datadesc}{ELIBACC} Can not access a needed shared library \end{datadesc}
\begin{datadesc}{ELIBBAD} Accessing a corrupted shared library \end{datadesc}
\begin{datadesc}{ELIBSCN} .lib section in a.out corrupted \end{datadesc}
\begin{datadesc}{ELIBMAX} Attempting to link in too many shared libraries \end{datadesc}
\begin{datadesc}{ELIBEXEC} Cannot exec a shared library directly \end{datadesc}
\begin{datadesc}{EILSEQ} Illegal byte sequence \end{datadesc}
\begin{datadesc}{ERESTART} Interrupted system call should be restarted \end{datadesc}
\begin{datadesc}{ESTRPIPE} Streams pipe error \end{datadesc}
\begin{datadesc}{EUSERS} Too many users \end{datadesc}
\begin{datadesc}{ENOTSOCK} Socket operation on non-socket \end{datadesc}
\begin{datadesc}{EDESTADDRREQ} Destination address required \end{datadesc}
\begin{datadesc}{EMSGSIZE} Message too long \end{datadesc}
\begin{datadesc}{EPROTOTYPE} Protocol wrong type for socket \end{datadesc}
\begin{datadesc}{ENOPROTOOPT} Protocol not available \end{datadesc}
\begin{datadesc}{EPROTONOSUPPORT} Protocol not supported \end{datadesc}
\begin{datadesc}{ESOCKTNOSUPPORT} Socket type not supported \end{datadesc}
\begin{datadesc}{EOPNOTSUPP} Operation not supported on transport endpoint \end{datadesc}
\begin{datadesc}{EPFNOSUPPORT} Protocol family not supported \end{datadesc}
\begin{datadesc}{EAFNOSUPPORT} Address family not supported by protocol \end{datadesc}
\begin{datadesc}{EADDRINUSE} Address already in use \end{datadesc}
\begin{datadesc}{EADDRNOTAVAIL} Cannot assign requested address \end{datadesc}
\begin{datadesc}{ENETDOWN} Network is down \end{datadesc}
\begin{datadesc}{ENETUNREACH} Network is unreachable \end{datadesc}
\begin{datadesc}{ENETRESET} Network dropped connection because of reset \end{datadesc}
\begin{datadesc}{ECONNABORTED} Software caused connection abort \end{datadesc}
\begin{datadesc}{ECONNRESET} Connection reset by peer \end{datadesc}
\begin{datadesc}{ENOBUFS} No buffer space available \end{datadesc}
\begin{datadesc}{EISCONN} Transport endpoint is already connected \end{datadesc}
\begin{datadesc}{ENOTCONN} Transport endpoint is not connected \end{datadesc}
\begin{datadesc}{ESHUTDOWN} Cannot send after transport endpoint shutdown \end{datadesc}
\begin{datadesc}{ETOOMANYREFS} Too many references: cannot splice \end{datadesc}
\begin{datadesc}{ETIMEDOUT} Connection timed out \end{datadesc}
\begin{datadesc}{ECONNREFUSED} Connection refused \end{datadesc}
\begin{datadesc}{EHOSTDOWN} Host is down \end{datadesc}
\begin{datadesc}{EHOSTUNREACH} No route to host \end{datadesc}
\begin{datadesc}{EALREADY} Operation already in progress \end{datadesc}
\begin{datadesc}{EINPROGRESS} Operation now in progress \end{datadesc}
\begin{datadesc}{ESTALE} Stale NFS file handle \end{datadesc}
\begin{datadesc}{EUCLEAN} Structure needs cleaning \end{datadesc}
\begin{datadesc}{ENOTNAM} Not a XENIX named type file \end{datadesc}
\begin{datadesc}{ENAVAIL} No XENIX semaphores available \end{datadesc}
\begin{datadesc}{EISNAM} Is a named type file \end{datadesc}
\begin{datadesc}{EREMOTEIO} Remote I/O error \end{datadesc}
\begin{datadesc}{EDQUOT} Quota exceeded \end{datadesc}


\section{Standard Module \sectcode{glob}}
\stmodindex{glob}
\renewcommand{\indexsubitem}{(in module glob)}

The \code{glob} module finds all the pathnames matching a specified
pattern according to the rules used by the \UNIX{} shell.  No tilde
expansion is done, but \verb\*\, \verb\?\, and character ranges
expressed with \verb\[]\ will be correctly matched.  This is done by
using the \code{os.listdir()} and \code{fnmatch.fnmatch()} functions
in concert, and not by actually invoking a subshell.  (For tilde and
shell variable expansion, use \code{os.path.expanduser(}) and
\code{os.path.expandvars()}.)

\begin{funcdesc}{glob}{pathname}
Returns a possibly-empty list of path names that match \var{pathname},
which must be a string containing a path specification.
\var{pathname} can be either absolute (like
\file{/usr/src/Python1.4/Makefile}) or relative (like
\file{../../Tools/*.gif}), and can contain shell-style wildcards.
\end{funcdesc}

For example, consider a directory containing only the following files:
\file{1.gif}, \file{2.txt}, and \file{card.gif}.  \code{glob.glob()}
will produce the following results.  Notice how any leading components
of the path are preserved.

\begin{verbatim}
>>> import glob
>>> glob.glob('./[0-9].*')
['./1.gif', './2.txt']
>>> glob.glob('*.gif')
['1.gif', 'card.gif']
>>> glob.glob('?.gif')
['1.gif']
\end{verbatim}

\section{Standard Module \sectcode{fnmatch}}
\label{module-fnmatch}
\stmodindex{fnmatch}

This module provides support for Unix shell-style wildcards, which are
\emph{not} the same as Python's regular expressions (which are
documented in the \code{re} module).  The special characters used
in shell-style wildcards are:
\begin{itemize}
\item[\code{*}] matches everything
\item[\code{?}]	matches any single character
\item[\code{[}\var{seq}\code{]}] matches any character in \var{seq}
\item[\code{[!}\var{seq}\code{]}] matches any character not in \var{seq}
\end{itemize}

Note that the filename separator (\code{'/'} on Unix) is \emph{not}
special to this module.  See module \code{glob} for pathname expansion
(\code{glob} uses \code{fnmatch} to match filename segments).

\renewcommand{\indexsubitem}{(in module fnmatch)}

\begin{funcdesc}{fnmatch}{filename\, pattern}
Test whether the \var{filename} string matches the \var{pattern}
string, returning true or false.  If the operating system is
case-insensitive, then both parameters will be normalized to all
lower- or upper-case before the comparision is performed.  If you
require a case-sensitive comparision regardless of whether that's
standard for your operating system, use \code{fnmatchcase()} instead.
\end{funcdesc}

\begin{funcdesc}{fnmatchcase}{}
Test whether \var{filename} matches \var{pattern}, returning true or
false; the comparision is case-sensitive.
\end{funcdesc}

\begin{funcdesc}{translate}{pattern}
Translate a shell pattern into a corresponding regular expression,
returning a string describing the pattern.  It does not compile the
expression.  
\end{funcdesc}


\section{\module{shutil} ---
         High-level file operations}

\declaremodule{standard}{shutil}
\modulesynopsis{High-level file operations, including copying.}
\sectionauthor{Fred L. Drake, Jr.}{fdrake@acm.org}
% partly based on the docstrings


The \module{shutil} module offers a number of high-level operations on
files and collections of files.  In particular, functions are provided 
which support file copying and removal.
\index{file!copying}
\index{copying files}

\strong{Caveat:}  On MacOS, the resource fork and other metadata are
not used.  For file copies, this means that resources will be lost and 
file type and creator codes will not be correct.


\begin{funcdesc}{copyfile}{src, dst}
  Copy the contents of \var{src} to \var{dst}.  If \var{dst} exists,
  it will be replaced, otherwise it will be created.
\end{funcdesc}

\begin{funcdesc}{copyfileobj}{fsrc, fdst\optional{, length}}
  Copy the contents of the file-like object \var{fsrc} to the
  file-like object \var{fdst}.  The integer \var{length}, if given,
  is the buffer size. In particular, a negative \var{length} value
  means to copy the data without looping over the source data in
  chunks; by default the data is read in chunks to avoid uncontrolled
  memory consumption.
\end{funcdesc}

\begin{funcdesc}{copymode}{src, dst}
  Copy the permission bits from \var{src} to \var{dst}.  The file
  contents, owner, and group are unaffected.
\end{funcdesc}

\begin{funcdesc}{copystat}{src, dst}
  Copy the permission bits, last access time, and last modification
  time from \var{src} to \var{dst}.  The file contents, owner, and
  group are unaffected.
\end{funcdesc}

\begin{funcdesc}{copy}{src, dst}
  Copy the file \var{src} to the file or directory \var{dst}.  If
  \var{dst} is a directory, a file with the same basename as \var{src} 
  is created (or overwritten) in the directory specified.  Permission
  bits are copied.
\end{funcdesc}

\begin{funcdesc}{copy2}{src, dst}
  Similar to \function{copy()}, but last access time and last
  modification time are copied as well.  This is similar to the
  \UNIX{} command \program{cp} \programopt{-p}.
\end{funcdesc}

\begin{funcdesc}{copytree}{src, dst\optional{, symlinks}}
  Recursively copy an entire directory tree rooted at \var{src}.  The
  destination directory, named by \var{dst}, must not already exist;
  it will be created.  Individual files are copied using
  \function{copy2()}.  If \var{symlinks} is true, symbolic links in
  the source tree are represented as symbolic links in the new tree;
  if false or omitted, the contents of the linked files are copied to
  the new tree.  Errors are reported to standard output.

  The source code for this should be considered an example rather than 
  a tool.
\end{funcdesc}

\begin{funcdesc}{rmtree}{path\optional{, ignore_errors\optional{, onerror}}}
\index{directory!deleting}
  Delete an entire directory tree.  If \var{ignore_errors} is true,
  errors will be ignored; if false or omitted, errors are handled by
  calling a handler specified by \var{onerror} or raise an exception.

  If \var{onerror} is provided, it must be a callable that accepts
  three parameters: \var{function}, \var{path}, and \var{excinfo}.
  The first parameter, \var{function}, is the function which raised
  the exception; it will be \function{os.remove()} or
  \function{os.rmdir()}.  The second parameter, \var{path}, will be
  the path name passed to \var{function}.  The third parameter,
  \var{excinfo}, will be the exception information return by
  \function{sys.exc_info()}.  Exceptions raised by \var{onerror} will
  not be caught.
\end{funcdesc}


\subsection{Example \label{shutil-example}}

This example is the implementation of the \function{copytree()}
function, described above, with the docstring omitted.  It
demonstrates many of the other functions provided by this module.

\begin{verbatim}
def copytree(src, dst, symlinks=0):
    names = os.listdir(src)
    os.mkdir(dst)
    for name in names:
        srcname = os.path.join(src, name)
        dstname = os.path.join(dst, name)
        try:
            if symlinks and os.path.islink(srcname):
                linkto = os.readlink(srcname)
                os.symlink(linkto, dstname)
            elif os.path.isdir(srcname):
                copytree(srcname, dstname)
            else:
                copy2(srcname, dstname)
            # XXX What about devices, sockets etc.?
        except (IOError, os.error), why:
            print "Can't copy %s to %s: %s" % (`srcname`, `dstname`, str(why))
\end{verbatim}

\section{\module{locale} ---
         Internationalization services}

\declaremodule{standard}{locale}
\modulesynopsis{Internationalization services.}
\moduleauthor{Martin von L\"owis}{martin@v.loewis.de}
\sectionauthor{Martin von L\"owis}{martin@v.loewis.de}


The \module{locale} module opens access to the \POSIX{} locale
database and functionality. The \POSIX{} locale mechanism allows
programmers to deal with certain cultural issues in an application,
without requiring the programmer to know all the specifics of each
country where the software is executed.

The \module{locale} module is implemented on top of the
\module{_locale}\refbimodindex{_locale} module, which in turn uses an
ANSI C locale implementation if available.

The \module{locale} module defines the following exception and
functions:


\begin{excdesc}{Error}
  Exception raised when \function{setlocale()} fails.
\end{excdesc}

\begin{funcdesc}{setlocale}{category\optional{, locale}}
  If \var{locale} is specified, it may be a string, a tuple of the
  form \code{(\var{language code}, \var{encoding})}, or \code{None}.
  If it is a tuple, it is converted to a string using the locale
  aliasing engine.  If \var{locale} is given and not \code{None},
  \function{setlocale()} modifies the locale setting for the
  \var{category}.  The available categories are listed in the data
  description below.  The value is the name of a locale.  An empty
  string specifies the user's default settings. If the modification of
  the locale fails, the exception \exception{Error} is raised.  If
  successful, the new locale setting is returned.

  If \var{locale} is omitted or \code{None}, the current setting for
  \var{category} is returned.

  \function{setlocale()} is not thread safe on most systems.
  Applications typically start with a call of

\begin{verbatim}
import locale
locale.setlocale(locale.LC_ALL, '')
\end{verbatim}

  This sets the locale for all categories to the user's default
  setting (typically specified in the \envvar{LANG} environment
  variable).  If the locale is not changed thereafter, using
  multithreading should not cause problems.

  \versionchanged[Added support for tuple values of the \var{locale}
                  parameter]{2.0}
\end{funcdesc}

\begin{funcdesc}{localeconv}{}
  Returns the database of the local conventions as a dictionary.
  This dictionary has the following strings as keys:

  \begin{tableiii}{l|l|p{3in}}{constant}{Key}{Category}{Meaning}
    \lineiii{LC_NUMERIC}{\code{'decimal_point'}}
            {Decimal point character.}
    \lineiii{}{\code{'grouping'}}
            {Sequence of numbers specifying which relative positions
             the \code{'thousands_sep'} is expected.  If the sequence is
             terminated with \constant{CHAR_MAX}, no further grouping
             is performed. If the sequence terminates with a \code{0}, 
             the last group size is repeatedly used.}
    \lineiii{}{\code{'thousands_sep'}}
            {Character used between groups.}\hline
    \lineiii{LC_MONETARY}{\code{'int_curr_symbol'}}
            {International currency symbol.}
    \lineiii{}{\code{'currency_symbol'}}
            {Local currency symbol.}
    \lineiii{}{\code{'mon_decimal_point'}}
            {Decimal point used for monetary values.}
    \lineiii{}{\code{'mon_thousands_sep'}}
            {Group separator used for monetary values.}
    \lineiii{}{\code{'mon_grouping'}}
            {Equivalent to \code{'grouping'}, used for monetary
             values.}
    \lineiii{}{\code{'positive_sign'}}
            {Symbol used to annotate a positive monetary value.}
    \lineiii{}{\code{'negative_sign'}}
            {Symbol used to annotate a nnegative monetary value.}
    \lineiii{}{\code{'frac_digits'}}
            {Number of fractional digits used in local formatting
             of monetary values.}
    \lineiii{}{\code{'int_frac_digits'}}
            {Number of fractional digits used in international
             formatting of monetary values.}
  \end{tableiii}

  The possible values for \code{'p_sign_posn'} and
  \code{'n_sign_posn'} are given below.

  \begin{tableii}{c|l}{code}{Value}{Explanation}
    \lineii{0}{Currency and value are surrounded by parentheses.}
    \lineii{1}{The sign should precede the value and currency symbol.}
    \lineii{2}{The sign should follow the value and currency symbol.}
    \lineii{3}{The sign should immediately precede the value.}
    \lineii{4}{The sign should immediately follow the value.}
    \lineii{\constant{LC_MAX}}{Nothing is specified in this locale.}
  \end{tableii}
\end{funcdesc}

\begin{funcdesc}{nl_langinfo}{option}

Return some locale-specific information as a string. This function is
not available on all systems, and the set of possible options might
also vary across platforms. The possible argument values are numbers,
for which symbolic constants are available in the locale module.

\end{funcdesc}

\begin{funcdesc}{getdefaultlocale}{\optional{envvars}}
  Tries to determine the default locale settings and returns
  them as a tuple of the form \code{(\var{language code},
  \var{encoding})}.

  According to \POSIX, a program which has not called
  \code{setlocale(LC_ALL, '')} runs using the portable \code{'C'}
  locale.  Calling \code{setlocale(LC_ALL, '')} lets it use the
  default locale as defined by the \envvar{LANG} variable.  Since we
  do not want to interfere with the current locale setting we thus
  emulate the behavior in the way described above.

  To maintain compatibility with other platforms, not only the
  \envvar{LANG} variable is tested, but a list of variables given as
  envvars parameter.  The first found to be defined will be
  used.  \var{envvars} defaults to the search path used in GNU gettext;
  it must always contain the variable name \samp{LANG}.  The GNU
  gettext search path contains \code{'LANGUAGE'}, \code{'LC_ALL'},
  \code{'LC_CTYPE'}, and \code{'LANG'}, in that order.

  Except for the code \code{'C'}, the language code corresponds to
  \rfc{1766}.  \var{language code} and \var{encoding} may be
  \code{None} if their values cannot be determined.
  \versionadded{2.0}
\end{funcdesc}

\begin{funcdesc}{getlocale}{\optional{category}}
  Returns the current setting for the given locale category as
  sequence containing \var{language code}, \var{encoding}.
  \var{category} may be one of the \constant{LC_*} values except
  \constant{LC_ALL}.  It defaults to \constant{LC_CTYPE}.

  Except for the code \code{'C'}, the language code corresponds to
  \rfc{1766}.  \var{language code} and \var{encoding} may be
  \code{None} if their values cannot be determined.
  \versionadded{2.0}
\end{funcdesc}

\begin{funcdesc}{getpreferredencoding}{\optional{do_setlocale}}
  Return the encoding used for text data, according to user
  preferences.  User preferences are expressed differently on
  different systems, and might not be available programmatically on
  some systems, so this function only returns a guess.

  On some systems, it is necessary to invoke \function{setlocale}
  to obtain the user preferences, so this function is not thread-safe.
  If invoking setlocale is not necessary or desired, \var{do_setlocale}
  should be set to \code{False}.

  \versionadded{2.3}
\end{funcdesc}

\begin{funcdesc}{normalize}{localename}
  Returns a normalized locale code for the given locale name.  The
  returned locale code is formatted for use with
  \function{setlocale()}.  If normalization fails, the original name
  is returned unchanged.

  If the given encoding is not known, the function defaults to
  the default encoding for the locale code just like
  \function{setlocale()}.
  \versionadded{2.0}
\end{funcdesc}

\begin{funcdesc}{resetlocale}{\optional{category}}
  Sets the locale for \var{category} to the default setting.

  The default setting is determined by calling
  \function{getdefaultlocale()}.  \var{category} defaults to
  \constant{LC_ALL}.
  \versionadded{2.0}
\end{funcdesc}

\begin{funcdesc}{strcoll}{string1, string2}
  Compares two strings according to the current
  \constant{LC_COLLATE} setting. As any other compare function,
  returns a negative, or a positive value, or \code{0}, depending on
  whether \var{string1} collates before or after \var{string2} or is
  equal to it.
\end{funcdesc}

\begin{funcdesc}{strxfrm}{string}
  Transforms a string to one that can be used for the built-in
  function \function{cmp()}\bifuncindex{cmp}, and still returns
  locale-aware results.  This function can be used when the same
  string is compared repeatedly, e.g. when collating a sequence of
  strings.
\end{funcdesc}

\begin{funcdesc}{format}{format, val\optional{, grouping}}
  Formats a number \var{val} according to the current
  \constant{LC_NUMERIC} setting.  The format follows the conventions
  of the \code{\%} operator.  For floating point values, the decimal
  point is modified if appropriate.  If \var{grouping} is true, also
  takes the grouping into account.
\end{funcdesc}

\begin{funcdesc}{str}{float}
  Formats a floating point number using the same format as the
  built-in function \code{str(\var{float})}, but takes the decimal
  point into account.
\end{funcdesc}

\begin{funcdesc}{atof}{string}
  Converts a string to a floating point number, following the
  \constant{LC_NUMERIC} settings.
\end{funcdesc}

\begin{funcdesc}{atoi}{string}
  Converts a string to an integer, following the
  \constant{LC_NUMERIC} conventions.
\end{funcdesc}

\begin{datadesc}{LC_CTYPE}
\refstmodindex{string}
  Locale category for the character type functions.  Depending on the
  settings of this category, the functions of module
  \refmodule{string} dealing with case change their behaviour.
\end{datadesc}

\begin{datadesc}{LC_COLLATE}
  Locale category for sorting strings.  The functions
  \function{strcoll()} and \function{strxfrm()} of the
  \module{locale} module are affected.
\end{datadesc}

\begin{datadesc}{LC_TIME}
  Locale category for the formatting of time.  The function
  \function{time.strftime()} follows these conventions.
\end{datadesc}

\begin{datadesc}{LC_MONETARY}
  Locale category for formatting of monetary values.  The available
  options are available from the \function{localeconv()} function.
\end{datadesc}

\begin{datadesc}{LC_MESSAGES}
  Locale category for message display. Python currently does not
  support application specific locale-aware messages.  Messages
  displayed by the operating system, like those returned by
  \function{os.strerror()} might be affected by this category.
\end{datadesc}

\begin{datadesc}{LC_NUMERIC}
  Locale category for formatting numbers.  The functions
  \function{format()}, \function{atoi()}, \function{atof()} and
  \function{str()} of the \module{locale} module are affected by that
  category.  All other numeric formatting operations are not
  affected.
\end{datadesc}

\begin{datadesc}{LC_ALL}
  Combination of all locale settings.  If this flag is used when the
  locale is changed, setting the locale for all categories is
  attempted. If that fails for any category, no category is changed at
  all.  When the locale is retrieved using this flag, a string
  indicating the setting for all categories is returned. This string
  can be later used to restore the settings.
\end{datadesc}

\begin{datadesc}{CHAR_MAX}
  This is a symbolic constant used for different values returned by
  \function{localeconv()}.
\end{datadesc}

The \function{nl_langinfo} function accepts one of the following keys.
Most descriptions are taken from the corresponding description in the
GNU C library.

\begin{datadesc}{CODESET}
Return a string with the name of the character encoding used in the
selected locale.
\end{datadesc}

\begin{datadesc}{D_T_FMT}
Return a string that can be used as a format string for strftime(3) to
represent time and date in a locale-specific way.
\end{datadesc}

\begin{datadesc}{D_FMT}
Return a string that can be used as a format string for strftime(3) to
represent a date in a locale-specific way.
\end{datadesc}

\begin{datadesc}{T_FMT}
Return a string that can be used as a format string for strftime(3) to
represent a time in a locale-specific way.
\end{datadesc}

\begin{datadesc}{T_FMT_AMPM}
The return value can be used as a format string for `strftime' to
represent time in the am/pm format.
\end{datadesc}

\begin{datadesc}{DAY_1 ... DAY_7}
Return name of the n-th day of the week. \warning{This
follows the US convention of \constant{DAY_1} being Sunday, not the
international convention (ISO 8601) that Monday is the first day of
the week.}
\end{datadesc}

\begin{datadesc}{ABDAY_1 ... ABDAY_7}
Return abbreviated name of the n-th day of the week.
\end{datadesc}

\begin{datadesc}{MON_1 ... MON_12}
Return name of the n-th month.
\end{datadesc}

\begin{datadesc}{ABMON_1 ... ABMON_12}
Return abbreviated name of the n-th month.
\end{datadesc}

\begin{datadesc}{RADIXCHAR}
Return radix character (decimal dot, decimal comma, etc.)
\end{datadesc}

\begin{datadesc}{THOUSEP}
Return separator character for thousands (groups of three digits).
\end{datadesc}

\begin{datadesc}{YESEXPR}
Return a regular expression that can be used with the regex
function to recognize a positive response to a yes/no question.
\warning{The expression is in the syntax suitable for the
\cfunction{regex()} function from the C library, which might differ
from the syntax used in \refmodule{re}.}
\end{datadesc}

\begin{datadesc}{NOEXPR}
Return a regular expression that can be used with the regex(3)
function to recognize a negative response to a yes/no question.
\end{datadesc}

\begin{datadesc}{CRNCYSTR}
Return the currency symbol, preceded by "-" if the symbol should
appear before the value, "+" if the symbol should appear after the
value, or "." if the symbol should replace the radix character.
\end{datadesc}

\begin{datadesc}{ERA}
The return value represents the era used in the current locale.

Most locales do not define this value.  An example of a locale which
does define this value is the Japanese one.  In Japan, the traditional
representation of dates includes the name of the era corresponding to
the then-emperor's reign.

Normally it should not be necessary to use this value directly.
Specifying the \code{E} modifier in their format strings causes the
\function{strftime} function to use this information.  The format of the
returned string is not specified, and therefore you should not assume
knowledge of it on different systems.
\end{datadesc}

\begin{datadesc}{ERA_YEAR}
The return value gives the year in the relevant era of the locale.
\end{datadesc}

\begin{datadesc}{ERA_D_T_FMT}
This return value can be used as a format string for
\function{strftime} to represent dates and times in a locale-specific
era-based way.
\end{datadesc}

\begin{datadesc}{ERA_D_FMT}
This return value can be used as a format string for
\function{strftime} to represent time in a locale-specific era-based
way.
\end{datadesc}

\begin{datadesc}{ALT_DIGITS}
The return value is a representation of up to 100 values used to
represent the values 0 to 99.
\end{datadesc}

Example:

\begin{verbatim}
>>> import locale
>>> loc = locale.setlocale(locale.LC_ALL) # get current locale
>>> locale.setlocale(locale.LC_ALL, 'de_DE') # use German locale; name might vary with platform
>>> locale.strcoll('f\xe4n', 'foo') # compare a string containing an umlaut 
>>> locale.setlocale(locale.LC_ALL, '') # use user's preferred locale
>>> locale.setlocale(locale.LC_ALL, 'C') # use default (C) locale
>>> locale.setlocale(locale.LC_ALL, loc) # restore saved locale
\end{verbatim}


\subsection{Background, details, hints, tips and caveats}

The C standard defines the locale as a program-wide property that may
be relatively expensive to change.  On top of that, some
implementation are broken in such a way that frequent locale changes
may cause core dumps.  This makes the locale somewhat painful to use
correctly.

Initially, when a program is started, the locale is the \samp{C} locale, no
matter what the user's preferred locale is.  The program must
explicitly say that it wants the user's preferred locale settings by
calling \code{setlocale(LC_ALL, '')}.

It is generally a bad idea to call \function{setlocale()} in some library
routine, since as a side effect it affects the entire program.  Saving
and restoring it is almost as bad: it is expensive and affects other
threads that happen to run before the settings have been restored.

If, when coding a module for general use, you need a locale
independent version of an operation that is affected by the locale
(such as \function{string.lower()}, or certain formats used with
\function{time.strftime()}), you will have to find a way to do it
without using the standard library routine.  Even better is convincing
yourself that using locale settings is okay.  Only as a last resort
should you document that your module is not compatible with
non-\samp{C} locale settings.

The case conversion functions in the
\refmodule{string}\refstmodindex{string} module are affected by the
locale settings.  When a call to the \function{setlocale()} function
changes the \constant{LC_CTYPE} settings, the variables
\code{string.lowercase}, \code{string.uppercase} and
\code{string.letters} are recalculated.  Note that this code that uses
these variable through `\keyword{from} ... \keyword{import} ...',
e.g.\ \code{from string import letters}, is not affected by subsequent
\function{setlocale()} calls.

The only way to perform numeric operations according to the locale
is to use the special functions defined by this module:
\function{atof()}, \function{atoi()}, \function{format()},
\function{str()}.

\subsection{For extension writers and programs that embed Python
            \label{embedding-locale}}

Extension modules should never call \function{setlocale()}, except to
find out what the current locale is.  But since the return value can
only be used portably to restore it, that is not very useful (except
perhaps to find out whether or not the locale is \samp{C}).

When Python code uses the \module{locale} module to change the locale,
this also affects the embedding application.  If the embedding
application doesn't want this to happen, it should remove the
\module{_locale} extension module (which does all the work) from the
table of built-in modules in the \file{config.c} file, and make sure
that the \module{_locale} module is not accessible as a shared library.


\subsection{Access to message catalogs \label{locale-gettext}}

The locale module exposes the C library's gettext interface on systems
that provide this interface.  It consists of the functions
\function{gettext()}, \function{dgettext()}, \function{dcgettext()},
\function{textdomain()}, \function{bindtextdomain()}, and
\function{bind_textdomain_codeset()}.  These are similar to the same
functions in the \refmodule{gettext} module, but use the C library's
binary format for message catalogs, and the C library's search
algorithms for locating message catalogs. 

Python applications should normally find no need to invoke these
functions, and should use \refmodule{gettext} instead.  A known
exception to this rule are applications that link use additional C
libraries which internally invoke \cfunction{gettext()} or
\function{dcgettext()}.  For these applications, it may be necessary to
bind the text domain, so that the libraries can properly locate their
message catalogs.


\chapter{Optional Operating System Services}
\label{someos}

The modules described in this chapter provide interfaces to operating
system features that are available on selected operating systems only.
The interfaces are generally modeled after the \UNIX{} or \C{}
interfaces but they are available on some other systems as well
(e.g. Windows or NT).  Here's an overview:

\localmoduletable
		% Optional Operating System Services
\section{\module{signal} ---
         Set handlers for asynchronous events}

\declaremodule{builtin}{signal}
\modulesynopsis{Set handlers for asynchronous events.}


This module provides mechanisms to use signal handlers in Python.
Some general rules for working with signals and their handlers:

\begin{itemize}

\item
A handler for a particular signal, once set, remains installed until
it is explicitly reset (Python emulates the BSD style interface
regardless of the underlying implementation), with the exception of
the handler for \constant{SIGCHLD}, which follows the underlying
implementation.

\item
There is no way to ``block'' signals temporarily from critical
sections (since this is not supported by all \UNIX{} flavors).

\item
Although Python signal handlers are called asynchronously as far as
the Python user is concerned, they can only occur between the
``atomic'' instructions of the Python interpreter.  This means that
signals arriving during long calculations implemented purely in C
(such as regular expression matches on large bodies of text) may be
delayed for an arbitrary amount of time.

\item
When a signal arrives during an I/O operation, it is possible that the
I/O operation raises an exception after the signal handler returns.
This is dependent on the underlying \UNIX{} system's semantics regarding
interrupted system calls.

\item
Because the \C{} signal handler always returns, it makes little sense to
catch synchronous errors like \constant{SIGFPE} or \constant{SIGSEGV}.

\item
Python installs a small number of signal handlers by default:
\constant{SIGPIPE} is ignored (so write errors on pipes and sockets can be
reported as ordinary Python exceptions) and \constant{SIGINT} is translated
into a \exception{KeyboardInterrupt} exception.  All of these can be
overridden.

\item
Some care must be taken if both signals and threads are used in the
same program.  The fundamental thing to remember in using signals and
threads simultaneously is:\ always perform \function{signal()} operations
in the main thread of execution.  Any thread can perform an
\function{alarm()}, \function{getsignal()}, or \function{pause()};
only the main thread can set a new signal handler, and the main thread
will be the only one to receive signals (this is enforced by the
Python \module{signal} module, even if the underlying thread
implementation supports sending signals to individual threads).  This
means that signals can't be used as a means of inter-thread
communication.  Use locks instead.

\end{itemize}

The variables defined in the \module{signal} module are:

\begin{datadesc}{SIG_DFL}
  This is one of two standard signal handling options; it will simply
  perform the default function for the signal.  For example, on most
  systems the default action for \constant{SIGQUIT} is to dump core
  and exit, while the default action for \constant{SIGCLD} is to
  simply ignore it.
\end{datadesc}

\begin{datadesc}{SIG_IGN}
  This is another standard signal handler, which will simply ignore
  the given signal.
\end{datadesc}

\begin{datadesc}{SIG*}
  All the signal numbers are defined symbolically.  For example, the
  hangup signal is defined as \constant{signal.SIGHUP}; the variable names
  are identical to the names used in C programs, as found in
  \code{<signal.h>}.
  The \UNIX{} man page for `\cfunction{signal()}' lists the existing
  signals (on some systems this is \manpage{signal}{2}, on others the
  list is in \manpage{signal}{7}).
  Note that not all systems define the same set of signal names; only
  those names defined by the system are defined by this module.
\end{datadesc}

\begin{datadesc}{NSIG}
  One more than the number of the highest signal number.
\end{datadesc}

The \module{signal} module defines the following functions:

\begin{funcdesc}{alarm}{time}
  If \var{time} is non-zero, this function requests that a
  \constant{SIGALRM} signal be sent to the process in \var{time} seconds.
  Any previously scheduled alarm is canceled (only one alarm can
  be scheduled at any time).  The returned value is then the number of
  seconds before any previously set alarm was to have been delivered.
  If \var{time} is zero, no alarm id scheduled, and any scheduled
  alarm is canceled.  The return value is the number of seconds
  remaining before a previously scheduled alarm.  If the return value
  is zero, no alarm is currently scheduled.  (See the \UNIX{} man page
  \manpage{alarm}{2}.)
  Availability: \UNIX.
\end{funcdesc}

\begin{funcdesc}{getsignal}{signalnum}
  Return the current signal handler for the signal \var{signalnum}.
  The returned value may be a callable Python object, or one of the
  special values \constant{signal.SIG_IGN}, \constant{signal.SIG_DFL} or
  \constant{None}.  Here, \constant{signal.SIG_IGN} means that the
  signal was previously ignored, \constant{signal.SIG_DFL} means that the
  default way of handling the signal was previously in use, and
  \code{None} means that the previous signal handler was not installed
  from Python.
\end{funcdesc}

\begin{funcdesc}{pause}{}
  Cause the process to sleep until a signal is received; the
  appropriate handler will then be called.  Returns nothing.  Not on
  Windows. (See the \UNIX{} man page \manpage{signal}{2}.)
\end{funcdesc}

\begin{funcdesc}{signal}{signalnum, handler}
  Set the handler for signal \var{signalnum} to the function
  \var{handler}.  \var{handler} can be a callable Python object
  taking two arguments (see below), or
  one of the special values \constant{signal.SIG_IGN} or
  \constant{signal.SIG_DFL}.  The previous signal handler will be returned
  (see the description of \function{getsignal()} above).  (See the
  \UNIX{} man page \manpage{signal}{2}.)

  When threads are enabled, this function can only be called from the
  main thread; attempting to call it from other threads will cause a
  \exception{ValueError} exception to be raised.

  The \var{handler} is called with two arguments: the signal number
  and the current stack frame (\code{None} or a frame object;
  for a description of frame objects, see the reference manual section
  on the standard type hierarchy or see the attribute descriptions in
  the \refmodule{inspect} module).
\end{funcdesc}

\subsection{Example}
\nodename{Signal Example}

Here is a minimal example program. It uses the \function{alarm()}
function to limit the time spent waiting to open a file; this is
useful if the file is for a serial device that may not be turned on,
which would normally cause the \function{os.open()} to hang
indefinitely.  The solution is to set a 5-second alarm before opening
the file; if the operation takes too long, the alarm signal will be
sent, and the handler raises an exception.

\begin{verbatim}
import signal, os

def handler(signum, frame):
    print 'Signal handler called with signal', signum
    raise IOError, "Couldn't open device!"

# Set the signal handler and a 5-second alarm
signal.signal(signal.SIGALRM, handler)
signal.alarm(5)

# This open() may hang indefinitely
fd = os.open('/dev/ttyS0', os.O_RDWR)  

signal.alarm(0)          # Disable the alarm
\end{verbatim}

\section{\module{socket} ---
         Low-level networking interface}

\declaremodule{builtin}{socket}
\modulesynopsis{Low-level networking interface.}


This module provides access to the BSD \emph{socket} interface.
It is available on \UNIX{} systems that support this interface.

For an introduction to socket programming (in C), see the following
papers: \emph{An Introductory 4.3BSD Interprocess Communication
Tutorial}, by Stuart Sechrest and \emph{An Advanced 4.3BSD Interprocess
Communication Tutorial}, by Samuel J.  Leffler et al, both in the
\UNIX{} Programmer's Manual, Supplementary Documents 1 (sections PS1:7
and PS1:8).  The \UNIX{} manual pages for the various socket-related
system calls are also a valuable source of information on the details of
socket semantics.

The Python interface is a straightforward transliteration of the
\UNIX{} system call and library interface for sockets to Python's
object-oriented style: the \function{socket()} function returns a
\dfn{socket object}\obindex{socket} whose methods implement the
various socket system calls.  Parameter types are somewhat
higher-level than in the C interface: as with \method{read()} and
\method{write()} operations on Python files, buffer allocation on
receive operations is automatic, and buffer length is implicit on send
operations.

Socket addresses are represented as a single string for the
\constant{AF_UNIX} address family and as a pair
\code{(\var{host}, \var{port})} for the \constant{AF_INET} address
family, where \var{host} is a string representing
either a hostname in Internet domain notation like
\code{'daring.cwi.nl'} or an IP address like \code{'100.50.200.5'},
and \var{port} is an integral port number.  Other address families are
currently not supported.  The address format required by a particular
socket object is automatically selected based on the address family
specified when the socket object was created.

For IP addresses, two special forms are accepted instead of a host
address: the empty string represents \constant{INADDR_ANY}, and the string
\code{'<broadcast>'} represents \constant{INADDR_BROADCAST}.

All errors raise exceptions.  The normal exceptions for invalid
argument types and out-of-memory conditions can be raised; errors
related to socket or address semantics raise the error
\exception{socket.error}.

Non-blocking mode is supported through the
\method{setblocking()} method.

The module \module{socket} exports the following constants and functions:


\begin{excdesc}{error}
This exception is raised for socket- or address-related errors.
The accompanying value is either a string telling what went wrong or a
pair \code{(\var{errno}, \var{string})}
representing an error returned by a system
call, similar to the value accompanying \exception{os.error}.
See the module \refmodule{errno}\refbimodindex{errno}, which contains
names for the error codes defined by the underlying operating system.
\end{excdesc}

\begin{datadesc}{AF_UNIX}
\dataline{AF_INET}
These constants represent the address (and protocol) families,
used for the first argument to \function{socket()}.  If the
\constant{AF_UNIX} constant is not defined then this protocol is
unsupported.
\end{datadesc}

\begin{datadesc}{SOCK_STREAM}
\dataline{SOCK_DGRAM}
\dataline{SOCK_RAW}
\dataline{SOCK_RDM}
\dataline{SOCK_SEQPACKET}
These constants represent the socket types,
used for the second argument to \function{socket()}.
(Only \constant{SOCK_STREAM} and
\constant{SOCK_DGRAM} appear to be generally useful.)
\end{datadesc}

\begin{datadesc}{SO_*}
\dataline{SOMAXCONN}
\dataline{MSG_*}
\dataline{SOL_*}
\dataline{IPPROTO_*}
\dataline{IPPORT_*}
\dataline{INADDR_*}
\dataline{IP_*}
Many constants of these forms, documented in the \UNIX{} documentation on
sockets and/or the IP protocol, are also defined in the socket module.
They are generally used in arguments to the \method{setsockopt()} and
\method{getsockopt()} methods of socket objects.  In most cases, only
those symbols that are defined in the \UNIX{} header files are defined;
for a few symbols, default values are provided.
\end{datadesc}

\begin{funcdesc}{gethostbyname}{hostname}
Translate a host name to IP address format.  The IP address is
returned as a string, e.g.,  \code{'100.50.200.5'}.  If the host name
is an IP address itself it is returned unchanged.  See
\function{gethostbyname_ex()} for a more complete interface.
\end{funcdesc}

\begin{funcdesc}{gethostbyname_ex}{hostname}
Translate a host name to IP address format, extended interface.
Return a triple \code{(hostname, aliaslist, ipaddrlist)} where
\code{hostname} is the primary host name responding to the given
\var{ip_address}, \code{aliaslist} is a (possibly empty) list of
alternative host names for the same address, and \code{ipaddrlist} is
a list of IP addresses for the same interface on the same
host (often but not always a single address).
\end{funcdesc}

\begin{funcdesc}{gethostname}{}
Return a string containing the hostname of the machine where 
the Python interpreter is currently executing.  If you want to know the
current machine's IP address, use \code{gethostbyname(gethostname())}.
Note: \function{gethostname()} doesn't always return the fully qualified
domain name; use \code{gethostbyaddr(gethostname())}
(see below).
\end{funcdesc}

\begin{funcdesc}{gethostbyaddr}{ip_address}
Return a triple \code{(\var{hostname}, \var{aliaslist},
\var{ipaddrlist})} where \var{hostname} is the primary host name
responding to the given \var{ip_address}, \var{aliaslist} is a
(possibly empty) list of alternative host names for the same address,
and \var{ipaddrlist} is a list of IP addresses for the same interface
on the same host (most likely containing only a single address).
To find the fully qualified domain name, check \var{hostname} and the
items of \var{aliaslist} for an entry containing at least one period.
\end{funcdesc}

\begin{funcdesc}{getprotobyname}{protocolname}
Translate an Internet protocol name (e.g.\ \code{'icmp'}) to a constant
suitable for passing as the (optional) third argument to the
\function{socket()} function.  This is usually only needed for sockets
opened in ``raw'' mode (\constant{SOCK_RAW}); for the normal socket
modes, the correct protocol is chosen automatically if the protocol is
omitted or zero.
\end{funcdesc}

\begin{funcdesc}{getservbyname}{servicename, protocolname}
Translate an Internet service name and protocol name to a port number
for that service.  The protocol name should be \code{'tcp'} or
\code{'udp'}.
\end{funcdesc}

\begin{funcdesc}{socket}{family, type\optional{, proto}}
Create a new socket using the given address family, socket type and
protocol number.  The address family should be \constant{AF_INET} or
\constant{AF_UNIX}.  The socket type should be \constant{SOCK_STREAM},
\constant{SOCK_DGRAM} or perhaps one of the other \samp{SOCK_} constants.
The protocol number is usually zero and may be omitted in that case.
\end{funcdesc}

\begin{funcdesc}{fromfd}{fd, family, type\optional{, proto}}
Build a socket object from an existing file descriptor (an integer as
returned by a file object's \method{fileno()} method).  Address family,
socket type and protocol number are as for the \function{socket()} function
above.  The file descriptor should refer to a socket, but this is not
checked --- subsequent operations on the object may fail if the file
descriptor is invalid.  This function is rarely needed, but can be
used to get or set socket options on a socket passed to a program as
standard input or output (e.g.\ a server started by the \UNIX{} inet
daemon).
\end{funcdesc}

\begin{funcdesc}{ntohl}{x}
Convert 32-bit integers from network to host byte order.  On machines
where the host byte order is the same as network byte order, this is a
no-op; otherwise, it performs a 4-byte swap operation.
\end{funcdesc}

\begin{funcdesc}{ntohs}{x}
Convert 16-bit integers from network to host byte order.  On machines
where the host byte order is the same as network byte order, this is a
no-op; otherwise, it performs a 2-byte swap operation.
\end{funcdesc}

\begin{funcdesc}{htonl}{x}
Convert 32-bit integers from host to network byte order.  On machines
where the host byte order is the same as network byte order, this is a
no-op; otherwise, it performs a 4-byte swap operation.
\end{funcdesc}

\begin{funcdesc}{htons}{x}
Convert 16-bit integers from host to network byte order.  On machines
where the host byte order is the same as network byte order, this is a
no-op; otherwise, it performs a 2-byte swap operation.
\end{funcdesc}

\begin{datadesc}{SocketType}
This is a Python type object that represents the socket object type.
It is the same as \code{type(socket(...))}.
\end{datadesc}

\subsection{Socket Objects}

Socket objects have the following methods.  Except for
\method{makefile()} these correspond to \UNIX{} system calls
applicable to sockets.

\begin{methoddesc}[socket]{accept}{}
Accept a connection.
The socket must be bound to an address and listening for connections.
The return value is a pair \code{(\var{conn}, \var{address})}
where \var{conn} is a \emph{new} socket object usable to send and
receive data on the connection, and \var{address} is the address bound
to the socket on the other end of the connection.
\end{methoddesc}

\begin{methoddesc}[socket]{bind}{address}
Bind the socket to \var{address}.  The socket must not already be bound.
(The format of \var{address} depends on the address family --- see above.)
\end{methoddesc}

\begin{methoddesc}[socket]{close}{}
Close the socket.  All future operations on the socket object will fail.
The remote end will receive no more data (after queued data is flushed).
Sockets are automatically closed when they are garbage-collected.
\end{methoddesc}

\begin{methoddesc}[socket]{connect}{address}
Connect to a remote socket at \var{address}.
(The format of \var{address} depends on the address family --- see
above.)
\end{methoddesc}

\begin{methoddesc}[socket]{connect_ex}{address}
Like \code{connect(\var{address})}, but return an error indicator
instead of raising an exception for errors returned by the C-level
\cfunction{connect()} call (other problems, such as ``host not found,''
can still raise exceptions).  The error indicator is \code{0} if the
operation succeeded, otherwise the value of the \cdata{errno}
variable.  This is useful, e.g., for asynchronous connects.
\end{methoddesc}

\begin{methoddesc}[socket]{fileno}{}
Return the socket's file descriptor (a small integer).  This is useful
with \function{select.select()}.
\end{methoddesc}

\begin{methoddesc}[socket]{getpeername}{}
Return the remote address to which the socket is connected.  This is
useful to find out the port number of a remote IP socket, for instance.
(The format of the address returned depends on the address family ---
see above.)  On some systems this function is not supported.
\end{methoddesc}

\begin{methoddesc}[socket]{getsockname}{}
Return the socket's own address.  This is useful to find out the port
number of an IP socket, for instance.
(The format of the address returned depends on the address family ---
see above.)
\end{methoddesc}

\begin{methoddesc}[socket]{getsockopt}{level, optname\optional{, buflen}}
Return the value of the given socket option (see the \UNIX{} man page
\manpage{getsockopt}{2}).  The needed symbolic constants
(\constant{SO_*} etc.) are defined in this module.  If \var{buflen}
is absent, an integer option is assumed and its integer value
is returned by the function.  If \var{buflen} is present, it specifies
the maximum length of the buffer used to receive the option in, and
this buffer is returned as a string.  It is up to the caller to decode
the contents of the buffer (see the optional built-in module
\refmodule{struct} for a way to decode C structures encoded as strings).
\end{methoddesc}

\begin{methoddesc}[socket]{listen}{backlog}
Listen for connections made to the socket.  The \var{backlog} argument
specifies the maximum number of queued connections and should be at
least 1; the maximum value is system-dependent (usually 5).
\end{methoddesc}

\begin{methoddesc}[socket]{makefile}{\optional{mode\optional{, bufsize}}}
Return a \dfn{file object} associated with the socket.  (File objects
were described earlier in \ref{bltin-file-objects}, ``File Objects.'')
The file object references a \cfunction{dup()}ped version of the
socket file descriptor, so the file object and socket object may be
closed or garbage-collected independently.  The optional \var{mode}
and \var{bufsize} arguments are interpreted the same way as by the
built-in \function{open()} function.
\end{methoddesc}

\begin{methoddesc}[socket]{recv}{bufsize\optional{, flags}}
Receive data from the socket.  The return value is a string representing
the data received.  The maximum amount of data to be received
at once is specified by \var{bufsize}.  See the \UNIX{} manual page
\manpage{recv}{2} for the meaning of the optional argument
\var{flags}; it defaults to zero.
\end{methoddesc}

\begin{methoddesc}[socket]{recvfrom}{bufsize\optional{, flags}}
Receive data from the socket.  The return value is a pair
\code{(\var{string}, \var{address})} where \var{string} is a string
representing the data received and \var{address} is the address of the
socket sending the data.  The optional \var{flags} argument has the
same meaning as for \method{recv()} above.
(The format of \var{address} depends on the address family --- see above.)
\end{methoddesc}

\begin{methoddesc}[socket]{send}{string\optional{, flags}}
Send data to the socket.  The socket must be connected to a remote
socket.  The optional \var{flags} argument has the same meaning as for
\method{recv()} above.  Returns the number of bytes sent.
\end{methoddesc}

\begin{methoddesc}[socket]{sendto}{string\optional{, flags}, address}
Send data to the socket.  The socket should not be connected to a
remote socket, since the destination socket is specified by
\var{address}.  The optional \var{flags} argument has the same
meaning as for \method{recv()} above.  Return the number of bytes sent.
(The format of \var{address} depends on the address family --- see above.)
\end{methoddesc}

\begin{methoddesc}[socket]{setblocking}{flag}
Set blocking or non-blocking mode of the socket: if \var{flag} is 0,
the socket is set to non-blocking, else to blocking mode.  Initially
all sockets are in blocking mode.  In non-blocking mode, if a
\method{recv()} call doesn't find any data, or if a
\method{send()} call can't immediately dispose of the data, a
\exception{error} exception is raised; in blocking mode, the calls
block until they can proceed.
\end{methoddesc}

\begin{methoddesc}[socket]{setsockopt}{level, optname, value}
Set the value of the given socket option (see the \UNIX{} man page
\manpage{setsockopt}{2}).  The needed symbolic constants are defined in
the \module{socket} module (\code{SO_*} etc.).  The value can be an
integer or a string representing a buffer.  In the latter case it is
up to the caller to ensure that the string contains the proper bits
(see the optional built-in module
\refmodule{struct}\refbimodindex{struct} for a way to encode C
structures as strings). 
\end{methoddesc}

\begin{methoddesc}[socket]{shutdown}{how}
Shut down one or both halves of the connection.  If \var{how} is
\code{0}, further receives are disallowed.  If \var{how} is \code{1},
further sends are disallowed.  If \var{how} is \code{2}, further sends
and receives are disallowed.
\end{methoddesc}

Note that there are no methods \method{read()} or \method{write()};
use \method{recv()} and \method{send()} without \var{flags} argument
instead.

\subsection{Example}
\nodename{Socket Example}

Here are two minimal example programs using the TCP/IP protocol:\ a
server that echoes all data that it receives back (servicing only one
client), and a client using it.  Note that a server must perform the
sequence \function{socket()}, \method{bind()}, \method{listen()},
\method{accept()} (possibly repeating the \method{accept()} to service
more than one client), while a client only needs the sequence
\function{socket()}, \method{connect()}.  Also note that the server
does not \method{send()}/\method{recv()} on the 
socket it is listening on but on the new socket returned by
\method{accept()}.

\begin{verbatim}
# Echo server program
from socket import *
HOST = ''                 # Symbolic name meaning the local host
PORT = 50007              # Arbitrary non-privileged server
s = socket(AF_INET, SOCK_STREAM)
s.bind(HOST, PORT)
s.listen(1)
conn, addr = s.accept()
print 'Connected by', addr
while 1:
    data = conn.recv(1024)
    if not data: break
    conn.send(data)
conn.close()
\end{verbatim}

\begin{verbatim}
# Echo client program
from socket import *
HOST = 'daring.cwi.nl'    # The remote host
PORT = 50007              # The same port as used by the server
s = socket(AF_INET, SOCK_STREAM)
s.connect(HOST, PORT)
s.send('Hello, world')
data = s.recv(1024)
s.close()
print 'Received', `data`
\end{verbatim}

\begin{seealso}
\seemodule{SocketServer}{classes that simplify writing network servers}
\end{seealso}

\section{\module{select} ---
         Waiting for I/O completion}

\declaremodule{builtin}{select}
\modulesynopsis{Wait for I/O completion on multiple streams.}


This module provides access to the \cfunction{select()}
and \cfunction{poll()} functions
available in most operating systems.  Note that on Windows, it only
works for sockets; on other operating systems, it also works for other
file types (in particular, on \UNIX{}, it works on pipes).  It cannot
be used on regular files to determine whether a file has grown since
it was last read.

The module defines the following:

\begin{excdesc}{error}
The exception raised when an error occurs.  The accompanying value is
a pair containing the numeric error code from \cdata{errno} and the
corresponding string, as would be printed by the \C{} function
\cfunction{perror()}.
\end{excdesc}

\begin{funcdesc}{poll}{}
(Not supported by all operating systems.)  Returns a polling object, 
which supports registering and unregistering file descriptors, and
then polling them for I/O events; 
see section~\ref{poll-objects} below for the methods supported by 
polling objects.
\end{funcdesc}

\begin{funcdesc}{select}{iwtd, owtd, ewtd\optional{, timeout}}
This is a straightforward interface to the \UNIX{} \cfunction{select()}
system call.  The first three arguments are lists of `waitable
objects': either integers representing file descriptors or
objects with a parameterless method named \method{fileno()} returning
such an integer.  The three lists of waitable objects are for input,
output and `exceptional conditions', respectively.  Empty lists are
allowed, but acceptance of three empty lists is platform-dependent.
(It is known to work on \UNIX{} but not on Windows.)  The optional
\var{timeout} argument specifies a time-out as a floating point number
in seconds.  When the \var{timeout} argument is omitted the function
blocks until at least one file descriptor is ready.  A time-out value
of zero specifies a poll and never blocks.

The return value is a triple of lists of objects that are ready:
subsets of the first three arguments.  When the time-out is reached
without a file descriptor becoming ready, three empty lists are
returned.

Amongst the acceptable object types in the lists are Python file
objects (e.g. \code{sys.stdin}, or objects returned by
\function{open()} or \function{os.popen()}), socket objects
returned by \function{socket.socket()},%
\withsubitem{(in module socket)}{\ttindex{socket()}}
\withsubitem{(in module os)}{\ttindex{popen()}}.
You may also define a \dfn{wrapper} class yourself, as long as it has
an appropriate \method{fileno()} method (that really returns a file
descriptor, not just a random integer).
\strong{Note:}\index{WinSock}  File objects on Windows are not
acceptable, but sockets are.  On Windows, the underlying
\cfunction{select()} function is provided by the WinSock library, and
does not handle file desciptors that don't originate from WinSock.
\end{funcdesc}

\subsection{Polling Objects
            \label{poll-objects}}

The \cfunction{poll()} system call, supported on most Unix systems,
provides better scalability for network servers that service many,
many clients at the same time.
\cfunction{poll()} scales better because the system call only
requires listing the file descriptors of interest, while \cfunction{select()}
builds a bitmap, turns on bits for the fds of interest, and then
afterward the whole bitmap has to be linearly scanned again.
\cfunction{select()} is O(highest file descriptor), while
\cfunction{poll()} is O(number of file descriptors).

\begin{methoddesc}{register}{fd\optional{, eventmask}}
Register a file descriptor with the polling object.  Future calls to
the \method{poll()} method will then check whether the file descriptor
has any pending I/O events.  \var{fd} can be either an integer, or an
object with a \method{fileno()} method that returns an integer.  File
objects implement
\method{fileno()}, so they can also be used as the argument.

\var{eventmask} is an optional bitmask describing the type of events you
want to check for, and can be a combination of the constants
\constant{POLLIN}, \constant{POLLPRI}, and \constant{POLLOUT},
described in the table below.  If not specified, the default value
used will check for all 3 types of events.

\begin{tableii}{l|l}{constant}{Constant}{Meaning}
  \lineii{POLLIN}{There is data to read}
  \lineii{POLLPRI}{There is urgent data to read}
  \lineii{POLLOUT}{Ready for output: writing will not block}
  \lineii{POLLERR}{Error condition of some sort}
  \lineii{POLLHUP}{Hung up}
  \lineii{POLLNVAL}{Invalid request: descriptor not open}
\end{tableii}

Registering a file descriptor that's already registered is not an
error, and has the same effect as registering the descriptor exactly
once. 
 
\end{methoddesc}

\begin{methoddesc}{unregister}{fd}
Remove a file descriptor being tracked by a polling object.  Just like
the \method{register()} method, \var{fd} can be an integer or an
object with a \method{fileno()} method that returns an integer.

Attempting to remove a file descriptor that was never registered
causes a \exception{KeyError} exception to be raised.
\end{methoddesc}

\begin{methoddesc}{poll}{\optional{timeout}}
Polls the set of registered file descriptors, and returns a
possibly-empty list containing \code{(\var{fd}, \var{event})} 2-tuples
for the descriptors that have events or errors to report.
\var{fd} is the file descriptor, and \var{event} is a bitmask 
with bits set for the reported events for that descriptor
--- \constant{POLLIN} for waiting input, 
\constant{POLLOUT} to indicate that the descriptor can be written to, and
so forth.
An empty list indicates that the call timed out and no file
descriptors had any events to report.
If \var{timeout} is given, it specifies the length of time in
milliseconds which the system will wait for events before returning.
If \var{timeout} is omitted, negative, or \code{None}, the call will
block until there is an event for this poll object.
\end{methoddesc}



\section{\module{thread} ---
         Multiple threads of control}

\declaremodule{builtin}{thread}
\modulesynopsis{Create multiple threads of control within one interpreter.}


This module provides low-level primitives for working with multiple
threads (a.k.a.\ \dfn{light-weight processes} or \dfn{tasks}) --- multiple
threads of control sharing their global data space.  For
synchronization, simple locks (a.k.a.\ \dfn{mutexes} or \dfn{binary
semaphores}) are provided.
\index{light-weight processes}
\index{processes, light-weight}
\index{binary semaphores}
\index{semaphores, binary}

The module is optional.  It is supported on Windows NT and '95, SGI
IRIX, Solaris 2.x, as well as on systems that have a \POSIX{} thread
(a.k.a. ``pthread'') implementation.
\index{pthreads}
\indexii{threads}{\POSIX{}}

It defines the following constant and functions:

\begin{excdesc}{error}
Raised on thread-specific errors.
\end{excdesc}

\begin{datadesc}{LockType}
This is the type of lock objects.
\end{datadesc}

\begin{funcdesc}{start_new_thread}{function, args\optional{, kwargs}}
Start a new thread.  The thread executes the function \var{function}
with the argument list \var{args} (which must be a tuple).  The
optional \var{kwargs} argument specifies a dictionary of keyword
arguments.  When the
function returns, the thread silently exits.  When the function
terminates with an unhandled exception, a stack trace is printed and
then the thread exits (but other threads continue to run).
\end{funcdesc}

\begin{funcdesc}{exit}{}
Raise the \exception{SystemExit} exception.  When not caught, this
will cause the thread to exit silently.
\end{funcdesc}

\begin{funcdesc}{exit_thread}{}
\deprecated{1.5.2}{Use \function{exit()}.}
This is an obsolete synonym for \function{exit()}.
\end{funcdesc}

%\begin{funcdesc}{exit_prog}{status}
%Exit all threads and report the value of the integer argument
%\var{status} as the exit status of the entire program.
%\strong{Caveat:} code in pending \keyword{finally} clauses, in this thread
%or in other threads, is not executed.
%\end{funcdesc}

\begin{funcdesc}{allocate_lock}{}
Return a new lock object.  Methods of locks are described below.  The
lock is initially unlocked.
\end{funcdesc}

\begin{funcdesc}{get_ident}{}
Return the `thread identifier' of the current thread.  This is a
nonzero integer.  Its value has no direct meaning; it is intended as a
magic cookie to be used e.g. to index a dictionary of thread-specific
data.  Thread identifiers may be recycled when a thread exits and
another thread is created.
\end{funcdesc}


Lock objects have the following methods:

\begin{methoddesc}[lock]{acquire}{\optional{waitflag}}
Without the optional argument, this method acquires the lock
unconditionally, if necessary waiting until it is released by another
thread (only one thread at a time can acquire a lock --- that's their
reason for existence), and returns \code{None}.  If the integer
\var{waitflag} argument is present, the action depends on its
value: if it is zero, the lock is only acquired if it can be acquired
immediately without waiting, while if it is nonzero, the lock is
acquired unconditionally as before.  If an argument is present, the
return value is \code{1} if the lock is acquired successfully,
\code{0} if not.
\end{methoddesc}

\begin{methoddesc}[lock]{release}{}
Releases the lock.  The lock must have been acquired earlier, but not
necessarily by the same thread.
\end{methoddesc}

\begin{methoddesc}[lock]{locked}{}
Return the status of the lock:\ \code{1} if it has been acquired by
some thread, \code{0} if not.
\end{methoddesc}

\strong{Caveats:}

\begin{itemize}
\item
Threads interact strangely with interrupts: the
\exception{KeyboardInterrupt} exception will be received by an
arbitrary thread.  (When the \refmodule{signal}\refbimodindex{signal}
module is available, interrupts always go to the main thread.)

\item
Calling \function{sys.exit()} or raising the \exception{SystemExit}
exception is equivalent to calling \function{exit()}.

\item
Not all built-in functions that may block waiting for I/O allow other
threads to run.  (The most popular ones (\function{time.sleep()},
\method{\var{file}.read()}, \function{select.select()}) work as
expected.)

\item
It is not possible to interrupt the \method{acquire()} method on a lock
--- the \exception{KeyboardInterrupt} exception will happen after the
lock has been acquired.

\item
When the main thread exits, it is system defined whether the other
threads survive.  On SGI IRIX using the native thread implementation,
they survive.  On most other systems, they are killed without
executing \keyword{try} ... \keyword{finally} clauses or executing
object destructors.
\indexii{threads}{IRIX}

\item
When the main thread exits, it does not do any of its usual cleanup
(except that \keyword{try} ... \keyword{finally} clauses are honored),
and the standard I/O files are not flushed.

\end{itemize}

\section{\module{threading} ---
         Higher-level threading interface}

\declaremodule{standard}{threading}
\modulesynopsis{Higher-level threading interface.}


This module constructs higher-level threading interfaces on top of the 
lower level \refmodule{thread} module.

The \refmodule[dummythreading]{dummy_threading} module is provided for
situations where \module{threading} cannot be used because
\refmodule{thread} is missing.

This module defines the following functions and objects:

\begin{funcdesc}{activeCount}{}
Return the number of currently active \class{Thread} objects.
The returned count is equal to the length of the list returned by
\function{enumerate()}.
A function that returns the number of currently active threads.
\end{funcdesc}

\begin{funcdesc}{Condition}{}
A factory function that returns a new condition variable object.
A condition variable allows one or more threads to wait until they
are notified by another thread.
\end{funcdesc}

\begin{funcdesc}{currentThread}{}
Return the current \class{Thread} object, corresponding to the
caller's thread of control.  If the caller's thread of control was not
created through the
\module{threading} module, a dummy thread object with limited functionality
is returned.
\end{funcdesc}

\begin{funcdesc}{enumerate}{}
Return a list of all currently active \class{Thread} objects.
The list includes daemonic threads, dummy thread objects created
by \function{currentThread()}, and the main thread.  It excludes terminated
threads and threads that have not yet been started.
\end{funcdesc}

\begin{funcdesc}{Event}{}
A factory function that returns a new event object.  An event manages
a flag that can be set to true with the \method{set()} method and
reset to false with the \method{clear()} method.  The \method{wait()}
method blocks until the flag is true.
\end{funcdesc}

\begin{classdesc*}{local}{}
A class that represents thread-local data.  Thread-local data are data
whose values are thread specific.  To manage thread-local data, just
create an instance of \class{local} (or a subclass) and store
attributes on it:

\begin{verbatim}
mydata = threading.local()
mydata.x = 1
\end{verbatim}

The instance's values will be different for separate threads.

For more details and extensive examples, see the documentation string
of the \module{_threading_local} module.

\versionadded{2.4}
\end{classdesc*}

\begin{funcdesc}{Lock}{}
A factory function that returns a new primitive lock object.  Once
a thread has acquired it, subsequent attempts to acquire it block,
until it is released; any thread may release it.
\end{funcdesc}

\begin{funcdesc}{RLock}{}
A factory function that returns a new reentrant lock object.
A reentrant lock must be released by the thread that acquired it.
Once a thread has acquired a reentrant lock, the same thread may
acquire it again without blocking; the thread must release it once
for each time it has acquired it.
\end{funcdesc}

\begin{funcdesc}{Semaphore}{\optional{value}}
A factory function that returns a new semaphore object.  A
semaphore manages a counter representing the number of \method{release()}
calls minus the number of \method{acquire()} calls, plus an initial value.
The \method{acquire()} method blocks if necessary until it can return
without making the counter negative.  If not given, \var{value} defaults to
1. 
\end{funcdesc}

\begin{funcdesc}{BoundedSemaphore}{\optional{value}}
A factory function that returns a new bounded semaphore object.  A bounded
semaphore checks to make sure its current value doesn't exceed its initial
value.  If it does, \exception{ValueError} is raised. In most situations
semaphores are used to guard resources with limited capacity.  If the
semaphore is released too many times it's a sign of a bug.  If not given,
\var{value} defaults to 1. 
\end{funcdesc}

\begin{classdesc*}{Thread}{}
A class that represents a thread of control.  This class can be safely
subclassed in a limited fashion.
\end{classdesc*}

\begin{classdesc*}{Timer}{}
A thread that executes a function after a specified interval has passed.
\end{classdesc*}

\begin{funcdesc}{settrace}{func}
Set a trace function\index{trace function} for all threads started
from the \module{threading} module.  The \var{func} will be passed to 
\function{sys.settrace()} for each thread, before its \method{run()}
method is called.
\versionadded{2.3}
\end{funcdesc}

\begin{funcdesc}{setprofile}{func}
Set a profile function\index{profile function} for all threads started
from the \module{threading} module.  The \var{func} will be passed to 
\function{sys.setprofile()} for each thread, before its \method{run()}
method is called.
\versionadded{2.3}
\end{funcdesc}

\begin{funcdesc}{stack_size}{\optional{size}}
Return the thread stack size used when creating new threads.  The
optional \var{size} argument specifies the stack size to be used for
subsequently created threads, and must be 0 (use platform or
configured default) or a positive integer value of at least 32,768 (32kB).
If changing the thread stack size is unsupported, or the specified size
is invalid, a RuntimeWarning is issued and the stack size is unmodified.
32kB is currently the minimum supported stack size value, to guarantee
sufficient stack space for the interpreter itself.
Note that some platforms may have particular restrictions on values for
the stack size, such as requiring allocation in multiples of the system
memory page size - platform documentation should be referred to for more
information (4kB pages are common; using multiples of 4096 for the
stack size is the suggested approach in the absence of more specific
information).
Availability: Windows, systems with \POSIX{} threads.
\versionadded{2.5}
\end{funcdesc}

Detailed interfaces for the objects are documented below.  

The design of this module is loosely based on Java's threading model.
However, where Java makes locks and condition variables basic behavior
of every object, they are separate objects in Python.  Python's \class{Thread}
class supports a subset of the behavior of Java's Thread class;
currently, there are no priorities, no thread groups, and threads
cannot be destroyed, stopped, suspended, resumed, or interrupted.  The
static methods of Java's Thread class, when implemented, are mapped to
module-level functions.

All of the methods described below are executed atomically.


\subsection{Lock Objects \label{lock-objects}}

A primitive lock is a synchronization primitive that is not owned
by a particular thread when locked.  In Python, it is currently
the lowest level synchronization primitive available, implemented
directly by the \refmodule{thread} extension module.

A primitive lock is in one of two states, ``locked'' or ``unlocked''.
It is created in the unlocked state.  It has two basic methods,
\method{acquire()} and \method{release()}.  When the state is
unlocked, \method{acquire()} changes the state to locked and returns
immediately.  When the state is locked, \method{acquire()} blocks
until a call to \method{release()} in another thread changes it to
unlocked, then the \method{acquire()} call resets it to locked and
returns.  The \method{release()} method should only be called in the
locked state; it changes the state to unlocked and returns
immediately.  When more than one thread is blocked in
\method{acquire()} waiting for the state to turn to unlocked, only one
thread proceeds when a \method{release()} call resets the state to
unlocked; which one of the waiting threads proceeds is not defined,
and may vary across implementations.

All methods are executed atomically.

\begin{methoddesc}{acquire}{\optional{blocking\code{ = 1}}}
Acquire a lock, blocking or non-blocking.

When invoked without arguments, block until the lock is
unlocked, then set it to locked, and return true.  

When invoked with the \var{blocking} argument set to true, do the
same thing as when called without arguments, and return true.

When invoked with the \var{blocking} argument set to false, do not
block.  If a call without an argument would block, return false
immediately; otherwise, do the same thing as when called
without arguments, and return true.
\end{methoddesc}

\begin{methoddesc}{release}{}
Release a lock.

When the lock is locked, reset it to unlocked, and return.  If
any other threads are blocked waiting for the lock to become
unlocked, allow exactly one of them to proceed.

Do not call this method when the lock is unlocked.

There is no return value.
\end{methoddesc}


\subsection{RLock Objects \label{rlock-objects}}

A reentrant lock is a synchronization primitive that may be
acquired multiple times by the same thread.  Internally, it uses
the concepts of ``owning thread'' and ``recursion level'' in
addition to the locked/unlocked state used by primitive locks.  In
the locked state, some thread owns the lock; in the unlocked
state, no thread owns it.

To lock the lock, a thread calls its \method{acquire()} method; this
returns once the thread owns the lock.  To unlock the lock, a
thread calls its \method{release()} method.
\method{acquire()}/\method{release()} call pairs may be nested; only
the final \method{release()} (the \method{release()} of the outermost
pair) resets the lock to unlocked and allows another thread blocked in
\method{acquire()} to proceed.

\begin{methoddesc}{acquire}{\optional{blocking\code{ = 1}}}
Acquire a lock, blocking or non-blocking.

When invoked without arguments: if this thread already owns
the lock, increment the recursion level by one, and return
immediately.  Otherwise, if another thread owns the lock,
block until the lock is unlocked.  Once the lock is unlocked
(not owned by any thread), then grab ownership, set the
recursion level to one, and return.  If more than one thread
is blocked waiting until the lock is unlocked, only one at a
time will be able to grab ownership of the lock.  There is no
return value in this case.

When invoked with the \var{blocking} argument set to true, do the
same thing as when called without arguments, and return true.

When invoked with the \var{blocking} argument set to false, do not
block.  If a call without an argument would block, return false
immediately; otherwise, do the same thing as when called
without arguments, and return true.
\end{methoddesc}

\begin{methoddesc}{release}{}
Release a lock, decrementing the recursion level.  If after the
decrement it is zero, reset the lock to unlocked (not owned by any
thread), and if any other threads are blocked waiting for the lock to
become unlocked, allow exactly one of them to proceed.  If after the
decrement the recursion level is still nonzero, the lock remains
locked and owned by the calling thread.

Only call this method when the calling thread owns the lock.
Do not call this method when the lock is unlocked.

There is no return value.
\end{methoddesc}


\subsection{Condition Objects \label{condition-objects}}

A condition variable is always associated with some kind of lock;
this can be passed in or one will be created by default.  (Passing
one in is useful when several condition variables must share the
same lock.)

A condition variable has \method{acquire()} and \method{release()}
methods that call the corresponding methods of the associated lock.
It also has a \method{wait()} method, and \method{notify()} and
\method{notifyAll()} methods.  These three must only be called when
the calling thread has acquired the lock.

The \method{wait()} method releases the lock, and then blocks until it
is awakened by a \method{notify()} or \method{notifyAll()} call for
the same condition variable in another thread.  Once awakened, it
re-acquires the lock and returns.  It is also possible to specify a
timeout.

The \method{notify()} method wakes up one of the threads waiting for
the condition variable, if any are waiting.  The \method{notifyAll()}
method wakes up all threads waiting for the condition variable.

Note: the \method{notify()} and \method{notifyAll()} methods don't
release the lock; this means that the thread or threads awakened will
not return from their \method{wait()} call immediately, but only when
the thread that called \method{notify()} or \method{notifyAll()}
finally relinquishes ownership of the lock.

Tip: the typical programming style using condition variables uses the
lock to synchronize access to some shared state; threads that are
interested in a particular change of state call \method{wait()}
repeatedly until they see the desired state, while threads that modify
the state call \method{notify()} or \method{notifyAll()} when they
change the state in such a way that it could possibly be a desired
state for one of the waiters.  For example, the following code is a
generic producer-consumer situation with unlimited buffer capacity:

\begin{verbatim}
# Consume one item
cv.acquire()
while not an_item_is_available():
    cv.wait()
get_an_available_item()
cv.release()

# Produce one item
cv.acquire()
make_an_item_available()
cv.notify()
cv.release()
\end{verbatim}

To choose between \method{notify()} and \method{notifyAll()}, consider
whether one state change can be interesting for only one or several
waiting threads.  E.g. in a typical producer-consumer situation,
adding one item to the buffer only needs to wake up one consumer
thread.

\begin{classdesc}{Condition}{\optional{lock}}
If the \var{lock} argument is given and not \code{None}, it must be a
\class{Lock} or \class{RLock} object, and it is used as the underlying
lock.  Otherwise, a new \class{RLock} object is created and used as
the underlying lock.
\end{classdesc}

\begin{methoddesc}{acquire}{*args}
Acquire the underlying lock.
This method calls the corresponding method on the underlying
lock; the return value is whatever that method returns.
\end{methoddesc}

\begin{methoddesc}{release}{}
Release the underlying lock.
This method calls the corresponding method on the underlying
lock; there is no return value.
\end{methoddesc}

\begin{methoddesc}{wait}{\optional{timeout}}
Wait until notified or until a timeout occurs.
This must only be called when the calling thread has acquired the
lock.

This method releases the underlying lock, and then blocks until it is
awakened by a \method{notify()} or \method{notifyAll()} call for the
same condition variable in another thread, or until the optional
timeout occurs.  Once awakened or timed out, it re-acquires the lock
and returns.

When the \var{timeout} argument is present and not \code{None}, it
should be a floating point number specifying a timeout for the
operation in seconds (or fractions thereof).

When the underlying lock is an \class{RLock}, it is not released using
its \method{release()} method, since this may not actually unlock the
lock when it was acquired multiple times recursively.  Instead, an
internal interface of the \class{RLock} class is used, which really
unlocks it even when it has been recursively acquired several times.
Another internal interface is then used to restore the recursion level
when the lock is reacquired.
\end{methoddesc}

\begin{methoddesc}{notify}{}
Wake up a thread waiting on this condition, if any.
This must only be called when the calling thread has acquired the
lock.

This method wakes up one of the threads waiting for the condition
variable, if any are waiting; it is a no-op if no threads are waiting.

The current implementation wakes up exactly one thread, if any are
waiting.  However, it's not safe to rely on this behavior.  A future,
optimized implementation may occasionally wake up more than one
thread.

Note: the awakened thread does not actually return from its
\method{wait()} call until it can reacquire the lock.  Since
\method{notify()} does not release the lock, its caller should.
\end{methoddesc}

\begin{methoddesc}{notifyAll}{}
Wake up all threads waiting on this condition.  This method acts like
\method{notify()}, but wakes up all waiting threads instead of one.
\end{methoddesc}


\subsection{Semaphore Objects \label{semaphore-objects}}

This is one of the oldest synchronization primitives in the history of
computer science, invented by the early Dutch computer scientist
Edsger W. Dijkstra (he used \method{P()} and \method{V()} instead of
\method{acquire()} and \method{release()}).

A semaphore manages an internal counter which is decremented by each
\method{acquire()} call and incremented by each \method{release()}
call.  The counter can never go below zero; when \method{acquire()}
finds that it is zero, it blocks, waiting until some other thread
calls \method{release()}.

\begin{classdesc}{Semaphore}{\optional{value}}
The optional argument gives the initial value for the internal
counter; it defaults to \code{1}.
\end{classdesc}

\begin{methoddesc}{acquire}{\optional{blocking}}
Acquire a semaphore.

When invoked without arguments: if the internal counter is larger than
zero on entry, decrement it by one and return immediately.  If it is
zero on entry, block, waiting until some other thread has called
\method{release()} to make it larger than zero.  This is done with
proper interlocking so that if multiple \method{acquire()} calls are
blocked, \method{release()} will wake exactly one of them up.  The
implementation may pick one at random, so the order in which blocked
threads are awakened should not be relied on.  There is no return
value in this case.

When invoked with \var{blocking} set to true, do the same thing as
when called without arguments, and return true.

When invoked with \var{blocking} set to false, do not block.  If a
call without an argument would block, return false immediately;
otherwise, do the same thing as when called without arguments, and
return true.
\end{methoddesc}

\begin{methoddesc}{release}{}
Release a semaphore,
incrementing the internal counter by one.  When it was zero on
entry and another thread is waiting for it to become larger
than zero again, wake up that thread.
\end{methoddesc}


\subsubsection{\class{Semaphore} Example \label{semaphore-examples}}

Semaphores are often used to guard resources with limited capacity, for
example, a database server.  In any situation where the size of the resource
size is fixed, you should use a bounded semaphore.  Before spawning any
worker threads, your main thread would initialize the semaphore:

\begin{verbatim}
maxconnections = 5
...
pool_sema = BoundedSemaphore(value=maxconnections)
\end{verbatim}

Once spawned, worker threads call the semaphore's acquire and release
methods when they need to connect to the server:

\begin{verbatim}
pool_sema.acquire()
conn = connectdb()
... use connection ...
conn.close()
pool_sema.release()
\end{verbatim}

The use of a bounded semaphore reduces the chance that a programming error
which causes the semaphore to be released more than it's acquired will go
undetected.


\subsection{Event Objects \label{event-objects}}

This is one of the simplest mechanisms for communication between
threads: one thread signals an event and other threads wait for it.

An event object manages an internal flag that can be set to true with
the \method{set()} method and reset to false with the \method{clear()}
method.  The \method{wait()} method blocks until the flag is true.


\begin{classdesc}{Event}{}
The internal flag is initially false.
\end{classdesc}

\begin{methoddesc}{isSet}{}
Return true if and only if the internal flag is true.
\end{methoddesc}

\begin{methoddesc}{set}{}
Set the internal flag to true.
All threads waiting for it to become true are awakened.
Threads that call \method{wait()} once the flag is true will not block
at all.
\end{methoddesc}

\begin{methoddesc}{clear}{}
Reset the internal flag to false.
Subsequently, threads calling \method{wait()} will block until
\method{set()} is called to set the internal flag to true again.
\end{methoddesc}

\begin{methoddesc}{wait}{\optional{timeout}}
Block until the internal flag is true.
If the internal flag is true on entry, return immediately.  Otherwise,
block until another thread calls \method{set()} to set the flag to
true, or until the optional timeout occurs.

When the timeout argument is present and not \code{None}, it should be a
floating point number specifying a timeout for the operation in
seconds (or fractions thereof).
\end{methoddesc}


\subsection{Thread Objects \label{thread-objects}}

This class represents an activity that is run in a separate thread
of control.  There are two ways to specify the activity: by
passing a callable object to the constructor, or by overriding the
\method{run()} method in a subclass.  No other methods (except for the
constructor) should be overridden in a subclass.  In other words, 
\emph{only}  override the \method{__init__()} and \method{run()}
methods of this class.

Once a thread object is created, its activity must be started by
calling the thread's \method{start()} method.  This invokes the
\method{run()} method in a separate thread of control.

Once the thread's activity is started, the thread is considered
'alive' and 'active' (these concepts are almost, but not quite
exactly, the same; their definition is intentionally somewhat
vague).  It stops being alive and active when its \method{run()}
method terminates -- either normally, or by raising an unhandled
exception.  The \method{isAlive()} method tests whether the thread is
alive.

Other threads can call a thread's \method{join()} method.  This blocks
the calling thread until the thread whose \method{join()} method is
called is terminated.

A thread has a name.  The name can be passed to the constructor,
set with the \method{setName()} method, and retrieved with the
\method{getName()} method.

A thread can be flagged as a ``daemon thread''.  The significance
of this flag is that the entire Python program exits when only
daemon threads are left.  The initial value is inherited from the
creating thread.  The flag can be set with the \method{setDaemon()}
method and retrieved with the \method{isDaemon()} method.

There is a ``main thread'' object; this corresponds to the
initial thread of control in the Python program.  It is not a
daemon thread.

There is the possibility that ``dummy thread objects'' are
created.  These are thread objects corresponding to ``alien
threads''.  These are threads of control started outside the
threading module, such as directly from C code.  Dummy thread objects
have limited functionality; they are always considered alive,
active, and daemonic, and cannot be \method{join()}ed.  They are never 
deleted, since it is impossible to detect the termination of alien
threads.


\begin{classdesc}{Thread}{group=None, target=None, name=None,
                          args=(), kwargs=\{\}}
This constructor should always be called with keyword
arguments.  Arguments are:

\var{group} should be \code{None}; reserved for future extension when
a \class{ThreadGroup} class is implemented.

\var{target} is the callable object to be invoked by the
\method{run()} method.  Defaults to \code{None}, meaning nothing is
called.

\var{name} is the thread name.  By default, a unique name is
constructed of the form ``Thread-\var{N}'' where \var{N} is a small
decimal number.

\var{args} is the argument tuple for the target invocation.  Defaults
to \code{()}.

\var{kwargs} is a dictionary of keyword arguments for the target
invocation.  Defaults to \code{\{\}}.

If the subclass overrides the constructor, it must make sure
to invoke the base class constructor (\code{Thread.__init__()})
before doing anything else to the thread.
\end{classdesc}

\begin{methoddesc}{start}{}
Start the thread's activity.

This must be called at most once per thread object.  It
arranges for the object's \method{run()} method to be invoked in a
separate thread of control.
\end{methoddesc}

\begin{methoddesc}{run}{}
Method representing the thread's activity.

You may override this method in a subclass.  The standard
\method{run()} method invokes the callable object passed to the
object's constructor as the \var{target} argument, if any, with
sequential and keyword arguments taken from the \var{args} and
\var{kwargs} arguments, respectively.
\end{methoddesc}

\begin{methoddesc}{join}{\optional{timeout}}
Wait until the thread terminates.
This blocks the calling thread until the thread whose \method{join()}
method is called terminates -- either normally or through an
unhandled exception -- or until the optional timeout occurs.

When the \var{timeout} argument is present and not \code{None}, it
should be a floating point number specifying a timeout for the
operation in seconds (or fractions thereof). As \method{join()} always 
returns \code{None}, you must call \method{isAlive()} to decide whether 
a timeout happened.

When the \var{timeout} argument is not present or \code{None}, the
operation will block until the thread terminates.

A thread can be \method{join()}ed many times.

A thread cannot join itself because this would cause a
deadlock.

It is an error to attempt to \method{join()} a thread before it has
been started.
\end{methoddesc}

\begin{methoddesc}{getName}{}
Return the thread's name.
\end{methoddesc}

\begin{methoddesc}{setName}{name}
Set the thread's name.

The name is a string used for identification purposes only.
It has no semantics.  Multiple threads may be given the same
name.  The initial name is set by the constructor.
\end{methoddesc}

\begin{methoddesc}{isAlive}{}
Return whether the thread is alive.

Roughly, a thread is alive from the moment the \method{start()} method
returns until its \method{run()} method terminates.
\end{methoddesc}

\begin{methoddesc}{isDaemon}{}
Return the thread's daemon flag.
\end{methoddesc}

\begin{methoddesc}{setDaemon}{daemonic}
Set the thread's daemon flag to the Boolean value \var{daemonic}.
This must be called before \method{start()} is called.

The initial value is inherited from the creating thread.

The entire Python program exits when no active non-daemon
threads are left.
\end{methoddesc}


\subsection{Timer Objects \label{timer-objects}}

This class represents an action that should be run only after a
certain amount of time has passed --- a timer.  \class{Timer} is a
subclass of \class{Thread} and as such also functions as an example of
creating custom threads.

Timers are started, as with threads, by calling their \method{start()}
method.  The timer can be stopped (before its action has begun) by
calling the \method{cancel()} method.  The interval the timer will
wait before executing its action may not be exactly the same as the
interval specified by the user.

For example:
\begin{verbatim}
def hello():
    print "hello, world"

t = Timer(30.0, hello)
t.start() # after 30 seconds, "hello, world" will be printed
\end{verbatim}

\begin{classdesc}{Timer}{interval, function, args=[], kwargs=\{\}}
Create a timer that will run \var{function} with arguments \var{args} and 
keyword arguments \var{kwargs}, after \var{interval} seconds have passed.
\end{classdesc}

\begin{methoddesc}{cancel}{}
Stop the timer, and cancel the execution of the timer's action.  This
will only work if the timer is still in its waiting stage.
\end{methoddesc}

\subsection{Using locks, conditions, and semaphores in the \keyword{with}
statement \label{with-locks}}

All of the objects provided by this module that have \method{acquire()} and
\method{release()} methods can be used as context managers for a \keyword{with}
statement.  The \method{acquire()} method will be called when the block is
entered, and \method{release()} will be called when the block is exited.

Currently, \class{Lock}, \class{RLock}, \class{Condition}, \class{Semaphore},
and \class{BoundedSemaphore} objects may be used as \keyword{with}
statement context managers.  For example:

\begin{verbatim}
from __future__ import with_statement
import threading

some_rlock = threading.RLock()

with some_rlock:
    print "some_rlock is locked while this executes"
\end{verbatim}


\section{\module{Queue} ---
         A synchronized queue class.}
\declaremodule{standard}{Queue}

\modulesynopsis{A synchronized queue class.}



The \module{Queue} module implements a multi-producer, multi-consumer
FIFO queue.  It is especially useful in threads programming when
information must be exchanged safely between multiple threads.  The
\class{Queue} class in this module implements all the required locking
semantics.  It depends on the availability of thread support in
Python.

The \module{Queue} module defines the following class and exception:


\begin{classdesc}{Queue}{maxsize}
Constructor for the class.  \var{maxsize} is an integer that sets the
upperbound limit on the number of items that can be placed in the
queue.  Insertion will block once this size has been reached, until
queue items are consumed.  If \var{maxsize} is less than or equal to
zero, the queue size is infinite.
\end{classdesc}

\begin{excdesc}{Empty}
Exception raised when non-blocking get (e.g. \method{get_nowait()}) is
called on a \class{Queue} object which is empty, or for which the
emptyiness cannot be determined (i.e. because the appropriate locks
cannot be acquired).
\end{excdesc}

\subsection{Queue Objects}
\label{QueueObjects}

Class \class{Queue} implements queue objects and has the methods
described below.  This class can be derived from in order to implement
other queue organizations (e.g. stack) but the inheritable interface
is not described here.  See the source code for details.  The public
methods are:

\begin{methoddesc}{qsize}{}
Returns the approximate size of the queue.  Because of multithreading
semantics, this number is not reliable.
\end{methoddesc}

\begin{methoddesc}{empty}{}
Returns \code{1} if the queue is empty, \code{0} otherwise.  Because
of multithreading semantics, this is not reliable.
\end{methoddesc}

\begin{methoddesc}{full}{}
Returns \code{1} if the queue is full, \code{0} otherwise.  Because of
multithreading semantics, this is not reliable.
\end{methoddesc}

\begin{methoddesc}{put}{item}
Puts \var{item} into the queue.
\end{methoddesc}

\begin{methoddesc}{get}{}
Gets and returns an item from the queue, blocking if necessary until
one is available.
\end{methoddesc}

\begin{methoddesc}{get_nowait}{}
Gets and returns an item from the queue if one is immediately
available.  Raises an \exception{Empty} exception if the queue is
empty or if the queue's emptiness cannot be determined.
\end{methoddesc}

\section{\module{anydbm} ---
         Generic access to DBM-style databases}

\declaremodule{standard}{anydbm}
\modulesynopsis{Generic interface to DBM-style database modules.}


\module{anydbm} is a generic interface to variants of the DBM
database --- \refmodule{dbhash}\refstmodindex{dbhash} (requires
\refmodule{bsddb}\refbimodindex{bsddb}),
\refmodule{gdbm}\refbimodindex{gdbm}, or
\refmodule{dbm}\refbimodindex{dbm}.  If none of these modules is
installed, the slow-but-simple implementation in module
\refmodule{dumbdbm}\refstmodindex{dumbdbm} will be used.

\begin{funcdesc}{open}{filename\optional{, flag\optional{, mode}}}
Open the database file \var{filename} and return a corresponding object.

If the database file already exists, the \refmodule{whichdb} module is 
used to determine its type and the appropriate module is used; if it
does not exist, the first module listed above that can be imported is
used.

The optional \var{flag} argument can be
\code{'r'} to open an existing database for reading only,
\code{'w'} to open an existing database for reading and writing,
\code{'c'} to create the database if it doesn't exist, or
\code{'n'}, which will always create a new empty database.  If not
specified, the default value is \code{'r'}.

The optional \var{mode} argument is the \UNIX{} mode of the file, used
only when the database has to be created.  It defaults to octal
\code{0666} (and will be modified by the prevailing umask).
\end{funcdesc}

\begin{excdesc}{error}
A tuple containing the exceptions that can be raised by each of the
supported modules, with a unique exception \exception{anydbm.error} as
the first item --- the latter is used when \exception{anydbm.error} is
raised.
\end{excdesc}

The object returned by \function{open()} supports most of the same
functionality as dictionaries; keys and their corresponding values can
be stored, retrieved, and deleted, and the \method{has_key()} and
\method{keys()} methods are available.  Keys and values must always be
strings.


\begin{seealso}
  \seemodule{anydbm}{Generic interface to \code{dbm}-style databases.}
  \seemodule{dbhash}{BSD \code{db} database interface.}
  \seemodule{dbm}{Standard \UNIX{} database interface.}
  \seemodule{dumbdbm}{Portable implementation of the \code{dbm} interface.}
  \seemodule{gdbm}{GNU database interface, based on the \code{dbm} interface.}
  \seemodule{shelve}{General object persistence built on top of 
                     the Python \code{dbm} interface.}
  \seemodule{whichdb}{Utility module used to determine the type of an
                      existing database.}
\end{seealso}


\section{\module{dumbdbm} ---
         Portable DBM implementation}

\declaremodule{standard}{dumbdbm}
\modulesynopsis{Portable implementation of the simple DBM interface.}


A simple and slow database implemented entirely in Python.  This
should only be used when no other DBM-style database is available.


\begin{funcdesc}{open}{filename\optional{, flag\optional{, mode}}}
Open the database file \var{filename} and return a corresponding object.
The optional \var{flag} argument can be
\code{'r'} to open an existing database for reading only,
\code{'w'} to open an existing database for reading and writing,
\code{'c'} to create the database if it doesn't exist, or
\code{'n'}, which will always create a new empty database.  If not
specified, the default value is \code{'r'}.

The optional \var{mode} argument is the \UNIX{} mode of the file, used
only when the database has to be created.  It defaults to octal
\code{0666} (and will be modified by the prevailing umask).
\end{funcdesc}

\begin{excdesc}{error}
Raised for errors not reported as \exception{KeyError} errors.
\end{excdesc}


\begin{seealso}
  \seemodule{anydbm}{Generic interface to \code{dbm}-style databases.}
  \seemodule{whichdb}{Utility module used to determine the type of an
                      existing database.}
\end{seealso}

\section{Standard Module \module{whichdb}}
\declaremodule{standard}{whichdb}

\modulesynopsis{Guess which DBM-style module created a given database.}


The single function in this module attempts to guess which of the
several simple database modules available--dbm, gdbm, or
dbhash--should be used to open a given file.

\begin{funcdesc}{whichdb}{filename}
Returns one of the following values: \code{None} if the file can't be
opened because it's unreadable or doesn't exist; the empty string
(\code{""}) if the file's format can't be guessed; or a string
containing the required module name, such as \code{"dbm"} or
\code{"gdbm"}.
\end{funcdesc}


% XXX The module has been extended (by Jeremy) but this documentation
% hasn't been updated yet

\section{Built-in Module \module{zlib}}
\label{module-zlib}
\bimodindex{zlib}

For applications that require data compression, the functions in this
module allow compression and decompression, using the zlib library,
which is based on the GNU \program{gzip} program.  The zlib library
has its own home page at
\url{http://www.cdrom.com/pub/infozip/zlib/}.  Version 1.0.4 is the
most recent version as of December, 1997; use a later version if one
is available.

The available exception and functions in this module are:

\begin{excdesc}{error}
  Exception raised on compression and decompression errors.
\end{excdesc}


\begin{funcdesc}{adler32}{string\optional{, value}}
   Computes a Adler-32 checksum of \var{string}.  (An Adler-32
   checksum is almost as reliable as a CRC32 but can be computed much
   more quickly.)  If \var{value} is present, it is used as the
   starting value of the checksum; otherwise, a fixed default value is
   used.  This allows computing a running checksum over the
   concatenation of several input strings.  The algorithm is not
   cryptographically strong, and should not be used for
   authentication or digital signatures.
\end{funcdesc}

\begin{funcdesc}{compress}{string\optional{, level}}
Compresses the data in \var{string}, returning a string contained
compressed data.  \var{level} is an integer from \code{1} to \code{9}
controlling the level of compression; \code{1} is fastest and produces
the least compression, \code{9} is slowest and produces the most.  The
default value is \code{6}.  Raises the \exception{error}
exception if any error occurs.
\end{funcdesc}

\begin{funcdesc}{compressobj}{\optional{level}}
  Returns a compression object, to be used for compressing data streams
  that won't fit into memory at once.  \var{level} is an integer from
  \code{1} to \code{9} controlling the level of compression; \code{1} is
  fastest and produces the least compression, \code{9} is slowest and
  produces the most.  The default value is \code{6}.
\end{funcdesc}

\begin{funcdesc}{crc32}{string\optional{, value}}
  Computes a CRC (Cyclic Redundancy Check)%
  \index{Cyclic Redundancy Check}
  \index{checksum!Cyclic Redundancy Check}
  checksum of \var{string}. If
  \var{value} is present, it is used as the starting value of the
  checksum; otherwise, a fixed default value is used.  This allows
  computing a running checksum over the concatenation of several
  input strings.  The algorithm is not cryptographically strong, and
  should not be used for authentication or digital signatures.
\end{funcdesc}

\begin{funcdesc}{decompress}{string}
Decompresses the data in \var{string}, returning a string containing
the uncompressed data.  Raises the \exception{error} exception if any
error occurs.
\end{funcdesc}

\begin{funcdesc}{decompressobj}{\optional{wbits}}
Returns a compression object, to be used for decompressing data streams
that won't fit into memory at once.  The \var{wbits} parameter
controls the size of the window buffer; usually this can be left
alone.
\end{funcdesc}

Compression objects support the following methods:

\begin{methoddesc}[Compress]{compress}{string}
Compress \var{string}, returning a string containing compressed data
for at least part of the data in \var{string}.  This data should be
concatenated to the output produced by any preceding calls to the
\method{compress()} method.  Some input may be kept in internal buffers
for later processing.
\end{methoddesc}

\begin{methoddesc}[Compress]{flush}{}
All pending input is processed, and an string containing the remaining
compressed output is returned.  After calling \method{flush()}, the
\method{compress()} method cannot be called again; the only realistic
action is to delete the object.
\end{methoddesc}

Decompression objects support the following methods:

\begin{methoddesc}[Decompress]{decompress}{string}
Decompress \var{string}, returning a string containing the
uncompressed data corresponding to at least part of the data in
\var{string}.  This data should be concatenated to the output produced
by any preceding calls to the
\method{decompress()} method.  Some of the input data may be preserved
in internal buffers for later processing.
\end{methoddesc}

\begin{methoddesc}[Decompress]{flush}{}
All pending input is processed, and a string containing the remaining
uncompressed output is returned.  After calling \method{flush()}, the
\method{decompress()} method cannot be called again; the only realistic
action is to delete the object.
\end{methoddesc}

\begin{seealso}
\seemodule{gzip}{reading and writing \program{gzip}-format files}
\end{seealso}



\section{Built-in Module \sectcode{gzip}}
\label{module-gzip}
\bimodindex{gzip}

The data compression provided by the \code{zlib} module is compatible
with that used by the GNU compression program \file{gzip}.
Accordingly, the \code{gzip} module provides the \code{GzipFile} class
to read and write \file{gzip}-format files, automatically compressing
or decompressing the data so it looks like an ordinary file object.

\code{GzipFile} objects simulate most of the methods of a file
object, though it's not possible to use the \code{seek()} and
\code{tell()} methods to access the file randomly.

\setindexsubitem{(in module gzip)}
\begin{funcdesc}{open}{fileobj\optional{\, filename\optional{\, mode\, compresslevel}}}
  Returns a new \code{GzipFile} object on top of \var{fileobj}, which
  can be a regular file, a \code{StringIO} object, or any object which
  simulates a file.

  The \file{gzip} file format includes the original filename of the
  uncompressed file; when opening a \code{GzipFile} object for
  writing, it can be set by the \var{filename} argument.  The default
  value is an empty string.

  \var{mode} can be either \code{'r'} or \code{'w'} depending on
  whether the file will be read or written.  \var{compresslevel} is an
  integer from 1 to 9 controlling the level of compression; 1 is
  fastest and produces the least compression, and 9 is slowest and
  produces the most compression.  The default value of
  \var{compresslevel} is 9.

  Calling a \code{GzipFile} object's \code{close()} method does not
  close \var{fileobj}, since you might wish to append more material
  after the compressed data.  This also allows you to pass a
  \code{StringIO} object opened for writing as \var{fileobj}, and
  retrieve the resulting memory buffer using the \code{StringIO}
  object's \code{getvalue()} method.
\end{funcdesc}

\begin{seealso}
\seemodule{zlib}{the basic data compression module}
\end{seealso}



\chapter{Unix Specific Services}
\label{unix}

The modules described in this chapter provide interfaces to features
that are unique to the \UNIX{} operating system, or in some cases to
some or many variants of it.  Here's an overview:

\localmoduletable
			% UNIX Specific Services
\section{\module{posix} ---
         The most common \POSIX{} system calls}

\declaremodule{builtin}{posix}
  \platform{Unix}
\modulesynopsis{The most common \POSIX\ system calls (normally used
                via module \refmodule{os}).}


This module provides access to operating system functionality that is
standardized by the C Standard and the \POSIX{} standard (a thinly
disguised \UNIX{} interface).

\strong{Do not import this module directly.}  Instead, import the
module \refmodule{os}, which provides a \emph{portable} version of this
interface.  On \UNIX{}, the \refmodule{os} module provides a superset of
the \module{posix} interface.  On non-\UNIX{} operating systems the
\module{posix} module is not available, but a subset is always
available through the \refmodule{os} interface.  Once \refmodule{os} is
imported, there is \emph{no} performance penalty in using it instead
of \module{posix}.  In addition, \refmodule{os}\refstmodindex{os}
provides some additional functionality, such as automatically calling
\function{putenv()} when an entry in \code{os.environ} is changed.

The descriptions below are very terse; refer to the corresponding
\UNIX{} manual (or \POSIX{} documentation) entry for more information.
Arguments called \var{path} refer to a pathname given as a string.

Errors are reported as exceptions; the usual exceptions are given for
type errors, while errors reported by the system calls raise
\exception{error} (a synonym for the standard exception
\exception{OSError}), described below.


\subsection{Large File Support \label{posix-large-files}}
\sectionauthor{Steve Clift}{clift@mail.anacapa.net}
\index{large files}
\index{file!large files}


Several operating systems (including AIX, HPUX, Irix and Solaris)
provide support for files that are larger than 2 Gb from a C
programming model where \ctype{int} and \ctype{long} are 32-bit
values. This is typically accomplished by defining the relevant size
and offset types as 64-bit values. Such files are sometimes referred
to as \dfn{large files}.

Large file support is enabled in Python when the size of an
\ctype{off_t} is larger than a \ctype{long} and the \ctype{long long}
type is available and is at least as large as an \ctype{off_t}. Python
longs are then used to represent file sizes, offsets and other values
that can exceed the range of a Python int. It may be necessary to
configure and compile Python with certain compiler flags to enable
this mode. For example, it is enabled by default with recent versions
of Irix, but with Solaris 2.6 and 2.7 you need to do something like:

\begin{verbatim}
CFLAGS="`getconf LFS_CFLAGS`" OPT="-g -O2 $CFLAGS" \
        ./configure
\end{verbatim} % $ <-- bow to font-lock

On large-file-capable Linux systems, this might work:

\begin{verbatim}
CFLAGS='-D_LARGEFILE64_SOURCE -D_FILE_OFFSET_BITS=64' OPT="-g -O2 $CFLAGS" \
        ./configure
\end{verbatim} % $ <-- bow to font-lock


\subsection{Module Contents \label{posix-contents}}


Module \module{posix} defines the following data item:

\begin{datadesc}{environ}
A dictionary representing the string environment at the time the
interpreter was started. For example, \code{environ['HOME']} is the
pathname of your home directory, equivalent to
\code{getenv("HOME")} in C.

Modifying this dictionary does not affect the string environment
passed on by \function{execv()}, \function{popen()} or
\function{system()}; if you need to change the environment, pass
\code{environ} to \function{execve()} or add variable assignments and
export statements to the command string for \function{system()} or
\function{popen()}.

\note{The \refmodule{os} module provides an alternate
implementation of \code{environ} which updates the environment on
modification.  Note also that updating \code{os.environ} will render
this dictionary obsolete.  Use of the \refmodule{os} module version of
this is recommended over direct access to the \module{posix} module.}
\end{datadesc}

Additional contents of this module should only be accessed via the
\refmodule{os} module; refer to the documentation for that module for
further information.

\section{\module{os.path} ---
         Common pathname manipulations}
\declaremodule{standard}{os.path}

\modulesynopsis{Common pathname manipulations.}

This module implements some useful functions on pathnames.
\index{path!operations}

\warning{On Windows, many of these functions do not properly
support UNC pathnames.  \function{splitunc()} and \function{ismount()}
do handle them correctly.}


\begin{funcdesc}{abspath}{path}
Return a normalized absolutized version of the pathname \var{path}.
On most platforms, this is equivalent to
\code{normpath(join(os.getcwd(), \var{path}))}.
\versionadded{1.5.2}
\end{funcdesc}

\begin{funcdesc}{basename}{path}
Return the base name of pathname \var{path}.  This is the second half
of the pair returned by \code{split(\var{path})}.  Note that the
result of this function is different from the
\UNIX{} \program{basename} program; where \program{basename} for
\code{'/foo/bar/'} returns \code{'bar'}, the \function{basename()}
function returns an empty string (\code{''}).
\end{funcdesc}

\begin{funcdesc}{commonprefix}{list}
Return the longest path prefix (taken character-by-character) that is a
prefix of all paths in 
\var{list}.  If \var{list} is empty, return the empty string
(\code{''}).  Note that this may return invalid paths because it works a
character at a time.
\end{funcdesc}

\begin{funcdesc}{dirname}{path}
Return the directory name of pathname \var{path}.  This is the first
half of the pair returned by \code{split(\var{path})}.
\end{funcdesc}

\begin{funcdesc}{exists}{path}
Return \code{True} if \var{path} refers to an existing path.
Returns \code{False} for broken symbolic links.
\end{funcdesc}

\begin{funcdesc}{lexists}{path}
Return \code{True} if \var{path} refers to an existing path.
Returns \code{True} for broken symbolic links.  
Equivalent to \function{exists()} on platforms lacking
\function{os.lstat()}.
\versionadded{2.4}
\end{funcdesc}

\begin{funcdesc}{expanduser}{path}
On \UNIX, return the argument with an initial component of \samp{\~} or
\samp{\~\var{user}} replaced by that \var{user}'s home directory.
An initial \samp{\~} is replaced by the environment variable
\envvar{HOME} if it is set; otherwise the current user's home directory
is looked up in the password directory through the built-in module
\refmodule{pwd}\refbimodindex{pwd}.
An initial \samp{\~\var{user}} is looked up directly in the
password directory.

On Windows, only \samp{\~} is supported; it is replaced by the
environment variable \envvar{HOME} or by a combination of
\envvar{HOMEDRIVE} and \envvar{HOMEPATH}.

If the expansion fails or if the
path does not begin with a tilde, the path is returned unchanged.
\end{funcdesc}

\begin{funcdesc}{expandvars}{path}
Return the argument with environment variables expanded.  Substrings
of the form \samp{\$\var{name}} or \samp{\$\{\var{name}\}} are
replaced by the value of environment variable \var{name}.  Malformed
variable names and references to non-existing variables are left
unchanged.
\end{funcdesc}

\begin{funcdesc}{getatime}{path}
Return the time of last access of \var{path}.  The return
value is a number giving the number of seconds since the epoch (see the 
\refmodule{time} module).  Raise \exception{os.error} if the file does
not exist or is inaccessible.
\versionadded{1.5.2}
\versionchanged[If \function{os.stat_float_times()} returns True, the result is a floating point number]{2.3}
\end{funcdesc}

\begin{funcdesc}{getmtime}{path}
Return the time of last modification of \var{path}.  The return
value is a number giving the number of seconds since the epoch (see the 
\refmodule{time} module).  Raise \exception{os.error} if the file does
not exist or is inaccessible.
\versionadded{1.5.2}
\versionchanged[If \function{os.stat_float_times()} returns True, the result is a floating point number]{2.3}
\end{funcdesc}

\begin{funcdesc}{getctime}{path}
Return the system's ctime which, on some systems (like \UNIX) is the
time of the last change, and, on others (like Windows), is the
creation time for \var{path}.  The return
value is a number giving the number of seconds since the epoch (see the 
\refmodule{time} module).  Raise \exception{os.error} if the file does
not exist or is inaccessible.
\versionadded{2.3}
\end{funcdesc}

\begin{funcdesc}{getsize}{path}
Return the size, in bytes, of \var{path}.  Raise
\exception{os.error} if the file does not exist or is inaccessible.
\versionadded{1.5.2}
\end{funcdesc}

\begin{funcdesc}{isabs}{path}
Return \code{True} if \var{path} is an absolute pathname (begins with a
slash).
\end{funcdesc}

\begin{funcdesc}{isfile}{path}
Return \code{True} if \var{path} is an existing regular file.  This follows
symbolic links, so both \function{islink()} and \function{isfile()}
can be true for the same path.
\end{funcdesc}

\begin{funcdesc}{isdir}{path}
Return \code{True} if \var{path} is an existing directory.  This follows
symbolic links, so both \function{islink()} and \function{isdir()} can
be true for the same path.
\end{funcdesc}

\begin{funcdesc}{islink}{path}
Return \code{True} if \var{path} refers to a directory entry that is a
symbolic link.  Always \code{False} if symbolic links are not supported.
\end{funcdesc}

\begin{funcdesc}{ismount}{path}
Return \code{True} if pathname \var{path} is a \dfn{mount point}: a point in
a file system where a different file system has been mounted.  The
function checks whether \var{path}'s parent, \file{\var{path}/..}, is
on a different device than \var{path}, or whether \file{\var{path}/..}
and \var{path} point to the same i-node on the same device --- this
should detect mount points for all \UNIX{} and \POSIX{} variants.
\end{funcdesc}

\begin{funcdesc}{join}{path1\optional{, path2\optional{, ...}}}
Joins one or more path components intelligently.  If any component is
an absolute path, all previous components are thrown away, and joining
continues.  The return value is the concatenation of \var{path1}, and
optionally \var{path2}, etc., with exactly one directory separator
(\code{os.sep}) inserted between components, unless \var{path2} is
empty.  Note that on Windows, since there is a current directory for
each drive, \function{os.path.join("c:", "foo")} represents a path
relative to the current directory on drive \file{C:} (\file{c:foo}), not
\file{c:\textbackslash\textbackslash foo}.
\end{funcdesc}

\begin{funcdesc}{normcase}{path}
Normalize the case of a pathname.  On \UNIX, this returns the path
unchanged; on case-insensitive filesystems, it converts the path to
lowercase.  On Windows, it also converts forward slashes to backward
slashes.
\end{funcdesc}

\begin{funcdesc}{normpath}{path}
Normalize a pathname.  This collapses redundant separators and
up-level references so that \code{A//B}, \code{A/./B} and
\code{A/foo/../B} all become \code{A/B}.  It does not normalize the
case (use \function{normcase()} for that).  On Windows, it converts
forward slashes to backward slashes. It should be understood that this may
change the meaning of the path if it contains symbolic links! 
\end{funcdesc}

\begin{funcdesc}{realpath}{path}
Return the canonical path of the specified filename, eliminating any
symbolic links encountered in the path.
Availability:  \UNIX.
\versionadded{2.2}
\end{funcdesc}

\begin{funcdesc}{samefile}{path1, path2}
Return \code{True} if both pathname arguments refer to the same file or
directory (as indicated by device number and i-node number).
Raise an exception if a \function{os.stat()} call on either pathname
fails.
Availability:  Macintosh, \UNIX.
\end{funcdesc}

\begin{funcdesc}{sameopenfile}{fp1, fp2}
Return \code{True} if the file objects \var{fp1} and \var{fp2} refer to the
same file.  The two file objects may represent different file
descriptors.
Availability:  Macintosh, \UNIX.
\end{funcdesc}

\begin{funcdesc}{samestat}{stat1, stat2}
Return \code{True} if the stat tuples \var{stat1} and \var{stat2} refer to
the same file.  These structures may have been returned by
\function{fstat()}, \function{lstat()}, or \function{stat()}.  This
function implements the underlying comparison used by
\function{samefile()} and \function{sameopenfile()}.
Availability:  Macintosh, \UNIX.
\end{funcdesc}

\begin{funcdesc}{split}{path}
Split the pathname \var{path} into a pair, \code{(\var{head},
\var{tail})} where \var{tail} is the last pathname component and
\var{head} is everything leading up to that.  The \var{tail} part will
never contain a slash; if \var{path} ends in a slash, \var{tail} will
be empty.  If there is no slash in \var{path}, \var{head} will be
empty.  If \var{path} is empty, both \var{head} and \var{tail} are
empty.  Trailing slashes are stripped from \var{head} unless it is the
root (one or more slashes only).  In nearly all cases,
\code{join(\var{head}, \var{tail})} equals \var{path} (the only
exception being when there were multiple slashes separating \var{head}
from \var{tail}).
\end{funcdesc}

\begin{funcdesc}{splitdrive}{path}
Split the pathname \var{path} into a pair \code{(\var{drive},
\var{tail})} where \var{drive} is either a drive specification or the
empty string.  On systems which do not use drive specifications,
\var{drive} will always be the empty string.  In all cases,
\code{\var{drive} + \var{tail}} will be the same as \var{path}.
\versionadded{1.3}
\end{funcdesc}

\begin{funcdesc}{splitext}{path}
Split the pathname \var{path} into a pair \code{(\var{root}, \var{ext})} 
such that \code{\var{root} + \var{ext} == \var{path}},
and \var{ext} is empty or begins with a period and contains
at most one period.
\end{funcdesc}

\begin{funcdesc}{walk}{path, visit, arg}
Calls the function \var{visit} with arguments
\code{(\var{arg}, \var{dirname}, \var{names})} for each directory in the
directory tree rooted at \var{path} (including \var{path} itself, if it
is a directory).  The argument \var{dirname} specifies the visited
directory, the argument \var{names} lists the files in the directory
(gotten from \code{os.listdir(\var{dirname})}).
The \var{visit} function may modify \var{names} to
influence the set of directories visited below \var{dirname}, e.g. to
avoid visiting certain parts of the tree.  (The object referred to by
\var{names} must be modified in place, using \keyword{del} or slice
assignment.)

\begin{notice}
Symbolic links to directories are not treated as subdirectories, and
that \function{walk()} therefore will not visit them. To visit linked
directories you must identify them with
\code{os.path.islink(\var{file})} and
\code{os.path.isdir(\var{file})}, and invoke \function{walk()} as
necessary.
\end{notice}

\note{The newer \function{\refmodule{os}.walk()} generator supplies
      similar functionality and can be easier to use.}
\end{funcdesc}

\begin{datadesc}{supports_unicode_filenames}
True if arbitrary Unicode strings can be used as file names (within
limitations imposed by the file system), and if
\function{os.listdir()} returns Unicode strings for a Unicode
argument.
\versionadded{2.3}
\end{datadesc}

\section{\module{pwd} ---
         The password database}

\declaremodule{builtin}{pwd}
  \platform{Unix}
\modulesynopsis{The password database (\function{getpwnam()} and friends).}

This module provides access to the \UNIX{} user account and password
database.  It is available on all \UNIX{} versions.

Password database entries are reported as a tuple-like object, whose
attributes correspond to the members of the \code{passwd} structure
(Attribute field below, see \code{<pwd.h>}):

\begin{tableiii}{r|l|l}{textrm}{Index}{Attribute}{Meaning}
  \lineiii{0}{\code{pw_name}}{Login name}
  \lineiii{1}{\code{pw_passwd}}{Optional encrypted password}
  \lineiii{2}{\code{pw_uid}}{Numerical user ID}
  \lineiii{3}{\code{pw_gid}}{Numerical group ID}
  \lineiii{4}{\code{pw_gecos}}{User name or comment field}
  \lineiii{5}{\code{pw_dir}}{User home directory}
  \lineiii{6}{\code{pw_shell}}{User command interpreter}
\end{tableiii}

The uid and gid items are integers, all others are strings.
\exception{KeyError} is raised if the entry asked for cannot be found.

\note{In traditional \UNIX{} the field \code{pw_passwd} usually
contains a password encrypted with a DES derived algorithm (see module
\refmodule{crypt}\refbimodindex{crypt}).  However most modern unices 
use a so-called \emph{shadow password} system.  On those unices the
\var{pw_passwd} field only contains an asterisk (\code{'*'}) or the 
letter \character{x} where the encrypted password is stored in a file
\file{/etc/shadow} which is not world readable.  Whether the \var{pw_passwd}
field contains anything useful is system-dependent.}

It defines the following items:

\begin{funcdesc}{getpwuid}{uid}
Return the password database entry for the given numeric user ID.
\end{funcdesc}

\begin{funcdesc}{getpwnam}{name}
Return the password database entry for the given user name.
\end{funcdesc}

\begin{funcdesc}{getpwall}{}
Return a list of all available password database entries, in arbitrary order.
\end{funcdesc}


\begin{seealso}
  \seemodule{grp}{An interface to the group database, similar to this.}
\end{seealso}

\section{Built-in Module \module{grp}}
\declaremodule{builtin}{grp}


\modulesynopsis{The group database (\function{getgrnam()} and friends).}

This module provides access to the \UNIX{} group database.
It is available on all \UNIX{} versions.

Group database entries are reported as 4-tuples containing the
following items from the group database (see \file{<grp.h>}), in order:
\code{gr_name},
\code{gr_passwd},
\code{gr_gid},
\code{gr_mem}.
The gid is an integer, name and password are strings, and the member
list is a list of strings.
(Note that most users are not explicitly listed as members of the
group they are in according to the password database.)
A \code{KeyError} exception is raised if the entry asked for cannot be found.

It defines the following items:

\begin{funcdesc}{getgrgid}{gid}
Return the group database entry for the given numeric group ID.
\end{funcdesc}

\begin{funcdesc}{getgrnam}{name}
Return the group database entry for the given group name.
\end{funcdesc}

\begin{funcdesc}{getgrall}{}
Return a list of all available group entries, in arbitrary order.
\end{funcdesc}

\section{Built-in Module \sectcode{crypt}}
\label{module-crypt}
\bimodindex{crypt}

This module implements an interface to the \manpage{crypt}{3} routine,
which is a one-way hash function based upon a modified DES algorithm;
see the \UNIX{} man page for further details.  Possible uses include
allowing Python scripts to accept typed passwords from the user, or
attempting to crack \UNIX{} passwords with a dictionary.
\index{crypt(3)}

\setindexsubitem{(in module crypt)}
\begin{funcdesc}{crypt}{word\, salt} 
\var{word} will usually be a user's password.  \var{salt} is a
2-character string which will be used to select one of 4096 variations
of DES\indexii{cipher}{DES}.  The characters in \var{salt} must be
either \code{.}, \code{/}, or an alphanumeric character.  Returns the
hashed password as a string, which will be composed of characters from
the same alphabet as the salt.
\end{funcdesc}

The module and documentation were written by Steve Majewski.
\index{Majewski, Steve}

\section{Built-in Module \sectcode{dbm}}
\label{module-dbm}
\bimodindex{dbm}

The \code{dbm} module provides an interface to the \UNIX{}
\code{(n)dbm} library.  Dbm objects behave like mappings
(dictionaries), except that keys and values are always strings.
Printing a dbm object doesn't print the keys and values, and the
\code{items()} and \code{values()} methods are not supported.

See also the \code{gdbm} module, which provides a similar interface
using the GNU GDBM library.
\refbimodindex{gdbm}

The module defines the following constant and functions:

\renewcommand{\indexsubitem}{(in module dbm)}
\begin{excdesc}{error}
Raised on dbm-specific errors, such as I/O errors. \code{KeyError} is
raised for general mapping errors like specifying an incorrect key.
\end{excdesc}

\begin{funcdesc}{open}{filename\, \optional{flag\, \optional{mode}}}
Open a dbm database and return a dbm object.  The \var{filename}
argument is the name of the database file (without the \file{.dir} or
\file{.pag} extensions).

The optional \var{flag} argument can be
\code{'r'} (to open an existing database for reading only --- default),
\code{'w'} (to open an existing database for reading and writing),
\code{'c'} (which creates the database if it doesn't exist), or
\code{'n'} (which always creates a new empty database).

The optional \var{mode} argument is the \UNIX{} mode of the file, used
only when the database has to be created.  It defaults to octal
\code{0666}.
\end{funcdesc}

\section{Built-in Module \module{gdbm}}
\label{module-gdbm}
\bimodindex{gdbm}

% Note that if this section appears on the same page as the first
% paragraph of the dbm module section, makeindex will produce the
% warning:
%
% ## Warning (input = lib.idx, line = 1184; output = lib.ind, line = 852):
%    -- Conflicting entries: multiple encaps for the same page under same key.
%
% This is because the \bimodindex{gdbm} and \refbimodindex{gdbm}
% entries in the .idx file are slightly different (the \bimodindex{}
% version includes "|textbf" at the end to make the defining occurance 
% bold).  There doesn't appear to be anything that can be done about
% this; it's just a little annoying.  The warning can be ignored, but
% the index produced uses the non-bold version.

This module is quite similar to the \code{dbm} module, but uses \code{gdbm}
instead to provide some additional functionality.  Please note that
the file formats created by \code{gdbm} and \code{dbm} are incompatible.
\refbimodindex{dbm}

The \code{gdbm} module provides an interface to the GNU DBM
library.  \code{gdbm} objects behave like mappings
(dictionaries), except that keys and values are always strings.
Printing a \code{gdbm} object doesn't print the keys and values, and the
\code{items()} and \code{values()} methods are not supported.

The module defines the following constant and functions:

\begin{excdesc}{error}
Raised on \code{gdbm}-specific errors, such as I/O errors. \code{KeyError} is
raised for general mapping errors like specifying an incorrect key.
\end{excdesc}

\begin{funcdesc}{open}{filename, \optional{flag, \optional{mode}}}
Open a \code{gdbm} database and return a \code{gdbm} object.  The
\var{filename} argument is the name of the database file.

The optional \var{flag} argument can be
\code{'r'} (to open an existing database for reading only --- default),
\code{'w'} (to open an existing database for reading and writing),
\code{'c'} (which creates the database if it doesn't exist), or
\code{'n'} (which always creates a new empty database).

Appending \code{f} to the flag opens the database in fast mode;
altered data will not automatically be written to the disk after every
change.  This results in faster writes to the database, but may result
in an inconsistent database if the program crashes while the database
is still open.  Use the \code{sync()} method to force any unwritten
data to be written to the disk.

The optional \var{mode} argument is the \UNIX{} mode of the file, used
only when the database has to be created.  It defaults to octal
\code{0666}.
\end{funcdesc}

In addition to the dictionary-like methods, \code{gdbm} objects have the
following methods:

\begin{funcdesc}{firstkey}{}
It's possible to loop over every key in the database using this method
and the \code{nextkey()} method.  The traversal is ordered by \code{gdbm}'s
internal hash values, and won't be sorted by the key values.  This
method returns the starting key.
\end{funcdesc}

\begin{funcdesc}{nextkey}{key}
Returns the key that follows \var{key} in the traversal.  The
following code prints every key in the database \code{db}, without having to
create a list in memory that contains them all:
\begin{verbatim}
k=db.firstkey()
while k!=None:
    print k
    k=db.nextkey(k)
\end{verbatim}
\end{funcdesc}

\begin{funcdesc}{reorganize}{}
If you have carried out a lot of deletions and would like to shrink
the space used by the \code{gdbm} file, this routine will reorganize the
database.  \code{gdbm} will not shorten the length of a database file except
by using this reorganization; otherwise, deleted file space will be
kept and reused as new (key,value) pairs are added.
\end{funcdesc}

\begin{funcdesc}{sync}{}
When the database has been opened in fast mode, this method forces any
unwritten data to be written to the disk.
\end{funcdesc}


\section{Built-in Module \sectcode{termios}}
\label{module-termios}
\bimodindex{termios}
\indexii{Posix}{I/O control}
\indexii{tty}{I/O control}

\renewcommand{\indexsubitem}{(in module termios)}

This module provides an interface to the Posix calls for tty I/O
control.  For a complete description of these calls, see the Posix or
\UNIX{} manual pages.  It is only available for those \UNIX{} versions
that support Posix \code{termios} style tty I/O control (and then
only if configured at installation time).

All functions in this module take a file descriptor \var{fd} as their
first argument.  This must be an integer file descriptor, such as
returned by \code{sys.stdin.fileno()}.

This module should be used in conjunction with the \code{TERMIOS}
module, which defines the relevant symbolic constants (see the next
section).

The module defines the following functions:

\begin{funcdesc}{tcgetattr}{fd}
Return a list containing the tty attributes for file descriptor
\var{fd}, as follows: \code{[\var{iflag}, \var{oflag}, \var{cflag},
\var{lflag}, \var{ispeed}, \var{ospeed}, \var{cc}]} where \var{cc} is
a list of the tty special characters (each a string of length 1,
except the items with indices \code{VMIN} and \code{VTIME}, which are
integers when these fields are defined).  The interpretation of the
flags and the speeds as well as the indexing in the \var{cc} array
must be done using the symbolic constants defined in the
\code{TERMIOS} module.
\end{funcdesc}

\begin{funcdesc}{tcsetattr}{fd\, when\, attributes}
Set the tty attributes for file descriptor \var{fd} from the
\var{attributes}, which is a list like the one returned by
\code{tcgetattr()}.  The \var{when} argument determines when the
attributes are changed: \code{TERMIOS.TCSANOW} to change immediately,
\code{TERMIOS.TCSADRAIN} to change after transmitting all queued
output, or \code{TERMIOS.TCSAFLUSH} to change after transmitting all
queued output and discarding all queued input.
\end{funcdesc}

\begin{funcdesc}{tcsendbreak}{fd\, duration}
Send a break on file descriptor \var{fd}.  A zero \var{duration} sends
a break for 0.25--0.5 seconds; a nonzero \var{duration} has a system
dependent meaning.
\end{funcdesc}

\begin{funcdesc}{tcdrain}{fd}
Wait until all output written to file descriptor \var{fd} has been
transmitted.
\end{funcdesc}

\begin{funcdesc}{tcflush}{fd\, queue}
Discard queued data on file descriptor \var{fd}.  The \var{queue}
selector specifies which queue: \code{TERMIOS.TCIFLUSH} for the input
queue, \code{TERMIOS.TCOFLUSH} for the output queue, or
\code{TERMIOS.TCIOFLUSH} for both queues.
\end{funcdesc}

\begin{funcdesc}{tcflow}{fd\, action}
Suspend or resume input or output on file descriptor \var{fd}.  The
\var{action} argument can be \code{TERMIOS.TCOOFF} to suspend output,
\code{TERMIOS.TCOON} to restart output, \code{TERMIOS.TCIOFF} to
suspend input, or \code{TERMIOS.TCION} to restart input.
\end{funcdesc}

\subsection{Example}
\nodename{termios Example}

Here's a function that prompts for a password with echoing turned off.
Note the technique using a separate \code{termios.tcgetattr()} call
and a \code{try \ldots{} finally} statement to ensure that the old tty
attributes are restored exactly no matter what happens:

\bcode\begin{verbatim}
def getpass(prompt = "Password: "):
    import termios, TERMIOS, sys
    fd = sys.stdin.fileno()
    old = termios.tcgetattr(fd)
    new = termios.tcgetattr(fd)
    new[3] = new[3] & \~TERMIOS.ECHO          # lflags
    try:
        termios.tcsetattr(fd, TERMIOS.TCSADRAIN, new)
        passwd = raw_input(prompt)
    finally:
        termios.tcsetattr(fd, TERMIOS.TCSADRAIN, old)
    return passwd
\end{verbatim}\ecode
%
\section{Standard Module \sectcode{TERMIOS}}
\stmodindex{TERMIOS}
\indexii{Posix}{I/O control}
\indexii{tty}{I/O control}

\renewcommand{\indexsubitem}{(in module TERMIOS)}

This module defines the symbolic constants required to use the
\code{termios} module (see the previous section).  See the Posix or
\UNIX{} manual pages (or the source) for a list of those constants.
\refbimodindex{termios}

Note: this module resides in a system-dependent subdirectory of the
Python library directory.  You may have to generate it for your
particular system using the script \file{Tools/scripts/h2py.py}.

\section{\module{fcntl} ---
         The \function{fcntl()} and \function{ioctl()} system calls}

\declaremodule{builtin}{fcntl}
  \platform{Unix}
\modulesynopsis{The \function{fcntl()} and \function{ioctl()} system calls.}
\sectionauthor{Jaap Vermeulen}{}

\indexii{UNIX@\UNIX{}}{file control}
\indexii{UNIX@\UNIX{}}{I/O control}

This module performs file control and I/O control on file descriptors.
It is an interface to the \cfunction{fcntl()} and \cfunction{ioctl()}
\UNIX{} routines.

All functions in this module take a file descriptor \var{fd} as their
first argument.  This can be an integer file descriptor, such as
returned by \code{sys.stdin.fileno()}, or a file object, such as
\code{sys.stdin} itself.

The module defines the following functions:


\begin{funcdesc}{fcntl}{fd, op\optional{, arg}}
  Perform the requested operation on file descriptor \var{fd}.
  The operation is defined by \var{op} and is operating system
  dependent.  These codes are also found in the \module{fcntl}
  module. The argument \var{arg} is optional, and defaults to the
  integer value \code{0}.  When present, it can either be an integer
  value, or a string.  With the argument missing or an integer value,
  the return value of this function is the integer return value of the
  C \cfunction{fcntl()} call.  When the argument is a string it
  represents a binary structure, e.g.\ created by
  \function{struct.pack()}. The binary data is copied to a buffer
  whose address is passed to the C \cfunction{fcntl()} call.  The
  return value after a successful call is the contents of the buffer,
  converted to a string object.  The length of the returned string
  will be the same as the length of the \var{arg} argument.  This is
  limited to 1024 bytes.  If the information returned in the buffer by
  the operating system is larger than 1024 bytes, this is most likely
  to result in a segmentation violation or a more subtle data
  corruption.

  If the \cfunction{fcntl()} fails, an \exception{IOError} is
  raised.
\end{funcdesc}

\begin{funcdesc}{ioctl}{fd, op, arg}
  This function is identical to the \function{fcntl()} function, except
  that the operations are typically defined in the library module
  \module{IOCTL}.
\end{funcdesc}

\begin{funcdesc}{flock}{fd, op}
Perform the lock operation \var{op} on file descriptor \var{fd}.
See the \UNIX{} manual \manpage{flock}{3} for details.  (On some
systems, this function is emulated using \cfunction{fcntl()}.)
\end{funcdesc}

\begin{funcdesc}{lockf}{fd, operation,
    \optional{len, \optional{start, \optional{whence}}}}
This is essentially a wrapper around the \function{fcntl()} locking
calls.  \var{fd} is the file descriptor of the file to lock or unlock,
and \var{operation} is one of the following values:

\begin{itemize}
\item \constant{LOCK_UN} -- unlock
\item \constant{LOCK_SH} -- acquire a shared lock
\item \constant{LOCK_EX} -- acquire an exclusive lock
\end{itemize}

When \var{operation} is \constant{LOCK_SH} or \constant{LOCK_EX}, it
can also be bit-wise OR'd with \constant{LOCK_NB} to avoid blocking on
lock acquisition.  If \constant{LOCK_NB} is used and the lock cannot
be acquired, an \exception{IOError} will be raised and the exception
will have an \var{errno} attribute set to \constant{EACCES} or
\constant{EAGAIN} (depending on the operating system; for portability,
check for both values).

\var{length} is the number of bytes to lock, \var{start} is the byte
offset at which the lock starts, relative to \var{whence}, and
\var{whence} is as with \function{fileobj.seek()}, specifically:

\begin{itemize}
\item \constant{0} -- relative to the start of the file
      (\constant{SEEK_SET})
\item \constant{1} -- relative to the current buffer position
      (\constant{SEEK_CUR})
\item \constant{2} -- relative to the end of the file
      (\constant{SEEK_END})
\end{itemize}

The default for \var{start} is 0, which means to start at the
beginning of the file.  The default for \var{length} is 0 which means
to lock to the end of the file.  The default for \var{whence} is also
0.
\end{funcdesc}

Examples (all on a SVR4 compliant system):

\begin{verbatim}
import struct, fcntl

file = open(...)
rv = fcntl(file, fcntl.F_SETFL, os.O_NDELAY)

lockdata = struct.pack('hhllhh', fcntl.F_WRLCK, 0, 0, 0, 0, 0)
rv = fcntl.fcntl(file, fcntl.F_SETLKW, lockdata)
\end{verbatim}

Note that in the first example the return value variable \var{rv} will
hold an integer value; in the second example it will hold a string
value.  The structure lay-out for the \var{lockdata} variable is
system dependent --- therefore using the \function{flock()} call may be
better.

% Manual text and implementation by Jaap Vermeulen
\section{Standard Module \sectcode{posixfile}}
\label{module-posixfile}
\bimodindex{posixfile}
\indexii{\POSIX{}}{file object}

\emph{Note:} This module will become obsolete in a future release.
The locking operation that it provides is done better and more
portably by the \function{fcntl.lockf()} call.%
\withsubitem{(in module fcntl)}{\ttindex{lockf()}}

This module implements some additional functionality over the built-in
file objects.  In particular, it implements file locking, control over
the file flags, and an easy interface to duplicate the file object.
The module defines a new file object, the posixfile object.  It
has all the standard file object methods and adds the methods
described below.  This module only works for certain flavors of
\UNIX{}, since it uses \function{fcntl.fcntl()} for file locking.%
\withsubitem{(in module fcntl)}{\ttindex{fcntl()}}

To instantiate a posixfile object, use the \function{open()} function
in the \module{posixfile} module.  The resulting object looks and
feels roughly the same as a standard file object.

The \module{posixfile} module defines the following constants:


\begin{datadesc}{SEEK_SET}
Offset is calculated from the start of the file.
\end{datadesc}

\begin{datadesc}{SEEK_CUR}
Offset is calculated from the current position in the file.
\end{datadesc}

\begin{datadesc}{SEEK_END}
Offset is calculated from the end of the file.
\end{datadesc}

The \module{posixfile} module defines the following functions:


\begin{funcdesc}{open}{filename\optional{, mode\optional{, bufsize}}}
 Create a new posixfile object with the given filename and mode.  The
 \var{filename}, \var{mode} and \var{bufsize} arguments are
 interpreted the same way as by the built-in \function{open()}
 function.
\end{funcdesc}

\begin{funcdesc}{fileopen}{fileobject}
 Create a new posixfile object with the given standard file object.
 The resulting object has the same filename and mode as the original
 file object.
\end{funcdesc}

The posixfile object defines the following additional methods:

\setindexsubitem{(posixfile method)}
\begin{funcdesc}{lock}{fmt, \optional{len\optional{, start\optional{, whence}}}}
 Lock the specified section of the file that the file object is
 referring to.  The format is explained
 below in a table.  The \var{len} argument specifies the length of the
 section that should be locked. The default is \code{0}. \var{start}
 specifies the starting offset of the section, where the default is
 \code{0}.  The \var{whence} argument specifies where the offset is
 relative to. It accepts one of the constants \constant{SEEK_SET},
 \constant{SEEK_CUR} or \constant{SEEK_END}.  The default is
 \constant{SEEK_SET}.  For more information about the arguments refer
 to the \manpage{fcntl}{2} manual page on your system.
\end{funcdesc}

\begin{funcdesc}{flags}{\optional{flags}}
 Set the specified flags for the file that the file object is referring
 to.  The new flags are ORed with the old flags, unless specified
 otherwise.  The format is explained below in a table.  Without
 the \var{flags} argument
 a string indicating the current flags is returned (this is
 the same as the \samp{?} modifier).  For more information about the
 flags refer to the \manpage{fcntl}{2} manual page on your system.
\end{funcdesc}

\begin{funcdesc}{dup}{}
 Duplicate the file object and the underlying file pointer and file
 descriptor.  The resulting object behaves as if it were newly
 opened.
\end{funcdesc}

\begin{funcdesc}{dup2}{fd}
 Duplicate the file object and the underlying file pointer and file
 descriptor.  The new object will have the given file descriptor.
 Otherwise the resulting object behaves as if it were newly opened.
\end{funcdesc}

\begin{funcdesc}{file}{}
 Return the standard file object that the posixfile object is based
 on.  This is sometimes necessary for functions that insist on a
 standard file object.
\end{funcdesc}

All methods raise \exception{IOError} when the request fails.

Format characters for the \method{lock()} method have the following
meaning:

\begin{tableii}{|c|l|}{samp}{Format}{Meaning}
  \lineii{u}{unlock the specified region}
  \lineii{r}{request a read lock for the specified section}
  \lineii{w}{request a write lock for the specified section}
\end{tableii}

In addition the following modifiers can be added to the format:

\begin{tableiii}{|c|l|c|}{samp}{Modifier}{Meaning}{Notes}
  \lineiii{|}{wait until the lock has been granted}{}
  \lineiii{?}{return the first lock conflicting with the requested lock, or
              \code{None} if there is no conflict.}{(1)} 
\end{tableiii}

Note:

(1) The lock returned is in the format \code{(\var{mode}, \var{len},
\var{start}, \var{whence}, \var{pid})} where \var{mode} is a character
representing the type of lock ('r' or 'w').  This modifier prevents a
request from being granted; it is for query purposes only.

Format characters for the \method{flags()} method have the following
meanings:

\begin{tableii}{|c|l|}{samp}{Format}{Meaning}
  \lineii{a}{append only flag}
  \lineii{c}{close on exec flag}
  \lineii{n}{no delay flag (also called non-blocking flag)}
  \lineii{s}{synchronization flag}
\end{tableii}

In addition the following modifiers can be added to the format:

\begin{tableiii}{|c|l|c|}{samp}{Modifier}{Meaning}{Notes}
  \lineiii{!}{turn the specified flags 'off', instead of the default 'on'}{(1)}
  \lineiii{=}{replace the flags, instead of the default 'OR' operation}{(1)}
  \lineiii{?}{return a string in which the characters represent the flags that
  are set.}{(2)}
\end{tableiii}

Note:

(1) The \code{!} and \code{=} modifiers are mutually exclusive.

(2) This string represents the flags after they may have been altered
by the same call.

Examples:

\begin{verbatim}
import posixfile

file = posixfile.open('/tmp/test', 'w')
file.lock('w|')
...
file.lock('u')
file.close()
\end{verbatim}

\section{Built-in Module \module{resource}}
\label{module-resource}

\bimodindex{resource}
This module provides basic mechanisms for measuring and controlling
system resources utilized by a program.

Symbolic constants are used to specify particular system resources and
to request usage information about either the current process or its
children.

A single exception is defined for errors:


\begin{excdesc}{error}
  The functions described below may raise this error if the underlying
  system call failures unexpectedly.
\end{excdesc}

\subsection{Resource Limits}

Resources usage can be limited using the \function{setrlimit()} function
described below. Each resource is controlled by a pair of limits: a
soft limit and a hard limit. The soft limit is the current limit, and
may be lowered or raised by a process over time. The soft limit can
never exceed the hard limit. The hard limit can be lowered to any
value greater than the soft limit, but not raised. (Only processes with
the effective UID of the super-user can raise a hard limit.)

The specific resources that can be limited are system dependent. They
are described in the \manpage{getrlimit}{2} man page.  The resources
listed below are supported when the underlying operating system
supports them; resources which cannot be checked or controlled by the
operating system are not defined in this module for those platforms.

\begin{funcdesc}{getrlimit}{resource}
  Returns a tuple \code{(\var{soft}, \var{hard})} with the current
  soft and hard limits of \var{resource}. Raises \exception{ValueError} if
  an invalid resource is specified, or \exception{error} if the
  underyling system call fails unexpectedly.
\end{funcdesc}

\begin{funcdesc}{setrlimit}{resource, limits}
  Sets new limits of consumption of \var{resource}. The \var{limits}
  argument must be a tuple \code{(\var{soft}, \var{hard})} of two
  integers describing the new limits. A value of \code{-1} can be used to
  specify the maximum possible upper limit.

  Raises \exception{ValueError} if an invalid resource is specified,
  if the new soft limit exceeds the hard limit, or if a process tries
  to raise its hard limit (unless the process has an effective UID of
  super-user).  Can also raise \exception{error} if the underyling
  system call fails.
\end{funcdesc}

These symbols define resources whose consumption can be controlled
using the \function{setrlimit()} and \function{getrlimit()} functions
described below. The values of these symbols are exactly the constants
used by \C{} programs.

The \UNIX{} man page for \manpage{getrlimit}{2} lists the available
resources.  Note that not all systems use the same symbol or same
value to denote the same resource.

\begin{datadesc}{RLIMIT_CORE}
  The maximum size (in bytes) of a core file that the current process
  can create.  This may result in the creation of a partial core file
  if a larger core would be required to contain the entire process
  image.
\end{datadesc}

\begin{datadesc}{RLIMIT_CPU}
  The maximum amount of CPU time (in seconds) that a process can
  use. If this limit is exceeded, a \constant{SIGXCPU} signal is sent to
  the process. (See the \module{signal} module documentation for
  information about how to catch this signal and do something useful,
  e.g. flush open files to disk.)
\end{datadesc}

\begin{datadesc}{RLIMIT_FSIZE}
  The maximum size of a file which the process may create.  This only
  affects the stack of the main thread in a multi-threaded process.
\end{datadesc}

\begin{datadesc}{RLIMIT_DATA}
  The maximum size (in bytes) of the process's heap.
\end{datadesc}

\begin{datadesc}{RLIMIT_STACK}
  The maximum size (in bytes) of the call stack for the current
  process.
\end{datadesc}

\begin{datadesc}{RLIMIT_RSS}
  The maximum resident set size that should be made available to the
  process.
\end{datadesc}

\begin{datadesc}{RLIMIT_NPROC}
  The maximum number of processes the current process may create.
\end{datadesc}

\begin{datadesc}{RLIMIT_NOFILE}
  The maximum number of open file descriptors for the current
  process.
\end{datadesc}

\begin{datadesc}{RLIMIT_OFILE}
  The BSD name for \constant{RLIMIT_NOFILE}.
\end{datadesc}

\begin{datadesc}{RLIMIT_MEMLOC}
  The maximm address space which may be locked in memory.
\end{datadesc}

\begin{datadesc}{RLIMIT_VMEM}
  The largest area of mapped memory which the process may occupy.
\end{datadesc}

\begin{datadesc}{RLIMIT_AS}
  The maximum area (in bytes) of address space which may be taken by
  the process.
\end{datadesc}

\subsection{Resource Usage}

These functiona are used to retrieve resource usage information:

\begin{funcdesc}{getrusage}{who}
  This function returns a large tuple that describes the resources
  consumed by either the current process or its children, as specified
  by the \var{who} parameter.  The \var{who} parameter should be
  specified using one of the \code{RUSAGE_*} constants described
  below.

  The elements of the return value each
  describe how a particular system resource has been used, e.g. amount
  of time spent running is user mode or number of times the process was
  swapped out of main memory. Some values are dependent on the clock
  tick internal, e.g. the amount of memory the process is using.

  The first two elements of the return value are floating point values
  representing the amount of time spent executing in user mode and the
  amount of time spent executing in system mode, respectively. The
  remaining values are integers. Consult the \manpage{getrusage}{2}
  man page for detailed information about these values. A brief
  summary is presented here:

\begin{tableii}{|r|l|}{code}{Offset}{Resource}
  \lineii{0}{time in user mode (float)}
  \lineii{1}{time in system mode (float)}
  \lineii{2}{maximum resident set size}
  \lineii{3}{shared memory size}
  \lineii{4}{unshared memory size}
  \lineii{5}{unshared stack size}
  \lineii{6}{page faults not requiring I/O}
  \lineii{7}{page faults requiring I/O}
  \lineii{8}{number of swap outs}
  \lineii{9}{block input operations}
  \lineii{10}{block output operations}
  \lineii{11}{messages sent}
  \lineii{12}{messages received}
  \lineii{13}{signals received}
  \lineii{14}{voluntary context switches}
  \lineii{15}{involuntary context switches}
\end{tableii}

  This function will raise a \exception{ValueError} if an invalid
  \var{who} parameter is specified. It may also raise
  \exception{error} exception in unusual circumstances.
\end{funcdesc}

\begin{funcdesc}{getpagesize}{}
  Returns the number of bytes in a system page. (This need not be the
  same as the hardware page size.) This function is useful for
  determining the number of bytes of memory a process is using. The
  third element of the tuple returned by \function{getrusage()} describes
  memory usage in pages; multiplying by page size produces number of
  bytes. 
\end{funcdesc}

The following \code{RUSAGE_*} symbols are passed to the
\function{getrusage()} function to specify which processes information
should be provided for.

\begin{datadesc}{RUSAGE_SELF}
  \constant{RUSAGE_SELF} should be used to
  request information pertaining only to the process itself.
\end{datadesc}

\begin{datadesc}{RUSAGE_CHILDREN}
  Pass to \function{getrusage()} to request resource information for
  child processes of the calling process.
\end{datadesc}

\begin{datadesc}{RUSAGE_BOTH}
  Pass to \function{getrusage()} to request resources consumed by both
  the current process and child processes.  May not be available on all
  systems.
\end{datadesc}

\section{Built-in Module \sectcode{syslog}}
\bimodindex{syslog}

This module provides an interface to the Unix \code{syslog} library
routines.  Refer to the \UNIX{} manual pages for a detailed description
of the \code{syslog} facility.

The module defines the following functions:

\begin{funcdesc}{syslog}{\optional{priority\,} message}
Send the string \var{message} to the system logger.
A trailing newline is added if necessary.
Each message is tagged with a priority composed of a \var{facility} and
a \var{level}.
The optional \var{priority} argument, which defaults to
\code{(LOG_USER | LOG_INFO)}, determines the message priority.
\end{funcdesc}

\begin{funcdesc}{openlog}{ident\, \optional{logopt\, \optional{facility}}}
Logging options other than the defaults can be set by explicitly opening
the log file with \code{openlog()} prior to calling \code{syslog()}.
The defaults are (usually) \var{ident} = \samp{syslog}, \var{logopt} = 0,
\var{facility} = \code{LOG_USER}.
The \var{ident} argument is a string which is prepended to every message.
The optional \var{logopt} argument is a bit field - see below for possible
values to combine.
The optional \var{facility} argument sets the default facility for messages
which do not have a facility explicitly encoded.
\end{funcdesc}

\begin{funcdesc}{closelog}{}
Close the log file.
\end{funcdesc}

\begin{funcdesc}{setlogmask}{maskpri}
This function set the priority mask to \var{maskpri} and returns the
previous mask value.
Calls to \code{syslog} with a priority level not set in \var{maskpri}
are ignored.
The default is to log all priorities.
The function \code{LOG_MASK(\var{pri})} calculates the mask for the
individual priority \var{pri}.
The function \code{LOG_UPTO(\var{pri})} calculates the mask for all priorities
up to and including \var{pri}.
\end{funcdesc}

The module defines the following constants:

\begin{description}

\item[Priority levels (high to low):]

\code{LOG_EMERG}, \code{LOG_ALERT}, \code{LOG_CRIT}, \code{LOG_ERR},
\code{LOG_WARNING}, \code{LOG_NOTICE}, \code{LOG_INFO}, \code{LOG_DEBUG}.

\item[Facilities:]

\code{LOG_KERN}, \code{LOG_USER}, \code{LOG_MAIL}, \code{LOG_DAEMON},
\code{LOG_AUTH}, \code{LOG_LPR}, \code{LOG_NEWS}, \code{LOG_UUCP},
\code{LOG_CRON} and \code{LOG_LOCAL0} to \code{LOG_LOCAL7}.

\item[Log options:]

\code{LOG_PID}, \code{LOG_CONS}, \code{LOG_NDELAY}, \code{LOG_NOWAIT}
and \code{LOG_PERROR} if defined in \file{syslog.h}.

\end{description}

\section{\module{stat} ---
         Interpreting \function{stat()} results}

\declaremodule{standard}{stat}
\modulesynopsis{Utilities for interpreting the results of
  \function{os.stat()}, \function{os.lstat()} and \function{os.fstat()}.}
\sectionauthor{Skip Montanaro}{skip@automatrix.com}


The \module{stat} module defines constants and functions for
interpreting the results of \function{os.stat()},
\function{os.fstat()} and \function{os.lstat()} (if they exist).  For
complete details about the \cfunction{stat()}, \cfunction{fstat()} and
\cfunction{lstat()} calls, consult the documentation for your system.

The \module{stat} module defines the following functions to test for
specific file types:


\begin{funcdesc}{S_ISDIR}{mode}
Return non-zero if the mode is from a directory.
\end{funcdesc}

\begin{funcdesc}{S_ISCHR}{mode}
Return non-zero if the mode is from a character special device file.
\end{funcdesc}

\begin{funcdesc}{S_ISBLK}{mode}
Return non-zero if the mode is from a block special device file.
\end{funcdesc}

\begin{funcdesc}{S_ISREG}{mode}
Return non-zero if the mode is from a regular file.
\end{funcdesc}

\begin{funcdesc}{S_ISFIFO}{mode}
Return non-zero if the mode is from a FIFO (named pipe).
\end{funcdesc}

\begin{funcdesc}{S_ISLNK}{mode}
Return non-zero if the mode is from a symbolic link.
\end{funcdesc}

\begin{funcdesc}{S_ISSOCK}{mode}
Return non-zero if the mode is from a socket.
\end{funcdesc}

Two additional functions are defined for more general manipulation of
the file's mode:

\begin{funcdesc}{S_IMODE}{mode}
Return the portion of the file's mode that can be set by
\function{os.chmod()}---that is, the file's permission bits, plus the
sticky bit, set-group-id, and set-user-id bits (on systems that support
them).
\end{funcdesc}

\begin{funcdesc}{S_IFMT}{mode}
Return the portion of the file's mode that describes the file type (used
by the \function{S_IS*()} functions above).
\end{funcdesc}

Normally, you would use the \function{os.path.is*()} functions for
testing the type of a file; the functions here are useful when you are
doing multiple tests of the same file and wish to avoid the overhead of
the \cfunction{stat()} system call for each test.  These are also
useful when checking for information about a file that isn't handled
by \refmodule{os.path}, like the tests for block and character
devices.

All the variables below are simply symbolic indexes into the 10-tuple
returned by \function{os.stat()}, \function{os.fstat()} or
\function{os.lstat()}.

\begin{datadesc}{ST_MODE}
Inode protection mode.
\end{datadesc}

\begin{datadesc}{ST_INO}
Inode number.
\end{datadesc}

\begin{datadesc}{ST_DEV}
Device inode resides on.
\end{datadesc}

\begin{datadesc}{ST_NLINK}
Number of links to the inode.
\end{datadesc}

\begin{datadesc}{ST_UID}
User id of the owner.
\end{datadesc}

\begin{datadesc}{ST_GID}
Group id of the owner.
\end{datadesc}

\begin{datadesc}{ST_SIZE}
Size in bytes of a plain file; amount of data waiting on some special
files.
\end{datadesc}

\begin{datadesc}{ST_ATIME}
Time of last access.
\end{datadesc}

\begin{datadesc}{ST_MTIME}
Time of last modification.
\end{datadesc}

\begin{datadesc}{ST_CTIME}
Time of last status change (see manual pages for details).
\end{datadesc}

The interpretation of ``file size'' changes according to the file
type.  For plain files this is the size of the file in bytes.  For
FIFOs and sockets under most Unixes (including Linux in particular),
the ``size'' is the number of bytes waiting to be read at the time of
the call to \function{os.stat()}, \function{os.fstat()}, or
\function{os.lstat()}; this can sometimes be useful, especially for
polling one of these special files after a non-blocking open.  The
meaning of the size field for other character and block devices varies
more, depending on the implementation of the underlying system call.

Example:

\begin{verbatim}
import os, sys
from stat import *

def walktree(dir, callback):
    '''recursively descend the directory rooted at dir,
       calling the callback function for each regular file'''

    for f in os.listdir(dir):
        pathname = '%s/%s' % (dir, f)
        mode = os.stat(pathname)[ST_MODE]
        if S_ISDIR(mode):
            # It's a directory, recurse into it
            walktree(pathname, callback)
        elif S_ISREG(mode):
            # It's a file, call the callback function
            callback(pathname)
        else:
            # Unknown file type, print a message
            print 'Skipping %s' % pathname

def visitfile(file):
    print 'visiting', file

if __name__ == '__main__':
    walktree(sys.argv[1], visitfile)
\end{verbatim}

\section{\module{popen2} ---
         Subprocesses with accessible I/O streams}

\declaremodule{standard}{popen2}
  \platform{Unix, Windows}
\modulesynopsis{Subprocesses with accessible standard I/O streams.}
\sectionauthor{Drew Csillag}{drew_csillag@geocities.com}


This module allows you to spawn processes and connect to their
input/output/error pipes and obtain their return codes under
\UNIX{} and Windows.

Note that starting with Python 2.0, this functionality is available
using functions from the \refmodule{os} module which have the same
names as the factory functions here, but the order of the return
values is more intuitive in the \refmodule{os} module variants.

The primary interface offered by this module is a trio of factory
functions.  For each of these, if \var{bufsize} is specified, 
it specifies the buffer size for the I/O pipes.  \var{mode}, if
provided, should be the string \code{'b'} or \code{'t'}; on Windows
this is needed to determine whether the file objects should be opened
in binary or text mode.  The default value for \var{mode} is
\code{'t'}.

On \UNIX, \var{cmd} may be a sequence, in which case arguments will be passed
directly to the program without shell intervention (as with
\function{os.spawnv()}). If \var{cmd} is a string it will be passed to the
shell (as with \function{os.system()}).

The only way to retrieve the return codes for the child processes is
by using the \method{poll()} or \method{wait()} methods on the
\class{Popen3} and \class{Popen4} classes; these are only available on
\UNIX.  This information is not available when using the
\function{popen2()}, \function{popen3()}, and \function{popen4()}
functions, or the equivalent functions in the \refmodule{os} module.
(Note that the tuples returned by the \refmodule{os} module's functions
are in a different order from the ones returned by the \module{popen2}
module.)

\begin{funcdesc}{popen2}{cmd\optional{, bufsize\optional{, mode}}}
Executes \var{cmd} as a sub-process.  Returns the file objects
\code{(\var{child_stdout}, \var{child_stdin})}.
\end{funcdesc}

\begin{funcdesc}{popen3}{cmd\optional{, bufsize\optional{, mode}}}
Executes \var{cmd} as a sub-process.  Returns the file objects
\code{(\var{child_stdout}, \var{child_stdin}, \var{child_stderr})}.
\end{funcdesc}

\begin{funcdesc}{popen4}{cmd\optional{, bufsize\optional{, mode}}}
Executes \var{cmd} as a sub-process.  Returns the file objects
\code{(\var{child_stdout_and_stderr}, \var{child_stdin})}.
\versionadded{2.0}
\end{funcdesc}


On \UNIX, a class defining the objects returned by the factory
functions is also available.  These are not used for the Windows
implementation, and are not available on that platform.

\begin{classdesc}{Popen3}{cmd\optional{, capturestderr\optional{, bufsize}}}
This class represents a child process.  Normally, \class{Popen3}
instances are created using the \function{popen2()} and
\function{popen3()} factory functions described above.

If not using one of the helper functions to create \class{Popen3}
objects, the parameter \var{cmd} is the shell command to execute in a
sub-process.  The \var{capturestderr} flag, if true, specifies that
the object should capture standard error output of the child process.
The default is false.  If the \var{bufsize} parameter is specified, it
specifies the size of the I/O buffers to/from the child process.
\end{classdesc}

\begin{classdesc}{Popen4}{cmd\optional{, bufsize}}
Similar to \class{Popen3}, but always captures standard error into the
same file object as standard output.  These are typically created
using \function{popen4()}.
\versionadded{2.0}
\end{classdesc}

\subsection{Popen3 and Popen4 Objects \label{popen3-objects}}

Instances of the \class{Popen3} and \class{Popen4} classes have the
following methods:

\begin{methoddesc}{poll}{}
Returns \code{-1} if child process hasn't completed yet, or its return 
code otherwise.
\end{methoddesc}

\begin{methoddesc}{wait}{}
Waits for and returns the status code of the child process.  The
status code encodes both the return code of the process and
information about whether it exited using the \cfunction{exit()}
system call or died due to a signal.  Functions to help interpret the
status code are defined in the \refmodule{os} module; see section
\ref{os-process} for the \function{W\var{*}()} family of functions.
\end{methoddesc}


The following attributes are also available: 

\begin{memberdesc}{fromchild}
A file object that provides output from the child process.  For
\class{Popen4} instances, this will provide both the standard output
and standard error streams.
\end{memberdesc}

\begin{memberdesc}{tochild}
A file object that provides input to the child process.
\end{memberdesc}

\begin{memberdesc}{childerr}
A file object that provides error output from the child process, if
\var{capturestderr} was true for the constructor, otherwise
\code{None}.  This will always be \code{None} for \class{Popen4}
instances.
\end{memberdesc}

\begin{memberdesc}{pid}
The process ID of the child process.
\end{memberdesc}


\subsection{Flow Control Issues \label{popen2-flow-control}}

Any time you are working with any form of inter-process communication,
control flow needs to be carefully thought out.  This remains the case
with the file objects provided by this module (or the \refmodule{os}
module equivalents).

% Example explanation and suggested work-arounds substantially stolen
% from Martin von L�wis:
% http://mail.python.org/pipermail/python-dev/2000-September/009460.html

When reading output from a child process that writes a lot of data to
standard error while the parent is reading from the child's standard
output, a deadlock can occur.  A similar situation can occur with other
combinations of reads and writes.  The essential factors are that more
than \constant{_PC_PIPE_BUF} bytes are being written by one process in
a blocking fashion, while the other process is reading from the other
process, also in a blocking fashion.

There are several ways to deal with this situation.

The simplest application change, in many cases, will be to follow this
model in the parent process:

\begin{verbatim}
import popen2

r, w, e = popen2.popen3('python slave.py')
e.readlines()
r.readlines()
r.close()
e.close()
w.close()
\end{verbatim}

with code like this in the child:

\begin{verbatim}
import os
import sys

# note that each of these print statements
# writes a single long string

print >>sys.stderr, 400 * 'this is a test\n'
os.close(sys.stderr.fileno())
print >>sys.stdout, 400 * 'this is another test\n'
\end{verbatim}

In particular, note that \code{sys.stderr} must be closed after
writing all data, or \method{readlines()} won't return.  Also note
that \function{os.close()} must be used, as \code{sys.stderr.close()}
won't close \code{stderr} (otherwise assigning to \code{sys.stderr}
will silently close it, so no further errors can be printed).

Applications which need to support a more general approach should
integrate I/O over pipes with their \function{select()} loops, or use
separate threads to read each of the individual files provided by
whichever \function{popen*()} function or \class{Popen*} class was
used.

\section{Standard module \sectcode{commands}}	% If implemented in Python
\label{module-commands}
\stmodindex{commands}

The \code{commands} module contains wrapper functions for \code{os.popen()} 
which take a system command as a string and return any output generated by 
the command, and optionally, the exit status.

The \code{commands} module is only usable on systems which support 
\code{popen()} (currently \UNIX{}).

The \code{commands} module defines the following functions:

\renewcommand{\indexsubitem}{(in module commands)}
\begin{funcdesc}{getstatusoutput}{cmd}
Execute the string \var{cmd} in a shell with \code{os.popen()} and return
a 2-tuple (status, output).  \var{cmd} is actually run as
\code{\{ cmd ; \} 2>\&1}, so that the returned output will contain output
or error messages. A trailing newline is stripped from the output.
The exit status for the  command can be interpreted according to the
rules for the \C{} function \code{wait()}.  
\end{funcdesc}

\begin{funcdesc}{getoutput}{cmd}
Like \code{getstatusoutput()}, except the exit status is ignored and
the return value is a string containing the command's output.  
\end{funcdesc}

\begin{funcdesc}{getstatus}{file}
Return the output of \samp{ls -ld \var{file}} as a string.  This
function uses the \code{getoutput()} function, and properly escapes
backslashes and dollar signs in the argument.
\end{funcdesc}

Example:

\begin{verbatim}
>>> import commands
>>> commands.getstatusoutput('ls /bin/ls')
(0, '/bin/ls')
>>> commands.getstatusoutput('cat /bin/junk')
(256, 'cat: /bin/junk: No such file or directory')
>>> commands.getstatusoutput('/bin/junk')
(256, 'sh: /bin/junk: not found')
>>> commands.getoutput('ls /bin/ls')
'/bin/ls'
>>> commands.getstatus('/bin/ls')
'-rwxr-xr-x  1 root        13352 Oct 14  1994 /bin/ls'
\end{verbatim}


\chapter{The Python Debugger}

\declaremodule{standard}{pdb}
\modulesynopsis{The Python debugger for interactive interpreters.}


The module \module{pdb} defines an interactive source code
debugger\index{debugging} for Python programs.  It supports setting
(conditional) breakpoints and single stepping at the source line
level, inspection of stack frames, source code listing, and evaluation
of arbitrary Python code in the context of any stack frame.  It also
supports post-mortem debugging and can be called under program
control.

The debugger is extensible --- it is actually defined as the class
\class{Pdb}\withsubitem{(class in pdb)}{\ttindex{Pdb}}.
This is currently undocumented but easily understood by reading the
source.  The extension interface uses the modules
\module{bdb}\refstmodindex{bdb} (undocumented) and
\refmodule{cmd}\refstmodindex{cmd}.

A primitive windowing version of the debugger also exists --- this is
module \module{wdb}\refstmodindex{wdb}, which requires
\module{stdwin}\refbimodindex{stdwin}.

The debugger's prompt is \samp{(Pdb) }.
Typical usage to run a program under control of the debugger is:

\begin{verbatim}
>>> import pdb
>>> import mymodule
>>> pdb.run('mymodule.test()')
> <string>(0)?()
(Pdb) continue
> <string>(1)?()
(Pdb) continue
NameError: 'spam'
> <string>(1)?()
(Pdb) 
\end{verbatim}

\file{pdb.py} can also be invoked as
a script to debug other scripts.  For example:

\begin{verbatim}
python /usr/local/lib/python1.5/pdb.py myscript.py
\end{verbatim}

Typical usage to inspect a crashed program is:

\begin{verbatim}
>>> import pdb
>>> import mymodule
>>> mymodule.test()
Traceback (innermost last):
  File "<stdin>", line 1, in ?
  File "./mymodule.py", line 4, in test
    test2()
  File "./mymodule.py", line 3, in test2
    print spam
NameError: spam
>>> pdb.pm()
> ./mymodule.py(3)test2()
-> print spam
(Pdb) 
\end{verbatim}

The module defines the following functions; each enters the debugger
in a slightly different way:

\begin{funcdesc}{run}{statement\optional{, globals\optional{, locals}}}
Execute the \var{statement} (given as a string) under debugger
control.  The debugger prompt appears before any code is executed; you
can set breakpoints and type \samp{continue}, or you can step through
the statement using \samp{step} or \samp{next} (all these commands are
explained below).  The optional \var{globals} and \var{locals}
arguments specify the environment in which the code is executed; by
default the dictionary of the module \module{__main__} is used.  (See
the explanation of the \keyword{exec} statement or the
\function{eval()} built-in function.)
\end{funcdesc}

\begin{funcdesc}{runeval}{expression\optional{, globals\optional{, locals}}}
Evaluate the \var{expression} (given as a a string) under debugger
control.  When \function{runeval()} returns, it returns the value of the
expression.  Otherwise this function is similar to
\function{run()}.
\end{funcdesc}

\begin{funcdesc}{runcall}{function\optional{, argument, ...}}
Call the \var{function} (a function or method object, not a string)
with the given arguments.  When \function{runcall()} returns, it returns
whatever the function call returned.  The debugger prompt appears as
soon as the function is entered.
\end{funcdesc}

\begin{funcdesc}{set_trace}{}
Enter the debugger at the calling stack frame.  This is useful to
hard-code a breakpoint at a given point in a program, even if the code
is not otherwise being debugged (e.g. when an assertion fails).
\end{funcdesc}

\begin{funcdesc}{post_mortem}{traceback}
Enter post-mortem debugging of the given \var{traceback} object.
\end{funcdesc}

\begin{funcdesc}{pm}{}
Enter post-mortem debugging of the traceback found in
\code{sys.last_traceback}.
\end{funcdesc}


\section{Debugger Commands \label{debugger-commands}}

The debugger recognizes the following commands.  Most commands can be
abbreviated to one or two letters; e.g. \samp{h(elp)} means that
either \samp{h} or \samp{help} can be used to enter the help
command (but not \samp{he} or \samp{hel}, nor \samp{H} or
\samp{Help} or \samp{HELP}).  Arguments to commands must be
separated by whitespace (spaces or tabs).  Optional arguments are
enclosed in square brackets (\samp{[]}) in the command syntax; the
square brackets must not be typed.  Alternatives in the command syntax
are separated by a vertical bar (\samp{|}).

Entering a blank line repeats the last command entered.  Exception: if
the last command was a \samp{list} command, the next 11 lines are
listed.

Commands that the debugger doesn't recognize are assumed to be Python
statements and are executed in the context of the program being
debugged.  Python statements can also be prefixed with an exclamation
point (\samp{!}).  This is a powerful way to inspect the program
being debugged; it is even possible to change a variable or call a
function.  When an
exception occurs in such a statement, the exception name is printed
but the debugger's state is not changed.

Multiple commands may be entered on a single line, separated by
\samp{;;}.  (A single \samp{;} is not used as it is
the separator for multiple commands in a line that is passed to
the Python parser.)
No intelligence is applied to separating the commands;
the input is split at the first \samp{;;} pair, even if it is in
the middle of a quoted string.

The debugger supports aliases.  Aliases can have parameters which
allows one a certain level of adaptability to the context under
examination.

If a file \file{.pdbrc}
\indexii{.pdbrc}{file}\indexiii{debugger}{configuration}{file}
exists in the user's home directory or in the current directory, it is
read in and executed as if it had been typed at the debugger prompt.
This is particularly useful for aliases.  If both files exist, the one
in the home directory is read first and aliases defined there can be
overridden by the local file.

\begin{description}

\item[h(elp) \optional{\var{command}}]

Without argument, print the list of available commands.  With a
\var{command} as argument, print help about that command.  \samp{help
pdb} displays the full documentation file; if the environment variable
\envvar{PAGER} is defined, the file is piped through that command
instead.  Since the \var{command} argument must be an identifier,
\samp{help exec} must be entered to get help on the \samp{!} command.

\item[w(here)]

Print a stack trace, with the most recent frame at the bottom.  An
arrow indicates the current frame, which determines the context of
most commands.

\item[d(own)]

Move the current frame one level down in the stack trace
(to an newer frame).

\item[u(p)]

Move the current frame one level up in the stack trace
(to a older frame).

\item[b(reak) \optional{\optional{\var{filename}:}\var{lineno}\code{\Large{|}}\var{function}\optional{, \var{condition}}}]

With a \var{lineno} argument, set a break there in the current
file.  With a \var{function} argument, set a break at the first
executable statement within that function.
The line number may be prefixed with a filename and a colon,
to specify a breakpoint in another file (probably one that
hasn't been loaded yet).  The file is searched on \code{sys.path}.
Note that each breakpoint is assigned a number to which all the other
breakpoint commands refer.

If a second argument is present, it is an expression which must
evaluate to true before the breakpoint is honored.

Without argument, list all breaks, including for each breakpoint,
the number of times that breakpoint has been hit, the current
ignore count, and the associated condition if any.

\item[tbreak \optional{\optional{\var{filename}:}\var{lineno}\code{\Large{|}}\var{function}\optional{, \var{condition}}}]

Temporary breakpoint, which is removed automatically when it is
first hit.  The arguments are the same as break.

\item[cl(ear) \optional{\var{bpnumber} \optional{\var{bpnumber ...}}}]

With a space separated list of breakpoint numbers, clear those
breakpoints.  Without argument, clear all breaks (but first
ask confirmation).

\item[disable \optional{\var{bpnumber} \optional{\var{bpnumber ...}}}]

Disables the breakpoints given as a space separated list of
breakpoint numbers.  Disabling a breakpoint means it cannot cause
the program to stop execution, but unlike clearing a breakpoint, it
remains in the list of breakpoints and can be (re-)enabled.

\item[enable \optional{\var{bpnumber} \optional{\var{bpnumber ...}}}]

Enables the breakpoints specified.

\item[ignore \var{bpnumber} \optional{\var{count}}]

Sets the ignore count for the given breakpoint number.  If
count is omitted, the ignore count is set to 0.  A breakpoint
becomes active when the ignore count is zero.  When non-zero,
the count is decremented each time the breakpoint is reached
and the breakpoint is not disabled and any associated condition
evaluates to true.

\item[condition \var{bpnumber} \optional{\var{condition}}]

Condition is an expression which must evaluate to true before
the breakpoint is honored.  If condition is absent, any existing
condition is removed; i.e., the breakpoint is made unconditional.

\item[s(tep)]

Execute the current line, stop at the first possible occasion
(either in a function that is called or on the next line in the
current function).

\item[n(ext)]

Continue execution until the next line in the current function
is reached or it returns.  (The difference between \samp{next} and
\samp{step} is that \samp{step} stops inside a called function, while
\samp{next} executes called functions at (nearly) full speed, only
stopping at the next line in the current function.)

\item[r(eturn)]

Continue execution until the current function returns.

\item[c(ont(inue))]

Continue execution, only stop when a breakpoint is encountered.

\item[l(ist) \optional{\var{first\optional{, last}}}]

List source code for the current file.  Without arguments, list 11
lines around the current line or continue the previous listing.  With
one argument, list 11 lines around at that line.  With two arguments,
list the given range; if the second argument is less than the first,
it is interpreted as a count.

\item[a(rgs)]

Print the argument list of the current function.

\item[p \var{expression}]

Evaluate the \var{expression} in the current context and print its
value.  (Note: \samp{print} can also be used, but is not a debugger
command --- this executes the Python \keyword{print} statement.)

\item[alias \optional{\var{name} \optional{command}}]

Creates an alias called \var{name} that executes \var{command}.  The
command must \emph{not} be enclosed in quotes.  Replaceable parameters
can be indicated by \samp{\%1}, \samp{\%2}, and so on, while \samp{\%*} is
replaced by all the parameters.  If no command is given, the current
alias for \var{name} is shown. If no arguments are given, all
aliases are listed.

Aliases may be nested and can contain anything that can be
legally typed at the pdb prompt.  Note that internal pdb commands
\emph{can} be overridden by aliases.  Such a command is
then hidden until the alias is removed.  Aliasing is recursively
applied to the first word of the command line; all other words
in the line are left alone.

As an example, here are two useful aliases (especially when placed
in the \file{.pdbrc} file):

\begin{verbatim}
#Print instance variables (usage "pi classInst")
alias pi for k in %1.__dict__.keys(): print "%1.",k,"=",%1.__dict__[k]
#Print instance variables in self
alias ps pi self
\end{verbatim}
		
\item[unalias \var{name}]

Deletes the specified alias.

\item[\optional{!}\var{statement}]

Execute the (one-line) \var{statement} in the context of
the current stack frame.
The exclamation point can be omitted unless the first word
of the statement resembles a debugger command.
To set a global variable, you can prefix the assignment
command with a \samp{global} command on the same line, e.g.:

\begin{verbatim}
(Pdb) global list_options; list_options = ['-l']
(Pdb)
\end{verbatim}

\item[q(uit)]

Quit from the debugger.
The program being executed is aborted.

\end{description}

\section{How It Works}

Some changes were made to the interpreter:

\begin{itemize}
\item \code{sys.settrace(\var{func})} sets the global trace function
\item there can also a local trace function (see later)
\end{itemize}

Trace functions have three arguments: \var{frame}, \var{event}, and
\var{arg}. \var{frame} is the current stack frame.  \var{event} is a
string: \code{'call'}, \code{'line'}, \code{'return'} or
\code{'exception'}.  \var{arg} depends on the event type.

The global trace function is invoked (with \var{event} set to
\code{'call'}) whenever a new local scope is entered; it should return
a reference to the local trace function to be used that scope, or
\code{None} if the scope shouldn't be traced.

The local trace function should return a reference to itself (or to
another function for further tracing in that scope), or \code{None} to
turn off tracing in that scope.

Instance methods are accepted (and very useful!) as trace functions.

The events have the following meaning:

\begin{description}

\item[\code{'call'}]
A function is called (or some other code block entered).  The global
trace function is called; arg is the argument list to the function;
the return value specifies the local trace function.

\item[\code{'line'}]
The interpreter is about to execute a new line of code (sometimes
multiple line events on one line exist).  The local trace function is
called; arg in None; the return value specifies the new local trace
function.

\item[\code{'return'}]
A function (or other code block) is about to return.  The local trace
function is called; arg is the value that will be returned.  The trace
function's return value is ignored.

\item[\code{'exception'}]
An exception has occurred.  The local trace function is called; arg is
a triple (exception, value, traceback); the return value specifies the
new local trace function

\end{description}

Note that as an exception is propagated down the chain of callers, an
\code{'exception'} event is generated at each level.

For more information on code and frame objects, refer to the
\citetitle[../ref/ref.html]{Python Reference Manual}.
			% The Python Debugger

\chapter{The Python Profilers \label{profile}}

\sectionauthor{James Roskind}{}

Copyright \copyright{} 1994, by InfoSeek Corporation, all rights reserved.
\index{InfoSeek Corporation}

Written by James Roskind.\footnote{
  Updated and converted to \LaTeX\ by Guido van Rossum.
  Further updated by Armin Rigo to integrate the documentation for the new
  \module{cProfile} module of Python 2.5.}

Permission to use, copy, modify, and distribute this Python software
and its associated documentation for any purpose (subject to the
restriction in the following sentence) without fee is hereby granted,
provided that the above copyright notice appears in all copies, and
that both that copyright notice and this permission notice appear in
supporting documentation, and that the name of InfoSeek not be used in
advertising or publicity pertaining to distribution of the software
without specific, written prior permission.  This permission is
explicitly restricted to the copying and modification of the software
to remain in Python, compiled Python, or other languages (such as C)
wherein the modified or derived code is exclusively imported into a
Python module.

INFOSEEK CORPORATION DISCLAIMS ALL WARRANTIES WITH REGARD TO THIS
SOFTWARE, INCLUDING ALL IMPLIED WARRANTIES OF MERCHANTABILITY AND
FITNESS. IN NO EVENT SHALL INFOSEEK CORPORATION BE LIABLE FOR ANY
SPECIAL, INDIRECT OR CONSEQUENTIAL DAMAGES OR ANY DAMAGES WHATSOEVER
RESULTING FROM LOSS OF USE, DATA OR PROFITS, WHETHER IN AN ACTION OF
CONTRACT, NEGLIGENCE OR OTHER TORTIOUS ACTION, ARISING OUT OF OR IN
CONNECTION WITH THE USE OR PERFORMANCE OF THIS SOFTWARE.


The profiler was written after only programming in Python for 3 weeks.
As a result, it is probably clumsy code, but I don't know for sure yet
'cause I'm a beginner :-).  I did work hard to make the code run fast,
so that profiling would be a reasonable thing to do.  I tried not to
repeat code fragments, but I'm sure I did some stuff in really awkward
ways at times.  Please send suggestions for improvements to:
\email{jar@netscape.com}.  I won't promise \emph{any} support.  ...but
I'd appreciate the feedback.


\section{Introduction to the profilers}
\nodename{Profiler Introduction}

A \dfn{profiler} is a program that describes the run time performance
of a program, providing a variety of statistics.  This documentation
describes the profiler functionality provided in the modules
\module{profile} and \module{pstats}.  This profiler provides
\dfn{deterministic profiling} of any Python programs.  It also
provides a series of report generation tools to allow users to rapidly
examine the results of a profile operation.
\index{deterministic profiling}
\index{profiling, deterministic}

The Python standard library provides three different profilers:

\begin{enumerate}
\item \module{profile}, a pure Python module, described in the sequel.
  Copyright \copyright{} 1994, by InfoSeek Corporation.
  \versionchanged[also reports the time spent in calls to built-in
  functions and methods]{2.4}

\item \module{cProfile}, a module written in C, with a reasonable
  overhead that makes it suitable for profiling long-running programs.
  Based on \module{lsprof}, contributed by Brett Rosen and Ted Czotter.
  \versionadded{2.5}

\item \module{hotshot}, a C module focusing on minimizing the overhead
  while profiling, at the expense of long data post-processing times.
  \versionchanged[the results should be more meaningful than in the
  past: the timing core contained a critical bug]{2.5}
\end{enumerate}

The \module{profile} and \module{cProfile} modules export the same
interface, so they are mostly interchangeables; \module{cProfile} has a
much lower overhead but is not so far as well-tested and might not be
available on all systems.  \module{cProfile} is really a compatibility
layer on top of the internal \module{_lsprof} module.  The
\module{hotshot} module is reserved to specialized usages.

%\section{How Is This Profiler Different From The Old Profiler?}
%\nodename{Profiler Changes}
%
%(This section is of historical importance only; the old profiler
%discussed here was last seen in Python 1.1.)
%
%The big changes from old profiling module are that you get more
%information, and you pay less CPU time.  It's not a trade-off, it's a
%trade-up.
%
%To be specific:
%
%\begin{description}
%
%\item[Bugs removed:]
%Local stack frame is no longer molested, execution time is now charged
%to correct functions.
%
%\item[Accuracy increased:]
%Profiler execution time is no longer charged to user's code,
%calibration for platform is supported, file reads are not done \emph{by}
%profiler \emph{during} profiling (and charged to user's code!).
%
%\item[Speed increased:]
%Overhead CPU cost was reduced by more than a factor of two (perhaps a
%factor of five), lightweight profiler module is all that must be
%loaded, and the report generating module (\module{pstats}) is not needed
%during profiling.
%
%\item[Recursive functions support:]
%Cumulative times in recursive functions are correctly calculated;
%recursive entries are counted.
%
%\item[Large growth in report generating UI:]
%Distinct profiles runs can be added together forming a comprehensive
%report; functions that import statistics take arbitrary lists of
%files; sorting criteria is now based on keywords (instead of 4 integer
%options); reports shows what functions were profiled as well as what
%profile file was referenced; output format has been improved.
%
%\end{description}


\section{Instant User's Manual \label{profile-instant}}

This section is provided for users that ``don't want to read the
manual.'' It provides a very brief overview, and allows a user to
rapidly perform profiling on an existing application.

To profile an application with a main entry point of \function{foo()},
you would add the following to your module:

\begin{verbatim}
import cProfile
cProfile.run('foo()')
\end{verbatim}

(Use \module{profile} instead of \module{cProfile} if the latter is not
available on your system.)

The above action would cause \function{foo()} to be run, and a series of
informative lines (the profile) to be printed.  The above approach is
most useful when working with the interpreter.  If you would like to
save the results of a profile into a file for later examination, you
can supply a file name as the second argument to the \function{run()}
function:

\begin{verbatim}
import cProfile
cProfile.run('foo()', 'fooprof')
\end{verbatim}

The file \file{cProfile.py} can also be invoked as
a script to profile another script.  For example:

\begin{verbatim}
python -m cProfile myscript.py
\end{verbatim}

\file{cProfile.py} accepts two optional arguments on the command line:

\begin{verbatim}
cProfile.py [-o output_file] [-s sort_order]
\end{verbatim}

\programopt{-s} only applies to standard output (\programopt{-o} is
not supplied).  Look in the \class{Stats} documentation for valid sort
values.

When you wish to review the profile, you should use the methods in the
\module{pstats} module.  Typically you would load the statistics data as
follows:

\begin{verbatim}
import pstats
p = pstats.Stats('fooprof')
\end{verbatim}

The class \class{Stats} (the above code just created an instance of
this class) has a variety of methods for manipulating and printing the
data that was just read into \code{p}.  When you ran
\function{cProfile.run()} above, what was printed was the result of three
method calls:

\begin{verbatim}
p.strip_dirs().sort_stats(-1).print_stats()
\end{verbatim}

The first method removed the extraneous path from all the module
names. The second method sorted all the entries according to the
standard module/line/name string that is printed.
%(this is to comply with the semantics of the old profiler).
The third method printed out
all the statistics.  You might try the following sort calls:

\begin{verbatim}
p.sort_stats('name')
p.print_stats()
\end{verbatim}

The first call will actually sort the list by function name, and the
second call will print out the statistics.  The following are some
interesting calls to experiment with:

\begin{verbatim}
p.sort_stats('cumulative').print_stats(10)
\end{verbatim}

This sorts the profile by cumulative time in a function, and then only
prints the ten most significant lines.  If you want to understand what
algorithms are taking time, the above line is what you would use.

If you were looking to see what functions were looping a lot, and
taking a lot of time, you would do:

\begin{verbatim}
p.sort_stats('time').print_stats(10)
\end{verbatim}

to sort according to time spent within each function, and then print
the statistics for the top ten functions.

You might also try:

\begin{verbatim}
p.sort_stats('file').print_stats('__init__')
\end{verbatim}

This will sort all the statistics by file name, and then print out
statistics for only the class init methods (since they are spelled
with \code{__init__} in them).  As one final example, you could try:

\begin{verbatim}
p.sort_stats('time', 'cum').print_stats(.5, 'init')
\end{verbatim}

This line sorts statistics with a primary key of time, and a secondary
key of cumulative time, and then prints out some of the statistics.
To be specific, the list is first culled down to 50\% (re: \samp{.5})
of its original size, then only lines containing \code{init} are
maintained, and that sub-sub-list is printed.

If you wondered what functions called the above functions, you could
now (\code{p} is still sorted according to the last criteria) do:

\begin{verbatim}
p.print_callers(.5, 'init')
\end{verbatim}

and you would get a list of callers for each of the listed functions.

If you want more functionality, you're going to have to read the
manual, or guess what the following functions do:

\begin{verbatim}
p.print_callees()
p.add('fooprof')
\end{verbatim}

Invoked as a script, the \module{pstats} module is a statistics
browser for reading and examining profile dumps.  It has a simple
line-oriented interface (implemented using \refmodule{cmd}) and
interactive help.

\section{What Is Deterministic Profiling?}
\nodename{Deterministic Profiling}

\dfn{Deterministic profiling} is meant to reflect the fact that all
\emph{function call}, \emph{function return}, and \emph{exception} events
are monitored, and precise timings are made for the intervals between
these events (during which time the user's code is executing).  In
contrast, \dfn{statistical profiling} (which is not done by this
module) randomly samples the effective instruction pointer, and
deduces where time is being spent.  The latter technique traditionally
involves less overhead (as the code does not need to be instrumented),
but provides only relative indications of where time is being spent.

In Python, since there is an interpreter active during execution, the
presence of instrumented code is not required to do deterministic
profiling.  Python automatically provides a \dfn{hook} (optional
callback) for each event.  In addition, the interpreted nature of
Python tends to add so much overhead to execution, that deterministic
profiling tends to only add small processing overhead in typical
applications.  The result is that deterministic profiling is not that
expensive, yet provides extensive run time statistics about the
execution of a Python program.

Call count statistics can be used to identify bugs in code (surprising
counts), and to identify possible inline-expansion points (high call
counts).  Internal time statistics can be used to identify ``hot
loops'' that should be carefully optimized.  Cumulative time
statistics should be used to identify high level errors in the
selection of algorithms.  Note that the unusual handling of cumulative
times in this profiler allows statistics for recursive implementations
of algorithms to be directly compared to iterative implementations.


\section{Reference Manual -- \module{profile} and \module{cProfile}}

\declaremodule{standard}{profile}
\declaremodule{standard}{cProfile}
\modulesynopsis{Python profiler}



The primary entry point for the profiler is the global function
\function{profile.run()} (resp. \function{cProfile.run()}).
It is typically used to create any profile
information.  The reports are formatted and printed using methods of
the class \class{pstats.Stats}.  The following is a description of all
of these standard entry points and functions.  For a more in-depth
view of some of the code, consider reading the later section on
Profiler Extensions, which includes discussion of how to derive
``better'' profilers from the classes presented, or reading the source
code for these modules.

\begin{funcdesc}{run}{command\optional{, filename}}

This function takes a single argument that has can be passed to the
\keyword{exec} statement, and an optional file name.  In all cases this
routine attempts to \keyword{exec} its first argument, and gather profiling
statistics from the execution. If no file name is present, then this
function automatically prints a simple profiling report, sorted by the
standard name string (file/line/function-name) that is presented in
each line.  The following is a typical output from such a call:

\begin{verbatim}
      2706 function calls (2004 primitive calls) in 4.504 CPU seconds

Ordered by: standard name

ncalls  tottime  percall  cumtime  percall filename:lineno(function)
     2    0.006    0.003    0.953    0.477 pobject.py:75(save_objects)
  43/3    0.533    0.012    0.749    0.250 pobject.py:99(evaluate)
 ...
\end{verbatim}

The first line indicates that 2706 calls were
monitored.  Of those calls, 2004 were \dfn{primitive}.  We define
\dfn{primitive} to mean that the call was not induced via recursion.
The next line: \code{Ordered by:\ standard name}, indicates that
the text string in the far right column was used to sort the output.
The column headings include:

\begin{description}

\item[ncalls ]
for the number of calls,

\item[tottime ]
for the total time spent in the given function (and excluding time
made in calls to sub-functions),

\item[percall ]
is the quotient of \code{tottime} divided by \code{ncalls}

\item[cumtime ]
is the total time spent in this and all subfunctions (from invocation
till exit). This figure is accurate \emph{even} for recursive
functions.

\item[percall ]
is the quotient of \code{cumtime} divided by primitive calls

\item[filename:lineno(function) ]
provides the respective data of each function

\end{description}

When there are two numbers in the first column (for example,
\samp{43/3}), then the latter is the number of primitive calls, and
the former is the actual number of calls.  Note that when the function
does not recurse, these two values are the same, and only the single
figure is printed.

\end{funcdesc}

\begin{funcdesc}{runctx}{command, globals, locals\optional{, filename}}
This function is similar to \function{run()}, with added
arguments to supply the globals and locals dictionaries for the
\var{command} string.
\end{funcdesc}

Analysis of the profiler data is done using this class from the
\module{pstats} module:

% now switch modules....
% (This \stmodindex use may be hard to change ;-( )
\stmodindex{pstats}

\begin{classdesc}{Stats}{filename\optional{, \moreargs}}
This class constructor creates an instance of a ``statistics object''
from a \var{filename} (or set of filenames).  \class{Stats} objects are
manipulated by methods, in order to print useful reports.

The file selected by the above constructor must have been created by
the corresponding version of \module{profile} or \module{cProfile}.
To be specific, there is
\emph{no} file compatibility guaranteed with future versions of this
profiler, and there is no compatibility with files produced by other
profilers.
%(such as the old system profiler).

If several files are provided, all the statistics for identical
functions will be coalesced, so that an overall view of several
processes can be considered in a single report.  If additional files
need to be combined with data in an existing \class{Stats} object, the
\method{add()} method can be used.
\end{classdesc}


\subsection{The \class{Stats} Class \label{profile-stats}}

\class{Stats} objects have the following methods:

\begin{methoddesc}[Stats]{strip_dirs}{}
This method for the \class{Stats} class removes all leading path
information from file names.  It is very useful in reducing the size
of the printout to fit within (close to) 80 columns.  This method
modifies the object, and the stripped information is lost.  After
performing a strip operation, the object is considered to have its
entries in a ``random'' order, as it was just after object
initialization and loading.  If \method{strip_dirs()} causes two
function names to be indistinguishable (they are on the same
line of the same filename, and have the same function name), then the
statistics for these two entries are accumulated into a single entry.
\end{methoddesc}


\begin{methoddesc}[Stats]{add}{filename\optional{, \moreargs}}
This method of the \class{Stats} class accumulates additional
profiling information into the current profiling object.  Its
arguments should refer to filenames created by the corresponding
version of \function{profile.run()} or \function{cProfile.run()}.
Statistics for identically named
(re: file, line, name) functions are automatically accumulated into
single function statistics.
\end{methoddesc}

\begin{methoddesc}[Stats]{dump_stats}{filename}
Save the data loaded into the \class{Stats} object to a file named
\var{filename}.  The file is created if it does not exist, and is
overwritten if it already exists.  This is equivalent to the method of
the same name on the \class{profile.Profile} and
\class{cProfile.Profile} classes.
\versionadded{2.3}
\end{methoddesc}

\begin{methoddesc}[Stats]{sort_stats}{key\optional{, \moreargs}}
This method modifies the \class{Stats} object by sorting it according
to the supplied criteria.  The argument is typically a string
identifying the basis of a sort (example: \code{'time'} or
\code{'name'}).

When more than one key is provided, then additional keys are used as
secondary criteria when there is equality in all keys selected
before them.  For example, \code{sort_stats('name', 'file')} will sort
all the entries according to their function name, and resolve all ties
(identical function names) by sorting by file name.

Abbreviations can be used for any key names, as long as the
abbreviation is unambiguous.  The following are the keys currently
defined:

\begin{tableii}{l|l}{code}{Valid Arg}{Meaning}
  \lineii{'calls'}{call count}
  \lineii{'cumulative'}{cumulative time}
  \lineii{'file'}{file name}
  \lineii{'module'}{file name}
  \lineii{'pcalls'}{primitive call count}
  \lineii{'line'}{line number}
  \lineii{'name'}{function name}
  \lineii{'nfl'}{name/file/line}
  \lineii{'stdname'}{standard name}
  \lineii{'time'}{internal time}
\end{tableii}

Note that all sorts on statistics are in descending order (placing
most time consuming items first), where as name, file, and line number
searches are in ascending order (alphabetical). The subtle
distinction between \code{'nfl'} and \code{'stdname'} is that the
standard name is a sort of the name as printed, which means that the
embedded line numbers get compared in an odd way.  For example, lines
3, 20, and 40 would (if the file names were the same) appear in the
string order 20, 3 and 40.  In contrast, \code{'nfl'} does a numeric
compare of the line numbers.  In fact, \code{sort_stats('nfl')} is the
same as \code{sort_stats('name', 'file', 'line')}.

%For compatibility with the old profiler,
For backward-compatibility reasons, the numeric arguments
\code{-1}, \code{0}, \code{1}, and \code{2} are permitted.  They are
interpreted as \code{'stdname'}, \code{'calls'}, \code{'time'}, and
\code{'cumulative'} respectively.  If this old style format (numeric)
is used, only one sort key (the numeric key) will be used, and
additional arguments will be silently ignored.
\end{methoddesc}


\begin{methoddesc}[Stats]{reverse_order}{}
This method for the \class{Stats} class reverses the ordering of the basic
list within the object.  %This method is provided primarily for
%compatibility with the old profiler.
Note that by default ascending vs descending order is properly selected
based on the sort key of choice.
\end{methoddesc}

\begin{methoddesc}[Stats]{print_stats}{\optional{restriction, \moreargs}}
This method for the \class{Stats} class prints out a report as described
in the \function{profile.run()} definition.

The order of the printing is based on the last \method{sort_stats()}
operation done on the object (subject to caveats in \method{add()} and
\method{strip_dirs()}).

The arguments provided (if any) can be used to limit the list down to
the significant entries.  Initially, the list is taken to be the
complete set of profiled functions.  Each restriction is either an
integer (to select a count of lines), or a decimal fraction between
0.0 and 1.0 inclusive (to select a percentage of lines), or a regular
expression (to pattern match the standard name that is printed; as of
Python 1.5b1, this uses the Perl-style regular expression syntax
defined by the \refmodule{re} module).  If several restrictions are
provided, then they are applied sequentially.  For example:

\begin{verbatim}
print_stats(.1, 'foo:')
\end{verbatim}

would first limit the printing to first 10\% of list, and then only
print functions that were part of filename \file{.*foo:}.  In
contrast, the command:

\begin{verbatim}
print_stats('foo:', .1)
\end{verbatim}

would limit the list to all functions having file names \file{.*foo:},
and then proceed to only print the first 10\% of them.
\end{methoddesc}


\begin{methoddesc}[Stats]{print_callers}{\optional{restriction, \moreargs}}
This method for the \class{Stats} class prints a list of all functions
that called each function in the profiled database.  The ordering is
identical to that provided by \method{print_stats()}, and the definition
of the restricting argument is also identical.  Each caller is reported on
its own line.  The format differs slightly depending on the profiler that
produced the stats:

\begin{itemize}
\item With \module{profile}, a number is shown in parentheses after each
  caller to show how many times this specific call was made.  For
  convenience, a second non-parenthesized number repeats the cumulative
  time spent in the function at the right.

\item With \module{cProfile}, each caller is preceeded by three numbers:
  the number of times this specific call was made, and the total and
  cumulative times spent in the current function while it was invoked by
  this specific caller.
\end{itemize}
\end{methoddesc}

\begin{methoddesc}[Stats]{print_callees}{\optional{restriction, \moreargs}}
This method for the \class{Stats} class prints a list of all function
that were called by the indicated function.  Aside from this reversal
of direction of calls (re: called vs was called by), the arguments and
ordering are identical to the \method{print_callers()} method.
\end{methoddesc}


\section{Limitations \label{profile-limits}}

One limitation has to do with accuracy of timing information.
There is a fundamental problem with deterministic profilers involving
accuracy.  The most obvious restriction is that the underlying ``clock''
is only ticking at a rate (typically) of about .001 seconds.  Hence no
measurements will be more accurate than the underlying clock.  If
enough measurements are taken, then the ``error'' will tend to average
out. Unfortunately, removing this first error induces a second source
of error.

The second problem is that it ``takes a while'' from when an event is
dispatched until the profiler's call to get the time actually
\emph{gets} the state of the clock.  Similarly, there is a certain lag
when exiting the profiler event handler from the time that the clock's
value was obtained (and then squirreled away), until the user's code
is once again executing.  As a result, functions that are called many
times, or call many functions, will typically accumulate this error.
The error that accumulates in this fashion is typically less than the
accuracy of the clock (less than one clock tick), but it
\emph{can} accumulate and become very significant.

The problem is more important with \module{profile} than with the
lower-overhead \module{cProfile}.  For this reason, \module{profile}
provides a means of calibrating itself for a given platform so that
this error can be probabilistically (on the average) removed.
After the profiler is calibrated, it will be more accurate (in a least
square sense), but it will sometimes produce negative numbers (when
call counts are exceptionally low, and the gods of probability work
against you :-). )  Do \emph{not} be alarmed by negative numbers in
the profile.  They should \emph{only} appear if you have calibrated
your profiler, and the results are actually better than without
calibration.


\section{Calibration \label{profile-calibration}}

The profiler of the \module{profile} module subtracts a constant from each
event handling time to compensate for the overhead of calling the time
function, and socking away the results.  By default, the constant is 0.
The following procedure can
be used to obtain a better constant for a given platform (see discussion
in section Limitations above).

\begin{verbatim}
import profile
pr = profile.Profile()
for i in range(5):
    print pr.calibrate(10000)
\end{verbatim}

The method executes the number of Python calls given by the argument,
directly and again under the profiler, measuring the time for both.
It then computes the hidden overhead per profiler event, and returns
that as a float.  For example, on an 800 MHz Pentium running
Windows 2000, and using Python's time.clock() as the timer,
the magical number is about 12.5e-6.

The object of this exercise is to get a fairly consistent result.
If your computer is \emph{very} fast, or your timer function has poor
resolution, you might have to pass 100000, or even 1000000, to get
consistent results.

When you have a consistent answer,
there are three ways you can use it:\footnote{Prior to Python 2.2, it
  was necessary to edit the profiler source code to embed the bias as
  a literal number.  You still can, but that method is no longer
  described, because no longer needed.}

\begin{verbatim}
import profile

# 1. Apply computed bias to all Profile instances created hereafter.
profile.Profile.bias = your_computed_bias

# 2. Apply computed bias to a specific Profile instance.
pr = profile.Profile()
pr.bias = your_computed_bias

# 3. Specify computed bias in instance constructor.
pr = profile.Profile(bias=your_computed_bias)
\end{verbatim}

If you have a choice, you are better off choosing a smaller constant, and
then your results will ``less often'' show up as negative in profile
statistics.


\section{Extensions --- Deriving Better Profilers}
\nodename{Profiler Extensions}

The \class{Profile} class of both modules, \module{profile} and
\module{cProfile}, were written so that
derived classes could be developed to extend the profiler.  The details
are not described here, as doing this successfully requires an expert
understanding of how the \class{Profile} class works internally.  Study
the source code of the module carefully if you want to
pursue this.

If all you want to do is change how current time is determined (for
example, to force use of wall-clock time or elapsed process time),
pass the timing function you want to the \class{Profile} class
constructor:

\begin{verbatim}
pr = profile.Profile(your_time_func)
\end{verbatim}

The resulting profiler will then call \function{your_time_func()}.

\begin{description}
\item[\class{profile.Profile}]
\function{your_time_func()} should return a single number, or a list of
numbers whose sum is the current time (like what \function{os.times()}
returns).  If the function returns a single time number, or the list of
returned numbers has length 2, then you will get an especially fast
version of the dispatch routine.

Be warned that you should calibrate the profiler class for the
timer function that you choose.  For most machines, a timer that
returns a lone integer value will provide the best results in terms of
low overhead during profiling.  (\function{os.times()} is
\emph{pretty} bad, as it returns a tuple of floating point values).  If
you want to substitute a better timer in the cleanest fashion,
derive a class and hardwire a replacement dispatch method that best
handles your timer call, along with the appropriate calibration
constant.

\item[\class{cProfile.Profile}]
\function{your_time_func()} should return a single number.  If it returns
plain integers, you can also invoke the class constructor with a second
argument specifying the real duration of one unit of time.  For example,
if \function{your_integer_time_func()} returns times measured in thousands
of seconds, you would constuct the \class{Profile} instance as follows:

\begin{verbatim}
pr = profile.Profile(your_integer_time_func, 0.001)
\end{verbatim}

As the \module{cProfile.Profile} class cannot be calibrated, custom
timer functions should be used with care and should be as fast as
possible.  For the best results with a custom timer, it might be
necessary to hard-code it in the C source of the internal
\module{_lsprof} module.

\end{description}
		% The Python Profiler

\chapter{Internet Protocols and Support \label{internet}}

\index{WWW}
\index{Internet}
\index{World-Wide Web}

The modules described in this chapter implement Internet protocols and 
support for related technology.  They are all implemented in Python.
Some of these modules require the presence of the system-dependent
module \module{socket}\refbimodindex{socket}, which is currently only
fully supported on \UNIX{} and Windows NT.  Here is an overview:

\localmoduletable
		% Internet Protocols
\section{Standard Module \sectcode{cgi}}
\label{module-cgi}
\stmodindex{cgi}
\indexii{WWW}{server}
\indexii{CGI}{protocol}
\indexii{HTTP}{protocol}
\indexii{MIME}{headers}
\index{URL}

\renewcommand{\indexsubitem}{(in module cgi)}

Support module for CGI (Common Gateway Interface) scripts.

This module defines a number of utilities for use by CGI scripts
written in Python.

\subsection{Introduction}
\nodename{Introduction to the CGI module}

A CGI script is invoked by an HTTP server, usually to process user
input submitted through an HTML \code{<FORM>} or \code{<ISINPUT>} element.

Most often, CGI scripts live in the server's special \code{cgi-bin}
directory.  The HTTP server places all sorts of information about the
request (such as the client's hostname, the requested URL, the query
string, and lots of other goodies) in the script's shell environment,
executes the script, and sends the script's output back to the client.

The script's input is connected to the client too, and sometimes the
form data is read this way; at other times the form data is passed via
the ``query string'' part of the URL.  This module (\code{cgi.py}) is intended
to take care of the different cases and provide a simpler interface to
the Python script.  It also provides a number of utilities that help
in debugging scripts, and the latest addition is support for file
uploads from a form (if your browser supports it -- Grail 0.3 and
Netscape 2.0 do).

The output of a CGI script should consist of two sections, separated
by a blank line.  The first section contains a number of headers,
telling the client what kind of data is following.  Python code to
generate a minimal header section looks like this:

\bcode\begin{verbatim}
print "Content-type: text/html"     # HTML is following
print                               # blank line, end of headers
\end{verbatim}\ecode
%
The second section is usually HTML, which allows the client software
to display nicely formatted text with header, in-line images, etc.
Here's Python code that prints a simple piece of HTML:

\bcode\begin{verbatim}
print "<TITLE>CGI script output</TITLE>"
print "<H1>This is my first CGI script</H1>"
print "Hello, world!"
\end{verbatim}\ecode
%
(It may not be fully legal HTML according to the letter of the
standard, but any browser will understand it.)

\subsection{Using the cgi module}
\nodename{Using the cgi module}

Begin by writing \code{import cgi}.  Don't use \code{from cgi import *} -- the
module defines all sorts of names for its own use or for backward 
compatibility that you don't want in your namespace.

It's best to use the \code{FieldStorage} class.  The other classes define in this 
module are provided mostly for backward compatibility.  Instantiate it 
exactly once, without arguments.  This reads the form contents from 
standard input or the environment (depending on the value of various 
environment variables set according to the CGI standard).  Since it may 
consume standard input, it should be instantiated only once.

The \code{FieldStorage} instance can be accessed as if it were a Python 
dictionary.  For instance, the following code (which assumes that the 
\code{Content-type} header and blank line have already been printed) checks that 
the fields \code{name} and \code{addr} are both set to a non-empty string:

\bcode\begin{verbatim}
form = cgi.FieldStorage()
form_ok = 0
if form.has_key("name") and form.has_key("addr"):
    if form["name"].value != "" and form["addr"].value != "":
        form_ok = 1
if not form_ok:
    print "<H1>Error</H1>"
    print "Please fill in the name and addr fields."
    return
...further form processing here...
\end{verbatim}\ecode
%
Here the fields, accessed through \code{form[key]}, are themselves instances
of \code{FieldStorage} (or \code{MiniFieldStorage}, depending on the form encoding).

If the submitted form data contains more than one field with the same
name, the object retrieved by \code{form[key]} is not a \code{(Mini)FieldStorage}
instance but a list of such instances.  If you expect this possibility
(i.e., when your HTML form comtains multiple fields with the same
name), use the \code{type()} function to determine whether you have a single
instance or a list of instances.  For example, here's code that
concatenates any number of username fields, separated by commas:

\bcode\begin{verbatim}
username = form["username"]
if type(username) is type([]):
    # Multiple username fields specified
    usernames = ""
    for item in username:
        if usernames:
            # Next item -- insert comma
            usernames = usernames + "," + item.value
        else:
            # First item -- don't insert comma
            usernames = item.value
else:
    # Single username field specified
    usernames = username.value
\end{verbatim}\ecode
%
If a field represents an uploaded file, the value attribute reads the 
entire file in memory as a string.  This may not be what you want.  You can 
test for an uploaded file by testing either the filename attribute or the 
file attribute.  You can then read the data at leasure from the file 
attribute:

\bcode\begin{verbatim}
fileitem = form["userfile"]
if fileitem.file:
    # It's an uploaded file; count lines
    linecount = 0
    while 1:
        line = fileitem.file.readline()
        if not line: break
        linecount = linecount + 1
\end{verbatim}\ecode
%
The file upload draft standard entertains the possibility of uploading
multiple files from one field (using a recursive \code{multipart/*}
encoding).  When this occurs, the item will be a dictionary-like
FieldStorage item.  This can be determined by testing its type
attribute, which should have the value \code{multipart/form-data} (or
perhaps another string beginning with \code{multipart/}  It this case, it
can be iterated over recursively just like the top-level form object.

When a form is submitted in the ``old'' format (as the query string or as a 
single data part of type \code{application/x-www-form-urlencoded}), the items 
will actually be instances of the class \code{MiniFieldStorage}.  In this case,
the list, file and filename attributes are always \code{None}.


\subsection{Old classes}

These classes, present in earlier versions of the \code{cgi} module, are still 
supported for backward compatibility.  New applications should use the
FieldStorage class.

\code{SvFormContentDict}: single value form content as dictionary; assumes each 
field name occurs in the form only once.

\code{FormContentDict}: multiple value form content as dictionary (the form
items are lists of values).  Useful if your form contains multiple
fields with the same name.

Other classes (\code{FormContent}, \code{InterpFormContentDict}) are present for
backwards compatibility with really old applications only.  If you still 
use these and would be inconvenienced when they disappeared from a next 
version of this module, drop me a note.


\subsection{Functions}
\nodename{Functions in cgi module}

These are useful if you want more control, or if you want to employ
some of the algorithms implemented in this module in other
circumstances.

\begin{funcdesc}{parse}{fp}: Parse a query in the environment or from a file (default \code{sys.stdin}).
\end{funcdesc}

\begin{funcdesc}{parse_qs}{qs}: parse a query string given as a string argument (data of type 
\code{application/x-www-form-urlencoded}).
\end{funcdesc}

\begin{funcdesc}{parse_multipart}{fp\, pdict}: parse input of type \code{multipart/form-data} (for 
file uploads).  Arguments are \code{fp} for the input file and 
    \code{pdict} for the dictionary containing other parameters of \code{content-type} header

    Returns a dictionary just like \code{parse_qs()}: keys are the field names, each 
    value is a list of values for that field.  This is easy to use but not 
    much good if you are expecting megabytes to be uploaded -- in that case, 
    use the \code{FieldStorage} class instead which is much more flexible.  Note 
    that \code{content-type} is the raw, unparsed contents of the \code{content-type} 
    header.

    Note that this does not parse nested multipart parts -- use \code{FieldStorage} for 
    that.
\end{funcdesc}

\begin{funcdesc}{parse_header}{string}: parse a header like \code{Content-type} into a main
content-type and a dictionary of parameters.
\end{funcdesc}

\begin{funcdesc}{test}{}: robust test CGI script, usable as main program.
    Writes minimal HTTP headers and formats all information provided to
    the script in HTML form.
\end{funcdesc}

\begin{funcdesc}{print_environ}{}: format the shell environment in HTML.
\end{funcdesc}

\begin{funcdesc}{print_form}{form}: format a form in HTML.
\end{funcdesc}

\begin{funcdesc}{print_directory}{}: format the current directory in HTML.
\end{funcdesc}

\begin{funcdesc}{print_environ_usage}{}: print a list of useful (used by CGI) environment variables in
HTML.
\end{funcdesc}

\begin{funcdesc}{escape}{}: convert the characters ``\code{\&}'', ``\code{<}'' and ``\code{>}'' to HTML-safe
sequences.  Use this if you need to display text that might contain
such characters in HTML.  To translate URLs for inclusion in the HREF
attribute of an \code{<A>} tag, use \code{urllib.quote()}.
\end{funcdesc}


\subsection{Caring about security}

There's one important rule: if you invoke an external program (e.g.
via the \code{os.system()} or \code{os.popen()} functions), make very sure you don't
pass arbitrary strings received from the client to the shell.  This is
a well-known security hole whereby clever hackers anywhere on the web
can exploit a gullible CGI script to invoke arbitrary shell commands.
Even parts of the URL or field names cannot be trusted, since the
request doesn't have to come from your form!

To be on the safe side, if you must pass a string gotten from a form
to a shell command, you should make sure the string contains only
alphanumeric characters, dashes, underscores, and periods.


\subsection{Installing your CGI script on a Unix system}

Read the documentation for your HTTP server and check with your local
system administrator to find the directory where CGI scripts should be
installed; usually this is in a directory \code{cgi-bin} in the server tree.

Make sure that your script is readable and executable by ``others''; the
Unix file mode should be 755 (use \code{chmod 755 filename}).  Make sure
that the first line of the script contains \code{\#!} starting in column 1
followed by the pathname of the Python interpreter, for instance:

\bcode\begin{verbatim}
#!/usr/local/bin/python
\end{verbatim}\ecode
%
Make sure the Python interpreter exists and is executable by ``others''.

Make sure that any files your script needs to read or write are
readable or writable, respectively, by ``others'' -- their mode should
be 644 for readable and 666 for writable.  This is because, for
security reasons, the HTTP server executes your script as user
``nobody'', without any special privileges.  It can only read (write,
execute) files that everybody can read (write, execute).  The current
directory at execution time is also different (it is usually the
server's cgi-bin directory) and the set of environment variables is
also different from what you get at login.  in particular, don't count
on the shell's search path for executables (\code{\$PATH}) or the Python
module search path (\code{\$PYTHONPATH}) to be set to anything interesting.

If you need to load modules from a directory which is not on Python's
default module search path, you can change the path in your script,
before importing other modules, e.g.:

\bcode\begin{verbatim}
import sys
sys.path.insert(0, "/usr/home/joe/lib/python")
sys.path.insert(0, "/usr/local/lib/python")
\end{verbatim}\ecode
%
(This way, the directory inserted last will be searched first!)

Instructions for non-Unix systems will vary; check your HTTP server's
documentation (it will usually have a section on CGI scripts).


\subsection{Testing your CGI script}

Unfortunately, a CGI script will generally not run when you try it
from the command line, and a script that works perfectly from the
command line may fail mysteriously when run from the server.  There's
one reason why you should still test your script from the command
line: if it contains a syntax error, the python interpreter won't
execute it at all, and the HTTP server will most likely send a cryptic
error to the client.

Assuming your script has no syntax errors, yet it does not work, you
have no choice but to read the next section:


\subsection{Debugging CGI scripts}

First of all, check for trivial installation errors -- reading the
section above on installing your CGI script carefully can save you a
lot of time.  If you wonder whether you have understood the
installation procedure correctly, try installing a copy of this module
file (\code{cgi.py}) as a CGI script.  When invoked as a script, the file
will dump its environment and the contents of the form in HTML form.
Give it the right mode etc, and send it a request.  If it's installed
in the standard \code{cgi-bin} directory, it should be possible to send it a
request by entering a URL into your browser of the form:

\bcode\begin{verbatim}
http://yourhostname/cgi-bin/cgi.py?name=Joe+Blow&addr=At+Home
\end{verbatim}\ecode
%
If this gives an error of type 404, the server cannot find the script
-- perhaps you need to install it in a different directory.  If it
gives another error (e.g.  500), there's an installation problem that
you should fix before trying to go any further.  If you get a nicely
formatted listing of the environment and form content (in this
example, the fields should be listed as ``addr'' with value ``At Home''
and ``name'' with value ``Joe Blow''), the \code{cgi.py} script has been
installed correctly.  If you follow the same procedure for your own
script, you should now be able to debug it.

The next step could be to call the \code{cgi} module's test() function from
your script: replace its main code with the single statement

\bcode\begin{verbatim}
cgi.test()
\end{verbatim}\ecode
%
This should produce the same results as those gotten from installing
the \code{cgi.py} file itself.

When an ordinary Python script raises an unhandled exception
(e.g. because of a typo in a module name, a file that can't be opened,
etc.), the Python interpreter prints a nice traceback and exits.
While the Python interpreter will still do this when your CGI script
raises an exception, most likely the traceback will end up in one of
the HTTP server's log file, or be discarded altogether.

Fortunately, once you have managed to get your script to execute
*some* code, it is easy to catch exceptions and cause a traceback to
be printed.  The \code{test()} function below in this module is an example.
Here are the rules:

\begin{enumerate}
	\item Import the traceback module (before entering the
	   try-except!)
	
	\item Make sure you finish printing the headers and the blank
	   line early
	
	\item Assign \code{sys.stderr} to \code{sys.stdout}
	
	\item Wrap all remaining code in a try-except statement
	
	\item In the except clause, call \code{traceback.print_exc()}
\end{enumerate}

For example:

\bcode\begin{verbatim}
import sys
import traceback
print "Content-type: text/html"
print
sys.stderr = sys.stdout
try:
    ...your code here...
except:
    print "\n\n<PRE>"
    traceback.print_exc()
\end{verbatim}\ecode
%
Notes: The assignment to \code{sys.stderr} is needed because the traceback
prints to \code{sys.stderr}.  The \code{print "$\backslash$n$\backslash$n<PRE>"} statement is necessary to
disable the word wrapping in HTML.

If you suspect that there may be a problem in importing the traceback
module, you can use an even more robust approach (which only uses
built-in modules):

\bcode\begin{verbatim}
import sys
sys.stderr = sys.stdout
print "Content-type: text/plain"
print
...your code here...
\end{verbatim}\ecode
%
This relies on the Python interpreter to print the traceback.  The
content type of the output is set to plain text, which disables all
HTML processing.  If your script works, the raw HTML will be displayed
by your client.  If it raises an exception, most likely after the
first two lines have been printed, a traceback will be displayed.
Because no HTML interpretation is going on, the traceback will
readable.


\subsection{Common problems and solutions}

\begin{itemize}
\item Most HTTP servers buffer the output from CGI scripts until the
script is completed.  This means that it is not possible to display a
progress report on the client's display while the script is running.

\item Check the installation instructions above.

\item Check the HTTP server's log files.  (\code{tail -f logfile} in a separate
window may be useful!)

\item Always check a script for syntax errors first, by doing something
like \code{python script.py}.

\item When using any of the debugging techniques, don't forget to add
\code{import sys} to the top of the script.

\item When invoking external programs, make sure they can be found.
Usually, this means using absolute path names -- \code{\$PATH} is usually not
set to a very useful value in a CGI script.

\item When reading or writing external files, make sure they can be read
or written by every user on the system.

\item Don't try to give a CGI script a set-uid mode.  This doesn't work on
most systems, and is a security liability as well.
\end{itemize}


\section{\module{urllib} ---
         Open an arbitrary resource by URL}

\declaremodule{standard}{urllib}
\modulesynopsis{Open an arbitrary network resource by URL (requires sockets).}

\index{WWW}
\index{World-Wide Web}
\index{URL}


This module provides a high-level interface for fetching data across
the World-Wide Web.  In particular, the \function{urlopen()} function
is similar to the built-in function \function{open()}, but accepts
Universal Resource Locators (URLs) instead of filenames.  Some
restrictions apply --- it can only open URLs for reading, and no seek
operations are available.

It defines the following public functions:

\begin{funcdesc}{urlopen}{url\optional{, data}}
Open a network object denoted by a URL for reading.  If the URL does
not have a scheme identifier, or if it has \file{file:} as its scheme
identifier, this opens a local file; otherwise it opens a socket to a
server somewhere on the network.  If the connection cannot be made, or
if the server returns an error code, the \exception{IOError} exception
is raised.  If all went well, a file-like object is returned.  This
supports the following methods: \method{read()}, \method{readline()},
\method{readlines()}, \method{fileno()}, \method{close()},
\method{info()} and \method{geturl()}.

Except for the \method{info()} and \method{geturl()} methods,
these methods have the same interface as for
file objects --- see section \ref{bltin-file-objects} in this
manual.  (It is not a built-in file object, however, so it can't be
used at those few places where a true built-in file object is
required.)

The \method{info()} method returns an instance of the class
\class{mimetools.Message} containing meta-information associated
with the URL.  When the method is HTTP, these headers are those
returned by the server at the head of the retrieved HTML page
(including Content-Length and Content-Type).  When the method is FTP,
a Content-Length header will be present if (as is now usual) the
server passed back a file length in response to the FTP retrieval
request.  When the method is local-file, returned headers will include
a Date representing the file's last-modified time, a Content-Length
giving file size, and a Content-Type containing a guess at the file's
type. See also the description of the
\refmodule{mimetools}\refstmodindex{mimetools} module.

The \method{geturl()} method returns the real URL of the page.  In
some cases, the HTTP server redirects a client to another URL.  The
\function{urlopen()} function handles this transparently, but in some
cases the caller needs to know which URL the client was redirected
to.  The \method{geturl()} method can be used to get at this
redirected URL.

If the \var{url} uses the \file{http:} scheme identifier, the optional
\var{data} argument may be given to specify a \code{POST} request
(normally the request type is \code{GET}).  The \var{data} argument
must in standard \file{application/x-www-form-urlencoded} format;
see the \function{urlencode()} function below.

The \function{urlopen()} function works transparently with proxies.
In a \UNIX{} or Windows environment, set the \envvar{http_proxy},
\envvar{ftp_proxy} or \envvar{gopher_proxy} environment variables to a
URL that identifies the proxy server before starting the Python
interpreter.  For example (the \character{\%} is the command prompt):

\begin{verbatim}
% http_proxy="http://www.someproxy.com:3128"
% export http_proxy
% python
...
\end{verbatim}

In a Macintosh environment, \function{urlopen()} will retrieve proxy
information from Internet\index{Internet Config} Config.

The \function{urlopen()} function works transparently with proxies.
In a \UNIX{} or Windows environment, set the \envvar{http_proxy},
\envvar{ftp_proxy} or \envvar{gopher_proxy} environment variables to a
URL that identifies the proxy server before starting the Python
interpreter, e.g.:

\begin{verbatim}
% http_proxy="http://www.someproxy.com:3128"
% export http_proxy
% python
...
\end{verbatim}

In a Macintosh environment, \function{urlopen()} will retrieve proxy
information from Internet Config.
\end{funcdesc}

\begin{funcdesc}{urlretrieve}{url\optional{, filename\optional{, hook}}}
Copy a network object denoted by a URL to a local file, if necessary.
If the URL points to a local file, or a valid cached copy of the
object exists, the object is not copied.  Return a tuple
\code{(\var{filename}, \var{headers})} where \var{filename} is the
local file name under which the object can be found, and \var{headers}
is either \code{None} (for a local object) or whatever the
\method{info()} method of the object returned by \function{urlopen()}
returned (for a remote object, possibly cached).  Exceptions are the
same as for \function{urlopen()}.

The second argument, if present, specifies the file location to copy
to (if absent, the location will be a tempfile with a generated name).
The third argument, if present, is a hook function that will be called
once on establishment of the network connection and once after each
block read thereafter.  The hook will be passed three arguments; a
count of blocks transferred so far, a block size in bytes, and the
total size of the file.  The third argument may be \code{-1} on older
FTP servers which do not return a file size in response to a retrieval 
request.

If the \var{url} uses the \file{http:} scheme identifier, the optional
\var{data} argument may be given to specify a \code{POST} request
(normally the request type is \code{GET}).  The \var{data} argument
must in standard \file{application/x-www-form-urlencoded} format;
see the \function{urlencode()} function below.
\end{funcdesc}

\begin{funcdesc}{urlcleanup}{}
Clear the cache that may have been built up by previous calls to
\function{urlretrieve()}.
\end{funcdesc}

\begin{funcdesc}{quote}{string\optional{, safe}}
Replace special characters in \var{string} using the \samp{\%xx} escape.
Letters, digits, and the characters \character{_,.-} are never quoted.
The optional \var{safe} parameter specifies additional characters
that should not be quoted --- its default value is \code{'/'}.

Example: \code{quote('/\~{}connolly/')} yields \code{'/\%7econnolly/'}.
\end{funcdesc}

\begin{funcdesc}{quote_plus}{string\optional{, safe}}
Like \function{quote()}, but also replaces spaces by plus signs, as
required for quoting HTML form values.  Plus signs in the original
string are escaped unless they are included in \var{safe}.
\end{funcdesc}

\begin{funcdesc}{unquote}{string}
Replace \samp{\%xx} escapes by their single-character equivalent.

Example: \code{unquote('/\%7Econnolly/')} yields \code{'/\~{}connolly/'}.
\end{funcdesc}

\begin{funcdesc}{unquote_plus}{string}
Like \function{unquote()}, but also replaces plus signs by spaces, as
required for unquoting HTML form values.
\end{funcdesc}

\begin{funcdesc}{urlencode}{dict}
Convert a dictionary to a ``url-encoded'' string, suitable to pass to
\function{urlopen()} above as the optional \var{data} argument.  This
is useful to pass a dictionary of form fields to a \code{POST}
request.  The resulting string is a series of
\code{\var{key}=\var{value}} pairs separated by \character{\&}
characters, where both \var{key} and \var{value} are quoted using
\function{quote_plus()} above.
\end{funcdesc}

The public functions \function{urlopen()} and \function{urlretrieve()}
create an instance of the \class{FancyURLopener} class and use it to perform
their requested actions.  To override this functionality, programmers can
create a subclass of \class{URLopener} or \class{FancyURLopener}, then
assign that class to the \var{urllib._urlopener} variable before calling the
desired function.  For example, applications may want to specify a different
\code{user-agent} header than \class{URLopener} defines.  This can be
accomplished with the following code:

\begin{verbatim}
class AppURLopener(urllib.FancyURLopener):
    def __init__(self, *args):
        apply(urllib.FancyURLopener.__init__, (self,) + args)
        self.version = "App/1.7"

urllib._urlopener = AppURLopener()
\end{verbatim}

\begin{classdesc}{URLopener}{\optional{proxies\optional{, **x509}}}
Base class for opening and reading URLs.  Unless you need to support
opening objects using schemes other than \file{http:}, \file{ftp:},
\file{gopher:} or \file{file:}, you probably want to use
\class{FancyURLopener}.

By default, the \class{URLopener} class sends a
\code{user-agent} header of \samp{urllib/\var{VVV}}, where
\var{VVV} is the \module{urllib} version number.  Applications can
define their own \code{user-agent} header by subclassing
\class{URLopener} or \class{FancyURLopener} and setting the instance
attribute \var{version} to an appropriate string value before the
\method{open()} method is called.

Additional keyword parameters, collected in \var{x509}, are used for
authentication with the \file{https:} scheme.  The keywords
\var{key_file} and \var{cert_file} are supported; both are needed to
actually retrieve a resource at an \file{https:} URL.
\end{classdesc}

\begin{classdesc}{FancyURLopener}{...}
\class{FancyURLopener} subclasses \class{URLopener} providing default
handling for the following HTTP response codes: 301, 302 or 401.  For
301 and 302 response codes, the \code{location} header is used to
fetch the actual URL.  For 401 response codes (authentication
required), basic HTTP authentication is performed.

The parameters to the constructor are the same as those for
\class{URLopener}.
\end{classdesc}

Restrictions:

\begin{itemize}

\item
Currently, only the following protocols are supported: HTTP, (versions
0.9 and 1.0), Gopher (but not Gopher-+), FTP, and local files.
\indexii{HTTP}{protocol}
\indexii{Gopher}{protocol}
\indexii{FTP}{protocol}

\item
The caching feature of \function{urlretrieve()} has been disabled
until I find the time to hack proper processing of Expiration time
headers.

\item
There should be a function to query whether a particular URL is in
the cache.

\item
For backward compatibility, if a URL appears to point to a local file
but the file can't be opened, the URL is re-interpreted using the FTP
protocol.  This can sometimes cause confusing error messages.

\item
The \function{urlopen()} and \function{urlretrieve()} functions can
cause arbitrarily long delays while waiting for a network connection
to be set up.  This means that it is difficult to build an interactive
web client using these functions without using threads.

\item
The data returned by \function{urlopen()} or \function{urlretrieve()}
is the raw data returned by the server.  This may be binary data
(e.g. an image), plain text or (for example) HTML\index{HTML}.  The
HTTP\indexii{HTTP}{protocol} protocol provides type information in the
reply header, which can be inspected by looking at the
\code{content-type} header.  For the Gopher\indexii{Gopher}{protocol}
protocol, type information is encoded in the URL; there is currently
no easy way to extract it.  If the returned data is HTML, you can use
the module \refmodule{htmllib}\refstmodindex{htmllib} to parse it.

\item
Although the \module{urllib} module contains (undocumented) routines
to parse and unparse URL strings, the recommended interface for URL
manipulation is in module \refmodule{urlparse}\refstmodindex{urlparse}.

\end{itemize}


\subsection{URLopener Objects \label{urlopener-objs}}
\sectionauthor{Skip Montanaro}{skip@mojam.com}

\class{URLopener} and \class{FancyURLopener} objects have the
following methods.

\begin{methoddesc}{open}{fullurl\optional{, data}}
Open \var{fullurl} using the appropriate protocol.  This method sets 
up cache and proxy information, then calls the appropriate open method with
its input arguments.  If the scheme is not recognized,
\method{open_unknown()} is called.  The \var{data} argument 
has the same meaning as the \var{data} argument of \function{urlopen()}.
\end{methoddesc}

\begin{methoddesc}{open_unknown}{fullurl\optional{, data}}
Overridable interface to open unknown URL types.
\end{methoddesc}

\begin{methoddesc}{retrieve}{url\optional{, filename\optional{, reporthook}}}
Retrieves the contents of \var{url} and places it in \var{filename}.  The
return value is a tuple consisting of a local filename and either a
\class{mimetools.Message} object containing the response headers (for remote
URLs) or None (for local URLs).  The caller must then open and read the
contents of \var{filename}.  If \var{filename} is not given and the URL
refers to a local file, the input filename is returned.  If the URL is
non-local and \var{filename} is not given, the filename is the output of
\function{tempfile.mktemp()} with a suffix that matches the suffix of the last
path component of the input URL.  If \var{reporthook} is given, it must be
a function accepting three numeric parameters.  It will be called after each
chunk of data is read from the network.  \var{reporthook} is ignored for
local URLs.

If the \var{url} uses the \file{http:} scheme identifier, the optional
\var{data} argument may be given to specify a \code{POST} request
(normally the request type is \code{GET}).  The \var{data} argument
must in standard \file{application/x-www-form-urlencoded} format;
see the \function{urlencode()} function below.
\end{methoddesc}


\subsection{Examples}
\nodename{Urllib Examples}

Here is an example session that uses the \samp{GET} method to retrieve
a URL containing parameters:

\begin{verbatim}
>>> import urllib
>>> params = urllib.urlencode({'spam': 1, 'eggs': 2, 'bacon': 0})
>>> f = urllib.urlopen("http://www.musi-cal.com/cgi-bin/query?%s" % params)
>>> print f.read()
\end{verbatim}

The following example uses the \samp{POST} method instead:

\begin{verbatim}
>>> import urllib
>>> params = urllib.urlencode({'spam': 1, 'eggs': 2, 'bacon': 0})
>>> f = urllib.urlopen("http://www.musi-cal.com/cgi-bin/query", params)
>>> print f.read()
\end{verbatim}

\section{\module{httplib} ---
         HTTP protocol client}

\declaremodule{standard}{httplib}
\modulesynopsis{HTTP and HTTPS protocol client (requires sockets).}

\indexii{HTTP}{protocol}
\index{HTTP!\module{httplib} (standard module)}

This module defines classes which implement the client side of the
HTTP and HTTPS protocols.  It is normally not used directly --- the
module \refmodule{urllib}\refstmodindex{urllib} uses it to handle URLs
that use HTTP and HTTPS.

\begin{notice}
  HTTPS support is only available if the \refmodule{socket} module was
  compiled with SSL support.
\end{notice}

\begin{notice}
  The public interface for this module changed substantially in Python
  2.0.  The \class{HTTP} class is retained only for backward
  compatibility with 1.5.2.  It should not be used in new code.  Refer
  to the online docstrings for usage.
\end{notice}

The module provides the following classes:

\begin{classdesc}{HTTPConnection}{host\optional{, port}}
An \class{HTTPConnection} instance represents one transaction with an HTTP
server.  It should be instantiated passing it a host and optional port number.
If no port number is passed, the port is extracted from the host string if it
has the form \code{\var{host}:\var{port}}, else the default HTTP port (80) is
used.  For example, the following calls all create instances that connect to
the server at the same host and port:

\begin{verbatim}
>>> h1 = httplib.HTTPConnection('www.cwi.nl')
>>> h2 = httplib.HTTPConnection('www.cwi.nl:80')
>>> h3 = httplib.HTTPConnection('www.cwi.nl', 80)
\end{verbatim}
\versionadded{2.0}
\end{classdesc}

\begin{classdesc}{HTTPSConnection}{host\optional{, port, key_file, cert_file}}
A subclass of \class{HTTPConnection} that uses SSL for communication with
secure servers.  Default port is \code{443}.
\var{key_file} is
the name of a PEM formatted file that contains your private
key. \var{cert_file} is a PEM formatted certificate chain file.

\warning{This does not do any certificate verification!}

\versionadded{2.0}
\end{classdesc}

\begin{classdesc}{HTTPResponse}{sock\optional{, debuglevel=0}\optional{, strict=0}}
Class whose instances are returned upon successful connection.  Not
instantiated directly by user.
\versionadded{2.0}
\end{classdesc}

The following exceptions are raised as appropriate:

\begin{excdesc}{HTTPException}
The base class of the other exceptions in this module.  It is a
subclass of \exception{Exception}.
\versionadded{2.0}
\end{excdesc}

\begin{excdesc}{NotConnected}
A subclass of \exception{HTTPException}.
\versionadded{2.0}
\end{excdesc}

\begin{excdesc}{InvalidURL}
A subclass of \exception{HTTPException}, raised if a port is given and is
either non-numeric or empty.
\versionadded{2.3}
\end{excdesc}

\begin{excdesc}{UnknownProtocol}
A subclass of \exception{HTTPException}.
\versionadded{2.0}
\end{excdesc}

\begin{excdesc}{UnknownTransferEncoding}
A subclass of \exception{HTTPException}.
\versionadded{2.0}
\end{excdesc}

\begin{excdesc}{UnimplementedFileMode}
A subclass of \exception{HTTPException}.
\versionadded{2.0}
\end{excdesc}

\begin{excdesc}{IncompleteRead}
A subclass of \exception{HTTPException}.
\versionadded{2.0}
\end{excdesc}

\begin{excdesc}{ImproperConnectionState}
A subclass of \exception{HTTPException}.
\versionadded{2.0}
\end{excdesc}

\begin{excdesc}{CannotSendRequest}
A subclass of \exception{ImproperConnectionState}.
\versionadded{2.0}
\end{excdesc}

\begin{excdesc}{CannotSendHeader}
A subclass of \exception{ImproperConnectionState}.
\versionadded{2.0}
\end{excdesc}

\begin{excdesc}{ResponseNotReady}
A subclass of \exception{ImproperConnectionState}.
\versionadded{2.0}
\end{excdesc}

\begin{excdesc}{BadStatusLine}
A subclass of \exception{HTTPException}.  Raised if a server responds with a
HTTP status code that we don't understand.
\versionadded{2.0}
\end{excdesc}

The constants defined in this module are:

\begin{datadesc}{HTTP_PORT}
  The default port for the HTTP protocol (always \code{80}).
\end{datadesc}

\begin{datadesc}{HTTPS_PORT}
  The default port for the HTTPS protocol (always \code{443}).
\end{datadesc}

and also the following constants for integer status codes:

\begin{tableiii}{l|c|l}{constant}{Constant}{Value}{Definition}
  \lineiii{CONTINUE}{\code{100}}
    {HTTP/1.1, \ulink{RFC 2616, Section 10.1.1}
      {http://www.w3.org/Protocols/rfc2616/rfc2616-sec10.html#sec10.1.1}}
  \lineiii{SWITCHING_PROTOCOLS}{\code{101}}
    {HTTP/1.1, \ulink{RFC 2616, Section 10.1.2}
      {http://www.w3.org/Protocols/rfc2616/rfc2616-sec10.html#sec10.1.2}}
  \lineiii{PROCESSING}{\code{102}}
    {WEBDAV, \ulink{RFC 2518, Section 10.1}
      {http://www.webdav.org/specs/rfc2518.html#STATUS_102}}

  \lineiii{OK}{\code{200}}
    {HTTP/1.1, \ulink{RFC 2616, Section 10.2.1}
      {http://www.w3.org/Protocols/rfc2616/rfc2616-sec10.html#sec10.2.1}}
  \lineiii{CREATED}{\code{201}}
    {HTTP/1.1, \ulink{RFC 2616, Section 10.2.2}
      {http://www.w3.org/Protocols/rfc2616/rfc2616-sec10.html#sec10.2.2}}
  \lineiii{ACCEPTED}{\code{202}}
    {HTTP/1.1, \ulink{RFC 2616, Section 10.2.3}
      {http://www.w3.org/Protocols/rfc2616/rfc2616-sec10.html#sec10.2.3}}
  \lineiii{NON_AUTHORITATIVE_INFORMATION}{\code{203}}
    {HTTP/1.1, \ulink{RFC 2616, Section 10.2.4}
      {http://www.w3.org/Protocols/rfc2616/rfc2616-sec10.html#sec10.2.4}}
  \lineiii{NO_CONTENT}{\code{204}}
    {HTTP/1.1, \ulink{RFC 2616, Section 10.2.5}
      {http://www.w3.org/Protocols/rfc2616/rfc2616-sec10.html#sec10.2.5}}
  \lineiii{RESET_CONTENT}{\code{205}}
    {HTTP/1.1, \ulink{RFC 2616, Section 10.2.6}
      {http://www.w3.org/Protocols/rfc2616/rfc2616-sec10.html#sec10.2.6}}
  \lineiii{PARTIAL_CONTENT}{\code{206}}
    {HTTP/1.1, \ulink{RFC 2616, Section 10.2.7}
      {http://www.w3.org/Protocols/rfc2616/rfc2616-sec10.html#sec10.2.7}}
  \lineiii{MULTI_STATUS}{\code{207}}
    {WEBDAV \ulink{RFC 2518, Section 10.2}
      {http://www.webdav.org/specs/rfc2518.html#STATUS_207}}
  \lineiii{IM_USED}{\code{226}}
    {Delta encoding in HTTP, \rfc{3229}, Section 10.4.1}

  \lineiii{MULTIPLE_CHOICES}{\code{300}}
    {HTTP/1.1, \ulink{RFC 2616, Section 10.3.1}
      {http://www.w3.org/Protocols/rfc2616/rfc2616-sec10.html#sec10.3.1}}
  \lineiii{MOVED_PERMANENTLY}{\code{301}}
    {HTTP/1.1, \ulink{RFC 2616, Section 10.3.2}
      {http://www.w3.org/Protocols/rfc2616/rfc2616-sec10.html#sec10.3.2}}
  \lineiii{FOUND}{\code{302}}
    {HTTP/1.1, \ulink{RFC 2616, Section 10.3.3}
      {http://www.w3.org/Protocols/rfc2616/rfc2616-sec10.html#sec10.3.3}}
  \lineiii{SEE_OTHER}{\code{303}}
    {HTTP/1.1, \ulink{RFC 2616, Section 10.3.4}
      {http://www.w3.org/Protocols/rfc2616/rfc2616-sec10.html#sec10.3.4}}
  \lineiii{NOT_MODIFIED}{\code{304}}
    {HTTP/1.1, \ulink{RFC 2616, Section 10.3.5}
      {http://www.w3.org/Protocols/rfc2616/rfc2616-sec10.html#sec10.3.5}}
  \lineiii{USE_PROXY}{\code{305}}
    {HTTP/1.1, \ulink{RFC 2616, Section 10.3.6}
      {http://www.w3.org/Protocols/rfc2616/rfc2616-sec10.html#sec10.3.6}}
  \lineiii{TEMPORARY_REDIRECT}{\code{307}}
    {HTTP/1.1, \ulink{RFC 2616, Section 10.3.8}
      {http://www.w3.org/Protocols/rfc2616/rfc2616-sec10.html#sec10.3.8}}

  \lineiii{BAD_REQUEST}{\code{400}}
    {HTTP/1.1, \ulink{RFC 2616, Section 10.4.1}
      {http://www.w3.org/Protocols/rfc2616/rfc2616-sec10.html#sec10.4.1}}
  \lineiii{UNAUTHORIZED}{\code{401}}
    {HTTP/1.1, \ulink{RFC 2616, Section 10.4.2}
      {http://www.w3.org/Protocols/rfc2616/rfc2616-sec10.html#sec10.4.2}}
  \lineiii{PAYMENT_REQUIRED}{\code{402}}
    {HTTP/1.1, \ulink{RFC 2616, Section 10.4.3}
      {http://www.w3.org/Protocols/rfc2616/rfc2616-sec10.html#sec10.4.3}}
  \lineiii{FORBIDDEN}{\code{403}}
    {HTTP/1.1, \ulink{RFC 2616, Section 10.4.4}
      {http://www.w3.org/Protocols/rfc2616/rfc2616-sec10.html#sec10.4.4}}
  \lineiii{NOT_FOUND}{\code{404}}
    {HTTP/1.1, \ulink{RFC 2616, Section 10.4.5}
      {http://www.w3.org/Protocols/rfc2616/rfc2616-sec10.html#sec10.4.5}}
  \lineiii{METHOD_NOT_ALLOWED}{\code{405}}
    {HTTP/1.1, \ulink{RFC 2616, Section 10.4.6}
      {http://www.w3.org/Protocols/rfc2616/rfc2616-sec10.html#sec10.4.6}}
  \lineiii{NOT_ACCEPTABLE}{\code{406}}
    {HTTP/1.1, \ulink{RFC 2616, Section 10.4.7}
      {http://www.w3.org/Protocols/rfc2616/rfc2616-sec10.html#sec10.4.7}}
  \lineiii{PROXY_AUTHENTICATION_REQUIRED}
    {\code{407}}{HTTP/1.1, \ulink{RFC 2616, Section 10.4.8}
      {http://www.w3.org/Protocols/rfc2616/rfc2616-sec10.html#sec10.4.8}}
  \lineiii{REQUEST_TIMEOUT}{\code{408}}
    {HTTP/1.1, \ulink{RFC 2616, Section 10.4.9}
      {http://www.w3.org/Protocols/rfc2616/rfc2616-sec10.html#sec10.4.9}}
  \lineiii{CONFLICT}{\code{409}}
    {HTTP/1.1, \ulink{RFC 2616, Section 10.4.10}
      {http://www.w3.org/Protocols/rfc2616/rfc2616-sec10.html#sec10.4.10}}
  \lineiii{GONE}{\code{410}}
    {HTTP/1.1, \ulink{RFC 2616, Section 10.4.11}
      {http://www.w3.org/Protocols/rfc2616/rfc2616-sec10.html#sec10.4.11}}
  \lineiii{LENGTH_REQUIRED}{\code{411}}
    {HTTP/1.1, \ulink{RFC 2616, Section 10.4.12}
      {http://www.w3.org/Protocols/rfc2616/rfc2616-sec10.html#sec10.4.12}}
  \lineiii{PRECONDITION_FAILED}{\code{412}}
    {HTTP/1.1, \ulink{RFC 2616, Section 10.4.13}
      {http://www.w3.org/Protocols/rfc2616/rfc2616-sec10.html#sec10.4.13}}
  \lineiii{REQUEST_ENTITY_TOO_LARGE}
    {\code{413}}{HTTP/1.1, \ulink{RFC 2616, Section 10.4.14}
      {http://www.w3.org/Protocols/rfc2616/rfc2616-sec10.html#sec10.4.14}}
  \lineiii{REQUEST_URI_TOO_LONG}{\code{414}}
    {HTTP/1.1, \ulink{RFC 2616, Section 10.4.15}
      {http://www.w3.org/Protocols/rfc2616/rfc2616-sec10.html#sec10.4.15}}
  \lineiii{UNSUPPORTED_MEDIA_TYPE}{\code{415}}
    {HTTP/1.1, \ulink{RFC 2616, Section 10.4.16}
      {http://www.w3.org/Protocols/rfc2616/rfc2616-sec10.html#sec10.4.16}}
  \lineiii{REQUESTED_RANGE_NOT_SATISFIABLE}{\code{416}}
    {HTTP/1.1, \ulink{RFC 2616, Section 10.4.17}
      {http://www.w3.org/Protocols/rfc2616/rfc2616-sec10.html#sec10.4.17}}
  \lineiii{EXPECTATION_FAILED}{\code{417}}
    {HTTP/1.1, \ulink{RFC 2616, Section 10.4.18}
      {http://www.w3.org/Protocols/rfc2616/rfc2616-sec10.html#sec10.4.18}}
  \lineiii{UNPROCESSABLE_ENTITY}{\code{422}}
    {WEBDAV, \ulink{RFC 2518, Section 10.3}
      {http://www.webdav.org/specs/rfc2518.html#STATUS_422}}
  \lineiii{LOCKED}{\code{423}}
    {WEBDAV \ulink{RFC 2518, Section 10.4}
      {http://www.webdav.org/specs/rfc2518.html#STATUS_423}}
  \lineiii{FAILED_DEPENDENCY}{\code{424}}
    {WEBDAV, \ulink{RFC 2518, Section 10.5}
      {http://www.webdav.org/specs/rfc2518.html#STATUS_424}}
  \lineiii{UPGRADE_REQUIRED}{\code{426}}
    {HTTP Upgrade to TLS, \rfc{2817}, Section 6}

  \lineiii{INTERNAL_SERVER_ERROR}{\code{500}}
    {HTTP/1.1, \ulink{RFC 2616, Section 10.5.1}
      {http://www.w3.org/Protocols/rfc2616/rfc2616-sec10.html#sec10.5.1}}
  \lineiii{NOT_IMPLEMENTED}{\code{501}}
    {HTTP/1.1, \ulink{RFC 2616, Section 10.5.2}
      {http://www.w3.org/Protocols/rfc2616/rfc2616-sec10.html#sec10.5.2}}
  \lineiii{BAD_GATEWAY}{\code{502}}
    {HTTP/1.1 \ulink{RFC 2616, Section 10.5.3}
      {http://www.w3.org/Protocols/rfc2616/rfc2616-sec10.html#sec10.5.3}}
  \lineiii{SERVICE_UNAVAILABLE}{\code{503}}
    {HTTP/1.1, \ulink{RFC 2616, Section 10.5.4}
      {http://www.w3.org/Protocols/rfc2616/rfc2616-sec10.html#sec10.5.4}}
  \lineiii{GATEWAY_TIMEOUT}{\code{504}}
    {HTTP/1.1 \ulink{RFC 2616, Section 10.5.5}
      {http://www.w3.org/Protocols/rfc2616/rfc2616-sec10.html#sec10.5.5}}
  \lineiii{HTTP_VERSION_NOT_SUPPORTED}{\code{505}}
    {HTTP/1.1, \ulink{RFC 2616, Section 10.5.6}
      {http://www.w3.org/Protocols/rfc2616/rfc2616-sec10.html#sec10.5.6}}
  \lineiii{INSUFFICIENT_STORAGE}{\code{507}}
    {WEBDAV, \ulink{RFC 2518, Section 10.6}
      {http://www.webdav.org/specs/rfc2518.html#STATUS_507}}
  \lineiii{NOT_EXTENDED}{\code{510}}
    {An HTTP Extension Framework, \rfc{2774}, Section 7}
\end{tableiii}

\subsection{HTTPConnection Objects \label{httpconnection-objects}}

\class{HTTPConnection} instances have the following methods:

\begin{methoddesc}{request}{method, url\optional{, body\optional{, headers}}}
This will send a request to the server using the HTTP request method
\var{method} and the selector \var{url}.  If the \var{body} argument is
present, it should be a string of data to send after the headers are finished.
The header Content-Length is automatically set to the correct value.
The \var{headers} argument should be a mapping of extra HTTP headers to send
with the request.
\end{methoddesc}

\begin{methoddesc}{getresponse}{}
Should be called after a request is sent to get the response from the server.
Returns an \class{HTTPResponse} instance.
\note{Note that you must have read the whole response before you can send a new
request to the server.}
\end{methoddesc}

\begin{methoddesc}{set_debuglevel}{level}
Set the debugging level (the amount of debugging output printed).
The default debug level is \code{0}, meaning no debugging output is
printed.
\end{methoddesc}

\begin{methoddesc}{connect}{}
Connect to the server specified when the object was created.
\end{methoddesc}

\begin{methoddesc}{close}{}
Close the connection to the server.
\end{methoddesc}

As an alternative to using the \method{request()} method described above,
you can also send your request step by step, by using the four functions
below.

\begin{methoddesc}{putrequest}{request, selector\optional{,
skip\_host\optional{, skip_accept_encoding}}}
This should be the first call after the connection to the server has
been made.  It sends a line to the server consisting of the
\var{request} string, the \var{selector} string, and the HTTP version
(\code{HTTP/1.1}).  To disable automatic sending of \code{Host:} or
\code{Accept-Encoding:} headers (for example to accept additional
content encodings), specify \var{skip_host} or \var{skip_accept_encoding}
with non-False values.
\versionchanged[\var{skip_accept_encoding} argument added]{2.4}
\end{methoddesc}

\begin{methoddesc}{putheader}{header, argument\optional{, ...}}
Send an \rfc{822}-style header to the server.  It sends a line to the
server consisting of the header, a colon and a space, and the first
argument.  If more arguments are given, continuation lines are sent,
each consisting of a tab and an argument.
\end{methoddesc}

\begin{methoddesc}{endheaders}{}
Send a blank line to the server, signalling the end of the headers.
\end{methoddesc}

\begin{methoddesc}{send}{data}
Send data to the server.  This should be used directly only after the
\method{endheaders()} method has been called and before
\method{getresponse()} is called.
\end{methoddesc}

\subsection{HTTPResponse Objects \label{httpresponse-objects}}

\class{HTTPResponse} instances have the following methods and attributes:

\begin{methoddesc}{read}{\optional{amt}}
Reads and returns the response body, or up to the next \var{amt} bytes.
\end{methoddesc}

\begin{methoddesc}{getheader}{name\optional{, default}}
Get the contents of the header \var{name}, or \var{default} if there is no
matching header.
\end{methoddesc}

\begin{methoddesc}{getheaders}{}
Return a list of (header, value) tuples. \versionadded{2.4}
\end{methoddesc}

\begin{datadesc}{msg}
  A \class{mimetools.Message} instance containing the response headers.
\end{datadesc}

\begin{datadesc}{version}
  HTTP protocol version used by server.  10 for HTTP/1.0, 11 for HTTP/1.1.
\end{datadesc}

\begin{datadesc}{status}
  Status code returned by server.
\end{datadesc}

\begin{datadesc}{reason}
  Reason phrase returned by server.
\end{datadesc}


\subsection{Examples \label{httplib-examples}}

Here is an example session that uses the \samp{GET} method:

\begin{verbatim}
>>> import httplib
>>> conn = httplib.HTTPConnection("www.python.org")
>>> conn.request("GET", "/index.html")
>>> r1 = conn.getresponse()
>>> print r1.status, r1.reason
200 OK
>>> data1 = r1.read()
>>> conn.request("GET", "/parrot.spam")
>>> r2 = conn.getresponse()
>>> print r2.status, r2.reason
404 Not Found
>>> data2 = r2.read()
>>> conn.close()
\end{verbatim}

Here is an example session that shows how to \samp{POST} requests:

\begin{verbatim}
>>> import httplib, urllib
>>> params = urllib.urlencode({'spam': 1, 'eggs': 2, 'bacon': 0})
>>> headers = {"Content-type": "application/x-www-form-urlencoded",
...            "Accept": "text/plain"}
>>> conn = httplib.HTTPConnection("musi-cal.mojam.com:80")
>>> conn.request("POST", "/cgi-bin/query", params, headers)
>>> response = conn.getresponse()
>>> print response.status, response.reason
200 OK
>>> data = response.read()
>>> conn.close()
\end{verbatim}

\section{Standard Module \module{ftplib}}
\label{module-ftplib}
\stmodindex{ftplib}
\indexii{FTP}{protocol}


This module defines the class \class{FTP} and a few related items.
The \class{FTP} class implements the client side of the FTP protocol.
You can use this to write Python programs that perform a variety of
automated FTP jobs, such as mirroring other ftp servers.  It is also
used by the module \module{urllib} to handle URLs that use FTP.  For
more information on FTP (File Transfer Protocol), see Internet
\rfc{959}.

Here's a sample session using the \module{ftplib} module:

\begin{verbatim}
>>> from ftplib import FTP
>>> ftp = FTP('ftp.cwi.nl')   # connect to host, default port
>>> ftp.login()               # user anonymous, passwd user@hostname
>>> ftp.retrlines('LIST')     # list directory contents
total 24418
drwxrwsr-x   5 ftp-usr  pdmaint     1536 Mar 20 09:48 .
dr-xr-srwt 105 ftp-usr  pdmaint     1536 Mar 21 14:32 ..
-rw-r--r--   1 ftp-usr  pdmaint     5305 Mar 20 09:48 INDEX
 .
 .
 .
>>> ftp.quit()
\end{verbatim}

The module defines the following items:

\begin{classdesc}{FTP}{\optional{host\optional{, user\optional{,
                       passwd\optional{, acct}}}}}
Return a new instance of the \class{FTP} class.  When
\var{host} is given, the method call \code{connect(\var{host})} is
made.  When \var{user} is given, additionally the method call
\code{login(\var{user}, \var{passwd}, \var{acct})} is made (where
\var{passwd} and \var{acct} default to the empty string when not given).
\end{classdesc}

\begin{datadesc}{all_errors}
The set of all exceptions (as a tuple) that methods of \class{FTP}
instances may raise as a result of problems with the FTP connection
(as opposed to programming errors made by the caller).  This set
includes the four exceptions listed below as well as
\exception{socket.error} and \exception{IOError}.
\end{datadesc}

\begin{excdesc}{error_reply}
Exception raised when an unexpected reply is received from the server.
\end{excdesc}

\begin{excdesc}{error_temp}
Exception raised when an error code in the range 400--499 is received.
\end{excdesc}

\begin{excdesc}{error_perm}
Exception raised when an error code in the range 500--599 is received.
\end{excdesc}

\begin{excdesc}{error_proto}
Exception raised when a reply is received from the server that does
not begin with a digit in the range 1--5.
\end{excdesc}


\subsection{FTP Objects}
\label{ftp-objects}

\class{FTP} instances have the following methods:

\begin{methoddesc}{set_debuglevel}{level}
Set the instance's debugging level.  This controls the amount of
debugging output printed.  The default, \code{0}, produces no
debugging output.  A value of \code{1} produces a moderate amount of
debugging output, generally a single line per request.  A value of
\code{2} or higher produces the maximum amount of debugging output,
logging each line sent and received on the control connection.
\end{methoddesc}

\begin{methoddesc}{connect}{host\optional{, port}}
Connect to the given host and port.  The default port number is \code{21}, as
specified by the FTP protocol specification.  It is rarely needed to
specify a different port number.  This function should be called only
once for each instance; it should not be called at all if a host was
given when the instance was created.  All other methods can only be
used after a connection has been made.
\end{methoddesc}

\begin{methoddesc}{getwelcome}{}
Return the welcome message sent by the server in reply to the initial
connection.  (This message sometimes contains disclaimers or help
information that may be relevant to the user.)
\end{methoddesc}

\begin{methoddesc}{login}{\optional{user\optional{, passwd\optional{, acct}}}}
Log in as the given \var{user}.  The \var{passwd} and \var{acct}
parameters are optional and default to the empty string.  If no
\var{user} is specified, it defaults to \code{'anonymous'}.  If
\var{user} is \code{anonymous}, the default \var{passwd} is
\samp{\var{realuser}@\var{host}} where \var{realuser} is the real user
name (glanced from the \envvar{LOGNAME} or \envvar{USER} environment
variable) and \var{host} is the hostname as returned by
\function{socket.gethostname()}.  This function should be called only
once for each instance, after a connection has been established; it
should not be called at all if a host and user were given when the
instance was created.  Most FTP commands are only allowed after the
client has logged in.
\end{methoddesc}

\begin{methoddesc}{abort}{}
Abort a file transfer that is in progress.  Using this does not always
work, but it's worth a try.
\end{methoddesc}

\begin{methoddesc}{sendcmd}{command}
Send a simple command string to the server and return the response
string.
\end{methoddesc}

\begin{methoddesc}{voidcmd}{command}
Send a simple command string to the server and handle the response.
Return nothing if a response code in the range 200--299 is received.
Raise an exception otherwise.
\end{methoddesc}

\begin{methoddesc}{retrbinary}{command, callback\optional{, maxblocksize}}
Retrieve a file in binary transfer mode.  \var{command} should be an
appropriate \samp{RETR} command, i.e.\ \code{'RETR \var{filename}'}.
The \var{callback} function is called for each block of data received,
with a single string argument giving the data block.
The optional \var{maxblocksize} argument specifies the maximum chunk size to
read on the low-level socket object created to do the actual transfer
(which will also be the largest size of the data blocks passed to
\var{callback}).  A reasonable default is chosen.
\end{methoddesc}

\begin{methoddesc}{retrlines}{command\optional{, callback}}
Retrieve a file or directory listing in \ASCII{} transfer mode.
\var{command} should be an appropriate \samp{RETR} command (see
\method{retrbinary()} or a \samp{LIST} command (usually just the string
\code{'LIST'}).  The \var{callback} function is called for each line,
with the trailing CRLF stripped.  The default \var{callback} prints
the line to \code{sys.stdout}.
\end{methoddesc}

\begin{methoddesc}{storbinary}{command, file, blocksize}
Store a file in binary transfer mode.  \var{command} should be an
appropriate \samp{STOR} command, i.e.\ \code{"STOR \var{filename}"}.
\var{file} is an open file object which is read until \EOF{} using its
\method{read()} method in blocks of size \var{blocksize} to provide the
data to be stored.
\end{methoddesc}

\begin{methoddesc}{storlines}{command, file}
Store a file in \ASCII{} transfer mode.  \var{command} should be an
appropriate \samp{STOR} command (see \method{storbinary()}).  Lines are
read until \EOF{} from the open file object \var{file} using its
\method{readline()} method to privide the data to be stored.
\end{methoddesc}

\begin{methoddesc}{transfercmd}{cmd}
Initiate a transfer over the data connection.  If the transfer is
active, send a \samp{PORT} command and the transfer command specified
by \var{cmd}, and accept the connection.  If the server is passive,
send a \samp{PASV} command, connect to it, and start the transfer
command.  Either way, return the socket for the connection.
\end{methoddesc}

\begin{methoddesc}{ntransfercmd}{cmd}
Like \method{transfercmd()}, but returns a tuple of the data
connection and the expected size of the data.  If the expected size
could not be computed, \code{None} will be returned as the expected
size.
\end{methoddesc}

\begin{methoddesc}{nlst}{argument\optional{, \ldots}}
Return a list of files as returned by the \samp{NLST} command.  The
optional \var{argument} is a directory to list (default is the current
server directory).  Multiple arguments can be used to pass
non-standard options to the \samp{NLST} command.
\end{methoddesc}

\begin{methoddesc}{dir}{argument\optional{, \ldots}}
Return a directory listing as returned by the \samp{LIST} command, as
a list of lines.  The optional \var{argument} is a directory to list
(default is the current server directory).  Multiple arguments can be
used to pass non-standard options to the \samp{LIST} command.  If the
last argument is a function, it is used as a \var{callback} function
as for \method{retrlines()}.
\end{methoddesc}

\begin{methoddesc}{rename}{fromname, toname}
Rename file \var{fromname} on the server to \var{toname}.
\end{methoddesc}

\begin{methoddesc}{delete}{filename}
Remove the file named \var{filename} from the server.  If successful,
returns the text of the response, otherwise raises
\exception{error_perm} on permission errors or \exception{error_reply} 
on other errors.
\end{methoddesc}

\begin{methoddesc}{cwd}{pathname}
Set the current directory on the server.
\end{methoddesc}

\begin{methoddesc}{mkd}{pathname}
Create a new directory on the server.
\end{methoddesc}

\begin{methoddesc}{pwd}{}
Return the pathname of the current directory on the server.
\end{methoddesc}

\begin{methoddesc}{rmd}{dirname}
Remove the directory named \var{dirname} on the server.
\end{methoddesc}

\begin{methoddesc}{size}{filename}
Request the size of the file named \var{filename} on the server.  On
success, the size of the file is returned as an integer, otherwise
\code{None} is returned.  Note that the \samp{SIZE} command is not 
standardized, but is supported by many common server implementations.
\end{methoddesc}

\begin{methoddesc}{quit}{}
Send a \samp{QUIT} command to the server and close the connection.
This is the ``polite'' way to close a connection, but it may raise an
exception of the server reponds with an error to the \samp{QUIT}
command.
\end{methoddesc}

\begin{methoddesc}{close}{}
Close the connection unilaterally.  This should not be applied to an
already closed connection (e.g.\ after a successful call to
\method{quit()}.
\end{methoddesc}

\section{\module{gopherlib} ---
         Gopher protocol client}

\declaremodule{standard}{gopherlib}
\modulesynopsis{Gopher protocol client (requires sockets).}

\deprecated{2.5}{The \code{gopher} protocol is not in active use
                 anymore.}

\indexii{Gopher}{protocol}

This module provides a minimal implementation of client side of the
Gopher protocol.  It is used by the module \refmodule{urllib} to
handle URLs that use the Gopher protocol.

The module defines the following functions:

\begin{funcdesc}{send_selector}{selector, host\optional{, port}}
Send a \var{selector} string to the gopher server at \var{host} and
\var{port} (default \code{70}).  Returns an open file object from
which the returned document can be read.
\end{funcdesc}

\begin{funcdesc}{send_query}{selector, query, host\optional{, port}}
Send a \var{selector} string and a \var{query} string to a gopher
server at \var{host} and \var{port} (default \code{70}).  Returns an
open file object from which the returned document can be read.
\end{funcdesc}

Note that the data returned by the Gopher server can be of any type,
depending on the first character of the selector string.  If the data
is text (first character of the selector is \samp{0}), lines are
terminated by CRLF, and the data is terminated by a line consisting of
a single \samp{.}, and a leading \samp{.} should be stripped from
lines that begin with \samp{..}.  Directory listings (first character
of the selector is \samp{1}) are transferred using the same protocol.

\section{\module{poplib} ---
         POP3 protocol client}

\declaremodule{standard}{poplib}
\modulesynopsis{POP3 protocol client (requires sockets).}

%By Andrew T. Csillag
%Even though I put it into LaTeX, I cannot really claim that I wrote
%it since I just stole most of it from the poplib.py source code and
%the imaplib ``chapter''.
%Revised by ESR, January 2000

\indexii{POP3}{protocol}

This module defines a class, \class{POP3}, which encapsulates a
connection to an POP3 server and implements the protocol as defined in
\rfc{1725}.  The \class{POP3} class supports both the minimal and
optional command sets.

Note that POP3, though widely supported, is obsolescent.  The
implementation quality of POP3 servers varies widely, and too many are
quite poor. If your mailserver supports IMAP, you would be better off
using the \code{\refmodule{imaplib}.\class{IMAP4}} class, as IMAP
servers tend to be better implemented.

A single class is provided by the \module{poplib} module:

\begin{classdesc}{POP3}{host\optional{, port}}
This class implements the actual POP3 protocol.  The connection is
created when the instance is initialized.
If \var{port} is omitted, the standard POP3 port (110) is used.
\end{classdesc}

One exception is defined as an attribute of the \module{poplib} module:

\begin{excdesc}{error_proto}
Exception raised on any errors.  The reason for the exception is
passed to the constructor as a string.
\end{excdesc}

\begin{seealso}
  \seemodule{imaplib}{The standard Python IMAP module.}
  \seetitle[http://www.tuxedo.org/\~{}esr/fetchmail/fetchmail-FAQ.html]
        {Frequently Asked Questions About Fetchmail}
        {The FAQ for the \program{fetchmail} POP/IMAP client collects
         information on POP3 server variations and RFC noncompliance
         that may be useful if you need to write an application based
         on the POP protocol.}
\end{seealso}


\subsection{POP3 Objects \label{pop3-objects}}

All POP3 commands are represented by methods of the same name,
in lower-case; most return the response text sent by the server.

An \class{POP3} instance has the following methods:


\begin{methoddesc}{set_debuglevel}{level}
Set the instance's debugging level.  This controls the amount of
debugging output printed.  The default, \code{0}, produces no
debugging output.  A value of \code{1} produces a moderate amount of
debugging output, generally a single line per request.  A value of
\code{2} or higher produces the maximum amount of debugging output,
logging each line sent and received on the control connection.
\end{methoddesc}

\begin{methoddesc}{getwelcome}{}
Returns the greeting string sent by the POP3 server.
\end{methoddesc}

\begin{methoddesc}{user}{username}
Send user command, response should indicate that a password is required.
\end{methoddesc}

\begin{methoddesc}{pass_}{password}
Send password, response includes message count and mailbox size.
Note: the mailbox on the server is locked until \method{quit()} is
called.
\end{methoddesc}

\begin{methoddesc}{apop}{user, secret}
Use the more secure APOP authentication to log into the POP3 server.
\end{methoddesc}

\begin{methoddesc}{rpop}{user}
Use RPOP authentication (similar to UNIX r-commands) to log into POP3 server.
\end{methoddesc}

\begin{methoddesc}{stat}{}
Get mailbox status.  The result is a tuple of 2 integers:
\code{(\var{message count}, \var{mailbox size})}.
\end{methoddesc}

\begin{methoddesc}{list}{\optional{which}}
Request message list, result is in the form
\code{(\var{response}, ['mesg_num octets', ...])}.  If \var{which} is
set, it is the message to list.
\end{methoddesc}

\begin{methoddesc}{retr}{which}
Retrieve whole message number \var{which}, and set its seen flag.
Result is in form  \code{(\var{response}, ['line', ...], \var{octets})}.
\end{methoddesc}

\begin{methoddesc}{dele}{which}
Flag message number \var{which} for deletion.  On most servers
deletions are not actually performed until QUIT (the major exception is
Eudora QPOP, which deliberately violates the RFCs by doing pending
deletes on any disconnect).
\end{methoddesc}

\begin{methoddesc}{rset}{}
Remove any deletion marks for the mailbox.
\end{methoddesc}

\begin{methoddesc}{noop}{}
Do nothing.  Might be used as a keep-alive.
\end{methoddesc}

\begin{methoddesc}{quit}{}
Signoff:  commit changes, unlock mailbox, drop connection.
\end{methoddesc}

\begin{methoddesc}{top}{which, howmuch}
Retrieves the message header plus \var{howmuch} lines of the message
after the header of message number \var{which}. Result is in form
\code{(\var{response}, ['line', ...], \var{octets})}.

The POP3 TOP command this method uses, unlike the RETR command,
doesn't set the message's seen flag; unfortunately, TOP is poorly
specified in the RFCs and is frequently broken in off-brand servers.
Test this method by hand against the POP3 servers you will use before
trusting it.
\end{methoddesc}

\begin{methoddesc}{uidl}{\optional{which}}
Return message digest (unique id) list.
If \var{which} is specified, result contains the unique id for that
message in the form \code{'\var{response}\ \var{mesgnum}\ \var{uid}},
otherwise result is list \code{(\var{response}, ['mesgnum uid', ...],
\var{octets})}.
\end{methoddesc}


\subsection{POP3 Example \label{pop3-example}}

Here is a minimal example (without error checking) that opens a
mailbox and retrieves and prints all messages:

\begin{verbatim}
import getpass, poplib

M = poplib.POP3('localhost')
M.user(getpass.getuser())
M.pass_(getpass.getpass())
numMessages = len(M.list()[1])
for i in range(numMessages):
    for j in M.retr(i+1)[1]:
        print j
\end{verbatim}

At the end of the module, there is a test section that contains a more
extensive example of usage.

\section{\module{imaplib} ---
         IMAP4 protocol client}

\declaremodule{standard}{imaplib}
\modulesynopsis{IMAP4 protocol client (requires sockets).}
\moduleauthor{Piers Lauder}{piers@communitysolutions.com.au}
\sectionauthor{Piers Lauder}{piers@communitysolutions.com.au}

% Based on HTML documentation by Piers Lauder <piers@communitysolutions.com.au>;
% converted by Fred L. Drake, Jr. <fdrake@acm.org>.
% Revised by ESR, January 2000.
% Changes for IMAP4_SSL by Tino Lange <Tino.Lange@isg.de>, March 2002 
% Changes for IMAP4_stream by Piers Lauder <piers@communitysolutions.com.au>, November 2002 

\indexii{IMAP4}{protocol}
\indexii{IMAP4_SSL}{protocol}
\indexii{IMAP4_stream}{protocol}

This module defines three classes, \class{IMAP4}, \class{IMAP4_SSL} and \class{IMAP4_stream}, which encapsulate a
connection to an IMAP4 server and implement a large subset of the
IMAP4rev1 client protocol as defined in \rfc{2060}. It is backward
compatible with IMAP4 (\rfc{1730}) servers, but note that the
\samp{STATUS} command is not supported in IMAP4.

Three classes are provided by the \module{imaplib} module, \class{IMAP4} is the base class:

\begin{classdesc}{IMAP4}{\optional{host\optional{, port}}}
This class implements the actual IMAP4 protocol.  The connection is
created and protocol version (IMAP4 or IMAP4rev1) is determined when
the instance is initialized.
If \var{host} is not specified, \code{''} (the local host) is used.
If \var{port} is omitted, the standard IMAP4 port (143) is used.
\end{classdesc}

Three exceptions are defined as attributes of the \class{IMAP4} class:

\begin{excdesc}{IMAP4.error}
Exception raised on any errors.  The reason for the exception is
passed to the constructor as a string.
\end{excdesc}

\begin{excdesc}{IMAP4.abort}
IMAP4 server errors cause this exception to be raised.  This is a
sub-class of \exception{IMAP4.error}.  Note that closing the instance
and instantiating a new one will usually allow recovery from this
exception.
\end{excdesc}

\begin{excdesc}{IMAP4.readonly}
This exception is raised when a writable mailbox has its status changed by the server.  This is a
sub-class of \exception{IMAP4.error}.  Some other client now has write permission,
and the mailbox will need to be re-opened to re-obtain write permission.
\end{excdesc}

There's also a subclass for secure connections:

\begin{classdesc}{IMAP4_SSL}{\optional{host\optional{, port\optional{, keyfile\optional{, certfile}}}}}
This is a subclass derived from \class{IMAP4} that connects over an SSL encrypted socket 
(to use this class you need a socket module that was compiled with SSL support).
If \var{host} is not specified, \code{''} (the local host) is used.
If \var{port} is omitted, the standard IMAP4-over-SSL port (993) is used.
\var{keyfile} and \var{certfile} are also optional - they can contain a PEM formatted
private key and certificate chain file for the SSL connection. 
\end{classdesc}

The second subclass allows for connections created by a child process:

\begin{classdesc}{IMAP4_stream}{command}
This is a subclass derived from \class{IMAP4} that connects
to the \code{stdin/stdout} file descriptors created by passing \var{command} to \code{os.popen2()}.
\versionadded{2.3}
\end{classdesc}

The following utility functions are defined:

\begin{funcdesc}{Internaldate2tuple}{datestr}
  Converts an IMAP4 INTERNALDATE string to Coordinated Universal
  Time. Returns a \refmodule{time} module tuple.
\end{funcdesc}

\begin{funcdesc}{Int2AP}{num}
  Converts an integer into a string representation using characters
  from the set [\code{A} .. \code{P}].
\end{funcdesc}

\begin{funcdesc}{ParseFlags}{flagstr}
  Converts an IMAP4 \samp{FLAGS} response to a tuple of individual
  flags.
\end{funcdesc}

\begin{funcdesc}{Time2Internaldate}{date_time}
  Converts a \refmodule{time} module tuple to an IMAP4
  \samp{INTERNALDATE} representation.  Returns a string in the form:
  \code{"DD-Mmm-YYYY HH:MM:SS +HHMM"} (including double-quotes).
\end{funcdesc}


Note that IMAP4 message numbers change as the mailbox changes; in
particular, after an \samp{EXPUNGE} command performs deletions the
remaining messages are renumbered. So it is highly advisable to use
UIDs instead, with the UID command.

At the end of the module, there is a test section that contains a more
extensive example of usage.

\begin{seealso}
  \seetext{Documents describing the protocol, and sources and binaries 
           for servers implementing it, can all be found at the
           University of Washington's \emph{IMAP Information Center}
           (\url{http://www.cac.washington.edu/imap/}).}
\end{seealso}


\subsection{IMAP4 Objects \label{imap4-objects}}

All IMAP4rev1 commands are represented by methods of the same name,
either upper-case or lower-case.

All arguments to commands are converted to strings, except for
\samp{AUTHENTICATE}, and the last argument to \samp{APPEND} which is
passed as an IMAP4 literal.  If necessary (the string contains IMAP4
protocol-sensitive characters and isn't enclosed with either
parentheses or double quotes) each string is quoted. However, the
\var{password} argument to the \samp{LOGIN} command is always quoted.
If you want to avoid having an argument string quoted
(eg: the \var{flags} argument to \samp{STORE}) then enclose the string in
parentheses (eg: \code{r'(\e Deleted)'}).

Each command returns a tuple: \code{(\var{type}, [\var{data},
...])} where \var{type} is usually \code{'OK'} or \code{'NO'},
and \var{data} is either the text from the command response, or
mandated results from the command. Each \var{data}
is either a string, or a tuple. If a tuple, then the first part
is the header of the response, and the second part contains
the data (ie: 'literal' value).

An \class{IMAP4} instance has the following methods:


\begin{methoddesc}{append}{mailbox, flags, date_time, message}
  Append \var{message} to named mailbox. 
\end{methoddesc}

\begin{methoddesc}{authenticate}{mechanism, authobject}
  Authenticate command --- requires response processing.

  \var{mechanism} specifies which authentication mechanism is to
  be used - it should appear in the instance variable \code{capabilities} in the
  form \code{AUTH=mechanism}.

  \var{authobject} must be a callable object:

\begin{verbatim}
data = authobject(response)
\end{verbatim}

  It will be called to process server continuation responses.
  It should return \code{data} that will be encoded and sent to server.
  It should return \code{None} if the client abort response \samp{*} should
  be sent instead.
\end{methoddesc}

\begin{methoddesc}{check}{}
  Checkpoint mailbox on server. 
\end{methoddesc}

\begin{methoddesc}{close}{}
  Close currently selected mailbox. Deleted messages are removed from
  writable mailbox. This is the recommended command before
  \samp{LOGOUT}.
\end{methoddesc}

\begin{methoddesc}{copy}{message_set, new_mailbox}
  Copy \var{message_set} messages onto end of \var{new_mailbox}. 
\end{methoddesc}

\begin{methoddesc}{create}{mailbox}
  Create new mailbox named \var{mailbox}.
\end{methoddesc}

\begin{methoddesc}{delete}{mailbox}
  Delete old mailbox named \var{mailbox}.
\end{methoddesc}

\begin{methoddesc}{deleteacl}{mailbox, who}
  Delete the ACLs (remove any rights) set for who on mailbox.
\versionadded{2.4}
\end{methoddesc}

\begin{methoddesc}{expunge}{}
  Permanently remove deleted items from selected mailbox. Generates an
  \samp{EXPUNGE} response for each deleted message. Returned data
  contains a list of \samp{EXPUNGE} message numbers in order
  received.
\end{methoddesc}

\begin{methoddesc}{fetch}{message_set, message_parts}
  Fetch (parts of) messages.  \var{message_parts} should be
  a string of message part names enclosed within parentheses,
  eg: \samp{"(UID BODY[TEXT])"}.  Returned data are tuples
  of message part envelope and data.
\end{methoddesc}

\begin{methoddesc}{getacl}{mailbox}
  Get the \samp{ACL}s for \var{mailbox}.
  The method is non-standard, but is supported by the \samp{Cyrus} server.
\end{methoddesc}

\begin{methoddesc}{getquota}{root}
  Get the \samp{quota} \var{root}'s resource usage and limits.
  This method is part of the IMAP4 QUOTA extension defined in rfc2087.
\versionadded{2.3}
\end{methoddesc}

\begin{methoddesc}{getquotaroot}{mailbox}
  Get the list of \samp{quota} \samp{roots} for the named \var{mailbox}.
  This method is part of the IMAP4 QUOTA extension defined in rfc2087.
\versionadded{2.3}
\end{methoddesc}

\begin{methoddesc}{list}{\optional{directory\optional{, pattern}}}
  List mailbox names in \var{directory} matching
  \var{pattern}.  \var{directory} defaults to the top-level mail
  folder, and \var{pattern} defaults to match anything.  Returned data
  contains a list of \samp{LIST} responses.
\end{methoddesc}

\begin{methoddesc}{login}{user, password}
  Identify the client using a plaintext password.
  The \var{password} will be quoted.
\end{methoddesc}

\begin{methoddesc}{login_cram_md5}{user, password}
  Force use of \samp{CRAM-MD5} authentication when identifying the client to protect the password.
  Will only work if the server \samp{CAPABILITY} response includes the phrase \samp{AUTH=CRAM-MD5}.
\versionadded{2.3}
\end{methoddesc}

\begin{methoddesc}{logout}{}
  Shutdown connection to server. Returns server \samp{BYE} response.
\end{methoddesc}

\begin{methoddesc}{lsub}{\optional{directory\optional{, pattern}}}
  List subscribed mailbox names in directory matching pattern.
  \var{directory} defaults to the top level directory and
  \var{pattern} defaults to match any mailbox.
  Returned data are tuples of message part envelope and data.
\end{methoddesc}

\begin{methoddes}{myrights}{mailbox}
  Show my ACLs for a mailbox (i.e. the rights that I have on mailbox).
\versionadded{2.4}
\end{methoddesc}

\begin{methoddesc}{namespace}{}
  Returns IMAP namespaces as defined in RFC2342.
\versionadded{2.3}
\end{methoddesc}

\begin{methoddesc}{noop}{}
  Send \samp{NOOP} to server.
\end{methoddesc}

\begin{methoddesc}{open}{host, port}
  Opens socket to \var{port} at \var{host}.
  The connection objects established by this method
  will be used in the \code{read}, \code{readline}, \code{send}, and \code{shutdown} methods.
  You may override this method.
\end{methoddesc}

\begin{methoddesc}{partial}{message_num, message_part, start, length}
  Fetch truncated part of a message.
  Returned data is a tuple of message part envelope and data.
\end{methoddesc}

\begin{methoddesc}{proxyauth}{user}
  Assume authentication as \var{user}.
  Allows an authorised administrator to proxy into any user's mailbox.
\versionadded{2.3}
\end{methoddesc}

\begin{methoddesc}{read}{size}
  Reads \var{size} bytes from the remote server.
  You may override this method.
\end{methoddesc}

\begin{methoddesc}{readline}{}
  Reads one line from the remote server.
  You may override this method.
\end{methoddesc}

\begin{methoddesc}{recent}{}
  Prompt server for an update. Returned data is \code{None} if no new
  messages, else value of \samp{RECENT} response.
\end{methoddesc}

\begin{methoddesc}{rename}{oldmailbox, newmailbox}
  Rename mailbox named \var{oldmailbox} to \var{newmailbox}.
\end{methoddesc}

\begin{methoddesc}{response}{code}
  Return data for response \var{code} if received, or
  \code{None}. Returns the given code, instead of the usual type.
\end{methoddesc}

\begin{methoddesc}{search}{charset, criterion\optional{, ...}}
  Search mailbox for matching messages.  Returned data contains a space
  separated list of matching message numbers.  \var{charset} may be
  \code{None}, in which case no \samp{CHARSET} will be specified in the
  request to the server.  The IMAP protocol requires that at least one
  criterion be specified; an exception will be raised when the server
  returns an error.

  Example:

\begin{verbatim}
# M is a connected IMAP4 instance...
msgnums = M.search(None, 'FROM', '"LDJ"')

# or:
msgnums = M.search(None, '(FROM "LDJ")')
\end{verbatim}
\end{methoddesc}

\begin{methoddesc}{select}{\optional{mailbox\optional{, readonly}}}
  Select a mailbox. Returned data is the count of messages in
  \var{mailbox} (\samp{EXISTS} response).  The default \var{mailbox}
  is \code{'INBOX'}.  If the \var{readonly} flag is set, modifications
  to the mailbox are not allowed.
\end{methoddesc}

\begin{methoddesc}{send}{data}
  Sends \code{data} to the remote server.
  You may override this method.
\end{methoddesc}

\begin{methoddesc}{setacl}{mailbox, who, what}
  Set an \samp{ACL} for \var{mailbox}.
  The method is non-standard, but is supported by the \samp{Cyrus} server.
\end{methoddesc}

\begin{methoddesc}{setquota}{root, limits}
  Set the \samp{quota} \var{root}'s resource \var{limits}.
  This method is part of the IMAP4 QUOTA extension defined in rfc2087.
\versionadded{2.3}
\end{methoddesc}

\begin{methoddesc}{shutdown}{}
  Close connection established in \code{open}.
  You may override this method.
\end{methoddesc}

\begin{methoddesc}{socket}{}
  Returns socket instance used to connect to server.
\end{methoddesc}

\begin{methoddesc}{sort}{sort_criteria, charset, search_criterion\optional{, ...}}
  The \code{sort} command is a variant of \code{search} with sorting semantics for
  the results.  Returned data contains a space
  separated list of matching message numbers.

  Sort has two arguments before the \var{search_criterion}
  argument(s); a parenthesized list of \var{sort_criteria}, and the searching \var{charset}.
  Note that unlike \code{search}, the searching \var{charset} argument is mandatory.
  There is also a \code{uid sort} command which corresponds to \code{sort} the way
  that \code{uid search} corresponds to \code{search}.
  The \code{sort} command first searches the mailbox for messages that
  match the given searching criteria using the charset argument for
  the interpretation of strings in the searching criteria.  It then
  returns the numbers of matching messages.

  This is an \samp{IMAP4rev1} extension command.
\end{methoddesc}

\begin{methoddesc}{status}{mailbox, names}
  Request named status conditions for \var{mailbox}. 
\end{methoddesc}

\begin{methoddesc}{store}{message_set, command, flag_list}
  Alters flag dispositions for messages in mailbox.
\end{methoddesc}

\begin{methoddesc}{subscribe}{mailbox}
  Subscribe to new mailbox.
\end{methoddesc}

\begin{methoddesc}{thread}{threading_algorithm, charset, search_criterion\optional{, ...}}
  The \code{thread} command is a variant of \code{search} with threading semantics for
  the results.  Returned data contains a space
  separated list of thread members.

  Thread members consist of zero or more messages numbers, delimited by spaces,
  indicating successive parent and child.

  Thread has two arguments before the \var{search_criterion}
  argument(s); a \var{threading_algorithm}, and the searching \var{charset}.
  Note that unlike \code{search}, the searching \var{charset} argument is mandatory.
  There is also a \code{uid thread} command which corresponds to \code{thread} the way
  that \code{uid search} corresponds to \code{search}.
  The \code{thread} command first searches the mailbox for messages that
  match the given searching criteria using the charset argument for
  the interpretation of strings in the searching criteria. It thren
  returns the matching messages threaded according to the specified
  threading algorithm.

  This is an \samp{IMAP4rev1} extension command. \versionadded{2.4}
\end{methoddesc}

\begin{methoddesc}{uid}{command, arg\optional{, ...}}
  Execute command args with messages identified by UID, rather than
  message number.  Returns response appropriate to command.  At least
  one argument must be supplied; if none are provided, the server will
  return an error and an exception will be raised.
\end{methoddesc}

\begin{methoddesc}{unsubscribe}{mailbox}
  Unsubscribe from old mailbox.
\end{methoddesc}

\begin{methoddesc}{xatom}{name\optional{, arg\optional{, ...}}}
  Allow simple extension commands notified by server in
  \samp{CAPABILITY} response.
\end{methoddesc}


Instances of \class{IMAP4_SSL} have just one additional method:

\begin{methoddesc}{ssl}{}
  Returns SSLObject instance used for the secure connection with the server.
\end{methoddesc}


The following attributes are defined on instances of \class{IMAP4}:


\begin{memberdesc}{PROTOCOL_VERSION}
The most recent supported protocol in the
\samp{CAPABILITY} response from the server.
\end{memberdesc}

\begin{memberdesc}{debug}
Integer value to control debugging output.  The initialize value is
taken from the module variable \code{Debug}.  Values greater than
three trace each command.
\end{memberdesc}


\subsection{IMAP4 Example \label{imap4-example}}

Here is a minimal example (without error checking) that opens a
mailbox and retrieves and prints all messages:

\begin{verbatim}
import getpass, imaplib

M = imaplib.IMAP4()
M.login(getpass.getuser(), getpass.getpass())
M.select()
typ, data = M.search(None, 'ALL')
for num in data[0].split():
    typ, data = M.fetch(num, '(RFC822)')
    print 'Message %s\n%s\n' % (num, data[0][1])
M.logout()
\end{verbatim}

\section{Standard Module \module{nntplib}}
\label{module-nntplib}
\stmodindex{nntplib}
\indexii{NNTP}{protocol}
\index{Network News Transfer Protocol}


This module defines the class \class{NNTP} which implements the client
side of the NNTP protocol.  It can be used to implement a news reader
or poster, or automated news processors.  For more information on NNTP
(Network News Transfer Protocol), see Internet \rfc{977}.

Here are two small examples of how it can be used.  To list some
statistics about a newsgroup and print the subjects of the last 10
articles:

\begin{verbatim}
>>> s = NNTP('news.cwi.nl')
>>> resp, count, first, last, name = s.group('comp.lang.python')
>>> print 'Group', name, 'has', count, 'articles, range', first, 'to', last
Group comp.lang.python has 59 articles, range 3742 to 3803
>>> resp, subs = s.xhdr('subject', first + '-' + last)
>>> for id, sub in subs[-10:]: print id, sub
... 
3792 Re: Removing elements from a list while iterating...
3793 Re: Who likes Info files?
3794 Emacs and doc strings
3795 a few questions about the Mac implementation
3796 Re: executable python scripts
3797 Re: executable python scripts
3798 Re: a few questions about the Mac implementation 
3799 Re: PROPOSAL: A Generic Python Object Interface for Python C Modules
3802 Re: executable python scripts 
3803 Re: \POSIX{} wait and SIGCHLD
>>> s.quit()
'205 news.cwi.nl closing connection.  Goodbye.'
\end{verbatim}

To post an article from a file (this assumes that the article has
valid headers):

\begin{verbatim}
>>> s = NNTP('news.cwi.nl')
>>> f = open('/tmp/article')
>>> s.post(f)
'240 Article posted successfully.'
>>> s.quit()
'205 news.cwi.nl closing connection.  Goodbye.'
\end{verbatim}
%
The module itself defines the following items:

\begin{classdesc}{NNTP}{host\optional{, port}}
Return a new instance of the \class{NNTP} class, representing a
connection to the NNTP server running on host \var{host}, listening at
port \var{port}.  The default \var{port} is 119.
\end{classdesc}

\begin{excdesc}{error_reply}
Exception raised when an unexpected reply is received from the server.
\end{excdesc}

\begin{excdesc}{error_temp}
Exception raised when an error code in the range 400--499 is received.
\end{excdesc}

\begin{excdesc}{error_perm}
Exception raised when an error code in the range 500--599 is received.
\end{excdesc}

\begin{excdesc}{error_proto}
Exception raised when a reply is received from the server that does
not begin with a digit in the range 1--5.
\end{excdesc}


\subsection{NNTP Objects}
\label{nntp-objects}

NNTP instances have the following methods.  The \var{response} that is
returned as the first item in the return tuple of almost all methods
is the server's response: a string beginning with a three-digit code.
If the server's response indicates an error, the method raises one of
the above exceptions.


\begin{methoddesc}{getwelcome}{}
Return the welcome message sent by the server in reply to the initial
connection.  (This message sometimes contains disclaimers or help
information that may be relevant to the user.)
\end{methoddesc}

\begin{methoddesc}{set_debuglevel}{level}
Set the instance's debugging level.  This controls the amount of
debugging output printed.  The default, \code{0}, produces no debugging
output.  A value of \code{1} produces a moderate amount of debugging
output, generally a single line per request or response.  A value of
\code{2} or higher produces the maximum amount of debugging output,
logging each line sent and received on the connection (including
message text).
\end{methoddesc}

\begin{methoddesc}{newgroups}{date, time}
Send a \samp{NEWGROUPS} command.  The \var{date} argument should be a
string of the form \code{"\var{yy}\var{mm}\var{dd}"} indicating the
date, and \var{time} should be a string of the form
\code{"\var{hh}\var{mm}\var{ss}"} indicating the time.  Return a pair
\code{(\var{response}, \var{groups})} where \var{groups} is a list of
group names that are new since the given date and time.
\end{methoddesc}

\begin{methoddesc}{newnews}{group, date, time}
Send a \samp{NEWNEWS} command.  Here, \var{group} is a group name or
\code{'*'}, and \var{date} and \var{time} have the same meaning as for
\method{newgroups()}.  Return a pair \code{(\var{response},
\var{articles})} where \var{articles} is a list of article ids.
\end{methoddesc}

\begin{methoddesc}{list}{}
Send a \samp{LIST} command.  Return a pair \code{(\var{response},
\var{list})} where \var{list} is a list of tuples.  Each tuple has the
form \code{(\var{group}, \var{last}, \var{first}, \var{flag})}, where
\var{group} is a group name, \var{last} and \var{first} are the last
and first article numbers (as strings), and \var{flag} is \code{'y'}
if posting is allowed, \code{'n'} if not, and \code{'m'} if the
newsgroup is moderated.  (Note the ordering: \var{last}, \var{first}.)
\end{methoddesc}

\begin{methoddesc}{group}{name}
Send a \samp{GROUP} command, where \var{name} is the group name.
Return a tuple \code{(}\var{response}\code{,} \var{count}\code{,}
\var{first}\code{,} \var{last}\code{,} \var{name}\code{)} where
\var{count} is the (estimated) number of articles in the group,
\var{first} is the first article number in the group, \var{last} is
the last article number in the group, and \var{name} is the group
name.  The numbers are returned as strings.
\end{methoddesc}

\begin{methoddesc}{help}{}
Send a \samp{HELP} command.  Return a pair \code{(\var{response},
\var{list})} where \var{list} is a list of help strings.
\end{methoddesc}

\begin{methoddesc}{stat}{id}
Send a \samp{STAT} command, where \var{id} is the message id (enclosed
in \character{<} and \character{>}) or an article number (as a string).
Return a triple \code{(\var{response}, \var{number}, \var{id})} where
\var{number} is the article number (as a string) and \var{id} is the
article id  (enclosed in \character{<} and \character{>}).
\end{methoddesc}

\begin{methoddesc}{next}{}
Send a \samp{NEXT} command.  Return as for \method{stat()}.
\end{methoddesc}

\begin{methoddesc}{last}{}
Send a \samp{LAST} command.  Return as for \method{stat()}.
\end{methoddesc}

\begin{methoddesc}{head}{id}
Send a \samp{HEAD} command, where \var{id} has the same meaning as for
\method{stat()}.  Return a tuple
\code{(\var{response}, \var{number}, \var{id}, \var{list})}
where the first three are the same as for \method{stat()},
and \var{list} is a list of the article's headers (an uninterpreted
list of lines, without trailing newlines).
\end{methoddesc}

\begin{methoddesc}{body}{id}
Send a \samp{BODY} command, where \var{id} has the same meaning as for
\method{stat()}.  Return as for \method{head()}.
\end{methoddesc}

\begin{methoddesc}{article}{id}
Send a \samp{ARTICLE} command, where \var{id} has the same meaning as
for \method{stat()}.  Return as for \method{head()}.
\end{methoddesc}

\begin{methoddesc}{slave}{}
Send a \samp{SLAVE} command.  Return the server's \var{response}.
\end{methoddesc}

\begin{methoddesc}{xhdr}{header, string}
Send an \samp{XHDR} command.  This command is not defined in the RFC
but is a common extension.  The \var{header} argument is a header
keyword, e.g. \code{'subject'}.  The \var{string} argument should have
the form \code{"\var{first}-\var{last}"} where \var{first} and
\var{last} are the first and last article numbers to search.  Return a
pair \code{(\var{response}, \var{list})}, where \var{list} is a list of
pairs \code{(\var{id}, \var{text})}, where \var{id} is an article id
(as a string) and \var{text} is the text of the requested header for
that article.
\end{methoddesc}

\begin{methoddesc}{post}{file}
Post an article using the \samp{POST} command.  The \var{file}
argument is an open file object which is read until EOF using its
\method{readline()} method.  It should be a well-formed news article,
including the required headers.  The \method{post()} method
automatically escapes lines beginning with \samp{.}.
\end{methoddesc}

\begin{methoddesc}{ihave}{id, file}
Send an \samp{IHAVE} command.  If the response is not an error, treat
\var{file} exactly as for the \method{post()} method.
\end{methoddesc}

\begin{methoddesc}{date}{}
Return a triple \code{(\var{response}, \var{date}, \var{time})},
containing the current date and time in a form suitable for the
\method{newnews()} and \method{newgroups()} methods.
This is an optional NNTP extension, and may not be supported by all
servers.
\end{methoddesc}

\begin{methoddesc}{xgtitle}{name}
Process an \samp{XGTITLE} command, returning a pair \code{(\var{response},
\var{list})}, where \var{list} is a list of tuples containing
\code{(\var{name}, \var{title})}.
% XXX huh?  Should that be name, description?
This is an optional NNTP extension, and may not be supported by all
servers.
\end{methoddesc}

\begin{methoddesc}{xover}{start, end}
Return a pair \code{(\var{resp}, \var{list})}.  \var{list} is a list
of tuples, one for each article in the range delimited by the \var{start}
and \var{end} article numbers.  Each tuple is of the form
\code{(}\var{article number}\code{,} \var{subject}\code{,}
\var{poster}\code{,} \var{date}\code{,} \var{id}\code{,}
\var{references}\code{,} \var{size}\code{,} \var{lines}\code{)}.
This is an optional NNTP extension, and may not be supported by all
servers.
\end{methoddesc}

\begin{methoddesc}{xpath}{id}
Return a pair \code{(\var{resp}, \var{path})}, where \var{path} is the
directory path to the article with message ID \var{id}.  This is an
optional NNTP extension, and may not be supported by all servers.
\end{methoddesc}

\begin{methoddesc}{quit}{}
Send a \samp{QUIT} command and close the connection.  Once this method
has been called, no other methods of the NNTP object should be called.
\end{methoddesc}

\section{\module{smtplib} ---
         SMTP protocol client.}
\declaremodule{standard}{smtplib}
\sectionauthor{Eric S. Raymond}{esr@snark.thyrsus.com}

\modulesynopsis{SMTP protocol client (requires sockets).}

\indexii{SMTP}{protocol}
\index{Simple Mail Transfer Protocol}

The \module{smtplib} module defines an SMTP client session object that
can be used to send mail to any Internet machine with an SMTP or ESMTP
listener daemon.  For details of SMTP and ESMTP operation, consult
\rfc{821} (\emph{Simple Mail Transfer Protocol}) and \rfc{1869}
(\emph{SMTP Service Extensions}).

\begin{classdesc}{SMTP}{\optional{host\optional{, port}}}
A \class{SMTP} instance encapsulates an SMTP connection.  It has
methods that support a full repertoire of SMTP and ESMTP
operations. If the optional host and port parameters are given, the
SMTP \method{connect()} method is called with those parameters during
initialization.

For normal use, you should only require the initialization/connect,
\method{sendmail()}, and \method{quit()} methods  An example is
included below.
\end{classdesc}


\subsection{SMTP Objects}
\label{SMTP-objects}

An \class{SMTP} instance has the following methods:

\begin{methoddesc}{set_debuglevel}{level}
Set the debug output level.  A true value for \var{level} results in
debug messages for connection and for all messages sent to and
received from the server.
\end{methoddesc}

\begin{methoddesc}{connect}{\optional{host\optional{, port}}}
Connect to a host on a given port.  The defaults are to connect to the 
local host at the standard SMTP port (25).

If the hostname ends with a colon (\character{:}) followed by a
number, that suffix will be stripped off and the number interpreted as 
the port number to use.

Note:  This method is automatically invoked by the constructor if a
host is specified during instantiation.
\end{methoddesc}

\begin{methoddesc}{docmd}{cmd, \optional{, argstring}}
Send a command \var{cmd} to the server.  The optional argument
\var{argstring} is simply concatenated to the command, separated by a
space.

This returns a 2-tuple composed of a numeric response code and the
actual response line (multiline responses are joined into one long
line.)

In normal operation it should not be necessary to call this method
explicitly.  It is used to implement other methods and may be useful
for testing private extensions.
\end{methoddesc}

\begin{methoddesc}{helo}{\optional{hostname}}
Identify yourself to the SMTP server using \samp{HELO}.  The hostname
argument defaults to the fully qualified domain name of the local
host.

In normal operation it should not be necessary to call this method
explicitly.  It will be implicitly called by the \method{sendmail()}
when necessary.
\end{methoddesc}

\begin{methoddesc}{ehlo}{\optional{hostname}}
Identify yourself to an ESMTP server using \samp{HELO}.  The hostname
argument defaults to the fully qualified domain name of the local
host.  Examine the response for ESMTP option and store them for use by
\method{has_option()}.

Unless you wish to use \method{has_option()} before sending
mail, it should not be necessary to call this method explicitly.  It
will be implicitly called by \method{sendmail()} when necessary.
\end{methoddesc}

\begin{methoddesc}{has_option}{name}
Return \code{1} if \var{name} is in the set of ESMTP options returned
by the server, \code{0} otherwise.  Case is ignored.
\end{methoddesc}

\begin{methoddesc}{verify}{address}
Check the validity of an address on this server using SMTP \samp{VRFY}.
Returns a tuple consisting of code 250 and a full \rfc{822} address
(including human name) if the user address is valid. Otherwise returns
an SMTP error code of 400 or greater and an error string.

Note: many sites disable SMTP \samp{VRFY} in order to foil spammers.
\end{methoddesc}

\begin{methoddesc}{sendmail}{from_addr, to_addrs, msg\optional{, options}}
Send mail.  The required arguments are an \rfc{822} from-address
string, a list of \rfc{822} to-address strings, and a message string.
The caller may pass a list of ESMTP options to be used in \samp{MAIL
FROM} commands as \var{options}.

If there has been no previous \samp{EHLO} or \samp{HELO} command this
session, this method tries ESMTP \samp{EHLO} first. If the server does
ESMTP, message size and each of the specified options will be passed
to it (if the option is in the feature set the server advertises).  If
\samp{EHLO} fails, \samp{HELO} will be tried and ESMTP options
suppressed.

This method will return normally if the mail is accepted for at least 
one recipient. Otherwise it will throw an exception (either
\exception{SMTPSenderRefused}, \exception{SMTPRecipientsRefused}, or
\exception{SMTPDataError}).  That is, if this method does not throw an
exception, then someone should get your mail.  If this method does not
throw an exception, it returns a dictionary, with one entry for each
recipient that was refused.
\end{methoddesc}

\begin{methoddesc}{quit}{}
Terminate the SMTP session and close the connection.
\end{methoddesc}

Low-level methods corresponding to the standard SMTP/ESMTP commands
\samp{HELP}, \samp{RSET}, \samp{NOOP}, \samp{MAIL}, \samp{RCPT}, and
\samp{DATA} are also supported.  Normally these do not need to be
called directly, so they are not documented here.  For details,
consult the module code.


\subsection{SMTP Example \label{SMTP-example}}

% really need a little description here...

\begin{verbatim}
import rfc822, string, sys
import smtplib

def prompt(prompt):
    sys.stdout.write(prompt + ": ")
    return string.strip(sys.stdin.readline())

fromaddr = prompt("From")
toaddrs  = string.splitfields(prompt("To"), ',')
print "Enter message, end with ^D:"
msg = ''
while 1:
    line = sys.stdin.readline()
    if not line:
        break
    msg = msg + line
print "Message length is " + `len(msg)`

server = smtplib.SMTP('localhost')
server.set_debuglevel(1)
server.sendmail(fromaddr, toaddrs, msg)
server.quit()
\end{verbatim}


\begin{seealso}
\seetext{\rfc{821}, \emph{Simple Mail Transfer Protocol}.  Available
online at \url{http://info.internet.isi.edu/in-notes/rfc/files/rfc821.txt}.}

\seetext{\rfc{1869}, \emph{SMTP Service Extensions}.  Available online 
at \url{http://info.internet.isi.edu/in-notes/rfc/files/rfc1869.txt}.}
\end{seealso}

\section{\module{urlparse} ---
         Parse URLs into components}
\declaremodule{standard}{urlparse}

\modulesynopsis{Parse URLs into components.}

\index{WWW}
\index{World Wide Web}
\index{URL}
\indexii{URL}{parsing}
\indexii{relative}{URL}


This module defines a standard interface to break Uniform Resource
Locator (URL) strings up in components (addressing scheme, network
location, path etc.), to combine the components back into a URL
string, and to convert a ``relative URL'' to an absolute URL given a
``base URL.''

The module has been designed to match the Internet RFC on Relative
Uniform Resource Locators (and discovered a bug in an earlier
draft!). It supports the following URL schemes:
\code{file}, \code{ftp}, \code{gopher}, \code{hdl}, \code{http}, 
\code{https}, \code{imap}, \code{mailto}, \code{mms}, \code{news}, 
\code{nntp}, \code{prospero}, \code{rsync}, \code{rtsp}, \code{rtspu}, 
\code{sftp}, \code{shttp}, \code{sip}, \code{snews}, \code{svn}, 
\code{svn+ssh}, \code{telnet}, \code{wais}.

The \module{urlparse} module defines the following functions:

\begin{funcdesc}{urlparse}{urlstring\optional{, default_scheme\optional{, allow_fragments}}}
Parse a URL into 6 components, returning a 6-tuple: (addressing
scheme, network location, path, parameters, query, fragment
identifier).  This corresponds to the general structure of a URL:
\code{\var{scheme}://\var{netloc}/\var{path};\var{parameters}?\var{query}\#\var{fragment}}.
Each tuple item is a string, possibly empty.
The components are not broken up in smaller parts (e.g. the network
location is a single string), and \% escapes are not expanded.
The delimiters as shown above are not part of the tuple items,
except for a leading slash in the \var{path} component, which is
retained if present.

Example:

\begin{verbatim}
urlparse('http://www.cwi.nl:80/%7Eguido/Python.html')
\end{verbatim}

yields the tuple

\begin{verbatim}
('http', 'www.cwi.nl:80', '/%7Eguido/Python.html', '', '', '')
\end{verbatim}

If the \var{default_scheme} argument is specified, it gives the
default addressing scheme, to be used only if the URL string does not
specify one.  The default value for this argument is the empty string.

If the \var{allow_fragments} argument is zero, fragment identifiers
are not allowed, even if the URL's addressing scheme normally does
support them.  The default value for this argument is \code{1}.
\end{funcdesc}

\begin{funcdesc}{urlunparse}{tuple}
Construct a URL string from a tuple as returned by \code{urlparse()}.
This may result in a slightly different, but equivalent URL, if the
URL that was parsed originally had redundant delimiters, e.g. a ? with
an empty query (the draft states that these are equivalent).
\end{funcdesc}

\begin{funcdesc}{urlsplit}{urlstring\optional{,
                           default_scheme\optional{, allow_fragments}}}
This is similar to \function{urlparse()}, but does not split the
params from the URL.  This should generally be used instead of
\function{urlparse()} if the more recent URL syntax allowing
parameters to be applied to each segment of the \var{path} portion of
the URL (see \rfc{2396}) is wanted.  A separate function is needed to
separate the path segments and parameters.  This function returns a
5-tuple: (addressing scheme, network location, path, query, fragment
identifier).
\versionadded{2.2}
\end{funcdesc}

\begin{funcdesc}{urlunsplit}{tuple}
Combine the elements of a tuple as returned by \function{urlsplit()}
into a complete URL as a string.
\versionadded{2.2}
\end{funcdesc}

\begin{funcdesc}{urljoin}{base, url\optional{, allow_fragments}}
Construct a full (``absolute'') URL by combining a ``base URL''
(\var{base}) with a ``relative URL'' (\var{url}).  Informally, this
uses components of the base URL, in particular the addressing scheme,
the network location and (part of) the path, to provide missing
components in the relative URL.

Example:

\begin{verbatim}
urljoin('http://www.cwi.nl/%7Eguido/Python.html', 'FAQ.html')
\end{verbatim}

yields the string

\begin{verbatim}
'http://www.cwi.nl/%7Eguido/FAQ.html'
\end{verbatim}

The \var{allow_fragments} argument has the same meaning as for
\code{urlparse()}.
\end{funcdesc}

\begin{funcdesc}{urldefrag}{url}
If \var{url} contains a fragment identifier, returns a modified
version of \var{url} with no fragment identifier, and the fragment
identifier as a separate string.  If there is no fragment identifier
in \var{url}, returns \var{url} unmodified and an empty string.
\end{funcdesc}


\begin{seealso}
  \seerfc{1738}{Uniform Resource Locators (URL)}{
        This specifies the formal syntax and semantics of absolute
        URLs.}
  \seerfc{1808}{Relative Uniform Resource Locators}{
        This Request For Comments includes the rules for joining an
        absolute and a relative URL, including a fair number of
        ``Abnormal Examples'' which govern the treatment of border
        cases.}
  \seerfc{2396}{Uniform Resource Identifiers (URI): Generic Syntax}{
        Document describing the generic syntactic requirements for
        both Uniform Resource Names (URNs) and Uniform Resource
        Locators (URLs).}
\end{seealso}

\section{\module{SocketServer} ---
         A framework for network servers.}
\declaremodule{standard}{SocketServer}

\modulesynopsis{A framework for network servers.}


The \module{SocketServer} module simplifies the task of writing network
servers.

There are four basic server classes: \class{TCPServer} uses the
Internet TCP protocol, which provides for continuous streams of data
between the client and server.  \class{UDPServer} uses datagrams, which
are discrete packets of information that may arrive out of order or be
lost while in transit.  The more infrequently used
\class{UnixStreamServer} and \class{UnixDatagramServer} classes are
similar, but use \UNIX{} domain sockets; they're not available on
non-\UNIX{} platforms.  For more details on network programming, consult
a book such as W. Richard Steven's \emph{UNIX Network Programming}
or Ralph Davis's \emph{Win32 Network Programming}.

These four classes process requests \dfn{synchronously}; each request
must be completed before the next request can be started.  This isn't
suitable if each request takes a long time to complete, because it
requires a lot of computation, or because it returns a lot of data
which the client is slow to process.  The solution is to create a
separate process or thread to handle each request; the
\class{ForkingMixIn} and \class{ThreadingMixIn} mix-in classes can be
used to support asynchronous behaviour.

Creating a server requires several steps.  First, you must create a
request handler class by subclassing the \class{BaseRequestHandler}
class and overriding its \method{handle()} method; this method will
process incoming requests.  Second, you must instantiate one of the
server classes, passing it the server's address and the request
handler class.  Finally, call the \method{handle_request()} or
\method{serve_forever()} method of the server object to process one or
many requests.

Server classes have the same external methods and attributes, no
matter what network protocol they use:

\setindexsubitem{(SocketServer protocol)}

%XXX should data and methods be intermingled, or separate?
% how should the distinction between class and instance variables be
% drawn?

\begin{funcdesc}{fileno}{}
Return an integer file descriptor for the socket on which the server
is listening.  This function is most commonly passed to
\function{select.select()}, to allow monitoring multiple servers in the
same process.
\end{funcdesc}

\begin{funcdesc}{handle_request}{}
Process a single request.  This function calls the following methods
in order: \method{get_request()}, \method{verify_request()}, and
\method{process_request()}.  If the user-provided \method{handle()}
method of the handler class raises an exception, the server's
\method{handle_error()} method will be called.
\end{funcdesc}

\begin{funcdesc}{serve_forever}{}
Handle an infinite number of requests.  This simply calls
\method{handle_request()} inside an infinite loop.
\end{funcdesc}

\begin{datadesc}{address_family}
The family of protocols to which the server's socket belongs.
\constant{socket.AF_INET} and \constant{socket.AF_UNIX} are two
possible values.
\end{datadesc}

\begin{datadesc}{RequestHandlerClass}
The user-provided request handler class; an instance of this class is
created for each request.
\end{datadesc}

\begin{datadesc}{server_address}
The address on which the server is listening.  The format of addresses
varies depending on the protocol family; see the documentation for the
socket module for details.  For Internet protocols, this is a tuple
containing a string giving the address, and an integer port number:
\code{('127.0.0.1', 80)}, for example.
\end{datadesc}

\begin{datadesc}{socket}
The socket object on which the server will listen for incoming requests.
\end{datadesc}

% XXX should class variables be covered before instance variables, or
% vice versa?

The server classes support the following class variables:

\begin{datadesc}{request_queue_size}
The size of the request queue.  If it takes a long time to process a
single request, any requests that arrive while the server is busy are
placed into a queue, up to \member{request_queue_size} requests.  Once
the queue is full, further requests from clients will get a
``Connection denied'' error.  The default value is usually 5, but this
can be overridden by subclasses.
\end{datadesc}

\begin{datadesc}{socket_type}
The type of socket used by the server; \constant{socket.SOCK_STREAM}
and \constant{socket.SOCK_DGRAM} are two possible values.
\end{datadesc}

There are various server methods that can be overridden by subclasses
of base server classes like \class{TCPServer}; these methods aren't
useful to external users of the server object.

% should the default implementations of these be documented, or should
% it be assumed that the user will look at SocketServer.py?

\begin{funcdesc}{finish_request}{}
Actually processes the request by instantiating
\member{RequestHandlerClass} and calling its \method{handle()} method.
\end{funcdesc}

\begin{funcdesc}{get_request}{}
Must accept a request from the socket, and return a 2-tuple containing
the \emph{new} socket object to be used to communicate with the
client, and the client's address.
\end{funcdesc}

\begin{funcdesc}{handle_error}{request, client_address}
This function is called if the \member{RequestHandlerClass}'s
\method{handle()} method raises an exception.  The default action is
to print the traceback to standard output and continue handling
further requests.
\end{funcdesc}

\begin{funcdesc}{process_request}{request, client_address}
Calls \method{finish_request()} to create an instance of the
\member{RequestHandlerClass}.  If desired, this function can create a
new process or thread to handle the request; the \class{ForkingMixIn}
and \class{ThreadingMixIn} classes do this.
\end{funcdesc}

% Is there any point in documenting the following two functions?
% What would the purpose of overriding them be: initializing server
% instance variables, adding new network families?

\begin{funcdesc}{server_activate}{}
Called by the server's constructor to activate the server.
May be overridden.
\end{funcdesc}

\begin{funcdesc}{server_bind}{}
Called by the server's constructor to bind the socket to the desired
address.  May be overridden.
\end{funcdesc}

\begin{funcdesc}{verify_request}{request, client_address}
Must return a Boolean value; if the value is true, the request will be
processed, and if it's false, the request will be denied.
This function can be overridden to implement access controls for a server.
The default implementation always return true.
\end{funcdesc}

The request handler class must define a new \method{handle()} method,
and can override any of the following methods.  A new instance is
created for each request.

\begin{funcdesc}{finish}{}
Called after the \method{handle()} method to perform any clean-up
actions required.  The default implementation does nothing.  If
\method{setup()} or \method{handle()} raise an exception, this
function will not be called.
\end{funcdesc}

\begin{funcdesc}{handle}{}
This function must do all the work required to service a request.
Several instance attributes are available to it; the request is
available as \member{self.request}; the client address as
\member{self.client_address}; and the server instance as
\member{self.server}, in case it needs access to per-server
information.

The type of \member{self.request} is different for datagram or stream
services.  For stream services, \member{self.request} is a socket
object; for datagram services, \member{self.request} is a string.
However, this can be hidden by using the mix-in request handler
classes
\class{StreamRequestHandler} or \class{DatagramRequestHandler}, which
override the \method{setup()} and \method{finish()} methods, and
provides \member{self.rfile} and \member{self.wfile} attributes.
\member{self.rfile} and \member{self.wfile} can be read or written,
respectively, to get the request data or return data to the client.
\end{funcdesc}

\begin{funcdesc}{setup}{}
Called before the \method{handle()} method to perform any
initialization actions required.  The default implementation does
nothing.
\end{funcdesc}

\section{\module{BaseHTTPServer} ---
         Basic HTTP server}

\declaremodule{standard}{BaseHTTPServer}
\modulesynopsis{Basic HTTP server (base class for
                \class{SimpleHTTPServer} and \class{CGIHTTPServer}).}


\indexii{WWW}{server}
\indexii{HTTP}{protocol}
\index{URL}
\index{httpd}

This module defines two classes for implementing HTTP servers
(Web servers). Usually, this module isn't used directly, but is used
as a basis for building functioning Web servers. See the
\refmodule{SimpleHTTPServer}\refstmodindex{SimpleHTTPServer} and
\refmodule{CGIHTTPServer}\refstmodindex{CGIHTTPServer} modules.

The first class, \class{HTTPServer}, is a
\class{SocketServer.TCPServer} subclass.  It creates and listens at the
HTTP socket, dispatching the requests to a handler.  Code to create and
run the server looks like this:

\begin{verbatim}
def run(server_class=BaseHTTPServer.HTTPServer,
        handler_class=BaseHTTPServer.BaseHTTPRequestHandler):
    server_address = ('', 8000)
    httpd = server_class(server_address, handler_class)
    httpd.serve_forever()
\end{verbatim}

\begin{classdesc}{HTTPServer}{server_address, RequestHandlerClass}
This class builds on the \class{TCPServer} class by
storing the server address as instance
variables named \member{server_name} and \member{server_port}. The
server is accessible by the handler, typically through the handler's
\member{server} instance variable.
\end{classdesc}

\begin{classdesc}{BaseHTTPRequestHandler}{request, client_address, server}
This class is used
to handle the HTTP requests that arrive at the server. By itself,
it cannot respond to any actual HTTP requests; it must be subclassed
to handle each request method (e.g. GET or POST).
\class{BaseHTTPRequestHandler} provides a number of class and instance
variables, and methods for use by subclasses.

The handler will parse the request and the headers, then call a
method specific to the request type. The method name is constructed
from the request. For example, for the request method \samp{SPAM}, the
\method{do_SPAM()} method will be called with no arguments. All of
the relevant information is stored in instance variables of the
handler.  Subclasses should not need to override or extend the
\method{__init__()} method.
\end{classdesc}


\class{BaseHTTPRequestHandler} has the following instance variables:

\begin{memberdesc}{client_address}
Contains a tuple of the form \code{(\var{host}, \var{port})} referring
to the client's address.
\end{memberdesc}

\begin{memberdesc}{command}
Contains the command (request type). For example, \code{'GET'}.
\end{memberdesc}

\begin{memberdesc}{path}
Contains the request path.
\end{memberdesc}

\begin{memberdesc}{request_version}
Contains the version string from the request. For example,
\code{'HTTP/1.0'}.
\end{memberdesc}

\begin{memberdesc}{headers}
Holds an instance of the class specified by the \member{MessageClass}
class variable. This instance parses and manages the headers in
the HTTP request.
\end{memberdesc}

\begin{memberdesc}{rfile}
Contains an input stream, positioned at the start of the optional
input data.
\end{memberdesc}

\begin{memberdesc}{wfile}
Contains the output stream for writing a response back to the client.
Proper adherence to the HTTP protocol must be used when writing
to this stream.
\end{memberdesc}


\class{BaseHTTPRequestHandler} has the following class variables:

\begin{memberdesc}{server_version}
Specifies the server software version.  You may want to override
this.
The format is multiple whitespace-separated strings,
where each string is of the form name[/version].
For example, \code{'BaseHTTP/0.2'}.
\end{memberdesc}

\begin{memberdesc}{sys_version}
Contains the Python system version, in a form usable by the
\member{version_string} method and the \member{server_version} class
variable. For example, \code{'Python/1.4'}.
\end{memberdesc}

\begin{memberdesc}{error_message_format}
Specifies a format string for building an error response to the
client. It uses parenthesized, keyed format specifiers, so the
format operand must be a dictionary. The \var{code} key should
be an integer, specifying the numeric HTTP error code value.
\var{message} should be a string containing a (detailed) error
message of what occurred, and \var{explain} should be an
explanation of the error code number. Default \var{message}
and \var{explain} values can found in the \var{responses}
class variable.
\end{memberdesc}

\begin{memberdesc}{protocol_version}
This specifies the HTTP protocol version used in responses.  If set
to \code{'HTTP/1.1'}, the server will permit HTTP persistent
connections; however, your server \emph{must} then include an
accurate \code{Content-Length} header (using \method{send_header()})
in all of its responses to clients.  For backwards compatibility,
the setting defaults to \code{'HTTP/1.0'}.
\end{memberdesc}

\begin{memberdesc}{MessageClass}
Specifies a \class{rfc822.Message}-like class to parse HTTP
headers. Typically, this is not overridden, and it defaults to
\class{mimetools.Message}.
\withsubitem{(in module mimetools)}{\ttindex{Message}}
\end{memberdesc}

\begin{memberdesc}{responses}
This variable contains a mapping of error code integers to two-element
tuples containing a short and long message. For example,
\code{\{\var{code}: (\var{shortmessage}, \var{longmessage})\}}. The
\var{shortmessage} is usually used as the \var{message} key in an
error response, and \var{longmessage} as the \var{explain} key
(see the \member{error_message_format} class variable).
\end{memberdesc}


A \class{BaseHTTPRequestHandler} instance has the following methods:

\begin{methoddesc}{handle}{}
Calls \method{handle_one_request()} once (or, if persistent connections
are enabled, multiple times) to handle incoming HTTP requests.
You should never need to override it; instead, implement appropriate
\method{do_*()} methods.
\end{methoddesc}

\begin{methoddesc}{handle_one_request}{}
This method will parse and dispatch
the request to the appropriate \method{do_*()} method.  You should
never need to override it.
\end{methoddesc}

\begin{methoddesc}{send_error}{code\optional{, message}}
Sends and logs a complete error reply to the client. The numeric
\var{code} specifies the HTTP error code, with \var{message} as
optional, more specific text. A complete set of headers is sent,
followed by text composed using the \member{error_message_format}
class variable.
\end{methoddesc}

\begin{methoddesc}{send_response}{code\optional{, message}}
Sends a response header and logs the accepted request. The HTTP
response line is sent, followed by \emph{Server} and \emph{Date}
headers. The values for these two headers are picked up from the
\method{version_string()} and \method{date_time_string()} methods,
respectively.
\end{methoddesc}

\begin{methoddesc}{send_header}{keyword, value}
Writes a specific HTTP header to the output stream. \var{keyword}
should specify the header keyword, with \var{value} specifying
its value.
\end{methoddesc}

\begin{methoddesc}{end_headers}{}
Sends a blank line, indicating the end of the HTTP headers in
the response.
\end{methoddesc}

\begin{methoddesc}{log_request}{\optional{code\optional{, size}}}
Logs an accepted (successful) request. \var{code} should specify
the numeric HTTP code associated with the response. If a size of
the response is available, then it should be passed as the
\var{size} parameter.
\end{methoddesc}

\begin{methoddesc}{log_error}{...}
Logs an error when a request cannot be fulfilled. By default,
it passes the message to \method{log_message()}, so it takes the
same arguments (\var{format} and additional values).
\end{methoddesc}

\begin{methoddesc}{log_message}{format, ...}
Logs an arbitrary message to \code{sys.stderr}. This is typically
overridden to create custom error logging mechanisms. The
\var{format} argument is a standard printf-style format string,
where the additional arguments to \method{log_message()} are applied
as inputs to the formatting. The client address and current date
and time are prefixed to every message logged.
\end{methoddesc}

\begin{methoddesc}{version_string}{}
Returns the server software's version string. This is a combination
of the \member{server_version} and \member{sys_version} class variables.
\end{methoddesc}

\begin{methoddesc}{date_time_string}{}
Returns the current date and time, formatted for a message header.
\end{methoddesc}

\begin{methoddesc}{log_data_time_string}{}
Returns the current date and time, formatted for logging.
\end{methoddesc}

\begin{methoddesc}{address_string}{}
Returns the client address, formatted for logging. A name lookup
is performed on the client's IP address.
\end{methoddesc}


\begin{seealso}
  \seemodule{CGIHTTPServer}{Extended request handler that supports CGI
                            scripts.}

  \seemodule{SimpleHTTPServer}{Basic request handler that limits response
                               to files actually under the document root.}
\end{seealso}


\chapter{Internet Data Handling \label{netdata}}

This chapter describes modules which support handling data formats
commonly used on the internet.  Some, like SGML and XML, may be useful 
for other applications as well.

\localmoduletable

\section{\module{sgmllib} ---
         Simple SGML parser}

\declaremodule{standard}{sgmllib}
\modulesynopsis{Only as much of an SGML parser as needed to parse HTML.}

\index{SGML}

This module defines a class \class{SGMLParser} which serves as the
basis for parsing text files formatted in SGML (Standard Generalized
Mark-up Language).  In fact, it does not provide a full SGML parser
--- it only parses SGML insofar as it is used by HTML, and the module
only exists as a base for the \refmodule{htmllib}\refstmodindex{htmllib}
module.


\begin{classdesc}{SGMLParser}{}
The \class{SGMLParser} class is instantiated without arguments.
The parser is hardcoded to recognize the following
constructs:

\begin{itemize}
\item
Opening and closing tags of the form
\samp{<\var{tag} \var{attr}="\var{value}" ...>} and
\samp{</\var{tag}>}, respectively.

\item
Numeric character references of the form \samp{\&\#\var{name};}.

\item
Entity references of the form \samp{\&\var{name};}.

\item
SGML comments of the form \samp{<!--\var{text}-->}.  Note that
spaces, tabs, and newlines are allowed between the trailing
\samp{>} and the immediately preceeding \samp{--}.

\end{itemize}
\end{classdesc}

\class{SGMLParser} instances have the following interface methods:


\begin{methoddesc}{reset}{}
Reset the instance.  Loses all unprocessed data.  This is called
implicitly at instantiation time.
\end{methoddesc}

\begin{methoddesc}{setnomoretags}{}
Stop processing tags.  Treat all following input as literal input
(CDATA).  (This is only provided so the HTML tag
\code{<PLAINTEXT>} can be implemented.)
\end{methoddesc}

\begin{methoddesc}{setliteral}{}
Enter literal mode (CDATA mode).
\end{methoddesc}

\begin{methoddesc}{feed}{data}
Feed some text to the parser.  It is processed insofar as it consists
of complete elements; incomplete data is buffered until more data is
fed or \method{close()} is called.
\end{methoddesc}

\begin{methoddesc}{close}{}
Force processing of all buffered data as if it were followed by an
end-of-file mark.  This method may be redefined by a derived class to
define additional processing at the end of the input, but the
redefined version should always call \method{close()}.
\end{methoddesc}

\begin{methoddesc}{get_starttag_text}{}
Return the text of the most recently opened start tag.  This should
not normally be needed for structured processing, but may be useful in
dealing with HTML ``as deployed'' or for re-generating input with
minimal changes (whitespace between attributes can be preserved,
etc.).
\end{methoddesc}

\begin{methoddesc}{handle_starttag}{tag, method, attributes}
This method is called to handle start tags for which either a
\method{start_\var{tag}()} or \method{do_\var{tag}()} method has been
defined.  The \var{tag} argument is the name of the tag converted to
lower case, and the \var{method} argument is the bound method which
should be used to support semantic interpretation of the start tag.
The \var{attributes} argument is a list of \code{(\var{name},
\var{value})} pairs containing the attributes found inside the tag's
\code{<>} brackets.  The \var{name} has been translated to lower case
and double quotes and backslashes in the \var{value} have been interpreted.
For instance, for the tag \code{<A HREF="http://www.cwi.nl/">}, this
method would be called as \samp{unknown_starttag('a', [('href',
'http://www.cwi.nl/')])}.  The base implementation simply calls
\var{method} with \var{attributes} as the only argument.
\end{methoddesc}

\begin{methoddesc}{handle_endtag}{tag, method}
This method is called to handle endtags for which an
\method{end_\var{tag}()} method has been defined.  The
\var{tag} argument is the name of the tag converted to lower case, and
the \var{method} argument is the bound method which should be used to
support semantic interpretation of the end tag.  If no
\method{end_\var{tag}()} method is defined for the closing element,
this handler is not called.  The base implementation simply calls
\var{method}.
\end{methoddesc}

\begin{methoddesc}{handle_data}{data}
This method is called to process arbitrary data.  It is intended to be
overridden by a derived class; the base class implementation does
nothing.
\end{methoddesc}

\begin{methoddesc}{handle_charref}{ref}
This method is called to process a character reference of the form
\samp{\&\#\var{ref};}.  In the base implementation, \var{ref} must
be a decimal number in the
range 0-255.  It translates the character to \ASCII{} and calls the
method \method{handle_data()} with the character as argument.  If
\var{ref} is invalid or out of range, the method
\code{unknown_charref(\var{ref})} is called to handle the error.  A
subclass must override this method to provide support for named
character entities.
\end{methoddesc}

\begin{methoddesc}{handle_entityref}{ref}
This method is called to process a general entity reference of the
form \samp{\&\var{ref};} where \var{ref} is an general entity
reference.  It looks for \var{ref} in the instance (or class)
variable \member{entitydefs} which should be a mapping from entity
names to corresponding translations.  If a translation is found, it
calls the method \method{handle_data()} with the translation;
otherwise, it calls the method \code{unknown_entityref(\var{ref})}.
The default \member{entitydefs} defines translations for
\code{\&amp;}, \code{\&apos}, \code{\&gt;}, \code{\&lt;}, and
\code{\&quot;}.
\end{methoddesc}

\begin{methoddesc}{handle_comment}{comment}
This method is called when a comment is encountered.  The
\var{comment} argument is a string containing the text between the
\samp{<!--} and \samp{-->} delimiters, but not the delimiters
themselves.  For example, the comment \samp{<!--text-->} will
cause this method to be called with the argument \code{'text'}.  The
default method does nothing.
\end{methoddesc}

\begin{methoddesc}{report_unbalanced}{tag}
This method is called when an end tag is found which does not
correspond to any open element.
\end{methoddesc}

\begin{methoddesc}{unknown_starttag}{tag, attributes}
This method is called to process an unknown start tag.  It is intended
to be overridden by a derived class; the base class implementation
does nothing.
\end{methoddesc}

\begin{methoddesc}{unknown_endtag}{tag}
This method is called to process an unknown end tag.  It is intended
to be overridden by a derived class; the base class implementation
does nothing.
\end{methoddesc}

\begin{methoddesc}{unknown_charref}{ref}
This method is called to process unresolvable numeric character
references.  Refer to \method{handle_charref()} to determine what is
handled by default.  It is intended to be overridden by a derived
class; the base class implementation does nothing.
\end{methoddesc}

\begin{methoddesc}{unknown_entityref}{ref}
This method is called to process an unknown entity reference.  It is
intended to be overridden by a derived class; the base class
implementation does nothing.
\end{methoddesc}

Apart from overriding or extending the methods listed above, derived
classes may also define methods of the following form to define
processing of specific tags.  Tag names in the input stream are case
independent; the \var{tag} occurring in method names must be in lower
case:

\begin{methoddescni}{start_\var{tag}}{attributes}
This method is called to process an opening tag \var{tag}.  It has
preference over \method{do_\var{tag}()}.  The
\var{attributes} argument has the same meaning as described for
\method{handle_starttag()} above.
\end{methoddescni}

\begin{methoddescni}{do_\var{tag}}{attributes}
This method is called to process an opening tag \var{tag} that does
not come with a matching closing tag.  The \var{attributes} argument
has the same meaning as described for \method{handle_starttag()} above.
\end{methoddescni}

\begin{methoddescni}{end_\var{tag}}{}
This method is called to process a closing tag \var{tag}.
\end{methoddescni}

Note that the parser maintains a stack of open elements for which no
end tag has been found yet.  Only tags processed by
\method{start_\var{tag}()} are pushed on this stack.  Definition of an
\method{end_\var{tag}()} method is optional for these tags.  For tags
processed by \method{do_\var{tag}()} or by \method{unknown_tag()}, no
\method{end_\var{tag}()} method must be defined; if defined, it will
not be used.  If both \method{start_\var{tag}()} and
\method{do_\var{tag}()} methods exist for a tag, the
\method{start_\var{tag}()} method takes precedence.

\section{\module{htmllib} ---
         A parser for HTML documents}

\declaremodule{standard}{htmllib}
\modulesynopsis{A parser for HTML documents.}

\index{HTML}
\index{hypertext}


This module defines a class which can serve as a base for parsing text
files formatted in the HyperText Mark-up Language (HTML).  The class
is not directly concerned with I/O --- it must be provided with input
in string form via a method, and makes calls to methods of a
``formatter'' object in order to produce output.  The
\class{HTMLParser} class is designed to be used as a base class for
other classes in order to add functionality, and allows most of its
methods to be extended or overridden.  In turn, this class is derived
from and extends the \class{SGMLParser} class defined in module
\refmodule{sgmllib}\refstmodindex{sgmllib}.  The \class{HTMLParser}
implementation supports the HTML 2.0 language as described in
\rfc{1866}.  Two implementations of formatter objects are provided in
the \refmodule{formatter}\refstmodindex{formatter}\ module; refer to the
documentation for that module for information on the formatter
interface.
\withsubitem{(in module sgmllib)}{\ttindex{SGMLParser}}

The following is a summary of the interface defined by
\class{sgmllib.SGMLParser}:

\begin{itemize}

\item
The interface to feed data to an instance is through the \method{feed()}
method, which takes a string argument.  This can be called with as
little or as much text at a time as desired; \samp{p.feed(a);
p.feed(b)} has the same effect as \samp{p.feed(a+b)}.  When the data
contains complete HTML tags, these are processed immediately;
incomplete elements are saved in a buffer.  To force processing of all
unprocessed data, call the \method{close()} method.

For example, to parse the entire contents of a file, use:
\begin{verbatim}
parser.feed(open('myfile.html').read())
parser.close()
\end{verbatim}

\item
The interface to define semantics for HTML tags is very simple: derive
a class and define methods called \method{start_\var{tag}()},
\method{end_\var{tag}()}, or \method{do_\var{tag}()}.  The parser will
call these at appropriate moments: \method{start_\var{tag}} or
\method{do_\var{tag}()} is called when an opening tag of the form
\code{<\var{tag} ...>} is encountered; \method{end_\var{tag}()} is called
when a closing tag of the form \code{<\var{tag}>} is encountered.  If
an opening tag requires a corresponding closing tag, like \code{<H1>}
... \code{</H1>}, the class should define the \method{start_\var{tag}()}
method; if a tag requires no closing tag, like \code{<P>}, the class
should define the \method{do_\var{tag}()} method.

\end{itemize}

The module defines a single class:

\begin{classdesc}{HTMLParser}{formatter}
This is the basic HTML parser class.  It supports all entity names
required by the HTML 2.0 specification (\rfc{1866}).  It also defines
handlers for all HTML 2.0 and many HTML 3.0 and 3.2 elements.
\end{classdesc}


\begin{seealso}
  \seemodule{formatter}{Interface definition for transforming an
                        abstract flow of formatting events into
                        specific output events on writer objects.}
  \seemodule{HTMLParser}{Alternate HTML parser that offers a slightly
                         lower-level view of the input, but is
                         designed to work with XHTML, and does not
                         implement some of the SGML syntax not used in
                         ``HTML as deployed'' and which isn't legal
                         for XHTML.}
  \seemodule{htmlentitydefs}{Definition of replacement text for HTML
                             2.0 entities.}
  \seemodule{sgmllib}{Base class for \class{HTMLParser}.}
\end{seealso}


\subsection{HTMLParser Objects \label{html-parser-objects}}

In addition to tag methods, the \class{HTMLParser} class provides some
additional methods and instance variables for use within tag methods.

\begin{memberdesc}{formatter}
This is the formatter instance associated with the parser.
\end{memberdesc}

\begin{memberdesc}{nofill}
Boolean flag which should be true when whitespace should not be
collapsed, or false when it should be.  In general, this should only
be true when character data is to be treated as ``preformatted'' text,
as within a \code{<PRE>} element.  The default value is false.  This
affects the operation of \method{handle_data()} and \method{save_end()}.
\end{memberdesc}


\begin{methoddesc}{anchor_bgn}{href, name, type}
This method is called at the start of an anchor region.  The arguments
correspond to the attributes of the \code{<A>} tag with the same
names.  The default implementation maintains a list of hyperlinks
(defined by the \code{HREF} attribute for \code{<A>} tags) within the
document.  The list of hyperlinks is available as the data attribute
\member{anchorlist}.
\end{methoddesc}

\begin{methoddesc}{anchor_end}{}
This method is called at the end of an anchor region.  The default
implementation adds a textual footnote marker using an index into the
list of hyperlinks created by \method{anchor_bgn()}.
\end{methoddesc}

\begin{methoddesc}{handle_image}{source, alt\optional{, ismap\optional{, align\optional{, width\optional{, height}}}}}
This method is called to handle images.  The default implementation
simply passes the \var{alt} value to the \method{handle_data()}
method.
\end{methoddesc}

\begin{methoddesc}{save_bgn}{}
Begins saving character data in a buffer instead of sending it to the
formatter object.  Retrieve the stored data via \method{save_end()}.
Use of the \method{save_bgn()} / \method{save_end()} pair may not be
nested.
\end{methoddesc}

\begin{methoddesc}{save_end}{}
Ends buffering character data and returns all data saved since the
preceding call to \method{save_bgn()}.  If the \member{nofill} flag is
false, whitespace is collapsed to single spaces.  A call to this
method without a preceding call to \method{save_bgn()} will raise a
\exception{TypeError} exception.
\end{methoddesc}



\section{\module{htmlentitydefs} ---
         Definitions of HTML general entities}

\declaremodule{standard}{htmlentitydefs}
\modulesynopsis{Definitions of HTML general entities.}
\sectionauthor{Fred L. Drake, Jr.}{fdrake@acm.org}

This module defines three dictionaries, \code{name2codepoint},
\code{codepoint2name}, and \code{entitydefs}. \code{entitydefs} is
used by the \refmodule{htmllib} module to provide the
\member{entitydefs} member of the \class{HTMLParser} class.  The
definition provided here contains all the entities defined by XHTML 1.0 
that can be handled using simple textual substitution in the Latin-1
character set (ISO-8859-1).


\begin{datadesc}{entitydefs}
  A dictionary mapping XHTML 1.0 entity definitions to their
  replacement text in ISO Latin-1.

\end{datadesc}

\begin{datadesc}{name2codepoint}
  A dictionary that maps HTML entity names to the Unicode codepoints.
  \versionadded{2.3}
\end{datadesc}

\begin{datadesc}{codepoint2name}
  A dictionary that maps Unicode codepoints to HTML entity names.
  \versionadded{2.3}
\end{datadesc}

\section{Standard Module \sectcode{xmllib}}
% Author: Sjoerd Mullender
\label{module-xmllib}
\stmodindex{xmllib}
\index{XML}

This module defines a class \code{XMLParser} which serves as the basis 
for parsing text files formatted in XML (eXtended Markup Language).

The \code{XMLParser} class must be instantiated without arguments.  It 
has the following interface methods:

\setindexsubitem{(XMLParser method)}

\begin{funcdesc}{reset}{}
Reset the instance.  Loses all unprocessed data.  This is called
implicitly at the instantiation time.
\end{funcdesc}

\begin{funcdesc}{setnomoretags}{}
Stop processing tags.  Treat all following input as literal input
(CDATA).
\end{funcdesc}

\begin{funcdesc}{setliteral}{}
Enter literal mode (CDATA mode).
\end{funcdesc}

\begin{funcdesc}{feed}{data}
Feed some text to the parser.  It is processed insofar as it consists
of complete elements; incomplete data is buffered until more data is
fed or \code{close()} is called.
\end{funcdesc}

\begin{funcdesc}{close}{}
Force processing of all buffered data as if it were followed by an
end-of-file mark.  This method may be redefined by a derived class to
define additional processing at the end of the input, but the
redefined version should always call \code{XMLParser.close()}.
\end{funcdesc}

\begin{funcdesc}{translate_references}{data}
Translate all entity and character references in \code{data} and
returns the translated string.
\end{funcdesc}

\begin{funcdesc}{handle_xml}{encoding\, standalone}
This method is called when the \code{<?xml ...?>} tag is processed.
The arguments are the values of the encoding and standalone attributes 
in the tag.  Both encoding and standalone are optional.  The values
passed to \code{handle_xml} default to \code{None} and the string
\code{'no'} respectively.
\end{funcdesc}

\begin{funcdesc}{handle_doctype}{tag\, data}
This method is called when the \code{<!DOCTYPE...>} tag is processed.
The arguments are the name of the root element and the uninterpreted
contents of the tag, starting after the white space after the name of
the root element.
\end{funcdesc}

\begin{funcdesc}{handle_starttag}{tag\, method\, attributes}
This method is called to handle start tags for which a
\code{start_\var{tag}()} method has been defined.  The \code{tag}
argument is the name of the tag, and the \code{method} argument is the
bound method which should be used to support semantic interpretation
of the start tag.  The \var{attributes} argument is a dictionary of
attributes, the key being the \var{name} and the value being the
\var{value} of the attribute found inside the tag's \code{<>} brackets.
Character and entity references in the \var{value} have
been interpreted.  For instance, for the tag
\code{<A HREF="http://www.cwi.nl/">}, this method would be called as
\code{handle_starttag('A', self.start_A, \{'HREF': 'http://www.cwi.nl/'\})}.
The base implementation simply calls \code{method} with \code{attributes}
as the only argument.
\end{funcdesc}

\begin{funcdesc}{handle_endtag}{tag\, method}
This method is called to handle endtags for which an
\code{end_\var{tag}()} method has been defined.  The \code{tag}
argument is the name of the tag, and the
\code{method} argument is the bound method which should be used to
support semantic interpretation of the end tag.  If no
\code{end_\var{tag}()} method is defined for the closing element, this
handler is not called.  The base implementation simply calls
\code{method}.
\end{funcdesc}

\begin{funcdesc}{handle_data}{data}
This method is called to process arbitrary data.  It is intended to be
overridden by a derived class; the base class implementation does
nothing.
\end{funcdesc}

\begin{funcdesc}{handle_charref}{ref}
This method is called to process a character reference of the form
\samp{\&\#\var{ref};}.  \var{ref} can either be a decimal number,
or a hexadecimal number when preceded by \code{x}.
In the base implementation, \var{ref} must be a number in the
range 0-255.  It translates the character to \ASCII{} and calls the
method \code{handle_data()} with the character as argument.  If
\var{ref} is invalid or out of range, the method
\code{unknown_charref(\var{ref})} is called to handle the error.  A
subclass must override this method to provide support for character
references outside of the \ASCII{} range.
\end{funcdesc}

\begin{funcdesc}{handle_entityref}{ref}
This method is called to process a general entity reference of the form
\samp{\&\var{ref};} where \var{ref} is an general entity
reference.  It looks for \var{ref} in the instance (or class)
variable \code{entitydefs} which should be a mapping from entity names
to corresponding translations.
If a translation is found, it calls the method \code{handle_data()}
with the translation; otherwise, it calls the method
\code{unknown_entityref(\var{ref})}.  The default \code{entitydefs}
defines translations for \code{\&amp;}, \code{\&apos}, \code{\&gt;},
\code{\&lt;}, and \code{\&quot;}.
\end{funcdesc}

\begin{funcdesc}{handle_comment}{comment}
This method is called when a comment is encountered.  The
\code{comment} argument is a string containing the text between the
\samp{<!--} and \samp{-->} delimiters, but not the delimiters
themselves.  For example, the comment \samp{<!--text-->} will
cause this method to be called with the argument \code{'text'}.  The
default method does nothing.
\end{funcdesc}

\begin{funcdesc}{handle_cdata}{data}
This method is called when a CDATA element is encountered.  The
\code{data} argument is a string containing the text between the
\samp{<![CDATA[} and \samp{]]>} delimiters, but not the delimiters
themselves.  For example, the entity \samp{<![CDATA[text]]>} will
cause this method to be called with the argument \code{'text'}.  The
default method does nothing.
\end{funcdesc}

\begin{funcdesc}{handle_proc}{name\, data}
This method is called when a processing instruction (PI) is encountered.  The
\code{name} is the PI target, and the \code{data} argument is a
string containing the text between the PI target and the closing delimiter,
but not the delimiter itself.  For example, the instruction
\samp{<?XML text?>} will cause this method to be called with the
arguments \code{'XML'} and \code{'text'}.  The default method does
nothing.  Note that if a document starts with a \code{<?xml ...?>}
tag, \code{handle_xml} is called to handle it.
\end{funcdesc}

\begin{funcdesc}{handle_special}{data}
This method is called when a declaration is encountered.  The
\code{data} argument is a string containing the text between the
\samp{<!} and \samp{>} delimiters, but not the delimiters
themselves.  For example, the entity \samp{<!ENTITY text>} will
cause this method to be called with the argument \code{'ENTITY text'}.  The
default method does nothing.  Note that \code{<!DOCTYPE ...>} is
handled separately if it is located at the start of the document.
\end{funcdesc}

\begin{funcdesc}{syntax_error}{message}
This method is called when a syntax error is encountered.  The
\code{message} is a description of what was wrong.  The default method 
raises a \code{RuntimeError} exception.  If this method is overridden, 
it is permissable for it to return.  This method is only called when
the error can be recovered from.  Unrecoverable errors raise a
\code{RuntimeError} without first calling \code{syntax_error}.
\end{funcdesc}

\begin{funcdesc}{unknown_starttag}{tag\, attributes}
This method is called to process an unknown start tag.  It is intended
to be overridden by a derived class; the base class implementation
does nothing.
\end{funcdesc}

\begin{funcdesc}{unknown_endtag}{tag}
This method is called to process an unknown end tag.  It is intended
to be overridden by a derived class; the base class implementation
does nothing.
\end{funcdesc}

\begin{funcdesc}{unknown_charref}{ref}
This method is called to process unresolvable numeric character
references.  It is intended to be overridden by a derived class; the
base class implementation does nothing.
\end{funcdesc}

\begin{funcdesc}{unknown_entityref}{ref}
This method is called to process an unknown entity reference.  It is
intended to be overridden by a derived class; the base class
implementation does nothing.
\end{funcdesc}

Apart from overriding or extending the methods listed above, derived
classes may also define methods and variables of the following form to
define processing of specific tags.  Tag names in the input stream are
case dependent; the \var{tag} occurring in method names must be in the
correct case:

\begin{funcdescni}{start_\var{tag}}{attributes}
This method is called to process an opening tag \var{tag}.  The
\var{attributes} argument has the same meaning as described for
\code{handle_starttag()} above.  In fact, the base implementation of
\code{handle_starttag()} calls this method.
\end{funcdescni}

\begin{funcdescni}{end_\var{tag}}{}
This method is called to process a closing tag \var{tag}.
\end{funcdescni}

\begin{datadescni}{\var{tag}_attributes}
If a class or instance variable \code{\var{tag}_attributes} exists, it 
should be a list or a dictionary.  If a list, the elements of the list 
are the valid attributes for the element \var{tag}; if a dictionary,
the keys are the valid attributes for the element \var{tag}, and the
values the default values of the attributes, or \code{None} if there
is no default.
In addition to the attributes that were present in the tag, the
attribute dictionary that is passed to \code{handle_starttag()} and
\code{unknown_starttag()} contains values for all attributes that have a
default value.
\end{datadescni}

\section{Standard Module \sectcode{formatter}}
\label{module-formatter}
\stmodindex{formatter}


This module supports two interface definitions, each with mulitple
implementations.  The \emph{formatter} interface is used by the
\class{HTMLParser} class of the \module{htmllib} module, and the
\emph{writer} interface is required by the formatter interface.
\withsubitem{(im module htmllib)}{\ttindex{HTMLParser}}

Formatter objects transform an abstract flow of formatting events into
specific output events on writer objects.  Formatters manage several
stack structures to allow various properties of a writer object to be
changed and restored; writers need not be able to handle relative
changes nor any sort of ``change back'' operation.  Specific writer
properties which may be controlled via formatter objects are
horizontal alignment, font, and left margin indentations.  A mechanism
is provided which supports providing arbitrary, non-exclusive style
settings to a writer as well.  Additional interfaces facilitate
formatting events which are not reversible, such as paragraph
separation.

Writer objects encapsulate device interfaces.  Abstract devices, such
as file formats, are supported as well as physical devices.  The
provided implementations all work with abstract devices.  The
interface makes available mechanisms for setting the properties which
formatter objects manage and inserting data into the output.


\subsection{The Formatter Interface}

Interfaces to create formatters are dependent on the specific
formatter class being instantiated.  The interfaces described below
are the required interfaces which all formatters must support once
initialized.

One data element is defined at the module level:


\begin{datadesc}{AS_IS}
Value which can be used in the font specification passed to the
\code{push_font()} method described below, or as the new value to any
other \code{push_\var{property}()} method.  Pushing the \code{AS_IS}
value allows the corresponding \code{pop_\var{property}()} method to
be called without having to track whether the property was changed.
\end{datadesc}

The following attributes are defined for formatter instance objects:


\begin{memberdesc}[formatter]{writer}
The writer instance with which the formatter interacts.
\end{memberdesc}


\begin{methoddesc}[formatter]{end_paragraph}{blanklines}
Close any open paragraphs and insert at least \var{blanklines}
before the next paragraph.
\end{methoddesc}

\begin{methoddesc}[formatter]{add_line_break}{}
Add a hard line break if one does not already exist.  This does not
break the logical paragraph.
\end{methoddesc}

\begin{methoddesc}[formatter]{add_hor_rule}{*args, **kw}
Insert a horizontal rule in the output.  A hard break is inserted if
there is data in the current paragraph, but the logical paragraph is
not broken.  The arguments and keywords are passed on to the writer's
\method{send_line_break()} method.
\end{methoddesc}

\begin{methoddesc}[formatter]{add_flowing_data}{data}
Provide data which should be formatted with collapsed whitespaces.
Whitespace from preceeding and successive calls to
\method{add_flowing_data()} is considered as well when the whitespace
collapse is performed.  The data which is passed to this method is
expected to be word-wrapped by the output device.  Note that any
word-wrapping still must be performed by the writer object due to the
need to rely on device and font information.
\end{methoddesc}

\begin{methoddesc}[formatter]{add_literal_data}{data}
Provide data which should be passed to the writer unchanged.
Whitespace, including newline and tab characters, are considered legal
in the value of \var{data}.  
\end{methoddesc}

\begin{methoddesc}[formatter]{add_label_data}{format, counter}
Insert a label which should be placed to the left of the current left
margin.  This should be used for constructing bulleted or numbered
lists.  If the \var{format} value is a string, it is interpreted as a
format specification for \var{counter}, which should be an integer.
The result of this formatting becomes the value of the label; if
\var{format} is not a string it is used as the label value directly.
The label value is passed as the only argument to the writer's
\method{send_label_data()} method.  Interpretation of non-string label
values is dependent on the associated writer.

Format specifications are strings which, in combination with a counter
value, are used to compute label values.  Each character in the format
string is copied to the label value, with some characters recognized
to indicate a transform on the counter value.  Specifically, the
character \character{1} represents the counter value formatter as an
arabic number, the characters \character{A} and \character{a}
represent alphabetic representations of the counter value in upper and
lower case, respectively, and \character{I} and \character{i}
represent the counter value in Roman numerals, in upper and lower
case.  Note that the alphabetic and roman transforms require that the
counter value be greater than zero.
\end{methoddesc}

\begin{methoddesc}[formatter]{flush_softspace}{}
Send any pending whitespace buffered from a previous call to
\method{add_flowing_data()} to the associated writer object.  This
should be called before any direct manipulation of the writer object.
\end{methoddesc}

\begin{methoddesc}[formatter]{push_alignment}{align}
Push a new alignment setting onto the alignment stack.  This may be
\constant{AS_IS} if no change is desired.  If the alignment value is
changed from the previous setting, the writer's \method{new_alignment()}
method is called with the \var{align} value.
\end{methoddesc}

\begin{methoddesc}[formatter]{pop_alignment}{}
Restore the previous alignment.
\end{methoddesc}

\begin{methoddesc}[formatter]{push_font}{\code{(}size, italic, bold, teletype\code{)}}
Change some or all font properties of the writer object.  Properties
which are not set to \constant{AS_IS} are set to the values passed in
while others are maintained at their current settings.  The writer's
\method{new_font()} method is called with the fully resolved font
specification.
\end{methoddesc}

\begin{methoddesc}[formatter]{pop_font}{}
Restore the previous font.
\end{methoddesc}

\begin{methoddesc}[formatter]{push_margin}{margin}
Increase the number of left margin indentations by one, associating
the logical tag \var{margin} with the new indentation.  The initial
margin level is \code{0}.  Changed values of the logical tag must be
true values; false values other than \constant{AS_IS} are not
sufficient to change the margin.
\end{methoddesc}

\begin{methoddesc}[formatter]{pop_margin}{}
Restore the previous margin.
\end{methoddesc}

\begin{methoddesc}[formatter]{push_style}{*styles}
Push any number of arbitrary style specifications.  All styles are
pushed onto the styles stack in order.  A tuple representing the
entire stack, including \constant{AS_IS} values, is passed to the
writer's \method{new_styles()} method.
\end{methoddesc}

\begin{methoddesc}[formatter]{pop_style}{\optional{n\code{ = 1}}}
Pop the last \var{n} style specifications passed to
\method{push_style()}.  A tuple representing the revised stack,
including \constant{AS_IS} values, is passed to the writer's
\method{new_styles()} method.
\end{methoddesc}

\begin{methoddesc}[formatter]{set_spacing}{spacing}
Set the spacing style for the writer.
\end{methoddesc}

\begin{methoddesc}[formatter]{assert_line_data}{\optional{flag\code{ = 1}}}
Inform the formatter that data has been added to the current paragraph
out-of-band.  This should be used when the writer has been manipulated
directly.  The optional \var{flag} argument can be set to false if
the writer manipulations produced a hard line break at the end of the
output.
\end{methoddesc}


\subsection{Formatter Implementations}

Two implementations of formatter objects are provided by this module.
Most applications may use one of these classes without modification or
subclassing.

\begin{classdesc}{NullFormatter}{\optional{writer}}
A formatter which does nothing.  If \var{writer} is omitted, a
\class{NullWriter} instance is created.  No methods of the writer are
called by \class{NullFormatter} instances.  Implementations should
inherit from this class if implementing a writer interface but don't
need to inherit any implementation.
\end{classdesc}

\begin{classdesc}{AbstractFormatter}{writer}
The standard formatter.  This implementation has demonstrated wide
applicability to many writers, and may be used directly in most
circumstances.  It has been used to implement a full-featured
world-wide web browser.
\end{classdesc}



\subsection{The Writer Interface}

Interfaces to create writers are dependent on the specific writer
class being instantiated.  The interfaces described below are the
required interfaces which all writers must support once initialized.
Note that while most applications can use the
\class{AbstractFormatter} class as a formatter, the writer must
typically be provided by the application.


\begin{methoddesc}[writer]{flush}{}
Flush any buffered output or device control events.
\end{methoddesc}

\begin{methoddesc}[writer]{new_alignment}{align}
Set the alignment style.  The \var{align} value can be any object,
but by convention is a string or \code{None}, where \code{None}
indicates that the writer's ``preferred'' alignment should be used.
Conventional \var{align} values are \code{'left'}, \code{'center'},
\code{'right'}, and \code{'justify'}.
\end{methoddesc}

\begin{methoddesc}[writer]{new_font}{font}
Set the font style.  The value of \var{font} will be \code{None},
indicating that the device's default font should be used, or a tuple
of the form \code{(}\var{size}, \var{italic}, \var{bold},
\var{teletype}\code{)}.  Size will be a string indicating the size of
font that should be used; specific strings and their interpretation
must be defined by the application.  The \var{italic}, \var{bold}, and
\var{teletype} values are boolean indicators specifying which of those
font attributes should be used.
\end{methoddesc}

\begin{methoddesc}[writer]{new_margin}{margin, level}
Set the margin level to the integer \var{level} and the logical tag
to \var{margin}.  Interpretation of the logical tag is at the
writer's discretion; the only restriction on the value of the logical
tag is that it not be a false value for non-zero values of
\var{level}.
\end{methoddesc}

\begin{methoddesc}[writer]{new_spacing}{spacing}
Set the spacing style to \var{spacing}.
\end{methoddesc}

\begin{methoddesc}[writer]{new_styles}{styles}
Set additional styles.  The \var{styles} value is a tuple of
arbitrary values; the value \constant{AS_IS} should be ignored.  The
\var{styles} tuple may be interpreted either as a set or as a stack
depending on the requirements of the application and writer
implementation.
\end{methoddesc}

\begin{methoddesc}[writer]{send_line_break}{}
Break the current line.
\end{methoddesc}

\begin{methoddesc}[writer]{send_paragraph}{blankline}
Produce a paragraph separation of at least \var{blankline} blank
lines, or the equivelent.  The \var{blankline} value will be an
integer.
\end{methoddesc}

\begin{methoddesc}[writer]{send_hor_rule}{*args, **kw}
Display a horizontal rule on the output device.  The arguments to this
method are entirely application- and writer-specific, and should be
interpreted with care.  The method implementation may assume that a
line break has already been issued via \method{send_line_break()}.
\end{methoddesc}

\begin{methoddesc}[writer]{send_flowing_data}{data}
Output character data which may be word-wrapped and re-flowed as
needed.  Within any sequence of calls to this method, the writer may
assume that spans of multiple whitespace characters have been
collapsed to single space characters.
\end{methoddesc}

\begin{methoddesc}[writer]{send_literal_data}{data}
Output character data which has already been formatted
for display.  Generally, this should be interpreted to mean that line
breaks indicated by newline characters should be preserved and no new
line breaks should be introduced.  The data may contain embedded
newline and tab characters, unlike data provided to the
\method{send_formatted_data()} interface.
\end{methoddesc}

\begin{methoddesc}[writer]{send_label_data}{data}
Set \var{data} to the left of the current left margin, if possible.
The value of \var{data} is not restricted; treatment of non-string
values is entirely application- and writer-dependent.  This method
will only be called at the beginning of a line.
\end{methoddesc}


\subsection{Writer Implementations}

Three implementations of the writer object interface are provided as
examples by this module.  Most applications will need to derive new
writer classes from the \class{NullWriter} class.

\begin{classdesc}{NullWriter}{}
A writer which only provides the interface definition; no actions are
taken on any methods.  This should be the base class for all writers
which do not need to inherit any implementation methods.
\end{classdesc}

\begin{classdesc}{AbstractWriter}{}
A writer which can be used in debugging formatters, but not much
else.  Each method simply announces itself by printing its name and
arguments on standard output.
\end{classdesc}

\begin{classdesc}{DumbWriter}{\optional{file\optional{, maxcol\code{ = 72}}}}
Simple writer class which writes output on the file object passed in
as \var{file} or, if \var{file} is omitted, on standard output.  The
output is simply word-wrapped to the number of columns specified by
\var{maxcol}.  This class is suitable for reflowing a sequence of
paragraphs.
\end{classdesc}

\section{\module{rfc822} ---
         Parse RFC 2822 mail headers}

\declaremodule{standard}{rfc822}
\modulesynopsis{Parse \rfc{2822} style mail messages.}

This module defines a class, \class{Message}, which represents an
``email message'' as defined by the Internet standard
\rfc{2822}.\footnote{This module originally conformed to \rfc{822},
hence the name.  Since then, \rfc{2822} has been released as an
update to \rfc{822}.  This module should be considered
\rfc{2822}-conformant, especially in cases where the
syntax or semantics have changed since \rfc{822}.}  Such messages
consist of a collection of message headers, and a message body.  This
module also defines a helper class
\class{AddressList} for parsing \rfc{2822} addresses.  Please refer to
the RFC for information on the specific syntax of \rfc{2822} messages.

The \refmodule{mailbox}\refstmodindex{mailbox} module provides classes 
to read mailboxes produced by various end-user mail programs.

\begin{classdesc}{Message}{file\optional{, seekable}}
A \class{Message} instance is instantiated with an input object as
parameter.  Message relies only on the input object having a
\method{readline()} method; in particular, ordinary file objects
qualify.  Instantiation reads headers from the input object up to a
delimiter line (normally a blank line) and stores them in the
instance.  The message body, following the headers, is not consumed.

This class can work with any input object that supports a
\method{readline()} method.  If the input object has seek and tell
capability, the \method{rewindbody()} method will work; also, illegal
lines will be pushed back onto the input stream.  If the input object
lacks seek but has an \method{unread()} method that can push back a
line of input, \class{Message} will use that to push back illegal
lines.  Thus this class can be used to parse messages coming from a
buffered stream.

The optional \var{seekable} argument is provided as a workaround for
certain stdio libraries in which \cfunction{tell()} discards buffered
data before discovering that the \cfunction{lseek()} system call
doesn't work.  For maximum portability, you should set the seekable
argument to zero to prevent that initial \method{tell()} when passing
in an unseekable object such as a a file object created from a socket
object.

Input lines as read from the file may either be terminated by CR-LF or
by a single linefeed; a terminating CR-LF is replaced by a single
linefeed before the line is stored.

All header matching is done independent of upper or lower case;
e.g.\ \code{\var{m}['From']}, \code{\var{m}['from']} and
\code{\var{m}['FROM']} all yield the same result.
\end{classdesc}

\begin{classdesc}{AddressList}{field}
You may instantiate the \class{AddressList} helper class using a single
string parameter, a comma-separated list of \rfc{2822} addresses to be
parsed.  (The parameter \code{None} yields an empty list.)
\end{classdesc}

\begin{funcdesc}{quote}{str}
Return a new string with backslashes in \var{str} replaced by two
backslashes and double quotes replaced by backslash-double quote.
\end{funcdesc}

\begin{funcdesc}{unquote}{str}
Return a new string which is an \emph{unquoted} version of \var{str}.
If \var{str} ends and begins with double quotes, they are stripped
off.  Likewise if \var{str} ends and begins with angle brackets, they
are stripped off.
\end{funcdesc}

\begin{funcdesc}{parseaddr}{address}
Parse \var{address}, which should be the value of some address-containing
field such as \code{To:} or \code{Cc:}, into its constituent
``realname'' and ``email address'' parts.  Returns a tuple of that
information, unless the parse fails, in which case a 2-tuple of
\code{(None, None)} is returned.
\end{funcdesc}

\begin{funcdesc}{dump_address_pair}{pair}
The inverse of \method{parseaddr()}, this takes a 2-tuple of the form
\code{(realname, email_address)} and returns the string value suitable
for a \code{To:} or \code{Cc:} header.  If the first element of
\var{pair} is false, then the second element is returned unmodified.
\end{funcdesc}

\begin{funcdesc}{parsedate}{date}
Attempts to parse a date according to the rules in \rfc{2822}.
however, some mailers don't follow that format as specified, so
\function{parsedate()} tries to guess correctly in such cases. 
\var{date} is a string containing an \rfc{2822} date, such as 
\code{'Mon, 20 Nov 1995 19:12:08 -0500'}.  If it succeeds in parsing
the date, \function{parsedate()} returns a 9-tuple that can be passed
directly to \function{time.mktime()}; otherwise \code{None} will be
returned.  Note that fields 6, 7, and 8 of the result tuple are not
usable.
\end{funcdesc}

\begin{funcdesc}{parsedate_tz}{date}
Performs the same function as \function{parsedate()}, but returns
either \code{None} or a 10-tuple; the first 9 elements make up a tuple
that can be passed directly to \function{time.mktime()}, and the tenth
is the offset of the date's timezone from UTC (which is the official
term for Greenwich Mean Time).  (Note that the sign of the timezone
offset is the opposite of the sign of the \code{time.timezone}
variable for the same timezone; the latter variable follows the
\POSIX{} standard while this module follows \rfc{2822}.)  If the input
string has no timezone, the last element of the tuple returned is
\code{None}.  Note that fields 6, 7, and 8 of the result tuple are not
usable.
\end{funcdesc}

\begin{funcdesc}{mktime_tz}{tuple}
Turn a 10-tuple as returned by \function{parsedate_tz()} into a UTC
timestamp.  It the timezone item in the tuple is \code{None}, assume
local time.  Minor deficiency: this first interprets the first 8
elements as a local time and then compensates for the timezone
difference; this may yield a slight error around daylight savings time
switch dates.  Not enough to worry about for common use.
\end{funcdesc}


\begin{seealso}
  \seemodule{mailbox}{Classes to read various mailbox formats produced 
                      by end-user mail programs.}
  \seemodule{mimetools}{Subclass of rfc.Message that handles MIME encoded
		        messages.} 
\end{seealso}


\subsection{Message Objects \label{message-objects}}

A \class{Message} instance has the following methods:

\begin{methoddesc}{rewindbody}{}
Seek to the start of the message body.  This only works if the file
object is seekable.
\end{methoddesc}

\begin{methoddesc}{isheader}{line}
Returns a line's canonicalized fieldname (the dictionary key that will
be used to index it) if the line is a legal \rfc{2822} header; otherwise
returns None (implying that parsing should stop here and the line be
pushed back on the input stream).  It is sometimes useful to override
this method in a subclass.
\end{methoddesc}

\begin{methoddesc}{islast}{line}
Return true if the given line is a delimiter on which Message should
stop.  The delimiter line is consumed, and the file object's read
location positioned immediately after it.  By default this method just
checks that the line is blank, but you can override it in a subclass.
\end{methoddesc}

\begin{methoddesc}{iscomment}{line}
Return true if the given line should be ignored entirely, just skipped.
By default this is a stub that always returns false, but you can
override it in a subclass.
\end{methoddesc}

\begin{methoddesc}{getallmatchingheaders}{name}
Return a list of lines consisting of all headers matching
\var{name}, if any.  Each physical line, whether it is a continuation
line or not, is a separate list item.  Return the empty list if no
header matches \var{name}.
\end{methoddesc}

\begin{methoddesc}{getfirstmatchingheader}{name}
Return a list of lines comprising the first header matching
\var{name}, and its continuation line(s), if any.  Return
\code{None} if there is no header matching \var{name}.
\end{methoddesc}

\begin{methoddesc}{getrawheader}{name}
Return a single string consisting of the text after the colon in the
first header matching \var{name}.  This includes leading whitespace,
the trailing linefeed, and internal linefeeds and whitespace if there
any continuation line(s) were present.  Return \code{None} if there is
no header matching \var{name}.
\end{methoddesc}

\begin{methoddesc}{getheader}{name\optional{, default}}
Like \code{getrawheader(\var{name})}, but strip leading and trailing
whitespace.  Internal whitespace is not stripped.  The optional
\var{default} argument can be used to specify a different default to
be returned when there is no header matching \var{name}.
\end{methoddesc}

\begin{methoddesc}{get}{name\optional{, default}}
An alias for \method{getheader()}, to make the interface more compatible 
with regular dictionaries.
\end{methoddesc}

\begin{methoddesc}{getaddr}{name}
Return a pair \code{(\var{full name}, \var{email address})} parsed
from the string returned by \code{getheader(\var{name})}.  If no
header matching \var{name} exists, return \code{(None, None)};
otherwise both the full name and the address are (possibly empty)
strings.

Example: If \var{m}'s first \code{From} header contains the string
\code{'jack@cwi.nl (Jack Jansen)'}, then
\code{m.getaddr('From')} will yield the pair
\code{('Jack Jansen', 'jack@cwi.nl')}.
If the header contained
\code{'Jack Jansen <jack@cwi.nl>'} instead, it would yield the
exact same result.
\end{methoddesc}

\begin{methoddesc}{getaddrlist}{name}
This is similar to \code{getaddr(\var{list})}, but parses a header
containing a list of email addresses (e.g.\ a \code{To} header) and
returns a list of \code{(\var{full name}, \var{email address})} pairs
(even if there was only one address in the header).  If there is no
header matching \var{name}, return an empty list.

If multiple headers exist that match the named header (e.g. if there
are several \code{Cc} headers), all are parsed for addresses.  Any
continuation lines the named headers contain are also parsed.
\end{methoddesc}

\begin{methoddesc}{getdate}{name}
Retrieve a header using \method{getheader()} and parse it into a 9-tuple
compatible with \function{time.mktime()}; note that fields 6, 7, and 8 
are not usable.  If there is no header matching
\var{name}, or it is unparsable, return \code{None}.

Date parsing appears to be a black art, and not all mailers adhere to
the standard.  While it has been tested and found correct on a large
collection of email from many sources, it is still possible that this
function may occasionally yield an incorrect result.
\end{methoddesc}

\begin{methoddesc}{getdate_tz}{name}
Retrieve a header using \method{getheader()} and parse it into a
10-tuple; the first 9 elements will make a tuple compatible with
\function{time.mktime()}, and the 10th is a number giving the offset
of the date's timezone from UTC.  Note that fields 6, 7, and 8 
are not usable.  Similarly to \method{getdate()}, if
there is no header matching \var{name}, or it is unparsable, return
\code{None}. 
\end{methoddesc}

\class{Message} instances also support a limited mapping interface.
In particular: \code{\var{m}[name]} is like
\code{\var{m}.getheader(name)} but raises \exception{KeyError} if
there is no matching header; and \code{len(\var{m})},
\code{\var{m}.get(name\optional{, deafult})},
\code{\var{m}.has_key(name)}, \code{\var{m}.keys()},
\code{\var{m}.values()} \code{\var{m}.items()}, and
\code{\var{m}.setdefault(name\optional{, default})} act as expected,
with the one difference that \method{get()} and \method{setdefault()}
use an empty string as the default value.  \class{Message} instances
also support the mapping writable interface \code{\var{m}[name] =
value} and \code{del \var{m}[name]}.  \class{Message} objects do not
support the \method{clear()}, \method{copy()}, \method{popitem()}, or
\method{update()} methods of the mapping interface.  (Support for
\method{get()} and \method{setdefault()} was only added in Python
2.2.)

Finally, \class{Message} instances have two public instance variables:

\begin{memberdesc}{headers}
A list containing the entire set of header lines, in the order in
which they were read (except that setitem calls may disturb this
order). Each line contains a trailing newline.  The
blank line terminating the headers is not contained in the list.
\end{memberdesc}

\begin{memberdesc}{fp}
The file or file-like object passed at instantiation time.  This can
be used to read the message content.
\end{memberdesc}


\subsection{AddressList Objects \label{addresslist-objects}}

An \class{AddressList} instance has the following methods:

\begin{methoddesc}{__len__}{}
Return the number of addresses in the address list.
\end{methoddesc}

\begin{methoddesc}{__str__}{}
Return a canonicalized string representation of the address list.
Addresses are rendered in "name" <host@domain> form, comma-separated.
\end{methoddesc}

\begin{methoddesc}{__add__}{alist}
Return a new \class{AddressList} instance that contains all addresses
in both \class{AddressList} operands, with duplicates removed (set
union).
\end{methoddesc}

\begin{methoddesc}{__iadd__}{alist}
In-place version of \method{__add__()}; turns this \class{AddressList}
instance into the union of itself and the right-hand instance,
\var{alist}.
\end{methoddesc}

\begin{methoddesc}{__sub__}{alist}
Return a new \class{AddressList} instance that contains every address
in the left-hand \class{AddressList} operand that is not present in
the right-hand address operand (set difference).
\end{methoddesc}

\begin{methoddesc}{__isub__}{alist}
In-place version of \method{__sub__()}, removing addresses in this
list which are also in \var{alist}.
\end{methoddesc}


Finally, \class{AddressList} instances have one public instance variable:

\begin{memberdesc}{addresslist}
A list of tuple string pairs, one per address.  In each member, the
first is the canonicalized name part, the second is the
actual route-address (\character{@}-separated username-host.domain
pair).
\end{memberdesc}

\section{\module{mimetools} ---
         Tools for parsing MIME messages}

\declaremodule{standard}{mimetools}
\modulesynopsis{Tools for parsing MIME-style message bodies.}

\deprecated{2.3}{The \refmodule{email} package should be used in
                 preference to the \module{mimetools} module.  This
                 module is present only to maintain backward
                 compatibility.}

This module defines a subclass of the
\refmodule{rfc822}\refstmodindex{rfc822} module's
\class{Message} class and a number of utility functions that are
useful for the manipulation for MIME multipart or encoded message.

It defines the following items:

\begin{classdesc}{Message}{fp\optional{, seekable}}
Return a new instance of the \class{Message} class.  This is a
subclass of the \class{rfc822.Message} class, with some additional
methods (see below).  The \var{seekable} argument has the same meaning
as for \class{rfc822.Message}.
\end{classdesc}

\begin{funcdesc}{choose_boundary}{}
Return a unique string that has a high likelihood of being usable as a
part boundary.  The string has the form
\code{'\var{hostipaddr}.\var{uid}.\var{pid}.\var{timestamp}.\var{random}'}.
\end{funcdesc}

\begin{funcdesc}{decode}{input, output, encoding}
Read data encoded using the allowed MIME \var{encoding} from open file
object \var{input} and write the decoded data to open file object
\var{output}.  Valid values for \var{encoding} include
\code{'base64'}, \code{'quoted-printable'}, \code{'uuencode'},
\code{'x-uuencode'}, \code{'uue'}, \code{'x-uue'}, \code{'7bit'}, and 
\code{'8bit'}.  Decoding messages encoded in \code{'7bit'} or \code{'8bit'}
has no effect.  The input is simply copied to the output.
\end{funcdesc}

\begin{funcdesc}{encode}{input, output, encoding}
Read data from open file object \var{input} and write it encoded using
the allowed MIME \var{encoding} to open file object \var{output}.
Valid values for \var{encoding} are the same as for \method{decode()}.
\end{funcdesc}

\begin{funcdesc}{copyliteral}{input, output}
Read lines from open file \var{input} until \EOF{} and write them to
open file \var{output}.
\end{funcdesc}

\begin{funcdesc}{copybinary}{input, output}
Read blocks until \EOF{} from open file \var{input} and write them to
open file \var{output}.  The block size is currently fixed at 8192.
\end{funcdesc}


\begin{seealso}
  \seemodule{email}{Comprehensive email handling package; supercedes
                    the \module{mimetools} module.}
  \seemodule{rfc822}{Provides the base class for
                     \class{mimetools.Message}.}
  \seemodule{multifile}{Support for reading files which contain
                        distinct parts, such as MIME data.}
  \seeurl{http://www.cs.uu.nl/wais/html/na-dir/mail/mime-faq/.html}{
          The MIME Frequently Asked Questions document.  For an
          overview of MIME, see the answer to question 1.1 in Part 1
          of this document.}
\end{seealso}


\subsection{Additional Methods of Message Objects
            \label{mimetools-message-objects}}

The \class{Message} class defines the following methods in
addition to the \class{rfc822.Message} methods:

\begin{methoddesc}{getplist}{}
Return the parameter list of the \mailheader{Content-Type} header.
This is a list of strings.  For parameters of the form
\samp{\var{key}=\var{value}}, \var{key} is converted to lower case but
\var{value} is not.  For example, if the message contains the header
\samp{Content-type: text/html; spam=1; Spam=2; Spam} then
\method{getplist()} will return the Python list \code{['spam=1',
'spam=2', 'Spam']}.
\end{methoddesc}

\begin{methoddesc}{getparam}{name}
Return the \var{value} of the first parameter (as returned by
\method{getplist()} of the form \samp{\var{name}=\var{value}} for the
given \var{name}.  If \var{value} is surrounded by quotes of the form
`\code{<}...\code{>}' or `\code{"}...\code{"}', these are removed.
\end{methoddesc}

\begin{methoddesc}{getencoding}{}
Return the encoding specified in the
\mailheader{Content-Transfer-Encoding} message header.  If no such
header exists, return \code{'7bit'}.  The encoding is converted to
lower case.
\end{methoddesc}

\begin{methoddesc}{gettype}{}
Return the message type (of the form \samp{\var{type}/\var{subtype}})
as specified in the \mailheader{Content-Type} header.  If no such
header exists, return \code{'text/plain'}.  The type is converted to
lower case.
\end{methoddesc}

\begin{methoddesc}{getmaintype}{}
Return the main type as specified in the \mailheader{Content-Type}
header.  If no such header exists, return \code{'text'}.  The main
type is converted to lower case.
\end{methoddesc}

\begin{methoddesc}{getsubtype}{}
Return the subtype as specified in the \mailheader{Content-Type}
header.  If no such header exists, return \code{'plain'}.  The subtype
is converted to lower case.
\end{methoddesc}

\section{\module{multifile} ---
         Support for files containing distinct parts}

\declaremodule{standard}{multifile}
\modulesynopsis{Support for reading files which contain distinct
                parts, such as some MIME data.}
\sectionauthor{Eric S. Raymond}{esr@snark.thyrsus.com}


The \class{MultiFile} object enables you to treat sections of a text
file as file-like input objects, with \code{''} being returned by
\method{readline()} when a given delimiter pattern is encountered.  The
defaults of this class are designed to make it useful for parsing
MIME multipart messages, but by subclassing it and overriding methods 
it can be easily adapted for more general use.

\begin{classdesc}{MultiFile}{fp\optional{, seekable}}
Create a multi-file.  You must instantiate this class with an input
object argument for the \class{MultiFile} instance to get lines from,
such as as a file object returned by \function{open()}.

\class{MultiFile} only ever looks at the input object's
\method{readline()}, \method{seek()} and \method{tell()} methods, and
the latter two are only needed if you want random access to the
individual MIME parts. To use \class{MultiFile} on a non-seekable
stream object, set the optional \var{seekable} argument to false; this
will prevent using the input object's \method{seek()} and
\method{tell()} methods.
\end{classdesc}

It will be useful to know that in \class{MultiFile}'s view of the world, text
is composed of three kinds of lines: data, section-dividers, and
end-markers.  MultiFile is designed to support parsing of
messages that may have multiple nested message parts, each with its
own pattern for section-divider and end-marker lines.


\subsection{MultiFile Objects \label{MultiFile-objects}}

A \class{MultiFile} instance has the following methods:

\begin{methoddesc}{push}{str}
Push a boundary string.  When an appropriately decorated version of
this boundary is found as an input line, it will be interpreted as a
section-divider or end-marker.  All subsequent
reads will return the empty string to indicate end-of-file, until a
call to \method{pop()} removes the boundary a or \method{next()} call
reenables it.

It is possible to push more than one boundary.  Encountering the
most-recently-pushed boundary will return EOF; encountering any other
boundary will raise an error.
\end{methoddesc}

\begin{methoddesc}{readline}{str}
Read a line.  If the line is data (not a section-divider or end-marker
or real EOF) return it.  If the line matches the most-recently-stacked
boundary, return \code{''} and set \code{self.last} to 1 or 0 according as
the match is or is not an end-marker.  If the line matches any other
stacked boundary, raise an error.  On encountering end-of-file on the
underlying stream object, the method raises \exception{Error} unless
all boundaries have been popped.
\end{methoddesc}

\begin{methoddesc}{readlines}{str}
Return all lines remaining in this part as a list of strings.
\end{methoddesc}

\begin{methoddesc}{read}{}
Read all lines, up to the next section.  Return them as a single
(multiline) string.  Note that this doesn't take a size argument!
\end{methoddesc}

\begin{methoddesc}{next}{}
Skip lines to the next section (that is, read lines until a
section-divider or end-marker has been consumed).  Return true if
there is such a section, false if an end-marker is seen.  Re-enable
the most-recently-pushed boundary.
\end{methoddesc}

\begin{methoddesc}{pop}{}
Pop a section boundary.  This boundary will no longer be interpreted
as EOF.
\end{methoddesc}

\begin{methoddesc}{seek}{pos\optional{, whence}}
Seek.  Seek indices are relative to the start of the current section.
The \var{pos} and \var{whence} arguments are interpreted as for a file
seek.
\end{methoddesc}

\begin{methoddesc}{tell}{}
Return the file position relative to the start of the current section.
\end{methoddesc}

\begin{methoddesc}{is_data}{str}
Return true if \var{str} is data and false if it might be a section
boundary.  As written, it tests for a prefix other than \code{'-}\code{-'} at
start of line (which all MIME boundaries have) but it is declared so
it can be overridden in derived classes.

Note that this test is used intended as a fast guard for the real
boundary tests; if it always returns false it will merely slow
processing, not cause it to fail.
\end{methoddesc}

\begin{methoddesc}{section_divider}{str}
Turn a boundary into a section-divider line.  By default, this
method prepends \code{'-}\code{-'} (which MIME section boundaries have) but
it is declared so it can be overridden in derived classes.  This
method need not append LF or CR-LF, as comparison with the result
ignores trailing whitespace. 
\end{methoddesc}

\begin{methoddesc}{end_marker}{str}
Turn a boundary string into an end-marker line.  By default, this
method prepends \code{'-}\code{-'} and appends \code{'-}\code{-'} (like a
MIME-multipart end-of-message marker) but it is declared so it can be
be overridden in derived classes.  This method need not append LF or
CR-LF, as comparison with the result ignores trailing whitespace.
\end{methoddesc}

Finally, \class{MultiFile} instances have two public instance variables:

\begin{memberdesc}{level}
Nesting depth of the current part.
\end{memberdesc}

\begin{memberdesc}{last}
True if the last end-of-file was for an end-of-message marker. 
\end{memberdesc}


\subsection{\class{MultiFile} Example \label{multifile-example}}

% This is almost unreadable; should be re-written when someone gets time.

\begin{verbatim}
fp = MultiFile(sys.stdin, 0)
fp.push(outer_boundary)
message1 = fp.readlines()
# We should now be either at real EOF or stopped on a message
# boundary. Re-enable the outer boundary.
fp.next()
# Read another message with the same delimiter
message2 = fp.readlines()
# Re-enable that delimiter again
fp.next()
# Now look for a message subpart with a different boundary
fp.push(inner_boundary)
sub_header = fp.readlines()
# If no exception has been thrown, we're looking at the start of
# the message subpart.  Reset and grab the subpart
fp.next()
sub_body = fp.readlines()
# Got it.  Now pop the inner boundary to re-enable the outer one.
fp.pop()
# Read to next outer boundary
message3 = fp.readlines()
\end{verbatim}

\section{\module{binhex} ---
         Encode and decode binhex4 files}

\declaremodule{standard}{binhex}
\modulesynopsis{Encode and decode files in binhex4 format.}


This module encodes and decodes files in binhex4 format, a format
allowing representation of Macintosh files in \ASCII.  On the Macintosh,
both forks of a file and the finder information are encoded (or
decoded), on other platforms only the data fork is handled.

The \module{binhex} module defines the following functions:

\begin{funcdesc}{binhex}{input, output}
Convert a binary file with filename \var{input} to binhex file
\var{output}. The \var{output} parameter can either be a filename or a
file-like object (any object supporting a \method{write()} and
\method{close()} method).
\end{funcdesc}

\begin{funcdesc}{hexbin}{input\optional{, output}}
Decode a binhex file \var{input}. \var{input} may be a filename or a
file-like object supporting \method{read()} and \method{close()} methods.
The resulting file is written to a file named \var{output}, unless the
argument is omitted in which case the output filename is read from the
binhex file.
\end{funcdesc}

The following exception is also defined:

\begin{excdesc}{Error}
Exception raised when something can't be encoded using the binhex
format (for example, a filename is too long to fit in the filename
field), or when input is not properly encoded binhex data.
\end{excdesc}


\begin{seealso}
  \seemodule{binascii}{Support module containing \ASCII-to-binary
                       and binary-to-\ASCII{} conversions.}
\end{seealso}


\subsection{Notes \label{binhex-notes}}

There is an alternative, more powerful interface to the coder and
decoder, see the source for details.

If you code or decode textfiles on non-Macintosh platforms they will
still use the Macintosh newline convention (carriage-return as end of
line).

As of this writing, \function{hexbin()} appears to not work in all
cases.

\section{Standard Module \module{uu}}
\label{module-uu}
\stmodindex{uu}

This module encodes and decodes files in uuencode format, allowing
arbitrary binary data to be transferred over ascii-only connections.
Wherever a file argument is expected, the methods accept a file-like
object.  For backwards compatibility, a string containing a pathname
is also accepted, and the corresponding file will be opened for
reading and writing; the pathname \code{'-'} is understood to mean the
standard input or output.  However, this interface is deprecated; it's
better for the caller to open the file itself, and be sure that, when
required, the mode is \code{'rb'} or \code{'wb'} on Windows or DOS.

This code was contributed by Lance Ellinghouse, and modified by Jack
Jansen.
\index{Jansen, Jack}
\index{Ellinghouse, Lance}

The \module{uu} module defines the following functions:

\begin{funcdesc}{encode}{in_file, out_file\optional{, name\optional{, mode}}}
Uuencode file \var{in_file} into file \var{out_file}.  The uuencoded
file will have the header specifying \var{name} and \var{mode} as the
defaults for the results of decoding the file. The default defaults
are taken from \var{in_file}, or \code{'-'} and \code{0666}
respectively. 
\end{funcdesc}

\begin{funcdesc}{decode}{in_file\optional{, out_file\optional{, mode}}}
This call decodes uuencoded file \var{in_file} placing the result on
file \var{out_file}. If \var{out_file} is a pathname the \var{mode} is
also set. Defaults for \var{out_file} and \var{mode} are taken from
the uuencode header.
\end{funcdesc}

\section{Standard Module \sectcode{binhex}}
\label{module-binhex}
\stmodindex{binhex}

This module encodes and decodes files in binhex4 format, a format
allowing representation of Macintosh files in ASCII. On the macintosh,
both forks of a file and the finder information are encoded (or
decoded), on other platforms only the data fork is handled.

The \code{binhex} module defines the following functions:

\setindexsubitem{(in module binhex)}

\begin{funcdesc}{binhex}{input\, output}
Convert a binary file with filename \var{input} to binhex file
\var{output}. The \var{output} parameter can either be a filename or a
file-like object (any object supporting a \var{write} and \var{close}
method).
\end{funcdesc}

\begin{funcdesc}{hexbin}{input\optional{\, output}}
Decode a binhex file \var{input}. \var{input} may be a filename or a
file-like object supporting \var{read} and \var{close} methods.
The resulting file is written to a file named \var{output}, unless the
argument is empty in which case the output filename is read from the
binhex file.
\end{funcdesc}

\subsection{Notes}
There is an alternative, more powerful interface to the coder and
decoder, see the source for details.

If you code or decode textfiles on non-Macintosh platforms they will
still use the macintosh newline convention (carriage-return as end of
line).

As of this writing, \var{hexbin} appears to not work in all cases.

\section{Standard Module \sectcode{uu}}
\stmodindex{uu}

This module encodes and decodes files in uuencode format, allowing
arbitrary binary data to be transferred over ascii-only connections.
Wherever a file argument is expected, the methods accept a file-like
object.  For backwards compatibility, a string containing a pathname
is also accepted, and the corresponding file will be opened for
reading and writing; the pathname \code{'-'} is understood to mean the
standard input or output.  However, this interface is deprecated; it's
better for the caller to open the file itself, and be sure that, when
required, the mode is \code{'rb'} or \code{'wb'} on Windows or DOS.

This code was contributed by Lance Ellinghouse, and modified by Jack
Jansen.

The \code{uu} module defines the following functions:

\setindexsubitem{(in module uu)}

\begin{funcdesc}{encode}{in_file\, out_file\optional{\, name\, mode}}
Uuencode file \var{in_file} into file \var{out_file}.  The uuencoded
file will have the header specifying \var{name} and \var{mode} as the
defaults for the results of decoding the file. The default defaults
are taken from \var{in_file}, or \code{'-'} and \code{0666}
respectively. 
\end{funcdesc}

\begin{funcdesc}{decode}{in_file\optional{\, out_file\, mode}}
This call decodes uuencoded file \var{in_file} placing the result on
file \var{out_file}. If \var{out_file} is a pathname the \var{mode} is
also set. Defaults for \var{out_file} and \var{mode} are taken from
the uuencode header.
\end{funcdesc}

\section{Built-in Module \sectcode{binascii}}	% If implemented in C
\bimodindex{binascii}

The binascii module contains a number of methods to convert between
binary and various ascii-encoded binary representations. Normally, you
will not use these modules directly but use wrapper modules like
\var{uu} or \var{hexbin} in stead, this module solely exists because
bit-manipuation of large amounts of data is slow in python.

The \code{binascii} module defines the following functions:

\setindexsubitem{(in module binascii)}

\begin{funcdesc}{a2b_uu}{string}
Convert a single line of uuencoded data back to binary and return the
binary data. Lines normally contain 45 (binary) bytes, except for the
last line. Line data may be followed by whitespace.
\end{funcdesc}

\begin{funcdesc}{b2a_uu}{data}
Convert binary data to a line of ascii characters, the return value is
the converted line, including a newline char. The length of \var{data}
should be at most 45.
\end{funcdesc}

\begin{funcdesc}{a2b_base64}{string}
Convert a block of base64 data back to binary and return the
binary data. More than one line may be passed at a time.
\end{funcdesc}

\begin{funcdesc}{b2a_base64}{data}
Convert binary data to a line of ascii characters in base64 coding.
The return value is the converted line, including a newline char.
The length of \var{data} should be at most 57 to adhere to the base64
standard.
\end{funcdesc}

\begin{funcdesc}{a2b_hqx}{string}
Convert binhex4 formatted ascii data to binary, without doing
rle-decompression. The string should contain a complete number of
binary bytes, or (in case of the last portion of the binhex4 data)
have the remaining bits zero.
\end{funcdesc}

\begin{funcdesc}{rledecode_hqx}{data}
Perform RLE-decompression on the data, as per the binhex4
standard. The algorithm uses \code{0x90} after a byte as a repeat
indicator, followed by a count. A count of \code{0} specifies a byte
value of \code{0x90}. The routine returns the decompressed data,
unless data input data ends in an orphaned repeat indicator, in which
case the \var{Incomplete} exception is raised.
\end{funcdesc}

\begin{funcdesc}{rlecode_hqx}{data}
Perform binhex4 style RLE-compression on \var{data} and return the
result.
\end{funcdesc}

\begin{funcdesc}{b2a_hqx}{data}
Perform hexbin4 binary-to-ascii translation and return the resulting
string. The argument should already be rle-coded, and have a length
divisible by 3 (except possibly the last fragment).
\end{funcdesc}

\begin{funcdesc}{crc_hqx}{data, crc}
Compute the binhex4 crc value of \var{data}, starting with an initial
\var{crc} and returning the result.
\end{funcdesc}
 
\begin{excdesc}{Error}
Exception raised on errors. These are usually programming errors.
\end{excdesc}

\begin{excdesc}{Incomplete}
Exception raised on incomplete data. These are usually not programming
errors, but handled by reading a little more data and trying again.
\end{excdesc}

\section{Standard Module \module{xdrlib}}
\declaremodule{standard}{xdrlib}

\modulesynopsis{The External Data Representation Standard as described
in \rfc{1014}.}

\index{XDR}
\index{External Data Representation}


The \module{xdrlib} module supports the External Data Representation
Standard as described in \rfc{1014}, written by Sun Microsystems,
Inc. June 1987.  It supports most of the data types described in the
RFC.

The \module{xdrlib} module defines two classes, one for packing
variables into XDR representation, and another for unpacking from XDR
representation.  There are also two exception classes.

\begin{classdesc}{Packer}{}
\class{Packer} is the class for packing data into XDR representation.
The \class{Packer} class is instantiated with no arguments.
\end{classdesc}

\begin{classdesc}{Unpacker}{data}
\code{Unpacker} is the complementary class which unpacks XDR data
values from a string buffer.  The input buffer is given as
\var{data}.
\end{classdesc}


\subsection{Packer Objects}
\label{xdr-packer-objects}

\class{Packer} instances have the following methods:

\begin{methoddesc}[Packer]{get_buffer}{}
Returns the current pack buffer as a string.
\end{methoddesc}

\begin{methoddesc}[Packer]{reset}{}
Resets the pack buffer to the empty string.
\end{methoddesc}

In general, you can pack any of the most common XDR data types by
calling the appropriate \code{pack_\var{type}()} method.  Each method
takes a single argument, the value to pack.  The following simple data
type packing methods are supported: \method{pack_uint()},
\method{pack_int()}, \method{pack_enum()}, \method{pack_bool()},
\method{pack_uhyper()}, and \method{pack_hyper()}.

\begin{methoddesc}[Packer]{pack_float}{value}
Packs the single-precision floating point number \var{value}.
\end{methoddesc}

\begin{methoddesc}[Packer]{pack_double}{value}
Packs the double-precision floating point number \var{value}.
\end{methoddesc}

The following methods support packing strings, bytes, and opaque data:

\begin{methoddesc}[Packer]{pack_fstring}{n, s}
Packs a fixed length string, \var{s}.  \var{n} is the length of the
string but it is \emph{not} packed into the data buffer.  The string
is padded with null bytes if necessary to guaranteed 4 byte alignment.
\end{methoddesc}

\begin{methoddesc}[Packer]{pack_fopaque}{n, data}
Packs a fixed length opaque data stream, similarly to
\method{pack_fstring()}.
\end{methoddesc}

\begin{methoddesc}[Packer]{pack_string}{s}
Packs a variable length string, \var{s}.  The length of the string is
first packed as an unsigned integer, then the string data is packed
with \method{pack_fstring()}.
\end{methoddesc}

\begin{methoddesc}[Packer]{pack_opaque}{data}
Packs a variable length opaque data string, similarly to
\method{pack_string()}.
\end{methoddesc}

\begin{methoddesc}[Packer]{pack_bytes}{bytes}
Packs a variable length byte stream, similarly to \method{pack_string()}.
\end{methoddesc}

The following methods support packing arrays and lists:

\begin{methoddesc}[Packer]{pack_list}{list, pack_item}
Packs a \var{list} of homogeneous items.  This method is useful for
lists with an indeterminate size; i.e. the size is not available until
the entire list has been walked.  For each item in the list, an
unsigned integer \code{1} is packed first, followed by the data value
from the list.  \var{pack_item} is the function that is called to pack
the individual item.  At the end of the list, an unsigned integer
\code{0} is packed.
\end{methoddesc}

\begin{methoddesc}[Packer]{pack_farray}{n, array, pack_item}
Packs a fixed length list (\var{array}) of homogeneous items.  \var{n}
is the length of the list; it is \emph{not} packed into the buffer,
but a \exception{ValueError} exception is raised if
\code{len(\var{array})} is not equal to \var{n}.  As above,
\var{pack_item} is the function used to pack each element.
\end{methoddesc}

\begin{methoddesc}[Packer]{pack_array}{list, pack_item}
Packs a variable length \var{list} of homogeneous items.  First, the
length of the list is packed as an unsigned integer, then each element
is packed as in \method{pack_farray()} above.
\end{methoddesc}


\subsection{Unpacker Objects}
\label{xdr-unpacker-objects}

The \class{Unpacker} class offers the following methods:

\begin{methoddesc}[Unpacker]{reset}{data}
Resets the string buffer with the given \var{data}.
\end{methoddesc}

\begin{methoddesc}[Unpacker]{get_position}{}
Returns the current unpack position in the data buffer.
\end{methoddesc}

\begin{methoddesc}[Unpacker]{set_position}{position}
Sets the data buffer unpack position to \var{position}.  You should be
careful about using \method{get_position()} and \method{set_position()}.
\end{methoddesc}

\begin{methoddesc}[Unpacker]{get_buffer}{}
Returns the current unpack data buffer as a string.
\end{methoddesc}

\begin{methoddesc}[Unpacker]{done}{}
Indicates unpack completion.  Raises an \exception{Error} exception
if all of the data has not been unpacked.
\end{methoddesc}

In addition, every data type that can be packed with a \class{Packer},
can be unpacked with an \class{Unpacker}.  Unpacking methods are of the
form \code{unpack_\var{type}()}, and take no arguments.  They return the
unpacked object.

\begin{methoddesc}[Unpacker]{unpack_float}{}
Unpacks a single-precision floating point number.
\end{methoddesc}

\begin{methoddesc}[Unpacker]{unpack_double}{}
Unpacks a double-precision floating point number, similarly to
\method{unpack_float()}.
\end{methoddesc}

In addition, the following methods unpack strings, bytes, and opaque
data:

\begin{methoddesc}[Unpacker]{unpack_fstring}{n}
Unpacks and returns a fixed length string.  \var{n} is the number of
characters expected.  Padding with null bytes to guaranteed 4 byte
alignment is assumed.
\end{methoddesc}

\begin{methoddesc}[Unpacker]{unpack_fopaque}{n}
Unpacks and returns a fixed length opaque data stream, similarly to
\method{unpack_fstring()}.
\end{methoddesc}

\begin{methoddesc}[Unpacker]{unpack_string}{}
Unpacks and returns a variable length string.  The length of the
string is first unpacked as an unsigned integer, then the string data
is unpacked with \method{unpack_fstring()}.
\end{methoddesc}

\begin{methoddesc}[Unpacker]{unpack_opaque}{}
Unpacks and returns a variable length opaque data string, similarly to
\method{unpack_string()}.
\end{methoddesc}

\begin{methoddesc}[Unpacker]{unpack_bytes}{}
Unpacks and returns a variable length byte stream, similarly to
\method{unpack_string()}.
\end{methoddesc}

The following methods support unpacking arrays and lists:

\begin{methoddesc}[Unpacker]{unpack_list}{unpack_item}
Unpacks and returns a list of homogeneous items.  The list is unpacked
one element at a time
by first unpacking an unsigned integer flag.  If the flag is \code{1},
then the item is unpacked and appended to the list.  A flag of
\code{0} indicates the end of the list.  \var{unpack_item} is the
function that is called to unpack the items.
\end{methoddesc}

\begin{methoddesc}[Unpacker]{unpack_farray}{n, unpack_item}
Unpacks and returns (as a list) a fixed length array of homogeneous
items.  \var{n} is number of list elements to expect in the buffer.
As above, \var{unpack_item} is the function used to unpack each element.
\end{methoddesc}

\begin{methoddesc}[Unpacker]{unpack_array}{unpack_item}
Unpacks and returns a variable length \var{list} of homogeneous items.
First, the length of the list is unpacked as an unsigned integer, then
each element is unpacked as in \method{unpack_farray()} above.
\end{methoddesc}


\subsection{Exceptions}
\nodename{xdr-exceptions}

Exceptions in this module are coded as class instances:

\begin{excdesc}{Error}
The base exception class.  \exception{Error} has a single public data
member \member{msg} containing the description of the error.
\end{excdesc}

\begin{excdesc}{ConversionError}
Class derived from \exception{Error}.  Contains no additional instance
variables.
\end{excdesc}

Here is an example of how you would catch one of these exceptions:

\begin{verbatim}
import xdrlib
p = xdrlib.Packer()
try:
    p.pack_double(8.01)
except xdrlib.ConversionError, instance:
    print 'packing the double failed:', instance.msg
\end{verbatim}

\section{Standard Module \sectcode{mailcap}}
\label{module-mailcap}
\stmodindex{mailcap}
\renewcommand{\indexsubitem}{(in module mailcap)}

Mailcap files are used to configure how MIME-aware applications such
as mail readers and Web browsers react to files with different MIME
types. (The name ``mailcap'' is derived from the phrase ``mail
capability''.)  For example, a mailcap file might contain a line like
\samp{video/mpeg; xmpeg \%s}.  Then, if the user encounters an email
message or Web document with the MIME type video/mpeg, \code{\%s} will be
replaced by a filename (usually one belonging to a temporary file) and
the xmpeg program can be automatically started to view the file.

The mailcap format is documented in RFC 1524, ``A User Agent
Configuration Mechanism For Multimedia Mail Format Information'', but
is not an Internet standard.  However, mailcap files are supported on
most Unix systems.

\begin{funcdesc}{findmatch}{caps\, MIMEtype\, key\, filename\, plist}
Return a 2-tuple; the first element is a string containing the command
line to be executed
(which can be passed to \code{os.system()}), and the second element is
the mailcap entry for a given MIME type.  If no matching MIME
type can be found, \code{(None, None)} is returned.

\var{key} is the name of the field desired, which represents the type of
activity to be performed; the default value is 'view', since in the
most common case you simply want to view the body of the MIME-typed
data.  Other possible values might be 'compose' and 'edit', if you
wanted to create a new body of the given MIME type or alter the
existing body data.  See RFC1524 for a complete list of these fields.

\var{filename} is the filename to be substituted for \%s in the
command line; the default value is
\file{/dev/null} which is almost certainly not what you want, so
usually you'll override it by specifying a filename.

\var{plist} can be a list containing named parameters; the default
value is simply an empty list.  Each entry in the list must be a
string containing the parameter name, an equals sign (\code{=}), and the
parameter's value.  Mailcap entries can contain 
named parameters like \code{\%\{foo\}}, which will be replaced by the
value of the parameter named 'foo'.  For example, if the command line
\samp{showpartial \%\{id\} \%\{number\} \%\{total\}}
was in a mailcap file, and \var{plist} was set to \code{['id=1',
'number=2', 'total=3']}, the resulting command line would be 
\code{"showpartial 1 2 3"}.  

In a mailcap file, the "test" field can optionally be specified to
test some external condition (e.g., the machine architecture, or the
window system in use) to determine whether or not the mailcap line
applies.  \code{findmatch()} will automatically check such conditions
and skip the entry if the check fails.
\end{funcdesc}

\begin{funcdesc}{getcaps}{}
Returns a dictionary mapping MIME types to a list of mailcap file
entries. This dictionary must be passed to the \code{findmatch()}
function.  An entry is stored as a list of dictionaries, but it
shouldn't be necessary to know the details of this representation.

The information is derived from all of the mailcap files found on the
system. Settings in the user's mailcap file \file{\$HOME/.mailcap}
will override settings in the system mailcap files
\file{/etc/mailcap}, \file{/usr/etc/mailcap}, and
\file{/usr/local/etc/mailcap}.
\end{funcdesc}

An example usage:
\bcode\begin{verbatim}
>>> import mailcap
>>> d=mailcap.getcaps()
>>> mailcap.findmatch(d, 'video/mpeg', filename='/tmp/tmp1223')
('xmpeg /tmp/tmp1223', {'view': 'xmpeg %s'})
\end{verbatim}\ecode

\section{\module{mimetypes} ---
         Map filenames to MIME types}

\declaremodule{standard}{mimetypes}
\modulesynopsis{Mapping of filename extensions to MIME types.}
\sectionauthor{Fred L. Drake, Jr.}{fdrake@acm.org}


\indexii{MIME}{content type}

The \module{mimetypes} module converts between a filename or URL and
the MIME type associated with the filename extension.  Conversions are
provided from filename to MIME type and from MIME type to filename
extension; encodings are not supported for the latter conversion.

The module provides one class and a number of convenience functions.
The functions are the normal interface to this module, but some
applications may be interested in the class as well.

The functions described below provide the primary interface for this
module.  If the module has not been initialized, they will call
\function{init()} if they rely on the information \function{init()}
sets up.


\begin{funcdesc}{guess_type}{filename\optional{, strict}}
Guess the type of a file based on its filename or URL, given by
\var{filename}.  The return value is a tuple \code{(\var{type},
\var{encoding})} where \var{type} is \code{None} if the type can't be
guessed (missing or unknown suffix) or a string of the form
\code{'\var{type}/\var{subtype}'}, usable for a MIME
\mailheader{content-type} header\indexii{MIME}{headers}.

\var{encoding} is \code{None} for no encoding or the name of the
program used to encode (e.g. \program{compress} or \program{gzip}).
The encoding is suitable for use as a \mailheader{Content-Encoding}
header, \emph{not} as a \mailheader{Content-Transfer-Encoding} header.
The mappings are table driven.  Encoding suffixes are case sensitive;
type suffixes are first tried case sensitively, then case
insensitively.

Optional \var{strict} is a flag specifying whether the list of known
MIME types is limited to only the official types \ulink{registered
with IANA}{http://www.isi.edu/in-notes/iana/assignments/media-types}
are recognized.  When \var{strict} is true (the default), only the
IANA types are supported; when \var{strict} is false, some additional
non-standard but commonly used MIME types are also recognized.
\end{funcdesc}

\begin{funcdesc}{guess_all_extensions}{type\optional{, strict}}
Guess the extensions for a file based on its MIME type, given by
\var{type}.
The return value is a list of strings giving all possible filename extensions,
including the leading dot (\character{.}).  The extensions are not guaranteed
to have been associated with any particular data stream, but would be mapped
to the MIME type \var{type} by \function{guess_type()}.

Optional \var{strict} has the same meaning as with the
\function{guess_type()} function.
\end{funcdesc}


\begin{funcdesc}{guess_extension}{type\optional{, strict}}
Guess the extension for a file based on its MIME type, given by
\var{type}.
The return value is a string giving a filename extension, including the
leading dot (\character{.}).  The extension is not guaranteed to have been
associated with any particular data stream, but would be mapped to the 
MIME type \var{type} by \function{guess_type()}.  If no extension can
be guessed for \var{type}, \code{None} is returned.

Optional \var{strict} has the same meaning as with the
\function{guess_type()} function.
\end{funcdesc}


Some additional functions and data items are available for controlling
the behavior of the module.


\begin{funcdesc}{init}{\optional{files}}
Initialize the internal data structures.  If given, \var{files} must
be a sequence of file names which should be used to augment the
default type map.  If omitted, the file names to use are taken from
\constant{knownfiles}.  Each file named in \var{files} or
\constant{knownfiles} takes precedence over those named before it.
Calling \function{init()} repeatedly is allowed.
\end{funcdesc}

\begin{funcdesc}{read_mime_types}{filename}
Load the type map given in the file \var{filename}, if it exists.  The 
type map is returned as a dictionary mapping filename extensions,
including the leading dot (\character{.}), to strings of the form
\code{'\var{type}/\var{subtype}'}.  If the file \var{filename} does
not exist or cannot be read, \code{None} is returned.
\end{funcdesc}


\begin{funcdesc}{add_type}{type, ext\optional{, strict}}
Add a mapping from the mimetype \var{type} to the extension \var{ext}.
When the extension is already known, the new type will replace the old
one. When the type is already known the extension will be added
to the list of known extensions.

When \var{strict} is the mapping will added to the official
MIME types, otherwise to the non-standard ones.
\end{funcdesc}


\begin{datadesc}{inited}
Flag indicating whether or not the global data structures have been
initialized.  This is set to true by \function{init()}.
\end{datadesc}

\begin{datadesc}{knownfiles}
List of type map file names commonly installed.  These files are
typically named \file{mime.types} and are installed in different
locations by different packages.\index{file!mime.types}
\end{datadesc}

\begin{datadesc}{suffix_map}
Dictionary mapping suffixes to suffixes.  This is used to allow
recognition of encoded files for which the encoding and the type are
indicated by the same extension.  For example, the \file{.tgz}
extension is mapped to \file{.tar.gz} to allow the encoding and type
to be recognized separately.
\end{datadesc}

\begin{datadesc}{encodings_map}
Dictionary mapping filename extensions to encoding types.
\end{datadesc}

\begin{datadesc}{types_map}
Dictionary mapping filename extensions to MIME types.
\end{datadesc}

\begin{datadesc}{common_types}
Dictionary mapping filename extensions to non-standard, but commonly
found MIME types.
\end{datadesc}


The \class{MimeTypes} class may be useful for applications which may
want more than one MIME-type database:

\begin{classdesc}{MimeTypes}{\optional{filenames}}
  This class represents a MIME-types database.  By default, it
  provides access to the same database as the rest of this module.
  The initial database is a copy of that provided by the module, and
  may be extended by loading additional \file{mime.types}-style files
  into the database using the \method{read()} or \method{readfp()}
  methods.  The mapping dictionaries may also be cleared before
  loading additional data if the default data is not desired.

  The optional \var{filenames} parameter can be used to cause
  additional files to be loaded ``on top'' of the default database.

  \versionadded{2.2}
\end{classdesc}

An example usage of the module:

\begin{verbatim}
>>> import mimetypes
>>> mimetypes.init()
>>> mimetypes.knownfiles
['/etc/mime.types', '/etc/httpd/mime.types', ... ]
>>> mimetypes.suffix_map['.tgz']
'.tar.gz'
>>> mimetypes.encodings_map['.gz']
'gzip'
>>> mimetypes.types_map['.tgz']
'application/x-tar-gz'
\end{verbatim}

\subsection{MimeTypes Objects \label{mimetypes-objects}}

\class{MimeTypes} instances provide an interface which is very like
that of the \refmodule{mimetypes} module.

\begin{memberdesc}[MimeTypes]{suffix_map}
  Dictionary mapping suffixes to suffixes.  This is used to allow
  recognition of encoded files for which the encoding and the type are
  indicated by the same extension.  For example, the \file{.tgz}
  extension is mapped to \file{.tar.gz} to allow the encoding and type
  to be recognized separately.  This is initially a copy of the global
  \code{suffix_map} defined in the module.
\end{memberdesc}

\begin{memberdesc}[MimeTypes]{encodings_map}
  Dictionary mapping filename extensions to encoding types.  This is
  initially a copy of the global \code{encodings_map} defined in the
  module.
\end{memberdesc}

\begin{memberdesc}[MimeTypes]{types_map}
  Dictionary mapping filename extensions to MIME types.  This is
  initially a copy of the global \code{types_map} defined in the
  module.
\end{memberdesc}

\begin{memberdesc}[MimeTypes]{common_types}
  Dictionary mapping filename extensions to non-standard, but commonly
  found MIME types.  This is initially a copy of the global
  \code{common_types} defined in the module.
\end{memberdesc}

\begin{methoddesc}[MimeTypes]{guess_extension}{type\optional{, strict}}
  Similar to the \function{guess_extension()} function, using the
  tables stored as part of the object.
\end{methoddesc}

\begin{methoddesc}[MimeTypes]{guess_type}{url\optional{, strict}}
  Similar to the \function{guess_type()} function, using the tables
  stored as part of the object.
\end{methoddesc}

\begin{methoddesc}[MimeTypes]{read}{path}
  Load MIME information from a file named \var{path}.  This uses
  \method{readfp()} to parse the file.
\end{methoddesc}

\begin{methoddesc}[MimeTypes]{readfp}{file}
  Load MIME type information from an open file.  The file must have
  the format of the standard \file{mime.types} files.
\end{methoddesc}

\section{\module{base64} ---
         Encode and decode MIME base64 data}

\declaremodule{standard}{base64}
\modulesynopsis{Encode and decode files using the MIME base64 data.}


\indexii{base64}{encoding}
\index{MIME!base64 encoding}

This module performs base64 encoding and decoding of arbitrary binary
strings into text strings that can be safely emailed or posted.  The
encoding scheme is defined in \rfc{1421} (``Privacy Enhancement for
Internet Electronic Mail: Part I: Message Encryption and
Authentication Procedures'', section 4.3.2.4, ``Step 4: Printable
Encoding'') and is used for MIME email and
various other Internet-related applications; it is not the same as the
output produced by the \program{uuencode} program.  For example, the
string \code{'www.python.org'} is encoded as the string
\code{'d3d3LnB5dGhvbi5vcmc=\e n'}.  


\begin{funcdesc}{decode}{input, output}
Decode the contents of the \var{input} file and write the resulting
binary data to the \var{output} file.
\var{input} and \var{output} must either be file objects or objects that
mimic the file object interface. \var{input} will be read until
\code{\var{input}.read()} returns an empty string.
\end{funcdesc}

\begin{funcdesc}{decodestring}{s}
Decode the string \var{s}, which must contain one or more lines of
base64 encoded data, and return a string containing the resulting
binary data.
\end{funcdesc}

\begin{funcdesc}{encode}{input, output}
Encode the contents of the \var{input} file and write the resulting
base64 encoded data to the \var{output} file.
\var{input} and \var{output} must either be file objects or objects that
mimic the file object interface. \var{input} will be read until
\code{\var{input}.read()} returns an empty string.
\end{funcdesc}

\begin{funcdesc}{encodestring}{s}
Encode the string \var{s}, which can contain arbitrary binary data,
and return a string containing one or more lines of
base64 encoded data.
\end{funcdesc}


\begin{seealso}
  \seemodule{binascii}{support module containing \ASCII{}-to-binary
                       and binary-to-\ASCII{} conversions}
\end{seealso}

\section{Standard Module \sectcode{quopri}}
\label{module-quopri}
\stmodindex{quopri}

This module performs quoted-printable transport encoding and decoding,
as defined in \rfc{1521}: ``MIME (Multipurpose Internet Mail Extensions)
Part One''.  The quoted-printable encoding is designed for data where
there are relatively few nonprintable characters; the base-64 encoding
scheme available via the \code{base64} module is more compact if there
are many such characters, as when sending a graphics file.
\indexii{quoted printable}{encoding}
\index{MIME!quoted-printable encoding}

\setindexsubitem{(in module quopri)}

\begin{funcdesc}{decode}{input, output}
Decode the contents of the \var{input} file and write the resulting
decoded binary data to the \var{output} file.
\var{input} and \var{output} must either be file objects or objects that
mimic the file object interface. \var{input} will be read until
\code{\var{input}.read()} returns an empty string.
\end{funcdesc}

\begin{funcdesc}{encode}{input, output, quotetabs}
Encode the contents of the \var{input} file and write the resulting
quoted-printable data to the \var{output} file.
\var{input} and \var{output} must either be file objects or objects that
mimic the file object interface. \var{input} will be read until
\code{\var{input}.read()} returns an empty string.
\end{funcdesc}




\section{\module{mailbox} ---
          Manipulate mailboxes in various formats}

\declaremodule{}{mailbox}
\moduleauthor{Gregory K.~Johnson}{gkj@gregorykjohnson.com}
\sectionauthor{Gregory K.~Johnson}{gkj@gregorykjohnson.com}
\modulesynopsis{Manipulate mailboxes in various formats}


This module defines two classes, \class{Mailbox} and \class{Message}, for
accessing and manipulating on-disk mailboxes and the messages they contain.
\class{Mailbox} offers a dictionary-like mapping from keys to messages.
\class{Message} extends the \module{email.Message} module's \class{Message}
class with format-specific state and behavior. Supported mailbox formats are
Maildir, mbox, MH, Babyl, and MMDF.

\begin{seealso}
    \seemodule{email}{Represent and manipulate messages.}
\end{seealso}

\subsection{\class{Mailbox} objects}
\label{mailbox-objects}

\begin{classdesc*}{Mailbox}
A mailbox, which may be inspected and modified.
\end{classdesc*}

The \class{Mailbox} interface is dictionary-like, with small keys
corresponding to messages. Keys are issued by the \class{Mailbox} instance
with which they will be used and are only meaningful to that \class{Mailbox}
instance. A key continues to identify a message even if the corresponding
message is modified, such as by replacing it with another message. Messages may
be added to a \class{Mailbox} instance using the set-like method
\method{add()} and removed using a \code{del} statement or the set-like methods
\method{remove()} and \method{discard()}.

\class{Mailbox} interface semantics differ from dictionary semantics in some
noteworthy ways. Each time a message is requested, a new representation
(typically a \class{Message} instance) is generated, based upon the current
state of the mailbox. Similarly, when a message is added to a \class{Mailbox}
instance, the provided message representation's contents are copied. In neither
case is a reference to the message representation kept by the \class{Mailbox}
instance.

The default \class{Mailbox} iterator iterates over message representations, not
keys as the default dictionary iterator does. Moreover, modification of a
mailbox during iteration is safe and well-defined. Messages added to the
mailbox after an iterator is created will not be seen by the iterator. Messages
removed from the mailbox before the iterator yields them will be silently
skipped, though using a key from an iterator may result in a
\exception{KeyError} exception if the corresponding message is subsequently
removed.

\class{Mailbox} itself is intended to define an interface and to be inherited
from by format-specific subclasses but is not intended to be instantiated.
Instead, you should instantiate a subclass.

\class{Mailbox} instances have the following methods:

\begin{methoddesc}{add}{message}
Add \var{message} to the mailbox and return the key that has been assigned to
it.

Parameter \var{message} may be a \class{Message} instance, an
\class{email.Message.Message} instance, a string, or a file-like object (which
should be open in text mode). If \var{message} is an instance of the
appropriate format-specific \class{Message} subclass (e.g., if it's an
\class{mboxMessage} instance and this is an \class{mbox} instance), its
format-specific information is used. Otherwise, reasonable defaults for
format-specific information are used.
\end{methoddesc}

\begin{methoddesc}{remove}{key}
\methodline{__delitem__}{key}
\methodline{discard}{key}
Delete the message corresponding to \var{key} from the mailbox.

If no such message exists, a \exception{KeyError} exception is raised if the
method was called as \method{remove()} or \method{__delitem__()} but no
exception is raised if the method was called as \method{discard()}. The
behavior of \method{discard()} may be preferred if the underlying mailbox
format supports concurrent modification by other processes.
\end{methoddesc}

\begin{methoddesc}{__setitem__}{key, message}
Replace the message corresponding to \var{key} with \var{message}. Raise a
\exception{KeyError} exception if no message already corresponds to \var{key}.

As with \method{add()}, parameter \var{message} may be a \class{Message}
instance, an \class{email.Message.Message} instance, a string, or a file-like
object (which should be open in text mode). If \var{message} is an instance of
the appropriate format-specific \class{Message} subclass (e.g., if it's an
\class{mboxMessage} instance and this is an \class{mbox} instance), its
format-specific information is used. Otherwise, the format-specific information
of the message that currently corresponds to \var{key} is left unchanged. 
\end{methoddesc}

\begin{methoddesc}{iterkeys}{}
\methodline{keys}{}
Return an iterator over all keys if called as \method{iterkeys()} or return a
list of keys if called as \method{keys()}.
\end{methoddesc}

\begin{methoddesc}{itervalues}{}
\methodline{__iter__}{}
\methodline{values}{}
Return an iterator over representations of all messages if called as
\method{itervalues()} or \method{__iter__()} or return a list of such
representations if called as \method{values()}. The messages are represented as
instances of the appropriate format-specific \class{Message} subclass unless a
custom message factory was specified when the \class{Mailbox} instance was
initialized. \note{The behavior of \method{__iter__()} is unlike that of
dictionaries, which iterate over keys.}
\end{methoddesc}

\begin{methoddesc}{iteritems}{}
\methodline{items}{}
Return an iterator over (\var{key}, \var{message}) pairs, where \var{key} is a
key and \var{message} is a message representation, if called as
\method{iteritems()} or return a list of such pairs if called as
\method{items()}. The messages are represented as instances of the appropriate
format-specific \class{Message} subclass unless a custom message factory was
specified when the \class{Mailbox} instance was initialized.
\end{methoddesc}

\begin{methoddesc}{get}{key\optional{, default=None}}
\methodline{__getitem__}{key}
Return a representation of the message corresponding to \var{key}. If no such
message exists, \var{default} is returned if the method was called as
\method{get()} and a \exception{KeyError} exception is raised if the method was
called as \method{__getitem__()}. The message is represented as an instance of
the appropriate format-specific \class{Message} subclass unless a custom
message factory was specified when the \class{Mailbox} instance was
initialized.
\end{methoddesc}

\begin{methoddesc}{get_message}{key}
Return a representation of the message corresponding to \var{key} as an
instance of the appropriate format-specific \class{Message} subclass, or raise
a \exception{KeyError} exception if no such message exists.
\end{methoddesc}

\begin{methoddesc}{get_string}{key}
Return a string representation of the message corresponding to \var{key}, or
raise a \exception{KeyError} exception if no such message exists.
\end{methoddesc}

\begin{methoddesc}{get_file}{key}
Return a file-like representation of the message corresponding to \var{key},
or raise a \exception{KeyError} exception if no such message exists. The
file-like object behaves as if open in binary mode. This file should be closed
once it is no longer needed.

\note{Unlike other representations of messages, file-like representations are
not necessarily independent of the \class{Mailbox} instance that created them
or of the underlying mailbox. More specific documentation is provided by each
subclass.}
\end{methoddesc}

\begin{methoddesc}{has_key}{key}
\methodline{__contains__}{key}
Return \code{True} if \var{key} corresponds to a message, \code{False}
otherwise.
\end{methoddesc}

\begin{methoddesc}{__len__}{}
Return a count of messages in the mailbox.
\end{methoddesc}

\begin{methoddesc}{clear}{}
Delete all messages from the mailbox.
\end{methoddesc}

\begin{methoddesc}{pop}{key\optional{, default}}
Return a representation of the message corresponding to \var{key} and delete
the message. If no such message exists, return \var{default} if it was supplied
or else raise a \exception{KeyError} exception. The message is represented as
an instance of the appropriate format-specific \class{Message} subclass unless
a custom message factory was specified when the \class{Mailbox} instance was
initialized.
\end{methoddesc}

\begin{methoddesc}{popitem}{}
Return an arbitrary (\var{key}, \var{message}) pair, where \var{key} is a key
and \var{message} is a message representation, and delete the corresponding
message. If the mailbox is empty, raise a \exception{KeyError} exception. The
message is represented as an instance of the appropriate format-specific
\class{Message} subclass unless a custom message factory was specified when the
\class{Mailbox} instance was initialized.
\end{methoddesc}

\begin{methoddesc}{update}{arg}
Parameter \var{arg} should be a \var{key}-to-\var{message} mapping or an
iterable of (\var{key}, \var{message}) pairs. Updates the mailbox so that, for
each given \var{key} and \var{message}, the message corresponding to \var{key}
is set to \var{message} as if by using \method{__setitem__()}. As with
\method{__setitem__()}, each \var{key} must already correspond to a message in
the mailbox or else a \exception{KeyError} exception will be raised, so in
general it is incorrect for \var{arg} to be a \class{Mailbox} instance.
\note{Unlike with dictionaries, keyword arguments are not supported.}
\end{methoddesc}

\begin{methoddesc}{flush}{}
Write any pending changes to the filesystem. For some \class{Mailbox}
subclasses, changes are always written immediately and this method does
nothing.
\end{methoddesc}

\begin{methoddesc}{lock}{}
Acquire an exclusive advisory lock on the mailbox so that other processes know
not to modify it. An \exception{ExternalClashError} is raised if the lock is
not available. The particular locking mechanisms used depend upon the mailbox
format.
\end{methoddesc}

\begin{methoddesc}{unlock}{}
Release the lock on the mailbox, if any.
\end{methoddesc}

\begin{methoddesc}{close}{}
Flush the mailbox, unlock it if necessary, and close any open files. For some
\class{Mailbox} subclasses, this method does nothing.
\end{methoddesc}


\subsubsection{\class{Maildir}}
\label{mailbox-maildir}

\begin{classdesc}{Maildir}{dirname\optional{, factory=rfc822.Message\optional{,
create=True}}}
A subclass of \class{Mailbox} for mailboxes in Maildir format. Parameter
\var{factory} is a callable object that accepts a file-like message
representation (which behaves as if opened in binary mode) and returns a custom
representation. If \var{factory} is \code{None}, \class{MaildirMessage} is used
as the default message representation. If \var{create} is \code{True}, the
mailbox is created if it does not exist.

It is for historical reasons that \var{factory} defaults to
\class{rfc822.Message} and that \var{dirname} is named as such rather than
\var{path}. For a \class{Maildir} instance that behaves like instances of other
\class{Mailbox} subclasses, set \var{factory} to \code{None}.
\end{classdesc}

Maildir is a directory-based mailbox format invented for the qmail mail
transfer agent and now widely supported by other programs. Messages in a
Maildir mailbox are stored in separate files within a common directory
structure. This design allows Maildir mailboxes to be accessed and modified by
multiple unrelated programs without data corruption, so file locking is
unnecessary.

Maildir mailboxes contain three subdirectories, namely: \file{tmp}, \file{new},
and \file{cur}. Messages are created momentarily in the \file{tmp} subdirectory
and then moved to the \file{new} subdirectory to finalize delivery. A mail user
agent may subsequently move the message to the \file{cur} subdirectory and
store information about the state of the message in a special "info" section
appended to its file name.

Folders of the style introduced by the Courier mail transfer agent are also
supported. Any subdirectory of the main mailbox is considered a folder if
\character{.} is the first character in its name. Folder names are represented
by \class{Maildir} without the leading \character{.}. Each folder is itself a
Maildir mailbox but should not contain other folders. Instead, a logical
nesting is indicated using \character{.} to delimit levels, e.g.,
"Archived.2005.07".

\begin{notice}
The Maildir specification requires the use of a colon (\character{:}) in
certain message file names. However, some operating systems do not permit this
character in file names, If you wish to use a Maildir-like format on such an
operating system, you should specify another character to use instead. The
exclamation point (\character{!}) is a popular choice. For example:
\begin{verbatim}
import mailbox
mailbox.Maildir.colon = '!'
\end{verbatim}
The \member{colon} attribute may also be set on a per-instance basis.
\end{notice}

\class{Maildir} instances have all of the methods of \class{Mailbox} in
addition to the following:

\begin{methoddesc}{list_folders}{}
Return a list of the names of all folders.
\end{methoddesc}

\begin{methoddesc}{get_folder}{folder}
Return a \class{Maildir} instance representing the folder whose name is
\var{folder}. A \exception{NoSuchMailboxError} exception is raised if the
folder does not exist.
\end{methoddesc}

\begin{methoddesc}{add_folder}{folder}
Create a folder whose name is \var{folder} and return a \class{Maildir}
instance representing it.
\end{methoddesc}

\begin{methoddesc}{remove_folder}{folder}
Delete the folder whose name is \var{folder}. If the folder contains any
messages, a \exception{NotEmptyError} exception will be raised and the folder
will not be deleted.
\end{methoddesc}

\begin{methoddesc}{clean}{}
Delete temporary files from the mailbox that have not been accessed in the
last 36 hours. The Maildir specification says that mail-reading programs
should do this occasionally.
\end{methoddesc}

Some \class{Mailbox} methods implemented by \class{Maildir} deserve special
remarks:

\begin{methoddesc}{add}{message}
\methodline[Maildir]{__setitem__}{key, message}
\methodline[Maildir]{update}{arg}
\warning{These methods generate unique file names based upon the current
process ID. When using multiple threads, undetected name clashes may occur and
cause corruption of the mailbox unless threads are coordinated to avoid using
these methods to manipulate the same mailbox simultaneously.}
\end{methoddesc}

\begin{methoddesc}{flush}{}
All changes to Maildir mailboxes are immediately applied, so this method does
nothing.
\end{methoddesc}

\begin{methoddesc}{lock}{}
\methodline{unlock}{}
Maildir mailboxes do not support (or require) locking, so these methods do
nothing. 
\end{methoddesc}

\begin{methoddesc}{close}{}
\class{Maildir} instances do not keep any open files and the underlying
mailboxes do not support locking, so this method does nothing.
\end{methoddesc}

\begin{methoddesc}{get_file}{key}
Depending upon the host platform, it may not be possible to modify or remove
the underlying message while the returned file remains open.
\end{methoddesc}

\begin{seealso}
    \seelink{http://www.qmail.org/man/man5/maildir.html}{maildir man page from
    qmail}{The original specification of the format.}
    \seelink{http://cr.yp.to/proto/maildir.html}{Using maildir format}{Notes
    on Maildir by its inventor. Includes an updated name-creation scheme and
    details on "info" semantics.}
    \seelink{http://www.courier-mta.org/?maildir.html}{maildir man page from
    Courier}{Another specification of the format. Describes a common extension
    for supporting folders.}
\end{seealso}

\subsubsection{\class{mbox}}
\label{mailbox-mbox}

\begin{classdesc}{mbox}{path\optional{, factory=None\optional{, create=True}}}
A subclass of \class{Mailbox} for mailboxes in mbox format. Parameter
\var{factory} is a callable object that accepts a file-like message
representation (which behaves as if opened in binary mode) and returns a custom
representation. If \var{factory} is \code{None}, \class{mboxMessage} is used as
the default message representation. If \var{create} is \code{True}, the mailbox
is created if it does not exist.
\end{classdesc}

The mbox format is the classic format for storing mail on \UNIX{} systems. All
messages in an mbox mailbox are stored in a single file with the beginning of
each message indicated by a line whose first five characters are "From~".

Several variations of the mbox format exist to address perceived shortcomings
in the original. In the interest of compatibility, \class{mbox} implements the
original format, which is sometimes referred to as \dfn{mboxo}. This means that
the \mailheader{Content-Length} header, if present, is ignored and that any
occurrences of "From~" at the beginning of a line in a message body are
transformed to ">From~" when storing the message, although occurences of
">From~" are not transformed to "From~" when reading the message.

Some \class{Mailbox} methods implemented by \class{mbox} deserve special
remarks:

\begin{methoddesc}{get_file}{key}
Using the file after calling \method{flush()} or \method{close()} on the
\class{mbox} instance may yield unpredictable results or raise an exception.
\end{methoddesc}

\begin{methoddesc}{lock}{}
\methodline{unlock}{}
Three locking mechanisms are used---dot locking and, if available, the
\cfunction{flock()} and \cfunction{lockf()} system calls.
\end{methoddesc}

\begin{seealso}
    \seelink{http://www.qmail.org/man/man5/mbox.html}{mbox man page from
    qmail}{A specification of the format and its variations.}
    \seelink{http://www.tin.org/bin/man.cgi?section=5\&topic=mbox}{mbox man
    page from tin}{Another specification of the format, with details on
    locking.}
    \seelink{http://home.netscape.com/eng/mozilla/2.0/relnotes/demo/content-length.html}
    {Configuring Netscape Mail on \UNIX{}: Why The Content-Length Format is
    Bad}{An argument for using the original mbox format rather than a
    variation.}
    \seelink{http://homepages.tesco.net./\tilde{}J.deBoynePollard/FGA/mail-mbox-formats.html}
    {"mbox" is a family of several mutually incompatible mailbox formats}{A
    history of mbox variations.}
\end{seealso}

\subsubsection{\class{MH}}
\label{mailbox-mh}

\begin{classdesc}{MH}{path\optional{, factory=None\optional{, create=True}}}
A subclass of \class{Mailbox} for mailboxes in MH format. Parameter
\var{factory} is a callable object that accepts a file-like message
representation (which behaves as if opened in binary mode) and returns a custom
representation. If \var{factory} is \code{None}, \class{MHMessage} is used as
the default message representation. If \var{create} is \code{True}, the mailbox
is created if it does not exist.
\end{classdesc}

MH is a directory-based mailbox format invented for the MH Message Handling
System, a mail user agent. Each message in an MH mailbox resides in its own
file. An MH mailbox may contain other MH mailboxes (called \dfn{folders}) in
addition to messages. Folders may be nested indefinitely. MH mailboxes also
support \dfn{sequences}, which are named lists used to logically group messages
without moving them to sub-folders. Sequences are defined in a file called
\file{.mh_sequences} in each folder.

The \class{MH} class manipulates MH mailboxes, but it does not attempt to
emulate all of \program{mh}'s behaviors. In particular, it does not modify and
is not affected by the \file{context} or \file{.mh_profile} files that are used
by \program{mh} to store its state and configuration.

\class{MH} instances have all of the methods of \class{Mailbox} in addition to
the following:

\begin{methoddesc}{list_folders}{}
Return a list of the names of all folders.
\end{methoddesc}

\begin{methoddesc}{get_folder}{folder}
Return an \class{MH} instance representing the folder whose name is
\var{folder}. A \exception{NoSuchMailboxError} exception is raised if the
folder does not exist.
\end{methoddesc}

\begin{methoddesc}{add_folder}{folder}
Create a folder whose name is \var{folder} and return an \class{MH} instance
representing it.
\end{methoddesc}

\begin{methoddesc}{remove_folder}{folder}
Delete the folder whose name is \var{folder}. If the folder contains any
messages, a \exception{NotEmptyError} exception will be raised and the folder
will not be deleted.
\end{methoddesc}

\begin{methoddesc}{get_sequences}{}
Return a dictionary of sequence names mapped to key lists. If there are no
sequences, the empty dictionary is returned.
\end{methoddesc}

\begin{methoddesc}{set_sequences}{sequences}
Re-define the sequences that exist in the mailbox based upon \var{sequences}, a
dictionary of names mapped to key lists, like returned by
\method{get_sequences()}.
\end{methoddesc}

\begin{methoddesc}{pack}{}
Rename messages in the mailbox as necessary to eliminate gaps in numbering.
Entries in the sequences list are updated correspondingly. \note{Already-issued
keys are invalidated by this operation and should not be subsequently used.}
\end{methoddesc}

Some \class{Mailbox} methods implemented by \class{MH} deserve special remarks:

\begin{methoddesc}{remove}{key}
\methodline{__delitem__}{key}
\methodline{discard}{key}
These methods immediately delete the message. The MH convention of marking a
message for deletion by prepending a comma to its name is not used.
\end{methoddesc}

\begin{methoddesc}{lock}{}
\methodline{unlock}{}
Three locking mechanisms are used---dot locking and, if available, the
\cfunction{flock()} and \cfunction{lockf()} system calls. For MH mailboxes,
locking the mailbox means locking the \file{.mh_sequences} file and, only for
the duration of any operations that affect them, locking individual message
files.
\end{methoddesc}

\begin{methoddesc}{get_file}{key}
Depending upon the host platform, it may not be possible to remove the
underlying message while the returned file remains open.
\end{methoddesc}

\begin{methoddesc}{flush}{}
All changes to MH mailboxes are immediately applied, so this method does
nothing.
\end{methoddesc}

\begin{methoddesc}{close}{}
\class{MH} instances do not keep any open files, so this method is equivelant
to \method{unlock()}.
\end{methoddesc}

\begin{seealso}
\seelink{http://www.nongnu.org/nmh/}{nmh - Message Handling System}{Home page
of \program{nmh}, an updated version of the original \program{mh}.}
\seelink{http://www.ics.uci.edu/\tilde{}mh/book/}{MH \& nmh: Email for Users \&
Programmers}{A GPL-licensed book on \program{mh} and \program{nmh}, with some
information on the mailbox format.}
\end{seealso}

\subsubsection{\class{Babyl}}
\label{mailbox-babyl}

\begin{classdesc}{Babyl}{path\optional{, factory=None\optional{, create=True}}}
A subclass of \class{Mailbox} for mailboxes in Babyl format. Parameter
\var{factory} is a callable object that accepts a file-like message
representation (which behaves as if opened in binary mode) and returns a custom
representation. If \var{factory} is \code{None}, \class{BabylMessage} is used
as the default message representation. If \var{create} is \code{True}, the
mailbox is created if it does not exist.
\end{classdesc}

Babyl is a single-file mailbox format used by the Rmail mail user agent
included with Emacs. The beginning of a message is indicated by a line
containing the two characters Control-Underscore
(\character{\textbackslash037}) and Control-L (\character{\textbackslash014}).
The end of a message is indicated by the start of the next message or, in the
case of the last message, a line containing a Control-Underscore
(\character{\textbackslash037}) character.

Messages in a Babyl mailbox have two sets of headers, original headers and
so-called visible headers. Visible headers are typically a subset of the
original headers that have been reformatted or abridged to be more attractive.
Each message in a Babyl mailbox also has an accompanying list of \dfn{labels},
or short strings that record extra information about the message, and a list of
all user-defined labels found in the mailbox is kept in the Babyl options
section.

\class{Babyl} instances have all of the methods of \class{Mailbox} in addition
to the following:

\begin{methoddesc}{get_labels}{}
Return a list of the names of all user-defined labels used in the mailbox.
\note{The actual messages are inspected to determine which labels exist in the
mailbox rather than consulting the list of labels in the Babyl options section,
but the Babyl section is updated whenever the mailbox is modified.}
\end{methoddesc}

Some \class{Mailbox} methods implemented by \class{Babyl} deserve special
remarks:

\begin{methoddesc}{get_file}{key}
In Babyl mailboxes, the headers of a message are not stored contiguously with
the body of the message. To generate a file-like representation, the headers
and body are copied together into a \class{StringIO} instance (from the
\module{StringIO} module), which has an API identical to that of a file. As a
result, the file-like object is truly independent of the underlying mailbox but
does not save memory compared to a string representation.
\end{methoddesc}

\begin{methoddesc}{lock}{}
\methodline{unlock}{}
Three locking mechanisms are used---dot locking and, if available, the
\cfunction{flock()} and \cfunction{lockf()} system calls.
\end{methoddesc}

\begin{seealso}
\seelink{http://quimby.gnus.org/notes/BABYL}{Format of Version 5 Babyl Files}{A
specification of the Babyl format.}
\seelink{http://www.gnu.org/software/emacs/manual/html_node/Rmail.html}{Reading
Mail with Rmail}{The Rmail manual, with some information on Babyl semantics.}
\end{seealso}

\subsubsection{\class{MMDF}}
\label{mailbox-mmdf}

\begin{classdesc}{MMDF}{path\optional{, factory=None\optional{, create=True}}}
A subclass of \class{Mailbox} for mailboxes in MMDF format. Parameter
\var{factory} is a callable object that accepts a file-like message
representation (which behaves as if opened in binary mode) and returns a custom
representation. If \var{factory} is \code{None}, \class{MMDFMessage} is used as
the default message representation. If \var{create} is \code{True}, the mailbox
is created if it does not exist.
\end{classdesc}

MMDF is a single-file mailbox format invented for the Multichannel Memorandum
Distribution Facility, a mail transfer agent. Each message is in the same form
as an mbox message but is bracketed before and after by lines containing four
Control-A (\character{\textbackslash001}) characters. As with the mbox format,
the beginning of each message is indicated by a line whose first five
characters are "From~", but additional occurrences of "From~" are not
transformed to ">From~" when storing messages because the extra message
separator lines prevent mistaking such occurrences for the starts of subsequent
messages.

Some \class{Mailbox} methods implemented by \class{MMDF} deserve special
remarks:

\begin{methoddesc}{get_file}{key}
Using the file after calling \method{flush()} or \method{close()} on the
\class{MMDF} instance may yield unpredictable results or raise an exception.
\end{methoddesc}

\begin{methoddesc}{lock}{}
\methodline{unlock}{}
Three locking mechanisms are used---dot locking and, if available, the
\cfunction{flock()} and \cfunction{lockf()} system calls.
\end{methoddesc}

\begin{seealso}
\seelink{http://www.tin.org/bin/man.cgi?section=5\&topic=mmdf}{mmdf man page
from tin}{A specification of MMDF format from the documentation of tin, a
newsreader.}
\seelink{http://en.wikipedia.org/wiki/MMDF}{MMDF}{A Wikipedia article
describing the Multichannel Memorandum Distribution Facility.}
\end{seealso}

\subsection{\class{Message} objects}
\label{mailbox-message-objects}

\begin{classdesc}{Message}{\optional{message}}
A subclass of the \module{email.Message} module's \class{Message}. Subclasses
of \class{mailbox.Message} add mailbox-format-specific state and behavior.

If \var{message} is omitted, the new instance is created in a default, empty
state. If \var{message} is an \class{email.Message.Message} instance, its
contents are copied; furthermore, any format-specific information is converted
insofar as possible if \var{message} is a \class{Message} instance. If
\var{message} is a string or a file, it should contain an \rfc{2822}-compliant
message, which is read and parsed.
\end{classdesc}

The format-specific state and behaviors offered by subclasses vary, but in
general it is only the properties that are not specific to a particular mailbox
that are supported (although presumably the properties are specific to a
particular mailbox format). For example, file offsets for single-file mailbox
formats and file names for directory-based mailbox formats are not retained,
because they are only applicable to the original mailbox. But state such as
whether a message has been read by the user or marked as important is retained,
because it applies to the message itself.

There is no requirement that \class{Message} instances be used to represent
messages retrieved using \class{Mailbox} instances. In some situations, the
time and memory required to generate \class{Message} representations might not
not acceptable. For such situations, \class{Mailbox} instances also offer
string and file-like representations, and a custom message factory may be
specified when a \class{Mailbox} instance is initialized. 

\subsubsection{\class{MaildirMessage}}
\label{mailbox-maildirmessage}

\begin{classdesc}{MaildirMessage}{\optional{message}}
A message with Maildir-specific behaviors. Parameter \var{message}
has the same meaning as with the \class{Message} constructor.
\end{classdesc}

Typically, a mail user agent application moves all of the messages in the
\file{new} subdirectory to the \file{cur} subdirectory after the first time the
user opens and closes the mailbox, recording that the messages are old whether
or not they've actually been read. Each message in \file{cur} has an "info"
section added to its file name to store information about its state. (Some mail
readers may also add an "info" section to messages in \file{new}.) The "info"
section may take one of two forms: it may contain "2," followed by a list of
standardized flags (e.g., "2,FR") or it may contain "1," followed by so-called
experimental information. Standard flags for Maildir messages are as follows:

\begin{tableiii}{l|l|l}{textrm}{Flag}{Meaning}{Explanation}
\lineiii{D}{Draft}{Under composition}
\lineiii{F}{Flagged}{Marked as important}
\lineiii{P}{Passed}{Forwarded, resent, or bounced}
\lineiii{R}{Replied}{Replied to}
\lineiii{S}{Seen}{Read}
\lineiii{T}{Trashed}{Marked for subsequent deletion}
\end{tableiii}

\class{MaildirMessage} instances offer the following methods:

\begin{methoddesc}{get_subdir}{}
Return either "new" (if the message should be stored in the \file{new}
subdirectory) or "cur" (if the message should be stored in the \file{cur}
subdirectory). \note{A message is typically moved from \file{new} to \file{cur}
after its mailbox has been accessed, whether or not the message is has been
read. A message \code{msg} has been read if \code{"S" not in msg.get_flags()}
is \code{True}.}
\end{methoddesc}

\begin{methoddesc}{set_subdir}{subdir}
Set the subdirectory the message should be stored in. Parameter \var{subdir}
must be either "new" or "cur".
\end{methoddesc}

\begin{methoddesc}{get_flags}{}
Return a string specifying the flags that are currently set. If the message
complies with the standard Maildir format, the result is the concatenation in
alphabetical order of zero or one occurrence of each of \character{D},
\character{F}, \character{P}, \character{R}, \character{S}, and \character{T}.
The empty string is returned if no flags are set or if "info" contains
experimental semantics.
\end{methoddesc}

\begin{methoddesc}{set_flags}{flags}
Set the flags specified by \var{flags} and unset all others.
\end{methoddesc}

\begin{methoddesc}{add_flag}{flag}
Set the flag(s) specified by \var{flag} without changing other flags. To add
more than one flag at a time, \var{flag} may be a string of more than one
character. The current "info" is overwritten whether or not it contains
experimental information rather than
flags.
\end{methoddesc}

\begin{methoddesc}{remove_flag}{flag}
Unset the flag(s) specified by \var{flag} without changing other flags. To
remove more than one flag at a time, \var{flag} maybe a string of more than one
character. If "info" contains experimental information rather than flags, the
current "info" is not modified.
\end{methoddesc}

\begin{methoddesc}{get_date}{}
Return the delivery date of the message as a floating-point number representing
seconds since the epoch.
\end{methoddesc}

\begin{methoddesc}{set_date}{date}
Set the delivery date of the message to \var{date}, a floating-point number
representing seconds since the epoch.
\end{methoddesc}

\begin{methoddesc}{get_info}{}
Return a string containing the "info" for a message. This is useful for
accessing and modifying "info" that is experimental (i.e., not a list of
flags).
\end{methoddesc}

\begin{methoddesc}{set_info}{info}
Set "info" to \var{info}, which should be a string.
\end{methoddesc}

When a \class{MaildirMessage} instance is created based upon an
\class{mboxMessage} or \class{MMDFMessage} instance, the \mailheader{Status}
and \mailheader{X-Status} headers are omitted and the following conversions
take place:

\begin{tableii}{l|l}{textrm}
    {Resulting state}{\class{mboxMessage} or \class{MMDFMessage} state}
\lineii{"cur" subdirectory}{O flag}
\lineii{F flag}{F flag}
\lineii{R flag}{A flag}
\lineii{S flag}{R flag}
\lineii{T flag}{D flag}
\end{tableii}

When a \class{MaildirMessage} instance is created based upon an
\class{MHMessage} instance, the following conversions take place:

\begin{tableii}{l|l}{textrm}
    {Resulting state}{\class{MHMessage} state}
\lineii{"cur" subdirectory}{"unseen" sequence}
\lineii{"cur" subdirectory and S flag}{no "unseen" sequence}
\lineii{F flag}{"flagged" sequence}
\lineii{R flag}{"replied" sequence}
\end{tableii}

When a \class{MaildirMessage} instance is created based upon a
\class{BabylMessage} instance, the following conversions take place:

\begin{tableii}{l|l}{textrm}
    {Resulting state}{\class{BabylMessage} state}
\lineii{"cur" subdirectory}{"unseen" label}
\lineii{"cur" subdirectory and S flag}{no "unseen" label}
\lineii{P flag}{"forwarded" or "resent" label}
\lineii{R flag}{"answered" label}
\lineii{T flag}{"deleted" label}
\end{tableii}

\subsubsection{\class{mboxMessage}}
\label{mailbox-mboxmessage}

\begin{classdesc}{mboxMessage}{\optional{message}}
A message with mbox-specific behaviors. Parameter \var{message} has the same
meaning as with the \class{Message} constructor.
\end{classdesc}

Messages in an mbox mailbox are stored together in a single file. The sender's
envelope address and the time of delivery are typically stored in a line
beginning with "From~" that is used to indicate the start of a message, though
there is considerable variation in the exact format of this data among mbox
implementations. Flags that indicate the state of the message, such as whether
it has been read or marked as important, are typically stored in
\mailheader{Status} and \mailheader{X-Status} headers.

Conventional flags for mbox messages are as follows:

\begin{tableiii}{l|l|l}{textrm}{Flag}{Meaning}{Explanation}
\lineiii{R}{Read}{Read}
\lineiii{O}{Old}{Previously detected by MUA}
\lineiii{D}{Deleted}{Marked for subsequent deletion}
\lineiii{F}{Flagged}{Marked as important}
\lineiii{A}{Answered}{Replied to}
\end{tableiii}

The "R" and "O" flags are stored in the \mailheader{Status} header, and the
"D", "F", and "A" flags are stored in the \mailheader{X-Status} header. The
flags and headers typically appear in the order mentioned.

\class{mboxMessage} instances offer the following methods:

\begin{methoddesc}{get_from}{}
Return a string representing the "From~" line that marks the start of the
message in an mbox mailbox. The leading "From~" and the trailing newline are
excluded.
\end{methoddesc}

\begin{methoddesc}{set_from}{from_\optional{, time_=None}}
Set the "From~" line to \var{from_}, which should be specified without a
leading "From~" or trailing newline. For convenience, \var{time_} may be
specified and will be formatted appropriately and appended to \var{from_}. If
\var{time_} is specified, it should be a \class{struct_time} instance, a tuple
suitable for passing to \method{time.strftime()}, or \code{True} (to use
\method{time.gmtime()}).
\end{methoddesc}

\begin{methoddesc}{get_flags}{}
Return a string specifying the flags that are currently set. If the message
complies with the conventional format, the result is the concatenation in the
following order of zero or one occurrence of each of \character{R},
\character{O}, \character{D}, \character{F}, and \character{A}.
\end{methoddesc}

\begin{methoddesc}{set_flags}{flags}
Set the flags specified by \var{flags} and unset all others. Parameter
\var{flags} should be the concatenation in any order of zero or more
occurrences of each of \character{R}, \character{O}, \character{D},
\character{F}, and \character{A}.
\end{methoddesc}

\begin{methoddesc}{add_flag}{flag}
Set the flag(s) specified by \var{flag} without changing other flags. To add
more than one flag at a time, \var{flag} may be a string of more than one
character.
\end{methoddesc}

\begin{methoddesc}{remove_flag}{flag}
Unset the flag(s) specified by \var{flag} without changing other flags. To
remove more than one flag at a time, \var{flag} maybe a string of more than one
character.
\end{methoddesc}

When an \class{mboxMessage} instance is created based upon a
\class{MaildirMessage} instance, a "From~" line is generated based upon the
\class{MaildirMessage} instance's delivery date, and the following conversions
take place:

\begin{tableii}{l|l}{textrm}
    {Resulting state}{\class{MaildirMessage} state}
\lineii{R flag}{S flag}
\lineii{O flag}{"cur" subdirectory}
\lineii{D flag}{T flag}
\lineii{F flag}{F flag}
\lineii{A flag}{R flag}
\end{tableii}

When an \class{mboxMessage} instance is created based upon an \class{MHMessage}
instance, the following conversions take place:

\begin{tableii}{l|l}{textrm}
    {Resulting state}{\class{MHMessage} state}
\lineii{R flag and O flag}{no "unseen" sequence}
\lineii{O flag}{"unseen" sequence}
\lineii{F flag}{"flagged" sequence}
\lineii{A flag}{"replied" sequence}
\end{tableii}

When an \class{mboxMessage} instance is created based upon a
\class{BabylMessage} instance, the following conversions take place:

\begin{tableii}{l|l}{textrm}
    {Resulting state}{\class{BabylMessage} state}
\lineii{R flag and O flag}{no "unseen" label}
\lineii{O flag}{"unseen" label}
\lineii{D flag}{"deleted" label}
\lineii{A flag}{"answered" label}
\end{tableii}

When a \class{Message} instance is created based upon an \class{MMDFMessage}
instance, the "From~" line is copied and all flags directly correspond:

\begin{tableii}{l|l}{textrm}
    {Resulting state}{\class{MMDFMessage} state}
\lineii{R flag}{R flag}
\lineii{O flag}{O flag}
\lineii{D flag}{D flag}
\lineii{F flag}{F flag}
\lineii{A flag}{A flag}
\end{tableii}

\subsubsection{\class{MHMessage}}
\label{mailbox-mhmessage}

\begin{classdesc}{MHMessage}{\optional{message}}
A message with MH-specific behaviors. Parameter \var{message} has the same
meaning as with the \class{Message} constructor.
\end{classdesc}

MH messages do not support marks or flags in the traditional sense, but they do
support sequences, which are logical groupings of arbitrary messages. Some mail
reading programs (although not the standard \program{mh} and \program{nmh}) use
sequences in much the same way flags are used with other formats, as follows:

\begin{tableii}{l|l}{textrm}{Sequence}{Explanation}
\lineii{unseen}{Not read, but previously detected by MUA}
\lineii{replied}{Replied to}
\lineii{flagged}{Marked as important}
\end{tableii}

\class{MHMessage} instances offer the following methods:

\begin{methoddesc}{get_sequences}{}
Return a list of the names of sequences that include this message.
\end{methoddesc}

\begin{methoddesc}{set_sequences}{sequences}
Set the list of sequences that include this message.
\end{methoddesc}

\begin{methoddesc}{add_sequence}{sequence}
Add \var{sequence} to the list of sequences that include this message.
\end{methoddesc}

\begin{methoddesc}{remove_sequence}{sequence}
Remove \var{sequence} from the list of sequences that include this message.
\end{methoddesc}

When an \class{MHMessage} instance is created based upon a
\class{MaildirMessage} instance, the following conversions take place:

\begin{tableii}{l|l}{textrm}
    {Resulting state}{\class{MaildirMessage} state}
\lineii{"unseen" sequence}{no S flag}
\lineii{"replied" sequence}{R flag}
\lineii{"flagged" sequence}{F flag}
\end{tableii}

When an \class{MHMessage} instance is created based upon an \class{mboxMessage}
or \class{MMDFMessage} instance, the \mailheader{Status} and
\mailheader{X-Status} headers are omitted and the following conversions take
place:

\begin{tableii}{l|l}{textrm}
    {Resulting state}{\class{mboxMessage} or \class{MMDFMessage} state}
\lineii{"unseen" sequence}{no R flag}
\lineii{"replied" sequence}{A flag}
\lineii{"flagged" sequence}{F flag}
\end{tableii}

When an \class{MHMessage} instance is created based upon a \class{BabylMessage}
instance, the following conversions take place:

\begin{tableii}{l|l}{textrm}
    {Resulting state}{\class{BabylMessage} state}
\lineii{"unseen" sequence}{"unseen" label}
\lineii{"replied" sequence}{"answered" label}
\end{tableii}

\subsubsection{\class{BabylMessage}}
\label{mailbox-babylmessage}

\begin{classdesc}{BabylMessage}{\optional{message}}
A message with Babyl-specific behaviors. Parameter \var{message} has the same
meaning as with the \class{Message} constructor.
\end{classdesc}

Certain message labels, called \dfn{attributes}, are defined by convention to
have special meanings. The attributes are as follows:

\begin{tableii}{l|l}{textrm}{Label}{Explanation}
\lineii{unseen}{Not read, but previously detected by MUA}
\lineii{deleted}{Marked for subsequent deletion}
\lineii{filed}{Copied to another file or mailbox}
\lineii{answered}{Replied to}
\lineii{forwarded}{Forwarded}
\lineii{edited}{Modified by the user}
\lineii{resent}{Resent}
\end{tableii}

By default, Rmail displays only
visible headers. The \class{BabylMessage} class, though, uses the original
headers because they are more complete. Visible headers may be accessed
explicitly if desired.

\class{BabylMessage} instances offer the following methods:

\begin{methoddesc}{get_labels}{}
Return a list of labels on the message.
\end{methoddesc}

\begin{methoddesc}{set_labels}{labels}
Set the list of labels on the message to \var{labels}.
\end{methoddesc}

\begin{methoddesc}{add_label}{label}
Add \var{label} to the list of labels on the message.
\end{methoddesc}

\begin{methoddesc}{remove_label}{label}
Remove \var{label} from the list of labels on the message.
\end{methoddesc}

\begin{methoddesc}{get_visible}{}
Return an \class{Message} instance whose headers are the message's visible
headers and whose body is empty.
\end{methoddesc}

\begin{methoddesc}{set_visible}{visible}
Set the message's visible headers to be the same as the headers in
\var{message}. Parameter \var{visible} should be a \class{Message} instance, an
\class{email.Message.Message} instance, a string, or a file-like object (which
should be open in text mode).
\end{methoddesc}

\begin{methoddesc}{update_visible}{}
When a \class{BabylMessage} instance's original headers are modified, the
visible headers are not automatically modified to correspond. This method
updates the visible headers as follows: each visible header with a
corresponding original header is set to the value of the original header, each
visible header without a corresponding original header is removed, and any of
\mailheader{Date}, \mailheader{From}, \mailheader{Reply-To}, \mailheader{To},
\mailheader{CC}, and \mailheader{Subject} that are present in the original
headers but not the visible headers are added to the visible headers.
\end{methoddesc}

When a \class{BabylMessage} instance is created based upon a
\class{MaildirMessage} instance, the following conversions take place:

\begin{tableii}{l|l}{textrm}
    {Resulting state}{\class{MaildirMessage} state}
\lineii{"unseen" label}{no S flag}
\lineii{"deleted" label}{T flag}
\lineii{"answered" label}{R flag}
\lineii{"forwarded" label}{P flag}
\end{tableii}

When a \class{BabylMessage} instance is created based upon an
\class{mboxMessage} or \class{MMDFMessage} instance, the \mailheader{Status}
and \mailheader{X-Status} headers are omitted and the following conversions
take place:

\begin{tableii}{l|l}{textrm}
    {Resulting state}{\class{mboxMessage} or \class{MMDFMessage} state}
\lineii{"unseen" label}{no R flag}
\lineii{"deleted" label}{D flag}
\lineii{"answered" label}{A flag}
\end{tableii}

When a \class{BabylMessage} instance is created based upon an \class{MHMessage}
instance, the following conversions take place:

\begin{tableii}{l|l}{textrm}
    {Resulting state}{\class{MHMessage} state}
\lineii{"unseen" label}{"unseen" sequence}
\lineii{"answered" label}{"replied" sequence}
\end{tableii}

\subsubsection{\class{MMDFMessage}}
\label{mailbox-mmdfmessage}

\begin{classdesc}{MMDFMessage}{\optional{message}}
A message with MMDF-specific behaviors. Parameter \var{message} has the same
meaning as with the \class{Message} constructor.
\end{classdesc}

As with message in an mbox mailbox, MMDF messages are stored with the sender's
address and the delivery date in an initial line beginning with "From ".
Likewise, flags that indicate the state of the message are typically stored in
\mailheader{Status} and \mailheader{X-Status} headers.

Conventional flags for MMDF messages are identical to those of mbox message and
are as follows:

\begin{tableiii}{l|l|l}{textrm}{Flag}{Meaning}{Explanation}
\lineiii{R}{Read}{Read}
\lineiii{O}{Old}{Previously detected by MUA}
\lineiii{D}{Deleted}{Marked for subsequent deletion}
\lineiii{F}{Flagged}{Marked as important}
\lineiii{A}{Answered}{Replied to}
\end{tableiii}

The "R" and "O" flags are stored in the \mailheader{Status} header, and the
"D", "F", and "A" flags are stored in the \mailheader{X-Status} header. The
flags and headers typically appear in the order mentioned.

\class{MMDFMessage} instances offer the following methods, which are identical
to those offered by \class{mboxMessage}:

\begin{methoddesc}{get_from}{}
Return a string representing the "From~" line that marks the start of the
message in an mbox mailbox. The leading "From~" and the trailing newline are
excluded.
\end{methoddesc}

\begin{methoddesc}{set_from}{from_\optional{, time_=None}}
Set the "From~" line to \var{from_}, which should be specified without a
leading "From~" or trailing newline. For convenience, \var{time_} may be
specified and will be formatted appropriately and appended to \var{from_}. If
\var{time_} is specified, it should be a \class{struct_time} instance, a tuple
suitable for passing to \method{time.strftime()}, or \code{True} (to use
\method{time.gmtime()}).
\end{methoddesc}

\begin{methoddesc}{get_flags}{}
Return a string specifying the flags that are currently set. If the message
complies with the conventional format, the result is the concatenation in the
following order of zero or one occurrence of each of \character{R},
\character{O}, \character{D}, \character{F}, and \character{A}.
\end{methoddesc}

\begin{methoddesc}{set_flags}{flags}
Set the flags specified by \var{flags} and unset all others. Parameter
\var{flags} should be the concatenation in any order of zero or more
occurrences of each of \character{R}, \character{O}, \character{D},
\character{F}, and \character{A}.
\end{methoddesc}

\begin{methoddesc}{add_flag}{flag}
Set the flag(s) specified by \var{flag} without changing other flags. To add
more than one flag at a time, \var{flag} may be a string of more than one
character.
\end{methoddesc}

\begin{methoddesc}{remove_flag}{flag}
Unset the flag(s) specified by \var{flag} without changing other flags. To
remove more than one flag at a time, \var{flag} maybe a string of more than one
character.
\end{methoddesc}

When an \class{MMDFMessage} instance is created based upon a
\class{MaildirMessage} instance, a "From~" line is generated based upon the
\class{MaildirMessage} instance's delivery date, and the following conversions
take place:

\begin{tableii}{l|l}{textrm}
    {Resulting state}{\class{MaildirMessage} state}
\lineii{R flag}{S flag}
\lineii{O flag}{"cur" subdirectory}
\lineii{D flag}{T flag}
\lineii{F flag}{F flag}
\lineii{A flag}{R flag}
\end{tableii}

When an \class{MMDFMessage} instance is created based upon an \class{MHMessage}
instance, the following conversions take place:

\begin{tableii}{l|l}{textrm}
    {Resulting state}{\class{MHMessage} state}
\lineii{R flag and O flag}{no "unseen" sequence}
\lineii{O flag}{"unseen" sequence}
\lineii{F flag}{"flagged" sequence}
\lineii{A flag}{"replied" sequence}
\end{tableii}

When an \class{MMDFMessage} instance is created based upon a
\class{BabylMessage} instance, the following conversions take place:

\begin{tableii}{l|l}{textrm}
    {Resulting state}{\class{BabylMessage} state}
\lineii{R flag and O flag}{no "unseen" label}
\lineii{O flag}{"unseen" label}
\lineii{D flag}{"deleted" label}
\lineii{A flag}{"answered" label}
\end{tableii}

When an \class{MMDFMessage} instance is created based upon an
\class{mboxMessage} instance, the "From~" line is copied and all flags directly
correspond:

\begin{tableii}{l|l}{textrm}
    {Resulting state}{\class{mboxMessage} state}
\lineii{R flag}{R flag}
\lineii{O flag}{O flag}
\lineii{D flag}{D flag}
\lineii{F flag}{F flag}
\lineii{A flag}{A flag}
\end{tableii}

\subsection{Exceptions}
\label{mailbox-deprecated}

The following exception classes are defined in the \module{mailbox} module:

\begin{classdesc}{Error}{}
The based class for all other module-specific exceptions.
\end{classdesc}

\begin{classdesc}{NoSuchMailboxError}{}
Raised when a mailbox is expected but is not found, such as when instantiating
a \class{Mailbox} subclass with a path that does not exist (and with the
\var{create} parameter set to \code{False}), or when opening a folder that does
not exist.
\end{classdesc}

\begin{classdesc}{NotEmptyErrorError}{}
Raised when a mailbox is not empty but is expected to be, such as when deleting
a folder that contains messages.
\end{classdesc}

\begin{classdesc}{ExternalClashError}{}
Raised when some mailbox-related condition beyond the control of the program
causes it to be unable to proceed, such as when failing to acquire a lock that
another program already holds a lock, or when a uniquely-generated file name
already exists.
\end{classdesc}

\begin{classdesc}{FormatError}{}
Raised when the data in a file cannot be parsed, such as when an \class{MH}
instance attempts to read a corrupted \file{.mh_sequences} file.
\end{classdesc}

\subsection{Deprecated classes and methods}
\label{mailbox-deprecated}

Older versions of the \module{mailbox} module do not support modification of
mailboxes, such as adding or removing message, and do not provide classes to
represent format-specific message properties. For backward compatibility, the
older mailbox classes are still available, but the newer classes should be used
in preference to them.

Older mailbox objects support only iteration and provide a single public
method:

\begin{methoddesc}{next}{}
Return the next message in the mailbox, created with the optional \var{factory}
argument passed into the mailbox object's constructor. By default this is an
\class{rfc822.Message} object (see the \refmodule{rfc822} module).  Depending
on the mailbox implementation the \var{fp} attribute of this object may be a
true file object or a class instance simulating a file object, taking care of
things like message boundaries if multiple mail messages are contained in a
single file, etc.  If no more messages are available, this method returns
\code{None}.
\end{methoddesc}

Most of the older mailbox classes have names that differ from the current
mailbox class names, except for \class{Maildir}. For this reason, the new
\class{Maildir} class defines a \method{next()} method and its constructor
differs slightly from those of the other new mailbox classes.

The older mailbox classes whose names are not the same as their newer
counterparts are as follows:

\begin{classdesc}{UnixMailbox}{fp\optional{, factory}}
Access to a classic \UNIX-style mailbox, where all messages are
contained in a single file and separated by \samp{From }
(a.k.a.\ \samp{From_}) lines.  The file object \var{fp} points to the
mailbox file.  The optional \var{factory} parameter is a callable that
should create new message objects.  \var{factory} is called with one
argument, \var{fp} by the \method{next()} method of the mailbox
object.  The default is the \class{rfc822.Message} class (see the
\refmodule{rfc822} module -- and the note below).

\begin{notice}
  For reasons of this module's internal implementation, you will
  probably want to open the \var{fp} object in binary mode.  This is
  especially important on Windows.
\end{notice}

For maximum portability, messages in a \UNIX-style mailbox are
separated by any line that begins exactly with the string \code{'From
'} (note the trailing space) if preceded by exactly two newlines.
Because of the wide-range of variations in practice, nothing else on
the From_ line should be considered.  However, the current
implementation doesn't check for the leading two newlines.  This is
usually fine for most applications.

The \class{UnixMailbox} class implements a more strict version of
From_ line checking, using a regular expression that usually correctly
matched From_ delimiters.  It considers delimiter line to be separated
by \samp{From \var{name} \var{time}} lines.  For maximum portability,
use the \class{PortableUnixMailbox} class instead.  This class is
identical to \class{UnixMailbox} except that individual messages are
separated by only \samp{From } lines.

For more information, see
\citetitle[http://home.netscape.com/eng/mozilla/2.0/relnotes/demo/content-length.html]{Configuring
Netscape Mail on \UNIX: Why the Content-Length Format is Bad}.
\end{classdesc}

\begin{classdesc}{PortableUnixMailbox}{fp\optional{, factory}}
A less-strict version of \class{UnixMailbox}, which considers only the
\samp{From } at the beginning of the line separating messages.  The
``\var{name} \var{time}'' portion of the From line is ignored, to
protect against some variations that are observed in practice.  This
works since lines in the message which begin with \code{'From '} are
quoted by mail handling software at delivery-time.
\end{classdesc}

\begin{classdesc}{MmdfMailbox}{fp\optional{, factory}}
Access an MMDF-style mailbox, where all messages are contained
in a single file and separated by lines consisting of 4 control-A
characters.  The file object \var{fp} points to the mailbox file.
Optional \var{factory} is as with the \class{UnixMailbox} class.
\end{classdesc}

\begin{classdesc}{MHMailbox}{dirname\optional{, factory}}
Access an MH mailbox, a directory with each message in a separate
file with a numeric name.
The name of the mailbox directory is passed in \var{dirname}.
\var{factory} is as with the \class{UnixMailbox} class.
\end{classdesc}

\begin{classdesc}{BabylMailbox}{fp\optional{, factory}}
Access a Babyl mailbox, which is similar to an MMDF mailbox.  In
Babyl format, each message has two sets of headers, the
\emph{original} headers and the \emph{visible} headers.  The original
headers appear before a line containing only \code{'*** EOOH ***'}
(End-Of-Original-Headers) and the visible headers appear after the
\code{EOOH} line.  Babyl-compliant mail readers will show you only the
visible headers, and \class{BabylMailbox} objects will return messages
containing only the visible headers.  You'll have to do your own
parsing of the mailbox file to get at the original headers.  Mail
messages start with the EOOH line and end with a line containing only
\code{'\e{}037\e{}014'}.  \var{factory} is as with the
\class{UnixMailbox} class.
\end{classdesc}

If you wish to use the older mailbox classes with the \module{email} module
rather than the deprecated \module{rfc822} module, you can do so as follows:

\begin{verbatim}
import email
import email.Errors
import mailbox

def msgfactory(fp):
    try:
        return email.message_from_file(fp)
    except email.Errors.MessageParseError:
        # Don't return None since that will
        # stop the mailbox iterator
        return ''

mbox = mailbox.UnixMailbox(fp, msgfactory)
\end{verbatim}

Alternatively, if you know your mailbox contains only well-formed MIME
messages, you can simplify this to:

\begin{verbatim}
import email
import mailbox

mbox = mailbox.UnixMailbox(fp, email.message_from_file)
\end{verbatim}

\subsection{Examples}
\label{mailbox-examples}

A simple example of printing the subjects of all messages in a mailbox that
seem interesting:

\begin{verbatim}
import mailbox
for message in mailbox.mbox('~/mbox'):
    subject = message['subject']       # Could possibly be None.
    if subject and 'python' in subject.lower():
        print subject
\end{verbatim}

To copy all mail from a Babyl mailbox to an MH mailbox, converting all
of the format-specific information that can be converted:

\begin{verbatim}
import mailbox
destination = mailbox.MH('~/Mail')
for message in mailbox.Babyl('~/RMAIL'):
    destination.add(MHMessage(message))
\end{verbatim}

An example of sorting mail from numerous mailing lists, being careful to avoid
mail corruption due to concurrent modification by other programs, mail loss due
to interruption of the program, or premature termination due to malformed
messages in the mailbox:

\begin{verbatim}
import mailbox
import email.Errors
list_names = ('python-list', 'python-dev', 'python-bugs')
boxes = dict((name, mailbox.mbox('~/email/%s' % name)) for name in list_names)
inbox = mailbox.Maildir('~/Maildir', None)
for key in inbox.iterkeys():
    try:
        message = inbox[key]
    except email.Errors.MessageParseError:
        continue                # The message is malformed. Just leave it.
    for name in list_names:
        list_id = message['list-id']
        if list_id and name in list_id:
            box = boxes[name]
            box.lock()
            box.add(message)
            box.flush()         # Write copy to disk before removing original.
            box.unlock()
            inbox.discard(key)
            break               # Found destination, so stop looking.
for box in boxes.itervalues():
    box.close()
\end{verbatim}

\section{Standard Module \sectcode{mimify}}
\label{module-mimify}
\stmodindex{mimify}

The mimify module defines two functions to convert mail messages to
and from MIME format.  The mail message can be either a simple message
or a so-called multipart message.  Each part is treated separately.
Mimifying (a part of) a message entails encoding the message as
quoted-printable if it contains any characters that cannot be
represented using 7-bit ASCII.  Unmimifying (a part of) a message
entails undoing the quoted-printable encoding.  Mimify and unmimify
are especially useful when a message has to be edited before being
sent.  Typical use would be:

\begin{verbatim}
unmimify message
edit message
mimify message
send message
\end{verbatim}

The modules defines the following user-callable functions and
user-settable variables:

\begin{funcdesc}{mimify}{infile, outfile}
Copy the message in \var{infile} to \var{outfile}, converting parts to
quoted-printable and adding MIME mail headers when necessary.
\var{infile} and \var{outfile} can be file objects (actually, any
object that has a \code{readline} method (for \var{infile}) or a
\code{write} method (for \var{outfile})) or strings naming the files.
If \var{infile} and \var{outfile} are both strings, they may have the
same value.
\end{funcdesc}

\begin{funcdesc}{unmimify}{infile, outfile, decode_base64 = 0} 
Copy the message in \var{infile} to \var{outfile}, decoding all
quoted-printable parts.  \var{infile} and \var{outfile} can be file
objects (actually, any object that has a \code{readline} method (for
\var{infile}) or a \code{write} method (for \var{outfile})) or strings
naming the files.  If \var{infile} and \var{outfile} are both strings,
they may have the same value.
If the \var{decode_base64} argument is provided and tests true, any
parts that are coded in the base64 encoding are decoded as well.
\end{funcdesc}

\begin{funcdesc}{mime_decode_header}{line}
Return a decoded version of the encoded header line in \var{line}.
\end{funcdesc}

\begin{funcdesc}{mime_encode_header}{line}
Return a MIME-encoded version of the header line in \var{line}.
\end{funcdesc}

\begin{datadesc}{MAXLEN}
By default, a part will be encoded as quoted-printable when it
contains any non-ASCII characters (i.e., characters with the 8th bit
set), or if there are any lines longer than \code{MAXLEN} characters
(default value 200).  
\end{datadesc}

\begin{datadesc}{CHARSET}
When not specified in the mail headers, a character set must be filled
in.  The string used is stored in \code{CHARSET}, and the default
value is ISO-8859-1 (also known as Latin1 (latin-one)).
\end{datadesc}

This module can also be used from the command line.  Usage is as
follows:
\begin{verbatim}
mimify.py -e [-l length] [infile [outfile]]
mimify.py -d [-b] [infile [outfile]]
\end{verbatim}
to encode (mimify) and decode (unmimify) respectively.  \var{infile}
defaults to standard input, \var{outfile} defaults to standard output.
The same file can be specified for input and output.

If the \code{-l} option is given when encoding, if there are any lines
longer than the specified \var{length}, the containing part will be
encoded.

If the \code{-b} option is given when decoding, any base64 parts will
be decoded as well.


% Module and documentation by Eric S. Raymond, 21 Dec 1998 
\section{Standard Module \module{netrc}}
\stmodindex{netrc}
\label{module-netrc}

The \code{netrc} class parses and encapsulates the netrc file format
used by Unix's ftp(1) and other FTP clientd

\begin{classdesc}{netrc}{\optional{file}}
A \class{netrc} instance or subclass instance enapsulates data from 
a netrc file.  The initialization argument, if present, specifies the file
to parse.  If no argument is given, the file .netrc in the user's home
directory will be read.  Parse errors will throw a SyntaxError
exception with associated diagnostic information including the file
name, line number, and terminating token.
\end{classdesc}

\subsection{netrc Objects}
\label{netrc-objects}

A \class{netrc} instance has the following methods:

\begin{methoddesc}{authenticators}{}
Return a 3-tuple (login, account, password) of authenticators for the
given host.  If the netrc file did not contain an entry for the given
host, return the tuple associated with the `default' entry.  If
neither matching host nor default entry is available, return None.
\end{methoddesc}

\begin{methoddesc}{__repr__}{host}
Dump the class data as a string in the format of a netrc file.
(This discards comments and may reorder the entries.)
\end{methoddesc}

Instances of \class{netrc} have public instance variables:

\begin{memberdesc}{hosts}
Dictionmary mapping host names to login/account/password tuples.  The
`default' entry, if any, is represented as a pseudo-host by that name.
\end{memberdesc}

\begin{memberdesc}{macros}
Dictionary mapping macro names to string lists.
\end{memberdesc}





\chapter{Restricted Execution}
\label{restricted}

In general, Python programs have complete access to the underlying
operating system throug the various functions and classes, For
example, a Python program can open any file for reading and writing by
using the \code{open()} built-in function (provided the underlying OS
gives you permission!).  This is exactly what you want for most
applications.

There exists a class of applications for which this ``openness'' is
inappropriate.  Take Grail: a web browser that accepts ``applets'',
snippets of Python code, from anywhere on the Internet for execution
on the local system.  This can be used to improve the user interface
of forms, for instance.  Since the originator of the code is unknown,
it is obvious that it cannot be trusted with the full resources of the
local machine.

\emph{Restricted execution} is the basic framework in Python that allows
for the segregation of trusted and untrusted code.  It is based on the
notion that trusted Python code (a \emph{supervisor}) can create a
``padded cell' (or environment) with limited permissions, and run the
untrusted code within this cell.  The untrusted code cannot break out
of its cell, and can only interact with sensitive system resources
through interfaces defined and managed by the trusted code.  The term
``restricted execution'' is favored over ``safe-Python''
since true safety is hard to define, and is determined by the way the
restricted environment is created.  Note that the restricted
environments can be nested, with inner cells creating subcells of
lesser, but never greater, privilege.

An interesting aspect of Python's restricted execution model is that
the interfaces presented to untrusted code usually have the same names
as those presented to trusted code.  Therefore no special interfaces
need to be learned to write code designed to run in a restricted
environment.  And because the exact nature of the padded cell is
determined by the supervisor, different restrictions can be imposed,
depending on the application.  For example, it might be deemed
``safe'' for untrusted code to read any file within a specified
directory, but never to write a file.  In this case, the supervisor
may redefine the built-in
\code{open()} function so that it raises an exception whenever the
\var{mode} parameter is \code{'w'}.  It might also perform a
\code{chroot()}-like operation on the \var{filename} parameter, such
that root is always relative to some safe ``sandbox'' area of the
filesystem.  In this case, the untrusted code would still see an
built-in \code{open()} function in its environment, with the same
calling interface.  The semantics would be identical too, with
\code{IOError}s being raised when the supervisor determined that an
unallowable parameter is being used.

The Python run-time determines whether a particular code block is
executing in restricted execution mode based on the identity of the
\code{__builtins__} object in its global variables: if this is (the
dictionary of) the standard \code{__builtin__} module, the code is
deemed to be unrestricted, else it is deemed to be restricted.

Python code executing in restricted mode faces a number of limitations
that are designed to prevent it from escaping from the padded cell.
For instance, the function object attribute \code{func_globals} and the
class and instance object attribute \code{__dict__} are unavailable.

Two modules provide the framework for setting up restricted execution
environments:

\begin{description}

\item[rexec]
--- Basic restricted execution framework.

\item[Bastion]
--- Providing restricted access to objects.

\end{description}

\section{\module{rexec} ---
         Restricted execution framework}

\declaremodule{standard}{rexec}
\modulesynopsis{Basic restricted execution framework.}
\versionchanged[Disabled module]{2.3}

\begin{notice}[warning]
  The documentation has been left in place to help in reading old code
  that uses the module.
\end{notice}

This module contains the \class{RExec} class, which supports
\method{r_eval()}, \method{r_execfile()}, \method{r_exec()}, and
\method{r_import()} methods, which are restricted versions of the standard
Python functions \method{eval()}, \method{execfile()} and
the \keyword{exec} and \keyword{import} statements.
Code executed in this restricted environment will
only have access to modules and functions that are deemed safe; you
can subclass \class{RExec} to add or remove capabilities as desired.

\begin{notice}[warning]
  While the \module{rexec} module is designed to perform as described
  below, it does have a few known vulnerabilities which could be
  exploited by carefully written code.  Thus it should not be relied
  upon in situations requiring ``production ready'' security.  In such
  situations, execution via sub-processes or very careful
  ``cleansing'' of both code and data to be processed may be
  necessary.  Alternatively, help in patching known \module{rexec}
  vulnerabilities would be welcomed.
\end{notice}

\begin{notice}
  The \class{RExec} class can prevent code from performing unsafe
  operations like reading or writing disk files, or using TCP/IP
  sockets.  However, it does not protect against code using extremely
  large amounts of memory or processor time.
\end{notice}

\begin{classdesc}{RExec}{\optional{hooks\optional{, verbose}}}
Returns an instance of the \class{RExec} class.  

\var{hooks} is an instance of the \class{RHooks} class or a subclass of it.
If it is omitted or \code{None}, the default \class{RHooks} class is
instantiated.
Whenever the \module{rexec} module searches for a module (even a
built-in one) or reads a module's code, it doesn't actually go out to
the file system itself.  Rather, it calls methods of an \class{RHooks}
instance that was passed to or created by its constructor.  (Actually,
the \class{RExec} object doesn't make these calls --- they are made by
a module loader object that's part of the \class{RExec} object.  This
allows another level of flexibility, which can be useful when changing
the mechanics of \keyword{import} within the restricted environment.)

By providing an alternate \class{RHooks} object, we can control the
file system accesses made to import a module, without changing the
actual algorithm that controls the order in which those accesses are
made.  For instance, we could substitute an \class{RHooks} object that
passes all filesystem requests to a file server elsewhere, via some
RPC mechanism such as ILU.  Grail's applet loader uses this to support
importing applets from a URL for a directory.

If \var{verbose} is true, additional debugging output may be sent to
standard output.
\end{classdesc}

It is important to be aware that code running in a restricted
environment can still call the \function{sys.exit()} function.  To
disallow restricted code from exiting the interpreter, always protect
calls that cause restricted code to run with a
\keyword{try}/\keyword{except} statement that catches the
\exception{SystemExit} exception.  Removing the \function{sys.exit()}
function from the restricted environment is not sufficient --- the
restricted code could still use \code{raise SystemExit}.  Removing
\exception{SystemExit} is not a reasonable option; some library code
makes use of this and would break were it not available.


\begin{seealso}
  \seetitle[http://grail.sourceforge.net/]{Grail Home Page}{Grail is a
            Web browser written entirely in Python.  It uses the
            \module{rexec} module as a foundation for supporting
            Python applets, and can be used as an example usage of
            this module.}
\end{seealso}


\subsection{RExec Objects \label{rexec-objects}}

\class{RExec} instances support the following methods:

\begin{methoddesc}{r_eval}{code}
\var{code} must either be a string containing a Python expression, or
a compiled code object, which will be evaluated in the restricted
environment's \module{__main__} module.  The value of the expression or
code object will be returned.
\end{methoddesc}

\begin{methoddesc}{r_exec}{code}
\var{code} must either be a string containing one or more lines of
Python code, or a compiled code object, which will be executed in the
restricted environment's \module{__main__} module.
\end{methoddesc}

\begin{methoddesc}{r_execfile}{filename}
Execute the Python code contained in the file \var{filename} in the
restricted environment's \module{__main__} module.
\end{methoddesc}

Methods whose names begin with \samp{s_} are similar to the functions
beginning with \samp{r_}, but the code will be granted access to
restricted versions of the standard I/O streams \code{sys.stdin},
\code{sys.stderr}, and \code{sys.stdout}.

\begin{methoddesc}{s_eval}{code}
\var{code} must be a string containing a Python expression, which will
be evaluated in the restricted environment.  
\end{methoddesc}

\begin{methoddesc}{s_exec}{code}
\var{code} must be a string containing one or more lines of Python code,
which will be executed in the restricted environment.  
\end{methoddesc}

\begin{methoddesc}{s_execfile}{code}
Execute the Python code contained in the file \var{filename} in the
restricted environment.
\end{methoddesc}

\class{RExec} objects must also support various methods which will be
implicitly called by code executing in the restricted environment.
Overriding these methods in a subclass is used to change the policies
enforced by a restricted environment.

\begin{methoddesc}{r_import}{modulename\optional{, globals\optional{,
                             locals\optional{, fromlist}}}}
Import the module \var{modulename}, raising an \exception{ImportError}
exception if the module is considered unsafe.
\end{methoddesc}

\begin{methoddesc}{r_open}{filename\optional{, mode\optional{, bufsize}}}
Method called when \function{open()} is called in the restricted
environment.  The arguments are identical to those of \function{open()},
and a file object (or a class instance compatible with file objects)
should be returned.  \class{RExec}'s default behaviour is allow opening
any file for reading, but forbidding any attempt to write a file.  See
the example below for an implementation of a less restrictive
\method{r_open()}.
\end{methoddesc}

\begin{methoddesc}{r_reload}{module}
Reload the module object \var{module}, re-parsing and re-initializing it.  
\end{methoddesc}

\begin{methoddesc}{r_unload}{module}
Unload the module object \var{module} (remove it from the
restricted environment's \code{sys.modules} dictionary).
\end{methoddesc}

And their equivalents with access to restricted standard I/O streams:

\begin{methoddesc}{s_import}{modulename\optional{, globals\optional{,
                             locals\optional{, fromlist}}}}
Import the module \var{modulename}, raising an \exception{ImportError}
exception if the module is considered unsafe.
\end{methoddesc}

\begin{methoddesc}{s_reload}{module}
Reload the module object \var{module}, re-parsing and re-initializing it.  
\end{methoddesc}

\begin{methoddesc}{s_unload}{module}
Unload the module object \var{module}.   
% XXX what are the semantics of this?  
\end{methoddesc}


\subsection{Defining restricted environments \label{rexec-extension}}

The \class{RExec} class has the following class attributes, which are
used by the \method{__init__()} method.  Changing them on an existing
instance won't have any effect; instead, create a subclass of
\class{RExec} and assign them new values in the class definition.
Instances of the new class will then use those new values.  All these
attributes are tuples of strings.

\begin{memberdesc}{nok_builtin_names}
Contains the names of built-in functions which will \emph{not} be
available to programs running in the restricted environment.  The
value for \class{RExec} is \code{('open', 'reload', '__import__')}.
(This gives the exceptions, because by far the majority of built-in
functions are harmless.  A subclass that wants to override this
variable should probably start with the value from the base class and
concatenate additional forbidden functions --- when new dangerous
built-in functions are added to Python, they will also be added to
this module.)
\end{memberdesc}

\begin{memberdesc}{ok_builtin_modules}
Contains the names of built-in modules which can be safely imported.
The value for \class{RExec} is \code{('audioop', 'array', 'binascii',
'cmath', 'errno', 'imageop', 'marshal', 'math', 'md5', 'operator',
'parser', 'regex', 'rotor', 'select', 'sha', '_sre', 'strop',
'struct', 'time')}.  A similar remark about overriding this variable
applies --- use the value from the base class as a starting point.
\end{memberdesc}

\begin{memberdesc}{ok_path}
Contains the directories which will be searched when an \keyword{import}
is performed in the restricted environment.  
The value for \class{RExec} is the same as \code{sys.path} (at the time
the module is loaded) for unrestricted code.
\end{memberdesc}

\begin{memberdesc}{ok_posix_names}
% Should this be called ok_os_names?
Contains the names of the functions in the \refmodule{os} module which will be
available to programs running in the restricted environment.  The
value for \class{RExec} is \code{('error', 'fstat', 'listdir',
'lstat', 'readlink', 'stat', 'times', 'uname', 'getpid', 'getppid',
'getcwd', 'getuid', 'getgid', 'geteuid', 'getegid')}.
\end{memberdesc}

\begin{memberdesc}{ok_sys_names}
Contains the names of the functions and variables in the \refmodule{sys}
module which will be available to programs running in the restricted
environment.  The value for \class{RExec} is \code{('ps1', 'ps2',
'copyright', 'version', 'platform', 'exit', 'maxint')}.
\end{memberdesc}

\begin{memberdesc}{ok_file_types}
Contains the file types from which modules are allowed to be loaded.
Each file type is an integer constant defined in the \refmodule{imp} module.
The meaningful values are \constant{PY_SOURCE}, \constant{PY_COMPILED}, and
\constant{C_EXTENSION}.  The value for \class{RExec} is \code{(C_EXTENSION,
PY_SOURCE)}.  Adding \constant{PY_COMPILED} in subclasses is not recommended;
an attacker could exit the restricted execution mode by putting a forged
byte-compiled file (\file{.pyc}) anywhere in your file system, for example
by writing it to \file{/tmp} or uploading it to the \file{/incoming}
directory of your public FTP server.
\end{memberdesc}


\subsection{An example}

Let us say that we want a slightly more relaxed policy than the
standard \class{RExec} class.  For example, if we're willing to allow
files in \file{/tmp} to be written, we can subclass the \class{RExec}
class:

\begin{verbatim}
class TmpWriterRExec(rexec.RExec):
    def r_open(self, file, mode='r', buf=-1):
        if mode in ('r', 'rb'):
            pass
        elif mode in ('w', 'wb', 'a', 'ab'):
            # check filename : must begin with /tmp/
            if file[:5]!='/tmp/': 
                raise IOError, "can't write outside /tmp"
            elif (string.find(file, '/../') >= 0 or
                 file[:3] == '../' or file[-3:] == '/..'):
                raise IOError, "'..' in filename forbidden"
        else: raise IOError, "Illegal open() mode"
        return open(file, mode, buf)
\end{verbatim}
%
Notice that the above code will occasionally forbid a perfectly valid
filename; for example, code in the restricted environment won't be
able to open a file called \file{/tmp/foo/../bar}.  To fix this, the
\method{r_open()} method would have to simplify the filename to
\file{/tmp/bar}, which would require splitting apart the filename and
performing various operations on it.  In cases where security is at
stake, it may be preferable to write simple code which is sometimes
overly restrictive, instead of more general code that is also more
complex and may harbor a subtle security hole.

\section{Standard Module \sectcode{Bastion}}
\stmodindex{Bastion}
\renewcommand{\indexsubitem}{(in module Bastion)}

% I'm concerned that the word 'bastion' won't be understood by people
% for whom English is a second language, making the module name
% somewhat mysterious.  Thus, the brief definition... --amk

According to the dictionary, a bastion is ``a fortified area or
position'', or ``something that is considered a stronghold.''  It's a
suitable name for this module, which provides a way to forbid access
to certain attributes of an object.  It must always be used with the
\code{rexec} module, in order to allow restricted-mode programs access
to certain safe attributes of an object, while denying access to
other, unsafe attributes.

% I've punted on the issue of documenting keyword arguments for now.

\begin{funcdesc}{Bastion}{object\optional{\, filter\, name\, class}}
Protect the class instance \var{object}, returning a bastion for the
object.  Any attempt to access one of the object's attributes will
have to be approved by the \var{filter} function; if the access is
denied an AttributeError exception will be raised.

If present, \var{filter} must be a function that accepts a string
containing an attribute name, and returns true if access to that
attribute will be permitted; if \var{filter} returns false, the access
is denied.  The default filter denies access to any function beginning
with an underscore (\code{_}).  The bastion's string representation
will be \code{<Bastion for \var{name}>} if a value for
\var{name} is provided; otherwise, \code{repr(\var{object})} will be used.

\var{class}, if present, would be a subclass of \code{BastionClass};
see the code in \file{bastion.py} for the details.  Overriding the
default \code{BastionClass} will rarely be required.  

\end{funcdesc}


\chapter{Multimedia Services}
\label{mmedia}

The modules described in this chapter implement various algorithms or
interfaces that are mainly useful for multimedia applications.  They
are available at the discretion of the installation.  Here's an overview:

\begin{description}

\item[audioop]
--- Manipulate raw audio data.

\item[imageop]
--- Manipulate raw image data.

\item[aifc]
--- Read and write audio files in AIFF or AIFC format.

\item[jpeg]
--- Read and write image files in compressed JPEG format.

\item[rgbimg]
--- Read and write image files in ``SGI RGB'' format (the module is
\emph{not} SGI specific though)!

\item[imghdr]
--- Determine the type of image contained in a file or byte stream.

\end{description}
			% Multimedia Services
\section{Built-in Module \sectcode{audioop}}
\label{module-audioop}
\bimodindex{audioop}

The \code{audioop} module contains some useful operations on sound fragments.
It operates on sound fragments consisting of signed integer samples
8, 16 or 32 bits wide, stored in Python strings.  This is the same
format as used by the \code{al} and \code{sunaudiodev} modules.  All
scalar items are integers, unless specified otherwise.

A few of the more complicated operations only take 16-bit samples,
otherwise the sample size (in bytes) is always a parameter of the operation.

The module defines the following variables and functions:

\renewcommand{\indexsubitem}{(in module audioop)}
\begin{excdesc}{error}
This exception is raised on all errors, such as unknown number of bytes
per sample, etc.
\end{excdesc}

\begin{funcdesc}{add}{fragment1\, fragment2\, width}
Return a fragment which is the addition of the two samples passed as
parameters.  \var{width} is the sample width in bytes, either
\code{1}, \code{2} or \code{4}.  Both fragments should have the same
length.
\end{funcdesc}

\begin{funcdesc}{adpcm2lin}{adpcmfragment\, width\, state}
Decode an Intel/DVI ADPCM coded fragment to a linear fragment.  See
the description of \code{lin2adpcm} for details on ADPCM coding.
Return a tuple \code{(\var{sample}, \var{newstate})} where the sample
has the width specified in \var{width}.
\end{funcdesc}

\begin{funcdesc}{adpcm32lin}{adpcmfragment\, width\, state}
Decode an alternative 3-bit ADPCM code.  See \code{lin2adpcm3} for
details.
\end{funcdesc}

\begin{funcdesc}{avg}{fragment\, width}
Return the average over all samples in the fragment.
\end{funcdesc}

\begin{funcdesc}{avgpp}{fragment\, width}
Return the average peak-peak value over all samples in the fragment.
No filtering is done, so the usefulness of this routine is
questionable.
\end{funcdesc}

\begin{funcdesc}{bias}{fragment\, width\, bias}
Return a fragment that is the original fragment with a bias added to
each sample.
\end{funcdesc}

\begin{funcdesc}{cross}{fragment\, width}
Return the number of zero crossings in the fragment passed as an
argument.
\end{funcdesc}

\begin{funcdesc}{findfactor}{fragment\, reference}
Return a factor \var{F} such that
\code{rms(add(fragment, mul(reference, -F)))} is minimal, i.e.,
return the factor with which you should multiply \var{reference} to
make it match as well as possible to \var{fragment}.  The fragments
should both contain 2-byte samples.

The time taken by this routine is proportional to \code{len(fragment)}. 
\end{funcdesc}

\begin{funcdesc}{findfit}{fragment\, reference}
This routine (which only accepts 2-byte sample fragments)

Try to match \var{reference} as well as possible to a portion of
\var{fragment} (which should be the longer fragment).  This is
(conceptually) done by taking slices out of \var{fragment}, using
\code{findfactor} to compute the best match, and minimizing the
result.  The fragments should both contain 2-byte samples.  Return a
tuple \code{(\var{offset}, \var{factor})} where \var{offset} is the
(integer) offset into \var{fragment} where the optimal match started
and \var{factor} is the (floating-point) factor as per
\code{findfactor}.
\end{funcdesc}

\begin{funcdesc}{findmax}{fragment\, length}
Search \var{fragment} for a slice of length \var{length} samples (not
bytes!)\ with maximum energy, i.e., return \var{i} for which
\code{rms(fragment[i*2:(i+length)*2])} is maximal.  The fragments
should both contain 2-byte samples.

The routine takes time proportional to \code{len(fragment)}.
\end{funcdesc}

\begin{funcdesc}{getsample}{fragment\, width\, index}
Return the value of sample \var{index} from the fragment.
\end{funcdesc}

\begin{funcdesc}{lin2lin}{fragment\, width\, newwidth}
Convert samples between 1-, 2- and 4-byte formats.
\end{funcdesc}

\begin{funcdesc}{lin2adpcm}{fragment\, width\, state}
Convert samples to 4 bit Intel/DVI ADPCM encoding.  ADPCM coding is an
adaptive coding scheme, whereby each 4 bit number is the difference
between one sample and the next, divided by a (varying) step.  The
Intel/DVI ADPCM algorithm has been selected for use by the IMA, so it
may well become a standard.

\code{State} is a tuple containing the state of the coder.  The coder
returns a tuple \code{(\var{adpcmfrag}, \var{newstate})}, and the
\var{newstate} should be passed to the next call of lin2adpcm.  In the
initial call \code{None} can be passed as the state.  \var{adpcmfrag}
is the ADPCM coded fragment packed 2 4-bit values per byte.
\end{funcdesc}

\begin{funcdesc}{lin2adpcm3}{fragment\, width\, state}
This is an alternative ADPCM coder that uses only 3 bits per sample.
It is not compatible with the Intel/DVI ADPCM coder and its output is
not packed (due to laziness on the side of the author).  Its use is
discouraged.
\end{funcdesc}

\begin{funcdesc}{lin2ulaw}{fragment\, width}
Convert samples in the audio fragment to U-LAW encoding and return
this as a Python string.  U-LAW is an audio encoding format whereby
you get a dynamic range of about 14 bits using only 8 bit samples.  It
is used by the Sun audio hardware, among others.
\end{funcdesc}

\begin{funcdesc}{minmax}{fragment\, width}
Return a tuple consisting of the minimum and maximum values of all
samples in the sound fragment.
\end{funcdesc}

\begin{funcdesc}{max}{fragment\, width}
Return the maximum of the {\em absolute value} of all samples in a
fragment.
\end{funcdesc}

\begin{funcdesc}{maxpp}{fragment\, width}
Return the maximum peak-peak value in the sound fragment.
\end{funcdesc}

\begin{funcdesc}{mul}{fragment\, width\, factor}
Return a fragment that has all samples in the original framgent
multiplied by the floating-point value \var{factor}.  Overflow is
silently ignored.
\end{funcdesc}

\begin{funcdesc}{ratecv}{fragment\, width\, nchannels\, inrate\, outrate\, state\optional{\, weightA\, weightB}}
Convert the frame rate of the input fragment.

\code{State} is a tuple containing the state of the converter.  The
converter returns a tupl \code{(\var{newfragment}, \var{newstate})},
and \var{newstate} should be passed to the next call of ratecv.

The \code{weightA} and \code{weightB} arguments are parameters for a
simple digital filter and default to 1 and 0 respectively.
\end{funcdesc}

\begin{funcdesc}{reverse}{fragment\, width}
Reverse the samples in a fragment and returns the modified fragment.
\end{funcdesc}

\begin{funcdesc}{rms}{fragment\, width}
Return the root-mean-square of the fragment, i.e.
\iftexi
the square root of the quotient of the sum of all squared sample value,
divided by the sumber of samples.
\else
% in eqn: sqrt { sum S sub i sup 2  over n }
\begin{displaymath}
\catcode`_=8
\sqrt{\frac{\sum{{S_{i}}^{2}}}{n}}
\end{displaymath}
\fi
This is a measure of the power in an audio signal.
\end{funcdesc}

\begin{funcdesc}{tomono}{fragment\, width\, lfactor\, rfactor} 
Convert a stereo fragment to a mono fragment.  The left channel is
multiplied by \var{lfactor} and the right channel by \var{rfactor}
before adding the two channels to give a mono signal.
\end{funcdesc}

\begin{funcdesc}{tostereo}{fragment\, width\, lfactor\, rfactor}
Generate a stereo fragment from a mono fragment.  Each pair of samples
in the stereo fragment are computed from the mono sample, whereby left
channel samples are multiplied by \var{lfactor} and right channel
samples by \var{rfactor}.
\end{funcdesc}

\begin{funcdesc}{ulaw2lin}{fragment\, width}
Convert sound fragments in ULAW encoding to linearly encoded sound
fragments.  ULAW encoding always uses 8 bits samples, so \var{width}
refers only to the sample width of the output fragment here.
\end{funcdesc}

Note that operations such as \code{mul} or \code{max} make no
distinction between mono and stereo fragments, i.e.\ all samples are
treated equal.  If this is a problem the stereo fragment should be split
into two mono fragments first and recombined later.  Here is an example
of how to do that:
\bcode\begin{verbatim}
def mul_stereo(sample, width, lfactor, rfactor):
    lsample = audioop.tomono(sample, width, 1, 0)
    rsample = audioop.tomono(sample, width, 0, 1)
    lsample = audioop.mul(sample, width, lfactor)
    rsample = audioop.mul(sample, width, rfactor)
    lsample = audioop.tostereo(lsample, width, 1, 0)
    rsample = audioop.tostereo(rsample, width, 0, 1)
    return audioop.add(lsample, rsample, width)
\end{verbatim}\ecode
%
If you use the ADPCM coder to build network packets and you want your
protocol to be stateless (i.e.\ to be able to tolerate packet loss)
you should not only transmit the data but also the state.  Note that
you should send the \var{initial} state (the one you passed to
\code{lin2adpcm}) along to the decoder, not the final state (as returned by
the coder).  If you want to use \code{struct} to store the state in
binary you can code the first element (the predicted value) in 16 bits
and the second (the delta index) in 8.

The ADPCM coders have never been tried against other ADPCM coders,
only against themselves.  It could well be that I misinterpreted the
standards in which case they will not be interoperable with the
respective standards.

The \code{find...} routines might look a bit funny at first sight.
They are primarily meant to do echo cancellation.  A reasonably
fast way to do this is to pick the most energetic piece of the output
sample, locate that in the input sample and subtract the whole output
sample from the input sample:
\bcode\begin{verbatim}
def echocancel(outputdata, inputdata):
    pos = audioop.findmax(outputdata, 800)    # one tenth second
    out_test = outputdata[pos*2:]
    in_test = inputdata[pos*2:]
    ipos, factor = audioop.findfit(in_test, out_test)
    # Optional (for better cancellation):
    # factor = audioop.findfactor(in_test[ipos*2:ipos*2+len(out_test)], 
    #              out_test)
    prefill = '\0'*(pos+ipos)*2
    postfill = '\0'*(len(inputdata)-len(prefill)-len(outputdata))
    outputdata = prefill + audioop.mul(outputdata,2,-factor) + postfill
    return audioop.add(inputdata, outputdata, 2)
\end{verbatim}\ecode

\section{\module{imageop} ---
         Manipulate raw image data}

\declaremodule{builtin}{imageop}
\modulesynopsis{Manipulate raw image data.}


The \module{imageop} module contains some useful operations on images.
It operates on images consisting of 8 or 32 bit pixels stored in
Python strings.  This is the same format as used by
\function{gl.lrectwrite()} and the \refmodule{imgfile} module.

The module defines the following variables and functions:

\begin{excdesc}{error}
This exception is raised on all errors, such as unknown number of bits
per pixel, etc.
\end{excdesc}


\begin{funcdesc}{crop}{image, psize, width, height, x0, y0, x1, y1}
Return the selected part of \var{image}, which should by
\var{width} by \var{height} in size and consist of pixels of
\var{psize} bytes. \var{x0}, \var{y0}, \var{x1} and \var{y1} are like
the \function{gl.lrectread()} parameters, i.e.\ the boundary is
included in the new image.  The new boundaries need not be inside the
picture.  Pixels that fall outside the old image will have their value
set to zero.  If \var{x0} is bigger than \var{x1} the new image is
mirrored.  The same holds for the y coordinates.
\end{funcdesc}

\begin{funcdesc}{scale}{image, psize, width, height, newwidth, newheight}
Return \var{image} scaled to size \var{newwidth} by \var{newheight}.
No interpolation is done, scaling is done by simple-minded pixel
duplication or removal.  Therefore, computer-generated images or
dithered images will not look nice after scaling.
\end{funcdesc}

\begin{funcdesc}{tovideo}{image, psize, width, height}
Run a vertical low-pass filter over an image.  It does so by computing
each destination pixel as the average of two vertically-aligned source
pixels.  The main use of this routine is to forestall excessive
flicker if the image is displayed on a video device that uses
interlacing, hence the name.
\end{funcdesc}

\begin{funcdesc}{grey2mono}{image, width, height, threshold}
Convert a 8-bit deep greyscale image to a 1-bit deep image by
thresholding all the pixels.  The resulting image is tightly packed and
is probably only useful as an argument to \function{mono2grey()}.
\end{funcdesc}

\begin{funcdesc}{dither2mono}{image, width, height}
Convert an 8-bit greyscale image to a 1-bit monochrome image using a
(simple-minded) dithering algorithm.
\end{funcdesc}

\begin{funcdesc}{mono2grey}{image, width, height, p0, p1}
Convert a 1-bit monochrome image to an 8 bit greyscale or color image.
All pixels that are zero-valued on input get value \var{p0} on output
and all one-value input pixels get value \var{p1} on output.  To
convert a monochrome black-and-white image to greyscale pass the
values \code{0} and \code{255} respectively.
\end{funcdesc}

\begin{funcdesc}{grey2grey4}{image, width, height}
Convert an 8-bit greyscale image to a 4-bit greyscale image without
dithering.
\end{funcdesc}

\begin{funcdesc}{grey2grey2}{image, width, height}
Convert an 8-bit greyscale image to a 2-bit greyscale image without
dithering.
\end{funcdesc}

\begin{funcdesc}{dither2grey2}{image, width, height}
Convert an 8-bit greyscale image to a 2-bit greyscale image with
dithering.  As for \function{dither2mono()}, the dithering algorithm
is currently very simple.
\end{funcdesc}

\begin{funcdesc}{grey42grey}{image, width, height}
Convert a 4-bit greyscale image to an 8-bit greyscale image.
\end{funcdesc}

\begin{funcdesc}{grey22grey}{image, width, height}
Convert a 2-bit greyscale image to an 8-bit greyscale image.
\end{funcdesc}

\section{Standard Module \sectcode{aifc}}
\label{module-aifc}
\stmodindex{aifc}

This module provides support for reading and writing AIFF and AIFF-C
files.  AIFF is Audio Interchange File Format, a format for storing
digital audio samples in a file.  AIFF-C is a newer version of the
format that includes the ability to compress the audio data.

Audio files have a number of parameters that describe the audio data.
The sampling rate or frame rate is the number of times per second the
sound is sampled.  The number of channels indicate if the audio is
mono, stereo, or quadro.  Each frame consists of one sample per
channel.  The sample size is the size in bytes of each sample.  Thus a
frame consists of \var{nchannels}*\var{samplesize} bytes, and a
second's worth of audio consists of
\var{nchannels}*\var{samplesize}*\var{framerate} bytes.

For example, CD quality audio has a sample size of two bytes (16
bits), uses two channels (stereo) and has a frame rate of 44,100
frames/second.  This gives a frame size of 4 bytes (2*2), and a
second's worth occupies 2*2*44100 bytes, i.e.\ 176,400 bytes.

Module \code{aifc} defines the following function:

\setindexsubitem{(in module aifc)}
\begin{funcdesc}{open}{file, mode}
Open an AIFF or AIFF-C file and return an object instance with
methods that are described below.  The argument file is either a
string naming a file or a file object.  The mode is either the string
\code{'r'} when the file must be opened for reading, or \code{'w'}
when the file must be opened for writing.  When used for writing, the
file object should be seekable, unless you know ahead of time how many
samples you are going to write in total and use
\code{writeframesraw()} and \code{setnframes()}.
\end{funcdesc}

Objects returned by \code{aifc.open()} when a file is opened for
reading have the following methods:

\setindexsubitem{(aifc object method)}
\begin{funcdesc}{getnchannels}{}
Return the number of audio channels (1 for mono, 2 for stereo).
\end{funcdesc}

\begin{funcdesc}{getsampwidth}{}
Return the size in bytes of individual samples.
\end{funcdesc}

\begin{funcdesc}{getframerate}{}
Return the sampling rate (number of audio frames per second).
\end{funcdesc}

\begin{funcdesc}{getnframes}{}
Return the number of audio frames in the file.
\end{funcdesc}

\begin{funcdesc}{getcomptype}{}
Return a four-character string describing the type of compression used
in the audio file.  For AIFF files, the returned value is
\code{'NONE'}.
\end{funcdesc}

\begin{funcdesc}{getcompname}{}
Return a human-readable description of the type of compression used in
the audio file.  For AIFF files, the returned value is \code{'not
compressed'}.
\end{funcdesc}

\begin{funcdesc}{getparams}{}
Return a tuple consisting of all of the above values in the above
order.
\end{funcdesc}

\begin{funcdesc}{getmarkers}{}
Return a list of markers in the audio file.  A marker consists of a
tuple of three elements.  The first is the mark ID (an integer), the
second is the mark position in frames from the beginning of the data
(an integer), the third is the name of the mark (a string).
\end{funcdesc}

\begin{funcdesc}{getmark}{id}
Return the tuple as described in \code{getmarkers} for the mark with
the given id.
\end{funcdesc}

\begin{funcdesc}{readframes}{nframes}
Read and return the next \var{nframes} frames from the audio file.  The
returned data is a string containing for each frame the uncompressed
samples of all channels.
\end{funcdesc}

\begin{funcdesc}{rewind}{}
Rewind the read pointer.  The next \code{readframes} will start from
the beginning.
\end{funcdesc}

\begin{funcdesc}{setpos}{pos}
Seek to the specified frame number.
\end{funcdesc}

\begin{funcdesc}{tell}{}
Return the current frame number.
\end{funcdesc}

\begin{funcdesc}{close}{}
Close the AIFF file.  After calling this method, the object can no
longer be used.
\end{funcdesc}

Objects returned by \code{aifc.open()} when a file is opened for
writing have all the above methods, except for \code{readframes} and
\code{setpos}.  In addition the following methods exist.  The
\code{get} methods can only be called after the corresponding
\code{set} methods have been called.  Before the first
\code{writeframes()} or \code{writeframesraw()}, all parameters except
for the number of frames must be filled in.

\begin{funcdesc}{aiff}{}
Create an AIFF file.  The default is that an AIFF-C file is created,
unless the name of the file ends in \code{'.aiff'} in which case the
default is an AIFF file.
\end{funcdesc}

\begin{funcdesc}{aifc}{}
Create an AIFF-C file.  The default is that an AIFF-C file is created,
unless the name of the file ends in \code{'.aiff'} in which case the
default is an AIFF file.
\end{funcdesc}

\begin{funcdesc}{setnchannels}{nchannels}
Specify the number of channels in the audio file.
\end{funcdesc}

\begin{funcdesc}{setsampwidth}{width}
Specify the size in bytes of audio samples.
\end{funcdesc}

\begin{funcdesc}{setframerate}{rate}
Specify the sampling frequency in frames per second.
\end{funcdesc}

\begin{funcdesc}{setnframes}{nframes}
Specify the number of frames that are to be written to the audio file.
If this parameter is not set, or not set correctly, the file needs to
support seeking.
\end{funcdesc}

\begin{funcdesc}{setcomptype}{type, name}
Specify the compression type.  If not specified, the audio data will
not be compressed.  In AIFF files, compression is not possible.  The
name parameter should be a human-readable description of the
compression type, the type parameter should be a four-character
string.  Currently the following compression types are supported:
NONE, ULAW, ALAW, G722.
\end{funcdesc}

\begin{funcdesc}{setparams}{nchannels, sampwidth, framerate, comptype, compname}
Set all the above parameters at once.  The argument is a tuple
consisting of the various parameters.  This means that it is possible
to use the result of a \code{getparams()} call as argument to
\code{setparams()}.
\end{funcdesc}

\begin{funcdesc}{setmark}{id, pos, name}
Add a mark with the given id (larger than 0), and the given name at
the given position.  This method can be called at any time before
\code{close()}.
\end{funcdesc}

\begin{funcdesc}{tell}{}
Return the current write position in the output file.  Useful in
combination with \code{setmark()}.
\end{funcdesc}

\begin{funcdesc}{writeframes}{data}
Write data to the output file.  This method can only be called after
the audio file parameters have been set.
\end{funcdesc}

\begin{funcdesc}{writeframesraw}{data}
Like \code{writeframes()}, except that the header of the audio file is
not updated.
\end{funcdesc}

\begin{funcdesc}{close}{}
Close the AIFF file.  The header of the file is updated to reflect the
actual size of the audio data. After calling this method, the object
can no longer be used.
\end{funcdesc}

\section{Built-in Module \module{jpeg}}
\label{module-jpeg}
\bimodindex{jpeg}

The module \module{jpeg} provides access to the jpeg compressor and
decompressor written by the Independent JPEG Group%
\index{Independent JPEG Group}%
. JPEG is a (draft?)
standard for compressing pictures.  For details on JPEG or the
Independent JPEG Group software refer to the JPEG standard or the
documentation provided with the software.

The \module{jpeg} module defines an exception and some functions.

\begin{excdesc}{error}
Exception raised by \function{compress()} and \function{decompress()}
in case of errors.
\end{excdesc}

\begin{funcdesc}{compress}{data, w, h, b}
Treat data as a pixmap of width \var{w} and height \var{h}, with
\var{b} bytes per pixel.  The data is in SGI GL order, so the first
pixel is in the lower-left corner. This means that \function{gl.lrectread()}
return data can immediately be passed to \function{compress()}.
Currently only 1 byte and 4 byte pixels are allowed, the former being
treated as greyscale and the latter as RGB color.
\function{compress()} returns a string that contains the compressed
picture, in JFIF\index{JFIF} format.
\end{funcdesc}

\begin{funcdesc}{decompress}{data}
Data is a string containing a picture in JFIF\index{JFIF} format. It
returns a tuple \code{(\var{data}, \var{width}, \var{height},
\var{bytesperpixel})}.  Again, the data is suitable to pass to
\function{gl.lrectwrite()}.
\end{funcdesc}

\begin{funcdesc}{setoption}{name, value}
Set various options.  Subsequent \function{compress()} and
\function{decompress()} calls will use these options.  The following
options are available:

\begin{tableii}{l|p{3in}}{code}{Option}{Effect}
  \lineii{'forcegray'}{%
    Force output to be grayscale, even if input is RGB.}
  \lineii{'quality'}{%
    Set the quality of the compressed image to a value between
    \code{0} and \code{100} (default is \code{75}).  This only affects
    compression.}
  \lineii{'optimize'}{%
    Perform Huffman table optimization.  Takes longer, but results in
    smaller compressed image.  This only affects compression.}
  \lineii{'smooth'}{%
    Perform inter-block smoothing on uncompressed image.  Only useful
    for low-quality images.  This only affects decompression.}
\end{tableii}
\end{funcdesc}

\section{\module{rgbimg} ---
         Read and write image files in ``SGI RGB'' format.}
\declaremodule{builtin}{rgbimg}

\modulesynopsis{Read and write image files in ``SGI RGB'' format (the module is
\emph{not} SGI specific though!).}


The \module{rgbimg} module allows Python programs to access SGI imglib image
files (also known as \file{.rgb} files).  The module is far from
complete, but is provided anyway since the functionality that there is
enough in some cases.  Currently, colormap files are not supported.

The module defines the following variables and functions:

\begin{excdesc}{error}
This exception is raised on all errors, such as unsupported file type, etc.
\end{excdesc}

\begin{funcdesc}{sizeofimage}{file}
This function returns a tuple \code{(\var{x}, \var{y})} where
\var{x} and \var{y} are the size of the image in pixels.
Only 4 byte RGBA pixels, 3 byte RGB pixels, and 1 byte greyscale pixels
are currently supported.
\end{funcdesc}

\begin{funcdesc}{longimagedata}{file}
This function reads and decodes the image on the specified file, and
returns it as a Python string. The string has 4 byte RGBA pixels.
The bottom left pixel is the first in
the string. This format is suitable to pass to \function{gl.lrectwrite()},
for instance.
\end{funcdesc}

\begin{funcdesc}{longstoimage}{data, x, y, z, file}
This function writes the RGBA data in \var{data} to image
file \var{file}. \var{x} and \var{y} give the size of the image.
\var{z} is 1 if the saved image should be 1 byte greyscale, 3 if the
saved image should be 3 byte RGB data, or 4 if the saved images should
be 4 byte RGBA data.  The input data always contains 4 bytes per pixel.
These are the formats returned by \function{gl.lrectread()}.
\end{funcdesc}

\begin{funcdesc}{ttob}{flag}
This function sets a global flag which defines whether the scan lines
of the image are read or written from bottom to top (flag is zero,
compatible with SGI GL) or from top to bottom(flag is one,
compatible with X).  The default is zero.
\end{funcdesc}

\section{Standard module \sectcode{imghdr}}
\label{module-imghdr}
\stmodindex{imghdr}

The \code{imghdr} module determines the type of image contained in a
file or byte stream.

The \code{imghdr} module defines the following function:

\renewcommand{\indexsubitem}{(in module imghdr)}

\begin{funcdesc}{what}{filename\optional{\, h}}
Tests the image data contained in the file named by \var{filename},
and returns a string describing the image type.  If optional \var{h}
is provided, the \var{filename} is ignored and \var{h} is assumed to
contain the byte stream to test.
\end{funcdesc}

The following image types are recognized, as listed below with the
return value from \code{what}:

\begin{enumerate}
\item[``rgb''] SGI ImgLib Files

\item[``gif''] GIF 87a and 89a Files

\item[``pbm''] Portable Bitmap Files

\item[``pgm''] Portable Graymap Files

\item[``ppm''] Portable Pixmap Files

\item[``tiff''] TIFF Files

\item[``rast''] Sun Raster Files

\item[``xbm''] X Bitmap Files

\item[``jpeg''] JPEG data in JIFF format
\end{enumerate}

You can extend the list of file types \code{imghdr} can recognize by
appending to this variable:

\begin{datadesc}{tests}
A list of functions performing the individual tests.  Each function
takes two arguments: the byte-stream and an open file-like object.
When \code{what()} is called with a byte-stream, the file-like
object will be \code{None}.

The test function should return a string describing the image type if
the test succeeded, or \code{None} if it failed.
\end{datadesc}

Example:

\bcode\begin{verbatim}
>>> import imghdr
>>> imghdr.what('/tmp/bass.gif')
'gif'
\end{verbatim}\ecode


\chapter{Cryptographic Services}
\label{crypto}
\index{cryptography}

The modules described in this chapter implement various algorithms of
a cryptographic nature.  They are available at the discretion of the
installation.  Here's an overview:

\localmoduletable

Hardcore cypherpunks will probably find the cryptographic modules
written by Andrew Kuchling of further interest; the package adds
built-in modules for DES and IDEA encryption, provides a Python module
for reading and decrypting PGP files, and then some.  These modules
are not distributed with Python but available separately.  See the URL
\url{http://starship.python.net/crew/amk/python/crypto.html} or
send email to \email{amk1@bigfoot.com} for more information.
\index{PGP}
\index{Pretty Good Privacy}
\indexii{DES}{cipher}
\indexii{IDEA}{cipher}
\index{cryptography}
\index{Kuchling, Andrew}
		% Cryptographic Services
\section{Built-in Module \sectcode{md5}}
\bimodindex{md5}

This module implements the interface to RSA's MD5 message digest
algorithm (see also Internet RFC 1321).  Its use is quite
straightforward:\ use the \code{md5.new()} to create an md5 object.
You can now feed this object with arbitrary strings using the
\code{update()} method, and at any point you can ask it for the
\dfn{digest} (a strong kind of 128-bit checksum,
a.k.a. ``fingerprint'') of the contatenation of the strings fed to it
so far using the \code{digest()} method.

For example, to obtain the digest of the string {\tt"Nobody inspects
the spammish repetition"}:

\bcode\begin{verbatim}
>>> import md5
>>> m = md5.new()
>>> m.update("Nobody inspects")
>>> m.update(" the spammish repetition")
>>> m.digest()
'\273d\234\203\335\036\245\311\331\336\311\241\215\360\377\351'
\end{verbatim}\ecode

More condensed:

\bcode\begin{verbatim}
>>> md5.new("Nobody inspects the spammish repetition").digest()
'\273d\234\203\335\036\245\311\331\336\311\241\215\360\377\351'
\end{verbatim}\ecode

\renewcommand{\indexsubitem}{(in module md5)}

\begin{funcdesc}{new}{\optional{arg}}
Return a new md5 object.  If \var{arg} is present, the method call
\code{update(\var{arg})} is made.
\end{funcdesc}

\begin{funcdesc}{md5}{\optional{arg}}
For backward compatibility reasons, this is an alternative name for the
\code{new()} function.
\end{funcdesc}

An md5 object has the following methods:

\renewcommand{\indexsubitem}{(md5 method)}
\begin{funcdesc}{update}{arg}
Update the md5 object with the string \var{arg}.  Repeated calls are
equivalent to a single call with the concatenation of all the
arguments, i.e.\ \code{m.update(a); m.update(b)} is equivalent to
\code{m.update(a+b)}.
\end{funcdesc}

\begin{funcdesc}{digest}{}
Return the digest of the strings passed to the \code{update()}
method so far.  This is an 16-byte string which may contain
non-\ASCII{} characters, including null bytes.
\end{funcdesc}

\begin{funcdesc}{copy}{}
Return a copy (``clone'') of the md5 object.  This can be used to
efficiently compute the digests of strings that share a common initial
substring.
\end{funcdesc}

\section{Built-in Module \sectcode{mpz}}
\label{module-mpz}
\bimodindex{mpz}

This is an optional module.  It is only available when Python is
configured to include it, which requires that the GNU MP software is
installed.

This module implements the interface to part of the GNU MP library,
which defines arbitrary precision integer and rational number
arithmetic routines.  Only the interfaces to the \emph{integer}
(\samp{mpz_{\rm \ldots}}) routines are provided. If not stated
otherwise, the description in the GNU MP documentation can be applied.

In general, \dfn{mpz}-numbers can be used just like other standard
Python numbers, e.g.\ you can use the built-in operators like \code{+},
\code{*}, etc., as well as the standard built-in functions like
\code{abs}, \code{int}, \ldots, \code{divmod}, \code{pow}.
\strong{Please note:} the \emph{bitwise-xor} operation has been implemented as
a bunch of \emph{and}s, \emph{invert}s and \emph{or}s, because the library
lacks an \code{mpz_xor} function, and I didn't need one.

You create an mpz-number by calling the function called \code{mpz} (see
below for an exact description). An mpz-number is printed like this:
\code{mpz(\var{value})}.

\setindexsubitem{(in module mpz)}
\begin{funcdesc}{mpz}{value}
  Create a new mpz-number. \var{value} can be an integer, a long,
  another mpz-number, or even a string. If it is a string, it is
  interpreted as an array of radix-256 digits, least significant digit
  first, resulting in a positive number. See also the \code{binary}
  method, described below.
\end{funcdesc}

A number of \emph{extra} functions are defined in this module. Non
mpz-arguments are converted to mpz-values first, and the functions
return mpz-numbers.

\begin{funcdesc}{powm}{base, exponent, modulus}
  Return \code{pow(\var{base}, \var{exponent}) \%{} \var{modulus}}. If
  \code{\var{exponent} == 0}, return \code{mpz(1)}. In contrast to the
  \C-library function, this version can handle negative exponents.
\end{funcdesc}

\begin{funcdesc}{gcd}{op1, op2}
  Return the greatest common divisor of \var{op1} and \var{op2}.
\end{funcdesc}

\begin{funcdesc}{gcdext}{a, b}
  Return a tuple \code{(\var{g}, \var{s}, \var{t})}, such that
  \code{\var{a}*\var{s} + \var{b}*\var{t} == \var{g} == gcd(\var{a}, \var{b})}.
\end{funcdesc}

\begin{funcdesc}{sqrt}{op}
  Return the square root of \var{op}. The result is rounded towards zero.
\end{funcdesc}

\begin{funcdesc}{sqrtrem}{op}
  Return a tuple \code{(\var{root}, \var{remainder})}, such that
  \code{\var{root}*\var{root} + \var{remainder} == \var{op}}.
\end{funcdesc}

\begin{funcdesc}{divm}{numerator, denominator, modulus}
  Returns a number \var{q}. such that
  \code{\var{q} * \var{denominator} \%{} \var{modulus} == \var{numerator}}.
  One could also implement this function in Python, using \code{gcdext}.
\end{funcdesc}

An mpz-number has one method:

\setindexsubitem{(mpz method)}
\begin{funcdesc}{binary}{}
  Convert this mpz-number to a binary string, where the number has been
  stored as an array of radix-256 digits, least significant digit first.

  The mpz-number must have a value greater than or equal to zero,
  otherwise a \code{ValueError}-exception will be raised.
\end{funcdesc}

\section{Built-in Module \sectcode{rotor}}
\bimodindex{rotor}

This module implements a rotor-based encryption algorithm, contributed by
Lance Ellinghouse.  The design is derived from the Enigma device, a machine
used during World War II to encipher messages.  A rotor is simply a
permutation.  For example, if the character `A' is the origin of the rotor,
then a given rotor might map `A' to `L', `B' to `Z', `C' to `G', and so on.
To encrypt, we choose several different rotors, and set the origins of the
rotors to known positions; their initial position is the ciphering key.  To
encipher a character, we permute the original character by the first rotor,
and then apply the second rotor's permutation to the result. We continue
until we've applied all the rotors; the resulting character is our
ciphertext.  We then change the origin of the final rotor by one position,
from `A' to `B'; if the final rotor has made a complete revolution, then we
rotate the next-to-last rotor by one position, and apply the same procedure
recursively.  In other words, after enciphering one character, we advance
the rotors in the same fashion as a car's odometer. Decoding works in the
same way, except we reverse the permutations and apply them in the opposite
order.
\index{Ellinghouse, Lance}
\indexii{Enigma}{cipher}

The available functions in this module are:

\renewcommand{\indexsubitem}{(in module rotor)}
\begin{funcdesc}{newrotor}{key\optional{\, numrotors}}
Return a rotor object. \var{key} is a string containing the encryption key
for the object; it can contain arbitrary binary data. The key will be used
to randomly generate the rotor permutations and their initial positions.
\var{numrotors} is the number of rotor permutations in the returned object;
if it is omitted, a default value of 6 will be used.
\end{funcdesc}

Rotor objects have the following methods:

\renewcommand{\indexsubitem}{(rotor method)}
\begin{funcdesc}{setkey}{key}
Sets the rotor's key to \var{key}.
\end{funcdesc}

\begin{funcdesc}{encrypt}{plaintext}
Reset the rotor object to its initial state and encrypt \var{plaintext},
returning a string containing the ciphertext.  The ciphertext is always the
same length as the original plaintext.
\end{funcdesc}

\begin{funcdesc}{encryptmore}{plaintext}
Encrypt \var{plaintext} without resetting the rotor object, and return a
string containing the ciphertext.
\end{funcdesc}

\begin{funcdesc}{decrypt}{ciphertext}
Reset the rotor object to its initial state and decrypt \var{ciphertext},
returning a string containing the ciphertext.  The plaintext string will
always be the same length as the ciphertext.
\end{funcdesc}

\begin{funcdesc}{decryptmore}{ciphertext}
Decrypt \var{ciphertext} without resetting the rotor object, and return a
string containing the ciphertext.
\end{funcdesc}

An example usage:
\bcode\begin{verbatim}
>>> import rotor
>>> rt = rotor.newrotor('key', 12)
>>> rt.encrypt('bar')
'\2534\363'
>>> rt.encryptmore('bar')
'\357\375$'
>>> rt.encrypt('bar')
'\2534\363'
>>> rt.decrypt('\2534\363')
'bar'
>>> rt.decryptmore('\357\375$')
'bar'
>>> rt.decrypt('\357\375$')
'l(\315'
>>> del rt
\end{verbatim}\ecode

The module's code is not an exact simulation of the original Enigma device;
it implements the rotor encryption scheme differently from the original. The
most important difference is that in the original Enigma, there were only 5
or 6 different rotors in existence, and they were applied twice to each
character; the cipher key was the order in which they were placed in the
machine.  The Python rotor module uses the supplied key to initialize a
random number generator; the rotor permutations and their initial positions
are then randomly generated.  The original device only enciphered the
letters of the alphabet, while this module can handle any 8-bit binary data;
it also produces binary output.  This module can also operate with an
arbitrary number of rotors.

The original Enigma cipher was broken in 1944. % XXX: Is this right?
The version implemented here is probably a good deal more difficult to crack
(especially if you use many rotors), but it won't be impossible for
a truly skilful and determined attacker to break the cipher.  So if you want
to keep the NSA out of your files, this rotor cipher may well be unsafe, but
for discouraging casual snooping through your files, it will probably be
just fine, and may be somewhat safer than using the Unix \file{crypt}
command.
\index{National Security Agency}\index{crypt(1)}
% XXX How were Unix commands represented in the docs?



%\chapter{Amoeba Specific Services}

\section{Built-in Module \sectcode{amoeba}}

\bimodindex{amoeba}
This module provides some object types and operations useful for
Amoeba applications.  It is only available on systems that support
Amoeba operations.  RPC errors and other Amoeba errors are reported as
the exception \code{amoeba.error = 'amoeba.error'}.

The module \code{amoeba} defines the following items:

\renewcommand{\indexsubitem}{(in module amoeba)}
\begin{funcdesc}{name_append}{path\, cap}
Stores a capability in the Amoeba directory tree.
Arguments are the pathname (a string) and the capability (a capability
object as returned by
\code{name_lookup()}).
\end{funcdesc}

\begin{funcdesc}{name_delete}{path}
Deletes a capability from the Amoeba directory tree.
Argument is the pathname.
\end{funcdesc}

\begin{funcdesc}{name_lookup}{path}
Looks up a capability.
Argument is the pathname.
Returns a
\dfn{capability}
object, to which various interesting operations apply, described below.
\end{funcdesc}

\begin{funcdesc}{name_replace}{path\, cap}
Replaces a capability in the Amoeba directory tree.
Arguments are the pathname and the new capability.
(This differs from
\code{name_append()}
in the behavior when the pathname already exists:
\code{name_append()}
finds this an error while
\code{name_replace()}
allows it, as its name suggests.)
\end{funcdesc}

\begin{datadesc}{capv}
A table representing the capability environment at the time the
interpreter was started.
(Alas, modifying this table does not affect the capability environment
of the interpreter.)
For example,
\code{amoeba.capv['ROOT']}
is the capability of your root directory, similar to
\code{getcap("ROOT")}
in C.
\end{datadesc}

\begin{excdesc}{error}
The exception raised when an Amoeba function returns an error.
The value accompanying this exception is a pair containing the numeric
error code and the corresponding string, as returned by the C function
\code{err_why()}.
\end{excdesc}

\begin{funcdesc}{timeout}{msecs}
Sets the transaction timeout, in milliseconds.
Returns the previous timeout.
Initially, the timeout is set to 2 seconds by the Python interpreter.
\end{funcdesc}

\subsection{Capability Operations}

Capabilities are written in a convenient \ASCII{} format, also used by the
Amoeba utilities
{\it c2a}(U)
and
{\it a2c}(U).
For example:

\bcode\begin{verbatim}
>>> amoeba.name_lookup('/profile/cap')
aa:1c:95:52:6a:fa/14(ff)/8e:ba:5b:8:11:1a
>>> 
\end{verbatim}\ecode
%
The following methods are defined for capability objects.

\renewcommand{\indexsubitem}{(capability method)}
\begin{funcdesc}{dir_list}{}
Returns a list of the names of the entries in an Amoeba directory.
\end{funcdesc}

\begin{funcdesc}{b_read}{offset\, maxsize}
Reads (at most)
\var{maxsize}
bytes from a bullet file at offset
\var{offset.}
The data is returned as a string.
EOF is reported as an empty string.
\end{funcdesc}

\begin{funcdesc}{b_size}{}
Returns the size of a bullet file.
\end{funcdesc}

\begin{funcdesc}{dir_append}{}
\funcline{dir_delete}{}\ 
\funcline{dir_lookup}{}\ 
\funcline{dir_replace}{}
Like the corresponding
\samp{name_}*
functions, but with a path relative to the capability.
(For paths beginning with a slash the capability is ignored, since this
is the defined semantics for Amoeba.)
\end{funcdesc}

\begin{funcdesc}{std_info}{}
Returns the standard info string of the object.
\end{funcdesc}

\begin{funcdesc}{tod_gettime}{}
Returns the time (in seconds since the Epoch, in UCT, as for POSIX) from
a time server.
\end{funcdesc}

\begin{funcdesc}{tod_settime}{t}
Sets the time kept by a time server.
\end{funcdesc}
		% AMOEBA ONLY

%\chapter{Standard Windowing Interface}

The modules in this chapter are available only on those systems where
the STDWIN library is available.  STDWIN runs on \UNIX{} under X11 and
on the Macintosh.  See CWI report CS-R8817.

\strong{Warning:} Using STDWIN is not recommended for new
applications.  It has never been ported to Microsoft Windows or
Windows NT, and for X11 or the Macintosh it lacks important
functionality --- in particular, it has no tools for the construction
of dialogs.  For most platforms, alternative, native solutions exist
(though none are currently documented in this manual): Tkinter for
\UNIX{} under X11, native Xt with Motif or Athena widgets for \UNIX{}
under X11, Win32 for Windows and Windows NT, and a collection of
native toolkit interfaces for the Macintosh.

\section{\module{stdwin} ---
         Platform-independent GUI System}
\declaremodule{builtin}{stdwin}

\modulesynopsis{Older GUI system for X11 and Macintosh}


This module defines several new object types and functions that
provide access to the functionality of STDWIN.

On \UNIX{} running X11, it can only be used if the \envvar{DISPLAY}
environment variable is set or an explicit \samp{-display
\var{displayname}} argument is passed to the Python interpreter.

Functions have names that usually resemble their C STDWIN counterparts
with the initial `w' dropped.
Points are represented by pairs of integers; rectangles
by pairs of points.
For a complete description of STDWIN please refer to the documentation
of STDWIN for C programmers (aforementioned CWI report).

\subsection{Functions Defined in Module \module{stdwin}}
\nodename{STDWIN Functions}

The following functions are defined in the \module{stdwin} module:

\begin{funcdesc}{open}{title}
Open a new window whose initial title is given by the string argument.
Return a window object; window object methods are described below.%
\footnote{The Python version of STDWIN does not support draw procedures; all
	drawing requests are reported as draw events.}
\end{funcdesc}

\begin{funcdesc}{getevent}{}
Wait for and return the next event.
An event is returned as a triple: the first element is the event
type, a small integer; the second element is the window object to which
the event applies, or
\code{None}
if it applies to no window in particular;
the third element is type-dependent.
Names for event types and command codes are defined in the standard
module \module{stdwinevent}.
\end{funcdesc}

\begin{funcdesc}{pollevent}{}
Return the next event, if one is immediately available.
If no event is available, return \code{()}.
\end{funcdesc}

\begin{funcdesc}{getactive}{}
Return the window that is currently active, or \code{None} if no
window is currently active.  (This can be emulated by monitoring
WE_ACTIVATE and WE_DEACTIVATE events.)
\end{funcdesc}

\begin{funcdesc}{listfontnames}{pattern}
Return the list of font names in the system that match the pattern (a
string).  The pattern should normally be \code{'*'}; returns all
available fonts.  If the underlying window system is X11, other
patterns follow the standard X11 font selection syntax (as used e.g.
in resource definitions), i.e. the wildcard character \code{'*'}
matches any sequence of characters (including none) and \code{'?'}
matches any single character.
On the Macintosh this function currently returns an empty list.
\end{funcdesc}

\begin{funcdesc}{setdefscrollbars}{hflag, vflag}
Set the flags controlling whether subsequently opened windows will
have horizontal and/or vertical scroll bars.
\end{funcdesc}

\begin{funcdesc}{setdefwinpos}{h, v}
Set the default window position for windows opened subsequently.
\end{funcdesc}

\begin{funcdesc}{setdefwinsize}{width, height}
Set the default window size for windows opened subsequently.
\end{funcdesc}

\begin{funcdesc}{getdefscrollbars}{}
Return the flags controlling whether subsequently opened windows will
have horizontal and/or vertical scroll bars.
\end{funcdesc}

\begin{funcdesc}{getdefwinpos}{}
Return the default window position for windows opened subsequently.
\end{funcdesc}

\begin{funcdesc}{getdefwinsize}{}
Return the default window size for windows opened subsequently.
\end{funcdesc}

\begin{funcdesc}{getscrsize}{}
Return the screen size in pixels.
\end{funcdesc}

\begin{funcdesc}{getscrmm}{}
Return the screen size in millimeters.
\end{funcdesc}

\begin{funcdesc}{fetchcolor}{colorname}
Return the pixel value corresponding to the given color name.
Return the default foreground color for unknown color names.
Hint: the following code tests whether you are on a machine that
supports more than two colors:
\begin{verbatim}
if stdwin.fetchcolor('black') <> \
          stdwin.fetchcolor('red') <> \
          stdwin.fetchcolor('white'):
    print 'color machine'
else:
    print 'monochrome machine'
\end{verbatim}
\end{funcdesc}

\begin{funcdesc}{setfgcolor}{pixel}
Set the default foreground color.
This will become the default foreground color of windows opened
subsequently, including dialogs.
\end{funcdesc}

\begin{funcdesc}{setbgcolor}{pixel}
Set the default background color.
This will become the default background color of windows opened
subsequently, including dialogs.
\end{funcdesc}

\begin{funcdesc}{getfgcolor}{}
Return the pixel value of the current default foreground color.
\end{funcdesc}

\begin{funcdesc}{getbgcolor}{}
Return the pixel value of the current default background color.
\end{funcdesc}

\begin{funcdesc}{setfont}{fontname}
Set the current default font.
This will become the default font for windows opened subsequently,
and is also used by the text measuring functions \function{textwidth()},
\function{textbreak()}, \function{lineheight()} and
\function{baseline()} below.  This accepts two more optional
parameters, size and style:  Size is the font size (in `points').
Style is a single character specifying the style, as follows:
\code{'b'} = bold,
\code{'i'} = italic,
\code{'o'} = bold + italic,
\code{'u'} = underline;
default style is roman.
Size and style are ignored under X11 but used on the Macintosh.
(Sorry for all this complexity --- a more uniform interface is being designed.)
\end{funcdesc}

\begin{funcdesc}{menucreate}{title}
Create a menu object referring to a global menu (a menu that appears in
all windows).
Methods of menu objects are described below.
Note: normally, menus are created locally; see the window method
\method{menucreate()} below.
\strong{Warning:} the menu only appears in a window as long as the object
returned by this call exists.
\end{funcdesc}

\begin{funcdesc}{newbitmap}{width, height}
Create a new bitmap object of the given dimensions.
Methods of bitmap objects are described below.
Not available on the Macintosh.
\end{funcdesc}

\begin{funcdesc}{fleep}{}
Cause a beep or bell (or perhaps a `visual bell' or flash, hence the
name).
\end{funcdesc}

\begin{funcdesc}{message}{string}
Display a dialog box containing the string.
The user must click OK before the function returns.
\end{funcdesc}

\begin{funcdesc}{askync}{prompt, default}
Display a dialog that prompts the user to answer a question with yes or
no.  Return 0 for no, 1 for yes.  If the user hits the Return key, the
default (which must be 0 or 1) is returned.  If the user cancels the
dialog, \exception{KeyboardInterrupt} is raised.
\end{funcdesc}

\begin{funcdesc}{askstr}{prompt, default}
Display a dialog that prompts the user for a string.
If the user hits the Return key, the default string is returned.
If the user cancels the dialog, \exception{KeyboardInterrupt} is
raised.
\end{funcdesc}

\begin{funcdesc}{askfile}{prompt, default, new}
Ask the user to specify a filename.  If \var{new} is zero it must be
an existing file; otherwise, it must be a new file.  If the user
cancels the dialog, \exception{KeyboardInterrupt} is raised.
\end{funcdesc}

\begin{funcdesc}{setcutbuffer}{i, string}
Store the string in the system's cut buffer number \var{i}, where it
can be found (for pasting) by other applications.  On X11, there are 8
cut buffers (numbered 0..7).  Cut buffer number 0 is the `clipboard'
on the Macintosh.
\end{funcdesc}

\begin{funcdesc}{getcutbuffer}{i}
Return the contents of the system's cut buffer number \var{i}.
\end{funcdesc}

\begin{funcdesc}{rotatecutbuffers}{n}
On X11, rotate the 8 cut buffers by \var{n}.  Ignored on the
Macintosh.
\end{funcdesc}

\begin{funcdesc}{getselection}{i}
Return X11 selection number \var{i.}  Selections are not cut buffers.
Selection numbers are defined in module \module{stdwinevents}.
Selection \constant{WS_PRIMARY} is the \dfn{primary} selection (used
by \program{xterm}, for instance); selection \constant{WS_SECONDARY}
is the \dfn{secondary} selection; selection \constant{WS_CLIPBOARD} is
the \dfn{clipboard} selection (used by \program{xclipboard}).  On the
Macintosh, this always returns an empty string.
\end{funcdesc}

\begin{funcdesc}{resetselection}{i}
Reset selection number \var{i}, if this process owns it.  (See window
method \method{setselection()}).
\end{funcdesc}

\begin{funcdesc}{baseline}{}
Return the baseline of the current font (defined by STDWIN as the
vertical distance between the baseline and the top of the
characters).
\end{funcdesc}

\begin{funcdesc}{lineheight}{}
Return the total line height of the current font.
\end{funcdesc}

\begin{funcdesc}{textbreak}{str, width}
Return the number of characters of the string that fit into a space of
\var{width}
bits wide when drawn in the curent font.
\end{funcdesc}

\begin{funcdesc}{textwidth}{str}
Return the width in bits of the string when drawn in the current font.
\end{funcdesc}

\begin{funcdesc}{connectionnumber}{}
\funcline{fileno}{}
(X11 under \UNIX{} only) Return the ``connection number'' used by the
underlying X11 implementation.  (This is normally the file number of
the socket.)  Both functions return the same value;
\method{connectionnumber()} is named after the corresponding function in
X11 and STDWIN, while \method{fileno()} makes it possible to use the
\module{stdwin} module as a ``file'' object parameter to
\function{select.select()}.  Note that if \constant{select()} implies that
input is possible on \module{stdwin}, this does not guarantee that an
event is ready --- it may be some internal communication going on
between the X server and the client library.  Thus, you should call
\function{stdwin.pollevent()} until it returns \code{None} to check for
events if you don't want your program to block.  Because of internal
buffering in X11, it is also possible that \function{stdwin.pollevent()}
returns an event while \function{select()} does not find \module{stdwin} to
be ready, so you should read any pending events with
\function{stdwin.pollevent()} until it returns \code{None} before entering
a blocking \function{select()} call.
\withsubitem{(in module select)}{\ttindex{select()}}
\end{funcdesc}

\subsection{Window Objects}
\nodename{STDWIN Window Objects}

Window objects are created by \function{stdwin.open()}.  They are closed
by their \method{close()} method or when they are garbage-collected.
Window objects have the following methods:

\begin{methoddesc}[window]{begindrawing}{}
Return a drawing object, whose methods (described below) allow drawing
in the window.
\end{methoddesc}

\begin{methoddesc}[window]{change}{rect}
Invalidate the given rectangle; this may cause a draw event.
\end{methoddesc}

\begin{methoddesc}[window]{gettitle}{}
Returns the window's title string.
\end{methoddesc}

\begin{methoddesc}[window]{getdocsize}{}
\begin{sloppypar}
Return a pair of integers giving the size of the document as set by
\method{setdocsize()}.
\end{sloppypar}
\end{methoddesc}

\begin{methoddesc}[window]{getorigin}{}
Return a pair of integers giving the origin of the window with respect
to the document.
\end{methoddesc}

\begin{methoddesc}[window]{gettitle}{}
Return the window's title string.
\end{methoddesc}

\begin{methoddesc}[window]{getwinsize}{}
Return a pair of integers giving the size of the window.
\end{methoddesc}

\begin{methoddesc}[window]{getwinpos}{}
Return a pair of integers giving the position of the window's upper
left corner (relative to the upper left corner of the screen).
\end{methoddesc}

\begin{methoddesc}[window]{menucreate}{title}
Create a menu object referring to a local menu (a menu that appears
only in this window).
Methods of menu objects are described below.
\strong{Warning:} the menu only appears as long as the object
returned by this call exists.
\end{methoddesc}

\begin{methoddesc}[window]{scroll}{rect, point}
Scroll the given rectangle by the vector given by the point.
\end{methoddesc}

\begin{methoddesc}[window]{setdocsize}{point}
Set the size of the drawing document.
\end{methoddesc}

\begin{methoddesc}[window]{setorigin}{point}
Move the origin of the window (its upper left corner)
to the given point in the document.
\end{methoddesc}

\begin{methoddesc}[window]{setselection}{i, str}
Attempt to set X11 selection number \var{i} to the string \var{str}.
(See \module{stdwin} function \function{getselection()} for the
meaning of \var{i}.)  Return true if it succeeds.
If  succeeds, the window ``owns'' the selection until
(a) another application takes ownership of the selection; or
(b) the window is deleted; or
(c) the application clears ownership by calling
\function{stdwin.resetselection(\var{i})}.  When another application
takes ownership of the selection, a \constant{WE_LOST_SEL} event is
received for no particular window and with the selection number as
detail.  Ignored on the Macintosh.
\end{methoddesc}

\begin{methoddesc}[window]{settimer}{dsecs}
Schedule a timer event for the window in \code{\var{dsecs}/10}
seconds.
\end{methoddesc}

\begin{methoddesc}[window]{settitle}{title}
Set the window's title string.
\end{methoddesc}

\begin{methoddesc}[window]{setwincursor}{name}
\begin{sloppypar}
Set the window cursor to a cursor of the given name.  It raises
\exception{RuntimeError} if no cursor of the given name exists.
Suitable names include
\code{'ibeam'},
\code{'arrow'},
\code{'cross'},
\code{'watch'}
and
\code{'plus'}.
On X11, there are many more (see \code{<X11/cursorfont.h>}).
\end{sloppypar}
\end{methoddesc}

\begin{methoddesc}[window]{setwinpos}{h, v}
Set the the position of the window's upper left corner (relative to
the upper left corner of the screen).
\end{methoddesc}

\begin{methoddesc}[window]{setwinsize}{width, height}
Set the window's size.
\end{methoddesc}

\begin{methoddesc}[window]{show}{rect}
Try to ensure that the given rectangle of the document is visible in
the window.
\end{methoddesc}

\begin{methoddesc}[window]{textcreate}{rect}
Create a text-edit object in the document at the given rectangle.
Methods of text-edit objects are described below.
\end{methoddesc}

\begin{methoddesc}[window]{setactive}{}
Attempt to make this window the active window.  If successful, this
will generate a WE_ACTIVATE event (and a WE_DEACTIVATE event in case
another window in this application became inactive).
\end{methoddesc}

\begin{methoddesc}[window]{close}{}
Discard the window object.  It should not be used again.
\end{methoddesc}

\subsection{Drawing Objects}

Drawing objects are created exclusively by the window method
\method{begindrawing()}.  Only one drawing object can exist at any
given time; the drawing object must be deleted to finish drawing.  No
drawing object may exist when \function{stdwin.getevent()} is called.
Drawing objects have the following methods:

\begin{methoddesc}[drawing]{box}{rect}
Draw a box just inside a rectangle.
\end{methoddesc}

\begin{methoddesc}[drawing]{circle}{center, radius}
Draw a circle with given center point and radius.
\end{methoddesc}

\begin{methoddesc}[drawing]{elarc}{center, (rh, rv), (a1, a2)}
Draw an elliptical arc with given center point.
\code{(\var{rh}, \var{rv})}
gives the half sizes of the horizontal and vertical radii.
\code{(\var{a1}, \var{a2})}
gives the angles (in degrees) of the begin and end points.
0 degrees is at 3 o'clock, 90 degrees is at 12 o'clock.
\end{methoddesc}

\begin{methoddesc}[drawing]{erase}{rect}
Erase a rectangle.
\end{methoddesc}

\begin{methoddesc}[drawing]{fillcircle}{center, radius}
Draw a filled circle with given center point and radius.
\end{methoddesc}

\begin{methoddesc}[drawing]{fillelarc}{center, (rh, rv), (a1, a2)}
Draw a filled elliptical arc; arguments as for \method{elarc()}.
\end{methoddesc}

\begin{methoddesc}[drawing]{fillpoly}{points}
Draw a filled polygon given by a list (or tuple) of points.
\end{methoddesc}

\begin{methoddesc}[drawing]{invert}{rect}
Invert a rectangle.
\end{methoddesc}

\begin{methoddesc}[drawing]{line}{p1, p2}
Draw a line from point
\var{p1}
to
\var{p2}.
\end{methoddesc}

\begin{methoddesc}[drawing]{paint}{rect}
Fill a rectangle.
\end{methoddesc}

\begin{methoddesc}[drawing]{poly}{points}
Draw the lines connecting the given list (or tuple) of points.
\end{methoddesc}

\begin{methoddesc}[drawing]{shade}{rect, percent}
Fill a rectangle with a shading pattern that is about
\var{percent}
percent filled.
\end{methoddesc}

\begin{methoddesc}[drawing]{text}{p, str}
Draw a string starting at point p (the point specifies the
top left coordinate of the string).
\end{methoddesc}

\begin{methoddesc}[drawing]{xorcircle}{center, radius}
\funcline{xorelarc}{center, (rh, rv), (a1, a2)}
\funcline{xorline}{p1, p2}
\funcline{xorpoly}{points}
Draw a circle, an elliptical arc, a line or a polygon, respectively,
in XOR mode.
\end{methoddesc}

\begin{methoddesc}[drawing]{setfgcolor}{}
\funcline{setbgcolor}{}
\funcline{getfgcolor}{}
\funcline{getbgcolor}{}
These functions are similar to the corresponding functions described
above for the \module{stdwin}
module, but affect or return the colors currently used for drawing
instead of the global default colors.
When a drawing object is created, its colors are set to the window's
default colors, which are in turn initialized from the global default
colors when the window is created.
\end{methoddesc}

\begin{methoddesc}[drawing]{setfont}{}
\funcline{baseline}{}
\funcline{lineheight}{}
\funcline{textbreak}{}
\funcline{textwidth}{}
These functions are similar to the corresponding functions described
above for the \module{stdwin}
module, but affect or use the current drawing font instead of
the global default font.
When a drawing object is created, its font is set to the window's
default font, which is in turn initialized from the global default
font when the window is created.
\end{methoddesc}

\begin{methoddesc}[drawing]{bitmap}{point, bitmap, mask}
Draw the \var{bitmap} with its top left corner at \var{point}.
If the optional \var{mask} argument is present, it should be either
the same object as \var{bitmap}, to draw only those bits that are set
in the bitmap, in the foreground color, or \code{None}, to draw all
bits (ones are drawn in the foreground color, zeros in the background
color).
Not available on the Macintosh.
\end{methoddesc}

\begin{methoddesc}[drawing]{cliprect}{rect}
Set the ``clipping region'' to a rectangle.
The clipping region limits the effect of all drawing operations, until
it is changed again or until the drawing object is closed.  When a
drawing object is created the clipping region is set to the entire
window.  When an object to be drawn falls partly outside the clipping
region, the set of pixels drawn is the intersection of the clipping
region and the set of pixels that would be drawn by the same operation
in the absence of a clipping region.
\end{methoddesc}

\begin{methoddesc}[drawing]{noclip}{}
Reset the clipping region to the entire window.
\end{methoddesc}

\begin{methoddesc}[drawing]{close}{}
\funcline{enddrawing}{}
Discard the drawing object.  It should not be used again.
\end{methoddesc}

\subsection{Menu Objects}

A menu object represents a menu.
The menu is destroyed when the menu object is deleted.
The following methods are defined:


\begin{methoddesc}[menu]{additem}{text, shortcut}
Add a menu item with given text.
The shortcut must be a string of length 1, or omitted (to specify no
shortcut).
\end{methoddesc}

\begin{methoddesc}[menu]{setitem}{i, text}
Set the text of item number \var{i}.
\end{methoddesc}

\begin{methoddesc}[menu]{enable}{i, flag}
Enable or disables item \var{i}.
\end{methoddesc}

\begin{methoddesc}[menu]{check}{i, flag}
Set or clear the \dfn{check mark} for item \var{i}.
\end{methoddesc}

\begin{methoddesc}[menu]{close}{}
Discard the menu object.  It should not be used again.
\end{methoddesc}

\subsection{Bitmap Objects}

A bitmap represents a rectangular array of bits.
The top left bit has coordinate (0, 0).
A bitmap can be drawn with the \method{bitmap()} method of a drawing object.
Bitmaps are currently not available on the Macintosh.

The following methods are defined:


\begin{methoddesc}[bitmap]{getsize}{}
Return a tuple representing the width and height of the bitmap.
(This returns the values that have been passed to the
\function{newbitmap()} function.)
\end{methoddesc}

\begin{methoddesc}[bitmap]{setbit}{point, bit}
Set the value of the bit indicated by \var{point} to \var{bit}.
\end{methoddesc}

\begin{methoddesc}[bitmap]{getbit}{point}
Return the value of the bit indicated by \var{point}.
\end{methoddesc}

\begin{methoddesc}[bitmap]{close}{}
Discard the bitmap object.  It should not be used again.
\end{methoddesc}

\subsection{Text-edit Objects}

A text-edit object represents a text-edit block.
For semantics, see the STDWIN documentation for \C{} programmers.
The following methods exist:


\begin{methoddesc}[text-edit]{arrow}{code}
Pass an arrow event to the text-edit block.
The \var{code} must be one of \constant{WC_LEFT}, \constant{WC_RIGHT}, 
\constant{WC_UP} or \constant{WC_DOWN} (see module
\module{stdwinevents}).
\end{methoddesc}

\begin{methoddesc}[text-edit]{draw}{rect}
Pass a draw event to the text-edit block.
The rectangle specifies the redraw area.
\end{methoddesc}

\begin{methoddesc}[text-edit]{event}{type, window, detail}
Pass an event gotten from
\function{stdwin.getevent()}
to the text-edit block.
Return true if the event was handled.
\end{methoddesc}

\begin{methoddesc}[text-edit]{getfocus}{}
Return 2 integers representing the start and end positions of the
focus, usable as slice indices on the string returned by
\method{gettext()}.
\end{methoddesc}

\begin{methoddesc}[text-edit]{getfocustext}{}
Return the text in the focus.
\end{methoddesc}

\begin{methoddesc}[text-edit]{getrect}{}
Return a rectangle giving the actual position of the text-edit block.
(The bottom coordinate may differ from the initial position because
the block automatically shrinks or grows to fit.)
\end{methoddesc}

\begin{methoddesc}[text-edit]{gettext}{}
Return the entire text buffer.
\end{methoddesc}

\begin{methoddesc}[text-edit]{move}{rect}
Specify a new position for the text-edit block in the document.
\end{methoddesc}

\begin{methoddesc}[text-edit]{replace}{str}
Replace the text in the focus by the given string.
The new focus is an insert point at the end of the string.
\end{methoddesc}

\begin{methoddesc}[text-edit]{setfocus}{i, j}
Specify the new focus.
Out-of-bounds values are silently clipped.
\end{methoddesc}

\begin{methoddesc}[text-edit]{settext}{str}
Replace the entire text buffer by the given string and set the focus
to \code{(0, 0)}.
\end{methoddesc}

\begin{methoddesc}[text-edit]{setview}{rect}
Set the view rectangle to \var{rect}.  If \var{rect} is \code{None},
viewing mode is reset.  In viewing mode, all output from the text-edit
object is clipped to the viewing rectangle.  This may be useful to
implement your own scrolling text subwindow.
\end{methoddesc}

\begin{methoddesc}[text-edit]{close}{}
Discard the text-edit object.  It should not be used again.
\end{methoddesc}

\subsection{Example}
\nodename{STDWIN Example}

Here is a minimal example of using STDWIN in Python.
It creates a window and draws the string ``Hello world'' in the top
left corner of the window.
The window will be correctly redrawn when covered and re-exposed.
The program quits when the close icon or menu item is requested.

\begin{verbatim}
import stdwin
from stdwinevents import *

def main():
    mywin = stdwin.open('Hello')
    #
    while 1:
        (type, win, detail) = stdwin.getevent()
        if type == WE_DRAW:
            draw = win.begindrawing()
            draw.text((0, 0), 'Hello, world')
            del draw
        elif type == WE_CLOSE:
            break

main()
\end{verbatim}

\section{\module{stdwinevents} ---
         Constants for use with \module{stdwin}}
\declaremodule{standard}{stdwinevents}

\modulesynopsis{Constant definitions for use with \module{stdwin}}


This module defines constants used by STDWIN for event types
(\constant{WE_ACTIVATE} etc.), command codes (\constant{WC_LEFT} etc.)
and selection types (\constant{WS_PRIMARY} etc.).
Read the file for details.
Suggested usage is

\begin{verbatim}
>>> from stdwinevents import *
>>> 
\end{verbatim}

\section{\module{rect} ---
         Functions for use with \module{stdwin}}
\declaremodule{standard}{rect}

\modulesynopsis{Geometry-related utility function for use with \module{stdwin}}


This module contains useful operations on rectangles.
A rectangle is defined as in module \module{stdwin}:
a pair of points, where a point is a pair of integers.
For example, the rectangle

\begin{verbatim}
(10, 20), (90, 80)
\end{verbatim}

is a rectangle whose left, top, right and bottom edges are 10, 20, 90
and 80, respectively.  Note that the positive vertical axis points
down (as in \module{stdwin}).

The module defines the following objects:

\begin{excdesc}{error}
The exception raised by functions in this module when they detect an
error.  The exception argument is a string describing the problem in
more detail.
\end{excdesc}

\begin{datadesc}{empty}
The rectangle returned when some operations return an empty result.
This makes it possible to quickly check whether a result is empty:

\begin{verbatim}
>>> import rect
>>> r1 = (10, 20), (90, 80)
>>> r2 = (0, 0), (10, 20)
>>> r3 = rect.intersect([r1, r2])
>>> if r3 is rect.empty: print 'Empty intersection'
Empty intersection
>>> 
\end{verbatim}
\end{datadesc}

\begin{funcdesc}{is_empty}{r}
Returns true if the given rectangle is empty.
A rectangle
\code{(\var{left}, \var{top}), (\var{right}, \var{bottom})}
is empty if
\begin{math}\var{left} \geq \var{right}\end{math} or
\begin{math}\var{top} \geq \var{bottom}\end{math}.
\end{funcdesc}

\begin{funcdesc}{intersect}{list}
Returns the intersection of all rectangles in the list argument.
It may also be called with a tuple argument.  Raises
\exception{rect.error} if the list is empty.  Returns
\constant{rect.empty} if the intersection of the rectangles is empty.
\end{funcdesc}

\begin{funcdesc}{union}{list}
Returns the smallest rectangle that contains all non-empty rectangles in
the list argument.  It may also be called with a tuple argument or
with two or more rectangles as arguments.  Returns
\constant{rect.empty} if the list is empty or all its rectangles are
empty.
\end{funcdesc}

\begin{funcdesc}{pointinrect}{point, rect}
Returns true if the point is inside the rectangle.  By definition, a
point \code{(\var{h}, \var{v})} is inside a rectangle
\code{(\var{left}, \var{top}), (\var{right}, \var{bottom})} if
\begin{math}\var{left} \leq \var{h} < \var{right}\end{math} and
\begin{math}\var{top} \leq \var{v} < \var{bottom}\end{math}.
\end{funcdesc}

\begin{funcdesc}{inset}{rect, (dh, dv)}
Returns a rectangle that lies inside the \var{rect} argument by
\var{dh} pixels horizontally and \var{dv} pixels vertically.  If
\var{dh} or \var{dv} is negative, the result lies outside \var{rect}.
\end{funcdesc}

\begin{funcdesc}{rect2geom}{rect}
Converts a rectangle to geometry representation:
\code{(\var{left}, \var{top}), (\var{width}, \var{height})}.
\end{funcdesc}

\begin{funcdesc}{geom2rect}{geom}
Converts a rectangle given in geometry representation back to the
standard rectangle representation
\code{(\var{left}, \var{top}), (\var{right}, \var{bottom})}.
\end{funcdesc}
		% STDWIN ONLY

\chapter{SGI IRIX Specific Services}
\label{sgi}

The modules described in this chapter provide interfaces to features
that are unique to SGI's IRIX operating system (versions 4 and 5).

\localmoduletable
			% SGI IRIX ONLY
\section{\module{al} ---
         Audio functions on the SGI}

\declaremodule{builtin}{al}
  \platform{IRIX}
\modulesynopsis{Audio functions on the SGI.}


This module provides access to the audio facilities of the SGI Indy
and Indigo workstations.  See section 3A of the IRIX man pages for
details.  You'll need to read those man pages to understand what these
functions do!  Some of the functions are not available in IRIX
releases before 4.0.5.  Again, see the manual to check whether a
specific function is available on your platform.

All functions and methods defined in this module are equivalent to
the C functions with \samp{AL} prefixed to their name.

Symbolic constants from the C header file \code{<audio.h>} are
defined in the standard module \module{AL}\refstmodindex{AL}, see
below.

\strong{Warning:} the current version of the audio library may dump core
when bad argument values are passed rather than returning an error
status.  Unfortunately, since the precise circumstances under which
this may happen are undocumented and hard to check, the Python
interface can provide no protection against this kind of problems.
(One example is specifying an excessive queue size --- there is no
documented upper limit.)

The module defines the following functions:


\begin{funcdesc}{openport}{name, direction\optional{, config}}
The name and direction arguments are strings.  The optional
\var{config} argument is a configuration object as returned by
\function{newconfig()}.  The return value is an \dfn{audio port
object}; methods of audio port objects are described below.
\end{funcdesc}

\begin{funcdesc}{newconfig}{}
The return value is a new \dfn{audio configuration object}; methods of
audio configuration objects are described below.
\end{funcdesc}

\begin{funcdesc}{queryparams}{device}
The device argument is an integer.  The return value is a list of
integers containing the data returned by \cfunction{ALqueryparams()}.
\end{funcdesc}

\begin{funcdesc}{getparams}{device, list}
The \var{device} argument is an integer.  The list argument is a list
such as returned by \function{queryparams()}; it is modified in place
(!).
\end{funcdesc}

\begin{funcdesc}{setparams}{device, list}
The \var{device} argument is an integer.  The \var{list} argument is a
list such as returned by \function{queryparams()}.
\end{funcdesc}


\subsection{Configuration Objects \label{al-config-objects}}

Configuration objects (returned by \function{newconfig()} have the
following methods:

\begin{methoddesc}[audio configuration]{getqueuesize}{}
Return the queue size.
\end{methoddesc}

\begin{methoddesc}[audio configuration]{setqueuesize}{size}
Set the queue size.
\end{methoddesc}

\begin{methoddesc}[audio configuration]{getwidth}{}
Get the sample width.
\end{methoddesc}

\begin{methoddesc}[audio configuration]{setwidth}{width}
Set the sample width.
\end{methoddesc}

\begin{methoddesc}[audio configuration]{getchannels}{}
Get the channel count.
\end{methoddesc}

\begin{methoddesc}[audio configuration]{setchannels}{nchannels}
Set the channel count.
\end{methoddesc}

\begin{methoddesc}[audio configuration]{getsampfmt}{}
Get the sample format.
\end{methoddesc}

\begin{methoddesc}[audio configuration]{setsampfmt}{sampfmt}
Set the sample format.
\end{methoddesc}

\begin{methoddesc}[audio configuration]{getfloatmax}{}
Get the maximum value for floating sample formats.
\end{methoddesc}

\begin{methoddesc}[audio configuration]{setfloatmax}{floatmax}
Set the maximum value for floating sample formats.
\end{methoddesc}


\subsection{Port Objects \label{al-port-objects}}

Port objects, as returned by \function{openport()}, have the following
methods:

\begin{methoddesc}[audio port]{closeport}{}
Close the port.
\end{methoddesc}

\begin{methoddesc}[audio port]{getfd}{}
Return the file descriptor as an int.
\end{methoddesc}

\begin{methoddesc}[audio port]{getfilled}{}
Return the number of filled samples.
\end{methoddesc}

\begin{methoddesc}[audio port]{getfillable}{}
Return the number of fillable samples.
\end{methoddesc}

\begin{methoddesc}[audio port]{readsamps}{nsamples}
Read a number of samples from the queue, blocking if necessary.
Return the data as a string containing the raw data, (e.g., 2 bytes per
sample in big-endian byte order (high byte, low byte) if you have set
the sample width to 2 bytes).
\end{methoddesc}

\begin{methoddesc}[audio port]{writesamps}{samples}
Write samples into the queue, blocking if necessary.  The samples are
encoded as described for the \method{readsamps()} return value.
\end{methoddesc}

\begin{methoddesc}[audio port]{getfillpoint}{}
Return the `fill point'.
\end{methoddesc}

\begin{methoddesc}[audio port]{setfillpoint}{fillpoint}
Set the `fill point'.
\end{methoddesc}

\begin{methoddesc}[audio port]{getconfig}{}
Return a configuration object containing the current configuration of
the port.
\end{methoddesc}

\begin{methoddesc}[audio port]{setconfig}{config}
Set the configuration from the argument, a configuration object.
\end{methoddesc}

\begin{methoddesc}[audio port]{getstatus}{list}
Get status information on last error.
\end{methoddesc}


\section{\module{AL} ---
         Constants used with the \module{al} module}

\declaremodule[al-constants]{standard}{AL}
  \platform{IRIX}
\modulesynopsis{Constants used with the \module{al} module.}


This module defines symbolic constants needed to use the built-in
module \module{al} (see above); they are equivalent to those defined
in the C header file \code{<audio.h>} except that the name prefix
\samp{AL_} is omitted.  Read the module source for a complete list of
the defined names.  Suggested use:

\begin{verbatim}
import al
from AL import *
\end{verbatim}

\section{\module{cd} ---
         CD-ROM access on SGI systems}

\declaremodule{builtin}{cd}
  \platform{IRIX}
\modulesynopsis{Interface to the CD-ROM on Silicon Graphics systems.}


This module provides an interface to the Silicon Graphics CD library.
It is available only on Silicon Graphics systems.

The way the library works is as follows.  A program opens the CD-ROM
device with \function{open()} and creates a parser to parse the data
from the CD with \function{createparser()}.  The object returned by
\function{open()} can be used to read data from the CD, but also to get
status information for the CD-ROM device, and to get information about
the CD, such as the table of contents.  Data from the CD is passed to
the parser, which parses the frames, and calls any callback
functions that have previously been added.

An audio CD is divided into \dfn{tracks} or \dfn{programs} (the terms
are used interchangeably).  Tracks can be subdivided into
\dfn{indices}.  An audio CD contains a \dfn{table of contents} which
gives the starts of the tracks on the CD.  Index 0 is usually the
pause before the start of a track.  The start of the track as given by
the table of contents is normally the start of index 1.

Positions on a CD can be represented in two ways.  Either a frame
number or a tuple of three values, minutes, seconds and frames.  Most
functions use the latter representation.  Positions can be both
relative to the beginning of the CD, and to the beginning of the
track.

Module \module{cd} defines the following functions and constants:


\begin{funcdesc}{createparser}{}
Create and return an opaque parser object.  The methods of the parser
object are described below.
\end{funcdesc}

\begin{funcdesc}{msftoframe}{minutes, seconds, frames}
Converts a \code{(\var{minutes}, \var{seconds}, \var{frames})} triple
representing time in absolute time code into the corresponding CD
frame number.
\end{funcdesc}

\begin{funcdesc}{open}{\optional{device\optional{, mode}}}
Open the CD-ROM device.  The return value is an opaque player object;
methods of the player object are described below.  The device is the
name of the SCSI device file, e.g. \code{'/dev/scsi/sc0d4l0'}, or
\code{None}.  If omitted or \code{None}, the hardware inventory is
consulted to locate a CD-ROM drive.  The \var{mode}, if not omitted,
should be the string \code{'r'}.
\end{funcdesc}

The module defines the following variables:

\begin{excdesc}{error}
Exception raised on various errors.
\end{excdesc}

\begin{datadesc}{DATASIZE}
The size of one frame's worth of audio data.  This is the size of the
audio data as passed to the callback of type \code{audio}.
\end{datadesc}

\begin{datadesc}{BLOCKSIZE}
The size of one uninterpreted frame of audio data.
\end{datadesc}

The following variables are states as returned by
\function{getstatus()}:

\begin{datadesc}{READY}
The drive is ready for operation loaded with an audio CD.
\end{datadesc}

\begin{datadesc}{NODISC}
The drive does not have a CD loaded.
\end{datadesc}

\begin{datadesc}{CDROM}
The drive is loaded with a CD-ROM.  Subsequent play or read operations
will return I/O errors.
\end{datadesc}

\begin{datadesc}{ERROR}
An error occurred while trying to read the disc or its table of
contents.
\end{datadesc}

\begin{datadesc}{PLAYING}
The drive is in CD player mode playing an audio CD through its audio
jacks.
\end{datadesc}

\begin{datadesc}{PAUSED}
The drive is in CD layer mode with play paused.
\end{datadesc}

\begin{datadesc}{STILL}
The equivalent of \constant{PAUSED} on older (non 3301) model Toshiba
CD-ROM drives.  Such drives have never been shipped by SGI.
\end{datadesc}

\begin{datadesc}{audio}
\dataline{pnum}
\dataline{index}
\dataline{ptime}
\dataline{atime}
\dataline{catalog}
\dataline{ident}
\dataline{control}
Integer constants describing the various types of parser callbacks
that can be set by the \method{addcallback()} method of CD parser
objects (see below).
\end{datadesc}


\subsection{Player Objects}
\label{player-objects}

Player objects (returned by \function{open()}) have the following
methods:

\begin{methoddesc}[CD player]{allowremoval}{}
Unlocks the eject button on the CD-ROM drive permitting the user to
eject the caddy if desired.
\end{methoddesc}

\begin{methoddesc}[CD player]{bestreadsize}{}
Returns the best value to use for the \var{num_frames} parameter of
the \method{readda()} method.  Best is defined as the value that
permits a continuous flow of data from the CD-ROM drive.
\end{methoddesc}

\begin{methoddesc}[CD player]{close}{}
Frees the resources associated with the player object.  After calling
\method{close()}, the methods of the object should no longer be used.
\end{methoddesc}

\begin{methoddesc}[CD player]{eject}{}
Ejects the caddy from the CD-ROM drive.
\end{methoddesc}

\begin{methoddesc}[CD player]{getstatus}{}
Returns information pertaining to the current state of the CD-ROM
drive.  The returned information is a tuple with the following values:
\var{state}, \var{track}, \var{rtime}, \var{atime}, \var{ttime},
\var{first}, \var{last}, \var{scsi_audio}, \var{cur_block}.
\var{rtime} is the time relative to the start of the current track;
\var{atime} is the time relative to the beginning of the disc;
\var{ttime} is the total time on the disc.  For more information on
the meaning of the values, see the man page \manpage{CDgetstatus}{3dm}.
The value of \var{state} is one of the following: \constant{ERROR},
\constant{NODISC}, \constant{READY}, \constant{PLAYING},
\constant{PAUSED}, \constant{STILL}, or \constant{CDROM}.
\end{methoddesc}

\begin{methoddesc}[CD player]{gettrackinfo}{track}
Returns information about the specified track.  The returned
information is a tuple consisting of two elements, the start time of
the track and the duration of the track.
\end{methoddesc}

\begin{methoddesc}[CD player]{msftoblock}{min, sec, frame}
Converts a minutes, seconds, frames triple representing a time in
absolute time code into the corresponding logical block number for the
given CD-ROM drive.  You should use \function{msftoframe()} rather than
\method{msftoblock()} for comparing times.  The logical block number
differs from the frame number by an offset required by certain CD-ROM
drives.
\end{methoddesc}

\begin{methoddesc}[CD player]{play}{start, play}
Starts playback of an audio CD in the CD-ROM drive at the specified
track.  The audio output appears on the CD-ROM drive's headphone and
audio jacks (if fitted).  Play stops at the end of the disc.
\var{start} is the number of the track at which to start playing the
CD; if \var{play} is 0, the CD will be set to an initial paused
state.  The method \method{togglepause()} can then be used to commence
play.
\end{methoddesc}

\begin{methoddesc}[CD player]{playabs}{minutes, seconds, frames, play}
Like \method{play()}, except that the start is given in minutes,
seconds, and frames instead of a track number.
\end{methoddesc}

\begin{methoddesc}[CD player]{playtrack}{start, play}
Like \method{play()}, except that playing stops at the end of the
track.
\end{methoddesc}

\begin{methoddesc}[CD player]{playtrackabs}{track, minutes, seconds, frames, play}
Like \method{play()}, except that playing begins at the specified
absolute time and ends at the end of the specified track.
\end{methoddesc}

\begin{methoddesc}[CD player]{preventremoval}{}
Locks the eject button on the CD-ROM drive thus preventing the user
from arbitrarily ejecting the caddy.
\end{methoddesc}

\begin{methoddesc}[CD player]{readda}{num_frames}
Reads the specified number of frames from an audio CD mounted in the
CD-ROM drive.  The return value is a string representing the audio
frames.  This string can be passed unaltered to the
\method{parseframe()} method of the parser object.
\end{methoddesc}

\begin{methoddesc}[CD player]{seek}{minutes, seconds, frames}
Sets the pointer that indicates the starting point of the next read of
digital audio data from a CD-ROM.  The pointer is set to an absolute
time code location specified in \var{minutes}, \var{seconds}, and
\var{frames}.  The return value is the logical block number to which
the pointer has been set.
\end{methoddesc}

\begin{methoddesc}[CD player]{seekblock}{block}
Sets the pointer that indicates the starting point of the next read of
digital audio data from a CD-ROM.  The pointer is set to the specified
logical block number.  The return value is the logical block number to
which the pointer has been set.
\end{methoddesc}

\begin{methoddesc}[CD player]{seektrack}{track}
Sets the pointer that indicates the starting point of the next read of
digital audio data from a CD-ROM.  The pointer is set to the specified
track.  The return value is the logical block number to which the
pointer has been set.
\end{methoddesc}

\begin{methoddesc}[CD player]{stop}{}
Stops the current playing operation.
\end{methoddesc}

\begin{methoddesc}[CD player]{togglepause}{}
Pauses the CD if it is playing, and makes it play if it is paused.
\end{methoddesc}


\subsection{Parser Objects}
\label{cd-parser-objects}

Parser objects (returned by \function{createparser()}) have the
following methods:

\begin{methoddesc}[CD parser]{addcallback}{type, func, arg}
Adds a callback for the parser.  The parser has callbacks for eight
different types of data in the digital audio data stream.  Constants
for these types are defined at the \module{cd} module level (see above).
The callback is called as follows: \code{\var{func}(\var{arg}, type,
data)}, where \var{arg} is the user supplied argument, \var{type} is
the particular type of callback, and \var{data} is the data returned
for this \var{type} of callback.  The type of the data depends on the
\var{type} of callback as follows:

\begin{tableii}{l|p{4in}}{code}{Type}{Value}
  \lineii{audio}{String which can be passed unmodified to
\function{al.writesamps()}.}
  \lineii{pnum}{Integer giving the program (track) number.}
  \lineii{index}{Integer giving the index number.}
  \lineii{ptime}{Tuple consisting of the program time in minutes,
seconds, and frames.}
  \lineii{atime}{Tuple consisting of the absolute time in minutes,
seconds, and frames.}
  \lineii{catalog}{String of 13 characters, giving the catalog number
of the CD.}
  \lineii{ident}{String of 12 characters, giving the ISRC
identification number of the recording.  The string consists of two
characters country code, three characters owner code, two characters
giving the year, and five characters giving a serial number.}
  \lineii{control}{Integer giving the control bits from the CD
subcode data}
\end{tableii}
\end{methoddesc}

\begin{methoddesc}[CD parser]{deleteparser}{}
Deletes the parser and frees the memory it was using.  The object
should not be used after this call.  This call is done automatically
when the last reference to the object is removed.
\end{methoddesc}

\begin{methoddesc}[CD parser]{parseframe}{frame}
Parses one or more frames of digital audio data from a CD such as
returned by \method{readda()}.  It determines which subcodes are
present in the data.  If these subcodes have changed since the last
frame, then \method{parseframe()} executes a callback of the
appropriate type passing to it the subcode data found in the frame.
Unlike the \C{} function, more than one frame of digital audio data
can be passed to this method.
\end{methoddesc}

\begin{methoddesc}[CD parser]{removecallback}{type}
Removes the callback for the given \var{type}.
\end{methoddesc}

\begin{methoddesc}[CD parser]{resetparser}{}
Resets the fields of the parser used for tracking subcodes to an
initial state.  \method{resetparser()} should be called after the disc
has been changed.
\end{methoddesc}

\section{Built-in Module \sectcode{fl}}
\label{module-fl}
\bimodindex{fl}

This module provides an interface to the FORMS Library by Mark
Overmars.  The source for the library can be retrieved by anonymous
ftp from host \samp{ftp.cs.ruu.nl}, directory \file{SGI/FORMS}.  It
was last tested with version 2.0b.

Most functions are literal translations of their C equivalents,
dropping the initial \samp{fl_} from their name.  Constants used by
the library are defined in module \code{FL} described below.

The creation of objects is a little different in Python than in C:
instead of the `current form' maintained by the library to which new
FORMS objects are added, all functions that add a FORMS object to a
form are methods of the Python object representing the form.
Consequently, there are no Python equivalents for the C functions
\code{fl_addto_form} and \code{fl_end_form}, and the equivalent of
\code{fl_bgn_form} is called \code{fl.make_form}.

Watch out for the somewhat confusing terminology: FORMS uses the word
\dfn{object} for the buttons, sliders etc. that you can place in a form.
In Python, `object' means any value.  The Python interface to FORMS
introduces two new Python object types: form objects (representing an
entire form) and FORMS objects (representing one button, slider etc.).
Hopefully this isn't too confusing...

There are no `free objects' in the Python interface to FORMS, nor is
there an easy way to add object classes written in Python.  The FORMS
interface to GL event handling is available, though, so you can mix
FORMS with pure GL windows.

\strong{Please note:} importing \code{fl} implies a call to the GL function
\code{foreground()} and to the FORMS routine \code{fl_init()}.

\subsection{Functions Defined in Module \sectcode{fl}}
\nodename{FL Functions}

Module \code{fl} defines the following functions.  For more information
about what they do, see the description of the equivalent C function
in the FORMS documentation:

\setindexsubitem{(in module fl)}
\begin{funcdesc}{make_form}{type\, width\, height}
Create a form with given type, width and height.  This returns a
\dfn{form} object, whose methods are described below.
\end{funcdesc}

\begin{funcdesc}{do_forms}{}
The standard FORMS main loop.  Returns a Python object representing
the FORMS object needing interaction, or the special value
\code{FL.EVENT}.
\end{funcdesc}

\begin{funcdesc}{check_forms}{}
Check for FORMS events.  Returns what \code{do_forms} above returns,
or \code{None} if there is no event that immediately needs
interaction.
\end{funcdesc}

\begin{funcdesc}{set_event_call_back}{function}
Set the event callback function.
\end{funcdesc}

\begin{funcdesc}{set_graphics_mode}{rgbmode\, doublebuffering}
Set the graphics modes.
\end{funcdesc}

\begin{funcdesc}{get_rgbmode}{}
Return the current rgb mode.  This is the value of the C global
variable \code{fl_rgbmode}.
\end{funcdesc}

\begin{funcdesc}{show_message}{str1\, str2\, str3}
Show a dialog box with a three-line message and an OK button.
\end{funcdesc}

\begin{funcdesc}{show_question}{str1\, str2\, str3}
Show a dialog box with a three-line message and YES and NO buttons.
It returns \code{1} if the user pressed YES, \code{0} if NO.
\end{funcdesc}

\begin{funcdesc}{show_choice}{str1, str2, str3, but1\optional{, but2, but3}}
Show a dialog box with a three-line message and up to three buttons.
It returns the number of the button clicked by the user
(\code{1}, \code{2} or \code{3}).
\end{funcdesc}

\begin{funcdesc}{show_input}{prompt\, default}
Show a dialog box with a one-line prompt message and text field in
which the user can enter a string.  The second argument is the default
input string.  It returns the string value as edited by the user.
\end{funcdesc}

\begin{funcdesc}{show_file_selector}{message\, directory\, pattern\, default}
Show a dialog box in which the user can select a file.  It returns
the absolute filename selected by the user, or \code{None} if the user
presses Cancel.
\end{funcdesc}

\begin{funcdesc}{get_directory}{}
\funcline{get_pattern}{}
\funcline{get_filename}{}
These functions return the directory, pattern and filename (the tail
part only) selected by the user in the last \code{show_file_selector}
call.
\end{funcdesc}

\begin{funcdesc}{qdevice}{dev}
\funcline{unqdevice}{dev}
\funcline{isqueued}{dev}
\funcline{qtest}{}
\funcline{qread}{}
%\funcline{blkqread}{?}
\funcline{qreset}{}
\funcline{qenter}{dev\, val}
\funcline{get_mouse}{}
\funcline{tie}{button\, valuator1\, valuator2}
These functions are the FORMS interfaces to the corresponding GL
functions.  Use these if you want to handle some GL events yourself
when using \code{fl.do_events}.  When a GL event is detected that
FORMS cannot handle, \code{fl.do_forms()} returns the special value
\code{FL.EVENT} and you should call \code{fl.qread()} to read the
event from the queue.  Don't use the equivalent GL functions!
\end{funcdesc}

\begin{funcdesc}{color}{}
\funcline{mapcolor}{}
\funcline{getmcolor}{}
See the description in the FORMS documentation of \code{fl_color},
\code{fl_mapcolor} and \code{fl_getmcolor}.
\end{funcdesc}

\subsection{Form Objects}

Form objects (returned by \code{fl.make_form()} above) have the
following methods.  Each method corresponds to a C function whose name
is prefixed with \samp{fl_}; and whose first argument is a form
pointer; please refer to the official FORMS documentation for
descriptions.

All the \samp{add_{\rm \ldots}} functions return a Python object representing
the FORMS object.  Methods of FORMS objects are described below.  Most
kinds of FORMS object also have some methods specific to that kind;
these methods are listed here.

\begin{flushleft}
\setindexsubitem{(form object method)}
\begin{funcdesc}{show_form}{placement\, bordertype\, name}
  Show the form.
\end{funcdesc}

\begin{funcdesc}{hide_form}{}
  Hide the form.
\end{funcdesc}

\begin{funcdesc}{redraw_form}{}
  Redraw the form.
\end{funcdesc}

\begin{funcdesc}{set_form_position}{x\, y}
Set the form's position.
\end{funcdesc}

\begin{funcdesc}{freeze_form}{}
Freeze the form.
\end{funcdesc}

\begin{funcdesc}{unfreeze_form}{}
  Unfreeze the form.
\end{funcdesc}

\begin{funcdesc}{activate_form}{}
  Activate the form.
\end{funcdesc}

\begin{funcdesc}{deactivate_form}{}
  Deactivate the form.
\end{funcdesc}

\begin{funcdesc}{bgn_group}{}
  Begin a new group of objects; return a group object.
\end{funcdesc}

\begin{funcdesc}{end_group}{}
  End the current group of objects.
\end{funcdesc}

\begin{funcdesc}{find_first}{}
  Find the first object in the form.
\end{funcdesc}

\begin{funcdesc}{find_last}{}
  Find the last object in the form.
\end{funcdesc}

%---

\begin{funcdesc}{add_box}{type\, x\, y\, w\, h\, name}
Add a box object to the form.
No extra methods.
\end{funcdesc}

\begin{funcdesc}{add_text}{type\, x\, y\, w\, h\, name}
Add a text object to the form.
No extra methods.
\end{funcdesc}

%\begin{funcdesc}{add_bitmap}{type\, x\, y\, w\, h\, name}
%Add a bitmap object to the form.
%\end{funcdesc}

\begin{funcdesc}{add_clock}{type\, x\, y\, w\, h\, name}
Add a clock object to the form. \\
Method:
\code{get_clock}.
\end{funcdesc}

%---

\begin{funcdesc}{add_button}{type\, x\, y\, w\, h\,  name}
Add a button object to the form. \\
Methods:
\code{get_button},
\code{set_button}.
\end{funcdesc}

\begin{funcdesc}{add_lightbutton}{type\, x\, y\, w\, h\, name}
Add a lightbutton object to the form. \\
Methods:
\code{get_button},
\code{set_button}.
\end{funcdesc}

\begin{funcdesc}{add_roundbutton}{type\, x\, y\, w\, h\, name}
Add a roundbutton object to the form. \\
Methods:
\code{get_button},
\code{set_button}.
\end{funcdesc}

%---

\begin{funcdesc}{add_slider}{type\, x\, y\, w\, h\, name}
Add a slider object to the form. \\
Methods:
\code{set_slider_value},
\code{get_slider_value},
\code{set_slider_bounds},
\code{get_slider_bounds},
\code{set_slider_return},
\code{set_slider_size},
\code{set_slider_precision},
\code{set_slider_step}.
\end{funcdesc}

\begin{funcdesc}{add_valslider}{type\, x\, y\, w\, h\, name}
Add a valslider object to the form. \\
Methods:
\code{set_slider_value},
\code{get_slider_value},
\code{set_slider_bounds},
\code{get_slider_bounds},
\code{set_slider_return},
\code{set_slider_size},
\code{set_slider_precision},
\code{set_slider_step}.
\end{funcdesc}

\begin{funcdesc}{add_dial}{type\, x\, y\, w\, h\, name}
Add a dial object to the form. \\
Methods:
\code{set_dial_value},
\code{get_dial_value},
\code{set_dial_bounds},
\code{get_dial_bounds}.
\end{funcdesc}

\begin{funcdesc}{add_positioner}{type\, x\, y\, w\, h\, name}
Add a positioner object to the form. \\
Methods:
\code{set_positioner_xvalue},
\code{set_positioner_yvalue},
\code{set_positioner_xbounds},
\code{set_positioner_ybounds},
\code{get_positioner_xvalue},
\code{get_positioner_yvalue},
\code{get_positioner_xbounds},
\code{get_positioner_ybounds}.
\end{funcdesc}

\begin{funcdesc}{add_counter}{type\, x\, y\, w\, h\, name}
Add a counter object to the form. \\
Methods:
\code{set_counter_value},
\code{get_counter_value},
\code{set_counter_bounds},
\code{set_counter_step},
\code{set_counter_precision},
\code{set_counter_return}.
\end{funcdesc}

%---

\begin{funcdesc}{add_input}{type\, x\, y\, w\, h\, name}
Add a input object to the form. \\
Methods:
\code{set_input},
\code{get_input},
\code{set_input_color},
\code{set_input_return}.
\end{funcdesc}

%---

\begin{funcdesc}{add_menu}{type\, x\, y\, w\, h\, name}
Add a menu object to the form. \\
Methods:
\code{set_menu},
\code{get_menu},
\code{addto_menu}.
\end{funcdesc}

\begin{funcdesc}{add_choice}{type\, x\, y\, w\, h\, name}
Add a choice object to the form. \\
Methods:
\code{set_choice},
\code{get_choice},
\code{clear_choice},
\code{addto_choice},
\code{replace_choice},
\code{delete_choice},
\code{get_choice_text},
\code{set_choice_fontsize},
\code{set_choice_fontstyle}.
\end{funcdesc}

\begin{funcdesc}{add_browser}{type\, x\, y\, w\, h\, name}
Add a browser object to the form. \\
Methods:
\code{set_browser_topline},
\code{clear_browser},
\code{add_browser_line},
\code{addto_browser},
\code{insert_browser_line},
\code{delete_browser_line},
\code{replace_browser_line},
\code{get_browser_line},
\code{load_browser},
\code{get_browser_maxline},
\code{select_browser_line},
\code{deselect_browser_line},
\code{deselect_browser},
\code{isselected_browser_line},
\code{get_browser},
\code{set_browser_fontsize},
\code{set_browser_fontstyle},
\code{set_browser_specialkey}.
\end{funcdesc}

%---

\begin{funcdesc}{add_timer}{type\, x\, y\, w\, h\, name}
Add a timer object to the form. \\
Methods:
\code{set_timer},
\code{get_timer}.
\end{funcdesc}
\end{flushleft}

Form objects have the following data attributes; see the FORMS
documentation:

\begin{tableiii}{|l|c|l|}{code}{Name}{Type}{Meaning}
  \lineiii{window}{int (read-only)}{GL window id}
  \lineiii{w}{float}{form width}
  \lineiii{h}{float}{form height}
  \lineiii{x}{float}{form x origin}
  \lineiii{y}{float}{form y origin}
  \lineiii{deactivated}{int}{nonzero if form is deactivated}
  \lineiii{visible}{int}{nonzero if form is visible}
  \lineiii{frozen}{int}{nonzero if form is frozen}
  \lineiii{doublebuf}{int}{nonzero if double buffering on}
\end{tableiii}

\subsection{FORMS Objects}

Besides methods specific to particular kinds of FORMS objects, all
FORMS objects also have the following methods:

\setindexsubitem{(FORMS object method)}
\begin{funcdesc}{set_call_back}{function\, argument}
Set the object's callback function and argument.  When the object
needs interaction, the callback function will be called with two
arguments: the object, and the callback argument.  (FORMS objects
without a callback function are returned by \code{fl.do_forms()} or
\code{fl.check_forms()} when they need interaction.)  Call this method
without arguments to remove the callback function.
\end{funcdesc}

\begin{funcdesc}{delete_object}{}
  Delete the object.
\end{funcdesc}

\begin{funcdesc}{show_object}{}
  Show the object.
\end{funcdesc}

\begin{funcdesc}{hide_object}{}
  Hide the object.
\end{funcdesc}

\begin{funcdesc}{redraw_object}{}
  Redraw the object.
\end{funcdesc}

\begin{funcdesc}{freeze_object}{}
  Freeze the object.
\end{funcdesc}

\begin{funcdesc}{unfreeze_object}{}
  Unfreeze the object.
\end{funcdesc}

%\begin{funcdesc}{handle_object}{} XXX
%\end{funcdesc}

%\begin{funcdesc}{handle_object_direct}{} XXX
%\end{funcdesc}

FORMS objects have these data attributes; see the FORMS documentation:

\begin{tableiii}{|l|c|l|}{code}{Name}{Type}{Meaning}
  \lineiii{objclass}{int (read-only)}{object class}
  \lineiii{type}{int (read-only)}{object type}
  \lineiii{boxtype}{int}{box type}
  \lineiii{x}{float}{x origin}
  \lineiii{y}{float}{y origin}
  \lineiii{w}{float}{width}
  \lineiii{h}{float}{height}
  \lineiii{col1}{int}{primary color}
  \lineiii{col2}{int}{secondary color}
  \lineiii{align}{int}{alignment}
  \lineiii{lcol}{int}{label color}
  \lineiii{lsize}{float}{label font size}
  \lineiii{label}{string}{label string}
  \lineiii{lstyle}{int}{label style}
  \lineiii{pushed}{int (read-only)}{(see FORMS docs)}
  \lineiii{focus}{int (read-only)}{(see FORMS docs)}
  \lineiii{belowmouse}{int (read-only)}{(see FORMS docs)}
  \lineiii{frozen}{int (read-only)}{(see FORMS docs)}
  \lineiii{active}{int (read-only)}{(see FORMS docs)}
  \lineiii{input}{int (read-only)}{(see FORMS docs)}
  \lineiii{visible}{int (read-only)}{(see FORMS docs)}
  \lineiii{radio}{int (read-only)}{(see FORMS docs)}
  \lineiii{automatic}{int (read-only)}{(see FORMS docs)}
\end{tableiii}

\section{Standard Module \sectcode{FL}}
\nodename{FL (uppercase)}
\label{module-FL}
\stmodindex{FL}

This module defines symbolic constants needed to use the built-in
module \code{fl} (see above); they are equivalent to those defined in
the C header file \file{<forms.h>} except that the name prefix
\samp{FL_} is omitted.  Read the module source for a complete list of
the defined names.  Suggested use:

\begin{verbatim}
import fl
from FL import *
\end{verbatim}
%
\section{Standard Module \sectcode{flp}}
\label{module-flp}
\stmodindex{flp}

This module defines functions that can read form definitions created
by the `form designer' (\code{fdesign}) program that comes with the
FORMS library (see module \code{fl} above).

For now, see the file \file{flp.doc} in the Python library source
directory for a description.

XXX A complete description should be inserted here!

\section{\module{fm} ---
         \emph{Font Manager} interface}

\declaremodule{builtin}{fm}
  \platform{IRIX}
\modulesynopsis{\emph{Font Manager} interface for SGI workstations.}


This module provides access to the IRIS \emph{Font Manager} library.
\index{Font Manager, IRIS}
\index{IRIS Font Manager}
It is available only on Silicon Graphics machines.
See also: \emph{4Sight User's Guide}, section 1, chapter 5: ``Using
the IRIS Font Manager.''

This is not yet a full interface to the IRIS Font Manager.
Among the unsupported features are: matrix operations; cache
operations; character operations (use string operations instead); some
details of font info; individual glyph metrics; and printer matching.

It supports the following operations:

\begin{funcdesc}{init}{}
Initialization function.
Calls \cfunction{fminit()}.
It is normally not necessary to call this function, since it is called
automatically the first time the \module{fm} module is imported.
\end{funcdesc}

\begin{funcdesc}{findfont}{fontname}
Return a font handle object.
Calls \code{fmfindfont(\var{fontname})}.
\end{funcdesc}

\begin{funcdesc}{enumerate}{}
Returns a list of available font names.
This is an interface to \cfunction{fmenumerate()}.
\end{funcdesc}

\begin{funcdesc}{prstr}{string}
Render a string using the current font (see the \function{setfont()} font
handle method below).
Calls \code{fmprstr(\var{string})}.
\end{funcdesc}

\begin{funcdesc}{setpath}{string}
Sets the font search path.
Calls \code{fmsetpath(\var{string})}.
(XXX Does not work!?!)
\end{funcdesc}

\begin{funcdesc}{fontpath}{}
Returns the current font search path.
\end{funcdesc}

Font handle objects support the following operations:

\setindexsubitem{(font handle method)}
\begin{funcdesc}{scalefont}{factor}
Returns a handle for a scaled version of this font.
Calls \code{fmscalefont(\var{fh}, \var{factor})}.
\end{funcdesc}

\begin{funcdesc}{setfont}{}
Makes this font the current font.
Note: the effect is undone silently when the font handle object is
deleted.
Calls \code{fmsetfont(\var{fh})}.
\end{funcdesc}

\begin{funcdesc}{getfontname}{}
Returns this font's name.
Calls \code{fmgetfontname(\var{fh})}.
\end{funcdesc}

\begin{funcdesc}{getcomment}{}
Returns the comment string associated with this font.
Raises an exception if there is none.
Calls \code{fmgetcomment(\var{fh})}.
\end{funcdesc}

\begin{funcdesc}{getfontinfo}{}
Returns a tuple giving some pertinent data about this font.
This is an interface to \code{fmgetfontinfo()}.
The returned tuple contains the following numbers:
\code{(}\var{printermatched}, \var{fixed_width}, \var{xorig},
\var{yorig}, \var{xsize}, \var{ysize}, \var{height},
\var{nglyphs}\code{)}.
\end{funcdesc}

\begin{funcdesc}{getstrwidth}{string}
Returns the width, in pixels, of \var{string} when drawn in this font.
Calls \code{fmgetstrwidth(\var{fh}, \var{string})}.
\end{funcdesc}

\section{Built-in Module \sectcode{gl}}
\label{module-gl}
\bimodindex{gl}

This module provides access to the Silicon Graphics
\emph{Graphics Library}.
It is available only on Silicon Graphics machines.

\strong{Warning:}
Some illegal calls to the GL library cause the Python interpreter to dump
core.
In particular, the use of most GL calls is unsafe before the first
window is opened.

The module is too large to document here in its entirety, but the
following should help you to get started.
The parameter conventions for the C functions are translated to Python as
follows:

\begin{itemize}
\item
All (short, long, unsigned) int values are represented by Python
integers.
\item
All float and double values are represented by Python floating point
numbers.
In most cases, Python integers are also allowed.
\item
All arrays are represented by one-dimensional Python lists.
In most cases, tuples are also allowed.
\item
\begin{sloppypar}
All string and character arguments are represented by Python strings,
for instance,
\code{winopen('Hi There!')}
and
\code{rotate(900, 'z')}.
\end{sloppypar}
\item
All (short, long, unsigned) integer arguments or return values that are
only used to specify the length of an array argument are omitted.
For example, the C call

\bcode\begin{verbatim}
lmdef(deftype, index, np, props)
\end{verbatim}\ecode
%
is translated to Python as

\bcode\begin{verbatim}
lmdef(deftype, index, props)
\end{verbatim}\ecode
%
\item
Output arguments are omitted from the argument list; they are
transmitted as function return values instead.
If more than one value must be returned, the return value is a tuple.
If the C function has both a regular return value (that is not omitted
because of the previous rule) and an output argument, the return value
comes first in the tuple.
Examples: the C call

\bcode\begin{verbatim}
getmcolor(i, &red, &green, &blue)
\end{verbatim}\ecode
%
is translated to Python as

\bcode\begin{verbatim}
red, green, blue = getmcolor(i)
\end{verbatim}\ecode
%
\end{itemize}

The following functions are non-standard or have special argument
conventions:

\renewcommand{\indexsubitem}{(in module gl)}
\begin{funcdesc}{varray}{argument}
%JHXXX the argument-argument added
Equivalent to but faster than a number of
\code{v3d()}
calls.
The \var{argument} is a list (or tuple) of points.
Each point must be a tuple of coordinates
\code{(\var{x}, \var{y}, \var{z})} or \code{(\var{x}, \var{y})}.
The points may be 2- or 3-dimensional but must all have the
same dimension.
Float and int values may be mixed however.
The points are always converted to 3D double precision points
by assuming \code{\var{z} = 0.0} if necessary (as indicated in the man page),
and for each point
\code{v3d()}
is called.
\end{funcdesc}

\begin{funcdesc}{nvarray}{}
Equivalent to but faster than a number of
\code{n3f}
and
\code{v3f}
calls.
The argument is an array (list or tuple) of pairs of normals and points.
Each pair is a tuple of a point and a normal for that point.
Each point or normal must be a tuple of coordinates
\code{(\var{x}, \var{y}, \var{z})}.
Three coordinates must be given.
Float and int values may be mixed.
For each pair,
\code{n3f()}
is called for the normal, and then
\code{v3f()}
is called for the point.
\end{funcdesc}

\begin{funcdesc}{vnarray}{}
Similar to 
\code{nvarray()}
but the pairs have the point first and the normal second.
\end{funcdesc}

\begin{funcdesc}{nurbssurface}{s_k\, t_k\, ctl\, s_ord\, t_ord\, type}
% XXX s_k[], t_k[], ctl[][]
Defines a nurbs surface.
The dimensions of
\code{\var{ctl}[][]}
are computed as follows:
\code{[len(\var{s_k}) - \var{s_ord}]},
\code{[len(\var{t_k}) - \var{t_ord}]}.
\end{funcdesc}

\begin{funcdesc}{nurbscurve}{knots\, ctlpoints\, order\, type}
Defines a nurbs curve.
The length of ctlpoints is
\code{len(\var{knots}) - \var{order}}.
\end{funcdesc}

\begin{funcdesc}{pwlcurve}{points\, type}
Defines a piecewise-linear curve.
\var{points}
is a list of points.
\var{type}
must be
\code{N_ST}.
\end{funcdesc}

\begin{funcdesc}{pick}{n}
\funcline{select}{n}
The only argument to these functions specifies the desired size of the
pick or select buffer.
\end{funcdesc}

\begin{funcdesc}{endpick}{}
\funcline{endselect}{}
These functions have no arguments.
They return a list of integers representing the used part of the
pick/select buffer.
No method is provided to detect buffer overrun.
\end{funcdesc}

Here is a tiny but complete example GL program in Python:

\bcode\begin{verbatim}
import gl, GL, time

def main():
    gl.foreground()
    gl.prefposition(500, 900, 500, 900)
    w = gl.winopen('CrissCross')
    gl.ortho2(0.0, 400.0, 0.0, 400.0)
    gl.color(GL.WHITE)
    gl.clear()
    gl.color(GL.RED)
    gl.bgnline()
    gl.v2f(0.0, 0.0)
    gl.v2f(400.0, 400.0)
    gl.endline()
    gl.bgnline()
    gl.v2f(400.0, 0.0)
    gl.v2f(0.0, 400.0)
    gl.endline()
    time.sleep(5)

main()
\end{verbatim}\ecode
%
\section{Standard Modules \sectcode{GL} and \sectcode{DEVICE}}
\nodename{GL and DEVICE}
\stmodindex{GL}
\stmodindex{DEVICE}

These modules define the constants used by the Silicon Graphics
\emph{Graphics Library}
that C programmers find in the header files
\file{<gl/gl.h>}
and
\file{<gl/device.h>}.
Read the module source files for details.

\section{Built-in Module \module{imgfile}}
\label{module-imgfile}
\bimodindex{imgfile}

The \module{imgfile} module allows Python programs to access SGI imglib image
files (also known as \file{.rgb} files).  The module is far from
complete, but is provided anyway since the functionality that there is
is enough in some cases.  Currently, colormap files are not supported.

The module defines the following variables and functions:

\begin{excdesc}{error}
This exception is raised on all errors, such as unsupported file type, etc.
\end{excdesc}

\begin{funcdesc}{getsizes}{file}
This function returns a tuple \code{(\var{x}, \var{y}, \var{z})} where
\var{x} and \var{y} are the size of the image in pixels and
\var{z} is the number of
bytes per pixel. Only 3 byte RGB pixels and 1 byte greyscale pixels
are currently supported.
\end{funcdesc}

\begin{funcdesc}{read}{file}
This function reads and decodes the image on the specified file, and
returns it as a Python string. The string has either 1 byte greyscale
pixels or 4 byte RGBA pixels. The bottom left pixel is the first in
the string. This format is suitable to pass to \function{gl.lrectwrite()},
for instance.
\end{funcdesc}

\begin{funcdesc}{readscaled}{file, x, y, filter\optional{, blur}}
This function is identical to read but it returns an image that is
scaled to the given \var{x} and \var{y} sizes. If the \var{filter} and
\var{blur} parameters are omitted scaling is done by
simply dropping or duplicating pixels, so the result will be less than
perfect, especially for computer-generated images.

Alternatively, you can specify a filter to use to smoothen the image
after scaling. The filter forms supported are \code{'impulse'},
\code{'box'}, \code{'triangle'}, \code{'quadratic'} and
\code{'gaussian'}. If a filter is specified \var{blur} is an optional
parameter specifying the blurriness of the filter. It defaults to \code{1.0}.

\code{readscaled} makes no
attempt to keep the aspect ratio correct, so that is the users'
responsibility.
\end{funcdesc}

\begin{funcdesc}{ttob}{flag}
This function sets a global flag which defines whether the scan lines
of the image are read or written from bottom to top (flag is zero,
compatible with SGI GL) or from top to bottom(flag is one,
compatible with X).  The default is zero.
\end{funcdesc}

\begin{funcdesc}{write}{file, data, x, y, z}
This function writes the RGB or greyscale data in \var{data} to image
file \var{file}. \var{x} and \var{y} give the size of the image,
\var{z} is 1 for 1 byte greyscale images or 3 for RGB images (which are
stored as 4 byte values of which only the lower three bytes are used).
These are the formats returned by \code{gl.lrectread}.
\end{funcdesc}

%\section{\module{panel} ---
         None}
\declaremodule{standard}{panel}

\modulesynopsis{None}


\strong{Please note:} The FORMS library, to which the
\code{fl}\refbimodindex{fl} module described above interfaces, is a
simpler and more accessible user interface library for use with GL
than the \code{panel} module (besides also being by a Dutch author).

This module should be used instead of the built-in module
\code{pnl}\refbimodindex{pnl}
to interface with the
\emph{Panel Library}.

The module is too large to document here in its entirety.
One interesting function:

\begin{funcdesc}{defpanellist}{filename}
Parses a panel description file containing S-expressions written by the
\emph{Panel Editor}
that accompanies the Panel Library and creates the described panels.
It returns a list of panel objects.
\end{funcdesc}

\strong{Warning:}
the Python interpreter will dump core if you don't create a GL window
before calling
\code{panel.mkpanel()}
or
\code{panel.defpanellist()}.

\section{\module{panelparser} ---
         None}
\declaremodule{standard}{panelparser}

\modulesynopsis{None}


This module defines a self-contained parser for S-expressions as output
by the Panel Editor (which is written in Scheme so it can't help writing
S-expressions).
The relevant function is
\code{panelparser.parse_file(\var{file})}
which has a file object (not a filename!) as argument and returns a list
of parsed S-expressions.
Each S-expression is converted into a Python list, with atoms converted
to Python strings and sub-expressions (recursively) to Python lists.
For more details, read the module file.
% XXXXJH should be funcdesc, I think

\section{\module{pnl} ---
         None}
\declaremodule{builtin}{pnl}

\modulesynopsis{None}


This module provides access to the
\emph{Panel Library}
built by NASA Ames\index{NASA} (to get it, send e-mail to
\code{panel-request@nas.nasa.gov}).
All access to it should be done through the standard module
\code{panel}\refstmodindex{panel},
which transparantly exports most functions from
\code{pnl}
but redefines
\code{pnl.dopanel()}.

\strong{Warning:}
the Python interpreter will dump core if you don't create a GL window
before calling
\code{pnl.mkpanel()}.

The module is too large to document here in its entirety.


\chapter{SunOS Specific Services}
\label{sunos}

The modules described in this chapter provide interfaces to features
that are unique to SunOS 5 (also known as Solaris version 2).
			% SUNOS ONLY
\section{\module{sunaudiodev} ---
         Access to Sun audio hardware.}
\declaremodule{builtin}{sunaudiodev}

\modulesynopsis{Access to Sun audio hardware.}


This module allows you to access the Sun audio interface. The Sun
audio hardware is capable of recording and playing back audio data
in u-LAW\index{u-LAW} format with a sample rate of 8K per second. A
full description can be found in the \manpage{audio}{7I} manual page.

The module defines the following variables and functions:

\begin{excdesc}{error}
This exception is raised on all errors. The argument is a string
describing what went wrong.
\end{excdesc}

\begin{funcdesc}{open}{mode}
This function opens the audio device and returns a Sun audio device
object. This object can then be used to do I/O on. The \var{mode} parameter
is one of \code{'r'} for record-only access, \code{'w'} for play-only
access, \code{'rw'} for both and \code{'control'} for access to the
control device. Since only one process is allowed to have the recorder
or player open at the same time it is a good idea to open the device
only for the activity needed. See \manpage{audio}{7I} for details.

As per the manpage, this module first looks in the environment
variable \code{AUDIODEV} for the base audio device filename.  If not
found, it falls back to \file{/dev/audio}.  The control device is
calculated by appending ``ctl'' to the base audio device.
\end{funcdesc}


\subsection{Audio Device Objects}
\label{audio-device-objects}

The audio device objects are returned by \function{open()} define the
following methods (except \code{control} objects which only provide
\method{getinfo()}, \method{setinfo()}, \method{fileno()}, and
\method{drain()}):

\begin{methoddesc}[audio device]{close}{}
This method explicitly closes the device. It is useful in situations
where deleting the object does not immediately close it since there
are other references to it. A closed device should not be used again.
\end{methoddesc}

\begin{methoddesc}[audio device]{fileno}{}
Returns the file descriptor associated with the device.  This can be
used to set up \code{SIGPOLL} notification, as described below.
\end{methoddesc}

\begin{methoddesc}[audio device]{drain}{}
This method waits until all pending output is processed and then returns.
Calling this method is often not necessary: destroying the object will
automatically close the audio device and this will do an implicit drain.
\end{methoddesc}

\begin{methoddesc}[audio device]{flush}{}
This method discards all pending output. It can be used avoid the
slow response to a user's stop request (due to buffering of up to one
second of sound).
\end{methoddesc}

\begin{methoddesc}[audio device]{getinfo}{}
This method retrieves status information like input and output volume,
etc. and returns it in the form of
an audio status object. This object has no methods but it contains a
number of attributes describing the current device status. The names
and meanings of the attributes are described in
\file{/usr/include/sun/audioio.h} and in the \manpage{audio}{7I}
manual page.  Member names
are slightly different from their \C{} counterparts: a status object is
only a single structure. Members of the \cdata{play} substructure have
\samp{o_} prepended to their name and members of the \cdata{record}
structure have \samp{i_}. So, the \C{} member \cdata{play.sample_rate} is
accessed as \member{o_sample_rate}, \cdata{record.gain} as \member{i_gain}
and \cdata{monitor_gain} plainly as \member{monitor_gain}.
\end{methoddesc}

\begin{methoddesc}[audio device]{ibufcount}{}
This method returns the number of samples that are buffered on the
recording side, i.e.\ the program will not block on a
\function{read()} call of so many samples.
\end{methoddesc}

\begin{methoddesc}[audio device]{obufcount}{}
This method returns the number of samples buffered on the playback
side. Unfortunately, this number cannot be used to determine a number
of samples that can be written without blocking since the kernel
output queue length seems to be variable.
\end{methoddesc}

\begin{methoddesc}[audio device]{read}{size}
This method reads \var{size} samples from the audio input and returns
them as a Python string. The function blocks until enough data is available.
\end{methoddesc}

\begin{methoddesc}[audio device]{setinfo}{status}
This method sets the audio device status parameters. The \var{status}
parameter is an device status object as returned by \function{getinfo()} and
possibly modified by the program.
\end{methoddesc}

\begin{methoddesc}[audio device]{write}{samples}
Write is passed a Python string containing audio samples to be played.
If there is enough buffer space free it will immediately return,
otherwise it will block.
\end{methoddesc}

There is a companion module,
\module{SUNAUDIODEV}\refstmodindex{SUNAUDIODEV}, which defines useful
symbolic constants like \constant{MIN_GAIN}, \constant{MAX_GAIN},
\constant{SPEAKER}, etc. The names of the constants are the same names
as used in the \C{} include file \code{<sun/audioio.h>}, with the
leading string \samp{AUDIO_} stripped.

The audio device supports asynchronous notification of various events,
through the SIGPOLL signal.  Here's an example of how you might enable 
this in Python:

\begin{verbatim}
def handle_sigpoll(signum, frame):
    print 'I got a SIGPOLL update'
pp
import fcntl, signal, STROPTS

signal.signal(signal.SIGPOLL, handle_sigpoll)
fcntl.ioctl(audio_obj.fileno(), STROPTS.I_SETSIG, STROPTS.S_MSG)
\end{verbatim}


\chapter{Undocumented Modules \label{undoc}}

Here's a quick listing of modules that are currently undocumented, but
that should be documented.  Feel free to contribute documentation for
them!  (Send via email to \email{python-docs@python.org}.)

The idea and original contents for this chapter were taken
from a posting by Fredrik Lundh; the specific contents of this chapter
have been substantially revised.


\section{Frameworks}

Frameworks tend to be harder to document, but are well worth the
effort spent.

\begin{description}
\item[\module{Tkinter}]
--- Interface to Tcl/Tk for graphical user interfaces; Fredrik Lundh
is working on this one!  See
\citetitle[http://www.pythonware.com/library.htm]{An Introduction to
Tkinter} at \url{http://www.pythonware.com/library.htm} for on-line
reference material.

\item[\module{Tkdnd}]
--- Drag-and-drop support for \module{Tkinter}.

\item[\module{turtle}]
--- Turtle graphics in a Tk window.

\item[\module{test}]
--- Regression testing framework.  This is used for the Python
regression test, but is useful for other Python libraries as well.
This is a package rather than a single module.
\end{description}


\section{Miscellaneous useful utilities}

Some of these are very old and/or not very robust; marked with ``hmm.''

\begin{description}
\item[\module{bdb}]
--- A generic Python debugger base class (used by pdb).

\item[\module{ihooks}]
--- Import hook support (for \refmodule{rexec}; may become obsolete).

\item[\module{smtpd}]
--- An SMTP daemon implementation which meets the minimum requirements
for \rfc{821} conformance.
\end{description}


\section{Platform specific modules}

These modules are used to implement the \refmodule{os.path} module,
and are not documented beyond this mention.  There's little need to
document these.

\begin{description}
\item[\module{dospath}]
--- Implementation of \module{os.path} on MS-DOS.

\item[\module{ntpath}]
--- Implementation on \module{os.path} on Win32, Win64, WinCE, and
OS/2 platforms.

\item[\module{posixpath}]
--- Implementation on \module{os.path} on \POSIX.
\end{description}


\section{Multimedia}

\begin{description}
\item[\module{audiodev}]
--- Platform-independent API for playing audio data.

\item[\module{sunaudio}]
--- Interpret Sun audio headers (may become obsolete or a tool/demo).

\item[\module{toaiff}]
--- Convert "arbitrary" sound files to AIFF files; should probably
become a tool or demo.  Requires the external program \program{sox}.
\end{description}


\section{Obsolete \label{obsolete-modules}}

These modules are not normally available for import; additional work
must be done to make them available.

Those which are written in Python will be installed into the directory 
\file{lib-old/} installed as part of the standard library.  To use
these, the directory must be added to \code{sys.path}, possibly using
\envvar{PYTHONPATH}.

Obsolete extension modules written in C are not built by default.
Under \UNIX, these must be enabled by uncommenting the appropriate
lines in \file{Modules/Setup} in the build tree and either rebuilding
Python if the modules are statically linked, or building and
installing the shared object if using dynamically-loaded extensions.

% XXX need Windows instructions!

\begin{description}
\item[\module{addpack}]
--- Alternate approach to packages.  Use the built-in package support
instead.

\item[\module{cmp}]
--- File comparison function.  Use the newer \refmodule{filecmp} instead.

\item[\module{cmpcache}]
--- Caching version of the obsolete \module{cmp} module.  Use the
newer \refmodule{filecmp} instead.

\item[\module{codehack}]
--- Extract function name or line number from a function
code object (these are now accessible as attributes:
\member{co.co_name}, \member{func.func_name},
\member{co.co_firstlineno}).

\item[\module{dircmp}]
--- Class to build directory diff tools on (may become a demo or tool).
\deprecated{2.0}{The \refmodule{filecmp} module replaces
\module{dircmp}.}

\item[\module{dump}]
--- Print python code that reconstructs a variable.

\item[\module{fmt}]
--- Text formatting abstractions (too slow).

\item[\module{lockfile}]
--- Wrapper around FCNTL file locking (use
\function{fcntl.lockf()}/\function{flock()} instead; see \refmodule{fcntl}).

\item[\module{newdir}]
--- New \function{dir()} function (the standard \function{dir()} is
now just as good).

\item[\module{Para}]
--- Helper for \module{fmt}.

\item[\module{poly}]
--- Polynomials.

\item[\module{regex}]
--- Emacs-style regular expression support; may still be used in some
old code (extension module).  Refer to the
\citetitle[http://www.python.org/doc/1.6/lib/module-regex.html]{Python
1.6 Documentation} for documentation.

\item[\module{regsub}]
--- Regular expression based string replacement utilities, for use
with \module{regex} (extension module).  Refer to the
\citetitle[http://www.python.org/doc/1.6/lib/module-regsub.html]{Python
1.6 Documentation} for documentation.

\item[\module{tb}]
--- Print tracebacks, with a dump of local variables (use
\function{pdb.pm()} or \refmodule{traceback} instead).

\item[\module{timing}]
--- Measure time intervals to high resolution (use
\function{time.clock()} instead).  (This is an extension module.)

\item[\module{tzparse}]
--- Parse a timezone specification (unfinished; may disappear in the
future, and does not work when the \envvar{TZ} environment variable is
not set).

\item[\module{util}]
--- Useful functions that don't fit elsewhere.

\item[\module{whatsound}]
--- Recognize sound files; use \refmodule{sndhdr} instead.

\item[\module{zmod}]
--- Compute properties of mathematical ``fields.''
\end{description}


The following modules are obsolete, but are likely to re-surface as
tools or scripts:

\begin{description}
\item[\module{find}]
--- Find files matching pattern in directory tree.

\item[\module{grep}]
--- \program{grep} implementation in Python.

\item[\module{packmail}]
--- Create a self-unpacking \UNIX{} shell archive.
\end{description}


The following modules were documented in previous versions of this
manual, but are now considered obsolete.  The source for the
documentation is still available as part of the documentation source
archive.

\begin{description}
\item[\module{ni}]
--- Import modules in ``packages.''  Basic package support is now
built in.  The built-in support is very similar to what is provided in
this module.

\item[\module{rand}]
--- Old interface to the random number generator.

\item[\module{soundex}]
--- Algorithm for collapsing names which sound similar to a shared
key.  The specific algorithm doesn't seem to match any published
algorithm.  (This is an extension module.)
\end{description}


\section{SGI-specific Extension modules}

The following are SGI specific, and may be out of touch with the
current version of reality.

\begin{description}
\item[\module{cl}]
--- Interface to the SGI compression library.

\item[\module{sv}]
--- Interface to the ``simple video'' board on SGI Indigo
(obsolete hardware).
\end{description}


%
%  The ugly "%begin{latexonly}" pseudo-environments are really just to
%  keep LaTeX2HTML quiet during the \renewcommand{} macros; they're
%  not really valuable.
%

%begin{latexonly}
\renewcommand{\indexname}{Module Index}
%end{latexonly}
\input{modlib.ind}		% Module Index

%begin{latexonly}
\renewcommand{\indexname}{Index}
%end{latexonly}
\documentclass{manual}

% NOTE: this file controls which chapters/sections of the library
% manual are actually printed.  It is easy to customize your manual
% by commenting out sections that you're not interested in.

\title{Python Library Reference}

\author{Guido van Rossum\\
	Fred L. Drake, Jr., editor}
\authoraddress{
	PythonLabs\\
	E-mail: \email{python-docs@python.org}
}

\date{\today}		% XXX update before release!
\release{2.1 alpha}		% software release, not documentation
\setshortversion{2.1}		% major.minor only for software


\makeindex			% tell \index to actually write the
				% .idx file
\makemodindex			% ... and the module index as well.


\begin{document}

\maketitle

\ifhtml
\chapter*{Front Matter\label{front}}
\fi

Copyright \copyright{} 2001-2006 Python Software Foundation.
All rights reserved.

Copyright \copyright{} 2000 BeOpen.com.
All rights reserved.

Copyright \copyright{} 1995-2000 Corporation for National Research Initiatives.
All rights reserved.

Copyright \copyright{} 1991-1995 Stichting Mathematisch Centrum.
All rights reserved.

See the end of this document for complete license and permissions
information.


\begin{abstract}

\noindent
Python is an extensible, interpreted, object-oriented programming
language.  It supports a wide range of applications, from simple text
processing scripts to interactive WWW browsers.

While the \emph{Python Reference Manual} describes the exact syntax and
semantics of the language, it does not describe the standard library
that is distributed with the language, and which greatly enhances its
immediate usability.  This library contains built-in modules (written
in C) that provide access to system functionality such as file I/O
that would otherwise be inaccessible to Python programmers, as well as
modules written in Python that provide standardized solutions for many
problems that occur in everyday programming.  Some of these modules
are explicitly designed to encourage and enhance the portability of
Python programs.

This library reference manual documents Python's standard library, as
well as many optional library modules (which may or may not be
available, depending on whether the underlying platform supports them
and on the configuration choices made at compile time).  It also
documents the standard types of the language and its built-in
functions and exceptions, many of which are not or incompletely
documented in the Reference Manual.

This manual assumes basic knowledge about the Python language.  For an
informal introduction to Python, see the \emph{Python Tutorial}; the
\emph{Python Reference Manual} remains the highest authority on
syntactic and semantic questions.  Finally, the manual entitled
\emph{Extending and Embedding the Python Interpreter} describes how to
add new extensions to Python and how to embed it in other applications.

\end{abstract}

\tableofcontents

				% Chapter title:

\chapter{Introduction}
\label{intro}

The ``Python library'' contains several different kinds of components.

It contains data types that would normally be considered part of the
``core'' of a language, such as numbers and lists.  For these types,
the Python language core defines the form of literals and places some
constraints on their semantics, but does not fully define the
semantics.  (On the other hand, the language core does define
syntactic properties like the spelling and priorities of operators.)

The library also contains built-in functions and exceptions ---
objects that can be used by all Python code without the need of an
\keyword{import} statement.  Some of these are defined by the core
language, but many are not essential for the core semantics and are
only described here.

The bulk of the library, however, consists of a collection of modules.
There are many ways to dissect this collection.  Some modules are
written in C and built in to the Python interpreter; others are
written in Python and imported in source form.  Some modules provide
interfaces that are highly specific to Python, like printing a stack
trace; some provide interfaces that are specific to particular
operating systems, like socket I/O; others provide interfaces that are
specific to a particular application domain, like the World-Wide Web.
Some modules are avaiable in all versions and ports of Python; others
are only available when the underlying system supports or requires
them; yet others are available only when a particular configuration
option was chosen at the time when Python was compiled and installed.

This manual is organized ``from the inside out'': it first describes
the built-in data types, then the built-in functions and exceptions,
and finally the modules, grouped in chapters of related modules.  The
ordering of the chapters as well as the ordering of the modules within
each chapter is roughly from most relevant to least important.

This means that if you start reading this manual from the start, and
skip to the next chapter when you get bored, you will get a reasonable
overview of the available modules and application areas that are
supported by the Python library.  Of course, you don't \emph{have} to
read it like a novel --- you can also browse the table of contents (in
front of the manual), or look for a specific function, module or term
in the index (in the back).  And finally, if you enjoy learning about
random subjects, you choose a random page number (see module
\module{rand}) and read a section or two.

Let the show begin!
		% Introduction

\chapter{Built-in Types, Exceptions and Functions}
\nodename{Built-in Objects}
\label{builtin}

Names for built-in exceptions and functions are found in a separate
symbol table.  This table is searched last when the interpreter looks
up the meaning of a name, so local and global
user-defined names can override built-in names.  Built-in types are
described together here for easy reference.\footnote{
	Most descriptions sorely lack explanations of the exceptions
	that may be raised --- this will be fixed in a future version of
	this manual.}
\indexii{built-in}{types}
\indexii{built-in}{exceptions}
\indexii{built-in}{functions}
\index{symbol table}

The tables in this chapter document the priorities of operators by
listing them in order of ascending priority (within a table) and
grouping operators that have the same priority in the same box.
Binary operators of the same priority group from left to right.
(Unary operators group from right to left, but there you have no real
choice.)  See Chapter 5 of the \emph{Python Reference Manual} for the
complete picture on operator priorities.
			% Built-in Types, Exceptions and Functions
\section{Built-in Types \label{types}}

The following sections describe the standard types that are built into
the interpreter.  These are the numeric types, sequence types, and
several others, including types themselves.  There is no explicit
Boolean type; use integers instead.
\indexii{built-in}{types}
\indexii{Boolean}{type}

Some operations are supported by several object types; in particular,
all objects can be compared, tested for truth value, and converted to
a string (with the \code{`\textrm{\ldots}`} notation).  The latter
conversion is implicitly used when an object is written by the
\keyword{print}\stindex{print} statement.


\subsection{Truth Value Testing \label{truth}}

Any object can be tested for truth value, for use in an \keyword{if} or
\keyword{while} condition or as operand of the Boolean operations below.
The following values are considered false:
\stindex{if}
\stindex{while}
\indexii{truth}{value}
\indexii{Boolean}{operations}
\index{false}

\begin{itemize}

\item	\code{None}
	\withsubitem{(Built-in object)}{\ttindex{None}}

\item	zero of any numeric type, for example, \code{0}, \code{0L},
        \code{0.0}, \code{0j}.

\item	any empty sequence, for example, \code{''}, \code{()}, \code{[]}.

\item	any empty mapping, for example, \code{\{\}}.

\item	instances of user-defined classes, if the class defines a
	\method{__nonzero__()} or \method{__len__()} method, when that
	method returns zero.\footnote{Additional information on these
special methods may be found in the \citetitle[../ref/ref.html]{Python
Reference Manual}.}

\end{itemize}

All other values are considered true --- so objects of many types are
always true.
\index{true}

Operations and built-in functions that have a Boolean result always
return \code{0} for false and \code{1} for true, unless otherwise
stated.  (Important exception: the Boolean operations
\samp{or}\opindex{or} and \samp{and}\opindex{and} always return one of
their operands.)


\subsection{Boolean Operations \label{boolean}}

These are the Boolean operations, ordered by ascending priority:
\indexii{Boolean}{operations}

\begin{tableiii}{c|l|c}{code}{Operation}{Result}{Notes}
  \lineiii{\var{x} or \var{y}}
          {if \var{x} is false, then \var{y}, else \var{x}}{(1)}
  \lineiii{\var{x} and \var{y}}
          {if \var{x} is false, then \var{x}, else \var{y}}{(1)}
  \hline
  \lineiii{not \var{x}}
          {if \var{x} is false, then \code{1}, else \code{0}}{(2)}
\end{tableiii}
\opindex{and}
\opindex{or}
\opindex{not}

\noindent
Notes:

\begin{description}

\item[(1)]
These only evaluate their second argument if needed for their outcome.

\item[(2)]
\samp{not} has a lower priority than non-Boolean operators, so
\code{not \var{a} == \var{b}} is interpreted as \code{not (\var{a} ==
\var{b})}, and \code{\var{a} == not \var{b}} is a syntax error.

\end{description}


\subsection{Comparisons \label{comparisons}}

Comparison operations are supported by all objects.  They all have the
same priority (which is higher than that of the Boolean operations).
Comparisons can be chained arbitrarily; for example, \code{\var{x} <
\var{y} <= \var{z}} is equivalent to \code{\var{x} < \var{y} and
\var{y} <= \var{z}}, except that \var{y} is evaluated only once (but
in both cases \var{z} is not evaluated at all when \code{\var{x} <
\var{y}} is found to be false).
\indexii{chaining}{comparisons}

This table summarizes the comparison operations:

\begin{tableiii}{c|l|c}{code}{Operation}{Meaning}{Notes}
  \lineiii{<}{strictly less than}{}
  \lineiii{<=}{less than or equal}{}
  \lineiii{>}{strictly greater than}{}
  \lineiii{>=}{greater than or equal}{}
  \lineiii{==}{equal}{}
  \lineiii{!=}{not equal}{(1)}
  \lineiii{<>}{not equal}{(1)}
  \lineiii{is}{object identity}{}
  \lineiii{is not}{negated object identity}{}
\end{tableiii}
\indexii{operator}{comparison}
\opindex{==} % XXX *All* others have funny characters < ! >
\opindex{is}
\opindex{is not}

\noindent
Notes:

\begin{description}

\item[(1)]
\code{<>} and \code{!=} are alternate spellings for the same operator.
(I couldn't choose between \ABC{} and C! :-)
\index{ABC language@\ABC{} language}
\index{language!ABC@\ABC{}}
\indexii{C}{language}
\code{!=} is the preferred spelling; \code{<>} is obsolescent.

\end{description}

Objects of different types, except different numeric types, never
compare equal; such objects are ordered consistently but arbitrarily
(so that sorting a heterogeneous array yields a consistent result).
Furthermore, some types (for example, file objects) support only a
degenerate notion of comparison where any two objects of that type are
unequal.  Again, such objects are ordered arbitrarily but
consistently.
\indexii{object}{numeric}
\indexii{objects}{comparing}

Instances of a class normally compare as non-equal unless the class
\withsubitem{(instance method)}{\ttindex{__cmp__()}}
defines the \method{__cmp__()} method.  Refer to the
\citetitle[../ref/customization.html]{Python Reference Manual} for
information on the use of this method to effect object comparisons.

\strong{Implementation note:} Objects of different types except
numbers are ordered by their type names; objects of the same types
that don't support proper comparison are ordered by their address.

Two more operations with the same syntactic priority,
\samp{in}\opindex{in} and \samp{not in}\opindex{not in}, are supported
only by sequence types (below).


\subsection{Numeric Types \label{typesnumeric}}

There are four numeric types: \dfn{plain integers}, \dfn{long integers}, 
\dfn{floating point numbers}, and \dfn{complex numbers}.
Plain integers (also just called \dfn{integers})
are implemented using \ctype{long} in C, which gives them at least 32
bits of precision.  Long integers have unlimited precision.  Floating
point numbers are implemented using \ctype{double} in C.  All bets on
their precision are off unless you happen to know the machine you are
working with.
\obindex{numeric}
\obindex{integer}
\obindex{long integer}
\obindex{floating point}
\obindex{complex number}
\indexii{C}{language}

Complex numbers have a real and imaginary part, which are both
implemented using \ctype{double} in C.  To extract these parts from
a complex number \var{z}, use \code{\var{z}.real} and \code{\var{z}.imag}.  

Numbers are created by numeric literals or as the result of built-in
functions and operators.  Unadorned integer literals (including hex
and octal numbers) yield plain integers.  Integer literals with an
\character{L} or \character{l} suffix yield long integers
(\character{L} is preferred because \samp{1l} looks too much like
eleven!).  Numeric literals containing a decimal point or an exponent
sign yield floating point numbers.  Appending \character{j} or
\character{J} to a numeric literal yields a complex number.
\indexii{numeric}{literals}
\indexii{integer}{literals}
\indexiii{long}{integer}{literals}
\indexii{floating point}{literals}
\indexii{complex number}{literals}
\indexii{hexadecimal}{literals}
\indexii{octal}{literals}

Python fully supports mixed arithmetic: when a binary arithmetic
operator has operands of different numeric types, the operand with the
``smaller'' type is converted to that of the other, where plain
integer is smaller than long integer is smaller than floating point is
smaller than complex.
Comparisons between numbers of mixed type use the same rule.\footnote{
	As a consequence, the list \code{[1, 2]} is considered equal
        to \code{[1.0, 2.0]}, and similar for tuples.
} The functions \function{int()}, \function{long()}, \function{float()},
and \function{complex()} can be used
to coerce numbers to a specific type.
\index{arithmetic}
\bifuncindex{int}
\bifuncindex{long}
\bifuncindex{float}
\bifuncindex{complex}

All numeric types support the following operations, sorted by
ascending priority (operations in the same box have the same
priority; all numeric operations have a higher priority than
comparison operations):

\begin{tableiii}{c|l|c}{code}{Operation}{Result}{Notes}
  \lineiii{\var{x} + \var{y}}{sum of \var{x} and \var{y}}{}
  \lineiii{\var{x} - \var{y}}{difference of \var{x} and \var{y}}{}
  \hline
  \lineiii{\var{x} * \var{y}}{product of \var{x} and \var{y}}{}
  \lineiii{\var{x} / \var{y}}{quotient of \var{x} and \var{y}}{(1)}
  \lineiii{\var{x} \%{} \var{y}}{remainder of \code{\var{x} / \var{y}}}{}
  \hline
  \lineiii{-\var{x}}{\var{x} negated}{}
  \lineiii{+\var{x}}{\var{x} unchanged}{}
  \hline
  \lineiii{abs(\var{x})}{absolute value or magnitude of \var{x}}{}
  \lineiii{int(\var{x})}{\var{x} converted to integer}{(2)}
  \lineiii{long(\var{x})}{\var{x} converted to long integer}{(2)}
  \lineiii{float(\var{x})}{\var{x} converted to floating point}{}
  \lineiii{complex(\var{re},\var{im})}{a complex number with real part \var{re}, imaginary part \var{im}.  \var{im} defaults to zero.}{}
  \lineiii{\var{c}.conjugate()}{conjugate of the complex number \var{c}}{}
  \lineiii{divmod(\var{x}, \var{y})}{the pair \code{(\var{x} / \var{y}, \var{x} \%{} \var{y})}}{(3)}
  \lineiii{pow(\var{x}, \var{y})}{\var{x} to the power \var{y}}{}
  \lineiii{\var{x} ** \var{y}}{\var{x} to the power \var{y}}{}
\end{tableiii}
\indexiii{operations on}{numeric}{types}
\withsubitem{(complex number method)}{\ttindex{conjugate()}}

\noindent
Notes:
\begin{description}

\item[(1)]
For (plain or long) integer division, the result is an integer.
The result is always rounded towards minus infinity: 1/2 is 0, 
(-1)/2 is -1, 1/(-2) is -1, and (-1)/(-2) is 0.  Note that the result
is a long integer if either operand is a long integer, regardless of
the numeric value.
\indexii{integer}{division}
\indexiii{long}{integer}{division}

\item[(2)]
Conversion from floating point to (long or plain) integer may round or
truncate as in C; see functions \function{floor()} and
\function{ceil()} in the \refmodule{math}\refbimodindex{math} module
for well-defined conversions.
\withsubitem{(in module math)}{\ttindex{floor()}\ttindex{ceil()}}
\indexii{numeric}{conversions}
\indexii{C}{language}

\item[(3)]
See section \ref{built-in-funcs}, ``Built-in Functions,'' for a full
description.

\end{description}
% XXXJH exceptions: overflow (when? what operations?) zerodivision

\subsubsection{Bit-string Operations on Integer Types \label{bitstring-ops}}
\nodename{Bit-string Operations}

Plain and long integer types support additional operations that make
sense only for bit-strings.  Negative numbers are treated as their 2's
complement value (for long integers, this assumes a sufficiently large
number of bits that no overflow occurs during the operation).

The priorities of the binary bit-wise operations are all lower than
the numeric operations and higher than the comparisons; the unary
operation \samp{\~} has the same priority as the other unary numeric
operations (\samp{+} and \samp{-}).

This table lists the bit-string operations sorted in ascending
priority (operations in the same box have the same priority):

\begin{tableiii}{c|l|c}{code}{Operation}{Result}{Notes}
  \lineiii{\var{x} | \var{y}}{bitwise \dfn{or} of \var{x} and \var{y}}{}
  \lineiii{\var{x} \^{} \var{y}}{bitwise \dfn{exclusive or} of \var{x} and \var{y}}{}
  \lineiii{\var{x} \&{} \var{y}}{bitwise \dfn{and} of \var{x} and \var{y}}{}
  \lineiii{\var{x} << \var{n}}{\var{x} shifted left by \var{n} bits}{(1), (2)}
  \lineiii{\var{x} >> \var{n}}{\var{x} shifted right by \var{n} bits}{(1), (3)}
  \hline
  \lineiii{\~\var{x}}{the bits of \var{x} inverted}{}
\end{tableiii}
\indexiii{operations on}{integer}{types}
\indexii{bit-string}{operations}
\indexii{shifting}{operations}
\indexii{masking}{operations}

\noindent
Notes:
\begin{description}
\item[(1)] Negative shift counts are illegal and cause a
\exception{ValueError} to be raised.
\item[(2)] A left shift by \var{n} bits is equivalent to
multiplication by \code{pow(2, \var{n})} without overflow check.
\item[(3)] A right shift by \var{n} bits is equivalent to
division by \code{pow(2, \var{n})} without overflow check.
\end{description}


\subsection{Sequence Types \label{typesseq}}

There are six sequence types: strings, Unicode strings, lists,
tuples, buffers, and xrange objects.

Strings literals are written in single or double quotes:
\code{'xyzzy'}, \code{"frobozz"}.  See chapter 2 of the
\citetitle[../ref/strings.html]{Python Reference Manual} for more about
string literals.  Unicode strings are much like strings, but are
specified in the syntax using a preceeding \character{u} character:
\code{u'abc'}, \code{u"def"}.  Lists are constructed with square brackets,
separating items with commas: \code{[a, b, c]}.  Tuples are
constructed by the comma operator (not within square brackets), with
or without enclosing parentheses, but an empty tuple must have the
enclosing parentheses, e.g., \code{a, b, c} or \code{()}.  A single
item tuple must have a trailing comma, e.g., \code{(d,)}.  Buffers are
not directly supported by Python syntax, but can be created by calling the
builtin function \function{buffer()}.\bifuncindex{buffer}  XRanges
objects are similar to buffers in that there is no specific syntax to
create them, but they are created using the \function{xrange()}
function.\bifuncindex{xrange}
\obindex{sequence}
\obindex{string}
\obindex{Unicode}
\obindex{buffer}
\obindex{tuple}
\obindex{list}
\obindex{xrange}

Sequence types support the following operations.  The \samp{in} and
\samp{not in} operations have the same priorities as the comparison
operations.  The \samp{+} and \samp{*} operations have the same
priority as the corresponding numeric operations.\footnote{They must
have since the parser can't tell the type of the operands.}

This table lists the sequence operations sorted in ascending priority
(operations in the same box have the same priority).  In the table,
\var{s} and \var{t} are sequences of the same type; \var{n}, \var{i}
and \var{j} are integers:

\begin{tableiii}{c|l|c}{code}{Operation}{Result}{Notes}
  \lineiii{\var{x} in \var{s}}{\code{1} if an item of \var{s} is equal to \var{x}, else \code{0}}{}
  \lineiii{\var{x} not in \var{s}}{\code{0} if an item of \var{s} is
equal to \var{x}, else \code{1}}{}
  \hline
  \lineiii{\var{s} + \var{t}}{the concatenation of \var{s} and \var{t}}{}
  \lineiii{\var{s} * \var{n}\textrm{,} \var{n} * \var{s}}{\var{n} copies of \var{s} concatenated}{(1)}
  \hline
  \lineiii{\var{s}[\var{i}]}{\var{i}'th item of \var{s}, origin 0}{(2)}
  \lineiii{\var{s}[\var{i}:\var{j}]}{slice of \var{s} from \var{i} to \var{j}}{(2), (3)}
  \hline
  \lineiii{len(\var{s})}{length of \var{s}}{}
  \lineiii{min(\var{s})}{smallest item of \var{s}}{}
  \lineiii{max(\var{s})}{largest item of \var{s}}{}
\end{tableiii}
\indexiii{operations on}{sequence}{types}
\bifuncindex{len}
\bifuncindex{min}
\bifuncindex{max}
\indexii{concatenation}{operation}
\indexii{repetition}{operation}
\indexii{subscript}{operation}
\indexii{slice}{operation}
\opindex{in}
\opindex{not in}

\noindent
Notes:

\begin{description}
\item[(1)] Values of \var{n} less than \code{0} are treated as
  \code{0} (which yields an empty sequence of the same type as
  \var{s}).

\item[(2)] If \var{i} or \var{j} is negative, the index is relative to
  the end of the string, i.e., \code{len(\var{s}) + \var{i}} or
  \code{len(\var{s}) + \var{j}} is substituted.  But note that \code{-0} is
  still \code{0}.
  
\item[(3)] The slice of \var{s} from \var{i} to \var{j} is defined as
  the sequence of items with index \var{k} such that \code{\var{i} <=
  \var{k} < \var{j}}.  If \var{i} or \var{j} is greater than
  \code{len(\var{s})}, use \code{len(\var{s})}.  If \var{i} is omitted,
  use \code{0}.  If \var{j} is omitted, use \code{len(\var{s})}.  If
  \var{i} is greater than or equal to \var{j}, the slice is empty.
\end{description}


\subsubsection{String Methods \label{string-methods}}

These are the string methods which both 8-bit strings and Unicode
objects support:

\begin{methoddesc}[string]{capitalize}{}
Return a copy of the string with only its first character capitalized.
\end{methoddesc}

\begin{methoddesc}[string]{center}{width}
Return centered in a string of length \var{width}. Padding is done
using spaces.
\end{methoddesc}

\begin{methoddesc}[string]{count}{sub\optional{, start\optional{, end}}}
Return the number of occurrences of substring \var{sub} in string
S\code{[\var{start}:\var{end}]}.  Optional arguments \var{start} and
\var{end} are interpreted as in slice notation.
\end{methoddesc}

\begin{methoddesc}[string]{encode}{\optional{encoding\optional{,errors}}}
Return an encoded version of the string.  Default encoding is the current
default string encoding.  \var{errors} may be given to set a different
error handling scheme.  The default for \var{errors} is
\code{'strict'}, meaning that encoding errors raise a
\exception{ValueError}.  Other possible values are \code{'ignore'} and
\code{'replace'}.
\versionadded{2.0}
\end{methoddesc}

\begin{methoddesc}[string]{endswith}{suffix\optional{, start\optional{, end}}}
Return true if the string ends with the specified \var{suffix},
otherwise return false.  With optional \var{start}, test beginning at
that position.  With optional \var{end}, stop comparing at that position.
\end{methoddesc}

\begin{methoddesc}[string]{expandtabs}{\optional{tabsize}}
Return a copy of the string where all tab characters are expanded
using spaces.  If \var{tabsize} is not given, a tab size of \code{8}
characters is assumed.
\end{methoddesc}

\begin{methoddesc}[string]{find}{sub\optional{, start\optional{, end}}}
Return the lowest index in the string where substring \var{sub} is
found, such that \var{sub} is contained in the range [\var{start},
\var{end}).  Optional arguments \var{start} and \var{end} are
interpreted as in slice notation.  Return \code{-1} if \var{sub} is
not found.
\end{methoddesc}

\begin{methoddesc}[string]{index}{sub\optional{, start\optional{, end}}}
Like \method{find()}, but raise \exception{ValueError} when the
substring is not found.
\end{methoddesc}

\begin{methoddesc}[string]{isalnum}{}
Return true if all characters in the string are alphanumeric and there
is at least one character, false otherwise.
\end{methoddesc}

\begin{methoddesc}[string]{isalpha}{}
Return true if all characters in the string are alphabetic and there
is at least one character, false otherwise.
\end{methoddesc}

\begin{methoddesc}[string]{isdigit}{}
Return true if there are only digit characters, false otherwise.
\end{methoddesc}

\begin{methoddesc}[string]{islower}{}
Return true if all cased characters in the string are lowercase and
there is at least one cased character, false otherwise.
\end{methoddesc}

\begin{methoddesc}[string]{isspace}{}
Return true if there are only whitespace characters in the string and
the string is not empty, false otherwise.
\end{methoddesc}

\begin{methoddesc}[string]{istitle}{}
Return true if the string is a titlecased string, i.e.\ uppercase
characters may only follow uncased characters and lowercase characters
only cased ones.  Return false otherwise.
\end{methoddesc}

\begin{methoddesc}[string]{isupper}{}
Return true if all cased characters in the string are uppercase and
there is at least one cased character, false otherwise.
\end{methoddesc}

\begin{methoddesc}[string]{join}{seq}
Return a string which is the concatenation of the strings in the
sequence \var{seq}.  The separator between elements is the string
providing this method.
\end{methoddesc}

\begin{methoddesc}[string]{ljust}{width}
Return the string left justified in a string of length \var{width}.
Padding is done using spaces.  The original string is returned if
\var{width} is less than \code{len(\var{s})}.
\end{methoddesc}

\begin{methoddesc}[string]{lower}{}
Return a copy of the string converted to lowercase.
\end{methoddesc}

\begin{methoddesc}[string]{lstrip}{}
Return a copy of the string with leading whitespace removed.
\end{methoddesc}

\begin{methoddesc}[string]{replace}{old, new\optional{, maxsplit}}
Return a copy of the string with all occurrences of substring
\var{old} replaced by \var{new}.  If the optional argument
\var{maxsplit} is given, only the first \var{maxsplit} occurrences are
replaced.
\end{methoddesc}

\begin{methoddesc}[string]{rfind}{sub \optional{,start \optional{,end}}}
Return the highest index in the string where substring \var{sub} is
found, such that \var{sub} is contained within s[start,end].  Optional
arguments \var{start} and \var{end} are interpreted as in slice
notation.  Return \code{-1} on failure.
\end{methoddesc}

\begin{methoddesc}[string]{rindex}{sub\optional{, start\optional{, end}}}
Like \method{rfind()} but raises \exception{ValueError} when the
substring \var{sub} is not found.
\end{methoddesc}

\begin{methoddesc}[string]{rjust}{width}
Return the string right justified in a string of length \var{width}.
Padding is done using spaces.  The original string is returned if
\var{width} is less than \code{len(\var{s})}.
\end{methoddesc}

\begin{methoddesc}[string]{rstrip}{}
Return a copy of the string with trailing whitespace removed.
\end{methoddesc}

\begin{methoddesc}[string]{split}{\optional{sep \optional{,maxsplit}}}
Return a list of the words in the string, using \var{sep} as the
delimiter string.  If \var{maxsplit} is given, at most \var{maxsplit}
splits are done.  If \var{sep} is not specified or \code{None}, any
whitespace string is a separator.
\end{methoddesc}

\begin{methoddesc}[string]{splitlines}{\optional{keepends}}
Return a list of the lines in the string, breaking at line
boundaries.  Line breaks are not included in the resulting list unless
\var{keepends} is given and true.
\end{methoddesc}

\begin{methoddesc}[string]{startswith}{prefix\optional{, start\optional{, end}}}
Return true if string starts with the \var{prefix}, otherwise
return false.  With optional \var{start}, test string beginning at
that position.  With optional \var{end}, stop comparing string at that
position.
\end{methoddesc}

\begin{methoddesc}[string]{strip}{}
Return a copy of the string with leading and trailing whitespace
removed.
\end{methoddesc}

\begin{methoddesc}[string]{swapcase}{}
Return a copy of the string with uppercase characters converted to
lowercase and vice versa.
\end{methoddesc}

\begin{methoddesc}[string]{title}{}
Return a titlecased version of, i.e.\ words start with uppercase
characters, all remaining cased characters are lowercase.
\end{methoddesc}

\begin{methoddesc}[string]{translate}{table\optional{, deletechars}}
Return a copy of the string where all characters occurring in the
optional argument \var{deletechars} are removed, and the remaining
characters have been mapped through the given translation table, which
must be a string of length 256.
\end{methoddesc}

\begin{methoddesc}[string]{upper}{}
Return a copy of the string converted to uppercase.
\end{methoddesc}


\subsubsection{String Formatting Operations \label{typesseq-strings}}

\index{formatting, string}
\index{string!formatting}
\index{printf-style formatting}
\index{sprintf-style formatting}

String and Unicode objects have one unique built-in operation: the
\code{\%} operator (modulo).  Given \code{\var{format} \%
\var{values}} (where \var{format} is a string or Unicode object),
\code{\%} conversion specifications in \var{format} are replaced with
zero or more elements of \var{values}.  The effect is similar to the
using \cfunction{sprintf()} in the C language.  If \var{format} is a
Unicode object, or if any of the objects being converted using the
\code{\%s} conversion are Unicode objects, the result will be a
Unicode object as well.

If \var{format} requires a single argument, \var{values} may be a
single non-tuple object. \footnote{A tuple object in this case should
  be a singleton.}  Otherwise, \var{values} must be a tuple with
exactly the number of items specified by the format string, or a
single mapping object (for example, a dictionary).

A conversion specifier contains two or more characters and has the
following components, which must occur in this order:

\begin{enumerate}
  \item  The \character{\%} character, which marks the start of the
         specifier.
  \item  Mapping key value (optional), consisting of an identifier in
         parentheses (for example, \code{(somename)}).
  \item  Conversion flags (optional), which affect the result of some
         conversion types.
  \item  Minimum field width (optional).  If specified as an
         \character{*} (asterisk), the actual width is read from the
         next element of the tuple in \var{values}, and the object to
         convert comes after the minimum field width and optional
         precision.
  \item  Precision (optional), given as a \character{.} (dot) followed
         by the precision.  If specified as \character{*} (an
         asterisk), the actual width is read from the next element of
         the tuple in \var{values}, and the value to convert comes after
         the precision.
  \item  Length modifier (optional).
  \item  Conversion type.
\end{enumerate}

If the right argument is a dictionary (or any kind of mapping), then
the formats in the string \emph{must} have a parenthesized key into
that dictionary inserted immediately after the \character{\%}
character, and each format formats the corresponding entry from the
mapping.  For example:

\begin{verbatim}
>>> count = 2
>>> language = 'Python'
>>> print '%(language)s has %(count)03d quote types.' % vars()
Python has 002 quote types.
\end{verbatim}

In this case no \code{*} specifiers may occur in a format (since they
require a sequential parameter list).

The conversion flag characters are:

\begin{tableii}{c|l}{character}{Flag}{Meaning}
  \lineii{\#}{The value conversion will use the ``alternate form''
              (where defined below).}
  \lineii{0}{The conversion will be zero padded.}
  \lineii{-}{The converted value is left adjusted (overrides
             \character{-}).}
  \lineii{{~}}{(a space) A blank should be left before a positive number
             (or empty string) produced by a signed conversion.}
  \lineii{+}{A sign character (\character{+} or \character{-}) will
             precede the conversion (overrides a "space" flag).}
\end{tableii}

The length modifier may be \code{h}, \code{l}, and \code{L} may be
present, but are ignored as they are not necessary for Python.

The conversion types are:

\begin{tableii}{c|l}{character}{Conversion}{Meaning}
  \lineii{d}{Signed integer decimal.}
  \lineii{i}{Signed integer decimal.}
  \lineii{o}{Unsigned octal.}
  \lineii{u}{Unsigned decimal.}
  \lineii{x}{Unsigned hexidecimal (lowercase).}
  \lineii{X}{Unsigned hexidecimal (uppercase).}
  \lineii{e}{Floating point exponential format (lowercase).}
  \lineii{E}{Floating point exponential format (uppercase).}
  \lineii{f}{Floating point decimal format.}
  \lineii{F}{Floating point decimal format.}
  \lineii{g}{Same as \character{e} if exponent is greater than -4 or
             less than precision, \character{f} otherwise.}
  \lineii{G}{Same as \character{E} if exponent is greater than -4 or
             less than precision, \character{F} otherwise.}
  \lineii{c}{Single character (accepts integer or single character
             string).}
  \lineii{r}{String (converts any python object using
             \function{repr()}).}
  \lineii{s}{String (converts any python object using
             \function{str()}).}
  \lineii{\%}{No argument is converted, results in a \character{\%}
              character in the result.  (The complete specification is
              \code{\%\%}.)}
\end{tableii}

% XXX Examples?


Since Python strings have an explicit length, \code{\%s} conversions
do not assume that \code{'\e0'} is the end of the string.

For safety reasons, floating point precisions are clipped to 50;
\code{\%f} conversions for numbers whose absolute value is over 1e25
are replaced by \code{\%g} conversions.\footnote{
  These numbers are fairly arbitrary.  They are intended to
  avoid printing endless strings of meaningless digits without hampering
  correct use and without having to know the exact precision of floating
  point values on a particular machine.
}  All other errors raise exceptions.

Additional string operations are defined in standard module
\refmodule{string} and in built-in module \refmodule{re}.
\refstmodindex{string}
\refstmodindex{re}


\subsubsection{XRange Type \label{typesseq-xrange}}

The xrange\obindex{xrange} type is an immutable sequence which is
commonly used for looping.  The advantage of the xrange type is that an
xrange object will always take the same amount of memory, no matter the
size of the range it represents.  There are no consistent performance
advantages.

XRange objects behave like tuples, and offer a single method:

\begin{methoddesc}[xrange]{tolist}{}
  Return a list object which represents the same values as the xrange
  object.
\end{methoddesc}


\subsubsection{Mutable Sequence Types \label{typesseq-mutable}}

List objects support additional operations that allow in-place
modification of the object.
These operations would be supported by other mutable sequence types
(when added to the language) as well.
Strings and tuples are immutable sequence types and such objects cannot
be modified once created.
The following operations are defined on mutable sequence types (where
\var{x} is an arbitrary object):
\indexiii{mutable}{sequence}{types}
\obindex{list}

\begin{tableiii}{c|l|c}{code}{Operation}{Result}{Notes}
  \lineiii{\var{s}[\var{i}] = \var{x}}
	{item \var{i} of \var{s} is replaced by \var{x}}{}
  \lineiii{\var{s}[\var{i}:\var{j}] = \var{t}}
  	{slice of \var{s} from \var{i} to \var{j} is replaced by \var{t}}{}
  \lineiii{del \var{s}[\var{i}:\var{j}]}
	{same as \code{\var{s}[\var{i}:\var{j}] = []}}{}
  \lineiii{\var{s}.append(\var{x})}
	{same as \code{\var{s}[len(\var{s}):len(\var{s})] = [\var{x}]}}{(1)}
  \lineiii{\var{s}.extend(\var{x})}
        {same as \code{\var{s}[len(\var{s}):len(\var{s})] = \var{x}}}{(2)}
  \lineiii{\var{s}.count(\var{x})}
    {return number of \var{i}'s for which \code{\var{s}[\var{i}] == \var{x}}}{}
  \lineiii{\var{s}.index(\var{x})}
    {return smallest \var{i} such that \code{\var{s}[\var{i}] == \var{x}}}{(3)}
  \lineiii{\var{s}.insert(\var{i}, \var{x})}
	{same as \code{\var{s}[\var{i}:\var{i}] = [\var{x}]}
	  if \code{\var{i} >= 0}}{}
  \lineiii{\var{s}.pop(\optional{\var{i}})}
    {same as \code{\var{x} = \var{s}[\var{i}]; del \var{s}[\var{i}]; return \var{x}}}{(4)}
  \lineiii{\var{s}.remove(\var{x})}
	{same as \code{del \var{s}[\var{s}.index(\var{x})]}}{(3)}
  \lineiii{\var{s}.reverse()}
	{reverses the items of \var{s} in place}{(5)}
  \lineiii{\var{s}.sort(\optional{\var{cmpfunc}})}
	{sort the items of \var{s} in place}{(5), (6)}
\end{tableiii}
\indexiv{operations on}{mutable}{sequence}{types}
\indexiii{operations on}{sequence}{types}
\indexiii{operations on}{list}{type}
\indexii{subscript}{assignment}
\indexii{slice}{assignment}
\stindex{del}
\withsubitem{(list method)}{
  \ttindex{append()}\ttindex{extend()}\ttindex{count()}\ttindex{index()}
  \ttindex{insert()}\ttindex{pop()}\ttindex{remove()}\ttindex{reverse()}
  \ttindex{sort()}}
\noindent
Notes:
\begin{description}
\item[(1)] The C implementation of Python has historically accepted
  multiple parameters and implicitly joined them into a tuple; this
  no longer works in Python 2.0.  Use of this misfeature has been
  deprecated since Python 1.4.

\item[(2)] Raises an exception when \var{x} is not a list object.  The 
  \method{extend()} method is experimental and not supported by
  mutable sequence types other than lists.

\item[(3)] Raises \exception{ValueError} when \var{x} is not found in
  \var{s}.

\item[(4)] The \method{pop()} method is only supported by the list and
  array types.  The optional argument \var{i} defaults to \code{-1},
  so that by default the last item is removed and returned.

\item[(5)] The \method{sort()} and \method{reverse()} methods modify the
  list in place for economy of space when sorting or reversing a large
  list.  They don't return the sorted or reversed list to remind you
  of this side effect.

\item[(6)] The \method{sort()} method takes an optional argument
  specifying a comparison function of two arguments (list items) which
  should return \code{-1}, \code{0} or \code{1} depending on whether
  the first argument is considered smaller than, equal to, or larger
  than the second argument.  Note that this slows the sorting process
  down considerably; e.g. to sort a list in reverse order it is much
  faster to use calls to the methods \method{sort()} and
  \method{reverse()} than to use the built-in function
  \function{sort()} with a comparison function that reverses the
  ordering of the elements.
\end{description}


\subsection{Mapping Types \label{typesmapping}}
\obindex{mapping}
\obindex{dictionary}

A \dfn{mapping} object maps values of one type (the key type) to
arbitrary objects.  Mappings are mutable objects.  There is currently
only one standard mapping type, the \dfn{dictionary}.  A dictionary's keys are
almost arbitrary values.  The only types of values not acceptable as
keys are values containing lists or dictionaries or other mutable
types that are compared by value rather than by object identity.
Numeric types used for keys obey the normal rules for numeric
comparison: if two numbers compare equal (e.g. \code{1} and
\code{1.0}) then they can be used interchangeably to index the same
dictionary entry.

Dictionaries are created by placing a comma-separated list of
\code{\var{key}: \var{value}} pairs within braces, for example:
\code{\{'jack': 4098, 'sjoerd': 4127\}} or
\code{\{4098: 'jack', 4127: 'sjoerd'\}}.

The following operations are defined on mappings (where \var{a} and
\var{b} are mappings, \var{k} is a key, and \var{v} and \var{x} are
arbitrary objects):
\indexiii{operations on}{mapping}{types}
\indexiii{operations on}{dictionary}{type}
\stindex{del}
\bifuncindex{len}
\withsubitem{(dictionary method)}{
  \ttindex{clear()}
  \ttindex{copy()}
  \ttindex{has_key()}
  \ttindex{items()}
  \ttindex{keys()}
  \ttindex{update()}
  \ttindex{values()}
  \ttindex{get()}}

\begin{tableiii}{c|l|c}{code}{Operation}{Result}{Notes}
  \lineiii{len(\var{a})}{the number of items in \var{a}}{}
  \lineiii{\var{a}[\var{k}]}{the item of \var{a} with key \var{k}}{(1)}
  \lineiii{\var{a}[\var{k}] = \var{v}}
          {set \code{\var{a}[\var{k}]} to \var{v}}
          {}
  \lineiii{del \var{a}[\var{k}]}
          {remove \code{\var{a}[\var{k}]} from \var{a}}
          {(1)}
  \lineiii{\var{a}.clear()}{remove all items from \code{a}}{}
  \lineiii{\var{a}.copy()}{a (shallow) copy of \code{a}}{}
  \lineiii{\var{k} \code{in} \var{a}}
          {\code{1} if \var{a} has a key \var{k}, else \code{0}}
          {}
  \lineiii{\var{k} not in \var{a}}
          {\code{0} if \var{a} has a key \var{k}, else \code{1}}
          {}
  \lineiii{\var{a}.has_key(\var{k})}
          {Equivalent to \var{k} \code{in} \var{a}}
          {}
  \lineiii{\var{a}.items()}
          {a copy of \var{a}'s list of (\var{key}, \var{value}) pairs}
          {(2)}
  \lineiii{\var{a}.keys()}{a copy of \var{a}'s list of keys}{(2)}
  \lineiii{\var{a}.update(\var{b})}
          {\code{for k in \var{b}.keys(): \var{a}[k] = \var{b}[k]}}
          {(3)}
  \lineiii{\var{a}.values()}{a copy of \var{a}'s list of values}{(2)}
  \lineiii{\var{a}.get(\var{k}\optional{, \var{x}})}
          {\code{\var{a}[\var{k}]} if \code{\var{k} in \var{a}}},
           else \var{x}}
          {(4)}
  \lineiii{\var{a}.setdefault(\var{k}\optional{, \var{x}})}
          {\code{\var{a}[\var{k}]} if \code{\var{k} in \var{a}}},
           else \var{x} (also setting it)}
          {(5)}
  \lineiii{\var{a}.popitem()}
          {remove and return an arbitrary (\var{key}, \var{value}) pair}
          {(6)}
\end{tableiii}

\noindent
Notes:
\begin{description}
\item[(1)] Raises a \exception{KeyError} exception if \var{k} is not
in the map.

\item[(2)] Keys and values are listed in random order.  If
\method{keys()} and \method{values()} are called with no intervening
modifications to the dictionary, the two lists will directly
correspond.  This allows the creation of \code{(\var{value},
\var{key})} pairs using \function{map()}: \samp{pairs = map(None,
\var{a}.values(), \var{a}.keys())}.

\item[(3)] \var{b} must be of the same type as \var{a}.

\item[(4)] Never raises an exception if \var{k} is not in the map,
instead it returns \var{x}.  \var{x} is optional; when \var{x} is not
provided and \var{k} is not in the map, \code{None} is returned.

\item[(5)] \function{setdefault()} is like \function{get()}, except
that if \var{k} is missing, \var{x} is both returned and inserted into
the dictionary as the value of \var{k}.

\item[(6)] \function{popitem()} is useful to destructively iterate
over a dictionary, as often used in set algorithms.
\end{description}


\subsection{Other Built-in Types \label{typesother}}

The interpreter supports several other kinds of objects.
Most of these support only one or two operations.


\subsubsection{Modules \label{typesmodules}}

The only special operation on a module is attribute access:
\code{\var{m}.\var{name}}, where \var{m} is a module and \var{name}
accesses a name defined in \var{m}'s symbol table.  Module attributes
can be assigned to.  (Note that the \keyword{import} statement is not,
strictly speaking, an operation on a module object; \code{import
\var{foo}} does not require a module object named \var{foo} to exist,
rather it requires an (external) \emph{definition} for a module named
\var{foo} somewhere.)

A special member of every module is \member{__dict__}.
This is the dictionary containing the module's symbol table.
Modifying this dictionary will actually change the module's symbol
table, but direct assignment to the \member{__dict__} attribute is not
possible (i.e., you can write \code{\var{m}.__dict__['a'] = 1}, which
defines \code{\var{m}.a} to be \code{1}, but you can't write
\code{\var{m}.__dict__ = \{\}}.

Modules built into the interpreter are written like this:
\code{<module 'sys' (built-in)>}.  If loaded from a file, they are
written as \code{<module 'os' from
'/usr/local/lib/python\shortversion/os.pyc'>}.


\subsubsection{Classes and Class Instances \label{typesobjects}}
\nodename{Classes and Instances}

See chapters 3 and 7 of the \citetitle[../ref/ref.html]{Python
Reference Manual} for these.


\subsubsection{Functions \label{typesfunctions}}

Function objects are created by function definitions.  The only
operation on a function object is to call it:
\code{\var{func}(\var{argument-list})}.

There are really two flavors of function objects: built-in functions
and user-defined functions.  Both support the same operation (to call
the function), but the implementation is different, hence the
different object types.

The implementation adds two special read-only attributes:
\code{\var{f}.func_code} is a function's \dfn{code
object}\obindex{code} (see below) and \code{\var{f}.func_globals} is
the dictionary used as the function's global namespace (this is the
same as \code{\var{m}.__dict__} where \var{m} is the module in which
the function \var{f} was defined).

Function objects also support getting and setting arbitrary
attributes, which can be used to, e.g. attach metadata to functions.
Regular attribute dot-notation is used to get and set such
attributes. \emph{Note that the current implementation only supports
function attributes on functions written in Python.  Function
attributes on built-ins may be supported in the future.}

Functions have another special attribute \code{\var{f}.__dict__}
(a.k.a. \code{\var{f}.func_dict}) which contains the namespace used to
support function attributes.  \code{__dict__} can be accessed
directly, set to a dictionary object, or \code{None}.  It can also be
deleted (but the following two lines are equivalent):

\begin{verbatim}
del func.__dict__
func.__dict__ = None
\end{verbatim}

\subsubsection{Methods \label{typesmethods}}
\obindex{method}

Methods are functions that are called using the attribute notation.
There are two flavors: built-in methods (such as \method{append()} on
lists) and class instance methods.  Built-in methods are described
with the types that support them.

The implementation adds two special read-only attributes to class
instance methods: \code{\var{m}.im_self} is the object on which the
method operates, and \code{\var{m}.im_func} is the function
implementing the method.  Calling \code{\var{m}(\var{arg-1},
\var{arg-2}, \textrm{\ldots}, \var{arg-n})} is completely equivalent to
calling \code{\var{m}.im_func(\var{m}.im_self, \var{arg-1},
\var{arg-2}, \textrm{\ldots}, \var{arg-n})}.

Class instance methods are either \emph{bound} or \emph{unbound},
referring to whether the method was accessed through an instance or a
class, respectively.  When a method is unbound, its \code{im_self}
attribute will be \code{None} and if called, an explicit \code{self}
object must be passed as the first argument.  In this case,
\code{self} must be an instance of the unbound method's class (or a
subclass of that class), otherwise a \code{TypeError} is raised.

Like function objects, methods objects support getting
arbitrary attributes.  However, since method attributes are actually
stored on the underlying function object (i.e. \code{meth.im_func}),
setting method attributes on either bound or unbound methods is
disallowed.  Attempting to set a method attribute results in a
\code{TypeError} being raised.  In order to set a method attribute,
you need to explicitly set it on the underlying function object:

\begin{verbatim}
class C:
    def method(self):
        pass

c = C()
c.method.im_func.whoami = 'my name is c'
\end{verbatim}

See the \citetitle[../ref/ref.html]{Python Reference Manual} for more
information.


\subsubsection{Code Objects \label{bltin-code-objects}}
\obindex{code}

Code objects are used by the implementation to represent
``pseudo-compiled'' executable Python code such as a function body.
They differ from function objects because they don't contain a
reference to their global execution environment.  Code objects are
returned by the built-in \function{compile()} function and can be
extracted from function objects through their \member{func_code}
attribute.
\bifuncindex{compile}
\withsubitem{(function object attribute)}{\ttindex{func_code}}

A code object can be executed or evaluated by passing it (instead of a
source string) to the \keyword{exec} statement or the built-in
\function{eval()} function.
\stindex{exec}
\bifuncindex{eval}

See the \citetitle[../ref/ref.html]{Python Reference Manual} for more
information.


\subsubsection{Type Objects \label{bltin-type-objects}}

Type objects represent the various object types.  An object's type is
accessed by the built-in function \function{type()}.  There are no special
operations on types.  The standard module \module{types} defines names
for all standard built-in types.
\bifuncindex{type}
\refstmodindex{types}

Types are written like this: \code{<type 'int'>}.


\subsubsection{The Null Object \label{bltin-null-object}}

This object is returned by functions that don't explicitly return a
value.  It supports no special operations.  There is exactly one null
object, named \code{None} (a built-in name).

It is written as \code{None}.


\subsubsection{The Ellipsis Object \label{bltin-ellipsis-object}}

This object is used by extended slice notation (see the
\citetitle[../ref/ref.html]{Python Reference Manual}).  It supports no
special operations.  There is exactly one ellipsis object, named
\constant{Ellipsis} (a built-in name).

It is written as \code{Ellipsis}.


\subsubsection{File Objects\obindex{file}
               \label{bltin-file-objects}}

File objects are implemented using C's \code{stdio} package and can be
created with the built-in function
\function{open()}\bifuncindex{open} described in section 
\ref{built-in-funcs}, ``Built-in Functions.''  They are also returned
by some other built-in functions and methods, e.g.,
\function{os.popen()} and \function{os.fdopen()} and the
\method{makefile()} method of socket objects.
\refstmodindex{os}
\refbimodindex{socket}

When a file operation fails for an I/O-related reason, the exception
\exception{IOError} is raised.  This includes situations where the
operation is not defined for some reason, like \method{seek()} on a tty
device or writing a file opened for reading.

Files have the following methods:


\begin{methoddesc}[file]{close}{}
  Close the file.  A closed file cannot be read or written anymore.
  Any operation which requires that the file be open will raise a
  \exception{ValueError} after the file has been closed.  Calling
  \method{close()} more than once is allowed.
\end{methoddesc}

\begin{methoddesc}[file]{flush}{}
  Flush the internal buffer, like \code{stdio}'s
  \cfunction{fflush()}.  This may be a no-op on some file-like
  objects.
\end{methoddesc}

\begin{methoddesc}[file]{isatty}{}
  Return true if the file is connected to a tty(-like) device, else
  false.  \strong{Note:} If a file-like object is not associated
  with a real file, this method should \emph{not} be implemented.
\end{methoddesc}

\begin{methoddesc}[file]{fileno}{}
  \index{file descriptor}
  \index{descriptor, file}
  Return the integer ``file descriptor'' that is used by the
  underlying implementation to request I/O operations from the
  operating system.  This can be useful for other, lower level
  interfaces that use file descriptors, e.g.\ module
  \refmodule{fcntl}\refbimodindex{fcntl} or \function{os.read()} and
  friends.  \strong{Note:} File-like objects which do not have a real
  file descriptor should \emph{not} provide this method!
\end{methoddesc}

\begin{methoddesc}[file]{read}{\optional{size}}
  Read at most \var{size} bytes from the file (less if the read hits
  \EOF{} before obtaining \var{size} bytes).  If the \var{size}
  argument is negative or omitted, read all data until \EOF{} is
  reached.  The bytes are returned as a string object.  An empty
  string is returned when \EOF{} is encountered immediately.  (For
  certain files, like ttys, it makes sense to continue reading after
  an \EOF{} is hit.)  Note that this method may call the underlying
  C function \cfunction{fread()} more than once in an effort to
  acquire as close to \var{size} bytes as possible.
\end{methoddesc}

\begin{methoddesc}[file]{readline}{\optional{size}}
  Read one entire line from the file.  A trailing newline character is
  kept in the string\footnote{
	The advantage of leaving the newline on is that an empty string 
	can be returned to mean \EOF{} without being ambiguous.  Another 
	advantage is that (in cases where it might matter, e.g. if you 
	want to make an exact copy of a file while scanning its lines) 
	you can tell whether the last line of a file ended in a newline
	or not (yes this happens!).
  } (but may be absent when a file ends with an
  incomplete line).  If the \var{size} argument is present and
  non-negative, it is a maximum byte count (including the trailing
  newline) and an incomplete line may be returned.
  An empty string is returned when \EOF{} is hit
  immediately.  Note: Unlike \code{stdio}'s \cfunction{fgets()}, the
  returned string contains null characters (\code{'\e 0'}) if they
  occurred in the input.
\end{methoddesc}

\begin{methoddesc}[file]{readlines}{\optional{sizehint}}
  Read until \EOF{} using \method{readline()} and return a list containing
  the lines thus read.  If the optional \var{sizehint} argument is
  present, instead of reading up to \EOF{}, whole lines totalling
  approximately \var{sizehint} bytes (possibly after rounding up to an
  internal buffer size) are read.  Objects implementing a file-like
  interface may choose to ignore \var{sizehint} if it cannot be
  implemented, or cannot be implemented efficiently.
\end{methoddesc}

\begin{methoddesc}[file]{xreadlines}{}
  Equivalent to \function{xreadlines.xreadlines(file)}.\refstmodindex{xreadlines}
\end{methoddesc}

\begin{methoddesc}[file]{seek}{offset\optional{, whence}}
  Set the file's current position, like \code{stdio}'s \cfunction{fseek()}.
  The \var{whence} argument is optional and defaults to \code{0}
  (absolute file positioning); other values are \code{1} (seek
  relative to the current position) and \code{2} (seek relative to the
  file's end).  There is no return value.  Note that if the file is
  opened for appending (mode \code{'a'} or \code{'a+'}), any
  \method{seek()} operations will be undone at the next write.  If the
  file is only opened for writing in append mode (mode \code{'a'}),
  this method is essentially a no-op, but it remains useful for files
  opened in append mode with reading enabled (mode \code{'a+'}).
\end{methoddesc}

\begin{methoddesc}[file]{tell}{}
  Return the file's current position, like \code{stdio}'s
  \cfunction{ftell()}.
\end{methoddesc}

\begin{methoddesc}[file]{truncate}{\optional{size}}
  Truncate the file's size.  If the optional \var{size} argument
  present, the file is truncated to (at most) that size.  The size
  defaults to the current position.  Availability of this function
  depends on the operating system version (for example, not all
  \UNIX{} versions support this operation).
\end{methoddesc}

\begin{methoddesc}[file]{write}{str}
  Write a string to the file.  There is no return value.  Note: Due to
  buffering, the string may not actually show up in the file until
  the \method{flush()} or \method{close()} method is called.
\end{methoddesc}

\begin{methoddesc}[file]{writelines}{list}
  Write a list of strings to the file.  There is no return value.
  (The name is intended to match \method{readlines()};
  \method{writelines()} does not add line separators.)
\end{methoddesc}

\begin{methoddesc}[file]{xreadlines}{}
  Equivalent to
  \function{xreadlines.xreadlines(\var{file})}.\refstmodindex{xreadlines}
  (See the \refmodule{xreadlines} module for more information.)
\end{methoddesc}


File objects also offer a number of other interesting attributes.
These are not required for file-like objects, but should be
implemented if they make sense for the particular object.

\begin{memberdesc}[file]{closed}
Boolean indicating the current state of the file object.  This is a
read-only attribute; the \method{close()} method changes the value.
It may not be available on all file-like objects.
\end{memberdesc}

\begin{memberdesc}[file]{mode}
The I/O mode for the file.  If the file was created using the
\function{open()} built-in function, this will be the value of the
\var{mode} parameter.  This is a read-only attribute and may not be
present on all file-like objects.
\end{memberdesc}

\begin{memberdesc}[file]{name}
If the file object was created using \function{open()}, the name of
the file.  Otherwise, some string that indicates the source of the
file object, of the form \samp{<\mbox{\ldots}>}.  This is a read-only
attribute and may not be present on all file-like objects.
\end{memberdesc}

\begin{memberdesc}[file]{softspace}
Boolean that indicates whether a space character needs to be printed
before another value when using the \keyword{print} statement.
Classes that are trying to simulate a file object should also have a
writable \member{softspace} attribute, which should be initialized to
zero.  This will be automatic for most classes implemented in Python
(care may be needed for objects that override attribute access); types
implemented in C will have to provide a writable
\member{softspace} attribute.
\strong{Note:}  This attribute is not used to control the
\keyword{print} statement, but to allow the implementation of
\keyword{print} to keep track of its internal state.
\end{memberdesc}


\subsubsection{Internal Objects \label{typesinternal}}

See the \citetitle[../ref/ref.html]{Python Reference Manual} for this
information.  It describes stack frame objects, traceback objects, and
slice objects.


\subsection{Special Attributes \label{specialattrs}}

The implementation adds a few special read-only attributes to several
object types, where they are relevant:

\begin{memberdesc}[object]{__dict__}
A dictionary or other mapping object used to store an
object's (writable) attributes.
\end{memberdesc}

\begin{memberdesc}[object]{__methods__}
List of the methods of many built-in object types,
e.g., \code{[].__methods__} yields
\code{['append', 'count', 'index', 'insert', 'pop', 'remove',
'reverse', 'sort']}.  This usually does not need to be explicitly
provided by the object.
\end{memberdesc}

\begin{memberdesc}[object]{__members__}
Similar to \member{__methods__}, but lists data attributes.  This
usually does not need to be explicitly provided by the object.
\end{memberdesc}

\begin{memberdesc}[instance]{__class__}
The class to which a class instance belongs.
\end{memberdesc}

\begin{memberdesc}[class]{__bases__}
The tuple of base classes of a class object.
\end{memberdesc}

\section{Built-in Exceptions}

\declaremodule{standard}{exceptions}
\modulesynopsis{Standard exceptions classes.}


Exceptions can be class objects or string objects.  Though most
exceptions have been string objects in past versions of Python, in
Python 1.5 and newer versions, all standard exceptions have been
converted to class objects, and users are encouraged to do the same.
The exceptions are defined in the module \module{exceptions}.  This
module never needs to be imported explicitly: the exceptions are
provided in the built-in namespace.

Two distinct string objects with the same value are considered different
exceptions.  This is done to force programmers to use exception names
rather than their string value when specifying exception handlers.
The string value of all built-in exceptions is their name, but this is
not a requirement for user-defined exceptions or exceptions defined by
library modules.

For class exceptions, in a \keyword{try}\stindex{try} statement with
an \keyword{except}\stindex{except} clause that mentions a particular
class, that clause also handles any exception classes derived from
that class (but not exception classes from which \emph{it} is
derived).  Two exception classes that are not related via subclassing
are never equivalent, even if they have the same name.

The built-in exceptions listed below can be generated by the
interpreter or built-in functions.  Except where mentioned, they have
an ``associated value'' indicating the detailed cause of the error.
This may be a string or a tuple containing several items of
information (e.g., an error code and a string explaining the code).
The associated value is the second argument to the
\keyword{raise}\stindex{raise} statement.  For string exceptions, the
associated value itself will be stored in the variable named as the
second argument of the \keyword{except} clause (if any).  For class
exceptions, that variable receives the exception instance.  If the
exception class is derived from the standard root class
\exception{Exception}, the associated value is present as the
exception instance's \member{args} attribute, and possibly on other
attributes as well.

User code can raise built-in exceptions.  This can be used to test an
exception handler or to report an error condition ``just like'' the
situation in which the interpreter raises the same exception; but
beware that there is nothing to prevent user code from raising an
inappropriate error.

\setindexsubitem{(built-in exception base class)}

The following exceptions are only used as base classes for other
exceptions.

\begin{excdesc}{Exception}
The root class for exceptions.  All built-in exceptions are derived
from this class.  All user-defined exceptions should also be derived
from this class, but this is not (yet) enforced.  The \function{str()}
function, when applied to an instance of this class (or most derived
classes) returns the string value of the argument or arguments, or an
empty string if no arguments were given to the constructor.  When used
as a sequence, this accesses the arguments given to the constructor
(handy for backward compatibility with old code).  The arguments are
also available on the instance's \member{args} attribute, as a tuple.
\end{excdesc}

\begin{excdesc}{StandardError}
The base class for all built-in exceptions except
\exception{StopIteration} and \exception{SystemExit}.
\exception{StandardError} itself is derived from the root class
\exception{Exception}.
\end{excdesc}

\begin{excdesc}{ArithmeticError}
The base class for those built-in exceptions that are raised for
various arithmetic errors: \exception{OverflowError},
\exception{ZeroDivisionError}, \exception{FloatingPointError}.
\end{excdesc}

\begin{excdesc}{LookupError}
The base class for the exceptions that are raised when a key or
index used on a mapping or sequence is invalid: \exception{IndexError},
\exception{KeyError}.  This can be raised directly by
\function{sys.setdefaultencoding()}.
\end{excdesc}

\begin{excdesc}{EnvironmentError}
The base class for exceptions that
can occur outside the Python system: \exception{IOError},
\exception{OSError}.  When exceptions of this type are created with a
2-tuple, the first item is available on the instance's \member{errno}
attribute (it is assumed to be an error number), and the second item
is available on the \member{strerror} attribute (it is usually the
associated error message).  The tuple itself is also available on the
\member{args} attribute.
\versionadded{1.5.2}

When an \exception{EnvironmentError} exception is instantiated with a
3-tuple, the first two items are available as above, while the third
item is available on the \member{filename} attribute.  However, for
backwards compatibility, the \member{args} attribute contains only a
2-tuple of the first two constructor arguments.

The \member{filename} attribute is \code{None} when this exception is
created with other than 3 arguments.  The \member{errno} and
\member{strerror} attributes are also \code{None} when the instance was
created with other than 2 or 3 arguments.  In this last case,
\member{args} contains the verbatim constructor arguments as a tuple.
\end{excdesc}


\setindexsubitem{(built-in exception)}

The following exceptions are the exceptions that are actually raised.

\begin{excdesc}{AssertionError}
\stindex{assert}
Raised when an \keyword{assert} statement fails.
\end{excdesc}

\begin{excdesc}{AttributeError}
% xref to attribute reference?
  Raised when an attribute reference or assignment fails.  (When an
  object does not support attribute references or attribute assignments
  at all, \exception{TypeError} is raised.)
\end{excdesc}

\begin{excdesc}{EOFError}
% XXXJH xrefs here
  Raised when one of the built-in functions (\function{input()} or
  \function{raw_input()}) hits an end-of-file condition (\EOF{}) without
  reading any data.
% XXXJH xrefs here
  (N.B.: the \method{read()} and \method{readline()} methods of file
  objects return an empty string when they hit \EOF{}.)
\end{excdesc}

\begin{excdesc}{FloatingPointError}
  Raised when a floating point operation fails.  This exception is
  always defined, but can only be raised when Python is configured
  with the \longprogramopt{with-fpectl} option, or the
  \constant{WANT_SIGFPE_HANDLER} symbol is defined in the
  \file{config.h} file.
\end{excdesc}

\begin{excdesc}{IOError}
% XXXJH xrefs here
  Raised when an I/O operation (such as a \keyword{print} statement,
  the built-in \function{open()} function or a method of a file
  object) fails for an I/O-related reason, e.g., ``file not found'' or
  ``disk full''.

  This class is derived from \exception{EnvironmentError}.  See the
  discussion above for more information on exception instance
  attributes.
\end{excdesc}

\begin{excdesc}{ImportError}
% XXXJH xref to import statement?
  Raised when an \keyword{import} statement fails to find the module
  definition or when a \code{from \textrm{\ldots} import} fails to find a
  name that is to be imported.
\end{excdesc}

\begin{excdesc}{IndexError}
% XXXJH xref to sequences
  Raised when a sequence subscript is out of range.  (Slice indices are
  silently truncated to fall in the allowed range; if an index is not a
  plain integer, \exception{TypeError} is raised.)
\end{excdesc}

\begin{excdesc}{KeyError}
% XXXJH xref to mapping objects?
  Raised when a mapping (dictionary) key is not found in the set of
  existing keys.
\end{excdesc}

\begin{excdesc}{KeyboardInterrupt}
  Raised when the user hits the interrupt key (normally
  \kbd{Control-C} or \kbd{Delete}).  During execution, a check for
  interrupts is made regularly.
% XXXJH xrefs here
  Interrupts typed when a built-in function \function{input()} or
  \function{raw_input()}) is waiting for input also raise this
  exception.
\end{excdesc}

\begin{excdesc}{MemoryError}
  Raised when an operation runs out of memory but the situation may
  still be rescued (by deleting some objects).  The associated value is
  a string indicating what kind of (internal) operation ran out of memory.
  Note that because of the underlying memory management architecture
  (C's \cfunction{malloc()} function), the interpreter may not
  always be able to completely recover from this situation; it
  nevertheless raises an exception so that a stack traceback can be
  printed, in case a run-away program was the cause.
\end{excdesc}

\begin{excdesc}{NameError}
  Raised when a local or global name is not found.  This applies only
  to unqualified names.  The associated value is the name that could
  not be found.
\end{excdesc}

\begin{excdesc}{NotImplementedError}
  This exception is derived from \exception{RuntimeError}.  In user
  defined base classes, abstract methods should raise this exception
  when they require derived classes to override the method.
  \versionadded{1.5.2}
\end{excdesc}

\begin{excdesc}{OSError}
  %xref for os module
  This class is derived from \exception{EnvironmentError} and is used
  primarily as the \refmodule{os} module's \code{os.error} exception.
  See \exception{EnvironmentError} above for a description of the
  possible associated values.
  \versionadded{1.5.2}
\end{excdesc}

\begin{excdesc}{OverflowError}
% XXXJH reference to long's and/or int's?
  Raised when the result of an arithmetic operation is too large to be
  represented.  This cannot occur for long integers (which would rather
  raise \exception{MemoryError} than give up).  Because of the lack of
  standardization of floating point exception handling in C, most
  floating point operations also aren't checked.  For plain integers,
  all operations that can overflow are checked except left shift, where
  typical applications prefer to drop bits than raise an exception.
\end{excdesc}

\begin{excdesc}{RuntimeError}
  Raised when an error is detected that doesn't fall in any of the
  other categories.  The associated value is a string indicating what
  precisely went wrong.  (This exception is mostly a relic from a
  previous version of the interpreter; it is not used very much any
  more.)
\end{excdesc}

\begin{excdesc}{StopIteration}
  Raised by an iterator's \method{next()} method to signal that there
  are no further values.
  This is derived from \exception{Exception} rather than
  \exception{StandardError}, since this is not considered an error in
  its normal application.
  \versionadded{2.2}
\end{excdesc}

\begin{excdesc}{SyntaxError}
% XXXJH xref to these functions?
  Raised when the parser encounters a syntax error.  This may occur in
  an \keyword{import} statement, in an \keyword{exec} statement, in a call
  to the built-in function \function{eval()} or \function{input()}, or
  when reading the initial script or standard input (also
  interactively).

When class exceptions are used, instances of this class have
atttributes \member{filename}, \member{lineno}, \member{offset} and
\member{text} for easier access to the details; for string exceptions,
the associated value is usually a tuple of the form
\code{(message, (filename, lineno, offset, text))}.
For class exceptions, \function{str()} returns only the message.
\end{excdesc}

\begin{excdesc}{SystemError}
  Raised when the interpreter finds an internal error, but the
  situation does not look so serious to cause it to abandon all hope.
  The associated value is a string indicating what went wrong (in
  low-level terms).
  
  You should report this to the author or maintainer of your Python
  interpreter.  Be sure to report the version string of the Python
  interpreter (\code{sys.version}; it is also printed at the start of an
  interactive Python session), the exact error message (the exception's
  associated value) and if possible the source of the program that
  triggered the error.
\end{excdesc}

\begin{excdesc}{SystemExit}
% XXXJH xref to module sys?
  This exception is raised by the \function{sys.exit()} function.  When it
  is not handled, the Python interpreter exits; no stack traceback is
  printed.  If the associated value is a plain integer, it specifies the
  system exit status (passed to C's \cfunction{exit()} function); if it is
  \code{None}, the exit status is zero; if it has another type (such as
  a string), the object's value is printed and the exit status is one.

  Instances have an attribute \member{code} which is set to the
  proposed exit status or error message (defaulting to \code{None}).
  Also, this exception derives directly from \exception{Exception} and
  not \exception{StandardError}, since it is not technically an error.

  A call to \function{sys.exit()} is translated into an exception so that
  clean-up handlers (\keyword{finally} clauses of \keyword{try} statements)
  can be executed, and so that a debugger can execute a script without
  running the risk of losing control.  The \function{os._exit()} function
  can be used if it is absolutely positively necessary to exit
  immediately (for example, in the child process after a call to
  \function{fork()}).
\end{excdesc}

\begin{excdesc}{TypeError}
  Raised when a built-in operation or function is applied to an object
  of inappropriate type.  The associated value is a string giving
  details about the type mismatch.
\end{excdesc}

\begin{excdesc}{UnboundLocalError}
  Raised when a reference is made to a local variable in a function or
  method, but no value has been bound to that variable.  This is a
  subclass of \exception{NameError}.
\versionadded{2.0}
\end{excdesc}

\begin{excdesc}{UnicodeError}
  Raised when a Unicode-related encoding or decoding error occurs.  It
  is a subclass of \exception{ValueError}.
\versionadded{2.0}
\end{excdesc}

\begin{excdesc}{ValueError}
  Raised when a built-in operation or function receives an argument
  that has the right type but an inappropriate value, and the
  situation is not described by a more precise exception such as
  \exception{IndexError}.
\end{excdesc}

\begin{excdesc}{WindowsError}
  Raised when a Windows-specific error occurs or when the error number
  does not correspond to an \cdata{errno} value.  The
  \member{errno} and \member{strerror} values are created from the
  return values of the \cfunction{GetLastError()} and
  \cfunction{FormatMessage()} functions from the Windows Platform API.
  This is a subclass of \exception{OSError}.
\versionadded{2.0}
\end{excdesc}

\begin{excdesc}{ZeroDivisionError}
  Raised when the second argument of a division or modulo operation is
  zero.  The associated value is a string indicating the type of the
  operands and the operation.
\end{excdesc}


\setindexsubitem{(built-in warning category)}

The following exceptions are used as warning categories; see the
\module{warnings} module for more information.

\begin{excdesc}{Warning}
Base class for warning categories.
\end{excdesc}

\begin{excdesc}{UserWarning}
Base class for warnings generated by user code.
\end{excdesc}

\begin{excdesc}{DeprecationWarning}
Base class for warnings about deprecated features.
\end{excdesc}

\begin{excdesc}{SyntaxWarning}
Base class for warnings about dubious syntax
\end{excdesc}

\begin{excdesc}{RuntimeWarning}
Base class for warnings about dubious runtime behavior.
\end{excdesc}

\section{Built-in Functions \label{built-in-funcs}}

The Python interpreter has a number of functions built into it that
are always available.  They are listed here in alphabetical order.


\setindexsubitem{(built-in function)}

\begin{funcdesc}{__import__}{name\optional{, globals\optional{, locals\optional{, fromlist}}}}
  This function is invoked by the \keyword{import}\stindex{import}
  statement.  It mainly exists so that you can replace it with another
  function that has a compatible interface, in order to change the
  semantics of the \keyword{import} statement.  For examples of why
  and how you would do this, see the standard library modules
  \module{ihooks}\refstmodindex{ihooks} and
  \refmodule{rexec}\refstmodindex{rexec}.  See also the built-in
  module \refmodule{imp}\refbimodindex{imp}, which defines some useful
  operations out of which you can build your own
  \function{__import__()} function.

  For example, the statement \samp{import spam} results in the
  following call: \code{__import__('spam',} \code{globals(),}
  \code{locals(), [])}; the statement \samp{from spam.ham import eggs}
  results in \samp{__import__('spam.ham', globals(), locals(),
  ['eggs'])}.  Note that even though \code{locals()} and
  \code{['eggs']} are passed in as arguments, the
  \function{__import__()} function does not set the local variable
  named \code{eggs}; this is done by subsequent code that is generated
  for the import statement.  (In fact, the standard implementation
  does not use its \var{locals} argument at all, and uses its
  \var{globals} only to determine the package context of the
  \keyword{import} statement.)

  When the \var{name} variable is of the form \code{package.module},
  normally, the top-level package (the name up till the first dot) is
  returned, \emph{not} the module named by \var{name}.  However, when
  a non-empty \var{fromlist} argument is given, the module named by
  \var{name} is returned.  This is done for compatibility with the
  bytecode generated for the different kinds of import statement; when
  using \samp{import spam.ham.eggs}, the top-level package \module{spam}
  must be placed in the importing namespace, but when using \samp{from
  spam.ham import eggs}, the \code{spam.ham} subpackage must be used
  to find the \code{eggs} variable.  As a workaround for this
  behavior, use \function{getattr()} to extract the desired
  components.  For example, you could define the following helper:

\begin{verbatim}
def my_import(name):
    mod = __import__(name)
    components = name.split('.')
    for comp in components[1:]:
        mod = getattr(mod, comp)
    return mod
\end{verbatim}
\end{funcdesc}

\begin{funcdesc}{abs}{x}
  Return the absolute value of a number.  The argument may be a plain
  or long integer or a floating point number.  If the argument is a
  complex number, its magnitude is returned.
\end{funcdesc}

\begin{funcdesc}{basestring}{}
  This abstract type is the superclass for \class{str} and \class{unicode}.
  It cannot be called or instantiated, but it can be used to test whether
  an object is an instance of \class{str} or \class{unicode}.
  \code{isinstance(obj, basestring)} is equivalent to
  \code{isinstance(obj, (str, unicode))}.
  \versionadded{2.3}
\end{funcdesc}

\begin{funcdesc}{bool}{\optional{x}}
  Convert a value to a Boolean, using the standard truth testing
  procedure.  If \var{x} is false or omitted, this returns
  \constant{False}; otherwise it returns \constant{True}.
  \class{bool} is also a class, which is a subclass of \class{int}.
  Class \class{bool} cannot be subclassed further.  Its only instances
  are \constant{False} and \constant{True}.

  \indexii{Boolean}{type}
  \versionadded{2.2.1}
  \versionchanged[If no argument is given, this function returns 
                  \constant{False}]{2.3}
\end{funcdesc}

\begin{funcdesc}{callable}{object}
  Return true if the \var{object} argument appears callable, false if
  not.  If this returns true, it is still possible that a call fails,
  but if it is false, calling \var{object} will never succeed.  Note
  that classes are callable (calling a class returns a new instance);
  class instances are callable if they have a \method{__call__()}
  method.
\end{funcdesc}

\begin{funcdesc}{chr}{i}
  Return a string of one character whose \ASCII{} code is the integer
  \var{i}.  For example, \code{chr(97)} returns the string \code{'a'}.
  This is the inverse of \function{ord()}.  The argument must be in
  the range [0..255], inclusive; \exception{ValueError} will be raised
  if \var{i} is outside that range.
\end{funcdesc}

\begin{funcdesc}{classmethod}{function}
  Return a class method for \var{function}.

  A class method receives the class as implicit first argument,
  just like an instance method receives the instance.
  To declare a class method, use this idiom:

\begin{verbatim}
class C:
    def f(cls, arg1, arg2, ...): ...
    f = classmethod(f)
\end{verbatim}

  It can be called either on the class (such as \code{C.f()}) or on an
  instance (such as \code{C().f()}).  The instance is ignored except for
  its class.
  If a class method is called for a derived class, the derived class
  object is passed as the implied first argument.

  Class methods are different than \Cpp{} or Java static methods.
  If you want those, see \function{staticmethod()} in this section.
  \versionadded{2.2}
\end{funcdesc}

\begin{funcdesc}{cmp}{x, y}
  Compare the two objects \var{x} and \var{y} and return an integer
  according to the outcome.  The return value is negative if \code{\var{x}
  < \var{y}}, zero if \code{\var{x} == \var{y}} and strictly positive if
  \code{\var{x} > \var{y}}.
\end{funcdesc}

\begin{funcdesc}{compile}{string, filename, kind\optional{,
                          flags\optional{, dont_inherit}}}
  Compile the \var{string} into a code object.  Code objects can be
  executed by an \keyword{exec} statement or evaluated by a call to
  \function{eval()}.  The \var{filename} argument should
  give the file from which the code was read; pass some recognizable value
  if it wasn't read from a file (\code{'<string>'} is commonly used).
  The \var{kind} argument specifies what kind of code must be
  compiled; it can be \code{'exec'} if \var{string} consists of a
  sequence of statements, \code{'eval'} if it consists of a single
  expression, or \code{'single'} if it consists of a single
  interactive statement (in the latter case, expression statements
  that evaluate to something else than \code{None} will printed).

  When compiling multi-line statements, two caveats apply: line
  endings must be represented by a single newline character
  (\code{'\e n'}), and the input must be terminated by at least one
  newline character.  If line endings are represented by
  \code{'\e r\e n'}, use the string \method{replace()} method to
  change them into \code{'\e n'}.

  The optional arguments \var{flags} and \var{dont_inherit}
  (which are new in Python 2.2) control which future statements (see
  \pep{236}) affect the compilation of \var{string}.  If neither is
  present (or both are zero) the code is compiled with those future
  statements that are in effect in the code that is calling compile.
  If the \var{flags} argument is given and \var{dont_inherit} is not
  (or is zero) then the future statements specified by the \var{flags}
  argument are used in addition to those that would be used anyway.
  If \var{dont_inherit} is a non-zero integer then the \var{flags}
  argument is it -- the future statements in effect around the call to
  compile are ignored.

  Future statemants are specified by bits which can be bitwise or-ed
  together to specify multiple statements.  The bitfield required to
  specify a given feature can be found as the \member{compiler_flag}
  attribute on the \class{_Feature} instance in the
  \module{__future__} module.
\end{funcdesc}

\begin{funcdesc}{complex}{\optional{real\optional{, imag}}}
  Create a complex number with the value \var{real} + \var{imag}*j or
  convert a string or number to a complex number.  If the first
  parameter is a string, it will be interpreted as a complex number
  and the function must be called without a second parameter.  The
  second parameter can never be a string.
  Each argument may be any numeric type (including complex).
  If \var{imag} is omitted, it defaults to zero and the function
  serves as a numeric conversion function like \function{int()},
  \function{long()} and \function{float()}.  If both arguments
  are omitted, returns \code{0j}.
\end{funcdesc}

\begin{funcdesc}{delattr}{object, name}
  This is a relative of \function{setattr()}.  The arguments are an
  object and a string.  The string must be the name
  of one of the object's attributes.  The function deletes
  the named attribute, provided the object allows it.  For example,
  \code{delattr(\var{x}, '\var{foobar}')} is equivalent to
  \code{del \var{x}.\var{foobar}}.
\end{funcdesc}

\begin{funcdesc}{dict}{\optional{mapping-or-sequence}}
  Return a new dictionary initialized from an optional positional
  argument or from a set of keyword arguments.
  If no arguments are given, return a new empty dictionary.
  If the positional argument is a mapping object, return a dictionary
  mapping the same keys to the same values as does the mapping object.
  Otherwise the positional argument must be a sequence, a container that
  supports iteration, or an iterator object.  The elements of the argument
  must each also be of one of those kinds, and each must in turn contain
  exactly two objects.  The first is used as a key in the new dictionary,
  and the second as the key's value.  If a given key is seen more than
  once, the last value associated with it is retained in the new
  dictionary.

  If keyword arguments are given, the keywords themselves with their
  associated values are added as items to the dictionary. If a key
  is specified both in the positional argument and as a keyword argument,
  the value associated with the keyword is retained in the dictionary.
  For example, these all return a dictionary equal to
  \code{\{"one": 2, "two": 3\}}:

  \begin{itemize}
    \item \code{dict(\{'one': 2, 'two': 3\})}
    \item \code{dict(\{'one': 2, 'two': 3\}.items())}
    \item \code{dict(\{'one': 2, 'two': 3\}.iteritems())}
    \item \code{dict(zip(('one', 'two'), (2, 3)))}
    \item \code{dict([['two', 3], ['one', 2]])}
    \item \code{dict(one=2, two=3)}
    \item \code{dict([(['one', 'two'][i-2], i) for i in (2, 3)])}
  \end{itemize}

  \versionadded{2.2}
  \versionchanged[Support for building a dictionary from keyword
                  arguments added]{2.3}
\end{funcdesc}

\begin{funcdesc}{dir}{\optional{object}}
  Without arguments, return the list of names in the current local
  symbol table.  With an argument, attempts to return a list of valid
  attributes for that object.  This information is gleaned from the
  object's \member{__dict__} attribute, if defined, and from the class
  or type object.  The list is not necessarily complete.
  If the object is a module object, the list contains the names of the
  module's attributes.
  If the object is a type or class object,
  the list contains the names of its attributes,
  and recursively of the attributes of its bases.
  Otherwise, the list contains the object's attributes' names,
  the names of its class's attributes,
  and recursively of the attributes of its class's base classes.
  The resulting list is sorted alphabetically.
  For example:

\begin{verbatim}
>>> import struct
>>> dir()
['__builtins__', '__doc__', '__name__', 'struct']
>>> dir(struct)
['__doc__', '__name__', 'calcsize', 'error', 'pack', 'unpack']
\end{verbatim}

  \note{Because \function{dir()} is supplied primarily as a convenience
  for use at an interactive prompt,
  it tries to supply an interesting set of names more than it tries to
  supply a rigorously or consistently defined set of names,
  and its detailed behavior may change across releases.}
\end{funcdesc}

\begin{funcdesc}{divmod}{a, b}
  Take two (non complex) numbers as arguments and return a pair of numbers
  consisting of their quotient and remainder when using long division.  With
  mixed operand types, the rules for binary arithmetic operators apply.  For
  plain and long integers, the result is the same as
  \code{(\var{a} / \var{b}, \var{a} \%{} \var{b})}.
  For floating point numbers the result is \code{(\var{q}, \var{a} \%{}
  \var{b})}, where \var{q} is usually \code{math.floor(\var{a} /
  \var{b})} but may be 1 less than that.  In any case \code{\var{q} *
  \var{b} + \var{a} \%{} \var{b}} is very close to \var{a}, if
  \code{\var{a} \%{} \var{b}} is non-zero it has the same sign as
  \var{b}, and \code{0 <= abs(\var{a} \%{} \var{b}) < abs(\var{b})}.

  \versionchanged[Using \function{divmod()} with complex numbers is
                  deprecated]{2.3}
\end{funcdesc}

\begin{funcdesc}{enumerate}{iterable}
  Return an enumerate object. \var{iterable} must be a sequence, an
  iterator, or some other object which supports iteration.  The
  \method{next()} method of the iterator returned by
  \function{enumerate()} returns a tuple containing a count (from
  zero) and the corresponding value obtained from iterating over
  \var{iterable}.  \function{enumerate()} is useful for obtaining an
  indexed series: \code{(0, seq[0])}, \code{(1, seq[1])}, \code{(2,
  seq[2])}, \ldots.
  \versionadded{2.3}
\end{funcdesc}

\begin{funcdesc}{eval}{expression\optional{, globals\optional{, locals}}}
  The arguments are a string and two optional dictionaries.  The
  \var{expression} argument is parsed and evaluated as a Python
  expression (technically speaking, a condition list) using the
  \var{globals} and \var{locals} dictionaries as global and local name
  space.  If the \var{globals} dictionary is present and lacks
  '__builtins__', the current globals are copied into \var{globals} before
  \var{expression} is parsed.  This means that \var{expression}
  normally has full access to the standard
  \refmodule[builtin]{__builtin__} module and restricted environments
  are propagated.  If the \var{locals} dictionary is omitted it defaults to
  the \var{globals} dictionary.  If both dictionaries are omitted, the
  expression is executed in the environment where \keyword{eval} is
  called.  The return value is the result of the evaluated expression.
  Syntax errors are reported as exceptions.  Example:

\begin{verbatim}
>>> x = 1
>>> print eval('x+1')
2
\end{verbatim}

  This function can also be used to execute arbitrary code objects
  (such as those created by \function{compile()}).  In this case pass
  a code object instead of a string.  The code object must have been
  compiled passing \code{'eval'} as the \var{kind} argument.

  Hints: dynamic execution of statements is supported by the
  \keyword{exec} statement.  Execution of statements from a file is
  supported by the \function{execfile()} function.  The
  \function{globals()} and \function{locals()} functions returns the
  current global and local dictionary, respectively, which may be
  useful to pass around for use by \function{eval()} or
  \function{execfile()}.
\end{funcdesc}

\begin{funcdesc}{execfile}{filename\optional{, globals\optional{, locals}}}
  This function is similar to the
  \keyword{exec} statement, but parses a file instead of a string.  It
  is different from the \keyword{import} statement in that it does not
  use the module administration --- it reads the file unconditionally
  and does not create a new module.\footnote{It is used relatively
  rarely so does not warrant being made into a statement.}

  The arguments are a file name and two optional dictionaries.  The
  file is parsed and evaluated as a sequence of Python statements
  (similarly to a module) using the \var{globals} and \var{locals}
  dictionaries as global and local namespace.  If the \var{locals}
  dictionary is omitted it defaults to the \var{globals} dictionary.
  If both dictionaries are omitted, the expression is executed in the
  environment where \function{execfile()} is called.  The return value is
  \code{None}.

  \warning{The default \var{locals} act as described for function
  \function{locals()} below:  modifications to the default \var{locals}
  dictionary should not be attempted.  Pass an explicit \var{locals}
  dictionary if you need to see effects of the code on \var{locals} after
  function \function{execfile()} returns.  \function{execfile()} cannot
  be used reliably to modify a function's locals.}
\end{funcdesc}

\begin{funcdesc}{file}{filename\optional{, mode\optional{, bufsize}}}
  Return a new file object (described in
  section~\ref{bltin-file-objects}, ``\ulink{File
  Objects}{bltin-file-objects.html}'').
  The first two arguments are the same as for \code{stdio}'s
  \cfunction{fopen()}: \var{filename} is the file name to be opened,
  \var{mode} indicates how the file is to be opened: \code{'r'} for
  reading, \code{'w'} for writing (truncating an existing file), and
  \code{'a'} opens it for appending (which on \emph{some} \UNIX{}
  systems means that \emph{all} writes append to the end of the file,
  regardless of the current seek position).

  Modes \code{'r+'}, \code{'w+'} and \code{'a+'} open the file for
  updating (note that \code{'w+'} truncates the file).  Append
  \code{'b'} to the mode to open the file in binary mode, on systems
  that differentiate between binary and text files (else it is
  ignored).  If the file cannot be opened, \exception{IOError} is
  raised.
  
  In addition to the standard \cfunction{fopen()} values \var{mode}
  may be \code{'U'} or \code{'rU'}. If Python is built with universal
  newline support (the default) the file is opened as a text file, but
  lines may be terminated by any of \code{'\e n'}, the Unix end-of-line
  convention,
  \code{'\e r'}, the Macintosh convention or \code{'\e r\e n'}, the Windows
  convention. All of these external representations are seen as
  \code{'\e n'}
  by the Python program. If Python is built without universal newline support
  \var{mode} \code{'U'} is the same as normal text mode.  Note that
  file objects so opened also have an attribute called
  \member{newlines} which has a value of \code{None} (if no newlines
  have yet been seen), \code{'\e n'}, \code{'\e r'}, \code{'\e r\e n'}, 
  or a tuple containing all the newline types seen.

  If \var{mode} is omitted, it defaults to \code{'r'}.  When opening a
  binary file, you should append \code{'b'} to the \var{mode} value
  for improved portability.  (It's useful even on systems which don't
  treat binary and text files differently, where it serves as
  documentation.)
  \index{line-buffered I/O}\index{unbuffered I/O}\index{buffer size, I/O}
  \index{I/O control!buffering}
  The optional \var{bufsize} argument specifies the
  file's desired buffer size: 0 means unbuffered, 1 means line
  buffered, any other positive value means use a buffer of
  (approximately) that size.  A negative \var{bufsize} means to use
  the system default, which is usually line buffered for tty
  devices and fully buffered for other files.  If omitted, the system
  default is used.\footnote{
    Specifying a buffer size currently has no effect on systems that
    don't have \cfunction{setvbuf()}.  The interface to specify the
    buffer size is not done using a method that calls
    \cfunction{setvbuf()}, because that may dump core when called
    after any I/O has been performed, and there's no reliable way to
    determine whether this is the case.}

  The \function{file()} constructor is new in Python 2.2.  The previous
  spelling, \function{open()}, is retained for compatibility, and is an
  alias for \function{file()}.
\end{funcdesc}

\begin{funcdesc}{filter}{function, list}
  Construct a list from those elements of \var{list} for which
  \var{function} returns true.  \var{list} may be either a sequence, a
  container which supports iteration, or an iterator,  If \var{list}
  is a string or a tuple, the result also has that type; otherwise it
  is always a list.  If \var{function} is \code{None}, the identity
  function is assumed, that is, all elements of \var{list} that are false
  (zero or empty) are removed.

  Note that \code{filter(function, \var{list})} is equivalent to
  \code{[item for item in \var{list} if function(item)]} if function is
  not \code{None} and \code{[item for item in \var{list} if item]} if
  function is \code{None}.
\end{funcdesc}

\begin{funcdesc}{float}{\optional{x}}
  Convert a string or a number to floating point.  If the argument is a
  string, it must contain a possibly signed decimal or floating point
  number, possibly embedded in whitespace. Otherwise, the argument may be a plain
  or long integer or a floating point number, and a floating point
  number with the same value (within Python's floating point
  precision) is returned.  If no argument is given, returns \code{0.0}.

  \note{When passing in a string, values for NaN\index{NaN}
  and Infinity\index{Infinity} may be returned, depending on the
  underlying C library.  The specific set of strings accepted which
  cause these values to be returned depends entirely on the C library
  and is known to vary.}
\end{funcdesc}

\begin{funcdesc}{frozenset}{\optional{iterable}}
  Return a frozenset object whose elements are taken from \var{iterable}.
  Frozensets are sets that have no update methods but can be hashed and
  used as members of other sets or as dictionary keys.  The elements of
  a frozenset must be immutable themselves.  To represent sets of sets,
  the inner sets should also be \class{frozenset} objects.  If
  \var{iterable} is not specified, returns a new empty set,
  \code{frozenset([])}.
  \versionadded{2.4}  
\end{funcdesc}

\begin{funcdesc}{getattr}{object, name\optional{, default}}
  Return the value of the named attributed of \var{object}.  \var{name}
  must be a string.  If the string is the name of one of the object's
  attributes, the result is the value of that attribute.  For example,
  \code{getattr(x, 'foobar')} is equivalent to \code{x.foobar}.  If the
  named attribute does not exist, \var{default} is returned if provided,
  otherwise \exception{AttributeError} is raised.
\end{funcdesc}

\begin{funcdesc}{globals}{}
  Return a dictionary representing the current global symbol table.
  This is always the dictionary of the current module (inside a
  function or method, this is the module where it is defined, not the
  module from which it is called).
\end{funcdesc}

\begin{funcdesc}{hasattr}{object, name}
  The arguments are an object and a string.  The result is \code{True} if the
  string is the name of one of the object's attributes, \code{False} if not.
  (This is implemented by calling \code{getattr(\var{object},
  \var{name})} and seeing whether it raises an exception or not.)
\end{funcdesc}

\begin{funcdesc}{hash}{object}
  Return the hash value of the object (if it has one).  Hash values
  are integers.  They are used to quickly compare dictionary
  keys during a dictionary lookup.  Numeric values that compare equal
  have the same hash value (even if they are of different types, as is
  the case for 1 and 1.0).
\end{funcdesc}

\begin{funcdesc}{help}{\optional{object}}
  Invoke the built-in help system.  (This function is intended for
  interactive use.)  If no argument is given, the interactive help
  system starts on the interpreter console.  If the argument is a
  string, then the string is looked up as the name of a module,
  function, class, method, keyword, or documentation topic, and a
  help page is printed on the console.  If the argument is any other
  kind of object, a help page on the object is generated.
  \versionadded{2.2}
\end{funcdesc}

\begin{funcdesc}{hex}{x}
  Convert an integer number (of any size) to a hexadecimal string.
  The result is a valid Python expression.  Note: this always yields
  an unsigned literal.  For example, on a 32-bit machine,
  \code{hex(-1)} yields \code{'0xffffffff'}.  When evaluated on a
  machine with the same word size, this literal is evaluated as -1; at
  a different word size, it may turn up as a large positive number or
  raise an \exception{OverflowError} exception.
\end{funcdesc}

\begin{funcdesc}{id}{object}
  Return the `identity' of an object.  This is an integer (or long
  integer) which is guaranteed to be unique and constant for this
  object during its lifetime.  Two objects whose lifetimes are
  disjunct may have the same \function{id()} value.  (Implementation
  note: this is the address of the object.)
\end{funcdesc}

\begin{funcdesc}{input}{\optional{prompt}}
  Equivalent to \code{eval(raw_input(\var{prompt}))}.
  \warning{This function is not safe from user errors!  It
  expects a valid Python expression as input; if the input is not
  syntactically valid, a \exception{SyntaxError} will be raised.
  Other exceptions may be raised if there is an error during
  evaluation.  (On the other hand, sometimes this is exactly what you
  need when writing a quick script for expert use.)}

  If the \refmodule{readline} module was loaded, then
  \function{input()} will use it to provide elaborate line editing and
  history features.

  Consider using the \function{raw_input()} function for general input
  from users.
\end{funcdesc}

\begin{funcdesc}{int}{\optional{x\optional{, radix}}}
  Convert a string or number to a plain integer.  If the argument is a
  string, it must contain a possibly signed decimal number
  representable as a Python integer, possibly embedded in whitespace.
  The \var{radix} parameter gives the base for the
  conversion and may be any integer in the range [2, 36], or zero.  If
  \var{radix} is zero, the proper radix is guessed based on the
  contents of string; the interpretation is the same as for integer
  literals.  If \var{radix} is specified and \var{x} is not a string,
  \exception{TypeError} is raised.
  Otherwise, the argument may be a plain or
  long integer or a floating point number.  Conversion of floating
  point numbers to integers truncates (towards zero).
  If the argument is outside the integer range a long object will
  be returned instead.  If no arguments are given, returns \code{0}.
\end{funcdesc}

\begin{funcdesc}{isinstance}{object, classinfo}
  Return true if the \var{object} argument is an instance of the
  \var{classinfo} argument, or of a (direct or indirect) subclass
  thereof.  Also return true if \var{classinfo} is a type object and
  \var{object} is an object of that type.  If \var{object} is not a
  class instance or an object of the given type, the function always
  returns false.  If \var{classinfo} is neither a class object nor a
  type object, it may be a tuple of class or type objects, or may
  recursively contain other such tuples (other sequence types are not
  accepted).  If \var{classinfo} is not a class, type, or tuple of
  classes, types, and such tuples, a \exception{TypeError} exception
  is raised.
  \versionchanged[Support for a tuple of type information was added]{2.2}
\end{funcdesc}

\begin{funcdesc}{issubclass}{class, classinfo}
  Return true if \var{class} is a subclass (direct or indirect) of
  \var{classinfo}.  A class is considered a subclass of itself.
  \var{classinfo} may be a tuple of class objects, in which case every
  entry in \var{classinfo} will be checked. In any other case, a
  \exception{TypeError} exception is raised.
  \versionchanged[Support for a tuple of type information was added]{2.3}
\end{funcdesc}

\begin{funcdesc}{iter}{o\optional{, sentinel}}
  Return an iterator object.  The first argument is interpreted very
  differently depending on the presence of the second argument.
  Without a second argument, \var{o} must be a collection object which
  supports the iteration protocol (the \method{__iter__()} method), or
  it must support the sequence protocol (the \method{__getitem__()}
  method with integer arguments starting at \code{0}).  If it does not
  support either of those protocols, \exception{TypeError} is raised.
  If the second argument, \var{sentinel}, is given, then \var{o} must
  be a callable object.  The iterator created in this case will call
  \var{o} with no arguments for each call to its \method{next()}
  method; if the value returned is equal to \var{sentinel},
  \exception{StopIteration} will be raised, otherwise the value will
  be returned.
  \versionadded{2.2}
\end{funcdesc}

\begin{funcdesc}{len}{s}
  Return the length (the number of items) of an object.  The argument
  may be a sequence (string, tuple or list) or a mapping (dictionary).
\end{funcdesc}

\begin{funcdesc}{list}{\optional{sequence}}
  Return a list whose items are the same and in the same order as
  \var{sequence}'s items.  \var{sequence} may be either a sequence, a
  container that supports iteration, or an iterator object.  If
  \var{sequence} is already a list, a copy is made and returned,
  similar to \code{\var{sequence}[:]}.  For instance,
  \code{list('abc')} returns \code{['a', 'b', 'c']} and \code{list(
  (1, 2, 3) )} returns \code{[1, 2, 3]}.  If no argument is given,
  returns a new empty list, \code{[]}.
\end{funcdesc}

\begin{funcdesc}{locals}{}
  Update and return a dictionary representing the current local symbol table.
  \warning{The contents of this dictionary should not be modified;
  changes may not affect the values of local variables used by the
  interpreter.}
\end{funcdesc}

\begin{funcdesc}{long}{\optional{x\optional{, radix}}}
  Convert a string or number to a long integer.  If the argument is a
  string, it must contain a possibly signed number of
  arbitrary size, possibly embedded in whitespace. The
  \var{radix} argument is interpreted in the same way as for
  \function{int()}, and may only be given when \var{x} is a string.
  Otherwise, the argument may be a plain or
  long integer or a floating point number, and a long integer with
  the same value is returned.    Conversion of floating
  point numbers to integers truncates (towards zero).  If no arguments
  are given, returns \code{0L}.
\end{funcdesc}

\begin{funcdesc}{map}{function, list, ...}
  Apply \var{function} to every item of \var{list} and return a list
  of the results.  If additional \var{list} arguments are passed,
  \var{function} must take that many arguments and is applied to the
  items of all lists in parallel; if a list is shorter than another it
  is assumed to be extended with \code{None} items.  If \var{function}
  is \code{None}, the identity function is assumed; if there are
  multiple list arguments, \function{map()} returns a list consisting
  of tuples containing the corresponding items from all lists (a kind
  of transpose operation).  The \var{list} arguments may be any kind
  of sequence; the result is always a list.
\end{funcdesc}

\begin{funcdesc}{max}{s\optional{, args...}}
  With a single argument \var{s}, return the largest item of a
  non-empty sequence (such as a string, tuple or list).  With more
  than one argument, return the largest of the arguments.
\end{funcdesc}

\begin{funcdesc}{min}{s\optional{, args...}}
  With a single argument \var{s}, return the smallest item of a
  non-empty sequence (such as a string, tuple or list).  With more
  than one argument, return the smallest of the arguments.
\end{funcdesc}

\begin{funcdesc}{object}{}
  Return a new featureless object.  \function{object()} is a base 
  for all new style classes.  It has the methods that are common
  to all instances of new style classes.
  \versionadded{2.2}

  \versionchanged[This function does not accept any arguments.
  Formerly, it accepted arguments but ignored them]{2.3}
\end{funcdesc}

\begin{funcdesc}{oct}{x}
  Convert an integer number (of any size) to an octal string.  The
  result is a valid Python expression.  Note: this always yields an
  unsigned literal.  For example, on a 32-bit machine, \code{oct(-1)}
  yields \code{'037777777777'}.  When evaluated on a machine with the
  same word size, this literal is evaluated as -1; at a different word
  size, it may turn up as a large positive number or raise an
  \exception{OverflowError} exception.
\end{funcdesc}

\begin{funcdesc}{open}{filename\optional{, mode\optional{, bufsize}}}
  An alias for the \function{file()} function above.
\end{funcdesc}

\begin{funcdesc}{ord}{c}
  Return the \ASCII{} value of a string of one character or a Unicode
  character.  E.g., \code{ord('a')} returns the integer \code{97},
  \code{ord(u'\e u2020')} returns \code{8224}.  This is the inverse of
  \function{chr()} for strings and of \function{unichr()} for Unicode
  characters.
\end{funcdesc}

\begin{funcdesc}{pow}{x, y\optional{, z}}
  Return \var{x} to the power \var{y}; if \var{z} is present, return
  \var{x} to the power \var{y}, modulo \var{z} (computed more
  efficiently than \code{pow(\var{x}, \var{y}) \%\ \var{z}}).  The
  arguments must have numeric types.  With mixed operand types, the
  coercion rules for binary arithmetic operators apply.  For int and
  long int operands, the result has the same type as the operands
  (after coercion) unless the second argument is negative; in that
  case, all arguments are converted to float and a float result is
  delivered.  For example, \code{10**2} returns \code{100}, but
  \code{10**-2} returns \code{0.01}.  (This last feature was added in
  Python 2.2.  In Python 2.1 and before, if both arguments were of integer
  types and the second argument was negative, an exception was raised.)
  If the second argument is negative, the third argument must be omitted.
  If \var{z} is present, \var{x} and \var{y} must be of integer types,
  and \var{y} must be non-negative.  (This restriction was added in
  Python 2.2.  In Python 2.1 and before, floating 3-argument \code{pow()}
  returned platform-dependent results depending on floating-point
  rounding accidents.)
\end{funcdesc}

\begin{funcdesc}{property}{\optional{fget\optional{, fset\optional{,
                           fdel\optional{, doc}}}}}
  Return a property attribute for new-style classes (classes that
  derive from \class{object}).

  \var{fget} is a function for getting an attribute value, likewise
  \var{fset} is a function for setting, and \var{fdel} a function
  for del'ing, an attribute.  Typical use is to define a managed attribute x:

\begin{verbatim}
class C(object):
    def getx(self): return self.__x
    def setx(self, value): self.__x = value
    def delx(self): del self.__x
    x = property(getx, setx, delx, "I'm the 'x' property.")
\end{verbatim}

  \versionadded{2.2}
\end{funcdesc}

\begin{funcdesc}{range}{\optional{start,} stop\optional{, step}}
  This is a versatile function to create lists containing arithmetic
  progressions.  It is most often used in \keyword{for} loops.  The
  arguments must be plain integers.  If the \var{step} argument is
  omitted, it defaults to \code{1}.  If the \var{start} argument is
  omitted, it defaults to \code{0}.  The full form returns a list of
  plain integers \code{[\var{start}, \var{start} + \var{step},
  \var{start} + 2 * \var{step}, \ldots]}.  If \var{step} is positive,
  the last element is the largest \code{\var{start} + \var{i} *
  \var{step}} less than \var{stop}; if \var{step} is negative, the last
  element is the largest \code{\var{start} + \var{i} * \var{step}}
  greater than \var{stop}.  \var{step} must not be zero (or else
  \exception{ValueError} is raised).  Example:

\begin{verbatim}
>>> range(10)
[0, 1, 2, 3, 4, 5, 6, 7, 8, 9]
>>> range(1, 11)
[1, 2, 3, 4, 5, 6, 7, 8, 9, 10]
>>> range(0, 30, 5)
[0, 5, 10, 15, 20, 25]
>>> range(0, 10, 3)
[0, 3, 6, 9]
>>> range(0, -10, -1)
[0, -1, -2, -3, -4, -5, -6, -7, -8, -9]
>>> range(0)
[]
>>> range(1, 0)
[]
\end{verbatim}
\end{funcdesc}

\begin{funcdesc}{raw_input}{\optional{prompt}}
  If the \var{prompt} argument is present, it is written to standard output
  without a trailing newline.  The function then reads a line from input,
  converts it to a string (stripping a trailing newline), and returns that.
  When \EOF{} is read, \exception{EOFError} is raised. Example:

\begin{verbatim}
>>> s = raw_input('--> ')
--> Monty Python's Flying Circus
>>> s
"Monty Python's Flying Circus"
\end{verbatim}

  If the \refmodule{readline} module was loaded, then
  \function{raw_input()} will use it to provide elaborate
  line editing and history features.
\end{funcdesc}

\begin{funcdesc}{reduce}{function, sequence\optional{, initializer}}
  Apply \var{function} of two arguments cumulatively to the items of
  \var{sequence}, from left to right, so as to reduce the sequence to
  a single value.  For example, \code{reduce(lambda x, y: x+y, [1, 2,
  3, 4, 5])} calculates \code{((((1+2)+3)+4)+5)}.  The left argument,
  \var{x}, is the accumulated value and the right argument, \var{y},
  is the update value from the \var{sequence}.  If the optional
  \var{initializer} is present, it is placed before the items of the
  sequence in the calculation, and serves as a default when the
  sequence is empty.  If \var{initializer} is not given and
  \var{sequence} contains only one item, the first item is returned.
\end{funcdesc}

\begin{funcdesc}{reload}{module}
  Reload a previously imported \var{module}.  The
  argument must be a module object, so it must have been successfully
  imported before.  This is useful if you have edited the module
  source file using an external editor and want to try out the new
  version without leaving the Python interpreter.  The return value is
  the module object (the same as the \var{module} argument).

  When \code{reload(module)} is executed:

\begin{itemize}

    \item{Python modules' code is recompiled and the module-level code
    reexecuted, defining a new set of objects which are bound to names in
    the module's dictionary.  The \code{init} function of extension
    modules is not called a second time.}

    \item{As with all other objects in Python the old objects are only
    reclaimed after their reference counts drop to zero.}

    \item{The names in the module namespace are updated to point to
    any new or changed objects.}

    \item{Other references to the old objects (such as names external
    to the module) are not rebound to refer to the new objects and
    must be updated in each namespace where they occur if that is
    desired.}

\end{itemize}

  There are a number of other caveats:

  If a module is syntactically correct but its initialization fails,
  the first \keyword{import} statement for it does not bind its name
  locally, but does store a (partially initialized) module object in
  \code{sys.modules}.  To reload the module you must first
  \keyword{import} it again (this will bind the name to the partially
  initialized module object) before you can \function{reload()} it.

  When a module is reloaded, its dictionary (containing the module's
  global variables) is retained.  Redefinitions of names will override
  the old definitions, so this is generally not a problem.  If the new
  version of a module does not define a name that was defined by the
  old version, the old definition remains.  This feature can be used
  to the module's advantage if it maintains a global table or cache of
  objects --- with a \keyword{try} statement it can test for the
  table's presence and skip its initialization if desired.

  It is legal though generally not very useful to reload built-in or
  dynamically loaded modules, except for \refmodule{sys},
  \refmodule[main]{__main__} and \refmodule[builtin]{__builtin__}.  In
  many cases, however, extension modules are not designed to be
  initialized more than once, and may fail in arbitrary ways when
  reloaded.

  If a module imports objects from another module using \keyword{from}
  \ldots{} \keyword{import} \ldots{}, calling \function{reload()} for
  the other module does not redefine the objects imported from it ---
  one way around this is to re-execute the \keyword{from} statement,
  another is to use \keyword{import} and qualified names
  (\var{module}.\var{name}) instead.

  If a module instantiates instances of a class, reloading the module
  that defines the class does not affect the method definitions of the
  instances --- they continue to use the old class definition.  The
  same is true for derived classes.
\end{funcdesc}

\begin{funcdesc}{repr}{object}
  Return a string containing a printable representation of an object.
  This is the same value yielded by conversions (reverse quotes).
  It is sometimes useful to be able to access this operation as an
  ordinary function.  For many types, this function makes an attempt
  to return a string that would yield an object with the same value
  when passed to \function{eval()}.
\end{funcdesc}

\begin{funcdesc}{reversed}{seq}
  Return a reverse iterator.  \var{seq} must be an object which
  supports the sequence protocol (the __len__() method and the
  \method{__getitem__()} method with integer arguments starting at
  \code{0}).
  \versionadded{2.4}
\end{funcdesc}

\begin{funcdesc}{round}{x\optional{, n}}
  Return the floating point value \var{x} rounded to \var{n} digits
  after the decimal point.  If \var{n} is omitted, it defaults to zero.
  The result is a floating point number.  Values are rounded to the
  closest multiple of 10 to the power minus \var{n}; if two multiples
  are equally close, rounding is done away from 0 (so. for example,
  \code{round(0.5)} is \code{1.0} and \code{round(-0.5)} is \code{-1.0}).
\end{funcdesc}

\begin{funcdesc}{set}{\optional{iterable}}
  Return a set whose elements are taken from \var{iterable}.  The elements
  must be immutable.  To represent sets of sets, the inner sets should
  be \class{frozenset} objects.  If \var{iterable} is not specified,
  returns a new empty set, \code{set([])}.
  \versionadded{2.4}  
\end{funcdesc}

\begin{funcdesc}{setattr}{object, name, value}
  This is the counterpart of \function{getattr()}.  The arguments are an
  object, a string and an arbitrary value.  The string may name an
  existing attribute or a new attribute.  The function assigns the
  value to the attribute, provided the object allows it.  For example,
  \code{setattr(\var{x}, '\var{foobar}', 123)} is equivalent to
  \code{\var{x}.\var{foobar} = 123}.
\end{funcdesc}

\begin{funcdesc}{slice}{\optional{start,} stop\optional{, step}}
  Return a slice object representing the set of indices specified by
  \code{range(\var{start}, \var{stop}, \var{step})}.  The \var{start}
  and \var{step} arguments default to \code{None}.  Slice objects have
  read-only data attributes \member{start}, \member{stop} and
  \member{step} which merely return the argument values (or their
  default).  They have no other explicit functionality; however they
  are used by Numerical Python\index{Numerical Python} and other third
  party extensions.  Slice objects are also generated when extended
  indexing syntax is used.  For example: \samp{a[start:stop:step]} or
  \samp{a[start:stop, i]}.
\end{funcdesc}

\begin{funcdesc}{sorted}{iterable\optional{, cmp\optional{,
                         key\optional{, reverse}}}}
  Return a new sorted list from the items in \var{iterable}.
  The optional arguments \var{cmp}, \var{key}, and \var{reverse}
  have the same meaning as those for the \method{list.sort()} method.
  \versionadded{2.4}    
\end{funcdesc}

\begin{funcdesc}{staticmethod}{function}
  Return a static method for \var{function}.

  A static method does not receive an implicit first argument.
  To declare a static method, use this idiom:

\begin{verbatim}
class C:
    def f(arg1, arg2, ...): ...
    f = staticmethod(f)
\end{verbatim}

  It can be called either on the class (such as \code{C.f()}) or on an
  instance (such as \code{C().f()}).  The instance is ignored except
  for its class.

  Static methods in Python are similar to those found in Java or \Cpp.
  For a more advanced concept, see \function{classmethod()} in this
  section.
  \versionadded{2.2}
\end{funcdesc}

\begin{funcdesc}{str}{\optional{object}}
  Return a string containing a nicely printable representation of an
  object.  For strings, this returns the string itself.  The
  difference with \code{repr(\var{object})} is that
  \code{str(\var{object})} does not always attempt to return a string
  that is acceptable to \function{eval()}; its goal is to return a
  printable string.  If no argument is given, returns the empty
  string, \code{''}.
\end{funcdesc}

\begin{funcdesc}{sum}{sequence\optional{, start}}
  Sums \var{start} and the items of a \var{sequence}, from left to
  right, and returns the total.  \var{start} defaults to \code{0}.
  The \var{sequence}'s items are normally numbers, and are not allowed
  to be strings.  The fast, correct way to concatenate sequence of
  strings is by calling \code{''.join(\var{sequence})}.
  Note that \code{sum(range(\var{n}), \var{m})} is equivalent to
  \code{reduce(operator.add, range(\var{n}), \var{m})}
  \versionadded{2.3}
\end{funcdesc}

\begin{funcdesc}{super}{type\optional{, object-or-type}}
  Return the superclass of \var{type}.  If the second argument is omitted
  the super object returned is unbound.  If the second argument is an
  object, \code{isinstance(\var{obj}, \var{type})} must be true.  If
  the second argument is a type, \code{issubclass(\var{type2},
  \var{type})} must be true.
  \function{super()} only works for new-style classes.

  A typical use for calling a cooperative superclass method is:
\begin{verbatim}
class C(B):
    def meth(self, arg):
        super(C, self).meth(arg)
\end{verbatim}
\versionadded{2.2}
\end{funcdesc}

\begin{funcdesc}{tuple}{\optional{sequence}}
  Return a tuple whose items are the same and in the same order as
  \var{sequence}'s items.  \var{sequence} may be a sequence, a
  container that supports iteration, or an iterator object.
  If \var{sequence} is already a tuple, it
  is returned unchanged.  For instance, \code{tuple('abc')} returns
  \code{('a', 'b', 'c')} and \code{tuple([1, 2, 3])} returns
  \code{(1, 2, 3)}.  If no argument is given, returns a new empty
  tuple, \code{()}.
\end{funcdesc}

\begin{funcdesc}{type}{object}
  Return the type of an \var{object}.  The return value is a
  type\obindex{type} object.  The standard module
  \module{types}\refstmodindex{types} defines names for all built-in
  types that don't already have built-in names.
  For instance:

\begin{verbatim}
>>> import types
>>> x = 'abc'
>>> if type(x) is str: print "It's a string"
...
It's a string
>>> def f(): pass
...
>>> if type(f) is types.FunctionType: print "It's a function"
...
It's a function
\end{verbatim}

  The \function{isinstance()} built-in function is recommended for
  testing the type of an object.
\end{funcdesc}

\begin{funcdesc}{unichr}{i}
  Return the Unicode string of one character whose Unicode code is the
  integer \var{i}.  For example, \code{unichr(97)} returns the string
  \code{u'a'}.  This is the inverse of \function{ord()} for Unicode
  strings.  The argument must be in the range [0..65535], inclusive.
  \exception{ValueError} is raised otherwise.
  \versionadded{2.0}
\end{funcdesc}

\begin{funcdesc}{unicode}{\optional{object\optional{, encoding
				    \optional{, errors}}}}
  Return the Unicode string version of \var{object} using one of the
  following modes:

  If \var{encoding} and/or \var{errors} are given, \code{unicode()}
  will decode the object which can either be an 8-bit string or a
  character buffer using the codec for \var{encoding}. The
  \var{encoding} parameter is a string giving the name of an encoding;
  if the encoding is not known, \exception{LookupError} is raised.
  Error handling is done according to \var{errors}; this specifies the
  treatment of characters which are invalid in the input encoding.  If
  \var{errors} is \code{'strict'} (the default), a
  \exception{ValueError} is raised on errors, while a value of
  \code{'ignore'} causes errors to be silently ignored, and a value of
  \code{'replace'} causes the official Unicode replacement character,
  \code{U+FFFD}, to be used to replace input characters which cannot
  be decoded.  See also the \refmodule{codecs} module.

  If no optional parameters are given, \code{unicode()} will mimic the
  behaviour of \code{str()} except that it returns Unicode strings
  instead of 8-bit strings. More precisely, if \var{object} is a
  Unicode string or subclass it will return that Unicode string without
  any additional decoding applied.

  For objects which provide a \method{__unicode__()} method, it will
  call this method without arguments to create a Unicode string. For
  all other objects, the 8-bit string version or representation is
  requested and then converted to a Unicode string using the codec for
  the default encoding in \code{'strict'} mode.

  \versionadded{2.0}
  \versionchanged[Support for \method{__unicode__()} added]{2.2}
\end{funcdesc}

\begin{funcdesc}{vars}{\optional{object}}
  Without arguments, return a dictionary corresponding to the current
  local symbol table.  With a module, class or class instance object
  as argument (or anything else that has a \member{__dict__}
  attribute), returns a dictionary corresponding to the object's
  symbol table.  The returned dictionary should not be modified: the
  effects on the corresponding symbol table are undefined.\footnote{
    In the current implementation, local variable bindings cannot
    normally be affected this way, but variables retrieved from
    other scopes (such as modules) can be.  This may change.}
\end{funcdesc}

\begin{funcdesc}{xrange}{\optional{start,} stop\optional{, step}}
  This function is very similar to \function{range()}, but returns an
  ``xrange object'' instead of a list.  This is an opaque sequence
  type which yields the same values as the corresponding list, without
  actually storing them all simultaneously.  The advantage of
  \function{xrange()} over \function{range()} is minimal (since
  \function{xrange()} still has to create the values when asked for
  them) except when a very large range is used on a memory-starved
  machine or when all of the range's elements are never used (such as
  when the loop is usually terminated with \keyword{break}).
\end{funcdesc}

\begin{funcdesc}{zip}{\optional{seq1, \moreargs}}
  This function returns a list of tuples, where the \var{i}-th tuple contains
  the \var{i}-th element from each of the argument sequences.
  The returned list is truncated in length to the length of
  the shortest argument sequence.  When there are multiple argument
  sequences which are all of the same length, \function{zip()} is
  similar to \function{map()} with an initial argument of \code{None}.
  With a single sequence argument, it returns a list of 1-tuples.
  With no arguments, it returns an empty list.
  \versionadded{2.0}

  \versionchanged[Formerly, \function{zip()} required at least one argument
  and \code{zip()} raised a \exception{TypeError} instead of returning
  an empty list.]{2.4} 
\end{funcdesc}


% ---------------------------------------------------------------------------	


\section{Non-essential Built-in Functions \label{non-essential-built-in-funcs}}

There are several built-in functions that are no longer essential to learn,
know or use in modern Python programming.  They have been kept here to
maintain backwards compatability with programs written for older versions
of Python.

Python programmers, trainers, students and bookwriters should feel free to
bypass these functions without concerns about missing something important.


\setindexsubitem{(non-essential built-in functions)}

\begin{funcdesc}{apply}{function, args\optional{, keywords}}
  The \var{function} argument must be a callable object (a
  user-defined or built-in function or method, or a class object) and
  the \var{args} argument must be a sequence.  The \var{function} is
  called with \var{args} as the argument list; the number of arguments
  is the length of the tuple.
  If the optional \var{keywords} argument is present, it must be a
  dictionary whose keys are strings.  It specifies keyword arguments
  to be added to the end of the argument list.
  Calling \function{apply()} is different from just calling
  \code{\var{function}(\var{args})}, since in that case there is always
  exactly one argument.  The use of \function{apply()} is equivalent
  to \code{\var{function}(*\var{args}, **\var{keywords})}.
  Use of \function{apply()} is not necessary since the ``extended call
  syntax,'' as used in the last example, is completely equivalent.

  \deprecated{2.3}{Use the extended call syntax instead, as described
                   above.}
\end{funcdesc}

\begin{funcdesc}{buffer}{object\optional{, offset\optional{, size}}}
  The \var{object} argument must be an object that supports the buffer
  call interface (such as strings, arrays, and buffers).  A new buffer
  object will be created which references the \var{object} argument.
  The buffer object will be a slice from the beginning of \var{object}
  (or from the specified \var{offset}). The slice will extend to the
  end of \var{object} (or will have a length given by the \var{size}
  argument).
\end{funcdesc}

\begin{funcdesc}{coerce}{x, y}
  Return a tuple consisting of the two numeric arguments converted to
  a common type, using the same rules as used by arithmetic
  operations.
\end{funcdesc}

\begin{funcdesc}{intern}{string}
  Enter \var{string} in the table of ``interned'' strings and return
  the interned string -- which is \var{string} itself or a copy.
  Interning strings is useful to gain a little performance on
  dictionary lookup -- if the keys in a dictionary are interned, and
  the lookup key is interned, the key comparisons (after hashing) can
  be done by a pointer compare instead of a string compare.  Normally,
  the names used in Python programs are automatically interned, and
  the dictionaries used to hold module, class or instance attributes
  have interned keys.  \versionchanged[Interned strings are not
  immortal (like they used to be in Python 2.2 and before);
  you must keep a reference to the return value of \function{intern()}
  around to benefit from it]{2.3}
\end{funcdesc}


\chapter{Python Services}

The modules described in this chapter provide a wide range of services
related to the Python interpreter and its interaction with its
environment.  Here's an overview:

\begin{description}

\item[sys]
--- Access system specific parameters and functions.

\item[types]
--- Names for all built-in types.

\item[traceback]
--- Print or retrieve a stack traceback.

\item[pickle]
--- Convert Python objects to streams of bytes and back.

\item[shelve]
--- Python object persistency.

\item[copy]
--- Shallow and deep copy operations.

\item[marshal]
--- Convert Python objects to streams of bytes and back (with
different constraints).

\item[imp]
--- Access the implementation of the \code{import} statement.

\item[parser]
--- Retrieve and submit parse trees from and to the runtime support
environment.

\item[__builtin__]
--- The set of built-in functions.

\item[__main__]
--- The environment where the top-level script is run.

\end{description}
		% Python Services
\section{\module{sys} ---
         System-specific parameters and functions}

\declaremodule{builtin}{sys}
\modulesynopsis{Access system-specific parameters and functions.}

This module provides access to some variables used or maintained by the
interpreter and to functions that interact strongly with the interpreter.
It is always available.


\begin{datadesc}{argv}
  The list of command line arguments passed to a Python script.
  \code{argv[0]} is the script name (it is operating system dependent
  whether this is a full pathname or not).  If the command was
  executed using the \programopt{-c} command line option to the
  interpreter, \code{argv[0]} is set to the string \code{'-c'}.  If no
  script name was passed to the Python interpreter, \code{argv} has
  zero length.
\end{datadesc}

\begin{datadesc}{byteorder}
  An indicator of the native byte order.  This will have the value
  \code{'big'} on big-endian (most-signigicant byte first) platforms,
  and \code{'little'} on little-endian (least-significant byte first)
  platforms.
  \versionadded{2.0}
\end{datadesc}

\begin{datadesc}{builtin_module_names}
  A tuple of strings giving the names of all modules that are compiled
  into this Python interpreter.  (This information is not available in
  any other way --- \code{modules.keys()} only lists the imported
  modules.)
\end{datadesc}

\begin{datadesc}{copyright}
  A string containing the copyright pertaining to the Python
  interpreter.
\end{datadesc}

\begin{datadesc}{dllhandle}
  Integer specifying the handle of the Python DLL.
  Availability: Windows.
\end{datadesc}

\begin{funcdesc}{displayhook}{\var{value}}
  If \var{value} is not \code{None}, this function prints it to
  \code{sys.stdout}, and saves it in \code{__builtin__._}.

  \code{sys.displayhook} is called on the result of evaluating an
  expression entered in an interactive Python session.  The display of
  these values can be customized by assigning another one-argument
  function to \code{sys.displayhook}.
\end{funcdesc}

\begin{funcdesc}{excepthook}{\var{type}, \var{value}, \var{traceback}}
  This function prints out a given traceback and exception to
  \code{sys.stderr}.

  When an exception is raised and uncaught, the interpreter calls
  \code{sys.excepthook} with three arguments, the exception class,
  exception instance, and a traceback object.  In an interactive
  session this happens just before control is returned to the prompt;
  in a Python program this happens just before the program exits.  The
  handling of such top-level exceptions can be customized by assigning
  another three-argument function to \code{sys.excepthook}.
\end{funcdesc}

\begin{datadesc}{__displayhook__}
\dataline{__excepthook__}
  These objects contain the original values of \code{displayhook} and
  \code{excepthook} at the start of the program.  They are saved so
  that \code{displayhook} and \code{excepthook} can be restored in
  case they happen to get replaced with broken objects.
\end{datadesc}

\begin{funcdesc}{exc_info}{}
  This function returns a tuple of three values that give information
  about the exception that is currently being handled.  The
  information returned is specific both to the current thread and to
  the current stack frame.  If the current stack frame is not handling
  an exception, the information is taken from the calling stack frame,
  or its caller, and so on until a stack frame is found that is
  handling an exception.  Here, ``handling an exception'' is defined
  as ``executing or having executed an except clause.''  For any stack
  frame, only information about the most recently handled exception is
  accessible.

  If no exception is being handled anywhere on the stack, a tuple
  containing three \code{None} values is returned.  Otherwise, the
  values returned are \code{(\var{type}, \var{value},
  \var{traceback})}.  Their meaning is: \var{type} gets the exception
  type of the exception being handled (a class object);
  \var{value} gets the exception parameter (its \dfn{associated value}
  or the second argument to \keyword{raise}, which is always a class
  instance if the exception type is a class object); \var{traceback}
  gets a traceback object (see the Reference Manual) which
  encapsulates the call stack at the point where the exception
  originally occurred.  \obindex{traceback}

  If \function{exc_clear()} is called, this function will return three
  \code{None} values until either another exception is raised in the
  current thread or the execution stack returns to a frame where
  another exception is being handled.

  \warning{Assigning the \var{traceback} return value to a
  local variable in a function that is handling an exception will
  cause a circular reference.  This will prevent anything referenced
  by a local variable in the same function or by the traceback from
  being garbage collected.  Since most functions don't need access to
  the traceback, the best solution is to use something like
  \code{exctype, value = sys.exc_info()[:2]} to extract only the
  exception type and value.  If you do need the traceback, make sure
  to delete it after use (best done with a \keyword{try}
  ... \keyword{finally} statement) or to call \function{exc_info()} in
  a function that does not itself handle an exception.} \note{Beginning
  with Python 2.2, such cycles are automatically reclaimed when garbage
  collection is enabled and they become unreachable, but it remains more
  efficient to avoid creating cycles.}
\end{funcdesc}

\begin{funcdesc}{exc_clear}{}
  This function clears all information relating to the current or last
  exception that occured in the current thread.  After calling this
  function, \function{exc_info()} will return three \code{None} values until
  another exception is raised in the current thread or the execution stack
  returns to a frame where another exception is being handled.
  
  This function is only needed in only a few obscure situations.  These
  include logging and error handling systems that report information on the
  last or current exception.  This function can also be used to try to free
  resources and trigger object finalization, though no guarantee is made as
  to what objects will be freed, if any.
\versionadded{2.3}
\end{funcdesc}

\begin{datadesc}{exc_type}
\dataline{exc_value}
\dataline{exc_traceback}
\deprecated {1.5}
            {Use \function{exc_info()} instead.}
  Since they are global variables, they are not specific to the
  current thread, so their use is not safe in a multi-threaded
  program.  When no exception is being handled, \code{exc_type} is set
  to \code{None} and the other two are undefined.
\end{datadesc}

\begin{datadesc}{exec_prefix}
  A string giving the site-specific directory prefix where the
  platform-dependent Python files are installed; by default, this is
  also \code{'/usr/local'}.  This can be set at build time with the
  \longprogramopt{exec-prefix} argument to the \program{configure}
  script.  Specifically, all configuration files (e.g. the
  \file{pyconfig.h} header file) are installed in the directory
  \code{exec_prefix + '/lib/python\var{version}/config'}, and shared
  library modules are installed in \code{exec_prefix +
  '/lib/python\var{version}/lib-dynload'}, where \var{version} is
  equal to \code{version[:3]}.
\end{datadesc}

\begin{datadesc}{executable}
  A string giving the name of the executable binary for the Python
  interpreter, on systems where this makes sense.
\end{datadesc}

\begin{funcdesc}{exit}{\optional{arg}}
  Exit from Python.  This is implemented by raising the
  \exception{SystemExit} exception, so cleanup actions specified by
  finally clauses of \keyword{try} statements are honored, and it is
  possible to intercept the exit attempt at an outer level.  The
  optional argument \var{arg} can be an integer giving the exit status
  (defaulting to zero), or another type of object.  If it is an
  integer, zero is considered ``successful termination'' and any
  nonzero value is considered ``abnormal termination'' by shells and
  the like.  Most systems require it to be in the range 0-127, and
  produce undefined results otherwise.  Some systems have a convention
  for assigning specific meanings to specific exit codes, but these
  are generally underdeveloped; \UNIX{} programs generally use 2 for
  command line syntax errors and 1 for all other kind of errors.  If
  another type of object is passed, \code{None} is equivalent to
  passing zero, and any other object is printed to \code{sys.stderr}
  and results in an exit code of 1.  In particular,
  \code{sys.exit("some error message")} is a quick way to exit a
  program when an error occurs.
\end{funcdesc}

\begin{datadesc}{exitfunc}
  This value is not actually defined by the module, but can be set by
  the user (or by a program) to specify a clean-up action at program
  exit.  When set, it should be a parameterless function.  This
  function will be called when the interpreter exits.  Only one
  function may be installed in this way; to allow multiple functions
  which will be called at termination, use the \refmodule{atexit}
  module.  \note{The exit function is not called when the program is
  killed by a signal, when a Python fatal internal error is detected,
  or when \code{os._exit()} is called.}
  \deprecated{2.4}{Use \refmodule{atexit} instead.}
\end{datadesc}

\begin{funcdesc}{getcheckinterval}{}
  Return the interpreter's ``check interval'';
  see \function{setcheckinterval()}.
  \versionadded{2.3}
\end{funcdesc}

\begin{funcdesc}{getdefaultencoding}{}
  Return the name of the current default string encoding used by the
  Unicode implementation.
  \versionadded{2.0}
\end{funcdesc}

\begin{funcdesc}{getdlopenflags}{}
  Return the current value of the flags that are used for
  \cfunction{dlopen()} calls. The flag constants are defined in the
  \refmodule{dl} and \module{DLFCN} modules.
  Availability: \UNIX.
  \versionadded{2.2}
\end{funcdesc}

\begin{funcdesc}{getfilesystemencoding}{}
  Return the name of the encoding used to convert Unicode filenames
  into system file names, or \code{None} if the system default encoding
  is used. The result value depends on the operating system:
\begin{itemize}
\item On Windows 9x, the encoding is ``mbcs''.
\item On Mac OS X, the encoding is ``utf-8''.
\item On Unix, the encoding is the user's preference 
      according to the result of nl_langinfo(CODESET), or None if
      the nl_langinfo(CODESET) failed.
\item On Windows NT+, file names are Unicode natively, so no conversion
      is performed. \code{getfilesystemencoding} still returns ``mbcs'',
      as this is the encoding that applications should use when they
      explicitly want to convert Unicode strings to byte strings that
      are equivalent when used as file names.
\end{itemize}
  \versionadded{2.3}
\end{funcdesc}

\begin{funcdesc}{getrefcount}{object}
  Return the reference count of the \var{object}.  The count returned
  is generally one higher than you might expect, because it includes
  the (temporary) reference as an argument to
  \function{getrefcount()}.
\end{funcdesc}

\begin{funcdesc}{getrecursionlimit}{}
  Return the current value of the recursion limit, the maximum depth
  of the Python interpreter stack.  This limit prevents infinite
  recursion from causing an overflow of the C stack and crashing
  Python.  It can be set by \function{setrecursionlimit()}.
\end{funcdesc}

\begin{funcdesc}{_getframe}{\optional{depth}}
  Return a frame object from the call stack.  If optional integer
  \var{depth} is given, return the frame object that many calls below
  the top of the stack.  If that is deeper than the call stack,
  \exception{ValueError} is raised.  The default for \var{depth} is
  zero, returning the frame at the top of the call stack.

  This function should be used for internal and specialized purposes
  only.
\end{funcdesc}

\begin{funcdesc}{getwindowsversion}{}
  Return a tuple containing five components, describing the Windows 
  version currently running.  The elements are \var{major}, \var{minor}, 
  \var{build}, \var{platform}, and \var{text}.  \var{text} contains
  a string while all other values are integers.

  \var{platform} may be one of the following values:
  \begin{list}{}{\leftmargin 0.7in \labelwidth 0.65in}
    \item[0 (\constant{VER_PLATFORM_WIN32s})]
      Win32s on Windows 3.1.
    \item[1 (\constant{VER_PLATFORM_WIN32_WINDOWS})] 
      Windows 95/98/ME
    \item[2 (\constant{VER_PLATFORM_WIN32_NT})] 
      Windows NT/2000/XP
    \item[3 (\constant{VER_PLATFORM_WIN32_CE})] 
      Windows CE.
  \end{list}
  
  This function wraps the Win32 \function{GetVersionEx()} function;
  see the Microsoft Documentation for more information about these
  fields.

  Availability: Windows.
  \versionadded{2.3}
\end{funcdesc}

\begin{datadesc}{hexversion}
  The version number encoded as a single integer.  This is guaranteed
  to increase with each version, including proper support for
  non-production releases.  For example, to test that the Python
  interpreter is at least version 1.5.2, use:

\begin{verbatim}
if sys.hexversion >= 0x010502F0:
    # use some advanced feature
    ...
else:
    # use an alternative implementation or warn the user
    ...
\end{verbatim}

  This is called \samp{hexversion} since it only really looks
  meaningful when viewed as the result of passing it to the built-in
  \function{hex()} function.  The \code{version_info} value may be
  used for a more human-friendly encoding of the same information.
  \versionadded{1.5.2}
\end{datadesc}

\begin{datadesc}{last_type}
\dataline{last_value}
\dataline{last_traceback}
  These three variables are not always defined; they are set when an
  exception is not handled and the interpreter prints an error message
  and a stack traceback.  Their intended use is to allow an
  interactive user to import a debugger module and engage in
  post-mortem debugging without having to re-execute the command that
  caused the error.  (Typical use is \samp{import pdb; pdb.pm()} to
  enter the post-mortem debugger; see chapter \ref{debugger}, ``The
  Python Debugger,'' for more information.)

  The meaning of the variables is the same as that of the return
  values from \function{exc_info()} above.  (Since there is only one
  interactive thread, thread-safety is not a concern for these
  variables, unlike for \code{exc_type} etc.)
\end{datadesc}

\begin{datadesc}{maxint}
  The largest positive integer supported by Python's regular integer
  type.  This is at least 2**31-1.  The largest negative integer is
  \code{-maxint-1} --- the asymmetry results from the use of 2's
  complement binary arithmetic.
\end{datadesc}

\begin{datadesc}{maxunicode}
  An integer giving the largest supported code point for a Unicode
  character.  The value of this depends on the configuration option
  that specifies whether Unicode characters are stored as UCS-2 or
  UCS-4.
\end{datadesc}

\begin{datadesc}{modules}
  This is a dictionary that maps module names to modules which have
  already been loaded.  This can be manipulated to force reloading of
  modules and other tricks.  Note that removing a module from this
  dictionary is \emph{not} the same as calling
  \function{reload()}\bifuncindex{reload} on the corresponding module
  object.
\end{datadesc}

\begin{datadesc}{path}
\indexiii{module}{search}{path}
  A list of strings that specifies the search path for modules.
  Initialized from the environment variable \envvar{PYTHONPATH}, plus an
  installation-dependent default.

  As initialized upon program startup,
  the first item of this list, \code{path[0]}, is the directory
  containing the script that was used to invoke the Python
  interpreter.  If the script directory is not available (e.g.  if the
  interpreter is invoked interactively or if the script is read from
  standard input), \code{path[0]} is the empty string, which directs
  Python to search modules in the current directory first.  Notice
  that the script directory is inserted \emph{before} the entries
  inserted as a result of \envvar{PYTHONPATH}.

  A program is free to modify this list for its own purposes.

  \versionchanged[Unicode strings are no longer ignored]{2.3}
\end{datadesc}

\begin{datadesc}{platform}
  This string contains a platform identifier, e.g. \code{'sunos5'} or
  \code{'linux1'}.  This can be used to append platform-specific
  components to \code{path}, for instance.
\end{datadesc}

\begin{datadesc}{prefix}
  A string giving the site-specific directory prefix where the
  platform independent Python files are installed; by default, this is
  the string \code{'/usr/local'}.  This can be set at build time with
  the \longprogramopt{prefix} argument to the \program{configure}
  script.  The main collection of Python library modules is installed
  in the directory \code{prefix + '/lib/python\var{version}'} while
  the platform independent header files (all except \file{pyconfig.h})
  are stored in \code{prefix + '/include/python\var{version}'}, where
  \var{version} is equal to \code{version[:3]}.
\end{datadesc}

\begin{datadesc}{ps1}
\dataline{ps2}
\index{interpreter prompts}
\index{prompts, interpreter}
  Strings specifying the primary and secondary prompt of the
  interpreter.  These are only defined if the interpreter is in
  interactive mode.  Their initial values in this case are
  \code{'>\code{>}> '} and \code{'... '}.  If a non-string object is
  assigned to either variable, its \function{str()} is re-evaluated
  each time the interpreter prepares to read a new interactive
  command; this can be used to implement a dynamic prompt.
\end{datadesc}

\begin{funcdesc}{setcheckinterval}{interval}
  Set the interpreter's ``check interval''.  This integer value
  determines how often the interpreter checks for periodic things such
  as thread switches and signal handlers.  The default is \code{100},
  meaning the check is performed every 100 Python virtual instructions.
  Setting it to a larger value may increase performance for programs
  using threads.  Setting it to a value \code{<=} 0 checks every
  virtual instruction, maximizing responsiveness as well as overhead.
\end{funcdesc}

\begin{funcdesc}{setdefaultencoding}{name}
  Set the current default string encoding used by the Unicode
  implementation.  If \var{name} does not match any available
  encoding, \exception{LookupError} is raised.  This function is only
  intended to be used by the \refmodule{site} module implementation
  and, where needed, by \module{sitecustomize}.  Once used by the
  \refmodule{site} module, it is removed from the \module{sys}
  module's namespace.
%  Note that \refmodule{site} is not imported if
%  the \programopt{-S} option is passed to the interpreter, in which
%  case this function will remain available.
  \versionadded{2.0}
\end{funcdesc}

\begin{funcdesc}{setdlopenflags}{n}
  Set the flags used by the interpreter for \cfunction{dlopen()}
  calls, such as when the interpreter loads extension modules.  Among
  other things, this will enable a lazy resolving of symbols when
  importing a module, if called as \code{sys.setdlopenflags(0)}.  To
  share symbols across extension modules, call as
  \code{sys.setdlopenflags(dl.RTLD_NOW | dl.RTLD_GLOBAL)}.  Symbolic
  names for the flag modules can be either found in the \refmodule{dl}
  module, or in the \module{DLFCN} module. If \module{DLFCN} is not
  available, it can be generated from \file{/usr/include/dlfcn.h}
  using the \program{h2py} script.
  Availability: \UNIX.
  \versionadded{2.2}
\end{funcdesc}

\begin{funcdesc}{setprofile}{profilefunc}
  Set the system's profile function,\index{profile function} which
  allows you to implement a Python source code profiler in
  Python.\index{profiler}  See chapter \ref{profile} for more
  information on the Python profiler.  The system's profile function
  is called similarly to the system's trace function (see
  \function{settrace()}), but it isn't called for each executed line
  of code (only on call and return, but the return event is reported
  even when an exception has been set).  The function is
  thread-specific, but there is no way for the profiler to know about
  context switches between threads, so it does not make sense to use
  this in the presence of multiple threads.
  Also, its return value is not used, so it can simply return
  \code{None}.
\end{funcdesc}

\begin{funcdesc}{setrecursionlimit}{limit}
  Set the maximum depth of the Python interpreter stack to
  \var{limit}.  This limit prevents infinite recursion from causing an
  overflow of the C stack and crashing Python.

  The highest possible limit is platform-dependent.  A user may need
  to set the limit higher when she has a program that requires deep
  recursion and a platform that supports a higher limit.  This should
  be done with care, because a too-high limit can lead to a crash.
\end{funcdesc}

\begin{funcdesc}{settrace}{tracefunc}
  Set the system's trace function,\index{trace function} which allows
  you to implement a Python source code debugger in Python.  See
  section \ref{debugger-hooks}, ``How It Works,'' in the chapter on
  the Python debugger.\index{debugger}  The function is
  thread-specific; for a debugger to support multiple threads, it must
  be registered using \function{settrace()} for each thread being
  debugged.  \note{The \function{settrace()} function is intended only
  for implementing debuggers, profilers, coverage tools and the like.
  Its behavior is part of the implementation platform, rather than 
  part of the language definition, and thus may not be available in
  all Python implementations.}
\end{funcdesc}

\begin{funcdesc}{settscdump}{on_flag}
  Activate dumping of VM measurements using the Pentium timestamp
  counter, if \var{on_flag} is true. Deactivate these dumps if
  \var{on_flag} is off. The function is available only if Python
  was compiled with \longprogramopt{with-tsc}. To understand the
  output of this dump, read \file{Python/ceval.c} in the Python
  sources.
  \versionadded{2.4}
\end{funcdesc}

\begin{datadesc}{stdin}
\dataline{stdout}
\dataline{stderr}
  File objects corresponding to the interpreter's standard input,
  output and error streams.  \code{stdin} is used for all interpreter
  input except for scripts but including calls to
  \function{input()}\bifuncindex{input} and
  \function{raw_input()}\bifuncindex{raw_input}.  \code{stdout} is
  used for the output of \keyword{print} and expression statements and
  for the prompts of \function{input()} and \function{raw_input()}.
  The interpreter's own prompts and (almost all of) its error messages
  go to \code{stderr}.  \code{stdout} and \code{stderr} needn't be
  built-in file objects: any object is acceptable as long as it has a
  \method{write()} method that takes a string argument.  (Changing
  these objects doesn't affect the standard I/O streams of processes
  executed by \function{os.popen()}, \function{os.system()} or the
  \function{exec*()} family of functions in the \refmodule{os}
  module.)
\end{datadesc}

\begin{datadesc}{__stdin__}
\dataline{__stdout__}
\dataline{__stderr__}
  These objects contain the original values of \code{stdin},
  \code{stderr} and \code{stdout} at the start of the program.  They
  are used during finalization, and could be useful to restore the
  actual files to known working file objects in case they have been
  overwritten with a broken object.
\end{datadesc}

\begin{datadesc}{tracebacklimit}
  When this variable is set to an integer value, it determines the
  maximum number of levels of traceback information printed when an
  unhandled exception occurs.  The default is \code{1000}.  When set
  to \code{0} or less, all traceback information is suppressed and
  only the exception type and value are printed.
\end{datadesc}

\begin{datadesc}{version}
  A string containing the version number of the Python interpreter
  plus additional information on the build number and compiler used.
  It has a value of the form \code{'\var{version}
  (\#\var{build_number}, \var{build_date}, \var{build_time})
  [\var{compiler}]'}.  The first three characters are used to identify
  the version in the installation directories (where appropriate on
  each platform).  An example:

\begin{verbatim}
>>> import sys
>>> sys.version
'1.5.2 (#0 Apr 13 1999, 10:51:12) [MSC 32 bit (Intel)]'
\end{verbatim}
\end{datadesc}

\begin{datadesc}{api_version}
  The C API version for this interpreter.  Programmers may find this useful
  when debugging version conflicts between Python and extension
  modules. \versionadded{2.3}
\end{datadesc}

\begin{datadesc}{version_info}
  A tuple containing the five components of the version number:
  \var{major}, \var{minor}, \var{micro}, \var{releaselevel}, and
  \var{serial}.  All values except \var{releaselevel} are integers;
  the release level is \code{'alpha'}, \code{'beta'},
  \code{'candidate'}, or \code{'final'}.  The \code{version_info}
  value corresponding to the Python version 2.0 is \code{(2, 0, 0,
  'final', 0)}.
  \versionadded{2.0}
\end{datadesc}

\begin{datadesc}{warnoptions}
  This is an implementation detail of the warnings framework; do not
  modify this value.  Refer to the \refmodule{warnings} module for
  more information on the warnings framework.
\end{datadesc}

\begin{datadesc}{winver}
  The version number used to form registry keys on Windows platforms.
  This is stored as string resource 1000 in the Python DLL.  The value
  is normally the first three characters of \constant{version}.  It is
  provided in the \module{sys} module for informational purposes;
  modifying this value has no effect on the registry keys used by
  Python.
  Availability: Windows.
\end{datadesc}


\begin{seealso}
  \seemodule{site}
    {This describes how to use .pth files to extend \code{sys.path}.}
\end{seealso}

\section{Built-in Types}
\label{types}

The following sections describe the standard types that are built into
the interpreter.  These are the numeric types, sequence types, and
several others, including types themselves.  There is no explicit
Boolean type; use integers instead.
\indexii{built-in}{types}
\indexii{Boolean}{type}

Some operations are supported by several object types; in particular,
all objects can be compared, tested for truth value, and converted to
a string (with the \code{`{\rm \ldots}`} notation).  The latter conversion is
implicitly used when an object is written by the \code{print} statement.
\stindex{print}


\subsection{Truth Value Testing}
\label{truth}

Any object can be tested for truth value, for use in an \code{if} or
\code{while} condition or as operand of the Boolean operations below.
The following values are considered false:
\stindex{if}
\stindex{while}
\indexii{truth}{value}
\indexii{Boolean}{operations}
\index{false}

\setindexsubitem{(Built-in object)}
\begin{itemize}

\item	\code{None}
	\ttindex{None}

\item	zero of any numeric type, e.g., \code{0}, \code{0L}, \code{0.0}.

\item	any empty sequence, e.g., \code{''}, \code{()}, \code{[]}.

\item	any empty mapping, e.g., \code{\{\}}.

\item	instances of user-defined classes, if the class defines a
	\code{__nonzero__()} or \code{__len__()} method, when that
	method returns zero.

\end{itemize}

All other values are considered true --- so objects of many types are
always true.
\index{true}

Operations and built-in functions that have a Boolean result always
return \code{0} for false and \code{1} for true, unless otherwise
stated.  (Important exception: the Boolean operations
\samp{or}\opindex{or} and \samp{and}\opindex{and} always return one of
their operands.)


\subsection{Boolean Operations}
\label{boolean}

These are the Boolean operations, ordered by ascending priority:
\indexii{Boolean}{operations}

\begin{tableiii}{|c|l|c|}{code}{Operation}{Result}{Notes}
  \lineiii{\var{x} or \var{y}}{if \var{x} is false, then \var{y}, else \var{x}}{(1)}
  \hline
  \lineiii{\var{x} and \var{y}}{if \var{x} is false, then \var{x}, else \var{y}}{(1)}
  \hline
  \lineiii{not \var{x}}{if \var{x} is false, then \code{1}, else \code{0}}{(2)}
\end{tableiii}
\opindex{and}
\opindex{or}
\opindex{not}

\noindent
Notes:

\begin{description}

\item[(1)]
These only evaluate their second argument if needed for their outcome.

\item[(2)]
\samp{not} has a lower priority than non-Boolean operators, so e.g.
\code{not a == b} is interpreted as \code{not(a == b)}, and
\code{a == not b} is a syntax error.

\end{description}


\subsection{Comparisons}
\label{comparisons}

Comparison operations are supported by all objects.  They all have the
same priority (which is higher than that of the Boolean operations).
Comparisons can be chained arbitrarily, e.g. \code{x < y <= z} is
equivalent to \code{x < y and y <= z}, except that \code{y} is
evaluated only once (but in both cases \code{z} is not evaluated at
all when \code{x < y} is found to be false).
\indexii{chaining}{comparisons}

This table summarizes the comparison operations:

\begin{tableiii}{|c|l|c|}{code}{Operation}{Meaning}{Notes}
  \lineiii{<}{strictly less than}{}
  \lineiii{<=}{less than or equal}{}
  \lineiii{>}{strictly greater than}{}
  \lineiii{>=}{greater than or equal}{}
  \lineiii{==}{equal}{}
  \lineiii{<>}{not equal}{(1)}
  \lineiii{!=}{not equal}{(1)}
  \lineiii{is}{object identity}{}
  \lineiii{is not}{negated object identity}{}
\end{tableiii}
\indexii{operator}{comparison}
\opindex{==} % XXX *All* others have funny characters < ! >
\opindex{is}
\opindex{is not}

\noindent
Notes:

\begin{description}

\item[(1)]
\code{<>} and \code{!=} are alternate spellings for the same operator.
(I couldn't choose between \ABC{} and \C{}! :-)
\indexii{ABC@\ABC{}}{language}
\indexii{\C{}}{language}

\end{description}

Objects of different types, except different numeric types, never
compare equal; such objects are ordered consistently but arbitrarily
(so that sorting a heterogeneous array yields a consistent result).
Furthermore, some types (e.g., windows) support only a degenerate
notion of comparison where any two objects of that type are unequal.
Again, such objects are ordered arbitrarily but consistently.
\indexii{types}{numeric}
\indexii{objects}{comparing}

(Implementation note: objects of different types except numbers are
ordered by their type names; objects of the same types that don't
support proper comparison are ordered by their address.)

Two more operations with the same syntactic priority, \code{in} and
\code{not in}, are supported only by sequence types (below).
\opindex{in}
\opindex{not in}


\subsection{Numeric Types}
\label{typesnumeric}

There are four numeric types: \dfn{plain integers}, \dfn{long integers}, 
\dfn{floating point numbers}, and \dfn{complex numbers}.
Plain integers (also just called \dfn{integers})
are implemented using \code{long} in \C{}, which gives them at least 32
bits of precision.  Long integers have unlimited precision.  Floating
point numbers are implemented using \code{double} in \C{}.  All bets on
their precision are off unless you happen to know the machine you are
working with.
\indexii{numeric}{types}
\indexii{integer}{types}
\indexii{integer}{type}
\indexiii{long}{integer}{type}
\indexii{floating point}{type}
\indexii{complex number}{type}
\indexii{\C{}}{language}

Complex numbers have a real and imaginary part, which are both
implemented using \code{double} in \C{}.  To extract these parts from
a complex number \code{z}, use \code{z.real} and \code{z.imag}.  

Numbers are created by numeric literals or as the result of built-in
functions and operators.  Unadorned integer literals (including hex
and octal numbers) yield plain integers.  Integer literals with an \samp{L}
or \samp{l} suffix yield long integers
(\samp{L} is preferred because \samp{1l} looks too much like eleven!).
Numeric literals containing a decimal point or an exponent sign yield
floating point numbers.  Appending \samp{j} or \samp{J} to a numeric
literal yields a complex number.
\indexii{numeric}{literals}
\indexii{integer}{literals}
\indexiii{long}{integer}{literals}
\indexii{floating point}{literals}
\indexii{complex number}{literals}
\indexii{hexadecimal}{literals}
\indexii{octal}{literals}

Python fully supports mixed arithmetic: when a binary arithmetic
operator has operands of different numeric types, the operand with the
``smaller'' type is converted to that of the other, where plain
integer is smaller than long integer is smaller than floating point is
smaller than complex.
Comparisons between numbers of mixed type use the same rule.%
\footnote{As a consequence, the list \code{[1, 2]} is considered equal
	to \code{[1.0, 2.0]}, and similar for tuples.}
The functions \code{int()}, \code{long()}, \code{float()},
and \code{complex()} can be used
to coerce numbers to a specific type.
\index{arithmetic}
\bifuncindex{int}
\bifuncindex{long}
\bifuncindex{float}
\bifuncindex{complex}

All numeric types support the following operations, sorted by
ascending priority (operations in the same box have the same
priority; all numeric operations have a higher priority than
comparison operations):

\begin{tableiii}{|c|l|c|}{code}{Operation}{Result}{Notes}
  \lineiii{\var{x} + \var{y}}{sum of \var{x} and \var{y}}{}
  \lineiii{\var{x} - \var{y}}{difference of \var{x} and \var{y}}{}
  \hline
  \lineiii{\var{x} * \var{y}}{product of \var{x} and \var{y}}{}
  \lineiii{\var{x} / \var{y}}{quotient of \var{x} and \var{y}}{(1)}
  \lineiii{\var{x} \%{} \var{y}}{remainder of \code{\var{x} / \var{y}}}{}
  \hline
  \lineiii{-\var{x}}{\var{x} negated}{}
  \lineiii{+\var{x}}{\var{x} unchanged}{}
  \hline
  \lineiii{abs(\var{x})}{absolute value or magnitude of \var{x}}{}
  \lineiii{int(\var{x})}{\var{x} converted to integer}{(2)}
  \lineiii{long(\var{x})}{\var{x} converted to long integer}{(2)}
  \lineiii{float(\var{x})}{\var{x} converted to floating point}{}
  \lineiii{complex(\var{re},\var{im})}{a complex number with real part \var{re}, imaginary part \var{im}.  \var{im} defaults to zero.}{}
  \lineiii{divmod(\var{x}, \var{y})}{the pair \code{(\var{x} / \var{y}, \var{x} \%{} \var{y})}}{(3)}
  \lineiii{pow(\var{x}, \var{y})}{\var{x} to the power \var{y}}{}
  \lineiii{\var{x}**\var{y}}{\var{x} to the power \var{y}}{}
\end{tableiii}
\indexiii{operations on}{numeric}{types}

\noindent
Notes:
\begin{description}

\item[(1)]
For (plain or long) integer division, the result is an integer.
The result is always rounded towards minus infinity: 1/2 is 0, 
(-1)/2 is -1, 1/(-2) is -1, and (-1)/(-2) is 0.
\indexii{integer}{division}
\indexiii{long}{integer}{division}

\item[(2)]
Conversion from floating point to (long or plain) integer may round or
truncate as in \C{}; see functions \code{floor()} and \code{ceil()} in
module \code{math} for well-defined conversions.
\bifuncindex{floor}
\bifuncindex{ceil}
\indexii{numeric}{conversions}
\refbimodindex{math}
\indexii{\C{}}{language}

\item[(3)]
See the section on built-in functions for an exact definition.

\end{description}
% XXXJH exceptions: overflow (when? what operations?) zerodivision

\subsubsection{Bit-string Operations on Integer Types}
\nodename{Bit-string Operations}

Plain and long integer types support additional operations that make
sense only for bit-strings.  Negative numbers are treated as their 2's
complement value (for long integers, this assumes a sufficiently large
number of bits that no overflow occurs during the operation).

The priorities of the binary bit-wise operations are all lower than
the numeric operations and higher than the comparisons; the unary
operation \samp{\~} has the same priority as the other unary numeric
operations (\samp{+} and \samp{-}).

This table lists the bit-string operations sorted in ascending
priority (operations in the same box have the same priority):

\begin{tableiii}{|c|l|c|}{code}{Operation}{Result}{Notes}
  \lineiii{\var{x} | \var{y}}{bitwise \dfn{or} of \var{x} and \var{y}}{}
  \hline
  \lineiii{\var{x} \^{} \var{y}}{bitwise \dfn{exclusive or} of \var{x} and \var{y}}{}
  \hline
  \lineiii{\var{x} \&{} \var{y}}{bitwise \dfn{and} of \var{x} and \var{y}}{}
  \hline
  \lineiii{\var{x} << \var{n}}{\var{x} shifted left by \var{n} bits}{(1), (2)}
  \lineiii{\var{x} >> \var{n}}{\var{x} shifted right by \var{n} bits}{(1), (3)}
  \hline
  \hline
  \lineiii{\~\var{x}}{the bits of \var{x} inverted}{}
\end{tableiii}
\indexiii{operations on}{integer}{types}
\indexii{bit-string}{operations}
\indexii{shifting}{operations}
\indexii{masking}{operations}

\noindent
Notes:
\begin{description}
\item[(1)] Negative shift counts are illegal and cause a
\exception{ValueError} to be raised.
\item[(2)] A left shift by \var{n} bits is equivalent to
multiplication by \code{pow(2, \var{n})} without overflow check.
\item[(3)] A right shift by \var{n} bits is equivalent to
division by \code{pow(2, \var{n})} without overflow check.
\end{description}


\subsection{Sequence Types}
\label{typesseq}

There are three sequence types: strings, lists and tuples.

Strings literals are written in single or double quotes:
\code{'xyzzy'}, \code{"frobozz"}.  See Chapter 2 of the \emph{Python
Reference Manual} for more about string literals.  Lists are
constructed with square brackets, separating items with commas:
\code{[a, b, c]}.  Tuples are constructed by the comma operator (not
within square brackets), with or without enclosing parentheses, but an
empty tuple must have the enclosing parentheses, e.g.,
\code{a, b, c} or \code{()}.  A single item tuple must have a trailing
comma, e.g., \code{(d,)}.
\indexii{sequence}{types}
\indexii{string}{type}
\indexii{tuple}{type}
\indexii{list}{type}

Sequence types support the following operations.  The \samp{in} and
\samp{not in} operations have the same priorities as the comparison
operations.  The \samp{+} and \samp{*} operations have the same
priority as the corresponding numeric operations.\footnote{They must
have since the parser can't tell the type of the operands.}

This table lists the sequence operations sorted in ascending priority
(operations in the same box have the same priority).  In the table,
\var{s} and \var{t} are sequences of the same type; \var{n}, \var{i}
and \var{j} are integers:

\begin{tableiii}{|c|l|c|}{code}{Operation}{Result}{Notes}
  \lineiii{\var{x} in \var{s}}{\code{1} if an item of \var{s} is equal to \var{x}, else \code{0}}{}
  \lineiii{\var{x} not in \var{s}}{\code{0} if an item of \var{s} is
equal to \var{x}, else \code{1}}{}
  \hline
  \lineiii{\var{s} + \var{t}}{the concatenation of \var{s} and \var{t}}{}
  \hline
  \lineiii{\var{s} * \var{n}{\rm ,} \var{n} * \var{s}}{\var{n} copies of \var{s} concatenated}{(3)}
  \hline
  \lineiii{\var{s}[\var{i}]}{\var{i}'th item of \var{s}, origin 0}{(1)}
  \lineiii{\var{s}[\var{i}:\var{j}]}{slice of \var{s} from \var{i} to \var{j}}{(1), (2)}
  \hline
  \lineiii{len(\var{s})}{length of \var{s}}{}
  \lineiii{min(\var{s})}{smallest item of \var{s}}{}
  \lineiii{max(\var{s})}{largest item of \var{s}}{}
\end{tableiii}
\indexiii{operations on}{sequence}{types}
\bifuncindex{len}
\bifuncindex{min}
\bifuncindex{max}
\indexii{concatenation}{operation}
\indexii{repetition}{operation}
\indexii{subscript}{operation}
\indexii{slice}{operation}
\opindex{in}
\opindex{not in}

\noindent
Notes:

\begin{description}
  
\item[(1)] If \var{i} or \var{j} is negative, the index is relative to
  the end of the string, i.e., \code{len(\var{s}) + \var{i}} or
  \code{len(\var{s}) + \var{j}} is substituted.  But note that \code{-0} is
  still \code{0}.
  
\item[(2)] The slice of \var{s} from \var{i} to \var{j} is defined as
  the sequence of items with index \var{k} such that \code{\var{i} <=
  \var{k} < \var{j}}.  If \var{i} or \var{j} is greater than
  \code{len(\var{s})}, use \code{len(\var{s})}.  If \var{i} is omitted,
  use \code{0}.  If \var{j} is omitted, use \code{len(\var{s})}.  If
  \var{i} is greater than or equal to \var{j}, the slice is empty.

\item[(3)] Values of \var{n} less than \code{0} are treated as
  \code{0} (which yields an empty sequence of the same type as
  \var{s}).

\end{description}

\subsubsection{More String Operations}

String objects have one unique built-in operation: the \code{\%}
operator (modulo) with a string left argument interprets this string
as a \C{} \cfunction{sprintf()} format string to be applied to the
right argument, and returns the string resulting from this formatting
operation.

The right argument should be a tuple with one item for each argument
required by the format string; if the string requires a single
argument, the right argument may also be a single non-tuple object.%
\footnote{A tuple object in this case should be a singleton.}
The following format characters are understood:
\%, c, s, i, d, u, o, x, X, e, E, f, g, G.
Width and precision may be a * to specify that an integer argument
specifies the actual width or precision.  The flag characters -, +,
blank, \# and 0 are understood.  The size specifiers h, l or L may be
present but are ignored.  The \code{\%s} conversion takes any Python
object and converts it to a string using \code{str()} before
formatting it.  The ANSI features \code{\%p} and \code{\%n}
are not supported.  Since Python strings have an explicit length,
\code{\%s} conversions don't assume that \code{'\e0'} is the end of
the string.

For safety reasons, floating point precisions are clipped to 50;
\code{\%f} conversions for numbers whose absolute value is over 1e25
are replaced by \code{\%g} conversions.%
\footnote{These numbers are fairly arbitrary.  They are intended to
avoid printing endless strings of meaningless digits without hampering
correct use and without having to know the exact precision of floating
point values on a particular machine.}
All other errors raise exceptions.

If the right argument is a dictionary (or any kind of mapping), then
the formats in the string must have a parenthesized key into that
dictionary inserted immediately after the \code{\%} character, and
each format formats the corresponding entry from the mapping.  E.g.
\begin{verbatim}
>>> count = 2
>>> language = 'Python'
>>> print '%(language)s has %(count)03d quote types.' % vars()
Python has 002 quote types.
>>> 
\end{verbatim}
In this case no * specifiers may occur in a format (since they
require a sequential parameter list).

Additional string operations are defined in standard module
\code{string} and in built-in module \code{re}.
\refstmodindex{string}
\refbimodindex{re}

\subsubsection{Mutable Sequence Types}

List objects support additional operations that allow in-place
modification of the object.
These operations would be supported by other mutable sequence types
(when added to the language) as well.
Strings and tuples are immutable sequence types and such objects cannot
be modified once created.
The following operations are defined on mutable sequence types (where
\var{x} is an arbitrary object):
\indexiii{mutable}{sequence}{types}
\indexii{list}{type}

\begin{tableiii}{|c|l|c|}{code}{Operation}{Result}{Notes}
  \lineiii{\var{s}[\var{i}] = \var{x}}
	{item \var{i} of \var{s} is replaced by \var{x}}{}
  \lineiii{\var{s}[\var{i}:\var{j}] = \var{t}}
  	{slice of \var{s} from \var{i} to \var{j} is replaced by \var{t}}{}
  \lineiii{del \var{s}[\var{i}:\var{j}]}
	{same as \code{\var{s}[\var{i}:\var{j}] = []}}{}
  \lineiii{\var{s}.append(\var{x})}
	{same as \code{\var{s}[len(\var{s}):len(\var{s})] = [\var{x}]}}{}
  \lineiii{\var{s}.count(\var{x})}
	{return number of \var{i}'s for which \code{\var{s}[\var{i}] == \var{x}}}{}
  \lineiii{\var{s}.index(\var{x})}
	{return smallest \var{i} such that \code{\var{s}[\var{i}] == \var{x}}}{(1)}
  \lineiii{\var{s}.insert(\var{i}, \var{x})}
	{same as \code{\var{s}[\var{i}:\var{i}] = [\var{x}]}
	  if \code{\var{i} >= 0}}{}
  \lineiii{\var{s}.remove(\var{x})}
	{same as \code{del \var{s}[\var{s}.index(\var{x})]}}{(1)}
  \lineiii{\var{s}.reverse()}
	{reverses the items of \var{s} in place}{(3)}
  \lineiii{\var{s}.sort()}
	{sort the items of \var{s} in place}{(2), (3)}
\end{tableiii}
\indexiv{operations on}{mutable}{sequence}{types}
\indexiii{operations on}{sequence}{types}
\indexiii{operations on}{list}{type}
\indexii{subscript}{assignment}
\indexii{slice}{assignment}
\stindex{del}
\setindexsubitem{(list method)}
\ttindex{append}
\ttindex{count}
\ttindex{index}
\ttindex{insert}
\ttindex{remove}
\ttindex{reverse}
\ttindex{sort}

\noindent
Notes:
\begin{description}
\item[(1)] Raises an exception when \var{x} is not found in \var{s}.
  
\item[(2)] The \code{sort()} method takes an optional argument
  specifying a comparison function of two arguments (list items) which
  should return \code{-1}, \code{0} or \code{1} depending on whether the
  first argument is considered smaller than, equal to, or larger than the
  second argument.  Note that this slows the sorting process down
  considerably; e.g. to sort a list in reverse order it is much faster
  to use calls to \code{sort()} and \code{reverse()} than to use
  \code{sort()} with a comparison function that reverses the ordering of
  the elements.

\item[(3)] The \code{sort()} and \code{reverse()} methods modify the
list in place for economy of space when sorting or reversing a large
list.  They don't return the sorted or reversed list to remind you of
this side effect.

\end{description}


\subsection{Mapping Types}
\label{typesmapping}

A \dfn{mapping} object maps values of one type (the key type) to
arbitrary objects.  Mappings are mutable objects.  There is currently
only one standard mapping type, the \dfn{dictionary}.  A dictionary's keys are
almost arbitrary values.  The only types of values not acceptable as
keys are values containing lists or dictionaries or other mutable
types that are compared by value rather than by object identity.
Numeric types used for keys obey the normal rules for numeric
comparison: if two numbers compare equal (e.g. \code{1} and
\code{1.0}) then they can be used interchangeably to index the same
dictionary entry.

\indexii{mapping}{types}
\indexii{dictionary}{type}

Dictionaries are created by placing a comma-separated list of
\code{\var{key}:\,\var{value}} pairs within braces, for example:
\code{\{'jack':\,4098, 'sjoerd':\,4127\}} or
\code{\{4098:\,'jack', 4127:\,'sjoerd'\}}.

The following operations are defined on mappings (where \var{a} is a
mapping, \var{k} is a key and \var{x} is an arbitrary object):

\begin{tableiii}{|c|l|c|}{code}{Operation}{Result}{Notes}
  \lineiii{len(\var{a})}{the number of items in \var{a}}{}
  \lineiii{\var{a}[\var{k}]}{the item of \var{a} with key \var{k}}{(1)}
  \lineiii{\var{a}[\var{k}] = \var{x}}{set \code{\var{a}[\var{k}]} to \var{x}}{}
  \lineiii{del \var{a}[\var{k}]}{remove \code{\var{a}[\var{k}]} from \var{a}}{(1)}
  \lineiii{\var{a}.clear()}{remove all items from \code{a}}{}
  \lineiii{\var{a}.copy()}{a (shallow) copy of \code{a}}{}
  \lineiii{\var{a}.has_key(\var{k})}{\code{1} if \var{a} has a key \var{k}, else \code{0}}{}
  \lineiii{\var{a}.items()}{a copy of \var{a}'s list of (key, item) pairs}{(2)}
  \lineiii{\var{a}.keys()}{a copy of \var{a}'s list of keys}{(2)}
  \lineiii{\var{a}.update(\var{b})}{\code{for k, v in \var{b}.items(): \var{a}[k] = v}}{(3)}
  \lineiii{\var{a}.values()}{a copy of \var{a}'s list of values}{(2)}
  \lineiii{\var{a}.get(\var{k}, \var{f})}{the item of \var{a} with key \var{k}}{(4)}
\end{tableiii}
\indexiii{operations on}{mapping}{types}
\indexiii{operations on}{dictionary}{type}
\stindex{del}
\bifuncindex{len}
\setindexsubitem{(dictionary method)}
\ttindex{keys}
\ttindex{has_key}

\noindent
Notes:
\begin{description}
\item[(1)] Raises an exception if \var{k} is not in the map.

\item[(2)] Keys and values are listed in random order.

\item[(3)] \var{b} must be of the same type as \var{a}.

\item[(4)] Never raises an exception if \var{k} is not in the map,
instead it returns \var{f}.  \var{f} is optional, when not provided
and \var{k} is not in the map, \code{None} is returned.
\end{description}


\subsection{Other Built-in Types}
\label{typesother}

The interpreter supports several other kinds of objects.
Most of these support only one or two operations.

\subsubsection{Modules}

The only special operation on a module is attribute access:
\code{\var{m}.\var{name}}, where \var{m} is a module and \var{name} accesses
a name defined in \var{m}'s symbol table.  Module attributes can be
assigned to.  (Note that the \code{import} statement is not, strictly
spoken, an operation on a module object; \code{import \var{foo}} does not
require a module object named \var{foo} to exist, rather it requires
an (external) \emph{definition} for a module named \var{foo}
somewhere.)

A special member of every module is \code{__dict__}.
This is the dictionary containing the module's symbol table.
Modifying this dictionary will actually change the module's symbol
table, but direct assignment to the \code{__dict__} attribute is not
possible (i.e., you can write \code{\var{m}.__dict__['a'] = 1}, which
defines \code{\var{m}.a} to be \code{1}, but you can't write \code{\var{m}.__dict__ = \{\}}.

Modules are written like this: \code{<module 'sys'>}.

\subsubsection{Classes and Class Instances}
\nodename{Classes and Instances}

See Chapters 3 and 7 of the \emph{Python Reference Manual} for these.

\subsubsection{Functions}

Function objects are created by function definitions.  The only
operation on a function object is to call it:
\code{\var{func}(\var{argument-list})}.

There are really two flavors of function objects: built-in functions
and user-defined functions.  Both support the same operation (to call
the function), but the implementation is different, hence the
different object types.

The implementation adds two special read-only attributes:
\code{\var{f}.func_code} is a function's \dfn{code object} (see below) and
\code{\var{f}.func_globals} is the dictionary used as the function's
global name space (this is the same as \code{\var{m}.__dict__} where
\var{m} is the module in which the function \var{f} was defined).

\subsubsection{Methods}
\obindex{method}

Methods are functions that are called using the attribute notation.
There are two flavors: built-in methods (such as \code{append()} on
lists) and class instance methods.  Built-in methods are described
with the types that support them.

The implementation adds two special read-only attributes to class
instance methods: \code{\var{m}.im_self} is the object whose method this
is, and \code{\var{m}.im_func} is the function implementing the method.
Calling \code{\var{m}(\var{arg-1}, \var{arg-2}, {\rm \ldots},
\var{arg-n})} is completely equivalent to calling
\code{\var{m}.im_func(\var{m}.im_self, \var{arg-1}, \var{arg-2}, {\rm
\ldots}, \var{arg-n})}.

See the \emph{Python Reference Manual} for more information.

\subsubsection{Code Objects}
\obindex{code}

Code objects are used by the implementation to represent
``pseudo-compiled'' executable Python code such as a function body.
They differ from function objects because they don't contain a
reference to their global execution environment.  Code objects are
returned by the built-in \code{compile()} function and can be
extracted from function objects through their \code{func_code}
attribute.
\bifuncindex{compile}
\ttindex{func_code}

A code object can be executed or evaluated by passing it (instead of a
source string) to the \code{exec} statement or the built-in
\code{eval()} function.
\stindex{exec}
\bifuncindex{eval}

See the \emph{Python Reference Manual} for more information.

\subsubsection{Type Objects}

Type objects represent the various object types.  An object's type is
accessed by the built-in function \code{type()}.  There are no special
operations on types.  The standard module \code{types} defines names
for all standard built-in types.
\bifuncindex{type}
\refstmodindex{types}

Types are written like this: \code{<type 'int'>}.

\subsubsection{The Null Object}

This object is returned by functions that don't explicitly return a
value.  It supports no special operations.  There is exactly one null
object, named \code{None} (a built-in name).

It is written as \code{None}.

\subsubsection{File Objects}

File objects are implemented using \C{}'s \code{stdio} package and can be
created with the built-in function \code{open()} described under
Built-in Functions below.  They are also returned by some other
built-in functions and methods, e.g.\ \code{posix.popen()} and
\code{posix.fdopen()} and the \code{makefile()} method of socket
objects.
\bifuncindex{open}
\refbimodindex{posix}
\refbimodindex{socket}

When a file operation fails for an I/O-related reason, the exception
\code{IOError} is raised.  This includes situations where the
operation is not defined for some reason, like \code{seek()} on a tty
device or writing a file opened for reading.

Files have the following methods:


\setindexsubitem{(file method)}

\begin{funcdesc}{close}{}
  Close the file.  A closed file cannot be read or written anymore.
\end{funcdesc}

\begin{funcdesc}{flush}{}
  Flush the internal buffer, like \code{stdio}'s \code{fflush()}.
\end{funcdesc}

\begin{funcdesc}{isatty}{}
  Return \code{1} if the file is connected to a tty(-like) device, else
  \code{0}.
\end{funcdesc}

\begin{funcdesc}{fileno}{}
Return the integer ``file descriptor'' that is used by the underlying
implementation to request I/O operations from the operating system.
This can be useful for other, lower level interfaces that use file
descriptors, e.g. module \code{fcntl} or \code{os.read()} and friends.
\refbimodindex{fcntl}
\end{funcdesc}

\begin{funcdesc}{read}{\optional{size}}
  Read at most \var{size} bytes from the file (less if the read hits
  \EOF{} or no more data is immediately available on a pipe, tty or
  similar device).  If the \var{size} argument is negative or omitted,
  read all data until \EOF{} is reached.  The bytes are returned as a string
  object.  An empty string is returned when \EOF{} is encountered
  immediately.  (For certain files, like ttys, it makes sense to
  continue reading after an \EOF{} is hit.)
\end{funcdesc}

\begin{funcdesc}{readline}{\optional{size}}
  Read one entire line from the file.  A trailing newline character is
  kept in the string%
\footnote{The advantage of leaving the newline on is that an empty string 
	can be returned to mean \EOF{} without being ambiguous.  Another 
	advantage is that (in cases where it might matter, e.g. if you 
	want to make an exact copy of a file while scanning its lines) 
	you can tell whether the last line of a file ended in a newline
	or not (yes this happens!).}
  (but may be absent when a file ends with an
  incomplete line).  If the \var{size} argument is present and
  non-negative, it is a maximum byte count (including the trailing
  newline) and an incomplete line may be returned.
  An empty string is returned when \EOF{} is hit
  immediately.  Note: unlike \code{stdio}'s \code{fgets()}, the returned
  string contains null characters (\code{'\e 0'}) if they occurred in the
  input.
\end{funcdesc}

\begin{funcdesc}{readlines}{\optional{sizehint}}
  Read until \EOF{} using \code{readline()} and return a list containing
  the lines thus read.  If the optional \var{sizehint} argument is
  present, instead of reading up to \EOF{}, whole lines totalling
  approximately \var{sizehint} bytes (possibly after rounding up to an
  internal buffer size) are read.
\end{funcdesc}

\begin{funcdesc}{seek}{offset\, whence}
  Set the file's current position, like \code{stdio}'s \code{fseek()}.
  The \var{whence} argument is optional and defaults to \code{0}
  (absolute file positioning); other values are \code{1} (seek
  relative to the current position) and \code{2} (seek relative to the
  file's end).  There is no return value.
\end{funcdesc}

\begin{funcdesc}{tell}{}
  Return the file's current position, like \code{stdio}'s \code{ftell()}.
\end{funcdesc}

\begin{funcdesc}{truncate}{\optional{size}}
Truncate the file's size.  If the optional size argument present, the
file is truncated to (at most) that size.  The size defaults to the
current position.  Availability of this function depends on the
operating system version (e.g., not all \UNIX{} versions support this
operation).
\end{funcdesc}

\begin{funcdesc}{write}{str}
Write a string to the file.  There is no return value.  Note: due to
buffering, the string may not actually show up in the file until
the \code{flush()} or \code{close()} method is called.
\end{funcdesc}

\begin{funcdesc}{writelines}{list}
Write a list of strings to the file.  There is no return value.
(The name is intended to match \code{readlines}; \code{writelines}
does not add line separators.)
\end{funcdesc}

File objects also offer the following attributes:

\setindexsubitem{(file attribute)}

\begin{datadesc}{closed}
Boolean indicating the current state of the file object.  This is a
read-only attribute; the \method{close()} method changes the value.
\end{datadesc}

\begin{datadesc}{mode}
The I/O mode for the file.  If the file was created using the
\function{open()} built-in function, this will be the value of the
\var{mode} parameter.  This is a read-only attribute.
\end{datadesc}

\begin{datadesc}{name}
If the file object was created using \function{open()}, the name of
the file.  Otherwise, some string that indicates the source of the
file object, of the form \samp{<\mbox{\ldots}>}.  This is a read-only
attribute.
\end{datadesc}

\begin{datadesc}{softspace}
Boolean that indicates whether a space character needs to be printed
before another value when using the \keyword{print} statement.
Classes that are trying to simulate a file object should also have a
writable \code{softspace} attribute, which should be initialized to
zero.  This will be automatic for classes implemented in Python; types
implemented in \C{} will have to provide a writable \code{softspace}
attribute.
\end{datadesc}

\subsubsection{Internal Objects}

See the \emph{Python Reference Manual} for this information.  It
describes code objects, stack frame objects, traceback objects, and
slice objects.


\subsection{Special Attributes}
\label{specialattrs}

The implementation adds a few special read-only attributes to several
object types, where they are relevant:

\begin{itemize}

\item
\code{\var{x}.__dict__} is a dictionary of some sort used to store an
object's (writable) attributes;

\item
\code{\var{x}.__methods__} lists the methods of many built-in object types,
e.g., \code{[].__methods__} yields
\code{['append', 'count', 'index', 'insert', 'remove', 'reverse', 'sort']};

\item
\code{\var{x}.__members__} lists data attributes;

\item
\code{\var{x}.__class__} is the class to which a class instance belongs;

\item
\code{\var{x}.__bases__} is the tuple of base classes of a class object.

\end{itemize}

\section{\module{UserDict} ---
         Class wrapper for dictionary objects}

\declaremodule{standard}{UserDict}
\modulesynopsis{Class wrapper for dictionary objects.}

This module defines a class that acts as a wrapper around
dictionary objects.  It is a useful base class for
your own dictionary-like classes, which can inherit from
them and override existing methods or add new ones.  In this way one
can add new behaviors to dictionaries.

The \module{UserDict} module defines the \class{UserDict} class:

\begin{classdesc}{UserDict}{\optional{initialdata}}
Return a class instance that simulates a dictionary.  The instance's
contents are kept in a regular dictionary, which is accessible via the
\member{data} attribute of \class{UserDict} instances.  If
\var{initialdata} is provided, \member{data} is initialized with its
contents; note that a reference to \var{initialdata} will not be kept, 
allowing it be used used for other purposes.
\end{classdesc}

In addition to supporting the methods and operations of mappings (see
section \ref{typesmapping}), \class{UserDict} instances provide the
following attribute:

\begin{memberdesc}{data}
A real dictionary used to store the contents of the \class{UserDict}
class.
\end{memberdesc}


\section{\module{UserList} ---
         Class wrapper for list objects}

\declaremodule{standard}{UserList}
\modulesynopsis{Class wrapper for list objects.}


This module defines a class that acts as a wrapper around
list objects.  It is a useful base class for
your own list-like classes, which can inherit from
them and override existing methods or add new ones.  In this way one
can add new behaviors to lists.

The \module{UserList} module defines the \class{UserList} class:

\begin{classdesc}{UserList}{\optional{list}}
Return a class instance that simulates a list.  The instance's
contents are kept in a regular list, which is accessible via the
\member{data} attribute of \class{UserList} instances.  The instance's
contents are initially set to a copy of \var{list}, defaulting to the
empty list \code{[]}.  \var{list} can be either a regular Python list,
or an instance of \class{UserList} (or a subclass).
\end{classdesc}

In addition to supporting the methods and operations of mutable
sequences (see section \ref{typesseq}), \class{UserList} instances
provide the following attribute:

\begin{memberdesc}{data}
A real Python list object used to store the contents of the
\class{UserList} class.
\end{memberdesc}

\strong{Subclassing requirements:}
Subclasses of \class{UserList} are expect to offer a constructor which
can be called with either no arguments or one argument.  List
operations which return a new sequence attempt to create an instance
of the actual implementation class.  To do so, it assumes that the
constructor can be called with a single parameter, which is a sequence
object used as a data source.

If a derived class does not wish to comply with this requirement, all
of the special methods supported by this class will need to be
overridden; please consult the sources for information about the
methods which need to be provided in that case.

\versionchanged[Python versions 1.5.2 and 1.6 also required that the
                constructor be callable with no parameters, and offer
                a mutable \member{data} attribute.  Earlier versions
                of Python did not attempt to create instances of the
                derived class]{2.0}


\section{\module{UserString} ---
         Class wrapper for string objects}

\declaremodule{standard}{UserString}
\modulesynopsis{Class wrapper for string objects.}
\moduleauthor{Peter Funk}{pf@artcom-gmbh.de}
\sectionauthor{Peter Funk}{pf@artcom-gmbh.de}

This module defines a class that acts as a wrapper around string
objects.  It is a useful base class for your own string-like classes,
which can inherit from them and override existing methods or add new
ones.  In this way one can add new behaviors to strings.

It should be noted that these classes are highly inefficient compared
to real string or Unicode objects; this is especially the case for
\class{MutableString}.

The \module{UserString} module defines the following classes:

\begin{classdesc}{UserString}{\optional{sequence}}
Return a class instance that simulates a string or a Unicode string
object.  The instance's content is kept in a regular string or Unicode
string object, which is accessible via the \member{data} attribute of
\class{UserString} instances.  The instance's contents are initially
set to a copy of \var{sequence}.  \var{sequence} can be either a
regular Python string or Unicode string, an instance of
\class{UserString} (or a subclass) or an arbitrary sequence which can
be converted into a string using the built-in \function{str()} function.
\end{classdesc}

\begin{classdesc}{MutableString}{\optional{sequence}}
This class is derived from the \class{UserString} above and redefines
strings to be \emph{mutable}.  Mutable strings can't be used as
dictionary keys, because dictionaries require \emph{immutable} objects as
keys.  The main intention of this class is to serve as an educational
example for inheritance and necessity to remove (override) the
\method{__hash__()} method in order to trap attempts to use a
mutable object as dictionary key, which would be otherwise very
error prone and hard to track down.
\end{classdesc}

In addition to supporting the methods and operations of string and
Unicode objects (see section \ref{string-methods}, ``String
Methods''), \class{UserString} instances provide the following
attribute:

\begin{memberdesc}{data}
A real Python string or Unicode object used to store the content of the
\class{UserString} class.
\end{memberdesc}

\section{\module{operator} ---
         Standard operators as functions.}
\declaremodule{builtin}{operator}
\sectionauthor{Skip Montanaro}{skip@automatrix.com}

\modulesynopsis{All Python's standard operators as built-in functions.}


The \module{operator} module exports a set of functions implemented in C
corresponding to the intrinsic operators of Python.  For example,
\code{operator.add(x, y)} is equivalent to the expression \code{x+y}.  The
function names are those used for special class methods; variants without
leading and trailing \samp{__} are also provided for convenience.

The functions fall into categories that perform object comparisons,
logical operations, mathematical operations, sequence operations, and
abstract type tests.

The object comparison functions are useful for all objects, and are
named after the rich comparison operators they support:

\begin{funcdesc}{lt}{a, b}
\funcline{le}{a, b}
\funcline{eq}{a, b}
\funcline{ne}{a, b}
\funcline{ge}{a, b}
\funcline{gt}{a, b}
\funcline{__lt__}{a, b}
\funcline{__le__}{a, b}
\funcline{__eq__}{a, b}
\funcline{__ne__}{a, b}
\funcline{__ge__}{a, b}
\funcline{__gt__}{a, b}
Perform ``rich comparisons'' between \var{a} and \var{b}. Specifically,
\code{lt(\var{a}, \var{b})} is equivalent to \code{\var{a} < \var{b}},
\code{le(\var{a}, \var{b})} is equivalent to \code{\var{a} <= \var{b}},
\code{eq(\var{a}, \var{b})} is equivalent to \code{\var{a} == \var{b}},
\code{ne(\var{a}, \var{b})} is equivalent to \code{\var{a} != \var{b}},
\code{gt(\var{a}, \var{b})} is equivalent to \code{\var{a} > \var{b}}
and
\code{ge(\var{a}, \var{b})} is equivalent to \code{\var{a} >= \var{b}}.
Note that unlike the built-in \function{cmp()}, these functions can
return any value, which may or may not be interpretable as a Boolean
value.  See the \citetitle[../ref/ref.html]{Python Reference Manual}
for more information about rich comparisons.
\versionadded{2.2}
\end{funcdesc}


The logical operations are also generally applicable to all objects,
and support truth tests, identity tests, and boolean operations:

\begin{funcdesc}{not_}{o}
\funcline{__not__}{o}
Return the outcome of \keyword{not} \var{o}.  (Note that there is no
\method{__not__()} method for object instances; only the interpreter
core defines this operation.  The result is affected by the
\method{__nonzero__()} and \method{__len__()} methods.)
\end{funcdesc}

\begin{funcdesc}{truth}{o}
Return \constant{True} if \var{o} is true, and \constant{False}
otherwise.  This is equivalent to using the \class{bool}
constructor.
\end{funcdesc}

\begin{funcdesc}{is_}{a, b}
Return \code{\var{a} is \var{b}}.  Tests object identity.
\versionadded{2.3}
\end{funcdesc}

\begin{funcdesc}{is_not}{a, b}
Return \code{\var{a} is not \var{b}}.  Tests object identity.
\versionadded{2.3}
\end{funcdesc}


The mathematical and bitwise operations are the most numerous:

\begin{funcdesc}{abs}{o}
\funcline{__abs__}{o}
Return the absolute value of \var{o}.
\end{funcdesc}

\begin{funcdesc}{add}{a, b}
\funcline{__add__}{a, b}
Return \var{a} \code{+} \var{b}, for \var{a} and \var{b} numbers.
\end{funcdesc}

\begin{funcdesc}{and_}{a, b}
\funcline{__and__}{a, b}
Return the bitwise and of \var{a} and \var{b}.
\end{funcdesc}

\begin{funcdesc}{div}{a, b}
\funcline{__div__}{a, b}
Return \var{a} \code{/} \var{b} when \code{__future__.division} is not
in effect.  This is also known as ``classic'' division.
\end{funcdesc}

\begin{funcdesc}{floordiv}{a, b}
\funcline{__floordiv__}{a, b}
Return \var{a} \code{//} \var{b}.
\versionadded{2.2}
\end{funcdesc}

\begin{funcdesc}{inv}{o}
\funcline{invert}{o}
\funcline{__inv__}{o}
\funcline{__invert__}{o}
Return the bitwise inverse of the number \var{o}.  This is equivalent
to \code{\textasciitilde}\var{o}.  The names \function{invert()} and
\function{__invert__()} were added in Python 2.0.
\end{funcdesc}

\begin{funcdesc}{lshift}{a, b}
\funcline{__lshift__}{a, b}
Return \var{a} shifted left by \var{b}.
\end{funcdesc}

\begin{funcdesc}{mod}{a, b}
\funcline{__mod__}{a, b}
Return \var{a} \code{\%} \var{b}.
\end{funcdesc}

\begin{funcdesc}{mul}{a, b}
\funcline{__mul__}{a, b}
Return \var{a} \code{*} \var{b}, for \var{a} and \var{b} numbers.
\end{funcdesc}

\begin{funcdesc}{neg}{o}
\funcline{__neg__}{o}
Return \var{o} negated.
\end{funcdesc}

\begin{funcdesc}{or_}{a, b}
\funcline{__or__}{a, b}
Return the bitwise or of \var{a} and \var{b}.
\end{funcdesc}

\begin{funcdesc}{pos}{o}
\funcline{__pos__}{o}
Return \var{o} positive.
\end{funcdesc}

\begin{funcdesc}{pow}{a, b}
\funcline{__pow__}{a, b}
Return \var{a} \code{**} \var{b}, for \var{a} and \var{b} numbers.
\versionadded{2.3}
\end{funcdesc}

\begin{funcdesc}{rshift}{a, b}
\funcline{__rshift__}{a, b}
Return \var{a} shifted right by \var{b}.
\end{funcdesc}

\begin{funcdesc}{sub}{a, b}
\funcline{__sub__}{a, b}
Return \var{a} \code{-} \var{b}.
\end{funcdesc}

\begin{funcdesc}{truediv}{a, b}
\funcline{__truediv__}{a, b}
Return \var{a} \code{/} \var{b} when \code{__future__.division} is in
effect.  This is also known as division.
\versionadded{2.2}
\end{funcdesc}

\begin{funcdesc}{xor}{a, b}
\funcline{__xor__}{a, b}
Return the bitwise exclusive or of \var{a} and \var{b}.
\end{funcdesc}


Operations which work with sequences include:

\begin{funcdesc}{concat}{a, b}
\funcline{__concat__}{a, b}
Return \var{a} \code{+} \var{b} for \var{a} and \var{b} sequences.
\end{funcdesc}

\begin{funcdesc}{contains}{a, b}
\funcline{__contains__}{a, b}
Return the outcome of the test \var{b} \code{in} \var{a}.
Note the reversed operands.  The name \function{__contains__()} was
added in Python 2.0.
\end{funcdesc}

\begin{funcdesc}{countOf}{a, b}
Return the number of occurrences of \var{b} in \var{a}.
\end{funcdesc}

\begin{funcdesc}{delitem}{a, b}
\funcline{__delitem__}{a, b}
Remove the value of \var{a} at index \var{b}.
\end{funcdesc}

\begin{funcdesc}{delslice}{a, b, c}
\funcline{__delslice__}{a, b, c}
Delete the slice of \var{a} from index \var{b} to index \var{c}\code{-1}.
\end{funcdesc}

\begin{funcdesc}{getitem}{a, b}
\funcline{__getitem__}{a, b}
Return the value of \var{a} at index \var{b}.
\end{funcdesc}

\begin{funcdesc}{getslice}{a, b, c}
\funcline{__getslice__}{a, b, c}
Return the slice of \var{a} from index \var{b} to index \var{c}\code{-1}.
\end{funcdesc}

\begin{funcdesc}{indexOf}{a, b}
Return the index of the first of occurrence of \var{b} in \var{a}.
\end{funcdesc}

\begin{funcdesc}{repeat}{a, b}
\funcline{__repeat__}{a, b}
Return \var{a} \code{*} \var{b} where \var{a} is a sequence and
\var{b} is an integer.
\end{funcdesc}

\begin{funcdesc}{sequenceIncludes}{\unspecified}
\deprecated{2.0}{Use \function{contains()} instead.}
Alias for \function{contains()}.
\end{funcdesc}

\begin{funcdesc}{setitem}{a, b, c}
\funcline{__setitem__}{a, b, c}
Set the value of \var{a} at index \var{b} to \var{c}.
\end{funcdesc}

\begin{funcdesc}{setslice}{a, b, c, v}
\funcline{__setslice__}{a, b, c, v}
Set the slice of \var{a} from index \var{b} to index \var{c}\code{-1} to the
sequence \var{v}.
\end{funcdesc}


The \module{operator} module also defines a few predicates to test the
type of objects.  \note{Be careful not to misinterpret the
results of these functions; only \function{isCallable()} has any
measure of reliability with instance objects.  For example:}

\begin{verbatim}
>>> class C:
...     pass
... 
>>> import operator
>>> o = C()
>>> operator.isMappingType(o)
True
\end{verbatim}

\begin{funcdesc}{isCallable}{o}
\deprecated{2.0}{Use the \function{callable()} built-in function instead.}
Returns true if the object \var{o} can be called like a function,
otherwise it returns false.  True is returned for functions, bound and
unbound methods, class objects, and instance objects which support the
\method{__call__()} method.
\end{funcdesc}

\begin{funcdesc}{isMappingType}{o}
Returns true if the object \var{o} supports the mapping interface.
This is true for dictionaries and all instance objects defining
\method{__getitem__}.
\warning{There is no reliable way to test if an instance
supports the complete mapping protocol since the interface itself is
ill-defined.  This makes this test less useful than it otherwise might
be.}
\end{funcdesc}

\begin{funcdesc}{isNumberType}{o}
Returns true if the object \var{o} represents a number.  This is true
for all numeric types implemented in C.
\warning{There is no reliable way to test if an instance
supports the complete numeric interface since the interface itself is
ill-defined.  This makes this test less useful than it otherwise might
be.}
\end{funcdesc}

\begin{funcdesc}{isSequenceType}{o}
Returns true if the object \var{o} supports the sequence protocol.
This returns true for all objects which define sequence methods in C,
and for all instance objects defining \method{__getitem__}.
\warning{There is no reliable
way to test if an instance supports the complete sequence interface
since the interface itself is ill-defined.  This makes this test less
useful than it otherwise might be.}
\end{funcdesc}


Example: Build a dictionary that maps the ordinals from \code{0} to
\code{255} to their character equivalents.

\begin{verbatim}
>>> import operator
>>> d = {}
>>> keys = range(256)
>>> vals = map(chr, keys)
>>> map(operator.setitem, [d]*len(keys), keys, vals)
\end{verbatim}


The \module{operator} module also defines tools for generalized attribute
and item lookups.  These are useful for making fast field extractors
as arguments for \function{map()}, \function{sorted()},
\method{itertools.groupby()}, or other functions that expect a
function argument.

\begin{funcdesc}{attrgetter}{attr\optional{, args...}}
Return a callable object that fetches \var{attr} from its operand.
If more than one attribute is requested, returns a tuple of attributes.
After, \samp{f=attrgetter('name')}, the call \samp{f(b)} returns
\samp{b.name}.  After, \samp{f=attrgetter('name', 'date')}, the call
\samp{f(b)} returns \samp{(b.name, b.date)}. 
\versionadded{2.4}
\versionchanged[Added support for multiple attributes]{2.5}
\end{funcdesc}
    
\begin{funcdesc}{itemgetter}{item\optional{, args...}}
Return a callable object that fetches \var{item} from its operand.
If more than one item is requested, returns a tuple of items.
After, \samp{f=itemgetter(2)}, the call \samp{f(b)} returns
\samp{b[2]}.
After, \samp{f=itemgetter(2,5,3)}, the call \samp{f(b)} returns
\samp{(b[2], b[5], b[3])}.		
\versionadded{2.4}
\versionchanged[Added support for multiple item extraction]{2.5}		
\end{funcdesc}

Examples:
                
\begin{verbatim}
>>> from operator import itemgetter
>>> inventory = [('apple', 3), ('banana', 2), ('pear', 5), ('orange', 1)]
>>> getcount = itemgetter(1)
>>> map(getcount, inventory)
[3, 2, 5, 1]
>>> sorted(inventory, key=getcount)
[('orange', 1), ('banana', 2), ('apple', 3), ('pear', 5)]
\end{verbatim}
                

\subsection{Mapping Operators to Functions \label{operator-map}}

This table shows how abstract operations correspond to operator
symbols in the Python syntax and the functions in the
\refmodule{operator} module.


\begin{tableiii}{l|c|l}{textrm}{Operation}{Syntax}{Function}
  \lineiii{Addition}{\code{\var{a} + \var{b}}}
          {\code{add(\var{a}, \var{b})}}
  \lineiii{Concatenation}{\code{\var{seq1} + \var{seq2}}}
          {\code{concat(\var{seq1}, \var{seq2})}}
  \lineiii{Containment Test}{\code{\var{o} in \var{seq}}}
          {\code{contains(\var{seq}, \var{o})}}
  \lineiii{Division}{\code{\var{a} / \var{b}}}
          {\code{div(\var{a}, \var{b}) \#} without \code{__future__.division}}
  \lineiii{Division}{\code{\var{a} / \var{b}}}
          {\code{truediv(\var{a}, \var{b}) \#} with \code{__future__.division}}
  \lineiii{Division}{\code{\var{a} // \var{b}}}
          {\code{floordiv(\var{a}, \var{b})}}
  \lineiii{Bitwise And}{\code{\var{a} \&\ \var{b}}}
          {\code{and_(\var{a}, \var{b})}}
  \lineiii{Bitwise Exclusive Or}{\code{\var{a} \^\ \var{b}}}
          {\code{xor(\var{a}, \var{b})}}
  \lineiii{Bitwise Inversion}{\code{\~{} \var{a}}}
          {\code{invert(\var{a})}}
  \lineiii{Bitwise Or}{\code{\var{a} | \var{b}}}
          {\code{or_(\var{a}, \var{b})}}
  \lineiii{Exponentiation}{\code{\var{a} ** \var{b}}}
          {\code{pow(\var{a}, \var{b})}}
  \lineiii{Identity}{\code{\var{a} is \var{b}}}
          {\code{is_(\var{a}, \var{b})}}
  \lineiii{Identity}{\code{\var{a} is not \var{b}}}
          {\code{is_not(\var{a}, \var{b})}}
  \lineiii{Indexed Assignment}{\code{\var{o}[\var{k}] = \var{v}}}
          {\code{setitem(\var{o}, \var{k}, \var{v})}}
  \lineiii{Indexed Deletion}{\code{del \var{o}[\var{k}]}}
          {\code{delitem(\var{o}, \var{k})}}
  \lineiii{Indexing}{\code{\var{o}[\var{k}]}}
          {\code{getitem(\var{o}, \var{k})}}
  \lineiii{Left Shift}{\code{\var{a} <\code{<} \var{b}}}
          {\code{lshift(\var{a}, \var{b})}}
  \lineiii{Modulo}{\code{\var{a} \%\ \var{b}}}
          {\code{mod(\var{a}, \var{b})}}
  \lineiii{Multiplication}{\code{\var{a} * \var{b}}}
          {\code{mul(\var{a}, \var{b})}}
  \lineiii{Negation (Arithmetic)}{\code{- \var{a}}}
          {\code{neg(\var{a})}}
  \lineiii{Negation (Logical)}{\code{not \var{a}}}
          {\code{not_(\var{a})}}
  \lineiii{Right Shift}{\code{\var{a} >\code{>} \var{b}}}
          {\code{rshift(\var{a}, \var{b})}}
  \lineiii{Sequence Repitition}{\code{\var{seq} * \var{i}}}
          {\code{repeat(\var{seq}, \var{i})}}
  \lineiii{Slice Assignment}{\code{\var{seq}[\var{i}:\var{j}]} = \var{values}}
          {\code{setslice(\var{seq}, \var{i}, \var{j}, \var{values})}}
  \lineiii{Slice Deletion}{\code{del \var{seq}[\var{i}:\var{j}]}}
          {\code{delslice(\var{seq}, \var{i}, \var{j})}}
  \lineiii{Slicing}{\code{\var{seq}[\var{i}:\var{j}]}}
          {\code{getslice(\var{seq}, \var{i}, \var{j})}}
  \lineiii{String Formatting}{\code{\var{s} \%\ \var{o}}}
          {\code{mod(\var{s}, \var{o})}}
  \lineiii{Subtraction}{\code{\var{a} - \var{b}}}
          {\code{sub(\var{a}, \var{b})}}
  \lineiii{Truth Test}{\code{\var{o}}}
          {\code{truth(\var{o})}}
  \lineiii{Ordering}{\code{\var{a} < \var{b}}}
          {\code{lt(\var{a}, \var{b})}}
  \lineiii{Ordering}{\code{\var{a} <= \var{b}}}
          {\code{le(\var{a}, \var{b})}}
  \lineiii{Equality}{\code{\var{a} == \var{b}}}
          {\code{eq(\var{a}, \var{b})}}
  \lineiii{Difference}{\code{\var{a} != \var{b}}}
          {\code{ne(\var{a}, \var{b})}}
  \lineiii{Ordering}{\code{\var{a} >= \var{b}}}
          {\code{ge(\var{a}, \var{b})}}
  \lineiii{Ordering}{\code{\var{a} > \var{b}}}
          {\code{gt(\var{a}, \var{b})}}
\end{tableiii}

\section{Standard Module \module{traceback}}
\declaremodule{standard}{traceback}

\modulesynopsis{Print or retrieve a stack traceback.}



This module provides a standard interface to extract, format and print
stack traces of Python programs.  It exactly mimics the behavior of
the Python interpreter when it prints a stack trace.  This is useful
when you want to print stack traces under program control, e.g. in a
``wrapper'' around the interpreter.

The module uses traceback objects --- this is the object type
that is stored in the variables \code{sys.exc_traceback} and
\code{sys.last_traceback} and returned as the third item from
\function{sys.exc_info()}.
\obindex{traceback}

The module defines the following functions:

\begin{funcdesc}{print_tb}{traceback\optional{, limit\optional{, file}}}
Print up to \var{limit} stack trace entries from \var{traceback}.  If
\var{limit} is omitted or \code{None}, all entries are printed.
If \var{file} is omitted or \code{None}, the output goes to
\code{sys.stderr}; otherwise it should be an open file or file-like
object to receive the output.
\end{funcdesc}

\begin{funcdesc}{extract_tb}{traceback\optional{, limit}}
Return a list of up to \var{limit} ``pre-processed'' stack trace
entries extracted from \var{traceback}.  It is useful for alternate
formatting of stack traces.  If \var{limit} is omitted or \code{None},
all entries are extracted.  A ``pre-processed'' stack trace entry is a
quadruple (\var{filename}, \var{line number}, \var{function name},
\var{line text}) representing the information that is usually printed
for a stack trace.  The \var{line text} is a string with leading and
trailing whitespace stripped; if the source is not available it is
\code{None}.
\end{funcdesc}

\begin{funcdesc}{print_exception}{type, value,
traceback\optional{, limit\optional{, file}}}
Print exception information and up to \var{limit} stack trace entries
from \var{traceback} to \var{file}.
This differs from \function{print_tb()} in the
following ways: (1) if \var{traceback} is not \code{None}, it prints a
header \samp{Traceback (innermost last):}; (2) it prints the
exception \var{type} and \var{value} after the stack trace; (3) if
\var{type} is \exception{SyntaxError} and \var{value} has the appropriate
format, it prints the line where the syntax error occurred with a
caret indicating the approximate position of the error.
\end{funcdesc}

\begin{funcdesc}{print_exc}{\optional{limit\optional{, file}}}
This is a shorthand for `\code{print_exception(sys.exc_type,}
\code{sys.exc_value,} \code{sys.exc_traceback,} \var{limit}\code{,}
\var{file}\code{)}'.  (In fact, it uses \code{sys.exc_info()} to
retrieve the same information in a thread-safe way.)
\end{funcdesc}

\begin{funcdesc}{print_last}{\optional{limit\optional{, file}}}
This is a shorthand for `\code{print_exception(sys.last_type,}
\code{sys.last_value,} \code{sys.last_traceback,} \var{limit}\code{,}
\var{file}\code{)}'.
\end{funcdesc}

\begin{funcdesc}{print_stack}{\optional{f\optional{, limit\optional{, file}}}}
This function prints a stack trace from its invocation point.  The
optional \var{f} argument can be used to specify an alternate stack
frame to start.  The optional \var{limit} and \var{file} arguments have the
same meaning as for \function{print_exception()}.
\end{funcdesc}

\begin{funcdesc}{extract_tb}{tb\optional{, limit}}
Return a list containing the raw (unformatted) traceback information
extracted from the traceback object \var{tb}.  The optional
\var{limit} argument has the same meaning as for
\function{print_exception()}.  The items in the returned list are
4-tuples containing the following values: filename, line number,
function name, and source text line.  The source text line is stripped 
of leading and trailing whitespace; it is \code{None} when the source
text file is unavailable.
\end{funcdesc}

\begin{funcdesc}{extract_stack}{\optional{f\optional{, limit}}}
Extract the raw traceback from the current stack frame.  The return
value has the same format as for \function{extract_tb()}.  The
optional \var{f} and \var{limit} arguments have the same meaning as
for \function{print_stack()}.
\end{funcdesc}

\begin{funcdesc}{format_list}{list}
Given a list of tuples as returned by \function{extract_tb()} or
\function{extract_stack()}, return a list of strings ready for
printing.  Each string in the resulting list corresponds to the item
with the same index in the argument list.  Each string ends in a
newline; the strings may contain internal newlines as well, for those
items whose source text line is not \code{None}.
\end{funcdesc}

\begin{funcdesc}{format_exception_only}{type, value}
Format the exception part of a traceback.  The arguments are the
exception type and value such as given by \code{sys.last_type} and
\code{sys.last_value}.  The return value is a list of strings, each
ending in a newline.  Normally, the list contains a single string;
however, for \code{SyntaxError} exceptions, it contains several lines
that (when printed) display detailed information about where the
syntax error occurred.  The message indicating which exception
occurred is the always last string in the list.
\end{funcdesc}

\begin{funcdesc}{format_exception}{type, value, tb\optional{, limit}}
Format a stack trace and the exception information.  The arguments 
have the same meaning as the corresponding arguments to
\function{print_exception()}.  The return value is a list of strings,
each ending in a newline and some containing internal newlines.  When
these lines are contatenated and printed, exactly the same text is
printed as does \function{print_exception()}.
\end{funcdesc}

\begin{funcdesc}{format_tb}{tb\optional{, limit}}
A shorthand for \code{format_list(extract_tb(\var{tb}, \var{limit}))}.
\end{funcdesc}

\begin{funcdesc}{format_stack}{\optional{f\optional{, limit}}}
A shorthand for \code{format_list(extract_stack(\var{f}, \var{limit}))}.
\end{funcdesc}

\begin{funcdesc}{tb_lineno}{tb}
This function returns the current line number set in the traceback
object.  This is normally the same as the \code{\var{tb}.tb_lineno}
field of the object, but when optimization is used (the -O flag) this
field is not updated correctly; this function calculates the correct
value.
\end{funcdesc}

A simple example follows:

\begin{verbatim}
import sys, traceback

def run_user_code(envdir):
    source = raw_input(">>> ")
    try:
        exec source in envdir
    except:
        print "Exception in user code:"
        print '-'*60
        traceback.print_exc(file=sys.stdout)
        print '-'*60

envdir = {}
while 1:
    run_user_code(envdir)
\end{verbatim}

\section{\module{pickle} ---
         Python object serialization}

\declaremodule{standard}{pickle}
\modulesynopsis{Convert Python objects to streams of bytes and back.}

\index{persistency}
\indexii{persistent}{objects}
\indexii{serializing}{objects}
\indexii{marshalling}{objects}
\indexii{flattening}{objects}
\indexii{pickling}{objects}


The \module{pickle} module implements a basic but powerful algorithm
for ``pickling'' (a.k.a.\ serializing, marshalling or flattening)
nearly arbitrary Python objects.  This is the act of converting
objects to a stream of bytes (and back: ``unpickling'').  This is a
more primitive notion than persistency --- although \module{pickle}
reads and writes file objects, it does not handle the issue of naming
persistent objects, nor the (even more complicated) area of concurrent
access to persistent objects.  The \module{pickle} module can
transform a complex object into a byte stream and it can transform the
byte stream into an object with the same internal structure.  The most
obvious thing to do with these byte streams is to write them onto a
file, but it is also conceivable to send them across a network or
store them in a database.  The module
\refmodule{shelve}\refstmodindex{shelve} provides a simple interface
to pickle and unpickle objects on DBM-style database files.


\strong{Note:} The \module{pickle} module is rather slow.  A
reimplementation of the same algorithm in C, which is up to 1000 times
faster, is available as the
\refmodule{cPickle}\refbimodindex{cPickle} module.  This has the same
interface except that \class{Pickler} and \class{Unpickler} are
factory functions, not classes (so they cannot be used as base classes
for inheritance).

Unlike the built-in module \refmodule{marshal}\refbimodindex{marshal},
\module{pickle} handles the following correctly:


\begin{itemize}

\item recursive objects (objects containing references to themselves)

\item object sharing (references to the same object in different places)

\item user-defined classes and their instances

\end{itemize}

The data format used by \module{pickle} is Python-specific.  This has
the advantage that there are no restrictions imposed by external
standards such as
XDR\index{XDR}\index{External Data Representation} (which can't
represent pointer sharing); however it means that non-Python programs
may not be able to reconstruct pickled Python objects.

By default, the \module{pickle} data format uses a printable \ASCII{}
representation.  This is slightly more voluminous than a binary
representation.  The big advantage of using printable \ASCII{} (and of
some other characteristics of \module{pickle}'s representation) is that
for debugging or recovery purposes it is possible for a human to read
the pickled file with a standard text editor.

A binary format, which is slightly more efficient, can be chosen by
specifying a nonzero (true) value for the \var{bin} argument to the
\class{Pickler} constructor or the \function{dump()} and \function{dumps()}
functions.  The binary format is not the default because of backwards
compatibility with the Python 1.4 pickle module.  In a future version,
the default may change to binary.

The \module{pickle} module doesn't handle code objects, which the
\refmodule{marshal}\refbimodindex{marshal} module does.  I suppose
\module{pickle} could, and maybe it should, but there's probably no
great need for it right now (as long as \refmodule{marshal} continues
to be used for reading and writing code objects), and at least this
avoids the possibility of smuggling Trojan horses into a program.

For the benefit of persistency modules written using \module{pickle}, it
supports the notion of a reference to an object outside the pickled
data stream.  Such objects are referenced by a name, which is an
arbitrary string of printable \ASCII{} characters.  The resolution of
such names is not defined by the \module{pickle} module --- the
persistent object module will have to implement a method
\method{persistent_load()}.  To write references to persistent objects,
the persistent module must define a method \method{persistent_id()} which
returns either \code{None} or the persistent ID of the object.

There are some restrictions on the pickling of class instances.

First of all, the class must be defined at the top level in a module.
Furthermore, all its instance variables must be picklable.

\setindexsubitem{(pickle protocol)}

When a pickled class instance is unpickled, its \method{__init__()} method
is normally \emph{not} invoked.  \strong{Note:} This is a deviation
from previous versions of this module; the change was introduced in
Python 1.5b2.  The reason for the change is that in many cases it is
desirable to have a constructor that requires arguments; it is a
(minor) nuisance to have to provide a \method{__getinitargs__()} method.

If it is desirable that the \method{__init__()} method be called on
unpickling, a class can define a method \method{__getinitargs__()},
which should return a \emph{tuple} containing the arguments to be
passed to the class constructor (\method{__init__()}).  This method is
called at pickle time; the tuple it returns is incorporated in the
pickle for the instance.
\withsubitem{(copy protocol)}{\ttindex{__getinitargs__()}}
\withsubitem{(instance constructor)}{\ttindex{__init__()}}

Classes can further influence how their instances are pickled --- if
the class
\withsubitem{(copy protocol)}{
  \ttindex{__getstate__()}\ttindex{__setstate__()}}
\withsubitem{(instance attribute)}{
  \ttindex{__dict__}}
defines the method \method{__getstate__()}, it is called and the return
state is pickled as the contents for the instance, and if the class
defines the method \method{__setstate__()}, it is called with the
unpickled state.  (Note that these methods can also be used to
implement copying class instances.)  If there is no
\method{__getstate__()} method, the instance's \member{__dict__} is
pickled.  If there is no \method{__setstate__()} method, the pickled
object must be a dictionary and its items are assigned to the new
instance's dictionary.  (If a class defines both \method{__getstate__()}
and \method{__setstate__()}, the state object needn't be a dictionary
--- these methods can do what they want.)  This protocol is also used
by the shallow and deep copying operations defined in the
\refmodule{copy}\refstmodindex{copy} module.

Note that when class instances are pickled, their class's code and
data are not pickled along with them.  Only the instance data are
pickled.  This is done on purpose, so you can fix bugs in a class or
add methods and still load objects that were created with an earlier
version of the class.  If you plan to have long-lived objects that
will see many versions of a class, it may be worthwhile to put a version
number in the objects so that suitable conversions can be made by the
class's \method{__setstate__()} method.

When a class itself is pickled, only its name is pickled --- the class
definition is not pickled, but re-imported by the unpickling process.
Therefore, the restriction that the class must be defined at the top
level in a module applies to pickled classes as well.

\setindexsubitem{(in module pickle)}

The interface can be summarized as follows.

To pickle an object \code{x} onto a file \code{f}, open for writing:

\begin{verbatim}
p = pickle.Pickler(f)
p.dump(x)
\end{verbatim}

A shorthand for this is:

\begin{verbatim}
pickle.dump(x, f)
\end{verbatim}

To unpickle an object \code{x} from a file \code{f}, open for reading:

\begin{verbatim}
u = pickle.Unpickler(f)
x = u.load()
\end{verbatim}

A shorthand is:

\begin{verbatim}
x = pickle.load(f)
\end{verbatim}

The \class{Pickler} class only calls the method \code{f.write()} with a
\withsubitem{(class in pickle)}{
  \ttindex{Unpickler}\ttindex{Pickler}}
string argument.  The \class{Unpickler} calls the methods \code{f.read()}
(with an integer argument) and \code{f.readline()} (without argument),
both returning a string.  It is explicitly allowed to pass non-file
objects here, as long as they have the right methods.

The constructor for the \class{Pickler} class has an optional second
argument, \var{bin}.  If this is present and nonzero, the binary
pickle format is used; if it is zero or absent, the (less efficient,
but backwards compatible) text pickle format is used.  The
\class{Unpickler} class does not have an argument to distinguish
between binary and text pickle formats; it accepts either format.

The following types can be pickled:

\begin{itemize}

\item \code{None}

\item integers, long integers, floating point numbers

\item strings

\item tuples, lists and dictionaries containing only picklable objects

\item classes that are defined at the top level in a module

\item instances of such classes whose \member{__dict__} or
\method{__setstate__()} is picklable

\end{itemize}

Attempts to pickle unpicklable objects will raise the
\exception{PicklingError} exception; when this happens, an unspecified
number of bytes may have been written to the file.

It is possible to make multiple calls to the \method{dump()} method of
the same \class{Pickler} instance.  These must then be matched to the
same number of calls to the \method{load()} method of the
corresponding \class{Unpickler} instance.  If the same object is
pickled by multiple \method{dump()} calls, the \method{load()} will all
yield references to the same object.  \emph{Warning}: this is intended
for pickling multiple objects without intervening modifications to the
objects or their parts.  If you modify an object and then pickle it
again using the same \class{Pickler} instance, the object is not
pickled again --- a reference to it is pickled and the
\class{Unpickler} will return the old value, not the modified one.
(There are two problems here: (a) detecting changes, and (b)
marshalling a minimal set of changes.  I have no answers.  Garbage
Collection may also become a problem here.)

Apart from the \class{Pickler} and \class{Unpickler} classes, the
module defines the following functions, and an exception:

\begin{funcdesc}{dump}{object, file\optional{, bin}}
Write a pickled representation of \var{obect} to the open file object
\var{file}.  This is equivalent to
\samp{Pickler(\var{file}, \var{bin}).dump(\var{object})}.
If the optional \var{bin} argument is present and nonzero, the binary
pickle format is used; if it is zero or absent, the (less efficient)
text pickle format is used.
\end{funcdesc}

\begin{funcdesc}{load}{file}
Read a pickled object from the open file object \var{file}.  This is
equivalent to \samp{Unpickler(\var{file}).load()}.
\end{funcdesc}

\begin{funcdesc}{dumps}{object\optional{, bin}}
Return the pickled representation of the object as a string, instead
of writing it to a file.  If the optional \var{bin} argument is
present and nonzero, the binary pickle format is used; if it is zero
or absent, the (less efficient) text pickle format is used.
\end{funcdesc}

\begin{funcdesc}{loads}{string}
Read a pickled object from a string instead of a file.  Characters in
the string past the pickled object's representation are ignored.
\end{funcdesc}

\begin{excdesc}{PicklingError}
This exception is raised when an unpicklable object is passed to
\method{Pickler.dump()}.
\end{excdesc}


\begin{seealso}
  \seemodule[copyreg]{copy_reg}{pickle interface constructor
                                registration}

  \seemodule{shelve}{indexed databases of objects; uses \module{pickle}}

  \seemodule{copy}{shallow and deep object copying}

  \seemodule{marshal}{high-performance serialization of built-in types}
\end{seealso}


\section{\module{cPickle} ---
         Alternate implementation of \module{pickle}}

\declaremodule{builtin}{cPickle}
\modulesynopsis{Faster version of \module{pickle}, but not subclassable.}
\moduleauthor{Jim Fulton}{jfulton@digicool.com}
\sectionauthor{Fred L. Drake, Jr.}{fdrake@acm.org}


The \module{cPickle} module provides a similar interface and identical
functionality as the \refmodule{pickle}\refstmodindex{pickle} module,
but can be up to 1000 times faster since it is implemented in C.  The
only other important difference to note is that \function{Pickler()}
and \function{Unpickler()} are functions and not classes, and so
cannot be subclassed.  This should not be an issue in most cases.

The format of the pickle data is identical to that produced using the
\refmodule{pickle} module, so it is possible to use \refmodule{pickle} and
\module{cPickle} interchangably with existing pickles.

(Since the pickle data format is actually a tiny stack-oriented
programming language, and there are some freedoms in the encodings of
certain objects, it's possible that the two modules produce different
pickled data for the same input objects; however they will always be
able to read each others pickles back in.)

\section{\module{copy_reg} ---
         Register \module{pickle} support functions}

\declaremodule[copyreg]{standard}{copy_reg}
\modulesynopsis{Register \module{pickle} support functions.}


The \module{copy_reg} module provides support for the
\refmodule{pickle}\refstmodindex{pickle}\ and
\refmodule{cPickle}\refbimodindex{cPickle}\ modules.  The
\refmodule{copy}\refstmodindex{copy}\ module is likely to use this in the
future as well.  It provides configuration information about object
constructors which are not classes.  Such constructors may be factory
functions or class instances.


\begin{funcdesc}{constructor}{object}
  Declares \var{object} to be a valid constructor.  If \var{object} is
  not callable (and hence not valid as a constructor), raises
  \exception{TypeError}.
\end{funcdesc}

\begin{funcdesc}{pickle}{type, function\optional{, constructor}}
  Declares that \var{function} should be used as a ``reduction''
  function for objects of type \var{type}; \var{type} must not be a
  ``classic'' class object.  (Classic classes are handled differently;
  see the documentation for the \refmodule{pickle} module for
  details.)  \var{function} should return either a string or a tuple
  containing two or three elements.

  The optional \var{constructor} parameter, if provided, is a
  callable object which can be used to reconstruct the object when
  called with the tuple of arguments returned by \var{function} at
  pickling time.  \exception{TypeError} will be raised if
  \var{object} is a class or \var{constructor} is not callable.

  See the \refmodule{pickle} module for more
  details on the interface expected of \var{function} and
  \var{constructor}.
\end{funcdesc}
		% really copy_reg
\section{Standard Module \module{shelve}}
\label{module-shelve}
\stmodindex{shelve}

A ``shelf'' is a persistent, dictionary-like object.  The difference
with ``dbm'' databases is that the values (not the keys!) in a shelf
can be essentially arbitrary Python objects --- anything that the
\code{pickle} module can handle.  This includes most class instances,
recursive data types, and objects containing lots of shared
sub-objects.  The keys are ordinary strings.
\refstmodindex{pickle}

To summarize the interface (\code{key} is a string, \code{data} is an
arbitrary object):

\begin{verbatim}
import shelve

d = shelve.open(filename) # open, with (g)dbm filename -- no suffix

d[key] = data   # store data at key (overwrites old data if
                # using an existing key)
data = d[key]   # retrieve data at key (raise KeyError if no
                # such key)
del d[key]      # delete data stored at key (raises KeyError
                # if no such key)
flag = d.has_key(key)   # true if the key exists
list = d.keys() # a list of all existing keys (slow!)

d.close()       # close it
\end{verbatim}
%
Restrictions:

\begin{itemize}

\item
The choice of which database package will be used (e.g. \code{dbm} or
\code{gdbm})
depends on which interface is available.  Therefore it isn't safe to
open the database directly using \code{dbm}.  The database is also
(unfortunately) subject to the limitations of \code{dbm}, if it is used ---
this means that (the pickled representation of) the objects stored in
the database should be fairly small, and in rare cases key collisions
may cause the database to refuse updates.
\refbimodindex{dbm}
\refbimodindex{gdbm}

\item
Dependent on the implementation, closing a persistent dictionary may
or may not be necessary to flush changes to disk.

\item
The \code{shelve} module does not support \emph{concurrent} read/write
access to shelved objects.  (Multiple simultaneous read accesses are
safe.)  When a program has a shelf open for writing, no other program
should have it open for reading or writing.  \UNIX{} file locking can
be used to solve this, but this differs across \UNIX{} versions and
requires knowledge about the database implementation used.

\end{itemize}

\section{\module{copy} ---
         Shallow and deep copy operations}
\declaremodule{standard}{copy}

\modulesynopsis{Shallow and deep copy operations.}


This module provides generic (shallow and deep) copying operations.
\withsubitem{(in copy)}{\ttindex{copy()}\ttindex{deepcopy()}}

Interface summary:

\begin{verbatim}
import copy

x = copy.copy(y)        # make a shallow copy of y
x = copy.deepcopy(y)    # make a deep copy of y
\end{verbatim}
%
For module specific errors, \exception{copy.error} is raised.

The difference between shallow and deep copying is only relevant for
compound objects (objects that contain other objects, like lists or
class instances):

\begin{itemize}

\item
A \emph{shallow copy} constructs a new compound object and then (to the
extent possible) inserts \emph{references} into it to the objects found
in the original.

\item
A \emph{deep copy} constructs a new compound object and then,
recursively, inserts \emph{copies} into it of the objects found in the
original.

\end{itemize}

Two problems often exist with deep copy operations that don't exist
with shallow copy operations:

\begin{itemize}

\item
Recursive objects (compound objects that, directly or indirectly,
contain a reference to themselves) may cause a recursive loop.

\item
Because deep copy copies \emph{everything} it may copy too much,
e.g., administrative data structures that should be shared even
between copies.

\end{itemize}

The \function{deepcopy()} function avoids these problems by:

\begin{itemize}

\item
keeping a ``memo'' dictionary of objects already copied during the current
copying pass; and

\item
letting user-defined classes override the copying operation or the
set of components copied.

\end{itemize}

This version does not copy types like module, class, function, method,
nor stack trace, stack frame, nor file, socket, window, nor array, nor
any similar types.

Classes can use the same interfaces to control copying that they use
to control pickling: they can define methods called
\method{__getinitargs__()}, \method{__getstate__()} and
\method{__setstate__()}.  See the description of module
\refmodule{pickle}\refstmodindex{pickle} for information on these
methods.  The \module{copy} module does not use the \module{copy_reg}
registration module.
\withsubitem{(copy protocol)}{\ttindex{__getinitargs__()}
  \ttindex{__getstate__()}\ttindex{__setstate__()}}

In order for a class to define its own copy implementation, it can
define special methods \method{__copy__()} and
\method{__deepcopy__()}.  The former is called to implement the
shallow copy operation; no additional arguments are passed.  The
latter is called to implement the deep copy operation; it is passed
one argument, the memo dictionary.  If the \method{__deepcopy__()}
implementation needs to make a deep copy of a component, it should
call the \function{deepcopy()} function with the component as first
argument and the memo dictionary as second argument.
\withsubitem{(copy protocol)}{\ttindex{__copy__()}\ttindex{__deepcopy__()}}

\begin{seealso}
\seemodule{pickle}{Discussion of the special disciplines used to
support object state retrieval and restoration.}
\end{seealso}

\section{\module{marshal} ---
         Internal Python object serialization}

\declaremodule{builtin}{marshal}
\modulesynopsis{Convert Python objects to streams of bytes and back
                (with different constraints).}


This module contains functions that can read and write Python
values in a binary format.  The format is specific to Python, but
independent of machine architecture issues (e.g., you can write a
Python value to a file on a PC, transport the file to a Sun, and read
it back there).  Details of the format are undocumented on purpose;
it may change between Python versions (although it rarely
does).\footnote{The name of this module stems from a bit of
  terminology used by the designers of Modula-3 (amongst others), who
  use the term ``marshalling'' for shipping of data around in a
  self-contained form. Strictly speaking, ``to marshal'' means to
  convert some data from internal to external form (in an RPC buffer for
  instance) and ``unmarshalling'' for the reverse process.}

This is not a general ``persistence'' module.  For general persistence
and transfer of Python objects through RPC calls, see the modules
\refmodule{pickle} and \refmodule{shelve}.  The \module{marshal} module exists
mainly to support reading and writing the ``pseudo-compiled'' code for
Python modules of \file{.pyc} files.  Therefore, the Python
maintainers reserve the right to modify the marshal format in backward
incompatible ways should the need arise.  If you're serializing and
de-serializing Python objects, use the \module{pickle} module instead.  
\refstmodindex{pickle}
\refstmodindex{shelve}
\obindex{code}

\begin{notice}[warning]
The \module{marshal} module is not intended to be secure against
erroneous or maliciously constructed data.  Never unmarshal data
received from an untrusted or unauthenticated source.
\end{notice}

Not all Python object types are supported; in general, only objects
whose value is independent from a particular invocation of Python can
be written and read by this module.  The following types are supported:
\code{None}, integers, long integers, floating point numbers,
strings, Unicode objects, tuples, lists, dictionaries, and code
objects, where it should be understood that tuples, lists and
dictionaries are only supported as long as the values contained
therein are themselves supported; and recursive lists and dictionaries
should not be written (they will cause infinite loops).

\strong{Caveat:} On machines where C's \code{long int} type has more than
32 bits (such as the DEC Alpha), it is possible to create plain Python
integers that are longer than 32 bits.
If such an integer is marshaled and read back in on a machine where
C's \code{long int} type has only 32 bits, a Python long integer object
is returned instead.  While of a different type, the numeric value is
the same.  (This behavior is new in Python 2.2.  In earlier versions,
all but the least-significant 32 bits of the value were lost, and a
warning message was printed.)

There are functions that read/write files as well as functions
operating on strings.

The module defines these functions:

\begin{funcdesc}{dump}{value, file\optional{, version}}
  Write the value on the open file.  The value must be a supported
  type.  The file must be an open file object such as
  \code{sys.stdout} or returned by \function{open()} or
  \function{posix.popen()}.  It must be opened in binary mode
  (\code{'wb'} or \code{'w+b'}).

  If the value has (or contains an object that has) an unsupported type,
  a \exception{ValueError} exception is raised --- but garbage data
  will also be written to the file.  The object will not be properly
  read back by \function{load()}.

  \versionadded[The \var{version} argument indicates the data
  format that \code{dump} should use (see below)]{2.4}
\end{funcdesc}

\begin{funcdesc}{load}{file}
  Read one value from the open file and return it.  If no valid value
  is read, raise \exception{EOFError}, \exception{ValueError} or
  \exception{TypeError}.  The file must be an open file object opened
  in binary mode (\code{'rb'} or \code{'r+b'}).

  \warning{If an object containing an unsupported type was
  marshalled with \function{dump()}, \function{load()} will substitute
  \code{None} for the unmarshallable type.}
\end{funcdesc}

\begin{funcdesc}{dumps}{value\optional{, version}}
  Return the string that would be written to a file by
  \code{dump(\var{value}, \var{file})}.  The value must be a supported
  type.  Raise a \exception{ValueError} exception if value has (or
  contains an object that has) an unsupported type.

  \versionadded[The \var{version} argument indicates the data
  format that \code{dumps} should use (see below)]{2.4}
\end{funcdesc}

\begin{funcdesc}{loads}{string}
  Convert the string to a value.  If no valid value is found, raise
  \exception{EOFError}, \exception{ValueError} or
  \exception{TypeError}.  Extra characters in the string are ignored.
\end{funcdesc}

In addition, the following constants are defined:

\begin{datadesc}{version}
  Indicates the format that the module uses. Version 0 is the
  historical format, version 1 (added in Python 2.4) shares
  interned strings. The current version is 1.

  \versionadded{2.4}
\end{datadesc}
\section{\module{imp} ---
         Access the \keyword{import} internals}

\declaremodule{builtin}{imp}
\modulesynopsis{Access the implementation of the \keyword{import} statement.}


This\stindex{import} module provides an interface to the mechanisms
used to implement the \keyword{import} statement.  It defines the
following constants and functions:


\begin{funcdesc}{get_magic}{}
\indexii{file}{byte-code}
Return the magic string value used to recognize byte-compiled code
files (\file{.pyc} files).  (This value may be different for each
Python version.)
\end{funcdesc}

\begin{funcdesc}{get_suffixes}{}
Return a list of triples, each describing a particular type of module.
Each triple has the form \code{(\var{suffix}, \var{mode},
\var{type})}, where \var{suffix} is a string to be appended to the
module name to form the filename to search for, \var{mode} is the mode
string to pass to the built-in \function{open()} function to open the
file (this can be \code{'r'} for text files or \code{'rb'} for binary
files), and \var{type} is the file type, which has one of the values
\constant{PY_SOURCE}, \constant{PY_COMPILED}, or
\constant{C_EXTENSION}, described below.
\end{funcdesc}

\begin{funcdesc}{find_module}{name\optional{, path}}
Try to find the module \var{name} on the search path \var{path}.  If
\var{path} is a list of directory names, each directory is searched
for files with any of the suffixes returned by \function{get_suffixes()}
above.  Invalid names in the list are silently ignored (but all list
items must be strings).  If \var{path} is omitted or \code{None}, the
list of directory names given by \code{sys.path} is searched, but
first it searches a few special places: it tries to find a built-in
module with the given name (\constant{C_BUILTIN}), then a frozen module
(\constant{PY_FROZEN}), and on some systems some other places are looked
in as well (on the Mac, it looks for a resource (\constant{PY_RESOURCE});
on Windows, it looks in the registry which may point to a specific
file).

If search is successful, the return value is a triple
\code{(\var{file}, \var{pathname}, \var{description})} where
\var{file} is an open file object positioned at the beginning,
\var{pathname} is the pathname of the
file found, and \var{description} is a triple as contained in the list
returned by \function{get_suffixes()} describing the kind of module found.
If the module does not live in a file, the returned \var{file} is
\code{None}, \var{filename} is the empty string, and the
\var{description} tuple contains empty strings for its suffix and
mode; the module type is as indicate in parentheses above.  If the
search is unsuccessful, \exception{ImportError} is raised.  Other
exceptions indicate problems with the arguments or environment.

This function does not handle hierarchical module names (names
containing dots).  In order to find \var{P}.\var{M}, that is, submodule
\var{M} of package \var{P}, use \function{find_module()} and
\function{load_module()} to find and load package \var{P}, and then use
\function{find_module()} with the \var{path} argument set to
\code{\var{P}.__path__}.  When \var{P} itself has a dotted name, apply
this recipe recursively.
\end{funcdesc}

\begin{funcdesc}{load_module}{name, file, filename, description}
Load a module that was previously found by \function{find_module()} (or by
an otherwise conducted search yielding compatible results).  This
function does more than importing the module: if the module was
already imported, it is equivalent to a
\function{reload()}\bifuncindex{reload}!  The \var{name} argument
indicates the full module name (including the package name, if this is
a submodule of a package).  The \var{file} argument is an open file,
and \var{filename} is the corresponding file name; these can be
\code{None} and \code{''}, respectively, when the module is not being
loaded from a file.  The \var{description} argument is a tuple, as
would be returned by \function{get_suffixes()}, describing what kind
of module must be loaded.

If the load is successful, the return value is the module object;
otherwise, an exception (usually \exception{ImportError}) is raised.

\strong{Important:} the caller is responsible for closing the
\var{file} argument, if it was not \code{None}, even when an exception
is raised.  This is best done using a \keyword{try}
... \keyword{finally} statement.
\end{funcdesc}

\begin{funcdesc}{new_module}{name}
Return a new empty module object called \var{name}.  This object is
\emph{not} inserted in \code{sys.modules}.
\end{funcdesc}

\begin{funcdesc}{lock_held}{}
Return \code{True} if the import lock is currently held, else \code{False}.
On platforms without threads, always return \code{False}.

On platforms with threads, a thread executing an import holds an internal
lock until the import is complete.
This lock blocks other threads from doing an import until the original
import completes, which in turn prevents other threads from seeing
incomplete module objects constructed by the original thread while in
the process of completing its import (and the imports, if any,
triggered by that).
\end{funcdesc}

\begin{funcdesc}{set_frozenmodules}{seq_of_tuples}
Set the global list of frozen modules.  \var{seq_of_tuples} is a sequence
of tuples of length 3: (\var{modulename}, \var{codedata}, \var{ispkg})
\var{modulename} is the name of the frozen module (may contain dots).
\var{codedata} is a marshalled code object. \var{ispkg} is a boolean
indicating whether the module is a package.
\versionadded{2.3}
\end{funcdesc}

\begin{funcdesc}{get_frozenmodules}{}
Return the global list of frozen modules as a tuple of tuples. See
\function{set_frozenmodules()} for a description of its contents.
\versionadded{2.3}
\end{funcdesc}

The following constants with integer values, defined in this module,
are used to indicate the search result of \function{find_module()}.

\begin{datadesc}{PY_SOURCE}
The module was found as a source file.
\end{datadesc}

\begin{datadesc}{PY_COMPILED}
The module was found as a compiled code object file.
\end{datadesc}

\begin{datadesc}{C_EXTENSION}
The module was found as dynamically loadable shared library.
\end{datadesc}

\begin{datadesc}{PY_RESOURCE}
The module was found as a Macintosh resource.  This value can only be
returned on a Macintosh.
\end{datadesc}

\begin{datadesc}{PKG_DIRECTORY}
The module was found as a package directory.
\end{datadesc}

\begin{datadesc}{C_BUILTIN}
The module was found as a built-in module.
\end{datadesc}

\begin{datadesc}{PY_FROZEN}
The module was found as a frozen module (see \function{init_frozen()}).
\end{datadesc}

The following constant and functions are obsolete; their functionality
is available through \function{find_module()} or \function{load_module()}.
They are kept around for backward compatibility:

\begin{datadesc}{SEARCH_ERROR}
Unused.
\end{datadesc}

\begin{funcdesc}{init_builtin}{name}
Initialize the built-in module called \var{name} and return its module
object.  If the module was already initialized, it will be initialized
\emph{again}.  A few modules cannot be initialized twice --- attempting
to initialize these again will raise an \exception{ImportError}
exception.  If there is no
built-in module called \var{name}, \code{None} is returned.
\end{funcdesc}

\begin{funcdesc}{init_frozen}{name}
Initialize the frozen module called \var{name} and return its module
object.  If the module was already initialized, it will be initialized
\emph{again}.  If there is no frozen module called \var{name},
\code{None} is returned.  (Frozen modules are modules written in
Python whose compiled byte-code object is incorporated into a
custom-built Python interpreter by Python's \program{freeze} utility.
See \file{Tools/freeze/} for now.)
\end{funcdesc}

\begin{funcdesc}{is_builtin}{name}
Return \code{1} if there is a built-in module called \var{name} which
can be initialized again.  Return \code{-1} if there is a built-in
module called \var{name} which cannot be initialized again (see
\function{init_builtin()}).  Return \code{0} if there is no built-in
module called \var{name}.
\end{funcdesc}

\begin{funcdesc}{is_frozen}{name}
Return \code{True} if there is a frozen module (see
\function{init_frozen()}) called \var{name}, or \code{False} if there is
no such module.
\end{funcdesc}

\begin{funcdesc}{load_compiled}{name, pathname, file}
\indexii{file}{byte-code}
Load and initialize a module implemented as a byte-compiled code file
and return its module object.  If the module was already initialized,
it will be initialized \emph{again}.  The \var{name} argument is used
to create or access a module object.  The \var{pathname} argument
points to the byte-compiled code file.  The \var{file}
argument is the byte-compiled code file, open for reading in binary
mode, from the beginning.
It must currently be a real file object, not a
user-defined class emulating a file.
\end{funcdesc}

\begin{funcdesc}{load_dynamic}{name, pathname\optional{, file}}
Load and initialize a module implemented as a dynamically loadable
shared library and return its module object.  If the module was
already initialized, it will be initialized \emph{again}.  Some modules
don't like that and may raise an exception.  The \var{pathname}
argument must point to the shared library.  The \var{name} argument is
used to construct the name of the initialization function: an external
C function called \samp{init\var{name}()} in the shared library is
called.  The optional \var{file} argument is ignored.  (Note: using
shared libraries is highly system dependent, and not all systems
support it.)
\end{funcdesc}

\begin{funcdesc}{load_source}{name, pathname, file}
Load and initialize a module implemented as a Python source file and
return its module object.  If the module was already initialized, it
will be initialized \emph{again}.  The \var{name} argument is used to
create or access a module object.  The \var{pathname} argument points
to the source file.  The \var{file} argument is the source
file, open for reading as text, from the beginning.
It must currently be a real file
object, not a user-defined class emulating a file.  Note that if a
properly matching byte-compiled file (with suffix \file{.pyc} or
\file{.pyo}) exists, it will be used instead of parsing the given
source file.
\end{funcdesc}


\subsection{Examples}
\label{examples-imp}

The following function emulates what was the standard import statement
up to Python 1.4 (no hierarchical module names).  (This
\emph{implementation} wouldn't work in that version, since
\function{find_module()} has been extended and
\function{load_module()} has been added in 1.4.)

\begin{verbatim}
import imp
import sys

def __import__(name, globals=None, locals=None, fromlist=None):
    # Fast path: see if the module has already been imported.
    try:
        return sys.modules[name]
    except KeyError:
        pass

    # If any of the following calls raises an exception,
    # there's a problem we can't handle -- let the caller handle it.

    fp, pathname, description = imp.find_module(name)
    
    try:
        return imp.load_module(name, fp, pathname, description)
    finally:
        # Since we may exit via an exception, close fp explicitly.
        if fp:
            fp.close()
\end{verbatim}

A more complete example that implements hierarchical module names and
includes a \function{reload()}\bifuncindex{reload} function can be
found in the module \module{knee}\refmodindex{knee}.  The
\module{knee} module can be found in \file{Demo/imputil/} in the
Python source distribution.

%\section{Built-in Module \sectcode{ni}}
\label{module-ni}
\bimodindex{ni}

\strong{Warning: This module is obsolete.}  As of Python 1.5a4,
package support (with different semantics for \code{__init__} and no
support for \code{__domain__} or\code{__}) is built in the
interpreter.  The ni module is retained only for backward
compatibility.

The \code{ni} module defines a new importing scheme, which supports
packages containing several Python modules.  To enable package
support, execute \code{import ni} before importing any packages.  Importing
this module automatically installs the relevant import hooks.  There
are no publicly-usable functions or variables in the \code{ni} module.

To create a package named \code{spam} containing sub-modules \code{ham}, \code{bacon} and
\code{eggs}, create a directory \file{spam} somewhere on Python's module search
path, as given in \code{sys.path}.  Then, create files called \file{ham.py}, \file{bacon.py} and
\file{eggs.py} inside \file{spam}.

To import module \code{ham} from package \code{spam} and use function
\code{hamneggs()} from that module, you can use any of the following
possibilities:

\bcode\begin{verbatim}
import spam.ham		# *not* "import spam" !!!
spam.ham.hamneggs()
\end{verbatim}\ecode
%
\bcode\begin{verbatim}
from spam import ham
ham.hamneggs()
\end{verbatim}\ecode
%
\bcode\begin{verbatim}
from spam.ham import hamneggs
hamneggs()
\end{verbatim}\ecode
%
\code{import spam} creates an
empty package named \code{spam} if one does not already exist, but it does
\emph{not} automatically import \code{spam}'s submodules.  
The only submodule that is guaranteed to be imported is
\code{spam.__init__}, if it exists; it would be in a file named
\file{__init__.py} in the \file{spam} directory.  Note that
\code{spam.__init__} is a submodule of package spam.  It can refer to
spam's namespace as \code{__} (two underscores):

\bcode\begin{verbatim}
__.spam_inited = 1		# Set a package-level variable
\end{verbatim}\ecode
%
Additional initialization code (setting up variables, importing other
submodules) can be performed in \file{spam/__init__.py}.

% libparser.tex
%
% Copyright 1995 Virginia Polytechnic Institute and State University
% and Fred L. Drake, Jr.  This copyright notice must be distributed on
% all copies, but this document otherwise may be distributed as part
% of the Python distribution.  No fee may be charged for this document
% in any representation, either on paper or electronically.  This
% restriction does not affect other elements in a distributed package
% in any way.
%

\section{Built-in Module \module{parser}}
\label{module-parser}
\bimodindex{parser}
\index{parsing!Python source code}

The \module{parser} module provides an interface to Python's internal
parser and byte-code compiler.  The primary purpose for this interface
is to allow Python code to edit the parse tree of a Python expression
and create executable code from this.  This is better than trying
to parse and modify an arbitrary Python code fragment as a string
because parsing is performed in a manner identical to the code
forming the application.  It is also faster.

The \module{parser} module was written and documented by Fred
L. Drake, Jr. (\email{fdrake@acm.org}).%
\index{Drake, Fred L., Jr.}

There are a few things to note about this module which are important
to making use of the data structures created.  This is not a tutorial
on editing the parse trees for Python code, but some examples of using
the \module{parser} module are presented.

Most importantly, a good understanding of the Python grammar processed
by the internal parser is required.  For full information on the
language syntax, refer to the \emph{Python Language Reference}.  The
parser itself is created from a grammar specification defined in the file
\file{Grammar/Grammar} in the standard Python distribution.  The parse
trees stored in the AST objects created by this module are the
actual output from the internal parser when created by the
\function{expr()} or \function{suite()} functions, described below.  The AST
objects created by \function{sequence2ast()} faithfully simulate those
structures.  Be aware that the values of the sequences which are
considered ``correct'' will vary from one version of Python to another
as the formal grammar for the language is revised.  However,
transporting code from one Python version to another as source text
will always allow correct parse trees to be created in the target
version, with the only restriction being that migrating to an older
version of the interpreter will not support more recent language
constructs.  The parse trees are not typically compatible from one
version to another, whereas source code has always been
forward-compatible.

Each element of the sequences returned by \function{ast2list()} or
\function{ast2tuple()} has a simple form.  Sequences representing
non-terminal elements in the grammar always have a length greater than
one.  The first element is an integer which identifies a production in
the grammar.  These integers are given symbolic names in the C header
file \file{Include/graminit.h} and the Python module
\module{symbol}.  Each additional element of the sequence represents
a component of the production as recognized in the input string: these
are always sequences which have the same form as the parent.  An
important aspect of this structure which should be noted is that
keywords used to identify the parent node type, such as the keyword
\keyword{if} in an \constant{if_stmt}, are included in the node tree without
any special treatment.  For example, the \keyword{if} keyword is
represented by the tuple \code{(1, 'if')}, where \code{1} is the
numeric value associated with all \code{NAME} tokens, including
variable and function names defined by the user.  In an alternate form
returned when line number information is requested, the same token
might be represented as \code{(1, 'if', 12)}, where the \code{12}
represents the line number at which the terminal symbol was found.

Terminal elements are represented in much the same way, but without
any child elements and the addition of the source text which was
identified.  The example of the \keyword{if} keyword above is
representative.  The various types of terminal symbols are defined in
the C header file \file{Include/token.h} and the Python module
\module{token}.

The AST objects are not required to support the functionality of this
module, but are provided for three purposes: to allow an application
to amortize the cost of processing complex parse trees, to provide a
parse tree representation which conserves memory space when compared
to the Python list or tuple representation, and to ease the creation
of additional modules in C which manipulate parse trees.  A simple
``wrapper'' class may be created in Python to hide the use of AST
objects.

The \module{parser} module defines functions for a few distinct
purposes.  The most important purposes are to create AST objects and
to convert AST objects to other representations such as parse trees
and compiled code objects, but there are also functions which serve to
query the type of parse tree represented by an AST object.


\subsection{Creating AST Objects}
\label{Creating ASTs}

AST objects may be created from source code or from a parse tree.
When creating an AST object from source, different functions are used
to create the \code{'eval'} and \code{'exec'} forms.

\begin{funcdesc}{expr}{string}
The \function{expr()} function parses the parameter \code{\var{string}}
as if it were an input to \samp{compile(\var{string}, 'eval')}.  If
the parse succeeds, an AST object is created to hold the internal
parse tree representation, otherwise an appropriate exception is
thrown.
\end{funcdesc}

\begin{funcdesc}{suite}{string}
The \function{suite()} function parses the parameter \code{\var{string}}
as if it were an input to \samp{compile(\var{string}, 'exec')}.  If
the parse succeeds, an AST object is created to hold the internal
parse tree representation, otherwise an appropriate exception is
thrown.
\end{funcdesc}

\begin{funcdesc}{sequence2ast}{sequence}
This function accepts a parse tree represented as a sequence and
builds an internal representation if possible.  If it can validate
that the tree conforms to the Python grammar and all nodes are valid
node types in the host version of Python, an AST object is created
from the internal representation and returned to the called.  If there
is a problem creating the internal representation, or if the tree
cannot be validated, a \exception{ParserError} exception is thrown.  An AST
object created this way should not be assumed to compile correctly;
normal exceptions thrown by compilation may still be initiated when
the AST object is passed to \function{compileast()}.  This may indicate
problems not related to syntax (such as a \exception{MemoryError}
exception), but may also be due to constructs such as the result of
parsing \code{del f(0)}, which escapes the Python parser but is
checked by the bytecode compiler.

Sequences representing terminal tokens may be represented as either
two-element lists of the form \code{(1, 'name')} or as three-element
lists of the form \code{(1, 'name', 56)}.  If the third element is
present, it is assumed to be a valid line number.  The line number
may be specified for any subset of the terminal symbols in the input
tree.
\end{funcdesc}

\begin{funcdesc}{tuple2ast}{sequence}
This is the same function as \function{sequence2ast()}.  This entry point
is maintained for backward compatibility.
\end{funcdesc}


\subsection{Converting AST Objects}
\label{Converting ASTs}

AST objects, regardless of the input used to create them, may be
converted to parse trees represented as list- or tuple- trees, or may
be compiled into executable code objects.  Parse trees may be
extracted with or without line numbering information.

\begin{funcdesc}{ast2list}{ast\optional{, line_info}}
This function accepts an AST object from the caller in
\code{\var{ast}} and returns a Python list representing the
equivelent parse tree.  The resulting list representation can be used
for inspection or the creation of a new parse tree in list form.  This
function does not fail so long as memory is available to build the
list representation.  If the parse tree will only be used for
inspection, \function{ast2tuple()} should be used instead to reduce memory
consumption and fragmentation.  When the list representation is
required, this function is significantly faster than retrieving a
tuple representation and converting that to nested lists.

If \code{\var{line_info}} is true, line number information will be
included for all terminal tokens as a third element of the list
representing the token.  Note that the line number provided specifies
the line on which the token \emph{ends}.  This information is
omitted if the flag is false or omitted.
\end{funcdesc}

\begin{funcdesc}{ast2tuple}{ast\optional{, line_info}}
This function accepts an AST object from the caller in
\code{\var{ast}} and returns a Python tuple representing the
equivelent parse tree.  Other than returning a tuple instead of a
list, this function is identical to \function{ast2list()}.

If \code{\var{line_info}} is true, line number information will be
included for all terminal tokens as a third element of the list
representing the token.  This information is omitted if the flag is
false or omitted.
\end{funcdesc}

\begin{funcdesc}{compileast}{ast\optional{, filename\code{ = '<ast>'}}}
The Python byte compiler can be invoked on an AST object to produce
code objects which can be used as part of an \code{exec} statement or
a call to the built-in \function{eval()}\bifuncindex{eval} function.
This function provides the interface to the compiler, passing the
internal parse tree from \code{\var{ast}} to the parser, using the
source file name specified by the \code{\var{filename}} parameter.
The default value supplied for \code{\var{filename}} indicates that
the source was an AST object.

Compiling an AST object may result in exceptions related to
compilation; an example would be a \exception{SyntaxError} caused by the
parse tree for \code{del f(0)}: this statement is considered legal
within the formal grammar for Python but is not a legal language
construct.  The \exception{SyntaxError} raised for this condition is
actually generated by the Python byte-compiler normally, which is why
it can be raised at this point by the \module{parser} module.  Most
causes of compilation failure can be diagnosed programmatically by
inspection of the parse tree.
\end{funcdesc}


\subsection{Queries on AST Objects}
\label{Querying ASTs}

Two functions are provided which allow an application to determine if
an AST was created as an expression or a suite.  Neither of these
functions can be used to determine if an AST was created from source
code via \function{expr()} or \function{suite()} or from a parse tree
via \function{sequence2ast()}.

\begin{funcdesc}{isexpr}{ast}
When \code{\var{ast}} represents an \code{'eval'} form, this function
returns true, otherwise it returns false.  This is useful, since code
objects normally cannot be queried for this information using existing
built-in functions.  Note that the code objects created by
\function{compileast()} cannot be queried like this either, and are
identical to those created by the built-in
\function{compile()}\bifuncindex{compile} function.
\end{funcdesc}


\begin{funcdesc}{issuite}{ast}
This function mirrors \function{isexpr()} in that it reports whether an
AST object represents an \code{'exec'} form, commonly known as a
``suite.''  It is not safe to assume that this function is equivelent
to \samp{not isexpr(\var{ast})}, as additional syntactic fragments may
be supported in the future.
\end{funcdesc}


\subsection{Exceptions and Error Handling}
\label{AST Errors}

The parser module defines a single exception, but may also pass other
built-in exceptions from other portions of the Python runtime
environment.  See each function for information about the exceptions
it can raise.

\begin{excdesc}{ParserError}
Exception raised when a failure occurs within the parser module.  This
is generally produced for validation failures rather than the built in
\exception{SyntaxError} thrown during normal parsing.
The exception argument is either a string describing the reason of the
failure or a tuple containing a sequence causing the failure from a parse
tree passed to \function{sequence2ast()} and an explanatory string.  Calls to
\function{sequence2ast()} need to be able to handle either type of exception,
while calls to other functions in the module will only need to be
aware of the simple string values.
\end{excdesc}

Note that the functions \function{compileast()}, \function{expr()}, and
\function{suite()} may throw exceptions which are normally thrown by the
parsing and compilation process.  These include the built in
exceptions \exception{MemoryError}, \exception{OverflowError},
\exception{SyntaxError}, and \exception{SystemError}.  In these cases, these
exceptions carry all the meaning normally associated with them.  Refer
to the descriptions of each function for detailed information.


\subsection{AST Objects}
\label{AST Objects}

AST objects returned by \function{expr()}, \function{suite()} and
\function{sequence2ast()} have no methods of their own.
Some of the functions defined which accept an AST object as their
first argument may change to object methods in the future.

Ordered and equality comparisons are supported between AST objects.
Pickling of AST objects (using the \module{pickle} module) is also
supported.

\begin{datadesc}{ASTType}
The type of the objects returned by \function{expr()},
\function{suite()} and \function{sequence2ast()}.
\end{datadesc}


\subsection{Examples}
\nodename{AST Examples}

The parser modules allows operations to be performed on the parse tree
of Python source code before the bytecode is generated, and provides
for inspection of the parse tree for information gathering purposes.
Two examples are presented.  The simple example demonstrates emulation
of the \function{compile()}\bifuncindex{compile} built-in function and
the complex example shows the use of a parse tree for information
discovery.

\subsubsection{Emulation of \function{compile()}}

While many useful operations may take place between parsing and
bytecode generation, the simplest operation is to do nothing.  For
this purpose, using the \module{parser} module to produce an
intermediate data structure is equivelent to the code

\begin{verbatim}
>>> code = compile('a + 5', 'eval')
>>> a = 5
>>> eval(code)
10
\end{verbatim}

The equivelent operation using the \module{parser} module is somewhat
longer, and allows the intermediate internal parse tree to be retained
as an AST object:

\begin{verbatim}
>>> import parser
>>> ast = parser.expr('a + 5')
>>> code = parser.compileast(ast)
>>> a = 5
>>> eval(code)
10
\end{verbatim}

An application which needs both AST and code objects can package this
code into readily available functions:

\begin{verbatim}
import parser

def load_suite(source_string):
    ast = parser.suite(source_string)
    code = parser.compileast(ast)
    return ast, code

def load_expression(source_string):
    ast = parser.expr(source_string)
    code = parser.compileast(ast)
    return ast, code
\end{verbatim}

\subsubsection{Information Discovery}

Some applications benefit from direct access to the parse tree.  The
remainder of this section demonstrates how the parse tree provides
access to module documentation defined in docstrings without requiring
that the code being examined be loaded into a running interpreter via
\keyword{import}.  This can be very useful for performing analyses of
untrusted code.

Generally, the example will demonstrate how the parse tree may be
traversed to distill interesting information.  Two functions and a set
of classes are developed which provide programmatic access to high
level function and class definitions provided by a module.  The
classes extract information from the parse tree and provide access to
the information at a useful semantic level, one function provides a
simple low-level pattern matching capability, and the other function
defines a high-level interface to the classes by handling file
operations on behalf of the caller.  All source files mentioned here
which are not part of the Python installation are located in the
\file{Demo/parser/} directory of the distribution.

The dynamic nature of Python allows the programmer a great deal of
flexibility, but most modules need only a limited measure of this when
defining classes, functions, and methods.  In this example, the only
definitions that will be considered are those which are defined in the
top level of their context, e.g., a function defined by a \keyword{def}
statement at column zero of a module, but not a function defined
within a branch of an \code{if} ... \code{else} construct, though
there are some good reasons for doing so in some situations.  Nesting
of definitions will be handled by the code developed in the example.

To construct the upper-level extraction methods, we need to know what
the parse tree structure looks like and how much of it we actually
need to be concerned about.  Python uses a moderately deep parse tree
so there are a large number of intermediate nodes.  It is important to
read and understand the formal grammar used by Python.  This is
specified in the file \file{Grammar/Grammar} in the distribution.
Consider the simplest case of interest when searching for docstrings:
a module consisting of a docstring and nothing else.  (See file
\file{docstring.py}.)

\begin{verbatim}
"""Some documentation.
"""
\end{verbatim}

Using the interpreter to take a look at the parse tree, we find a
bewildering mass of numbers and parentheses, with the documentation
buried deep in nested tuples.

\begin{verbatim}
>>> import parser
>>> import pprint
>>> ast = parser.suite(open('docstring.py').read())
>>> tup = parser.ast2tuple(ast)
>>> pprint.pprint(tup)
(257,
 (264,
  (265,
   (266,
    (267,
     (307,
      (287,
       (288,
        (289,
         (290,
          (292,
           (293,
            (294,
             (295,
              (296,
               (297,
                (298,
                 (299,
                  (300, (3, '"""Some documentation.\012"""'))))))))))))))))),
   (4, ''))),
 (4, ''),
 (0, ''))
\end{verbatim}

The numbers at the first element of each node in the tree are the node
types; they map directly to terminal and non-terminal symbols in the
grammar.  Unfortunately, they are represented as integers in the
internal representation, and the Python structures generated do not
change that.  However, the \module{symbol} and \module{token} modules
provide symbolic names for the node types and dictionaries which map
from the integers to the symbolic names for the node types.

In the output presented above, the outermost tuple contains four
elements: the integer \code{257} and three additional tuples.  Node
type \code{257} has the symbolic name \constant{file_input}.  Each of
these inner tuples contains an integer as the first element; these
integers, \code{264}, \code{4}, and \code{0}, represent the node types
\constant{stmt}, \constant{NEWLINE}, and \constant{ENDMARKER},
respectively.
Note that these values may change depending on the version of Python
you are using; consult \file{symbol.py} and \file{token.py} for
details of the mapping.  It should be fairly clear that the outermost
node is related primarily to the input source rather than the contents
of the file, and may be disregarded for the moment.  The \constant{stmt}
node is much more interesting.  In particular, all docstrings are
found in subtrees which are formed exactly as this node is formed,
with the only difference being the string itself.  The association
between the docstring in a similar tree and the defined entity (class,
function, or module) which it describes is given by the position of
the docstring subtree within the tree defining the described
structure.

By replacing the actual docstring with something to signify a variable
component of the tree, we allow a simple pattern matching approach to
check any given subtree for equivelence to the general pattern for
docstrings.  Since the example demonstrates information extraction, we
can safely require that the tree be in tuple form rather than list
form, allowing a simple variable representation to be
\code{['variable_name']}.  A simple recursive function can implement
the pattern matching, returning a boolean and a dictionary of variable
name to value mappings.  (See file \file{example.py}.)

\begin{verbatim}
from types import ListType, TupleType

def match(pattern, data, vars=None):
    if vars is None:
        vars = {}
    if type(pattern) is ListType:
        vars[pattern[0]] = data
        return 1, vars
    if type(pattern) is not TupleType:
        return (pattern == data), vars
    if len(data) != len(pattern):
        return 0, vars
    for pattern, data in map(None, pattern, data):
        same, vars = match(pattern, data, vars)
        if not same:
            break
    return same, vars
\end{verbatim}

Using this simple representation for syntactic variables and the symbolic
node types, the pattern for the candidate docstring subtrees becomes
fairly readable.  (See file \file{example.py}.)

\begin{verbatim}
import symbol
import token

DOCSTRING_STMT_PATTERN = (
    symbol.stmt,
    (symbol.simple_stmt,
     (symbol.small_stmt,
      (symbol.expr_stmt,
       (symbol.testlist,
        (symbol.test,
         (symbol.and_test,
          (symbol.not_test,
           (symbol.comparison,
            (symbol.expr,
             (symbol.xor_expr,
              (symbol.and_expr,
               (symbol.shift_expr,
                (symbol.arith_expr,
                 (symbol.term,
                  (symbol.factor,
                   (symbol.power,
                    (symbol.atom,
                     (token.STRING, ['docstring'])
                     )))))))))))))))),
     (token.NEWLINE, '')
     ))
\end{verbatim}

Using the \function{match()} function with this pattern, extracting the
module docstring from the parse tree created previously is easy:

\begin{verbatim}
>>> found, vars = match(DOCSTRING_STMT_PATTERN, tup[1])
>>> found
1
>>> vars
{'docstring': '"""Some documentation.\012"""'}
\end{verbatim}

Once specific data can be extracted from a location where it is
expected, the question of where information can be expected
needs to be answered.  When dealing with docstrings, the answer is
fairly simple: the docstring is the first \constant{stmt} node in a code
block (\constant{file_input} or \constant{suite} node types).  A module
consists of a single \constant{file_input} node, and class and function
definitions each contain exactly one \constant{suite} node.  Classes and
functions are readily identified as subtrees of code block nodes which
start with \code{(stmt, (compound_stmt, (classdef, ...} or
\code{(stmt, (compound_stmt, (funcdef, ...}.  Note that these subtrees
cannot be matched by \function{match()} since it does not support multiple
sibling nodes to match without regard to number.  A more elaborate
matching function could be used to overcome this limitation, but this
is sufficient for the example.

Given the ability to determine whether a statement might be a
docstring and extract the actual string from the statement, some work
needs to be performed to walk the parse tree for an entire module and
extract information about the names defined in each context of the
module and associate any docstrings with the names.  The code to
perform this work is not complicated, but bears some explanation.

The public interface to the classes is straightforward and should
probably be somewhat more flexible.  Each ``major'' block of the
module is described by an object providing several methods for inquiry
and a constructor which accepts at least the subtree of the complete
parse tree which it represents.  The \class{ModuleInfo} constructor
accepts an optional \var{name} parameter since it cannot
otherwise determine the name of the module.

The public classes include \class{ClassInfo}, \class{FunctionInfo},
and \class{ModuleInfo}.  All objects provide the
methods \method{get_name()}, \method{get_docstring()},
\method{get_class_names()}, and \method{get_class_info()}.  The
\class{ClassInfo} objects support \method{get_method_names()} and
\method{get_method_info()} while the other classes provide
\method{get_function_names()} and \method{get_function_info()}.

Within each of the forms of code block that the public classes
represent, most of the required information is in the same form and is
accessed in the same way, with classes having the distinction that
functions defined at the top level are referred to as ``methods.''
Since the difference in nomenclature reflects a real semantic
distinction from functions defined outside of a class, the
implementation needs to maintain the distinction.
Hence, most of the functionality of the public classes can be
implemented in a common base class, \class{SuiteInfoBase}, with the
accessors for function and method information provided elsewhere.
Note that there is only one class which represents function and method
information; this parallels the use of the \keyword{def} statement to
define both types of elements.

Most of the accessor functions are declared in \class{SuiteInfoBase}
and do not need to be overriden by subclasses.  More importantly, the
extraction of most information from a parse tree is handled through a
method called by the \class{SuiteInfoBase} constructor.  The example
code for most of the classes is clear when read alongside the formal
grammar, but the method which recursively creates new information
objects requires further examination.  Here is the relevant part of
the \class{SuiteInfoBase} definition from \file{example.py}:

\begin{verbatim}
class SuiteInfoBase:
    _docstring = ''
    _name = ''

    def __init__(self, tree = None):
        self._class_info = {}
        self._function_info = {}
        if tree:
            self._extract_info(tree)

    def _extract_info(self, tree):
        # extract docstring
        if len(tree) == 2:
            found, vars = match(DOCSTRING_STMT_PATTERN[1], tree[1])
        else:
            found, vars = match(DOCSTRING_STMT_PATTERN, tree[3])
        if found:
            self._docstring = eval(vars['docstring'])
        # discover inner definitions
        for node in tree[1:]:
            found, vars = match(COMPOUND_STMT_PATTERN, node)
            if found:
                cstmt = vars['compound']
                if cstmt[0] == symbol.funcdef:
                    name = cstmt[2][1]
                    self._function_info[name] = FunctionInfo(cstmt)
                elif cstmt[0] == symbol.classdef:
                    name = cstmt[2][1]
                    self._class_info[name] = ClassInfo(cstmt)
\end{verbatim}

After initializing some internal state, the constructor calls the
\method{_extract_info()} method.  This method performs the bulk of the
information extraction which takes place in the entire example.  The
extraction has two distinct phases: the location of the docstring for
the parse tree passed in, and the discovery of additional definitions
within the code block represented by the parse tree.

The initial \keyword{if} test determines whether the nested suite is of
the ``short form'' or the ``long form.''  The short form is used when
the code block is on the same line as the definition of the code
block, as in

\begin{verbatim}
def square(x): "Square an argument."; return x ** 2
\end{verbatim}

while the long form uses an indented block and allows nested
definitions:

\begin{verbatim}
def make_power(exp):
    "Make a function that raises an argument to the exponent `exp'."
    def raiser(x, y=exp):
        return x ** y
    return raiser
\end{verbatim}

When the short form is used, the code block may contain a docstring as
the first, and possibly only, \constant{small_stmt} element.  The
extraction of such a docstring is slightly different and requires only
a portion of the complete pattern used in the more common case.  As
implemented, the docstring will only be found if there is only
one \constant{small_stmt} node in the \constant{simple_stmt} node.
Since most functions and methods which use the short form do not
provide a docstring, this may be considered sufficient.  The
extraction of the docstring proceeds using the \function{match()} function
as described above, and the value of the docstring is stored as an
attribute of the \class{SuiteInfoBase} object.

After docstring extraction, a simple definition discovery
algorithm operates on the \constant{stmt} nodes of the
\constant{suite} node.  The special case of the short form is not
tested; since there are no \constant{stmt} nodes in the short form,
the algorithm will silently skip the single \constant{simple_stmt}
node and correctly not discover any nested definitions.

Each statement in the code block is categorized as
a class definition, function or method definition, or
something else.  For the definition statements, the name of the
element defined is extracted and a representation object
appropriate to the definition is created with the defining subtree
passed as an argument to the constructor.  The repesentation objects
are stored in instance variables and may be retrieved by name using
the appropriate accessor methods.

The public classes provide any accessors required which are more
specific than those provided by the \class{SuiteInfoBase} class, but
the real extraction algorithm remains common to all forms of code
blocks.  A high-level function can be used to extract the complete set
of information from a source file.  (See file \file{example.py}.)

\begin{verbatim}
def get_docs(fileName):
    source = open(fileName).read()
    import os
    basename = os.path.basename(os.path.splitext(fileName)[0])
    import parser
    ast = parser.suite(source)
    tup = parser.ast2tuple(ast)
    return ModuleInfo(tup, basename)
\end{verbatim}

This provides an easy-to-use interface to the documentation of a
module.  If information is required which is not extracted by the code
of this example, the code may be extended at clearly defined points to
provide additional capabilities.

\begin{seealso}

\seemodule{symbol}{useful constants representing internal nodes of the
parse tree}

\seemodule{token}{useful constants representing leaf nodes of the
parse tree and functions for testing node values}

\end{seealso}

\section{\module{symbol} ---
         Constants used with Python parse trees}

\declaremodule{standard}{symbol}
\modulesynopsis{Constants representing internal nodes of the parse tree.}
\sectionauthor{Fred L. Drake, Jr.}{fdrake@acm.org}


This module provides constants which represent the numeric values of
internal nodes of the parse tree.  Unlike most Python constants, these
use lower-case names.  Refer to the file \file{Grammar/Grammar} in the
Python distribution for the definitions of the names in the context of
the language grammar.  The specific numeric values which the names map
to may change between Python versions.

This module also provides one additional data object:


\begin{datadesc}{sym_name}
  Dictionary mapping the numeric values of the constants defined in
  this module back to name strings, allowing more human-readable
  representation of parse trees to be generated.
\end{datadesc}


\begin{seealso}
  \seemodule{parser}{The second example for the \refmodule{parser}
                     module shows how to use the \module{symbol}
                     module.}
\end{seealso}

\section{\module{token} ---
         Constants used with Python parse trees}

\declaremodule{standard}{token}
\modulesynopsis{Constants representing terminal nodes of the parse tree.}
\sectionauthor{Fred L. Drake, Jr.}{fdrake@acm.org}


This module provides constants which represent the numeric values of
leaf nodes of the parse tree (terminal tokens).  Refer to the file
\file{Grammar/Grammar} in the Python distribution for the definitions
of the names in the context of the language grammar.  The specific
numeric values which the names map to may change between Python
versions.

This module also provides one data object and some functions.  The
functions mirror definitions in the Python C header files.



\begin{datadesc}{tok_name}
Dictionary mapping the numeric values of the constants defined in this
module back to name strings, allowing more human-readable
representation of parse trees to be generated.
\end{datadesc}

\begin{funcdesc}{ISTERMINAL}{x}
Return true for terminal token values.
\end{funcdesc}

\begin{funcdesc}{ISNONTERMINAL}{x}
Return true for non-terminal token values.
\end{funcdesc}

\begin{funcdesc}{ISEOF}{x}
Return true if \var{x} is the marker indicating the end of input.
\end{funcdesc}


\begin{seealso}
  \seemodule{parser}{The second example for the \refmodule{parser}
                     module shows how to use the \module{symbol}
                     module.}
\end{seealso}

\section{Standard Module \module{keyword}}
\label{module-keyword}
\stmodindex{keyword}

This module allows a Python program to determine if a string is a
keyword.  A single function is provided:

\begin{funcdesc}{iskeyword}{s}
Return true if \var{s} is a Python keyword.
\end{funcdesc}

\section{Standard Module \sectcode{code}}
\label{module-code}
\stmodindex{code}

The \code{code} module defines operations pertaining to Python code
objects.

The \code{code} module defines the following functions:

\renewcommand{\indexsubitem}{(in module code)}

\begin{funcdesc}{compile_command}{source\,
\optional{filename\optional{\, symbol}}}
This function is useful for programs that want to emulate Python's
interpreter main loop (a.k.a. the read-eval-print loop).  The tricky
part is to determine when the user has entered an incomplete command
that can be completed by entering more text (as opposed to a complete
command or a syntax error).  This function \emph{almost} always makes
the same decision as the real interpreter main loop.

Arguments: \var{source} is the source string; \var{filename} is the
optional filename from which source was read, defaulting to
\code{"<input>"}; and \var{symbol} is the optional grammar start
symbol, which should be either \code{"single"} (the default) or
\code{"eval"}.

Return a code object (the same as \code{compile(\var{source},
\var{filename}, \var{symbol})}) if the command is complete and valid;
return \code{None} if the command is incomplete; raise
\code{SyntaxError} if the command is a syntax error.


\end{funcdesc}

\section{\module{pprint} ---
         Data pretty printer}

\declaremodule{standard}{pprint}
\modulesynopsis{Data pretty printer.}
\moduleauthor{Fred L. Drake, Jr.}{fdrake@acm.org}
\sectionauthor{Fred L. Drake, Jr.}{fdrake@acm.org}


The \module{pprint} module provides a capability to ``pretty-print''
arbitrary Python data structures in a form which can be used as input
to the interpreter.  If the formatted structures include objects which
are not fundamental Python types, the representation may not be
loadable.  This may be the case if objects such as files, sockets,
classes, or instances are included, as well as many other builtin
objects which are not representable as Python constants.

The formatted representation keeps objects on a single line if it can,
and breaks them onto multiple lines if they don't fit within the
allowed width.  Construct \class{PrettyPrinter} objects explicitly if
you need to adjust the width constraint.

The \module{pprint} module defines one class:


% First the implementation class:

\begin{classdesc}{PrettyPrinter}{...}
Construct a \class{PrettyPrinter} instance.  This constructor
understands several keyword parameters.  An output stream may be set
using the \var{stream} keyword; the only method used on the stream
object is the file protocol's \method{write()} method.  If not
specified, the \class{PrettyPrinter} adopts \code{sys.stdout}.  Three
additional parameters may be used to control the formatted
representation.  The keywords are \var{indent}, \var{depth}, and
\var{width}.  The amount of indentation added for each recursive level
is specified by \var{indent}; the default is one.  Other values can
cause output to look a little odd, but can make nesting easier to
spot.  The number of levels which may be printed is controlled by
\var{depth}; if the data structure being printed is too deep, the next
contained level is replaced by \samp{...}.  By default, there is no
constraint on the depth of the objects being formatted.  The desired
output width is constrained using the \var{width} parameter; the
default is eighty characters.  If a structure cannot be formatted
within the constrained width, a best effort will be made.

\begin{verbatim}
>>> import pprint, sys
>>> stuff = sys.path[:]
>>> stuff.insert(0, stuff[:])
>>> pp = pprint.PrettyPrinter(indent=4)
>>> pp.pprint(stuff)
[   [   '',
        '/usr/local/lib/python1.5',
        '/usr/local/lib/python1.5/test',
        '/usr/local/lib/python1.5/sunos5',
        '/usr/local/lib/python1.5/sharedmodules',
        '/usr/local/lib/python1.5/tkinter'],
    '',
    '/usr/local/lib/python1.5',
    '/usr/local/lib/python1.5/test',
    '/usr/local/lib/python1.5/sunos5',
    '/usr/local/lib/python1.5/sharedmodules',
    '/usr/local/lib/python1.5/tkinter']
>>>
>>> import parser
>>> tup = parser.ast2tuple(
...     parser.suite(open('pprint.py').read()))[1][1][1]
>>> pp = pprint.PrettyPrinter(depth=6)
>>> pp.pprint(tup)
(266, (267, (307, (287, (288, (...))))))
\end{verbatim}
\end{classdesc}


% Now the derivative functions:

The \class{PrettyPrinter} class supports several derivative functions:

\begin{funcdesc}{pformat}{object\optional{, indent\optional{,
width\optional{, depth}}}}
Return the formatted representation of \var{object} as a string.  \var{indent},
\var{width} and \var{depth} will be passed to the \class{PrettyPrinter}
constructor as formatting parameters.
\versionchanged[The parameters \var{indent}, \var{width} and \var{depth}
were added]{2.4}
\end{funcdesc}

\begin{funcdesc}{pprint}{object\optional{, stream\optional{,
indent\optional{, width\optional{, depth}}}}}
Prints the formatted representation of \var{object} on \var{stream},
followed by a newline.  If \var{stream} is omitted, \code{sys.stdout}
is used.  This may be used in the interactive interpreter instead of a
\keyword{print} statement for inspecting values.    \var{indent},
\var{width} and \var{depth} will be passed to the \class{PrettyPrinter}
constructor as formatting parameters.

\begin{verbatim}
>>> stuff = sys.path[:]
>>> stuff.insert(0, stuff)
>>> pprint.pprint(stuff)
[<Recursion on list with id=869440>,
 '',
 '/usr/local/lib/python1.5',
 '/usr/local/lib/python1.5/test',
 '/usr/local/lib/python1.5/sunos5',
 '/usr/local/lib/python1.5/sharedmodules',
 '/usr/local/lib/python1.5/tkinter']
\end{verbatim}
\versionchanged[The parameters \var{indent}, \var{width} and \var{depth}
were added]{2.4}
\end{funcdesc}

\begin{funcdesc}{isreadable}{object}
Determine if the formatted representation of \var{object} is
``readable,'' or can be used to reconstruct the value using
\function{eval()}\bifuncindex{eval}.  This always returns false for
recursive objects.

\begin{verbatim}
>>> pprint.isreadable(stuff)
False
\end{verbatim}
\end{funcdesc}

\begin{funcdesc}{isrecursive}{object}
Determine if \var{object} requires a recursive representation.
\end{funcdesc}


One more support function is also defined:

\begin{funcdesc}{saferepr}{object}
Return a string representation of \var{object}, protected against
recursive data structures.  If the representation of \var{object}
exposes a recursive entry, the recursive reference will be represented
as \samp{<Recursion on \var{typename} with id=\var{number}>}.  The
representation is not otherwise formatted.
\end{funcdesc}

% This example is outside the {funcdesc} to keep it from running over
% the right margin.
\begin{verbatim}
>>> pprint.saferepr(stuff)
"[<Recursion on list with id=682968>, '', '/usr/local/lib/python1.5', '/usr/loca
l/lib/python1.5/test', '/usr/local/lib/python1.5/sunos5', '/usr/local/lib/python
1.5/sharedmodules', '/usr/local/lib/python1.5/tkinter']"
\end{verbatim}


\subsection{PrettyPrinter Objects}
\label{PrettyPrinter Objects}

\class{PrettyPrinter} instances have the following methods:


\begin{methoddesc}{pformat}{object}
Return the formatted representation of \var{object}.  This takes into
account the options passed to the \class{PrettyPrinter} constructor.
\end{methoddesc}

\begin{methoddesc}{pprint}{object}
Print the formatted representation of \var{object} on the configured
stream, followed by a newline.
\end{methoddesc}

The following methods provide the implementations for the
corresponding functions of the same names.  Using these methods on an
instance is slightly more efficient since new \class{PrettyPrinter}
objects don't need to be created.

\begin{methoddesc}{isreadable}{object}
Determine if the formatted representation of the object is
``readable,'' or can be used to reconstruct the value using
\function{eval()}\bifuncindex{eval}.  Note that this returns false for
recursive objects.  If the \var{depth} parameter of the
\class{PrettyPrinter} is set and the object is deeper than allowed,
this returns false.
\end{methoddesc}

\begin{methoddesc}{isrecursive}{object}
Determine if the object requires a recursive representation.
\end{methoddesc}

This method is provided as a hook to allow subclasses to modify the
way objects are converted to strings.  The default implementation uses
the internals of the \function{saferepr()} implementation.

\begin{methoddesc}{format}{object, context, maxlevels, level}
Returns three values: the formatted version of \var{object} as a
string, a flag indicating whether the result is readable, and a flag
indicating whether recursion was detected.  The first argument is the
object to be presented.  The second is a dictionary which contains the
\function{id()} of objects that are part of the current presentation
context (direct and indirect containers for \var{object} that are
affecting the presentation) as the keys; if an object needs to be
presented which is already represented in \var{context}, the third
return value should be true.  Recursive calls to the \method{format()}
method should add additional entries for containers to this
dictionary.  The third argument, \var{maxlevels}, gives the requested
limit to recursion; this will be \code{0} if there is no requested
limit.  This argument should be passed unmodified to recursive calls.
The fourth argument, \var{level}, gives the current level; recursive
calls should be passed a value less than that of the current call.
\versionadded{2.3}
\end{methoddesc}

\section{\module{py_compile} ---
         Compile Python source files}

% Documentation based on module docstrings, by Fred L. Drake, Jr.
% <fdrake@acm.org>

\declaremodule[pycompile]{standard}{py_compile}

\modulesynopsis{Compile Python source files to byte-code files.}


\indexii{file}{byte-code}
The \module{py_compile} module provides a single function to generate
a byte-code file from a source file.

Though not often needed, this function can be useful when installing
modules for shared use, especially if some of the users may not have
permission to write the byte-code cache files in the directory
containing the source code.


\begin{funcdesc}{compile}{file\optional{, cfile\optional{, dfile}}}
  Compile a source file to byte-code and write out the byte-code cache 
  file.  The source code is loaded from the file name \var{file}.  The 
  byte-code is written to \var{cfile}, which defaults to \var{file}
  \code{+} \code{'c'} (\code{'o'} if optimization is enabled in the
  current interpreter).  If \var{dfile} is specified, it is used as
  the name of the source file in error messages instead of \var{file}. 
\end{funcdesc}


\begin{seealso}
  \seemodule{compileall}{Utilities to compile all Python source files
                         in a directory tree.}
\end{seealso}
		% really py_compile
\section{\module{compileall} ---
         Byte-compile Python libraries}

\declaremodule{standard}{compileall}
\modulesynopsis{Tools for byte-compiling all Python source files in a
                directory tree.}


This module provides some utility functions to support installing
Python libraries.  These functions compile Python source files in a
directory tree, allowing users without permission to write to the
libraries to take advantage of cached byte-code files.

The source file for this module may also be used as a script to
compile Python sources in directories named on the command line or in
\code{sys.path}.


\begin{funcdesc}{compile_dir}{dir\optional{, maxlevels\optional{,
                              ddir\optional{, force\optional{, 
                              rx\optional{, quiet}}}}}}
  Recursively descend the directory tree named by \var{dir}, compiling
  all \file{.py} files along the way.  The \var{maxlevels} parameter
  is used to limit the depth of the recursion; it defaults to
  \code{10}.  If \var{ddir} is given, it is used as the base path from 
  which the filenames used in error messages will be generated.  If
  \var{force} is true, modules are re-compiled even if the timestamps
  are up to date. 

  If \var{rx} is given, it specifies a regular expression of file
  names to exclude from the search; that expression is searched for in
  the full path.

  If \var{quiet} is true, nothing is printed to the standard output
  in normal operation.
\end{funcdesc}

\begin{funcdesc}{compile_path}{\optional{skip_curdir\optional{,
                               maxlevels\optional{, force}}}}
  Byte-compile all the \file{.py} files found along \code{sys.path}.
  If \var{skip_curdir} is true (the default), the current directory is
  not included in the search.  The \var{maxlevels} and
  \var{force} parameters default to \code{0} and are passed to the
  \function{compile_dir()} function.
\end{funcdesc}


\begin{seealso}
  \seemodule[pycompile]{py_compile}{Byte-compile a single source file.}
\end{seealso}

\section{\module{dis} ---
         Disassembler for Python byte code}

\declaremodule{standard}{dis}
\modulesynopsis{Disassembler for Python byte code.}


The \module{dis} module supports the analysis of Python byte code by
disassembling it.  Since there is no Python assembler, this module
defines the Python assembly language.  The Python byte code which
this module takes as an input is defined in the file 
\file{Include/opcode.h} and used by the compiler and the interpreter.

Example: Given the function \function{myfunc}:

\begin{verbatim}
def myfunc(alist):
    return len(alist)
\end{verbatim}

the following command can be used to get the disassembly of
\function{myfunc()}:

\begin{verbatim}
>>> dis.dis(myfunc)
          0 SET_LINENO          1

          3 SET_LINENO          2
          6 LOAD_GLOBAL         0 (len)
          9 LOAD_FAST           0 (alist)
         12 CALL_FUNCTION       1
         15 RETURN_VALUE   
         16 LOAD_CONST          0 (None)
         19 RETURN_VALUE   
\end{verbatim}

The \module{dis} module defines the following functions and constants:

\begin{funcdesc}{dis}{\optional{bytesource}}
Disassemble the \var{bytesource} object. \var{bytesource} can denote
either a class, a method, a function, or a code object.  For a class,
it disassembles all methods.  For a single code sequence, it prints
one line per byte code instruction.  If no object is provided, it
disassembles the last traceback.
\end{funcdesc}

\begin{funcdesc}{distb}{\optional{tb}}
Disassembles the top-of-stack function of a traceback, using the last
traceback if none was passed.  The instruction causing the exception
is indicated.
\end{funcdesc}

\begin{funcdesc}{disassemble}{code\optional{, lasti}}
Disassembles a code object, indicating the last instruction if \var{lasti}
was provided.  The output is divided in the following columns:

\begin{enumerate}
\item the current instruction, indicated as \samp{-->},
\item a labelled instruction, indicated with \samp{>>},
\item the address of the instruction,
\item the operation code name,
\item operation parameters, and
\item interpretation of the parameters in parentheses.
\end{enumerate}

The parameter interpretation recognizes local and global
variable names, constant values, branch targets, and compare
operators.
\end{funcdesc}

\begin{funcdesc}{disco}{code\optional{, lasti}}
A synonym for disassemble.  It is more convenient to type, and kept
for compatibility with earlier Python releases.
\end{funcdesc}

\begin{datadesc}{opname}
Sequence of operation names, indexable using the byte code.
\end{datadesc}

\begin{datadesc}{cmp_op}
Sequence of all compare operation names.
\end{datadesc}

\begin{datadesc}{hasconst}
Sequence of byte codes that have a constant parameter.
\end{datadesc}

\begin{datadesc}{hasname}
Sequence of byte codes that access an attribute by name.
\end{datadesc}

\begin{datadesc}{hasjrel}
Sequence of byte codes that have a relative jump target.
\end{datadesc}

\begin{datadesc}{hasjabs}
Sequence of byte codes that have an absolute jump target.
\end{datadesc}

\begin{datadesc}{haslocal}
Sequence of byte codes that access a local variable.
\end{datadesc}

\begin{datadesc}{hascompare}
Sequence of byte codes of boolean operations.
\end{datadesc}

\subsection{Python Byte Code Instructions}
\label{bytecodes}

The Python compiler currently generates the following byte code
instructions.

\setindexsubitem{(byte code insns)}

\begin{opcodedesc}{STOP_CODE}{}
Indicates end-of-code to the compiler, not used by the interpreter.
\end{opcodedesc}

\begin{opcodedesc}{POP_TOP}{}
Removes the top-of-stack (TOS) item.
\end{opcodedesc}

\begin{opcodedesc}{ROT_TWO}{}
Swaps the two top-most stack items.
\end{opcodedesc}

\begin{opcodedesc}{ROT_THREE}{}
Lifts second and third stack item one position up, moves top down
to position three.
\end{opcodedesc}

\begin{opcodedesc}{ROT_FOUR}{}
Lifts second, third and forth stack item one position up, moves top down to
position four.
\end{opcodedesc}

\begin{opcodedesc}{DUP_TOP}{}
Duplicates the reference on top of the stack.
\end{opcodedesc}

Unary Operations take the top of the stack, apply the operation, and
push the result back on the stack.

\begin{opcodedesc}{UNARY_POSITIVE}{}
Implements \code{TOS = +TOS}.
\end{opcodedesc}

\begin{opcodedesc}{UNARY_NEGATIVE}{}
Implements \code{TOS = -TOS}.
\end{opcodedesc}

\begin{opcodedesc}{UNARY_NOT}{}
Implements \code{TOS = not TOS}.
\end{opcodedesc}

\begin{opcodedesc}{UNARY_CONVERT}{}
Implements \code{TOS = `TOS`}.
\end{opcodedesc}

\begin{opcodedesc}{UNARY_INVERT}{}
Implements \code{TOS = \~{}TOS}.
\end{opcodedesc}

Binary operations remove the top of the stack (TOS) and the second top-most
stack item (TOS1) from the stack.  They perform the operation, and put the
result back on the stack.

\begin{opcodedesc}{BINARY_POWER}{}
Implements \code{TOS = TOS1 ** TOS}.
\end{opcodedesc}

\begin{opcodedesc}{BINARY_MULTIPLY}{}
Implements \code{TOS = TOS1 * TOS}.
\end{opcodedesc}

\begin{opcodedesc}{BINARY_DIVIDE}{}
Implements \code{TOS = TOS1 / TOS}.
\end{opcodedesc}

\begin{opcodedesc}{BINARY_MODULO}{}
Implements \code{TOS = TOS1 \%{} TOS}.
\end{opcodedesc}

\begin{opcodedesc}{BINARY_ADD}{}
Implements \code{TOS = TOS1 + TOS}.
\end{opcodedesc}

\begin{opcodedesc}{BINARY_SUBTRACT}{}
Implements \code{TOS = TOS1 - TOS}.
\end{opcodedesc}

\begin{opcodedesc}{BINARY_SUBSCR}{}
Implements \code{TOS = TOS1[TOS]}.
\end{opcodedesc}

\begin{opcodedesc}{BINARY_LSHIFT}{}
Implements \code{TOS = TOS1 <\code{}< TOS}.
\end{opcodedesc}

\begin{opcodedesc}{BINARY_RSHIFT}{}
Implements \code{TOS = TOS1 >\code{}> TOS}.
\end{opcodedesc}

\begin{opcodedesc}{BINARY_AND}{}
Implements \code{TOS = TOS1 \&\ TOS}.
\end{opcodedesc}

\begin{opcodedesc}{BINARY_XOR}{}
Implements \code{TOS = TOS1 \^\ TOS}.
\end{opcodedesc}

\begin{opcodedesc}{BINARY_OR}{}
Implements \code{TOS = TOS1 | TOS}.
\end{opcodedesc}

In-place operations are like binary operations, in that they remove TOS and
TOS1, and push the result back on the stack, but the operation is done
in-place when TOS1 supports it, and the resulting TOS may be (but does not
have to be) the original TOS1.

\begin{opcodedesc}{INPLACE_POWER}{}
Implements in-place \code{TOS = TOS1 ** TOS}.
\end{opcodedesc}

\begin{opcodedesc}{INPLACE_MULTIPLY}{}
Implements in-place \code{TOS = TOS1 * TOS}.
\end{opcodedesc}

\begin{opcodedesc}{INPLACE_DIVIDE}{}
Implements in-place \code{TOS = TOS1 / TOS}.
\end{opcodedesc}

\begin{opcodedesc}{INPLACE_MODULO}{}
Implements in-place \code{TOS = TOS1 \%{} TOS}.
\end{opcodedesc}

\begin{opcodedesc}{INPLACE_ADD}{}
Implements in-place \code{TOS = TOS1 + TOS}.
\end{opcodedesc}

\begin{opcodedesc}{INPLACE_SUBTRACT}{}
Implements in-place \code{TOS = TOS1 - TOS}.
\end{opcodedesc}

\begin{opcodedesc}{INPLACE_LSHIFT}{}
Implements in-place \code{TOS = TOS1 <\code{}< TOS}.
\end{opcodedesc}

\begin{opcodedesc}{INPLACE_RSHIFT}{}
Implements in-place \code{TOS = TOS1 >\code{}> TOS}.
\end{opcodedesc}

\begin{opcodedesc}{INPLACE_AND}{}
Implements in-place \code{TOS = TOS1 \&\ TOS}.
\end{opcodedesc}

\begin{opcodedesc}{INPLACE_XOR}{}
Implements in-place \code{TOS = TOS1 \^\ TOS}.
\end{opcodedesc}

\begin{opcodedesc}{INPLACE_OR}{}
Implements in-place \code{TOS = TOS1 | TOS}.
\end{opcodedesc}

The slice opcodes take up to three parameters.

\begin{opcodedesc}{SLICE+0}{}
Implements \code{TOS = TOS[:]}.
\end{opcodedesc}

\begin{opcodedesc}{SLICE+1}{}
Implements \code{TOS = TOS1[TOS:]}.
\end{opcodedesc}

\begin{opcodedesc}{SLICE+2}{}
Implements \code{TOS = TOS1[:TOS1]}.
\end{opcodedesc}

\begin{opcodedesc}{SLICE+3}{}
Implements \code{TOS = TOS2[TOS1:TOS]}.
\end{opcodedesc}

Slice assignment needs even an additional parameter.  As any statement,
they put nothing on the stack.

\begin{opcodedesc}{STORE_SLICE+0}{}
Implements \code{TOS[:] = TOS1}.
\end{opcodedesc}

\begin{opcodedesc}{STORE_SLICE+1}{}
Implements \code{TOS1[TOS:] = TOS2}.
\end{opcodedesc}

\begin{opcodedesc}{STORE_SLICE+2}{}
Implements \code{TOS1[:TOS] = TOS2}.
\end{opcodedesc}

\begin{opcodedesc}{STORE_SLICE+3}{}
Implements \code{TOS2[TOS1:TOS] = TOS3}.
\end{opcodedesc}

\begin{opcodedesc}{DELETE_SLICE+0}{}
Implements \code{del TOS[:]}.
\end{opcodedesc}

\begin{opcodedesc}{DELETE_SLICE+1}{}
Implements \code{del TOS1[TOS:]}.
\end{opcodedesc}

\begin{opcodedesc}{DELETE_SLICE+2}{}
Implements \code{del TOS1[:TOS]}.
\end{opcodedesc}

\begin{opcodedesc}{DELETE_SLICE+3}{}
Implements \code{del TOS2[TOS1:TOS]}.
\end{opcodedesc}

\begin{opcodedesc}{STORE_SUBSCR}{}
Implements \code{TOS1[TOS] = TOS2}.
\end{opcodedesc}

\begin{opcodedesc}{DELETE_SUBSCR}{}
Implements \code{del TOS1[TOS]}.
\end{opcodedesc}

\begin{opcodedesc}{PRINT_EXPR}{}
Implements the expression statement for the interactive mode.  TOS is
removed from the stack and printed.  In non-interactive mode, an
expression statement is terminated with \code{POP_STACK}.
\end{opcodedesc}

\begin{opcodedesc}{PRINT_ITEM}{}
Prints TOS to the file-like object bound to \code{sys.stdout}.  There
is one such instruction for each item in the \keyword{print} statement.
\end{opcodedesc}

\begin{opcodedesc}{PRINT_ITEM_TO}{}
Like \code{PRINT_ITEM}, but prints the item second from TOS to the
file-like object at TOS.  This is used by the extended print statement.
\end{opcodedesc}

\begin{opcodedesc}{PRINT_NEWLINE}{}
Prints a new line on \code{sys.stdout}.  This is generated as the
last operation of a \keyword{print} statement, unless the statement
ends with a comma.
\end{opcodedesc}

\begin{opcodedesc}{PRINT_NEWLINE_TO}{}
Like \code{PRINT_NEWLINE}, but prints the new line on the file-like
object on the TOS.  This is used by the extended print statement.
\end{opcodedesc}

\begin{opcodedesc}{BREAK_LOOP}{}
Terminates a loop due to a \keyword{break} statement.
\end{opcodedesc}

\begin{opcodedesc}{LOAD_LOCALS}{}
Pushes a reference to the locals of the current scope on the stack.
This is used in the code for a class definition: After the class body
is evaluated, the locals are passed to the class definition.
\end{opcodedesc}

\begin{opcodedesc}{RETURN_VALUE}{}
Returns with TOS to the caller of the function.
\end{opcodedesc}

\begin{opcodedesc}{IMPORT_STAR}{}
Loads all symbols not starting with \character{_} directly from the module TOS
to the local namespace. The module is popped after loading all names.
This opcode implements \code{from module import *}.
\end{opcodedesc}

\begin{opcodedesc}{EXEC_STMT}{}
Implements \code{exec TOS2,TOS1,TOS}.  The compiler fills
missing optional parameters with \code{None}.
\end{opcodedesc}

\begin{opcodedesc}{POP_BLOCK}{}
Removes one block from the block stack.  Per frame, there is a 
stack of blocks, denoting nested loops, try statements, and such.
\end{opcodedesc}

\begin{opcodedesc}{END_FINALLY}{}
Terminates a \keyword{finally} clause.  The interpreter recalls
whether the exception has to be re-raised, or whether the function
returns, and continues with the outer-next block.
\end{opcodedesc}

\begin{opcodedesc}{BUILD_CLASS}{}
Creates a new class object.  TOS is the methods dictionary, TOS1
the tuple of the names of the base classes, and TOS2 the class name.
\end{opcodedesc}

All of the following opcodes expect arguments.  An argument is two
bytes, with the more significant byte last.

\begin{opcodedesc}{STORE_NAME}{namei}
Implements \code{name = TOS}. \var{namei} is the index of \var{name}
in the attribute \member{co_names} of the code object.
The compiler tries to use \code{STORE_LOCAL} or \code{STORE_GLOBAL}
if possible.
\end{opcodedesc}

\begin{opcodedesc}{DELETE_NAME}{namei}
Implements \code{del name}, where \var{namei} is the index into
\member{co_names} attribute of the code object.
\end{opcodedesc}

\begin{opcodedesc}{UNPACK_SEQUENCE}{count}
Unpacks TOS into \var{count} individual values, which are put onto
the stack right-to-left.
\end{opcodedesc}

%\begin{opcodedesc}{UNPACK_LIST}{count}
%This opcode is obsolete.
%\end{opcodedesc}

%\begin{opcodedesc}{UNPACK_ARG}{count}
%This opcode is obsolete.
%\end{opcodedesc}

\begin{opcodedesc}{DUP_TOPX}{count}
Duplicate \var{count} items, keeping them in the same order. Due to
implementation limits, \var{count} should be between 1 and 5 inclusive.
\end{opcodedesc}

\begin{opcodedesc}{STORE_ATTR}{namei}
Implements \code{TOS.name = TOS1}, where \var{namei} is the index
of name in \member{co_names}.
\end{opcodedesc}

\begin{opcodedesc}{DELETE_ATTR}{namei}
Implements \code{del TOS.name}, using \var{namei} as index into
\member{co_names}.
\end{opcodedesc}

\begin{opcodedesc}{STORE_GLOBAL}{namei}
Works as \code{STORE_NAME}, but stores the name as a global.
\end{opcodedesc}

\begin{opcodedesc}{DELETE_GLOBAL}{namei}
Works as \code{DELETE_NAME}, but deletes a global name.
\end{opcodedesc}

%\begin{opcodedesc}{UNPACK_VARARG}{argc}
%This opcode is obsolete.
%\end{opcodedesc}

\begin{opcodedesc}{LOAD_CONST}{consti}
Pushes \samp{co_consts[\var{consti}]} onto the stack.
\end{opcodedesc}

\begin{opcodedesc}{LOAD_NAME}{namei}
Pushes the value associated with \samp{co_names[\var{namei}]} onto the stack.
\end{opcodedesc}

\begin{opcodedesc}{BUILD_TUPLE}{count}
Creates a tuple consuming \var{count} items from the stack, and pushes
the resulting tuple onto the stack.
\end{opcodedesc}

\begin{opcodedesc}{BUILD_LIST}{count}
Works as \code{BUILD_TUPLE}, but creates a list.
\end{opcodedesc}

\begin{opcodedesc}{BUILD_MAP}{zero}
Pushes a new empty dictionary object onto the stack.  The argument is
ignored and set to zero by the compiler.
\end{opcodedesc}

\begin{opcodedesc}{LOAD_ATTR}{namei}
Replaces TOS with \code{getattr(TOS, co_names[\var{namei}]}.
\end{opcodedesc}

\begin{opcodedesc}{COMPARE_OP}{opname}
Performs a boolean operation.  The operation name can be found
in \code{cmp_op[\var{opname}]}.
\end{opcodedesc}

\begin{opcodedesc}{IMPORT_NAME}{namei}
Imports the module \code{co_names[\var{namei}]}.  The module object is
pushed onto the stack.  The current namespace is not affected: for a
proper import statement, a subsequent \code{STORE_FAST} instruction
modifies the namespace.
\end{opcodedesc}

\begin{opcodedesc}{IMPORT_FROM}{namei}
Loads the attribute \code{co_names[\var{namei}]} from the module found in
TOS. The resulting object is pushed onto the stack, to be subsequently
stored by a \code{STORE_FAST} instruction.
\end{opcodedesc}

\begin{opcodedesc}{JUMP_FORWARD}{delta}
Increments byte code counter by \var{delta}.
\end{opcodedesc}

\begin{opcodedesc}{JUMP_IF_TRUE}{delta}
If TOS is true, increment the byte code counter by \var{delta}.  TOS is
left on the stack.
\end{opcodedesc}

\begin{opcodedesc}{JUMP_IF_FALSE}{delta}
If TOS is false, increment the byte code counter by \var{delta}.  TOS
is not changed. 
\end{opcodedesc}

\begin{opcodedesc}{JUMP_ABSOLUTE}{target}
Set byte code counter to \var{target}.
\end{opcodedesc}

\begin{opcodedesc}{FOR_LOOP}{delta}
Iterate over a sequence.  TOS is the current index, TOS1 the sequence.
First, the next element is computed.  If the sequence is exhausted,
increment byte code counter by \var{delta}.  Otherwise, push the
sequence, the incremented counter, and the current item onto the stack.
\end{opcodedesc}

%\begin{opcodedesc}{LOAD_LOCAL}{namei}
%This opcode is obsolete.
%\end{opcodedesc}

\begin{opcodedesc}{LOAD_GLOBAL}{namei}
Loads the global named \code{co_names[\var{namei}]} onto the stack.
\end{opcodedesc}

%\begin{opcodedesc}{SET_FUNC_ARGS}{argc}
%This opcode is obsolete.
%\end{opcodedesc}

\begin{opcodedesc}{SETUP_LOOP}{delta}
Pushes a block for a loop onto the block stack.  The block spans
from the current instruction with a size of \var{delta} bytes.
\end{opcodedesc}

\begin{opcodedesc}{SETUP_EXCEPT}{delta}
Pushes a try block from a try-except clause onto the block stack.
\var{delta} points to the first except block.
\end{opcodedesc}

\begin{opcodedesc}{SETUP_FINALLY}{delta}
Pushes a try block from a try-except clause onto the block stack.
\var{delta} points to the finally block.
\end{opcodedesc}

\begin{opcodedesc}{LOAD_FAST}{var_num}
Pushes a reference to the local \code{co_varnames[\var{var_num}]} onto
the stack.
\end{opcodedesc}

\begin{opcodedesc}{STORE_FAST}{var_num}
Stores TOS into the local \code{co_varnames[\var{var_num}]}.
\end{opcodedesc}

\begin{opcodedesc}{DELETE_FAST}{var_num}
Deletes local \code{co_varnames[\var{var_num}]}.
\end{opcodedesc}

\begin{opcodedesc}{LOAD_CLOSURE}{i}
Pushes a reference to the cell contained in slot \var{i} of the
cell and free variable storage.  The name of the variable is 
\code{co_cellvars[\var{i}]} if \var{i} is less than the length of
\var{co_cellvars}.  Otherwise it is 
\code{co_freevars[\var{i} - len(co_cellvars)]}.
\end{opcodedesc}

\begin{opcodedesc}{LOAD_DEREF}{i}
Loads the cell contained in slot \var{i} of the cell and free variable
storage.  Pushes a reference to the object the cell contains on the
stack. 
\end{opcodedesc}

\begin{opcodedesc}{STORE_DEREF}{i}
Stores TOS into the cell contained in slot \var{i} of the cell and
free variable storage.
\end{opcodedesc}

\begin{opcodedesc}{SET_LINENO}{lineno}
Sets the current line number to \var{lineno}.
\end{opcodedesc}

\begin{opcodedesc}{RAISE_VARARGS}{argc}
Raises an exception. \var{argc} indicates the number of parameters
to the raise statement, ranging from 0 to 3.  The handler will find
the traceback as TOS2, the parameter as TOS1, and the exception
as TOS.
\end{opcodedesc}

\begin{opcodedesc}{CALL_FUNCTION}{argc}
Calls a function.  The low byte of \var{argc} indicates the number of
positional parameters, the high byte the number of keyword parameters.
On the stack, the opcode finds the keyword parameters first.  For each
keyword argument, the value is on top of the key.  Below the keyword
parameters, the positional parameters are on the stack, with the
right-most parameter on top.  Below the parameters, the function object
to call is on the stack.
\end{opcodedesc}

\begin{opcodedesc}{MAKE_FUNCTION}{argc}
Pushes a new function object on the stack.  TOS is the code associated
with the function.  The function object is defined to have \var{argc}
default parameters, which are found below TOS.
\end{opcodedesc}

\begin{opcodedesc}{MAKE_CLOSURE}{argc}
Creates a new function object, sets its \var{func_closure} slot, and
pushes it on the stack.  TOS is the code associated with the function.
If the code object has N free variables, the next N items on the stack
are the cells for these variables.  The function also has \var{argc}
default parameters, where are found before the cells.
\end{opcodedesc}

\begin{opcodedesc}{BUILD_SLICE}{argc}
Pushes a slice object on the stack.  \var{argc} must be 2 or 3.  If it
is 2, \code{slice(TOS1, TOS)} is pushed; if it is 3,
\code{slice(TOS2, TOS1, TOS)} is pushed.
See the \code{slice()}\bifuncindex{slice} built-in function for more
information.
\end{opcodedesc}

\begin{opcodedesc}{EXTENDED_ARG}{ext}
Prefixes any opcode which has an argument too big to fit into the
default two bytes.  \var{ext} holds two additional bytes which, taken
together with the subsequent opcode's argument, comprise a four-byte
argument, \var{ext} being the two most-significant bytes.
\end{opcodedesc}

\begin{opcodedesc}{CALL_FUNCTION_VAR}{argc}
Calls a function. \var{argc} is interpreted as in \code{CALL_FUNCTION}.
The top element on the stack contains the variable argument list, followed
by keyword and positional arguments.
\end{opcodedesc}

\begin{opcodedesc}{CALL_FUNCTION_KW}{argc}
Calls a function. \var{argc} is interpreted as in \code{CALL_FUNCTION}.
The top element on the stack contains the keyword arguments dictionary, 
followed by explicit keyword and positional arguments.
\end{opcodedesc}

\begin{opcodedesc}{CALL_FUNCTION_VAR_KW}{argc}
Calls a function. \var{argc} is interpreted as in
\code{CALL_FUNCTION}.  The top element on the stack contains the
keyword arguments dictionary, followed by the variable-arguments
tuple, followed by explicit keyword and positional arguments.
\end{opcodedesc}

\section{\module{site} ---
         Site-specific configuration hook}

\declaremodule{standard}{site}
\modulesynopsis{A standard way to reference site-specific modules.}


\strong{This module is automatically imported during initialization.}
The automatic import can be suppressed using the interpreter's
\programopt{-S} option.

Importing this module will append site-specific paths to the module
search path.
\indexiii{module}{search}{path}

It starts by constructing up to four directories from a head and a
tail part.  For the head part, it uses \code{sys.prefix} and
\code{sys.exec_prefix}; empty heads are skipped.  For
the tail part, it uses the empty string (on Windows) or
\file{lib/python\shortversion/site-packages} (on \UNIX{} and Macintosh)
and then \file{lib/site-python}.  For each of the distinct
head-tail combinations, it sees if it refers to an existing directory,
and if so, adds it to \code{sys.path} and also inspects the newly added 
path for configuration files.
\indexii{site-python}{directory}
\indexii{site-packages}{directory}

A path configuration file is a file whose name has the form
\file{\var{package}.pth} and exists in one of the four directories
mentioned above; its contents are additional items (one
per line) to be added to \code{sys.path}.  Non-existing items are
never added to \code{sys.path}, but no check is made that the item
refers to a directory (rather than a file).  No item is added to
\code{sys.path} more than once.  Blank lines and lines beginning with
\code{\#} are skipped.  Lines starting with \code{import} are executed.
\index{package}
\indexiii{path}{configuration}{file}

For example, suppose \code{sys.prefix} and \code{sys.exec_prefix} are
set to \file{/usr/local}.  The Python \version\ library is then
installed in \file{/usr/local/lib/python\shortversion} (where only the
first three characters of \code{sys.version} are used to form the
installation path name).  Suppose this has a subdirectory
\file{/usr/local/lib/python\shortversion/site-packages} with three
subsubdirectories, \file{foo}, \file{bar} and \file{spam}, and two
path configuration files, \file{foo.pth} and \file{bar.pth}.  Assume
\file{foo.pth} contains the following:

\begin{verbatim}
# foo package configuration

foo
bar
bletch
\end{verbatim}

and \file{bar.pth} contains:

\begin{verbatim}
# bar package configuration

bar
\end{verbatim}

Then the following directories are added to \code{sys.path}, in this
order:

\begin{verbatim}
/usr/local/lib/python2.3/site-packages/bar
/usr/local/lib/python2.3/site-packages/foo
\end{verbatim}

Note that \file{bletch} is omitted because it doesn't exist; the
\file{bar} directory precedes the \file{foo} directory because
\file{bar.pth} comes alphabetically before \file{foo.pth}; and
\file{spam} is omitted because it is not mentioned in either path
configuration file.

After these path manipulations, an attempt is made to import a module
named \module{sitecustomize}\refmodindex{sitecustomize}, which can
perform arbitrary site-specific customizations.  If this import fails
with an \exception{ImportError} exception, it is silently ignored.

Note that for some non-\UNIX{} systems, \code{sys.prefix} and
\code{sys.exec_prefix} are empty, and the path manipulations are
skipped; however the import of
\module{sitecustomize}\refmodindex{sitecustomize} is still attempted.

\section{\module{user} ---
         User-specific configuration hook}

\declaremodule{standard}{user}
\modulesynopsis{A standard way to reference user-specific modules.}


\indexii{.pythonrc.py}{file}
\indexiii{user}{configuration}{file}

As a policy, Python doesn't run user-specified code on startup of
Python programs.  (Only interactive sessions execute the script
specified in the \envvar{PYTHONSTARTUP} environment variable if it
exists).

However, some programs or sites may find it convenient to allow users
to have a standard customization file, which gets run when a program
requests it.  This module implements such a mechanism.  A program
that wishes to use the mechanism must execute the statement

\begin{verbatim}
import user
\end{verbatim}

The \module{user} module looks for a file \file{.pythonrc.py} in the user's
home directory and if it can be opened, executes it (using
\function{exec()}\bifuncindex{exec}) in its own (the
module \module{user}'s) global namespace.  Errors during this phase
are not caught; that's up to the program that imports the
\module{user} module, if it wishes.  The home directory is assumed to
be named by the \envvar{HOME} environment variable; if this is not set,
the current directory is used.

The user's \file{.pythonrc.py} could conceivably test for
\code{sys.version} if it wishes to do different things depending on
the Python version.

A warning to users: be very conservative in what you place in your
\file{.pythonrc.py} file.  Since you don't know which programs will
use it, changing the behavior of standard modules or functions is
generally not a good idea.

A suggestion for programmers who wish to use this mechanism: a simple
way to let users specify options for your package is to have them
define variables in their \file{.pythonrc.py} file that you test in
your module.  For example, a module \module{spam} that has a verbosity
level can look for a variable \code{user.spam_verbose}, as follows:

\begin{verbatim}
import user

verbose = bool(getattr(user, "spam_verbose", 0))
\end{verbatim}

(The three-argument form of \function{getattr()} is used in case
the user has not defined \code{spam_verbose} in their
\file{.pythonrc.py} file.)

Programs with extensive customization needs are better off reading a
program-specific customization file.

Programs with security or privacy concerns should \emph{not} import
this module; a user can easily break into a program by placing
arbitrary code in the \file{.pythonrc.py} file.

Modules for general use should \emph{not} import this module; it may
interfere with the operation of the importing program.

\begin{seealso}
  \seemodule{site}{Site-wide customization mechanism.}
\end{seealso}

\section{\module{__builtin__} ---
         Built-in functions.}
\declaremodule[builtin]{builtin}{__builtin__}

\modulesynopsis{The set of built-in functions.}


This module provides direct access to all `built-in' identifiers of
Python; e.g. \code{__builtin__.open} is the full name for the built-in
function \code{open()}.  See section \ref{built-in-funcs}, ``Built-in
Functions.''
		% really __builtin__
\section{\module{__main__} ---
         Top-level script environment}

\declaremodule[main]{builtin}{__main__}
\modulesynopsis{The environment where the top-level script is run.}

This module represents the (otherwise anonymous) scope in which the
interpreter's main program executes --- commands read either from
standard input, from a script file, or from an interactive prompt.  It
is this environment in which the idiomatic ``conditional script''
stanza causes a script to run:

\begin{verbatim}
if __name__ == "__main__":
    main()
\end{verbatim}
			% really __main__

\chapter{String Services}

The modules described in this chapter provide a wide range of string
manipulation operations.  Here's an overview:

\begin{description}

\item[string]
--- Common string operations.

\item[re]
--- New Perl-style regular expression search and match operations.

\item[regex]
--- Regular expression search and match operations.

\item[regsub]
--- Substitution and splitting operations that use regular expressions.

\item[struct]
--- Interpret strings as packed binary data.

\item[StringIO]
--- Read and write strings as if they were files.

\item[soundex]
--- Compute hash values for English words.

\end{description}
		% String Services
\section{\module{string} ---
         Common string operations}

\declaremodule{standard}{string}
\modulesynopsis{Common string operations.}

The \module{string} module contains a number of useful constants and classes,
as well as some deprecated legacy functions that are also available as methods
on strings.  See the module \refmodule{re}\refstmodindex{re} for string
functions based on regular expressions.

\subsection{String constants}

The constants defined in this module are:

\begin{datadesc}{ascii_letters}
  The concatenation of the \constant{ascii_lowercase} and
  \constant{ascii_uppercase} constants described below.  This value is
  not locale-dependent.
\end{datadesc}

\begin{datadesc}{ascii_lowercase}
  The lowercase letters \code{'abcdefghijklmnopqrstuvwxyz'}.  This
  value is not locale-dependent and will not change.
\end{datadesc}

\begin{datadesc}{ascii_uppercase}
  The uppercase letters \code{'ABCDEFGHIJKLMNOPQRSTUVWXYZ'}.  This
  value is not locale-dependent and will not change.
\end{datadesc}

\begin{datadesc}{digits}
  The string \code{'0123456789'}.
\end{datadesc}

\begin{datadesc}{hexdigits}
  The string \code{'0123456789abcdefABCDEF'}.
\end{datadesc}

\begin{datadesc}{letters}
  The concatenation of the strings \constant{lowercase} and
  \constant{uppercase} described below.  The specific value is
  locale-dependent, and will be updated when
  \function{locale.setlocale()} is called.
\end{datadesc}

\begin{datadesc}{lowercase}
  A string containing all the characters that are considered lowercase
  letters.  On most systems this is the string
  \code{'abcdefghijklmnopqrstuvwxyz'}.  Do not change its definition ---
  the effect on the routines \function{upper()} and
  \function{swapcase()} is undefined.  The specific value is
  locale-dependent, and will be updated when
  \function{locale.setlocale()} is called.
\end{datadesc}

\begin{datadesc}{octdigits}
  The string \code{'01234567'}.
\end{datadesc}

\begin{datadesc}{punctuation}
  String of \ASCII{} characters which are considered punctuation
  characters in the \samp{C} locale.
\end{datadesc}

\begin{datadesc}{printable}
  String of characters which are considered printable.  This is a
  combination of \constant{digits}, \constant{letters},
  \constant{punctuation}, and \constant{whitespace}.
\end{datadesc}

\begin{datadesc}{uppercase}
  A string containing all the characters that are considered uppercase
  letters.  On most systems this is the string
  \code{'ABCDEFGHIJKLMNOPQRSTUVWXYZ'}.  Do not change its definition ---
  the effect on the routines \function{lower()} and
  \function{swapcase()} is undefined.  The specific value is
  locale-dependent, and will be updated when
  \function{locale.setlocale()} is called.
\end{datadesc}

\begin{datadesc}{whitespace}
  A string containing all characters that are considered whitespace.
  On most systems this includes the characters space, tab, linefeed,
  return, formfeed, and vertical tab.  Do not change its definition ---
  the effect on the routines \function{strip()} and \function{split()}
  is undefined.
\end{datadesc}

\subsection{Template strings}

Templates provide simpler string substitutions as described in \pep{292}.
Instead of the normal \samp{\%}-based substitutions, Templates support
\samp{\$}-based substitutions, using the following rules:

\begin{itemize}
\item \samp{\$\$} is an escape; it is replaced with a single \samp{\$}.

\item \samp{\$identifier} names a substitution placeholder matching a mapping
       key of "identifier".  By default, "identifier" must spell a Python
       identifier.  The first non-identifier character after the \samp{\$}
       character terminates this placeholder specification.

\item \samp{\$\{identifier\}} is equivalent to \samp{\$identifier}.  It is
      required when valid identifier characters follow the placeholder but are
      not part of the placeholder, such as "\$\{noun\}ification".
\end{itemize}

Any other appearance of \samp{\$} in the string will result in a
\exception{ValueError} being raised.

\versionadded{2.4}

The \module{string} module provides a \class{Template} class that implements
these rules.  The methods of \class{Template} are:

\begin{classdesc}{Template}{template}
The constructor takes a single argument which is the template string.
\end{classdesc}

\begin{methoddesc}[Template]{substitute}{mapping\optional{, **kws}}
Performs the template substitution, returning a new string.  \var{mapping} is
any dictionary-like object with keys that match the placeholders in the
template.  Alternatively, you can provide keyword arguments, where the
keywords are the placeholders.  When both \var{mapping} and \var{kws} are
given and there are duplicates, the placeholders from \var{kws} take
precedence.
\end{methoddesc}

\begin{methoddesc}[Template]{safe_substitute}{mapping\optional{, **kws}}
Like \method{substitute()}, except that if placeholders are missing from
\var{mapping} and \var{kws}, instead of raising a \exception{KeyError}
exception, the original placeholder will appear in the resulting string
intact.  Also, unlike with \method{substitute()}, any other appearances of the
\samp{\$} will simply return \samp{\$} instead of raising
\exception{ValueError}.

While other exceptions may still occur, this method is called ``safe'' because
substitutions always tries to return a usable string instead of raising an
exception.  In another sense, \method{safe_substitute()} may be anything other
than safe, since it will silently ignore malformed templates containing
dangling delimiters, unmatched braces, or placeholders that are not valid
Python identifiers.
\end{methoddesc}

\class{Template} instances also provide one public data attribute:

\begin{memberdesc}[string]{template}
This is the object passed to the constructor's \var{template} argument.  In
general, you shouldn't change it, but read-only access is not enforced.
\end{memberdesc}

Here is an example of how to use a Template:

\begin{verbatim}
>>> from string import Template
>>> s = Template('$who likes $what')
>>> s.substitute(who='tim', what='kung pao')
'tim likes kung pao'
>>> d = dict(who='tim')
>>> Template('Give $who $100').substitute(d)
Traceback (most recent call last):
[...]
ValueError: Invalid placeholder in string: line 1, col 10
>>> Template('$who likes $what').substitute(d)
Traceback (most recent call last):
[...]
KeyError: 'what'
>>> Template('$who likes $what').safe_substitute(d)
'tim likes $what'
\end{verbatim}

Advanced usage: you can derive subclasses of \class{Template} to customize the
placeholder syntax, delimiter character, or the entire regular expression used
to parse template strings.  To do this, you can override these class
attributes:

\begin{itemize}
\item \var{delimiter} -- This is the literal string describing a placeholder
      introducing delimiter.  The default value \samp{\$}.  Note that this
      should \emph{not} be a regular expression, as the implementation will
      call \method{re.escape()} on this string as needed.
\item \var{idpattern} -- This is the regular expression describing the pattern
      for non-braced placeholders (the braces will be added automatically as
      appropriate).  The default value is the regular expression
      \samp{[_a-z][_a-z0-9]*}.
\end{itemize}

Alternatively, you can provide the entire regular expression pattern by
overriding the class attribute \var{pattern}.  If you do this, the value must
be a regular expression object with four named capturing groups.  The
capturing groups correspond to the rules given above, along with the invalid
placeholder rule:

\begin{itemize}
\item \var{escaped} -- This group matches the escape sequence,
      e.g. \samp{\$\$}, in the default pattern.
\item \var{named} -- This group matches the unbraced placeholder name; it
      should not include the delimiter in capturing group.
\item \var{braced} -- This group matches the brace enclosed placeholder name;
      it should not include either the delimiter or braces in the capturing
      group.
\item \var{invalid} -- This group matches any other delimiter pattern (usually
      a single delimiter), and it should appear last in the regular
      expression.
\end{itemize}

\subsection{String functions}

The following functions are available to operate on string and Unicode
objects.  They are not available as string methods.

\begin{funcdesc}{capwords}{s}
  Split the argument into words using \function{split()}, capitalize
  each word using \function{capitalize()}, and join the capitalized
  words using \function{join()}.  Note that this replaces runs of
  whitespace characters by a single space, and removes leading and
  trailing whitespace.
\end{funcdesc}

\begin{funcdesc}{maketrans}{from, to}
  Return a translation table suitable for passing to
  \function{translate()}, that will map
  each character in \var{from} into the character at the same position
  in \var{to}; \var{from} and \var{to} must have the same length.

  \warning{Don't use strings derived from \constant{lowercase}
  and \constant{uppercase} as arguments; in some locales, these don't have
  the same length.  For case conversions, always use
  \function{lower()} and \function{upper()}.}
\end{funcdesc}

\subsection{Deprecated string functions}

The following list of functions are also defined as methods of string and
Unicode objects; see ``String Methods'' (section
\ref{string-methods}) for more information on those.  You should consider
these functions as deprecated, although they will not be removed until Python
3.0.  The functions defined in this module are:

\begin{funcdesc}{atof}{s}
  \deprecated{2.0}{Use the \function{float()} built-in function.}
  Convert a string to a floating point number.  The string must have
  the standard syntax for a floating point literal in Python,
  optionally preceded by a sign (\samp{+} or \samp{-}).  Note that
  this behaves identical to the built-in function
  \function{float()}\bifuncindex{float} when passed a string.

  \note{When passing in a string, values for NaN\index{NaN}
  and Infinity\index{Infinity} may be returned, depending on the
  underlying C library.  The specific set of strings accepted which
  cause these values to be returned depends entirely on the C library
  and is known to vary.}
\end{funcdesc}

\begin{funcdesc}{atoi}{s\optional{, base}}
  \deprecated{2.0}{Use the \function{int()} built-in function.}
  Convert string \var{s} to an integer in the given \var{base}.  The
  string must consist of one or more digits, optionally preceded by a
  sign (\samp{+} or \samp{-}).  The \var{base} defaults to 10.  If it
  is 0, a default base is chosen depending on the leading characters
  of the string (after stripping the sign): \samp{0x} or \samp{0X}
  means 16, \samp{0} means 8, anything else means 10.  If \var{base}
  is 16, a leading \samp{0x} or \samp{0X} is always accepted, though
  not required.  This behaves identically to the built-in function
  \function{int()} when passed a string.  (Also note: for a more
  flexible interpretation of numeric literals, use the built-in
  function \function{eval()}\bifuncindex{eval}.)
\end{funcdesc}

\begin{funcdesc}{atol}{s\optional{, base}}
  \deprecated{2.0}{Use the \function{long()} built-in function.}
  Convert string \var{s} to a long integer in the given \var{base}.
  The string must consist of one or more digits, optionally preceded
  by a sign (\samp{+} or \samp{-}).  The \var{base} argument has the
  same meaning as for \function{atoi()}.  A trailing \samp{l} or
  \samp{L} is not allowed, except if the base is 0.  Note that when
  invoked without \var{base} or with \var{base} set to 10, this
  behaves identical to the built-in function
  \function{long()}\bifuncindex{long} when passed a string.
\end{funcdesc}

\begin{funcdesc}{capitalize}{word}
  Return a copy of \var{word} with only its first character capitalized.
\end{funcdesc}

\begin{funcdesc}{expandtabs}{s\optional{, tabsize}}
  Expand tabs in a string replacing them by one or more spaces,
  depending on the current column and the given tab size.  The column
  number is reset to zero after each newline occurring in the string.
  This doesn't understand other non-printing characters or escape
  sequences.  The tab size defaults to 8.
\end{funcdesc}

\begin{funcdesc}{find}{s, sub\optional{, start\optional{,end}}}
  Return the lowest index in \var{s} where the substring \var{sub} is
  found such that \var{sub} is wholly contained in
  \code{\var{s}[\var{start}:\var{end}]}.  Return \code{-1} on failure.
  Defaults for \var{start} and \var{end} and interpretation of
  negative values is the same as for slices.
\end{funcdesc}

\begin{funcdesc}{rfind}{s, sub\optional{, start\optional{, end}}}
  Like \function{find()} but find the highest index.
\end{funcdesc}

\begin{funcdesc}{index}{s, sub\optional{, start\optional{, end}}}
  Like \function{find()} but raise \exception{ValueError} when the
  substring is not found.
\end{funcdesc}

\begin{funcdesc}{rindex}{s, sub\optional{, start\optional{, end}}}
  Like \function{rfind()} but raise \exception{ValueError} when the
  substring is not found.
\end{funcdesc}

\begin{funcdesc}{count}{s, sub\optional{, start\optional{, end}}}
  Return the number of (non-overlapping) occurrences of substring
  \var{sub} in string \code{\var{s}[\var{start}:\var{end}]}.
  Defaults for \var{start} and \var{end} and interpretation of
  negative values are the same as for slices.
\end{funcdesc}

\begin{funcdesc}{lower}{s}
  Return a copy of \var{s}, but with upper case letters converted to
  lower case.
\end{funcdesc}

\begin{funcdesc}{split}{s\optional{, sep\optional{, maxsplit}}}
  Return a list of the words of the string \var{s}.  If the optional
  second argument \var{sep} is absent or \code{None}, the words are
  separated by arbitrary strings of whitespace characters (space, tab, 
  newline, return, formfeed).  If the second argument \var{sep} is
  present and not \code{None}, it specifies a string to be used as the 
  word separator.  The returned list will then have one more item
  than the number of non-overlapping occurrences of the separator in
  the string.  The optional third argument \var{maxsplit} defaults to
  0.  If it is nonzero, at most \var{maxsplit} number of splits occur,
  and the remainder of the string is returned as the final element of
  the list (thus, the list will have at most \code{\var{maxsplit}+1}
  elements).

  The behavior of split on an empty string depends on the value of \var{sep}.
  If \var{sep} is not specified, or specified as \code{None}, the result will
  be an empty list.  If \var{sep} is specified as any string, the result will
  be a list containing one element which is an empty string.
\end{funcdesc}

\begin{funcdesc}{rsplit}{s\optional{, sep\optional{, maxsplit}}}
  Return a list of the words of the string \var{s}, scanning \var{s}
  from the end.  To all intents and purposes, the resulting list of
  words is the same as returned by \function{split()}, except when the
  optional third argument \var{maxsplit} is explicitly specified and
  nonzero.  When \var{maxsplit} is nonzero, at most \var{maxsplit}
  number of splits -- the \emph{rightmost} ones -- occur, and the remainder
  of the string is returned as the first element of the list (thus, the
  list will have at most \code{\var{maxsplit}+1} elements).
  \versionadded{2.4}
\end{funcdesc}

\begin{funcdesc}{splitfields}{s\optional{, sep\optional{, maxsplit}}}
  This function behaves identically to \function{split()}.  (In the
  past, \function{split()} was only used with one argument, while
  \function{splitfields()} was only used with two arguments.)
\end{funcdesc}

\begin{funcdesc}{join}{words\optional{, sep}}
  Concatenate a list or tuple of words with intervening occurrences of 
  \var{sep}.  The default value for \var{sep} is a single space
  character.  It is always true that
  \samp{string.join(string.split(\var{s}, \var{sep}), \var{sep})}
  equals \var{s}.
\end{funcdesc}

\begin{funcdesc}{joinfields}{words\optional{, sep}}
  This function behaves identically to \function{join()}.  (In the past, 
  \function{join()} was only used with one argument, while
  \function{joinfields()} was only used with two arguments.)
  Note that there is no \method{joinfields()} method on string
  objects; use the \method{join()} method instead.
\end{funcdesc}

\begin{funcdesc}{lstrip}{s\optional{, chars}}
Return a copy of the string with leading characters removed.  If
\var{chars} is omitted or \code{None}, whitespace characters are
removed.  If given and not \code{None}, \var{chars} must be a string;
the characters in the string will be stripped from the beginning of
the string this method is called on.
\versionchanged[The \var{chars} parameter was added.  The \var{chars}
parameter cannot be passed in earlier 2.2 versions]{2.2.3}
\end{funcdesc}

\begin{funcdesc}{rstrip}{s\optional{, chars}}
Return a copy of the string with trailing characters removed.  If
\var{chars} is omitted or \code{None}, whitespace characters are
removed.  If given and not \code{None}, \var{chars} must be a string;
the characters in the string will be stripped from the end of the
string this method is called on.
\versionchanged[The \var{chars} parameter was added.  The \var{chars}
parameter cannot be passed in earlier 2.2 versions]{2.2.3}
\end{funcdesc}

\begin{funcdesc}{strip}{s\optional{, chars}}
Return a copy of the string with leading and trailing characters
removed.  If \var{chars} is omitted or \code{None}, whitespace
characters are removed.  If given and not \code{None}, \var{chars}
must be a string; the characters in the string will be stripped from
the both ends of the string this method is called on.
\versionchanged[The \var{chars} parameter was added.  The \var{chars}
parameter cannot be passed in earlier 2.2 versions]{2.2.3}
\end{funcdesc}

\begin{funcdesc}{swapcase}{s}
  Return a copy of \var{s}, but with lower case letters
  converted to upper case and vice versa.
\end{funcdesc}

\begin{funcdesc}{translate}{s, table\optional{, deletechars}}
  Delete all characters from \var{s} that are in \var{deletechars} (if 
  present), and then translate the characters using \var{table}, which 
  must be a 256-character string giving the translation for each
  character value, indexed by its ordinal.
\end{funcdesc}

\begin{funcdesc}{upper}{s}
  Return a copy of \var{s}, but with lower case letters converted to
  upper case.
\end{funcdesc}

\begin{funcdesc}{ljust}{s, width}
\funcline{rjust}{s, width}
\funcline{center}{s, width}
  These functions respectively left-justify, right-justify and center
  a string in a field of given width.  They return a string that is at
  least \var{width} characters wide, created by padding the string
  \var{s} with spaces until the given width on the right, left or both
  sides.  The string is never truncated.
\end{funcdesc}

\begin{funcdesc}{zfill}{s, width}
  Pad a numeric string on the left with zero digits until the given
  width is reached.  Strings starting with a sign are handled
  correctly.
\end{funcdesc}

\begin{funcdesc}{replace}{str, old, new\optional{, maxreplace}}
  Return a copy of string \var{str} with all occurrences of substring
  \var{old} replaced by \var{new}.  If the optional argument
  \var{maxreplace} is given, the first \var{maxreplace} occurrences are
  replaced.
\end{funcdesc}

\section{\module{re} ---
         New Perl-style regular expression search and match operations.}
\declaremodule{standard}{re}
\moduleauthor{Andrew M. Kuchling}{akuchling@acm.org}
\sectionauthor{Andrew M. Kuchling}{akuchling@acm.org}


\modulesynopsis{New Perl-style regular expression search and match
operations.}


This module provides regular expression matching operations similar to
those found in Perl.  It's 8-bit clean: the strings being processed
may contain both null bytes and characters whose high bit is set.  Regular
expression patterns may not contain null bytes, but they may contain
characters with the high bit set.  The \module{re} module is always
available.

Regular expressions use the backslash character (\character{\e}) to
indicate special forms or to allow special characters to be used
without invoking their special meaning.  This collides with Python's
usage of the same character for the same purpose in string literals;
for example, to match a literal backslash, one might have to write
\code{'\e\e\e\e'} as the pattern string, because the regular expression
must be \samp{\e\e}, and each backslash must be expressed as
\samp{\e\e} inside a regular Python string literal. 

The solution is to use Python's raw string notation for regular
expression patterns; backslashes are not handled in any special way in
a string literal prefixed with \character{r}.  So \code{r"\e n"} is a
two-character string containing \character{\e} and \character{n},
while \code{"\e n"} is a one-character string containing a newline.
Usually patterns will be expressed in Python code using this raw
string notation.

\subsection{Regular Expression Syntax \label{re-syntax}}

A regular expression (or RE) specifies a set of strings that matches
it; the functions in this module let you check if a particular string
matches a given regular expression (or if a given regular expression
matches a particular string, which comes down to the same thing).

Regular expressions can be concatenated to form new regular
expressions; if \emph{A} and \emph{B} are both regular expressions,
then \emph{AB} is also an regular expression.  If a string \emph{p}
matches A and another string \emph{q} matches B, the string \emph{pq}
will match AB.  Thus, complex expressions can easily be constructed
from simpler primitive expressions like the ones described here.  For
details of the theory and implementation of regular expressions,
consult the Friedl book referenced below, or almost any textbook about
compiler construction.

A brief explanation of the format of regular expressions follows.  For
further information and a gentler presentation, consult the Regular
Expression HOWTO, accessible from \url{http://www.python.org/doc/howto/}.

Regular expressions can contain both special and ordinary characters.
Most ordinary characters, like \character{A}, \character{a}, or \character{0},
are the simplest regular expressions; they simply match themselves.  
You can concatenate ordinary characters, so \regexp{last} matches the
string \code{'last'}.  (In the rest of this section, we'll write RE's in
\regexp{this special style}, usually without quotes, and strings to be
matched \code{'in single quotes'}.)

Some characters, like \character{|} or \character{(}, are special.  Special
characters either stand for classes of ordinary characters, or affect
how the regular expressions around them are interpreted.

The special characters are:

\begin{list}{}{\leftmargin 0.7in \labelwidth 0.65in}

\item[\character{.}] (Dot.)  In the default mode, this matches any
character except a newline.  If the \constant{DOTALL} flag has been
specified, this matches any character including a newline.

\item[\character{\^}] (Caret.)  Matches the start of the string, and in
\constant{MULTILINE} mode also matches immediately after each newline.

\item[\character{\$}] Matches the end of the string, and in
\constant{MULTILINE} mode also matches before a newline.
\regexp{foo} matches both 'foo' and 'foobar', while the regular
expression \regexp{foo\$} matches only 'foo'.

\item[\character{*}] Causes the resulting RE to
match 0 or more repetitions of the preceding RE, as many repetitions
as are possible.  \regexp{ab*} will
match 'a', 'ab', or 'a' followed by any number of 'b's.

\item[\character{+}] Causes the
resulting RE to match 1 or more repetitions of the preceding RE.
\regexp{ab+} will match 'a' followed by any non-zero number of 'b's; it
will not match just 'a'.

\item[\character{?}] Causes the resulting RE to
match 0 or 1 repetitions of the preceding RE.  \regexp{ab?} will
match either 'a' or 'ab'.
\item[\code{*?}, \code{+?}, \code{??}] The \character{*}, \character{+}, and
\character{?} qualifiers are all \dfn{greedy}; they match as much text as
possible.  Sometimes this behaviour isn't desired; if the RE
\regexp{<.*>} is matched against \code{'<H1>title</H1>'}, it will match the
entire string, and not just \code{'<H1>'}.
Adding \character{?} after the qualifier makes it perform the match in
\dfn{non-greedy} or \dfn{minimal} fashion; as \emph{few} characters as
possible will be matched.  Using \regexp{.*?} in the previous
expression will match only \code{'<H1>'}.

\item[\code{\{\var{m},\var{n}\}}] Causes the resulting RE to match from
\var{m} to \var{n} repetitions of the preceding RE, attempting to
match as many repetitions as possible.  For example, \regexp{a\{3,5\}}
will match from 3 to 5 \character{a} characters.  Omitting \var{n}
specifies an infinite upper bound; you can't omit \var{m}.

\item[\code{\{\var{m},\var{n}\}?}] Causes the resulting RE to
match from \var{m} to \var{n} repetitions of the preceding RE,
attempting to match as \emph{few} repetitions as possible.  This is
the non-greedy version of the previous qualifier.  For example, on the
6-character string \code{'aaaaaa'}, \regexp{a\{3,5\}} will match 5
\character{a} characters, while \regexp{a\{3,5\}?} will only match 3
characters.

\item[\character{\e}] Either escapes special characters (permitting
you to match characters like \character{*}, \character{?}, and so
forth), or signals a special sequence; special sequences are discussed
below.

If you're not using a raw string to
express the pattern, remember that Python also uses the
backslash as an escape sequence in string literals; if the escape
sequence isn't recognized by Python's parser, the backslash and
subsequent character are included in the resulting string.  However,
if Python would recognize the resulting sequence, the backslash should
be repeated twice.  This is complicated and hard to understand, so
it's highly recommended that you use raw strings for all but the
simplest expressions.

\item[\code{[]}] Used to indicate a set of characters.  Characters can
be listed individually, or a range of characters can be indicated by
giving two characters and separating them by a \character{-}.  Special
characters are not active inside sets.  For example, \regexp{[akm\$]}
will match any of the characters \character{a}, \character{k},
\character{m}, or \character{\$}; \regexp{[a-z]}
will match any lowercase letter, and \code{[a-zA-Z0-9]} matches any
letter or digit.  Character classes such as \code{\e w} or \code{\e S}
(defined below) are also acceptable inside a range.  If you want to
include a \character{]} or a \character{-} inside a set, precede it with a
backslash, or place it as the first character.  The 
pattern \regexp{[]]} will match \code{']'}, for example.  

You can match the characters not within a range by \dfn{complementing}
the set.  This is indicated by including a
\character{\^} as the first character of the set; \character{\^} elsewhere will
simply match the \character{\^} character.  For example, \regexp{[{\^}5]}
will match any character except \character{5}.

\item[\character{|}]\code{A|B}, where A and B can be arbitrary REs,
creates a regular expression that will match either A or B.  This can
be used inside groups (see below) as well.  To match a literal \character{|},
use \regexp{\e|}, or enclose it inside a character class, as in  \regexp{[|]}.

\item[\code{(...)}] Matches whatever regular expression is inside the
parentheses, and indicates the start and end of a group; the contents
of a group can be retrieved after a match has been performed, and can
be matched later in the string with the \regexp{\e \var{number}} special
sequence, described below.  To match the literals \character{(} or
\character{')}, use \regexp{\e(} or \regexp{\e)}, or enclose them
inside a character class: \regexp{[(] [)]}.

\item[\code{(?...)}] This is an extension notation (a \character{?}
following a \character{(} is not meaningful otherwise).  The first
character after the \character{?} 
determines what the meaning and further syntax of the construct is.
Extensions usually do not create a new group;
\regexp{(?P<\var{name}>...)} is the only exception to this rule.
Following are the currently supported extensions.

\item[\code{(?iLmsx)}] (One or more letters from the set \character{i},
\character{L}, \character{m}, \character{s}, \character{x}.)  The group matches
the empty string; the letters set the corresponding flags
(\constant{re.I}, \constant{re.L}, \constant{re.M}, \constant{re.S},
\constant{re.X}) for the entire regular expression.  This is useful if
you wish to include the flags as part of the regular expression, instead
of passing a \var{flag} argument to the \function{compile()} function. 

\item[\code{(?:...)}] A non-grouping version of regular parentheses.
Matches whatever regular expression is inside the parentheses, but the
substring matched by the 
group \emph{cannot} be retrieved after performing a match or
referenced later in the pattern. 

\item[\code{(?P<\var{name}>...)}] Similar to regular parentheses, but
the substring matched by the group is accessible via the symbolic group
name \var{name}.  Group names must be valid Python identifiers.  A
symbolic group is also a numbered group, just as if the group were not
named.  So the group named 'id' in the example above can also be
referenced as the numbered group 1.

For example, if the pattern is
\regexp{(?P<id>[a-zA-Z_]\e w*)}, the group can be referenced by its
name in arguments to methods of match objects, such as \code{m.group('id')}
or \code{m.end('id')}, and also by name in pattern text
(e.g. \regexp{(?P=id)}) and replacement text (e.g. \code{\e g<id>}).

\item[\code{(?P=\var{name})}] Matches whatever text was matched by the
earlier group named \var{name}.

\item[\code{(?\#...)}] A comment; the contents of the parentheses are
simply ignored.

\item[\code{(?=...)}] Matches if \regexp{...} matches next, but doesn't
consume any of the string.  This is called a lookahead assertion.  For
example, \regexp{Isaac (?=Asimov)} will match \code{'Isaac~'} only if it's
followed by \code{'Asimov'}.

\item[\code{(?!...)}] Matches if \regexp{...} doesn't match next.  This
is a negative lookahead assertion.  For example,
\regexp{Isaac (?!Asimov)} will match \code{'Isaac~'} only if it's \emph{not}
followed by \code{'Asimov'}.

\end{list}

The special sequences consist of \character{\e} and a character from the
list below.  If the ordinary character is not on the list, then the
resulting RE will match the second character.  For example,
\regexp{\e\$} matches the character \character{\$}.

\begin{list}{}{\leftmargin 0.7in \labelwidth 0.65in}

%
\item[\code{\e \var{number}}] Matches the contents of the group of the
same number.  Groups are numbered starting from 1.  For example,
\regexp{(.+) \e 1} matches \code{'the the'} or \code{'55 55'}, but not
\code{'the end'} (note 
the space after the group).  This special sequence can only be used to
match one of the first 99 groups.  If the first digit of \var{number}
is 0, or \var{number} is 3 octal digits long, it will not be interpreted
as a group match, but as the character with octal value \var{number}.
Inside the \character{[} and \character{]} of a character class, all numeric
escapes are treated as characters. 
%
\item[\code{\e A}] Matches only at the start of the string.
%
\item[\code{\e b}] Matches the empty string, but only at the
beginning or end of a word.  A word is defined as a sequence of
alphanumeric characters, so the end of a word is indicated by
whitespace or a non-alphanumeric character.  Inside a character range,
\regexp{\e b} represents the backspace character, for compatibility with
Python's string literals.
%
\item[\code{\e B}] Matches the empty string, but only when it is
\emph{not} at the beginning or end of a word.
%
\item[\code{\e d}]Matches any decimal digit; this is
equivalent to the set \regexp{[0-9]}.
%
\item[\code{\e D}]Matches any non-digit character; this is
equivalent to the set \regexp{[{\^}0-9]}.
%
\item[\code{\e s}]Matches any whitespace character; this is
equivalent to the set \regexp{[ \e t\e n\e r\e f\e v]}.
%
\item[\code{\e S}]Matches any non-whitespace character; this is
equivalent to the set \regexp{[\^\ \e t\e n\e r\e f\e v]}.
%
\item[\code{\e w}]When the \constant{LOCALE} flag is not specified,
matches any alphanumeric character; this is equivalent to the set
\regexp{[a-zA-Z0-9_]}.  With \constant{LOCALE}, it will match the set
\regexp{[0-9_]} plus whatever characters are defined as letters for the
current locale.
%
\item[\code{\e W}]When the \constant{LOCALE} flag is not specified,
matches any non-alphanumeric character; this is equivalent to the set
\regexp{[{\^}a-zA-Z0-9_]}.   With \constant{LOCALE}, it will match any
character not in the set \regexp{[0-9_]}, and not defined as a letter
for the current locale.

\item[\code{\e Z}]Matches only at the end of the string.
%

\item[\code{\e \e}] Matches a literal backslash.

\end{list}


\subsection{Matching vs. Searching \label{matching-searching}}
\sectionauthor{Fred L. Drake, Jr.}{fdrake@acm.org}

\strong{XXX This section is still incomplete!}

Python offers two different primitive operations based on regular
expressions: match and search.  If you are accustomed to Perl's
semantics, the search operation is what you're looking for.  See the
\function{search()} function and corresponding method of compiled
regular expression objects.

Note that match may differ from search using a regular expression
beginning with \character{\^}:  \character{\^} matches only at the start
of the string, or in \constant{MULTILINE} mode also immediately
following a newline.  "match" succeeds only if the pattern matches at
the start of the string regardless of mode, or at the starting
position given by the optional \var{pos} argument regardless of
whether a newline precedes it.

% Examples from Tim Peters:
\begin{verbatim}
re.compile("a").match("ba", 1)           # succeeds
re.compile("^a").search("ba", 1)         # fails; 'a' not at start
re.compile("^a").search("\na", 1)        # fails; 'a' not at start
re.compile("^a", re.M).search("\na", 1)  # succeeds
re.compile("^a", re.M).search("ba", 1)   # fails; no preceding \n
\end{verbatim}


\subsection{Module Contents}
\nodename{Contents of Module re}

The module defines the following functions and constants, and an exception:


\begin{funcdesc}{compile}{pattern\optional{, flags}}
  Compile a regular expression pattern into a regular expression
  object, which can be used for matching using its \function{match()} and
  \function{search()} methods, described below.  

  The expression's behaviour can be modified by specifying a
  \var{flags} value.  Values can be any of the following variables,
  combined using bitwise OR (the \code{|} operator).

The sequence

\begin{verbatim}
prog = re.compile(pat)
result = prog.match(str)
\end{verbatim}

is equivalent to

\begin{verbatim}
result = re.match(pat, str)
\end{verbatim}

but the version using \function{compile()} is more efficient when the
expression will be used several times in a single program.
%(The compiled version of the last pattern passed to
%\function{regex.match()} or \function{regex.search()} is cached, so
%programs that use only a single regular expression at a time needn't
%worry about compiling regular expressions.)
\end{funcdesc}

\begin{datadesc}{I}
\dataline{IGNORECASE}
Perform case-insensitive matching; expressions like \regexp{[A-Z]} will match
lowercase letters, too.  This is not affected by the current locale.
\end{datadesc}

\begin{datadesc}{L}
\dataline{LOCALE}
Make \regexp{\e w}, \regexp{\e W}, \regexp{\e b},
\regexp{\e B}, dependent on the current locale. 
\end{datadesc}

\begin{datadesc}{M}
\dataline{MULTILINE}
When specified, the pattern character \character{\^} matches at the
beginning of the string and at the beginning of each line
(immediately following each newline); and the pattern character
\character{\$} matches at the end of the string and at the end of each line
(immediately preceding each newline).
By default, \character{\^} matches only at the beginning of the string, and
\character{\$} only at the end of the string and immediately before the
newline (if any) at the end of the string. 
\end{datadesc}

\begin{datadesc}{S}
\dataline{DOTALL}
Make the \character{.} special character match any character at all, including a
newline; without this flag, \character{.} will match anything \emph{except}
a newline.
\end{datadesc}

\begin{datadesc}{X}
\dataline{VERBOSE}
This flag allows you to write regular expressions that look nicer.
Whitespace within the pattern is ignored, 
except when in a character class or preceded by an unescaped
backslash, and, when a line contains a \character{\#} neither in a character
class or preceded by an unescaped backslash, all characters from the
leftmost such \character{\#} through the end of the line are ignored.
% XXX should add an example here
\end{datadesc}


\begin{funcdesc}{search}{pattern, string\optional{, flags}}
  Scan through \var{string} looking for a location where the regular
  expression \var{pattern} produces a match, and return a
  corresponding \class{MatchObject} instance.
  Return \code{None} if no
  position in the string matches the pattern; note that this is
  different from finding a zero-length match at some point in the string.
\end{funcdesc}

\begin{funcdesc}{match}{pattern, string\optional{, flags}}
  If zero or more characters at the beginning of \var{string} match
  the regular expression \var{pattern}, return a corresponding
  \class{MatchObject} instance.  Return \code{None} if the string does not
  match the pattern; note that this is different from a zero-length
  match.

  \strong{Note:}  If you want to locate a match anywhere in
  \var{string}, use \method{search()} instead.
\end{funcdesc}

\begin{funcdesc}{split}{pattern, string, \optional{, maxsplit\code{ = 0}}}
  Split \var{string} by the occurrences of \var{pattern}.  If
  capturing parentheses are used in \var{pattern}, then the text of all
  groups in the pattern are also returned as part of the resulting list.
  If \var{maxsplit} is nonzero, at most \var{maxsplit} splits
  occur, and the remainder of the string is returned as the final
  element of the list.  (Incompatibility note: in the original Python
  1.5 release, \var{maxsplit} was ignored.  This has been fixed in
  later releases.)

\begin{verbatim}
>>> re.split('\W+', 'Words, words, words.')
['Words', 'words', 'words', '']
>>> re.split('(\W+)', 'Words, words, words.')
['Words', ', ', 'words', ', ', 'words', '.', '']
>>> re.split('\W+', 'Words, words, words.', 1)
['Words', 'words, words.']
\end{verbatim}

  This function combines and extends the functionality of
  the old \function{regsub.split()} and \function{regsub.splitx()}.
\end{funcdesc}

\begin{funcdesc}{findall}{pattern, string}
\versionadded{1.5.2}
Return a list of all non-overlapping matches of \var{pattern} in
\var{string}.  If one or more groups are present in the pattern,
return a list of groups; this will be a list of tuples if the pattern
has more than one group.  Empty matches are included in the result.
\end{funcdesc}

\begin{funcdesc}{sub}{pattern, repl, string\optional{, count\code{ = 0}}}
Return the string obtained by replacing the leftmost non-overlapping
occurrences of \var{pattern} in \var{string} by the replacement
\var{repl}.  If the pattern isn't found, \var{string} is returned
unchanged.  \var{repl} can be a string or a function; if a function,
it is called for every non-overlapping occurance of \var{pattern}.
The function takes a single match object argument, and returns the
replacement string.  For example:

\begin{verbatim}
>>> def dashrepl(matchobj):
....    if matchobj.group(0) == '-': return ' '
....    else: return '-'
>>> re.sub('-{1,2}', dashrepl, 'pro----gram-files')
'pro--gram files'
\end{verbatim}

The pattern may be a string or a 
regex object; if you need to specify
regular expression flags, you must use a regex object, or use
embedded modifiers in a pattern; e.g.
\samp{sub("(?i)b+", "x", "bbbb BBBB")} returns \code{'x x'}.

The optional argument \var{count} is the maximum number of pattern
occurrences to be replaced; \var{count} must be a non-negative integer, and
the default value of 0 means to replace all occurrences.

Empty matches for the pattern are replaced only when not adjacent to a
previous match, so \samp{sub('x*', '-', 'abc')} returns \code{'-a-b-c-'}.

If \var{repl} is a string, any backslash escapes in it are processed.
That is, \samp{\e n} is converted to a single newline character,
\samp{\e r} is converted to a linefeed, and so forth.  Unknown escapes
such as \samp{\e j} are left alone.  Backreferences, such as \samp{\e 6}, are
replaced with the substring matched by group 6 in the pattern. 

In addition to character escapes and backreferences as described
above, \samp{\e g<name>} will use the substring matched by the group
named \samp{name}, as defined by the \regexp{(?P<name>...)} syntax.
\samp{\e g<number>} uses the corresponding group number; \samp{\e
g<2>} is therefore equivalent to \samp{\e 2}, but isn't ambiguous in a
replacement such as \samp{\e g<2>0}.  \samp{\e 20} would be
interpreted as a reference to group 20, not a reference to group 2
followed by the literal character \character{0}.  
\end{funcdesc}

\begin{funcdesc}{subn}{pattern, repl, string\optional{, count\code{ = 0}}}
Perform the same operation as \function{sub()}, but return a tuple
\code{(\var{new_string}, \var{number_of_subs_made})}.
\end{funcdesc}

\begin{funcdesc}{escape}{string}
  Return \var{string} with all non-alphanumerics backslashed; this is
  useful if you want to match an arbitrary literal string that may have
  regular expression metacharacters in it.
\end{funcdesc}

\begin{excdesc}{error}
  Exception raised when a string passed to one of the functions here
  is not a valid regular expression (e.g., unmatched parentheses) or
  when some other error occurs during compilation or matching.  It is
  never an error if a string contains no match for a pattern.
\end{excdesc}


\subsection{Regular Expression Objects \label{re-objects}}

Compiled regular expression objects support the following methods and
attributes:

\begin{methoddesc}[RegexObject]{search}{string\optional{, pos}\optional{,
                                        endpos}}
  Scan through \var{string} looking for a location where this regular
  expression produces a match, and return a
  corresponding \class{MatchObject} instance.  Return \code{None} if no
  position in the string matches the pattern; note that this is
  different from finding a zero-length match at some point in the string.
  
  The optional \var{pos} and \var{endpos} parameters have the same
  meaning as for the \method{match()} method.
\end{methoddesc}

\begin{methoddesc}[RegexObject]{match}{string\optional{, pos}\optional{,
                                       endpos}}
  If zero or more characters at the beginning of \var{string} match
  this regular expression, return a corresponding
  \class{MatchObject} instance.  Return \code{None} if the string does not
  match the pattern; note that this is different from a zero-length
  match.

  \strong{Note:}  If you want to locate a match anywhere in
  \var{string}, use \method{search()} instead.

  The optional second parameter \var{pos} gives an index in the string
  where the search is to start; it defaults to \code{0}.  This is not
  completely equivalent to slicing the string; the \code{'\^'} pattern
  character matches at the real beginning of the string and at positions
  just after a newline, but not necessarily at the index where the search
  is to start.

  The optional parameter \var{endpos} limits how far the string will
  be searched; it will be as if the string is \var{endpos} characters
  long, so only the characters from \var{pos} to \var{endpos} will be
  searched for a match.
\end{methoddesc}

\begin{methoddesc}[RegexObject]{split}{string, \optional{,
                                       maxsplit\code{ = 0}}}
Identical to the \function{split()} function, using the compiled pattern.
\end{methoddesc}

\begin{methoddesc}[RegexObject]{findall}{string}
Identical to the \function{findall()} function, using the compiled pattern.
\end{methoddesc}

\begin{methoddesc}[RegexObject]{sub}{repl, string\optional{, count\code{ = 0}}}
Identical to the \function{sub()} function, using the compiled pattern.
\end{methoddesc}

\begin{methoddesc}[RegexObject]{subn}{repl, string\optional{,
                                      count\code{ = 0}}}
Identical to the \function{subn()} function, using the compiled pattern.
\end{methoddesc}


\begin{memberdesc}[RegexObject]{flags}
The flags argument used when the regex object was compiled, or
\code{0} if no flags were provided.
\end{memberdesc}

\begin{memberdesc}[RegexObject]{groupindex}
A dictionary mapping any symbolic group names defined by 
\regexp{(?P<\var{id}>)} to group numbers.  The dictionary is empty if no
symbolic groups were used in the pattern.
\end{memberdesc}

\begin{memberdesc}[RegexObject]{pattern}
The pattern string from which the regex object was compiled.
\end{memberdesc}


\subsection{Match Objects \label{match-objects}}

\class{MatchObject} instances support the following methods and attributes:

\begin{methoddesc}[MatchObject]{group}{\optional{group1, group2, ...}}
Returns one or more subgroups of the match.  If there is a single
argument, the result is a single string; if there are
multiple arguments, the result is a tuple with one item per argument.
Without arguments, \var{group1} defaults to zero (i.e. the whole match
is returned).
If a \var{groupN} argument is zero, the corresponding return value is the
entire matching string; if it is in the inclusive range [1..99], it is
the string matching the the corresponding parenthesized group.  If a
group number is negative or larger than the number of groups defined
in the pattern, an \exception{IndexError} exception is raised.
If a group is contained in a part of the pattern that did not match,
the corresponding result is \code{None}.  If a group is contained in a 
part of the pattern that matched multiple times, the last match is
returned.

If the regular expression uses the \regexp{(?P<\var{name}>...)} syntax,
the \var{groupN} arguments may also be strings identifying groups by
their group name.  If a string argument is not used as a group name in 
the pattern, an \exception{IndexError} exception is raised.

A moderately complicated example:

\begin{verbatim}
m = re.match(r"(?P<int>\d+)\.(\d*)", '3.14')
\end{verbatim}

After performing this match, \code{m.group(1)} is \code{'3'}, as is
\code{m.group('int')}, and \code{m.group(2)} is \code{'14'}.
\end{methoddesc}

\begin{methoddesc}[MatchObject]{groups}{\optional{default}}
Return a tuple containing all the subgroups of the match, from 1 up to
however many groups are in the pattern.  The \var{default} argument is
used for groups that did not participate in the match; it defaults to
\code{None}.  (Incompatibility note: in the original Python 1.5
release, if the tuple was one element long, a string would be returned
instead.  In later versions (from 1.5.1 on), a singleton tuple is
returned in such cases.)
\end{methoddesc}

\begin{methoddesc}[MatchObject]{groupdict}{\optional{default}}
Return a dictionary containing all the \emph{named} subgroups of the
match, keyed by the subgroup name.  The \var{default} argument is
used for groups that did not participate in the match; it defaults to
\code{None}.
\end{methoddesc}

\begin{methoddesc}[MatchObject]{start}{\optional{group}}
\funcline{end}{\optional{group}}
Return the indices of the start and end of the substring
matched by \var{group}; \var{group} defaults to zero (meaning the whole
matched substring).
Return \code{None} if \var{group} exists but
did not contribute to the match.  For a match object
\var{m}, and a group \var{g} that did contribute to the match, the
substring matched by group \var{g} (equivalent to
\code{\var{m}.group(\var{g})}) is

\begin{verbatim}
m.string[m.start(g):m.end(g)]
\end{verbatim}

Note that
\code{m.start(\var{group})} will equal \code{m.end(\var{group})} if
\var{group} matched a null string.  For example, after \code{\var{m} =
re.search('b(c?)', 'cba')}, \code{\var{m}.start(0)} is 1,
\code{\var{m}.end(0)} is 2, \code{\var{m}.start(1)} and
\code{\var{m}.end(1)} are both 2, and \code{\var{m}.start(2)} raises
an \exception{IndexError} exception.
\end{methoddesc}

\begin{methoddesc}[MatchObject]{span}{\optional{group}}
For \class{MatchObject} \var{m}, return the 2-tuple
\code{(\var{m}.start(\var{group}), \var{m}.end(\var{group}))}.
Note that if \var{group} did not contribute to the match, this is
\code{(None, None)}.  Again, \var{group} defaults to zero.
\end{methoddesc}

\begin{memberdesc}[MatchObject]{pos}
The value of \var{pos} which was passed to the
\function{search()} or \function{match()} function.  This is the index into
the string at which the regex engine started looking for a match. 
\end{memberdesc}

\begin{memberdesc}[MatchObject]{endpos}
The value of \var{endpos} which was passed to the
\function{search()} or \function{match()} function.  This is the index into
the string beyond which the regex engine will not go.
\end{memberdesc}

\begin{memberdesc}[MatchObject]{re}
The regular expression object whose \method{match()} or
\method{search()} method produced this \class{MatchObject} instance.
\end{memberdesc}

\begin{memberdesc}[MatchObject]{string}
The string passed to \function{match()} or \function{search()}.
\end{memberdesc}

\begin{seealso}
\seetext{Jeffrey Friedl, \emph{Mastering Regular Expressions},
O'Reilly.  The Python material in this book dates from before the
\module{re} module, but it covers writing good regular expression
patterns in great detail.}
\end{seealso}


\section{Built-in Module \sectcode{regex}}
\label{module-regex}

\bimodindex{regex}
This module provides regular expression matching operations similar to
those found in Emacs.

\strong{Obsolescence note:}
This module is obsolete as of Python version 1.5; it is still being
maintained because much existing code still uses it.  All new code in
need of regular expressions should use the new \code{re} module, which
supports the more powerful and regular Perl-style regular expressions.
Existing code should be converted.  The standard library module
\code{reconvert} helps in converting \code{regex} style regular
expressions to \code{re} style regular expressions.  (For more
conversion help, see the URL
\file{http://starship.skyport.net/crew/amk/regex/regex-to-re.html}.)

By default the patterns are Emacs-style regular expressions
(with one exception).  There is
a way to change the syntax to match that of several well-known
\UNIX{} utilities.  The exception is that Emacs' \samp{\e s}
pattern is not supported, since the original implementation references
the Emacs syntax tables.

This module is 8-bit clean: both patterns and strings may contain null
bytes and characters whose high bit is set.

\strong{Please note:} There is a little-known fact about Python string
literals which means that you don't usually have to worry about
doubling backslashes, even though they are used to escape special
characters in string literals as well as in regular expressions.  This
is because Python doesn't remove backslashes from string literals if
they are followed by an unrecognized escape character.
\emph{However}, if you want to include a literal \dfn{backslash} in a
regular expression represented as a string literal, you have to
\emph{quadruple} it or enclose it in a singleton character class.
E.g.\  to extract \LaTeX\ \samp{\e section\{{\rm
\ldots}\}} headers from a document, you can use this pattern:
\code{'[\e ]section\{\e (.*\e )\}'}.  \emph{Another exception:}
the escape sequece \samp{\e b} is significant in string literals
(where it means the ASCII bell character) as well as in Emacs regular
expressions (where it stands for a word boundary), so in order to
search for a word boundary, you should use the pattern \code{'\e \e b'}.
Similarly, a backslash followed by a digit 0-7 should be doubled to
avoid interpretation as an octal escape.

\subsection{Regular Expressions}

A regular expression (or RE) specifies a set of strings that matches
it; the functions in this module let you check if a particular string
matches a given regular expression (or if a given regular expression
matches a particular string, which comes down to the same thing).

Regular expressions can be concatenated to form new regular
expressions; if \emph{A} and \emph{B} are both regular expressions,
then \emph{AB} is also an regular expression.  If a string \emph{p}
matches A and another string \emph{q} matches B, the string \emph{pq}
will match AB.  Thus, complex expressions can easily be constructed
from simpler ones like the primitives described here.  For details of
the theory and implementation of regular expressions, consult almost
any textbook about compiler construction.

% XXX The reference could be made more specific, say to 
% "Compilers: Principles, Techniques and Tools", by Alfred V. Aho, 
% Ravi Sethi, and Jeffrey D. Ullman, or some FA text.   

A brief explanation of the format of regular expressions follows.

Regular expressions can contain both special and ordinary characters.
Ordinary characters, like '\code{A}', '\code{a}', or '\code{0}', are
the simplest regular expressions; they simply match themselves.  You
can concatenate ordinary characters, so '\code{last}' matches the
characters 'last'.  (In the rest of this section, we'll write RE's in
\code{this special font}, usually without quotes, and strings to be
matched 'in single quotes'.)

Special characters either stand for classes of ordinary characters, or
affect how the regular expressions around them are interpreted.

The special characters are:
\begin{itemize}
\item[\code{.}] (Dot.)  Matches any character except a newline.
\item[\code{\^}] (Caret.)  Matches the start of the string.
\item[\code{\$}] Matches the end of the string.  
\code{foo} matches both 'foo' and 'foobar', while the regular
expression '\code{foo\$}' matches only 'foo'.
\item[\code{*}] Causes the resulting RE to
match 0 or more repetitions of the preceding RE.  \code{ab*} will
match 'a', 'ab', or 'a' followed by any number of 'b's.
\item[\code{+}] Causes the
resulting RE to match 1 or more repetitions of the preceding RE.
\code{ab+} will match 'a' followed by any non-zero number of 'b's; it
will not match just 'a'.
\item[\code{?}] Causes the resulting RE to
match 0 or 1 repetitions of the preceding RE.  \code{ab?} will
match either 'a' or 'ab'.

\item[\code{\e}] Either escapes special characters (permitting you to match
characters like '*?+\&\$'), or signals a special sequence; special
sequences are discussed below.  Remember that Python also uses the
backslash as an escape sequence in string literals; if the escape
sequence isn't recognized by Python's parser, the backslash and
subsequent character are included in the resulting string.  However,
if Python would recognize the resulting sequence, the backslash should
be repeated twice.  

\item[\code{[]}] Used to indicate a set of characters.  Characters can
be listed individually, or a range is indicated by giving two
characters and separating them by a '-'.  Special characters are
not active inside sets.  For example, \code{[akm\$]}
will match any of the characters 'a', 'k', 'm', or '\$'; \code{[a-z]} will
match any lowercase letter.  

If you want to include a \code{]} inside a
set, it must be the first character of the set; to include a \code{-},
place it as the first or last character. 

Characters \emph{not} within a range can be matched by including a
\code{\^} as the first character of the set; \code{\^} elsewhere will
simply match the '\code{\^}' character.  
\end{itemize}

The special sequences consist of '\code{\e}' and a character
from the list below.  If the ordinary character is not on the list,
then the resulting RE will match the second character.  For example,
\code{\e\$} matches the character '\$'.  Ones where the backslash
should be doubled in string literals are indicated.

\begin{itemize}
\item[\code{\e|}]\code{A\e|B}, where A and B can be arbitrary REs,
creates a regular expression that will match either A or B.  This can
be used inside groups (see below) as well.
%
\item[\code{\e( \e)}] Indicates the start and end of a group; the
contents of a group can be matched later in the string with the
\code{\e [1-9]} special sequence, described next.
%
{\fulllineitems\item[\code{\e \e 1, ... \e \e 7, \e 8, \e 9}]
Matches the contents of the group of the same
number.  For example, \code{\e (.+\e ) \e \e 1} matches 'the the' or
'55 55', but not 'the end' (note the space after the group).  This
special sequence can only be used to match one of the first 9 groups;
groups with higher numbers can be matched using the \code{\e v}
sequence.  (\code{\e 8} and \code{\e 9} don't need a double backslash
because they are not octal digits.)}
%
\item[\code{\e \e b}] Matches the empty string, but only at the
beginning or end of a word.  A word is defined as a sequence of
alphanumeric characters, so the end of a word is indicated by
whitespace or a non-alphanumeric character.
%
\item[\code{\e B}] Matches the empty string, but when it is \emph{not} at the
beginning or end of a word.
%
\item[\code{\e v}] Must be followed by a two digit decimal number, and
matches the contents of the group of the same number.  The group number must be between 1 and 99, inclusive.
%
\item[\code{\e w}]Matches any alphanumeric character; this is
equivalent to the set \code{[a-zA-Z0-9]}.
%
\item[\code{\e W}] Matches any non-alphanumeric character; this is
equivalent to the set \code{[\^a-zA-Z0-9]}.
\item[\code{\e <}] Matches the empty string, but only at the beginning of a
word.  A word is defined as a sequence of alphanumeric characters, so
the end of a word is indicated by whitespace or a non-alphanumeric 
character.
\item[\code{\e >}] Matches the empty string, but only at the end of a
word.

\item[\code{\e \e \e \e}] Matches a literal backslash.

% In Emacs, the following two are start of buffer/end of buffer.  In
% Python they seem to be synonyms for ^$.
\item[\code{\e `}] Like \code{\^}, this only matches at the start of the
string.
\item[\code{\e \e '}] Like \code{\$}, this only matches at the end of the
string.
% end of buffer
\end{itemize}

\subsection{Module Contents}

The module defines these functions, and an exception:

\renewcommand{\indexsubitem}{(in module regex)}

\begin{funcdesc}{match}{pattern\, string}
  Return how many characters at the beginning of \var{string} match
  the regular expression \var{pattern}.  Return \code{-1} if the
  string does not match the pattern (this is different from a
  zero-length match!).
\end{funcdesc}

\begin{funcdesc}{search}{pattern\, string}
  Return the first position in \var{string} that matches the regular
  expression \var{pattern}.  Return \code{-1} if no position in the string
  matches the pattern (this is different from a zero-length match
  anywhere!).
\end{funcdesc}

\begin{funcdesc}{compile}{pattern\optional{\, translate}}
  Compile a regular expression pattern into a regular expression
  object, which can be used for matching using its \code{match} and
  \code{search} methods, described below.  The optional argument
  \var{translate}, if present, must be a 256-character string
  indicating how characters (both of the pattern and of the strings to
  be matched) are translated before comparing them; the \code{i}-th
  element of the string gives the translation for the character with
  \ASCII{} code \code{i}.  This can be used to implement
  case-insensitive matching; see the \code{casefold} data item below.

  The sequence

\bcode\begin{verbatim}
prog = regex.compile(pat)
result = prog.match(str)
\end{verbatim}\ecode
%
is equivalent to

\bcode\begin{verbatim}
result = regex.match(pat, str)
\end{verbatim}\ecode
%
but the version using \code{compile()} is more efficient when multiple
regular expressions are used concurrently in a single program.  (The
compiled version of the last pattern passed to \code{regex.match()} or
\code{regex.search()} is cached, so programs that use only a single
regular expression at a time needn't worry about compiling regular
expressions.)
\end{funcdesc}

\begin{funcdesc}{set_syntax}{flags}
  Set the syntax to be used by future calls to \code{compile},
  \code{match} and \code{search}.  (Already compiled expression objects
  are not affected.)  The argument is an integer which is the OR of
  several flag bits.  The return value is the previous value of
  the syntax flags.  Names for the flags are defined in the standard
  module \code{regex_syntax}; read the file \file{regex_syntax.py} for
  more information.
\end{funcdesc}

\begin{funcdesc}{get_syntax}{}
  Returns the current value of the syntax flags as an integer.
\end{funcdesc}

\begin{funcdesc}{symcomp}{pattern\optional{\, translate}}
This is like \code{compile}, but supports symbolic group names: if a
parenthesis-enclosed group begins with a group name in angular
brackets, e.g. \code{'\e(<id>[a-z][a-z0-9]*\e)'}, the group can
be referenced by its name in arguments to the \code{group} method of
the resulting compiled regular expression object, like this:
\code{p.group('id')}.  Group names may contain alphanumeric characters
and \code{'_'} only.
\end{funcdesc}

\begin{excdesc}{error}
  Exception raised when a string passed to one of the functions here
  is not a valid regular expression (e.g., unmatched parentheses) or
  when some other error occurs during compilation or matching.  (It is
  never an error if a string contains no match for a pattern.)
\end{excdesc}

\begin{datadesc}{casefold}
A string suitable to pass as \var{translate} argument to
\code{compile} to map all upper case characters to their lowercase
equivalents.
\end{datadesc}

\noindent
Compiled regular expression objects support these methods:

\renewcommand{\indexsubitem}{(regex method)}
\begin{funcdesc}{match}{string\optional{\, pos}}
  Return how many characters at the beginning of \var{string} match
  the compiled regular expression.  Return \code{-1} if the string
  does not match the pattern (this is different from a zero-length
  match!).
  
  The optional second parameter \var{pos} gives an index in the string
  where the search is to start; it defaults to \code{0}.  This is not
  completely equivalent to slicing the string; the \code{'\^'} pattern
  character matches at the real begin of the string and at positions
  just after a newline, not necessarily at the index where the search
  is to start.
\end{funcdesc}

\begin{funcdesc}{search}{string\optional{\, pos}}
  Return the first position in \var{string} that matches the regular
  expression \code{pattern}.  Return \code{-1} if no position in the
  string matches the pattern (this is different from a zero-length
  match anywhere!).
  
  The optional second parameter has the same meaning as for the
  \code{match} method.
\end{funcdesc}

\begin{funcdesc}{group}{index\, index\, ...}
This method is only valid when the last call to the \code{match}
or \code{search} method found a match.  It returns one or more
groups of the match.  If there is a single \var{index} argument,
the result is a single string; if there are multiple arguments, the
result is a tuple with one item per argument.  If the \var{index} is
zero, the corresponding return value is the entire matching string; if
it is in the inclusive range [1..99], it is the string matching the
the corresponding parenthesized group (using the default syntax,
groups are parenthesized using \code{{\e}(} and \code{{\e})}).  If no
such group exists, the corresponding result is \code{None}.

If the regular expression was compiled by \code{symcomp} instead of
\code{compile}, the \var{index} arguments may also be strings
identifying groups by their group name.
\end{funcdesc}

\noindent
Compiled regular expressions support these data attributes:

\renewcommand{\indexsubitem}{(regex attribute)}

\begin{datadesc}{regs}
When the last call to the \code{match} or \code{search} method found a
match, this is a tuple of pairs of indices corresponding to the
beginning and end of all parenthesized groups in the pattern.  Indices
are relative to the string argument passed to \code{match} or
\code{search}.  The 0-th tuple gives the beginning and end or the
whole pattern.  When the last match or search failed, this is
\code{None}.
\end{datadesc}

\begin{datadesc}{last}
When the last call to the \code{match} or \code{search} method found a
match, this is the string argument passed to that method.  When the
last match or search failed, this is \code{None}.
\end{datadesc}

\begin{datadesc}{translate}
This is the value of the \var{translate} argument to
\code{regex.compile} that created this regular expression object.  If
the \var{translate} argument was omitted in the \code{regex.compile}
call, this is \code{None}.
\end{datadesc}

\begin{datadesc}{givenpat}
The regular expression pattern as passed to \code{compile} or
\code{symcomp}.
\end{datadesc}

\begin{datadesc}{realpat}
The regular expression after stripping the group names for regular
expressions compiled with \code{symcomp}.  Same as \code{givenpat}
otherwise.
\end{datadesc}

\begin{datadesc}{groupindex}
A dictionary giving the mapping from symbolic group names to numerical
group indices for regular expressions compiled with \code{symcomp}.
\code{None} otherwise.
\end{datadesc}

\section{\module{regsub} ---
         String operations using regular expressions}

\declaremodule{standard}{regsub}
\modulesynopsis{Substitution and splitting operations that use
                regular expressions.  \strong{Obsolete!}}


This module defines a number of functions useful for working with
regular expressions (see built-in module \refmodule{regex}).

Warning: these functions are not thread-safe.

\strong{Obsolescence note:}
This module is obsolete as of Python version 1.5; it is still being
maintained because much existing code still uses it.  All new code in
need of regular expressions should use the new \refmodule{re} module, which
supports the more powerful and regular Perl-style regular expressions.
Existing code should be converted.  The standard library module
\module{reconvert} helps in converting \refmodule{regex} style regular
expressions to \refmodule{re} style regular expressions.  (For more
conversion help, see Andrew Kuchling's\index{Kuchling, Andrew}
``regex-to-re HOWTO'' at
\url{http://www.python.org/doc/howto/regex-to-re/}.)


\begin{funcdesc}{sub}{pat, repl, str}
Replace the first occurrence of pattern \var{pat} in string
\var{str} by replacement \var{repl}.  If the pattern isn't found,
the string is returned unchanged.  The pattern may be a string or an
already compiled pattern.  The replacement may contain references
\samp{\e \var{digit}} to subpatterns and escaped backslashes.
\end{funcdesc}

\begin{funcdesc}{gsub}{pat, repl, str}
Replace all (non-overlapping) occurrences of pattern \var{pat} in
string \var{str} by replacement \var{repl}.  The same rules as for
\code{sub()} apply.  Empty matches for the pattern are replaced only
when not adjacent to a previous match, so e.g.
\code{gsub('', '-', 'abc')} returns \code{'-a-b-c-'}.
\end{funcdesc}

\begin{funcdesc}{split}{str, pat\optional{, maxsplit}}
Split the string \var{str} in fields separated by delimiters matching
the pattern \var{pat}, and return a list containing the fields.  Only
non-empty matches for the pattern are considered, so e.g.
\code{split('a:b', ':*')} returns \code{['a', 'b']} and
\code{split('abc', '')} returns \code{['abc']}.  The \var{maxsplit}
defaults to 0. If it is nonzero, only \var{maxsplit} number of splits
occur, and the remainder of the string is returned as the final
element of the list.
\end{funcdesc}

\begin{funcdesc}{splitx}{str, pat\optional{, maxsplit}}
Split the string \var{str} in fields separated by delimiters matching
the pattern \var{pat}, and return a list containing the fields as well
as the separators.  For example, \code{splitx('a:::b', ':*')} returns
\code{['a', ':::', 'b']}.  Otherwise, this function behaves the same
as \code{split}.
\end{funcdesc}

\begin{funcdesc}{capwords}{s\optional{, pat}}
Capitalize words separated by optional pattern \var{pat}.  The default
pattern uses any characters except letters, digits and underscores as
word delimiters.  Capitalization is done by changing the first
character of each word to upper case.
\end{funcdesc}

\begin{funcdesc}{clear_cache}{}
The regsub module maintains a cache of compiled regular expressions,
keyed on the regular expression string and the syntax of the regex
module at the time the expression was compiled.  This function clears
that cache.
\end{funcdesc}

\section{\module{struct} ---
         Interpret strings as packed binary data}
\declaremodule{builtin}{struct}

\modulesynopsis{Interpret strings as packed binary data.}

\indexii{C}{structures}
\indexiii{packing}{binary}{data}

This module performs conversions between Python values and C
structs represented as Python strings.  It uses \dfn{format strings}
(explained below) as compact descriptions of the lay-out of the C
structs and the intended conversion to/from Python values.  This can
be used in handling binary data stored in files or from network
connections, among other sources.

The module defines the following exception and functions:


\begin{excdesc}{error}
  Exception raised on various occasions; argument is a string
  describing what is wrong.
\end{excdesc}

\begin{funcdesc}{pack}{fmt, v1, v2, \textrm{\ldots}}
  Return a string containing the values
  \code{\var{v1}, \var{v2}, \textrm{\ldots}} packed according to the given
  format.  The arguments must match the values required by the format
  exactly.
\end{funcdesc}

\begin{funcdesc}{unpack}{fmt, string}
  Unpack the string (presumably packed by \code{pack(\var{fmt},
  \textrm{\ldots})}) according to the given format.  The result is a
  tuple even if it contains exactly one item.  The string must contain
  exactly the amount of data required by the format
  (\code{len(\var{string})} must equal \code{calcsize(\var{fmt})}).
\end{funcdesc}

\begin{funcdesc}{calcsize}{fmt}
  Return the size of the struct (and hence of the string)
  corresponding to the given format.
\end{funcdesc}

Format characters have the following meaning; the conversion between
C and Python values should be obvious given their types:

\begin{tableiv}{c|l|l|c}{samp}{Format}{C Type}{Python}{Notes}
  \lineiv{x}{pad byte}{no value}{}
  \lineiv{c}{\ctype{char}}{string of length 1}{}
  \lineiv{b}{\ctype{signed char}}{integer}{}
  \lineiv{B}{\ctype{unsigned char}}{integer}{}
  \lineiv{h}{\ctype{short}}{integer}{}
  \lineiv{H}{\ctype{unsigned short}}{integer}{}
  \lineiv{i}{\ctype{int}}{integer}{}
  \lineiv{I}{\ctype{unsigned int}}{long}{}
  \lineiv{l}{\ctype{long}}{integer}{}
  \lineiv{L}{\ctype{unsigned long}}{long}{}
  \lineiv{q}{\ctype{long long}}{long}{(1)}
  \lineiv{Q}{\ctype{unsigned long long}}{long}{(1)}
  \lineiv{f}{\ctype{float}}{float}{}
  \lineiv{d}{\ctype{double}}{float}{}
  \lineiv{s}{\ctype{char[]}}{string}{}
  \lineiv{p}{\ctype{char[]}}{string}{}
  \lineiv{P}{\ctype{void *}}{integer}{}
\end{tableiv}

\noindent
Notes:

\begin{description}
\item[(1)]
  The \character{q} and \character{Q} conversion codes are available in
  native mode only if the platform C compiler supports C \ctype{long long},
  or, on Windows, \ctype{__int64}.  They are always available in standard
  modes.
  \versionadded{2.2}
\end{description}


A format character may be preceded by an integral repeat count.  For
example, the format string \code{'4h'} means exactly the same as
\code{'hhhh'}.

Whitespace characters between formats are ignored; a count and its
format must not contain whitespace though.

For the \character{s} format character, the count is interpreted as the
size of the string, not a repeat count like for the other format
characters; for example, \code{'10s'} means a single 10-byte string, while
\code{'10c'} means 10 characters.  For packing, the string is
truncated or padded with null bytes as appropriate to make it fit.
For unpacking, the resulting string always has exactly the specified
number of bytes.  As a special case, \code{'0s'} means a single, empty
string (while \code{'0c'} means 0 characters).

The \character{p} format character can be used to encode a Pascal
string.  The first byte is the length of the stored string, with the
bytes of the string following.  If count is given, it is used as the
total number of bytes used, including the length byte.  If the string
passed in to \function{pack()} is too long, the stored representation
is truncated.  If the string is too short, padding is used to ensure
that exactly enough bytes are used to satisfy the count.

For the \character{I}, \character{L}, \character{q} and \character{Q}
format characters, the return value is a Python long integer.

For the \character{P} format character, the return value is a Python
integer or long integer, depending on the size needed to hold a
pointer when it has been cast to an integer type.  A \NULL{} pointer will
always be returned as the Python integer \code{0}. When packing pointer-sized
values, Python integer or long integer objects may be used.  For
example, the Alpha and Merced processors use 64-bit pointer values,
meaning a Python long integer will be used to hold the pointer; other
platforms use 32-bit pointers and will use a Python integer.

By default, C numbers are represented in the machine's native format
and byte order, and properly aligned by skipping pad bytes if
necessary (according to the rules used by the C compiler).

Alternatively, the first character of the format string can be used to
indicate the byte order, size and alignment of the packed data,
according to the following table:

\begin{tableiii}{c|l|l}{samp}{Character}{Byte order}{Size and alignment}
  \lineiii{@}{native}{native}
  \lineiii{=}{native}{standard}
  \lineiii{<}{little-endian}{standard}
  \lineiii{>}{big-endian}{standard}
  \lineiii{!}{network (= big-endian)}{standard}
\end{tableiii}

If the first character is not one of these, \character{@} is assumed.

Native byte order is big-endian or little-endian, depending on the
host system.  For example, Motorola and Sun processors are big-endian;
Intel and DEC processors are little-endian.

Native size and alignment are determined using the C compiler's
\keyword{sizeof} expression.  This is always combined with native byte
order.

Standard size and alignment are as follows: no alignment is required
for any type (so you have to use pad bytes);
\ctype{short} is 2 bytes;
\ctype{int} and \ctype{long} are 4 bytes;
\ctype{long long} (\ctype{__int64} on Windows) is 8 bytes;
\ctype{float} and \ctype{double} are 32-bit and 64-bit
IEEE floating point numbers, respectively.

Note the difference between \character{@} and \character{=}: both use
native byte order, but the size and alignment of the latter is
standardized.

The form \character{!} is available for those poor souls who claim they
can't remember whether network byte order is big-endian or
little-endian.

There is no way to indicate non-native byte order (force
byte-swapping); use the appropriate choice of \character{<} or
\character{>}.

The \character{P} format character is only available for the native
byte ordering (selected as the default or with the \character{@} byte
order character). The byte order character \character{=} chooses to
use little- or big-endian ordering based on the host system. The
struct module does not interpret this as native ordering, so the
\character{P} format is not available.

Examples (all using native byte order, size and alignment, on a
big-endian machine):

\begin{verbatim}
>>> from struct import *
>>> pack('hhl', 1, 2, 3)
'\x00\x01\x00\x02\x00\x00\x00\x03'
>>> unpack('hhl', '\x00\x01\x00\x02\x00\x00\x00\x03')
(1, 2, 3)
>>> calcsize('hhl')
8
\end{verbatim}

Hint: to align the end of a structure to the alignment requirement of
a particular type, end the format with the code for that type with a
repeat count of zero.  For example, the format \code{'llh0l'}
specifies two pad bytes at the end, assuming longs are aligned on
4-byte boundaries.  This only works when native size and alignment are
in effect; standard size and alignment does not enforce any alignment.

\begin{seealso}
  \seemodule{array}{Packed binary storage of homogeneous data.}
  \seemodule{xdrlib}{Packing and unpacking of XDR data.}
\end{seealso}

\section{\module{StringIO} ---
         Read and write strings as if they were files.}
\declaremodule{standard}{StringIO}


\modulesynopsis{Read and write strings as if they were files.}


This module implements a file-like class, \class{StringIO},
that reads and writes a string buffer (also known as \emph{memory
files}). See the description on file objects for operations.

\begin{classdesc}{StringIO}{\optional{buffer}}
When a \class{StringIO} object is created, it can be initialized
to an existing string by passing the string to the constructor.
If no string is given, the \class{StringIO} will start empty.
\end{classdesc}

The following methods of \class{StringIO} objects require special
mention:

\begin{methoddesc}{getvalue}{}
Retrieve the entire contents of the ``file'' at any time before the
\class{StringIO} object's \method{close()} method is called.
\end{methoddesc}

\begin{methoddesc}{close}{}
Free the memory buffer.
\end{methoddesc}


\section{\module{cStringIO} ---
         Faster version of \module{StringIO}, but not subclassable.}

\declaremodule{builtin}{cStringIO}
\modulesynopsis{Faster version of \module{StringIO}, but not subclassable.}
\sectionauthor{Fred L. Drake, Jr.}{fdrake@acm.org}

The module \module{cStringIO} provides an interface similar to that of
the \module{StringIO} module.  Heavy use of \class{StringIO.StringIO}
objects can be made more efficient by using the function
\function{StringIO()} from this module instead.

Since this module provides a factory function which returns objects of
built-in types, there's no way to build your own version using
subclassing.  Use the original \module{StringIO} module in that case.

%\section{Built-in Module \sectcode{soundex}}
\label{module-soundex}
\bimodindex{soundex}

\setindexsubitem{(in module soundex)}
The soundex algorithm takes an English word, and returns an
easily-computed hash of it; this hash is intended to be the same for
words that sound alike.  This module provides an interface to the
soundex algorithm.

Note that the soundex algorithm is quite simple-minded, and isn't
perfect by any measure.  Its main purpose is to help looking up names
in databases, when the name may be misspelled --- soundex hashes common
misspellings together.

\begin{funcdesc}{get_soundex}{string}
Return the soundex hash value for a word; it will always be a
6-character string.  \var{string} must contain the word to be hashed,
with no leading whitespace; the case of the word is ignored.
\end{funcdesc}

\begin{funcdesc}{sound_similar}{string1, string2}
Compare the word in \var{string1} with the word in \var{string2}; this
is equivalent to 
\code{get_soundex(\var{string1})} \code{==}
\code{get_soundex(\var{string2})}.
\end{funcdesc}


\chapter{Miscellaneous Services}
\label{misc}

The modules described in this chapter provide miscellaneous services
that are available in all Python versions.  Here's an overview:

\begin{description}

\item[math]
--- Mathematical functions (\function{sin()} etc.).

\item[cmath]
--- Mathematical functions for complex numbers.

\item[whrandom]
--- Floating point pseudo-random number generator.

\item[random]
--- Generate pseudo-random numbers with various common distributions.

\item[array]
--- Efficient arrays of uniformly typed numeric values.

\item[fileinput]
--- Perl-like iteration over lines from multiple input streams, with
``save in place'' capability.

\end{description}
			% Miscellaneous Services
\section{Built-in Module \sectcode{math}}
\label{module-math}

\bimodindex{math}
\renewcommand{\indexsubitem}{(in module math)}
This module is always available.
It provides access to the mathematical functions defined by the C
standard.
They are:

\begin{funcdesc}{acos}{x}
Return the arc cosine of \var{x}.
\end{funcdesc}

\begin{funcdesc}{asin}{x}
Return the arc sine of \var{x}.
\end{funcdesc}

\begin{funcdesc}{atan}{x}
Return the arc tangent of \var{x}.
\end{funcdesc}

\begin{funcdesc}{atan2}{x, y}
Return \code{atan(x / y)}.
\end{funcdesc}

\begin{funcdesc}{ceil}{x}
Return the ceiling of \var{x}.
\end{funcdesc}

\begin{funcdesc}{cos}{x}
Return the cosine of \var{x}.
\end{funcdesc}

\begin{funcdesc}{cosh}{x}
Return the hyperbolic cosine of \var{x}.
\end{funcdesc}

\begin{funcdesc}{exp}{x}
Return the exponential value $\mbox{e}^x$.
\end{funcdesc}

\begin{funcdesc}{fabs}{x}
Return the absolute value of the real \var{x}.
\end{funcdesc}

\begin{funcdesc}{floor}{x}
Return the floor of \var{x}.
\end{funcdesc}

\begin{funcdesc}{fmod}{x, y}
Return \code{x \% y}.
\end{funcdesc}

\begin{funcdesc}{frexp}{x}
Return the matissa and exponent for \var{x}.  The mantissa is
positive.
\end{funcdesc}

\begin{funcdesc}{hypot}{x, y}
Return the Euclidean distance, \code{sqrt(x*x + y*y)}.
\end{funcdesc}

\begin{funcdesc}{ldexp}{x, i}
Return $x {\times} 2^i$.
\end{funcdesc}

\begin{funcdesc}{modf}{x}
Return the fractional and integer parts of \var{x}.  Both results
carry the sign of \var{x}.
\end{funcdesc}

\begin{funcdesc}{pow}{x, y}
Return $x^y$.
\end{funcdesc}

\begin{funcdesc}{sin}{x}
Return the sine of \var{x}.
\end{funcdesc}

\begin{funcdesc}{sinh}{x}
Return the hyperbolic sine of \var{x}.
\end{funcdesc}

\begin{funcdesc}{sqrt}{x}
Return the square root of \var{x}.
\end{funcdesc}

\begin{funcdesc}{tan}{x}
Return the tangent of \var{x}.
\end{funcdesc}

\begin{funcdesc}{tanh}{x}
Return the hyperbolic tangent of \var{x}.
\end{funcdesc}

Note that \code{frexp} and \code{modf} have a different call/return
pattern than their C equivalents: they take a single argument and
return a pair of values, rather than returning their second return
value through an `output parameter' (there is no such thing in Python).

The module also defines two mathematical constants:

\begin{datadesc}{pi}
The mathematical constant \emph{pi}.
\end{datadesc}

\begin{datadesc}{e}
The mathematical constant \emph{e}.
\end{datadesc}

\begin{seealso}
  \seemodule{cmath}{Complex number versions of many of these functions.}
\end{seealso}

\section{Built-in Module \sectcode{cmath}}
\label{module-cmath}

\bimodindex{cmath}
This module is always available.
It provides access to mathematical functions for complex numbers.
The functions are:

\begin{funcdesc}{acos}{x}
Return the arc cosine of \var{x}.
\end{funcdesc}

\begin{funcdesc}{acosh}{x}
Return the hyperbolic arc cosine of \var{x}.
\end{funcdesc}

\begin{funcdesc}{asin}{x}
Return the arc sine of \var{x}.
\end{funcdesc}

\begin{funcdesc}{asinh}{x}
Return the hyperbolic arc sine of \var{x}.
\end{funcdesc}

\begin{funcdesc}{atan}{x}
Return the arc tangent of \var{x}.
\end{funcdesc}

\begin{funcdesc}{atanh}{x}
Return the hyperbolic arc tangent of \var{x}.
\end{funcdesc}

\begin{funcdesc}{cos}{x}
Return the cosine of \var{x}.
\end{funcdesc}

\begin{funcdesc}{cosh}{x}
Return the hyperbolic cosine of \var{x}.
\end{funcdesc}

\begin{funcdesc}{exp}{x}
Return the exponential value \code{e**\var{x}}.
\end{funcdesc}

\begin{funcdesc}{log}{x}
Return the natural logarithm of \var{x}.
\end{funcdesc}

\begin{funcdesc}{log10}{x}
Return the base-10 logarithm of \var{x}.
\end{funcdesc}

\begin{funcdesc}{sin}{x}
Return the sine of \var{x}.
\end{funcdesc}

\begin{funcdesc}{sinh}{x}
Return the hyperbolic sine of \var{x}.
\end{funcdesc}

\begin{funcdesc}{sqrt}{x}
Return the square root of \var{x}.
\end{funcdesc}

\begin{funcdesc}{tan}{x}
Return the tangent of \var{x}.
\end{funcdesc}

\begin{funcdesc}{tanh}{x}
Return the hyperbolic tangent of \var{x}.
\end{funcdesc}

The module also defines two mathematical constants:

\begin{datadesc}{pi}
The mathematical constant \emph{pi}, as a real.
\end{datadesc}

\begin{datadesc}{e}
The mathematical constant \emph{e}, as a real.
\end{datadesc}

Note that the selection of functions is similar, but not identical, to
that in module \code{math}\refbimodindex{math}.  The reason for having
two modules is, that some users aren't interested in complex numbers,
and perhaps don't even know what they are.  They would rather have
\code{math.sqrt(-1)} raise an exception than return a complex number.
Also note that the functions defined in \code{cmath} always return a
complex number, even if the answer can be expressed as a real number
(in which case the complex number has an imaginary part of zero).

\section{\module{whrandom} ---
         Pseudo-random number generator}

\declaremodule{standard}{whrandom}
\modulesynopsis{Floating point pseudo-random number generator.}

\deprecated{2.1}{Use \refmodule{random} instead.}

\note{This module was an implementation detail of the
\refmodule{random} module in releases of Python prior to 2.1.  It is
no longer used.  Please do not use this module directly; use
\refmodule{random} instead.}

This module implements a Wichmann-Hill pseudo-random number generator
class that is also named \class{whrandom}.  Instances of the
\class{whrandom} class conform to the Random Number Generator
interface described in the docs for the \refmodule{random} module.
They also offer the 
following method, specific to the Wichmann-Hill algorithm:

\begin{methoddesc}[whrandom]{seed}{\optional{x, y, z}}
  Initializes the random number generator from the integers \var{x},
  \var{y} and \var{z}.  When the module is first imported, the random
  number is initialized using values derived from the current time.
  If \var{x}, \var{y}, and \var{z} are either omitted or \code{0}, the 
  seed will be computed from the current system time.  If one or two
  of the parameters are \code{0}, but not all three, the zero values
  are replaced by ones.  This causes some apparently different seeds
  to be equal, with the corresponding result on the pseudo-random
  series produced by the generator.
\end{methoddesc}

Other supported methods include:

\begin{funcdesc}{choice}{seq}
Chooses a random element from the non-empty sequence \var{seq} and returns it.
\end{funcdesc}

\begin{funcdesc}{randint}{a, b}
Returns a random integer \var{N} such that \code{\var{a}<=\var{N}<=\var{b}}.
\end{funcdesc}

\begin{funcdesc}{random}{}
Returns the next random floating point number in the range [0.0 ... 1.0).
\end{funcdesc}

\begin{funcdesc}{seed}{x, y, z}
Initializes the random number generator from the integers \var{x},
\var{y} and \var{z}.  When the module is first imported, the random
number is initialized using values derived from the current time.
\end{funcdesc}

\begin{funcdesc}{uniform}{a, b}
Returns a random real number \var{N} such that \code{\var{a}<=\var{N}<\var{b}}.
\end{funcdesc}

When imported, the \module{whrandom} module also creates an instance of
the \class{whrandom} class, and makes the methods of that instance
available at the module level.  Therefore one can write either 
\code{N = whrandom.random()} or:

\begin{verbatim}
generator = whrandom.whrandom()
N = generator.random()
\end{verbatim}

Note that using separate instances of the generator leads to
independent sequences of pseudo-random numbers.

\begin{seealso}
  \seemodule{random}{Generators for various random distributions and
                     documentation for the Random Number Generator
                     interface.}
  \seetext{Wichmann, B. A. \& Hill, I. D., ``Algorithm AS 183: 
           An efficient and portable pseudo-random number generator'',
           \citetitle{Applied Statistics} 31 (1982) 188-190.}
\end{seealso}

\section{Standard Module \sectcode{random}}
\label{module-random}
\stmodindex{random}

This module implements pseudo-random number generators for various
distributions: on the real line, there are functions to compute normal
or Gaussian, lognormal, negative exponential, gamma, and beta
distributions.  For generating distribution of angles, the circular
uniform and von Mises distributions are available.

The module exports the following functions, which are exactly
equivalent to those in the \code{whrandom} module: \code{choice},
\code{randint}, \code{random}, \code{uniform}.  See the documentation
for the \code{whrandom} module for these functions.

The following functions specific to the \code{random} module are also
defined, and all return real values.  Function parameters are named
after the corresponding variables in the distribution's equation, as
used in common mathematical practice; most of these equations can be
found in any statistics text.

\renewcommand{\indexsubitem}{(in module random)}
\begin{funcdesc}{betavariate}{alpha\, beta}
Beta distribution.  Conditions on the parameters are \code{alpha>-1}
and \code{beta>-1}.
Returned values will range between 0 and 1.
\end{funcdesc}

\begin{funcdesc}{cunifvariate}{mean\, arc}
Circular uniform distribution.  \var{mean} is the mean angle, and
\var{arc} is the range of the distribution, centered around the mean
angle.  Both values must be expressed in radians, and can range
between 0 and \code{pi}.  Returned values will range between
\code{mean - arc/2} and \code{mean + arc/2}.
\end{funcdesc}

\begin{funcdesc}{expovariate}{lambd}
Exponential distribution.  \var{lambd} is 1.0 divided by the desired mean.
(The parameter would be called ``lambda'', but that's also a reserved
word in Python.)  Returned values will range from 0 to positive infinity.
\end{funcdesc}

\begin{funcdesc}{gamma}{alpha\, beta}
Gamma distribution.  (\emph{Not} the gamma function!) 
Conditions on the parameters are \code{alpha>-1} and \code{beta>0}.
\end{funcdesc}

\begin{funcdesc}{gauss}{mu\, sigma}
Gaussian distribution.  \var{mu} is the mean, and \var{sigma} is the
standard deviation.  This is slightly faster than the
\code{normalvariate} function defined below.
\end{funcdesc}

\begin{funcdesc}{lognormvariate}{mu\, sigma}
Log normal distribution.  If you take the natural logarithm of this
distribution, you'll get a normal distribution with mean \var{mu} and
standard deviation \var{sigma}  \var{mu} can have any value, and \var{sigma}
must be greater than zero.  
\end{funcdesc}

\begin{funcdesc}{normalvariate}{mu\, sigma}
Normal distribution.  \var{mu} is the mean, and \var{sigma} is the
standard deviation.
\end{funcdesc}

\begin{funcdesc}{vonmisesvariate}{mu\, kappa}
\var{mu} is the mean angle, expressed in radians between 0 and pi,
and \var{kappa} is the concentration parameter, which must be greater
then or equal to zero.  If \var{kappa} is equal to zero, this
distribution reduces to a uniform random angle over the range 0 to
\code{2*pi}.
\end{funcdesc}


\begin{seealso}
\seemodule{whrandom}{the standard Python random number generator}
\end{seealso}

%\section{Standard Module \sectcode{rand}}
\label{module-rand}
\stmodindex{rand}

The \code{rand} module simulates the C library's \code{rand()}
interface, though the results aren't necessarily compatible with any
given library's implementation.  While still supported for
compatibility, the \code{rand} module is now considered obsolete; if
possible, use the \code{whrandom} module instead.

\begin{funcdesc}{choice}{seq}
Returns a random element from the sequence \var{seq}.
\end{funcdesc}

\begin{funcdesc}{rand}{}
Return a random integer between 0 and 32767, inclusive.
\end{funcdesc}

\begin{funcdesc}{srand}{seed}
Set a starting seed value for the random number generator; \var{seed}
can be an arbitrary integer. 
\end{funcdesc}

\begin{seealso}
\seemodule{whrandom}{the standard Python random number generator}
\end{seealso}


\section{\module{bisect} ---
         Array bisection algorithm}

\declaremodule{standard}{bisect}
\modulesynopsis{Array bisection algorithms for binary searching.}
\sectionauthor{Fred L. Drake, Jr.}{fdrake@acm.org}
% LaTeX produced by Fred L. Drake, Jr. <fdrake@acm.org>, with an
% example based on the PyModules FAQ entry by Aaron Watters
% <arw@pythonpros.com>.


This module provides support for maintaining a list in sorted order
without having to sort the list after each insertion.  For long lists
of items with expensive comparison operations, this can be an
improvement over the more common approach.  The module is called
\module{bisect} because it uses a basic bisection algorithm to do its
work.  The source code may be most useful as a working example of the
algorithm (the boundary conditions are already right!).

The following functions are provided:

\begin{funcdesc}{bisect_left}{list, item\optional{, lo\optional{, hi}}}
  Locate the proper insertion point for \var{item} in \var{list} to
  maintain sorted order.  The parameters \var{lo} and \var{hi} may be
  used to specify a subset of the list which should be considered; by
  default the entire list is used.  If \var{item} is already present
  in \var{list}, the insertion point will be before (to the left of)
  any existing entries.  The return value is suitable for use as the
  first parameter to \code{\var{list}.insert()}.  This assumes that
  \var{list} is already sorted.
\versionadded{2.1}
\end{funcdesc}

\begin{funcdesc}{bisect_right}{list, item\optional{, lo\optional{, hi}}}
  Similar to \function{bisect_left()}, but returns an insertion point
  which comes after (to the right of) any existing entries of
  \var{item} in \var{list}.
\versionadded{2.1}
\end{funcdesc}

\begin{funcdesc}{bisect}{\unspecified}
  Alias for \function{bisect_right()}.
\end{funcdesc}

\begin{funcdesc}{insort_left}{list, item\optional{, lo\optional{, hi}}}
  Insert \var{item} in \var{list} in sorted order.  This is equivalent
  to \code{\var{list}.insert(bisect.bisect_left(\var{list}, \var{item},
  \var{lo}, \var{hi}), \var{item})}.  This assumes that \var{list} is
  already sorted.
\versionadded{2.1}
\end{funcdesc}

\begin{funcdesc}{insort_right}{list, item\optional{, lo\optional{, hi}}}
  Similar to \function{insort_left()}, but inserting \var{item} in
  \var{list} after any existing entries of \var{item}.
\versionadded{2.1}
\end{funcdesc}

\begin{funcdesc}{insort}{\unspecified}
  Alias for \function{insort_right()}.
\end{funcdesc}


\subsection{Examples}
\nodename{bisect-example}

The \function{bisect()} function is generally useful for categorizing
numeric data.  This example uses \function{bisect()} to look up a
letter grade for an exam total (say) based on a set of ordered numeric
breakpoints: 85 and up is an `A', 75..84 is a `B', etc.

\begin{verbatim}
>>> grades = "FEDCBA"
>>> breakpoints = [30, 44, 66, 75, 85]
>>> from bisect import bisect
>>> def grade(total):
...           return grades[bisect(breakpoints, total)]
...
>>> grade(66)
'C'
>>> map(grade, [33, 99, 77, 44, 12, 88])
['E', 'A', 'B', 'D', 'F', 'A']
\end{verbatim}

The bisect module can be used with the Queue module to implement a priority
queue (example courtesy of Fredrik Lundh): \index{Priority Queue}

\begin{verbatim}
import Queue, bisect

class PriorityQueue(Queue.Queue):
    def _put(self, item):
        bisect.insort(self.queue, item)

# usage
queue = PriorityQueue(0)
queue.put((2, "second"))
queue.put((1, "first"))
queue.put((3, "third"))
priority, value = queue.get()
\end{verbatim}

\section{\module{array} ---
         Efficient arrays of numeric values}

\declaremodule{builtin}{array}
\modulesynopsis{Efficient arrays of uniformly typed numeric values.}


This module defines a new object type which can efficiently represent
an array of basic values: characters, integers, floating point
numbers.  Arrays\index{arrays} are sequence types and behave very much
like lists, except that the type of objects stored in them is
constrained.  The type is specified at object creation time by using a
\dfn{type code}, which is a single character.  The following type
codes are defined:

\begin{tableiii}{c|l|c}{code}{Type code}{C Type}{Minimum size in bytes}
\lineiii{'c'}{character}{1}
\lineiii{'b'}{signed int}{1}
\lineiii{'B'}{unsigned int}{1}
\lineiii{'h'}{signed int}{2}
\lineiii{'H'}{unsigned int}{2}
\lineiii{'i'}{signed int}{2}
\lineiii{'I'}{unsigned int}{2}
\lineiii{'l'}{signed int}{4}
\lineiii{'L'}{unsigned int}{4}
\lineiii{'f'}{float}{4}
\lineiii{'d'}{double}{8}
\end{tableiii}

The actual representation of values is determined by the machine
architecture (strictly speaking, by the C implementation).  The actual
size can be accessed through the \member{itemsize} attribute.  The values
stored  for \code{'L'} and \code{'I'} items will be represented as
Python long integers when retrieved, because Python's plain integer
type cannot represent the full range of C's unsigned (long) integers.


The module defines the following function and type object:

\begin{funcdesc}{array}{typecode\optional{, initializer}}
Return a new array whose items are restricted by \var{typecode}, and
initialized from the optional \var{initializer} value, which must be a
list or a string.  The list or string is passed to the new array's
\method{fromlist()} or \method{fromstring()} method (see below) to add
initial items to the array.
\end{funcdesc}

\begin{datadesc}{ArrayType}
Type object corresponding to the objects returned by
\function{array()}.
\end{datadesc}


Array objects support the following data items and methods:

\begin{memberdesc}[array]{typecode}
The typecode character used to create the array.
\end{memberdesc}

\begin{memberdesc}[array]{itemsize}
The length in bytes of one array item in the internal representation.
\end{memberdesc}


\begin{methoddesc}[array]{append}{x}
Append a new item with value \var{x} to the end of the array.
\end{methoddesc}

\begin{methoddesc}[array]{buffer_info}{}
Return a tuple \code{(\var{address}, \var{length})} giving the current
memory address and the length in bytes of the buffer used to hold
array's contents.  This is occasionally useful when working with
low-level (and inherently unsafe) I/O interfaces that require memory
addresses, such as certain \cfunction{ioctl()} operations.  The returned
numbers are valid as long as the array exists and no length-changing
operations are applied to it.
\end{methoddesc}

\begin{methoddesc}[array]{byteswap}{x}
``Byteswap'' all items of the array.  This is only supported for
integer values.  It is useful when reading data from a file written
on a machine with a different byte order.
\end{methoddesc}

\begin{methoddesc}[array]{fromfile}{f, n}
Read \var{n} items (as machine values) from the file object \var{f}
and append them to the end of the array.  If less than \var{n} items
are available, \exception{EOFError} is raised, but the items that were
available are still inserted into the array.  \var{f} must be a real
built-in file object; something else with a \method{read()} method won't
do.
\end{methoddesc}

\begin{methoddesc}[array]{fromlist}{list}
Append items from the list.  This is equivalent to
\samp{for x in \var{list}:\ a.append(x)}
except that if there is a type error, the array is unchanged.
\end{methoddesc}

\begin{methoddesc}[array]{fromstring}{s}
Appends items from the string, interpreting the string as an
array of machine values (i.e. as if it had been read from a
file using the \method{fromfile()} method).
\end{methoddesc}

\begin{methoddesc}[array]{insert}{i, x}
Insert a new item with value \var{x} in the array before position
\var{i}.
\end{methoddesc}

\begin{methoddesc}[array]{read}{f, n}
\deprecated {1.5.1}
  {Use the \method{fromfile()} method.}
Read \var{n} items (as machine values) from the file object \var{f}
and append them to the end of the array.  If less than \var{n} items
are available, \exception{EOFError} is raised, but the items that were
available are still inserted into the array.  \var{f} must be a real
built-in file object; something else with a \method{read()} method won't
do.
\end{methoddesc}

\begin{methoddesc}[array]{reverse}{}
Reverse the order of the items in the array.
\end{methoddesc}

\begin{methoddesc}[array]{tofile}{f}
Write all items (as machine values) to the file object \var{f}.
\end{methoddesc}

\begin{methoddesc}[array]{tolist}{}
Convert the array to an ordinary list with the same items.
\end{methoddesc}

\begin{methoddesc}[array]{tostring}{}
Convert the array to an array of machine values and return the
string representation (the same sequence of bytes that would
be written to a file by the \method{tofile()} method.)
\end{methoddesc}

\begin{methoddesc}[array]{write}{f}
\deprecated {1.5.1}
  {Use the \method{tofile()} method.}
Write all items (as machine values) to the file object \var{f}.
\end{methoddesc}

When an array object is printed or converted to a string, it is
represented as \code{array(\var{typecode}, \var{initializer})}.  The
\var{initializer} is omitted if the array is empty, otherwise it is a
string if the \var{typecode} is \code{'c'}, otherwise it is a list of
numbers.  The string is guaranteed to be able to be converted back to
an array with the same type and value using reverse quotes
(\code{``}).  Examples:

\begin{verbatim}
array('l')
array('c', 'hello world')
array('l', [1, 2, 3, 4, 5])
array('d', [1.0, 2.0, 3.14])
\end{verbatim}


\begin{seealso}
  \seemodule{struct}{packing and unpacking of heterogeneous binary data}
  \seemodule{xdrlib{{packing and unpacking of XDR data}
\end{seealso}

\section{\module{fileinput} ---
         Iterate over lines from multiple input streams}
\declaremodule{standard}{fileinput}
\moduleauthor{Guido van Rossum}{guido@python.org}
\sectionauthor{Fred L. Drake, Jr.}{fdrake@acm.org}

\modulesynopsis{Perl-like iteration over lines from multiple input
streams, with ``save in place'' capability.}


This module implements a helper class and functions to quickly write a
loop over standard input or a list of files.

The typical use is:

\begin{verbatim}
import fileinput
for line in fileinput.input():
    process(line)
\end{verbatim}

This iterates over the lines of all files listed in
\code{sys.argv[1:]}, defaulting to \code{sys.stdin} if the list is
empty.  If a filename is \code{'-'}, it is also replaced by
\code{sys.stdin}.  To specify an alternative list of filenames, pass
it as the first argument to \function{input()}.  A single file name is
also allowed.

All files are opened in text mode.  If an I/O error occurs during
opening or reading a file, \exception{IOError} is raised.

If \code{sys.stdin} is used more than once, the second and further use
will return no lines, except perhaps for interactive use, or if it has
been explicitly reset (e.g. using \code{sys.stdin.seek(0)}).

Empty files are opened and immediately closed; the only time their
presence in the list of filenames is noticeable at all is when the
last file opened is empty.

It is possible that the last line of a file does not end in a newline
character; lines are returned including the trailing newline when it
is present.

The following function is the primary interface of this module:

\begin{funcdesc}{input}{\optional{files\optional{,
                       inplace\optional{, backup}}}}
  Create an instance of the \class{FileInput} class.  The instance
  will be used as global state for the functions of this module, and
  is also returned to use during iteration.
\end{funcdesc}


The following functions use the global state created by
\function{input()}; if there is no active state,
\exception{RuntimeError} is raised.

\begin{funcdesc}{filename}{}
  Return the name of the file currently being read.  Before the first
  line has been read, returns \code{None}.
\end{funcdesc}

\begin{funcdesc}{lineno}{}
  Return the cumulative line number of the line that has just been
  read.  Before the first line has been read, returns \code{0}.  After
  the last line of the last file has been read, returns the line
  number of that line.
\end{funcdesc}

\begin{funcdesc}{filelineno}{}
  Return the line number in the current file.  Before the first line
  has been read, returns \code{0}.  After the last line of the last
  file has been read, returns the line number of that line within the
  file.
\end{funcdesc}

\begin{funcdesc}{isfirstline}{}
  Returns true the line just read is the first line of its file,
  otherwise returns false.
\end{funcdesc}

\begin{funcdesc}{isstdin}{}
  Returns true if the last line was read from \code{sys.stdin},
  otherwise returns false.
\end{funcdesc}

\begin{funcdesc}{nextfile}{}
  Close the current file so that the next iteration will read the
  first line from the next file (if any); lines not read from the file
  will not count towards the cumulative line count.  The filename is
  not changed until after the first line of the next file has been
  read.  Before the first line has been read, this function has no
  effect; it cannot be used to skip the first file.  After the last
  line of the last file has been read, this function has no effect.
\end{funcdesc}

\begin{funcdesc}{close}{}
  Close the sequence.
\end{funcdesc}


The class which implements the sequence behavior provided by the
module is available for subclassing as well:

\begin{classdesc}{FileInput}{\optional{files\optional{,
                             inplace\optional{, backup}}}}
  Class \class{FileInput} is the implementation; its methods
  \method{filename()}, \method{lineno()}, \method{fileline()},
  \method{isfirstline()}, \method{isstdin()}, \method{nextfile()} and
  \method{close()} correspond to the functions of the same name in the
  module.  In addition it has a \method{readline()} method which
  returns the next input line, and a \method{__getitem__()} method
  which implements the sequence behavior.  The sequence must be
  accessed in strictly sequential order; random access and
  \method{readline()} cannot be mixed.
\end{classdesc}

\strong{Optional in-place filtering:} if the keyword argument
\code{\var{inplace}=1} is passed to \function{input()} or to the
\class{FileInput} constructor, the file is moved to a backup file and
standard output is directed to the input file.
This makes it possible to write a filter that rewrites its input file
in place.  If the keyword argument \code{\var{backup}='.<some
extension>'} is also given, it specifies the extension for the backup
file, and the backup file remains around; by default, the extension is
\code{'.bak'} and it is deleted when the output file is closed.  In-place
filtering is disabled when standard input is read.

\strong{Caveat:} The current implementation does not work for MS-DOS
8+3 filesystems.

% This section was contributed by Drew Csillag <drew_csillag@geocities.com>.

\section{\module{calendar} ---
         Functions that emulate the \UNIX{} \program{cal} program.}
\declaremodule{standard}{calendar}

\modulesynopsis{Functions that emulate the \UNIX{} \program{cal}
program.}


This module allows you to output calendars like the \UNIX{}
\manpage{cal}{1} program.

\begin{funcdesc}{isleap}{year}
Returns \code{1} if \var{year} is a leap year.
\end{funcdesc}

\begin{funcdesc}{leapdays}{year1, year2}
Return the number of leap years in the range
[\var{year1}\ldots\var{year2}].
\end{funcdesc}

\begin{funcdesc}{weekday}{year, month, day}
Returns the day of the week (\code{0} is Monday) for \var{year}
(\code{1970}--\ldots), \var{month} (\code{1}--\code{12}), \var{day}
(\code{1}--\code{31}).
\end{funcdesc}

\begin{funcdesc}{monthrange}{year, month}
Returns weekday of first day of the month and number of days in month, 
for the specified \var{year} and \var{month}.
\end{funcdesc}

\begin{funcdesc}{monthcalendar}{year, month}
Returns a matrix representing a month's calendar.  Each row represents
a week; days outside of the month a represented by zeros.
\end{funcdesc}

\begin{funcdesc}{prmonth}{year, month\optional{, width\optional{, length}}}
Prints a month's calendar.  If \var{width} is provided, it specifies
the width of the columns that the numbers are centered in.  If
\var{length} is given, it specifies the number of lines that each
week will use.
\end{funcdesc}

\begin{funcdesc}{prcal}{year}
Prints the calendar for the year \var{year}.
\end{funcdesc}

\section{\module{cmd} ---
         Support for line-oriented command interpreters}

\declaremodule{standard}{cmd}
\sectionauthor{Eric S. Raymond}{esr@snark.thyrsus.com}
\modulesynopsis{Build line-oriented command interpreters.}


The \class{Cmd} class provides a simple framework for writing
line-oriented command interpreters.  These are often useful for
test harnesses, administrative tools, and prototypes that will
later be wrapped in a more sophisticated interface.

\begin{classdesc}{Cmd}{\optional{completekey\optional{,
                       stdin\optional{, stdout}}}}
A \class{Cmd} instance or subclass instance is a line-oriented
interpreter framework.  There is no good reason to instantiate
\class{Cmd} itself; rather, it's useful as a superclass of an
interpreter class you define yourself in order to inherit
\class{Cmd}'s methods and encapsulate action methods.

The optional argument \var{completekey} is the \refmodule{readline} name
of a completion key; it defaults to \kbd{Tab}. If \var{completekey} is
not \constant{None} and \refmodule{readline} is available, command completion
is done automatically.

The optional arguments \var{stdin} and \var{stdout} specify the 
input and output file objects that the Cmd instance or subclass 
instance will use for input and output. If not specified, they
will default to \var{sys.stdin} and \var{sys.stdout}.

\versionchanged[The \var{stdin} and \var{stdout} parameters were added]{2.3}
\end{classdesc}

\subsection{Cmd Objects}
\label{Cmd-objects}

A \class{Cmd} instance has the following methods:

\begin{methoddesc}{cmdloop}{\optional{intro}}
Repeatedly issue a prompt, accept input, parse an initial prefix off
the received input, and dispatch to action methods, passing them the
remainder of the line as argument.

The optional argument is a banner or intro string to be issued before the
first prompt (this overrides the \member{intro} class member).

If the \refmodule{readline} module is loaded, input will automatically
inherit \program{bash}-like history-list editing (e.g. \kbd{Control-P}
scrolls back to the last command, \kbd{Control-N} forward to the next
one, \kbd{Control-F} moves the cursor to the right non-destructively,
\kbd{Control-B} moves the cursor to the left non-destructively, etc.).

An end-of-file on input is passed back as the string \code{'EOF'}.

An interpreter instance will recognize a command name \samp{foo} if
and only if it has a method \method{do_foo()}.  As a special case,
a line beginning with the character \character{?} is dispatched to
the method \method{do_help()}.  As another special case, a line
beginning with the character \character{!} is dispatched to the
method \method{do_shell()} (if such a method is defined).

This method will return when the \method{postcmd()} method returns a
true value.  The \var{stop} argument to \method{postcmd()} is the
return value from the command's corresponding \method{do_*()} method.

If completion is enabled, completing commands will be done
automatically, and completing of commands args is done by calling
\method{complete_foo()} with arguments \var{text}, \var{line},
\var{begidx}, and \var{endidx}.  \var{text} is the string prefix we
are attempting to match: all returned matches must begin with it.
\var{line} is the current input line with leading whitespace removed,
\var{begidx} and \var{endidx} are the beginning and ending indexes
of the prefix text, which could be used to provide different
completion depending upon which position the argument is in.

All subclasses of \class{Cmd} inherit a predefined \method{do_help()}.
This method, called with an argument \code{'bar'}, invokes the
corresponding method \method{help_bar()}.  With no argument,
\method{do_help()} lists all available help topics (that is, all
commands with corresponding \method{help_*()} methods), and also lists
any undocumented commands.
\end{methoddesc}

\begin{methoddesc}{onecmd}{str}
Interpret the argument as though it had been typed in response to the
prompt.  This may be overridden, but should not normally need to be;
see the \method{precmd()} and \method{postcmd()} methods for useful
execution hooks.  The return value is a flag indicating whether
interpretation of commands by the interpreter should stop.  If there
is a \method{do_*()} method for the command \var{str}, the return
value of that method is returned, otherwise the return value from the
\method{default()} method is returned.
\end{methoddesc}

\begin{methoddesc}{emptyline}{}
Method called when an empty line is entered in response to the prompt.
If this method is not overridden, it repeats the last nonempty command
entered.  
\end{methoddesc}

\begin{methoddesc}{default}{line}
Method called on an input line when the command prefix is not
recognized. If this method is not overridden, it prints an
error message and returns.
\end{methoddesc}

\begin{methoddesc}{completedefault}{text, line, begidx, endidx}
Method called to complete an input line when no command-specific
\method{complete_*()} method is available.  By default, it returns an
empty list.
\end{methoddesc}

\begin{methoddesc}{precmd}{line}
Hook method executed just before the command line \var{line} is
interpreted, but after the input prompt is generated and issued.  This
method is a stub in \class{Cmd}; it exists to be overridden by
subclasses.  The return value is used as the command which will be
executed by the \method{onecmd()} method; the \method{precmd()}
implementation may re-write the command or simply return \var{line}
unchanged.
\end{methoddesc}

\begin{methoddesc}{postcmd}{stop, line}
Hook method executed just after a command dispatch is finished.  This
method is a stub in \class{Cmd}; it exists to be overridden by
subclasses.  \var{line} is the command line which was executed, and
\var{stop} is a flag which indicates whether execution will be
terminated after the call to \method{postcmd()}; this will be the
return value of the \method{onecmd()} method.  The return value of
this method will be used as the new value for the internal flag which
corresponds to \var{stop}; returning false will cause interpretation
to continue.
\end{methoddesc}

\begin{methoddesc}{preloop}{}
Hook method executed once when \method{cmdloop()} is called.  This
method is a stub in \class{Cmd}; it exists to be overridden by
subclasses.
\end{methoddesc}

\begin{methoddesc}{postloop}{}
Hook method executed once when \method{cmdloop()} is about to return.
This method is a stub in \class{Cmd}; it exists to be overridden by
subclasses.
\end{methoddesc}

Instances of \class{Cmd} subclasses have some public instance variables:

\begin{memberdesc}{prompt}
The prompt issued to solicit input.
\end{memberdesc}

\begin{memberdesc}{identchars}
The string of characters accepted for the command prefix.
\end{memberdesc}

\begin{memberdesc}{lastcmd}
The last nonempty command prefix seen. 
\end{memberdesc}

\begin{memberdesc}{intro}
A string to issue as an intro or banner.  May be overridden by giving
the \method{cmdloop()} method an argument.
\end{memberdesc}

\begin{memberdesc}{doc_header}
The header to issue if the help output has a section for documented
commands.
\end{memberdesc}

\begin{memberdesc}{misc_header}
The header to issue if the help output has a section for miscellaneous 
help topics (that is, there are \method{help_*()} methods without
corresponding \method{do_*()} methods).
\end{memberdesc}

\begin{memberdesc}{undoc_header}
The header to issue if the help output has a section for undocumented 
commands (that is, there are \method{do_*()} methods without
corresponding \method{help_*()} methods).
\end{memberdesc}

\begin{memberdesc}{ruler}
The character used to draw separator lines under the help-message
headers.  If empty, no ruler line is drawn.  It defaults to
\character{=}.
\end{memberdesc}

\section{\module{shlex} ---
         Simple lexical analysis}

\declaremodule{standard}{shlex}
\modulesynopsis{Simple lexical analysis for \UNIX{} shell-like languages.}
\moduleauthor{Eric S. Raymond}{esr@snark.thyrsus.com}
\sectionauthor{Eric S. Raymond}{esr@snark.thyrsus.com}

\versionadded{1.5.2}

The \class{shlex} class makes it easy to write lexical analyzers for
simple syntaxes resembling that of the \UNIX{} shell.  This will often
be useful for writing minilanguages, e.g.\ in run control files for
Python applications.

\begin{classdesc}{shlex}{\optional{stream\optional{, file}}}
A \class{shlex} instance or subclass instance is a lexical analyzer
object.  The initialization argument, if present, specifies where to
read characters from. It must be a file- or stream-like object with
\method{read()} and \method{readline()} methods.  If no argument is given,
input will be taken from \code{sys.stdin}.  The second optional 
argument is a filename string, which sets the initial value of the
\member{infile} member.  If the stream argument is omitted or
equal to \code{sys.stdin}, this second argument defaults to ``stdin''.
\end{classdesc}


\begin{seealso}
  \seemodule{ConfigParser}{Parser for configuration files similar to the
                           Windows \file{.ini} files.}
\end{seealso}


\subsection{shlex Objects \label{shlex-objects}}

A \class{shlex} instance has the following methods:


\begin{methoddesc}{get_token}{}
Return a token.  If tokens have been stacked using
\method{push_token()}, pop a token off the stack.  Otherwise, read one
from the input stream.  If reading encounters an immediate
end-of-file, an empty string is returned. 
\end{methoddesc}

\begin{methoddesc}{push_token}{str}
Push the argument onto the token stack.
\end{methoddesc}

\begin{methoddesc}{read_token}{}
Read a raw token.  Ignore the pushback stack, and do not interpret source
requests.  (This is not ordinarily a useful entry point, and is
documented here only for the sake of completeness.)
\end{methoddesc}

\begin{methoddesc}{sourcehook}{filename}
When \class{shlex} detects a source request (see
\member{source} below) this method is given the following token as
argument, and expected to return a tuple consisting of a filename and
an open file-like object.

Normally, this method first strips any quotes off the argument.  If
the result is an absolute pathname, or there was no previous source
request in effect, or the previous source was a stream
(e.g. \code{sys.stdin}), the result is left alone.  Otherwise, if the
result is a relative pathname, the directory part of the name of the
file immediately before it on the source inclusion stack is prepended
(this behavior is like the way the C preprocessor handles
\code{\#include "file.h"}).

The result of the manipulations is treated as a filename, and returned
as the first component of the tuple, with
\function{open()} called on it to yield the second component. (Note:
this is the reverse of the order of arguments in instance initialization!)

This hook is exposed so that you can use it to implement directory
search paths, addition of file extensions, and other namespace hacks.
There is no corresponding `close' hook, but a shlex instance will call
the \method{close()} method of the sourced input stream when it
returns \EOF.

For more explicit control of source stacking, use the
\method{push_source()} and \method{pop_source()} methods. 
\end{methoddesc}

\begin{methoddesc}{push_source}{stream\optional{, filename}}
Push an input source stream onto the input stack.  If the filename
argument is specified it will later be available for use in error
messages.  This is the same method used internally by the
\method{sourcehook} method.
\versionadded{2.1}
\end{methoddesc}

\begin{methoddesc}{pop_source}{}
Pop the last-pushed input source from the input stack.
This is the same method used internally when the lexer reaches
\EOF on a stacked input stream.
\versionadded{2.1}
\end{methoddesc}

\begin{methoddesc}{error_leader}{\optional{file\optional{, line}}}
This method generates an error message leader in the format of a
\UNIX{} C compiler error label; the format is \code{'"\%s", line \%d: '},
where the \samp{\%s} is replaced with the name of the current source
file and the \samp{\%d} with the current input line number (the
optional arguments can be used to override these).

This convenience is provided to encourage \module{shlex} users to
generate error messages in the standard, parseable format understood
by Emacs and other \UNIX{} tools.
\end{methoddesc}

Instances of \class{shlex} subclasses have some public instance
variables which either control lexical analysis or can be used for
debugging:

\begin{memberdesc}{commenters}
The string of characters that are recognized as comment beginners.
All characters from the comment beginner to end of line are ignored.
Includes just \character{\#} by default.   
\end{memberdesc}

\begin{memberdesc}{wordchars}
The string of characters that will accumulate into multi-character
tokens.  By default, includes all \ASCII{} alphanumerics and
underscore.
\end{memberdesc}

\begin{memberdesc}{whitespace}
Characters that will be considered whitespace and skipped.  Whitespace
bounds tokens.  By default, includes space, tab, linefeed and
carriage-return.
\end{memberdesc}

\begin{memberdesc}{quotes}
Characters that will be considered string quotes.  The token
accumulates until the same quote is encountered again (thus, different
quote types protect each other as in the shell.)  By default, includes
\ASCII{} single and double quotes.
\end{memberdesc}

\begin{memberdesc}{infile}
The name of the current input file, as initially set at class
instantiation time or stacked by later source requests.  It may
be useful to examine this when constructing error messages.
\end{memberdesc}

\begin{memberdesc}{instream}
The input stream from which this \class{shlex} instance is reading
characters.
\end{memberdesc}

\begin{memberdesc}{source}
This member is \code{None} by default.  If you assign a string to it,
that string will be recognized as a lexical-level inclusion request
similar to the \samp{source} keyword in various shells.  That is, the
immediately following token will opened as a filename and input taken
from that stream until \EOF, at which point the \method{close()}
method of that stream will be called and the input source will again
become the original input stream. Source requests may be stacked any
number of levels deep.
\end{memberdesc}

\begin{memberdesc}{debug}
If this member is numeric and \code{1} or more, a \class{shlex}
instance will print verbose progress output on its behavior.  If you
need to use this, you can read the module source code to learn the
details.
\end{memberdesc}

Note that any character not declared to be a word character,
whitespace, or a quote will be returned as a single-character token.

Quote and comment characters are not recognized within words.  Thus,
the bare words \samp{ain't} and \samp{ain\#t} would be returned as single
tokens by the default parser.

\begin{memberdesc}{lineno}
Source line number (count of newlines seen so far plus one).
\end{memberdesc}

\begin{memberdesc}{token}
The token buffer.  It may be useful to examine this when catching
exceptions.
\end{memberdesc}


\chapter{Generic Operating System Services}
\label{allos}

The modules described in this chapter provide interfaces to operating
system features that are available on (almost) all operating systems,
such as files and a clock.  The interfaces are generally modelled
after the \UNIX{} or \C{} interfaces but they are available on most
other systems as well.  Here's an overview:

\begin{description}

\item[os]
--- Miscellaneous OS interfaces.

\item[time]
--- Time access and conversions.

\item[getopt]
--- Parser for command line options.

\item[tempfile]
--- Generate temporary file names.

\item[errno]
--- Standard errno system symbols.

\item[glob]
--- \UNIX{} shell style pathname pattern expansion.

\item[fnmatch]
--- \UNIX{} shell style pathname pattern matching.

\item[locale]
--- Internationalization services.

\end{description}
		% Generic Operating System Services
\section{\module{os} ---
         Miscellaneous operating system interfaces}

\declaremodule{standard}{os}
\modulesynopsis{Miscellaneous operating system interfaces.}


This module provides a more portable way of using operating system
dependent functionality than importing a operating system dependent
built-in module like \refmodule{posix} or \module{nt}.

This module searches for an operating system dependent built-in module like
\module{mac} or \refmodule{posix} and exports the same functions and data
as found there.  The design of all Python's built-in operating system dependent
modules is such that as long as the same functionality is available,
it uses the same interface; for example, the function
\code{os.stat(\var{path})} returns stat information about \var{path} in
the same format (which happens to have originated with the
\POSIX{} interface).

Extensions peculiar to a particular operating system are also
available through the \module{os} module, but using them is of course a
threat to portability!

Note that after the first time \module{os} is imported, there is
\emph{no} performance penalty in using functions from \module{os}
instead of directly from the operating system dependent built-in module,
so there should be \emph{no} reason not to use \module{os}!


% Frank Stajano <fstajano@uk.research.att.com> complained that it
% wasn't clear that the entries described in the subsections were all
% available at the module level (most uses of subsections are
% different); I think this is only a problem for the HTML version,
% where the relationship may not be as clear.
%
\ifhtml
The \module{os} module contains many functions and data values.
The items below and in the following sub-sections are all available
directly from the \module{os} module.
\fi


\begin{excdesc}{error}
This exception is raised when a function returns a system-related
error (not for illegal argument types or other incidental errors).
This is also known as the built-in exception \exception{OSError}.  The
accompanying value is a pair containing the numeric error code from
\cdata{errno} and the corresponding string, as would be printed by the
C function \cfunction{perror()}.  See the module
\refmodule{errno}\refbimodindex{errno}, which contains names for the
error codes defined by the underlying operating system.

When exceptions are classes, this exception carries two attributes,
\member{errno} and \member{strerror}.  The first holds the value of
the C \cdata{errno} variable, and the latter holds the corresponding
error message from \cfunction{strerror()}.  For exceptions that
involve a file system path (such as \function{chdir()} or
\function{unlink()}), the exception instance will contain a third
attribute, \member{filename}, which is the file name passed to the
function.
\end{excdesc}

\begin{datadesc}{name}
The name of the operating system dependent module imported.  The
following names have currently been registered: \code{'posix'},
\code{'nt'}, \code{'mac'}, \code{'os2'}, \code{'ce'},
\code{'java'}, \code{'riscos'}.
\end{datadesc}

\begin{datadesc}{path}
The corresponding operating system dependent standard module for pathname
operations, such as \module{posixpath} or \module{macpath}.  Thus,
given the proper imports, \code{os.path.split(\var{file})} is
equivalent to but more portable than
\code{posixpath.split(\var{file})}.  Note that this is also an
importable module: it may be imported directly as
\refmodule{os.path}.
\end{datadesc}



\subsection{Process Parameters \label{os-procinfo}}

These functions and data items provide information and operate on the
current process and user.

\begin{datadesc}{environ}
A mapping object representing the string environment. For example,
\code{environ['HOME']} is the pathname of your home directory (on some
platforms), and is equivalent to \code{getenv("HOME")} in C.

This mapping is captured the first time the \module{os} module is
imported, typically during Python startup as part of processing
\file{site.py}.  Changes to the environment made after this time are
not reflected in \code{os.environ}, except for changes made by modifying
\code{os.environ} directly.

If the platform supports the \function{putenv()} function, this
mapping may be used to modify the environment as well as query the
environment.  \function{putenv()} will be called automatically when
the mapping is modified.
\note{Calling \function{putenv()} directly does not change
\code{os.environ}, so it's better to modify \code{os.environ}.}
\note{On some platforms, including FreeBSD and Mac OS X, setting
\code{environ} may cause memory leaks.  Refer to the system documentation
for \cfunction{putenv()}.}

If \function{putenv()} is not provided, a modified copy of this mapping 
may be passed to the appropriate process-creation functions to cause 
child processes to use a modified environment.

If the platform supports the \function{unsetenv()} function, you can 
delete items in this mapping to unset environment variables.
\function{unsetenv()} will be called automatically when an item is
deleted from \code{os.environ}.

\end{datadesc}

\begin{funcdescni}{chdir}{path}
\funclineni{fchdir}{fd}
\funclineni{getcwd}{}
These functions are described in ``Files and Directories'' (section
\ref{os-file-dir}).
\end{funcdescni}

\begin{funcdesc}{ctermid}{}
Return the filename corresponding to the controlling terminal of the
process.
Availability: \UNIX.
\end{funcdesc}

\begin{funcdesc}{getegid}{}
Return the effective group id of the current process.  This
corresponds to the `set id' bit on the file being executed in the
current process.
Availability: \UNIX.
\end{funcdesc}

\begin{funcdesc}{geteuid}{}
\index{user!effective id}
Return the current process' effective user id.
Availability: \UNIX.
\end{funcdesc}

\begin{funcdesc}{getgid}{}
\index{process!group}
Return the real group id of the current process.
Availability: \UNIX.
\end{funcdesc}

\begin{funcdesc}{getgroups}{}
Return list of supplemental group ids associated with the current
process.
Availability: \UNIX.
\end{funcdesc}

\begin{funcdesc}{getlogin}{}
Return the name of the user logged in on the controlling terminal of
the process.  For most purposes, it is more useful to use the
environment variable \envvar{LOGNAME} to find out who the user is,
or \code{pwd.getpwuid(os.getuid())[0]} to get the login name
of the currently effective user ID.
Availability: \UNIX.
\end{funcdesc}

\begin{funcdesc}{getpgid}{pid}
Return the process group id of the process with process id \var{pid}.
If \var{pid} is 0, the process group id of the current process is
returned. Availability: \UNIX.
\versionadded{2.3}
\end{funcdesc}

\begin{funcdesc}{getpgrp}{}
\index{process!group}
Return the id of the current process group.
Availability: \UNIX.
\end{funcdesc}

\begin{funcdesc}{getpid}{}
\index{process!id}
Return the current process id.
Availability: \UNIX, Windows.
\end{funcdesc}

\begin{funcdesc}{getppid}{}
\index{process!id of parent}
Return the parent's process id.
Availability: \UNIX.
\end{funcdesc}

\begin{funcdesc}{getuid}{}
\index{user!id}
Return the current process' user id.
Availability: \UNIX.
\end{funcdesc}

\begin{funcdesc}{getenv}{varname\optional{, value}}
Return the value of the environment variable \var{varname} if it
exists, or \var{value} if it doesn't.  \var{value} defaults to
\code{None}.
Availability: most flavors of \UNIX, Windows.
\end{funcdesc}

\begin{funcdesc}{putenv}{varname, value}
\index{environment variables!setting}
Set the environment variable named \var{varname} to the string
\var{value}.  Such changes to the environment affect subprocesses
started with \function{os.system()}, \function{popen()} or
\function{fork()} and \function{execv()}.
Availability: most flavors of \UNIX, Windows.

\note{On some platforms, including FreeBSD and Mac OS X,
setting \code{environ} may cause memory leaks.
Refer to the system documentation for putenv.}

When \function{putenv()} is
supported, assignments to items in \code{os.environ} are automatically
translated into corresponding calls to \function{putenv()}; however,
calls to \function{putenv()} don't update \code{os.environ}, so it is
actually preferable to assign to items of \code{os.environ}.
\end{funcdesc}

\begin{funcdesc}{setegid}{egid}
Set the current process's effective group id.
Availability: \UNIX.
\end{funcdesc}

\begin{funcdesc}{seteuid}{euid}
Set the current process's effective user id.
Availability: \UNIX.
\end{funcdesc}

\begin{funcdesc}{setgid}{gid}
Set the current process' group id.
Availability: \UNIX.
\end{funcdesc}

\begin{funcdesc}{setgroups}{groups}
Set the list of supplemental group ids associated with the current
process to \var{groups}. \var{groups} must be a sequence, and each
element must be an integer identifying a group. This operation is
typical available only to the superuser.
Availability: \UNIX.
\versionadded{2.2}
\end{funcdesc}

\begin{funcdesc}{setpgrp}{}
Calls the system call \cfunction{setpgrp()} or \cfunction{setpgrp(0,
0)} depending on which version is implemented (if any).  See the
\UNIX{} manual for the semantics.
Availability: \UNIX.
\end{funcdesc}

\begin{funcdesc}{setpgid}{pid, pgrp} Calls the system call
\cfunction{setpgid()} to set the process group id of the process with
id \var{pid} to the process group with id \var{pgrp}.  See the \UNIX{}
manual for the semantics.
Availability: \UNIX.
\end{funcdesc}

\begin{funcdesc}{setreuid}{ruid, euid}
Set the current process's real and effective user ids.
Availability: \UNIX.
\end{funcdesc}

\begin{funcdesc}{setregid}{rgid, egid}
Set the current process's real and effective group ids.
Availability: \UNIX.
\end{funcdesc}

\begin{funcdesc}{getsid}{pid}
Calls the system call \cfunction{getsid()}.  See the \UNIX{} manual
for the semantics.
Availability: \UNIX. \versionadded{2.4}
\end{funcdesc}

\begin{funcdesc}{setsid}{}
Calls the system call \cfunction{setsid()}.  See the \UNIX{} manual
for the semantics.
Availability: \UNIX.
\end{funcdesc}

\begin{funcdesc}{setuid}{uid}
\index{user!id, setting}
Set the current process' user id.
Availability: \UNIX.
\end{funcdesc}

% placed in this section since it relates to errno.... a little weak
\begin{funcdesc}{strerror}{code}
Return the error message corresponding to the error code in
\var{code}.
Availability: \UNIX, Windows.
\end{funcdesc}

\begin{funcdesc}{umask}{mask}
Set the current numeric umask and returns the previous umask.
Availability: \UNIX, Windows.
\end{funcdesc}

\begin{funcdesc}{uname}{}
Return a 5-tuple containing information identifying the current
operating system.  The tuple contains 5 strings:
\code{(\var{sysname}, \var{nodename}, \var{release}, \var{version},
\var{machine})}.  Some systems truncate the nodename to 8
characters or to the leading component; a better way to get the
hostname is \function{socket.gethostname()}
\withsubitem{(in module socket)}{\ttindex{gethostname()}}
or even
\withsubitem{(in module socket)}{\ttindex{gethostbyaddr()}}
\code{socket.gethostbyaddr(socket.gethostname())}.
Availability: recent flavors of \UNIX.
\end{funcdesc}

\begin{funcdesc}{unsetenv}{varname}
\index{environment variables!deleting}
Unset (delete) the environment variable named \var{varname}. Such
changes to the environment affect subprocesses started with
\function{os.system()}, \function{popen()} or \function{fork()} and
\function{execv()}. Availability: most flavors of \UNIX, Windows.

When \function{unsetenv()} is
supported, deletion of items in \code{os.environ} is automatically
translated into a corresponding call to \function{unsetenv()}; however,
calls to \function{unsetenv()} don't update \code{os.environ}, so it is
actually preferable to delete items of \code{os.environ}.
\end{funcdesc}

\subsection{File Object Creation \label{os-newstreams}}

These functions create new file objects.


\begin{funcdesc}{fdopen}{fd\optional{, mode\optional{, bufsize}}}
Return an open file object connected to the file descriptor \var{fd}.
\index{I/O control!buffering}
The \var{mode} and \var{bufsize} arguments have the same meaning as
the corresponding arguments to the built-in \function{open()}
function.
Availability: Macintosh, \UNIX, Windows.

\versionchanged[When specified, the \var{mode} argument must now start
  with one of the letters \character{r}, \character{w}, or \character{a},
  otherwise a \exception{ValueError} is raised]{2.3}
\end{funcdesc}

\begin{funcdesc}{popen}{command\optional{, mode\optional{, bufsize}}}
Open a pipe to or from \var{command}.  The return value is an open
file object connected to the pipe, which can be read or written
depending on whether \var{mode} is \code{'r'} (default) or \code{'w'}.
The \var{bufsize} argument has the same meaning as the corresponding
argument to the built-in \function{open()} function.  The exit status of
the command (encoded in the format specified for \function{wait()}) is
available as the return value of the \method{close()} method of the file
object, except that when the exit status is zero (termination without
errors), \code{None} is returned.
Availability: Macintosh, \UNIX, Windows.

\versionchanged[This function worked unreliably under Windows in
  earlier versions of Python.  This was due to the use of the
  \cfunction{_popen()} function from the libraries provided with
  Windows.  Newer versions of Python do not use the broken
  implementation from the Windows libraries]{2.0}
\end{funcdesc}

\begin{funcdesc}{tmpfile}{}
Return a new file object opened in update mode (\samp{w+b}).  The file
has no directory entries associated with it and will be automatically
deleted once there are no file descriptors for the file.
Availability: Macintosh, \UNIX, Windows.
\end{funcdesc}


For each of the following \function{popen()} variants, if \var{bufsize} is
specified, it specifies the buffer size for the I/O pipes.
\var{mode}, if provided, should be the string \code{'b'} or
\code{'t'}; on Windows this is needed to determine whether the file
objects should be opened in binary or text mode.  The default value
for \var{mode} is \code{'t'}.

Also, for each of these variants, on \UNIX, \var{cmd} may be a sequence, in
which case arguments will be passed directly to the program without shell
intervention (as with \function{os.spawnv()}). If \var{cmd} is a string it will
be passed to the shell (as with \function{os.system()}).

These methods do not make it possible to retrieve the exit status from
the child processes.  The only way to control the input and output
streams and also retrieve the return codes is to use the
\class{Popen3} and \class{Popen4} classes from the \refmodule{popen2}
module; these are only available on \UNIX.

For a discussion of possible deadlock conditions related to the use
of these functions, see ``\ulink{Flow Control
Issues}{popen2-flow-control.html}''
(section~\ref{popen2-flow-control}).

\begin{funcdesc}{popen2}{cmd\optional{, mode\optional{, bufsize}}}
Executes \var{cmd} as a sub-process.  Returns the file objects
\code{(\var{child_stdin}, \var{child_stdout})}.
Availability: Macintosh, \UNIX, Windows.
\versionadded{2.0}
\end{funcdesc}

\begin{funcdesc}{popen3}{cmd\optional{, mode\optional{, bufsize}}}
Executes \var{cmd} as a sub-process.  Returns the file objects
\code{(\var{child_stdin}, \var{child_stdout}, \var{child_stderr})}.
Availability: Macintosh, \UNIX, Windows.
\versionadded{2.0}
\end{funcdesc}

\begin{funcdesc}{popen4}{cmd\optional{, mode\optional{, bufsize}}}
Executes \var{cmd} as a sub-process.  Returns the file objects
\code{(\var{child_stdin}, \var{child_stdout_and_stderr})}.
Availability: Macintosh, \UNIX, Windows.
\versionadded{2.0}
\end{funcdesc}

(Note that \code{\var{child_stdin}, \var{child_stdout}, and
\var{child_stderr}} are named from the point of view of the child
process, i.e. \var{child_stdin} is the child's standard input.)

This functionality is also available in the \refmodule{popen2} module
using functions of the same names, but the return values of those
functions have a different order.


\subsection{File Descriptor Operations \label{os-fd-ops}}

These functions operate on I/O streams referenced using file
descriptors.  

File descriptors are small integers corresponding to a file that has
been opened by the current process.  For example, standard input is
usually file descriptor 0, standard output is 1, and standard error is
2.  Further files opened by a process will then be assigned 3, 4, 5,
and so forth.  The name ``file descriptor'' is slightly deceptive; on
{\UNIX} platforms, sockets and pipes are also referenced by file descriptors.


\begin{funcdesc}{close}{fd}
Close file descriptor \var{fd}.
Availability: Macintosh, \UNIX, Windows.

\begin{notice}
This function is intended for low-level I/O and must be applied
to a file descriptor as returned by \function{open()} or
\function{pipe()}.  To close a ``file object'' returned by the
built-in function \function{open()} or by \function{popen()} or
\function{fdopen()}, use its \method{close()} method.
\end{notice}
\end{funcdesc}

\begin{funcdesc}{dup}{fd}
Return a duplicate of file descriptor \var{fd}.
Availability: Macintosh, \UNIX, Windows.
\end{funcdesc}

\begin{funcdesc}{dup2}{fd, fd2}
Duplicate file descriptor \var{fd} to \var{fd2}, closing the latter
first if necessary.
Availability: Macintosh, \UNIX, Windows.
\end{funcdesc}

\begin{funcdesc}{fdatasync}{fd}
Force write of file with filedescriptor \var{fd} to disk.
Does not force update of metadata.
Availability: \UNIX.
\end{funcdesc}

\begin{funcdesc}{fpathconf}{fd, name}
Return system configuration information relevant to an open file.
\var{name} specifies the configuration value to retrieve; it may be a
string which is the name of a defined system value; these names are
specified in a number of standards (\POSIX.1, \UNIX{} 95, \UNIX{} 98, and
others).  Some platforms define additional names as well.  The names
known to the host operating system are given in the
\code{pathconf_names} dictionary.  For configuration variables not
included in that mapping, passing an integer for \var{name} is also
accepted.
Availability: Macintosh, \UNIX.

If \var{name} is a string and is not known, \exception{ValueError} is
raised.  If a specific value for \var{name} is not supported by the
host system, even if it is included in \code{pathconf_names}, an
\exception{OSError} is raised with \constant{errno.EINVAL} for the
error number.
\end{funcdesc}

\begin{funcdesc}{fstat}{fd}
Return status for file descriptor \var{fd}, like \function{stat()}.
Availability: Macintosh, \UNIX, Windows.
\end{funcdesc}

\begin{funcdesc}{fstatvfs}{fd}
Return information about the filesystem containing the file associated
with file descriptor \var{fd}, like \function{statvfs()}.
Availability: \UNIX.
\end{funcdesc}

\begin{funcdesc}{fsync}{fd}
Force write of file with filedescriptor \var{fd} to disk.  On \UNIX,
this calls the native \cfunction{fsync()} function; on Windows, the
MS \cfunction{_commit()} function.

If you're starting with a Python file object \var{f}, first do
\code{\var{f}.flush()}, and then do \code{os.fsync(\var{f}.fileno())},
to ensure that all internal buffers associated with \var{f} are written
to disk.
Availability: Macintosh, \UNIX, and Windows starting in 2.2.3.
\end{funcdesc}

\begin{funcdesc}{ftruncate}{fd, length}
Truncate the file corresponding to file descriptor \var{fd},
so that it is at most \var{length} bytes in size.
Availability: Macintosh, \UNIX.
\end{funcdesc}

\begin{funcdesc}{isatty}{fd}
Return \code{True} if the file descriptor \var{fd} is open and
connected to a tty(-like) device, else \code{False}.
Availability: Macintosh, \UNIX.
\end{funcdesc}

\begin{funcdesc}{lseek}{fd, pos, how}
Set the current position of file descriptor \var{fd} to position
\var{pos}, modified by \var{how}: \code{0} to set the position
relative to the beginning of the file; \code{1} to set it relative to
the current position; \code{2} to set it relative to the end of the
file.
Availability: Macintosh, \UNIX, Windows.
\end{funcdesc}

\begin{funcdesc}{open}{file, flags\optional{, mode}}
Open the file \var{file} and set various flags according to
\var{flags} and possibly its mode according to \var{mode}.
The default \var{mode} is \code{0777} (octal), and the current umask
value is first masked out.  Return the file descriptor for the newly
opened file.
Availability: Macintosh, \UNIX, Windows.

For a description of the flag and mode values, see the C run-time
documentation; flag constants (like \constant{O_RDONLY} and
\constant{O_WRONLY}) are defined in this module too (see below).

\begin{notice}
This function is intended for low-level I/O.  For normal usage,
use the built-in function \function{open()}, which returns a ``file
object'' with \method{read()} and \method{write()} methods (and many
more).
\end{notice}
\end{funcdesc}

\begin{funcdesc}{openpty}{}
Open a new pseudo-terminal pair. Return a pair of file descriptors
\code{(\var{master}, \var{slave})} for the pty and the tty,
respectively. For a (slightly) more portable approach, use the
\refmodule{pty}\refstmodindex{pty} module.
Availability: Macintosh, Some flavors of \UNIX.
\end{funcdesc}

\begin{funcdesc}{pipe}{}
Create a pipe.  Return a pair of file descriptors \code{(\var{r},
\var{w})} usable for reading and writing, respectively.
Availability: Macintosh, \UNIX, Windows.
\end{funcdesc}

\begin{funcdesc}{read}{fd, n}
Read at most \var{n} bytes from file descriptor \var{fd}.
Return a string containing the bytes read.  If the end of the file
referred to by \var{fd} has been reached, an empty string is
returned.
Availability: Macintosh, \UNIX, Windows.

\begin{notice}
This function is intended for low-level I/O and must be applied
to a file descriptor as returned by \function{open()} or
\function{pipe()}.  To read a ``file object'' returned by the
built-in function \function{open()} or by \function{popen()} or
\function{fdopen()}, or \code{sys.stdin}, use its
\method{read()} or \method{readline()} methods.
\end{notice}
\end{funcdesc}

\begin{funcdesc}{tcgetpgrp}{fd}
Return the process group associated with the terminal given by
\var{fd} (an open file descriptor as returned by \function{open()}).
Availability: Macintosh, \UNIX.
\end{funcdesc}

\begin{funcdesc}{tcsetpgrp}{fd, pg}
Set the process group associated with the terminal given by
\var{fd} (an open file descriptor as returned by \function{open()})
to \var{pg}.
Availability: Macintosh, \UNIX.
\end{funcdesc}

\begin{funcdesc}{ttyname}{fd}
Return a string which specifies the terminal device associated with
file-descriptor \var{fd}.  If \var{fd} is not associated with a terminal
device, an exception is raised.
Availability:Macintosh,  \UNIX.
\end{funcdesc}

\begin{funcdesc}{write}{fd, str}
Write the string \var{str} to file descriptor \var{fd}.
Return the number of bytes actually written.
Availability: Macintosh, \UNIX, Windows.

\begin{notice}
This function is intended for low-level I/O and must be applied
to a file descriptor as returned by \function{open()} or
\function{pipe()}.  To write a ``file object'' returned by the
built-in function \function{open()} or by \function{popen()} or
\function{fdopen()}, or \code{sys.stdout} or \code{sys.stderr}, use
its \method{write()} method.
\end{notice}
\end{funcdesc}


The following data items are available for use in constructing the
\var{flags} parameter to the \function{open()} function.  Some items will
not be available on all platforms.  For descriptions of their availability
and use, consult \manpage{open}{2}.

\begin{datadesc}{O_RDONLY}
\dataline{O_WRONLY}
\dataline{O_RDWR}
\dataline{O_APPEND}
\dataline{O_CREAT}
\dataline{O_EXCL}
\dataline{O_TRUNC}
Options for the \var{flag} argument to the \function{open()} function.
These can be bit-wise OR'd together.
Availability: Macintosh, \UNIX, Windows.
\end{datadesc}

\begin{datadesc}{O_DSYNC}
\dataline{O_RSYNC}
\dataline{O_SYNC}
\dataline{O_NDELAY}
\dataline{O_NONBLOCK}
\dataline{O_NOCTTY}
\dataline{O_SHLOCK}
\dataline{O_EXLOCK}
More options for the \var{flag} argument to the \function{open()} function.
Availability: Macintosh, \UNIX.
\end{datadesc}

\begin{datadesc}{O_BINARY}
Option for the \var{flag} argument to the \function{open()} function.
This can be bit-wise OR'd together with those listed above.
Availability: Windows.
% XXX need to check on the availability of this one.
\end{datadesc}

\begin{datadesc}{O_NOINHERIT}
\dataline{O_SHORT_LIVED}
\dataline{O_TEMPORARY}
\dataline{O_RANDOM}
\dataline{O_SEQUENTIAL}
\dataline{O_TEXT}
Options for the \var{flag} argument to the \function{open()} function.
These can be bit-wise OR'd together.
Availability: Windows.
\end{datadesc}

\begin{datadesc}{SEEK_SET}
\dataline{SEEK_CUR}
\dataline{SEEK_END}
Parameters to the \function{lseek()} function.
Their values are 0, 1, and 2, respectively.
Availability: Windows, Macintosh, \UNIX.
\versionadded{2.5}
\end{datadesc}

\subsection{Files and Directories \label{os-file-dir}}

\begin{funcdesc}{access}{path, mode}
Use the real uid/gid to test for access to \var{path}.  Note that most
operations will use the effective uid/gid, therefore this routine can
be used in a suid/sgid environment to test if the invoking user has the
specified access to \var{path}.  \var{mode} should be \constant{F_OK}
to test the existence of \var{path}, or it can be the inclusive OR of
one or more of \constant{R_OK}, \constant{W_OK}, and \constant{X_OK} to
test permissions.  Return \constant{True} if access is allowed,
\constant{False} if not.
See the \UNIX{} man page \manpage{access}{2} for more information.
Availability: Macintosh, \UNIX, Windows.

\note{Using \function{access()} to check if a user is authorized to e.g.
open a file before actually doing so using \function{open()} creates a 
security hole, because the user might exploit the short time interval 
between checking and opening the file to manipulate it.}

\note{I/O operations may fail even when \function{access()}
indicates that they would succeed, particularly for operations
on network filesystems which may have permissions semantics
beyond the usual \POSIX{} permission-bit model.}
\end{funcdesc}

\begin{datadesc}{F_OK}
  Value to pass as the \var{mode} parameter of \function{access()} to
  test the existence of \var{path}.
\end{datadesc}

\begin{datadesc}{R_OK}
  Value to include in the \var{mode} parameter of \function{access()}
  to test the readability of \var{path}.
\end{datadesc}

\begin{datadesc}{W_OK}
  Value to include in the \var{mode} parameter of \function{access()}
  to test the writability of \var{path}.
\end{datadesc}

\begin{datadesc}{X_OK}
  Value to include in the \var{mode} parameter of \function{access()}
  to determine if \var{path} can be executed.
\end{datadesc}

\begin{funcdesc}{chdir}{path}
\index{directory!changing}
Change the current working directory to \var{path}.
Availability: Macintosh, \UNIX, Windows.
\end{funcdesc}

\begin{funcdesc}{fchdir}{fd}
Change the current working directory to the directory represented by
the file descriptor \var{fd}.  The descriptor must refer to an opened
directory, not an open file.
Availability: \UNIX.
\versionadded{2.3}
\end{funcdesc}

\begin{funcdesc}{getcwd}{}
Return a string representing the current working directory.
Availability: Macintosh, \UNIX, Windows.
\end{funcdesc}

\begin{funcdesc}{getcwdu}{}
Return a Unicode object representing the current working directory.
Availability: Macintosh, \UNIX, Windows.
\versionadded{2.3}
\end{funcdesc}

\begin{funcdesc}{chroot}{path}
Change the root directory of the current process to \var{path}.
Availability: Macintosh, \UNIX.
\versionadded{2.2}
\end{funcdesc}

\begin{funcdesc}{chmod}{path, mode}
Change the mode of \var{path} to the numeric \var{mode}.
\var{mode} may take one of the following values
(as defined in the \module{stat} module) or bitwise or-ed
combinations of them:
\begin{itemize}
  \item \code{S_ISUID}
  \item \code{S_ISGID}
  \item \code{S_ENFMT}
  \item \code{S_ISVTX}
  \item \code{S_IREAD}
  \item \code{S_IWRITE}
  \item \code{S_IEXEC}
  \item \code{S_IRWXU}
  \item \code{S_IRUSR}
  \item \code{S_IWUSR}
  \item \code{S_IXUSR}
  \item \code{S_IRWXG}
  \item \code{S_IRGRP}
  \item \code{S_IWGRP}
  \item \code{S_IXGRP}
  \item \code{S_IRWXO}
  \item \code{S_IROTH}
  \item \code{S_IWOTH}
  \item \code{S_IXOTH}
\end{itemize}
Availability: Macintosh, \UNIX, Windows.

\note{Although Windows supports \function{chmod()}, you can only 
set the file's read-only flag with it (via the \code{S_IWRITE} 
and \code{S_IREAD} constants or a corresponding integer value). 
All other bits are ignored.}
\end{funcdesc}

\begin{funcdesc}{chown}{path, uid, gid}
Change the owner and group id of \var{path} to the numeric \var{uid}
and \var{gid}. To leave one of the ids unchanged, set it to -1.
Availability: Macintosh, \UNIX.
\end{funcdesc}

\begin{funcdesc}{lchown}{path, uid, gid}
Change the owner and group id of \var{path} to the numeric \var{uid}
and gid. This function will not follow symbolic links.
Availability: Macintosh, \UNIX.
\versionadded{2.3}
\end{funcdesc}

\begin{funcdesc}{link}{src, dst}
Create a hard link pointing to \var{src} named \var{dst}.
Availability: Macintosh, \UNIX.
\end{funcdesc}

\begin{funcdesc}{listdir}{path}
Return a list containing the names of the entries in the directory.
The list is in arbitrary order.  It does not include the special
entries \code{'.'} and \code{'..'} even if they are present in the
directory.
Availability: Macintosh, \UNIX, Windows.

\versionchanged[On Windows NT/2k/XP and \UNIX, if \var{path} is a Unicode
object, the result will be a list of Unicode objects.]{2.3}
\end{funcdesc}

\begin{funcdesc}{lstat}{path}
Like \function{stat()}, but do not follow symbolic links.
Availability: Macintosh, \UNIX.
\end{funcdesc}

\begin{funcdesc}{mkfifo}{path\optional{, mode}}
Create a FIFO (a named pipe) named \var{path} with numeric mode
\var{mode}.  The default \var{mode} is \code{0666} (octal).  The current
umask value is first masked out from the mode.
Availability: Macintosh, \UNIX.

FIFOs are pipes that can be accessed like regular files.  FIFOs exist
until they are deleted (for example with \function{os.unlink()}).
Generally, FIFOs are used as rendezvous between ``client'' and
``server'' type processes: the server opens the FIFO for reading, and
the client opens it for writing.  Note that \function{mkfifo()}
doesn't open the FIFO --- it just creates the rendezvous point.
\end{funcdesc}

\begin{funcdesc}{mknod}{filename\optional{, mode=0600, device}}
Create a filesystem node (file, device special file or named pipe)
named \var{filename}. \var{mode} specifies both the permissions to use and
the type of node to be created, being combined (bitwise OR) with one
of S_IFREG, S_IFCHR, S_IFBLK, and S_IFIFO (those constants are
available in \module{stat}). For S_IFCHR and S_IFBLK, \var{device}
defines the newly created device special file (probably using
\function{os.makedev()}), otherwise it is ignored.
\versionadded{2.3}
\end{funcdesc}

\begin{funcdesc}{major}{device}
Extracts the device major number from a raw device number (usually
the \member{st_dev} or \member{st_rdev} field from \ctype{stat}).
\versionadded{2.3}
\end{funcdesc}

\begin{funcdesc}{minor}{device}
Extracts the device minor number from a raw device number (usually
the \member{st_dev} or \member{st_rdev} field from \ctype{stat}).
\versionadded{2.3}
\end{funcdesc}

\begin{funcdesc}{makedev}{major, minor}
Composes a raw device number from the major and minor device numbers.
\versionadded{2.3}
\end{funcdesc}

\begin{funcdesc}{mkdir}{path\optional{, mode}}
Create a directory named \var{path} with numeric mode \var{mode}.
The default \var{mode} is \code{0777} (octal).  On some systems,
\var{mode} is ignored.  Where it is used, the current umask value is
first masked out.
Availability: Macintosh, \UNIX, Windows.
\end{funcdesc}

\begin{funcdesc}{makedirs}{path\optional{, mode}}
Recursive directory creation function.\index{directory!creating}
\index{UNC paths!and \function{os.makedirs()}}
Like \function{mkdir()},
but makes all intermediate-level directories needed to contain the
leaf directory.  Throws an \exception{error} exception if the leaf
directory already exists or cannot be created.  The default \var{mode}
is \code{0777} (octal).  On some systems, \var{mode} is ignored.
Where it is used, the current umask value is first masked out.
\note{\function{makedirs()} will become confused if the path elements
to create include \var{os.pardir}.}
\versionadded{1.5.2}
\versionchanged[This function now handles UNC paths correctly]{2.3}
\end{funcdesc}

\begin{funcdesc}{pathconf}{path, name}
Return system configuration information relevant to a named file.
\var{name} specifies the configuration value to retrieve; it may be a
string which is the name of a defined system value; these names are
specified in a number of standards (\POSIX.1, \UNIX{} 95, \UNIX{} 98, and
others).  Some platforms define additional names as well.  The names
known to the host operating system are given in the
\code{pathconf_names} dictionary.  For configuration variables not
included in that mapping, passing an integer for \var{name} is also
accepted.
Availability: Macintosh, \UNIX.

If \var{name} is a string and is not known, \exception{ValueError} is
raised.  If a specific value for \var{name} is not supported by the
host system, even if it is included in \code{pathconf_names}, an
\exception{OSError} is raised with \constant{errno.EINVAL} for the
error number.
\end{funcdesc}

\begin{datadesc}{pathconf_names}
Dictionary mapping names accepted by \function{pathconf()} and
\function{fpathconf()} to the integer values defined for those names
by the host operating system.  This can be used to determine the set
of names known to the system.
Availability: Macintosh, \UNIX.
\end{datadesc}

\begin{funcdesc}{readlink}{path}
Return a string representing the path to which the symbolic link
points.  The result may be either an absolute or relative pathname; if
it is relative, it may be converted to an absolute pathname using
\code{os.path.join(os.path.dirname(\var{path}), \var{result})}.
Availability: Macintosh, \UNIX.
\end{funcdesc}

\begin{funcdesc}{remove}{path}
Remove the file \var{path}.  If \var{path} is a directory,
\exception{OSError} is raised; see \function{rmdir()} below to remove
a directory.  This is identical to the \function{unlink()} function
documented below.  On Windows, attempting to remove a file that is in
use causes an exception to be raised; on \UNIX, the directory entry is
removed but the storage allocated to the file is not made available
until the original file is no longer in use.
Availability: Macintosh, \UNIX, Windows.
\end{funcdesc}

\begin{funcdesc}{removedirs}{path}
\index{directory!deleting}
Removes directories recursively.  Works like
\function{rmdir()} except that, if the leaf directory is
successfully removed, \function{removedirs()} 
tries to successively remove every parent directory mentioned in 
\var{path} until an error is raised (which is ignored, because
it generally means that a parent directory is not empty).
For example, \samp{os.removedirs('foo/bar/baz')} will first remove
the directory \samp{'foo/bar/baz'}, and then remove \samp{'foo/bar'}
and \samp{'foo'} if they are empty.
Raises \exception{OSError} if the leaf directory could not be 
successfully removed.
\versionadded{1.5.2}
\end{funcdesc}

\begin{funcdesc}{rename}{src, dst}
Rename the file or directory \var{src} to \var{dst}.  If \var{dst} is
a directory, \exception{OSError} will be raised.  On \UNIX, if
\var{dst} exists and is a file, it will be removed silently if the
user has permission.  The operation may fail on some \UNIX{} flavors
if \var{src} and \var{dst} are on different filesystems.  If
successful, the renaming will be an atomic operation (this is a
\POSIX{} requirement).  On Windows, if \var{dst} already exists,
\exception{OSError} will be raised even if it is a file; there may be
no way to implement an atomic rename when \var{dst} names an existing
file.
Availability: Macintosh, \UNIX, Windows.
\end{funcdesc}

\begin{funcdesc}{renames}{old, new}
Recursive directory or file renaming function.
Works like \function{rename()}, except creation of any intermediate
directories needed to make the new pathname good is attempted first.
After the rename, directories corresponding to rightmost path segments
of the old name will be pruned away using \function{removedirs()}.
\versionadded{1.5.2}

\begin{notice}
This function can fail with the new directory structure made if
you lack permissions needed to remove the leaf directory or file.
\end{notice}
\end{funcdesc}

\begin{funcdesc}{rmdir}{path}
Remove the directory \var{path}.
Availability: Macintosh, \UNIX, Windows.
\end{funcdesc}

\begin{funcdesc}{stat}{path}
Perform a \cfunction{stat()} system call on the given path.  The
return value is an object whose attributes correspond to the members of
the \ctype{stat} structure, namely:
\member{st_mode} (protection bits),
\member{st_ino} (inode number),
\member{st_dev} (device),
\member{st_nlink} (number of hard links),
\member{st_uid} (user ID of owner),
\member{st_gid} (group ID of owner),
\member{st_size} (size of file, in bytes),
\member{st_atime} (time of most recent access),
\member{st_mtime} (time of most recent content modification),
\member{st_ctime}
(platform dependent; time of most recent metadata change on \UNIX, or
the time of creation on Windows):

\begin{verbatim}
>>> import os
>>> statinfo = os.stat('somefile.txt')
>>> statinfo
(33188, 422511L, 769L, 1, 1032, 100, 926L, 1105022698,1105022732, 1105022732)
>>> statinfo.st_size
926L
>>>
\end{verbatim}

\versionchanged [If \function{stat_float_times} returns true, the time
values are floats, measuring seconds. Fractions of a second may be
reported if the system supports that. On Mac OS, the times are always
floats. See \function{stat_float_times} for further discussion. ]{2.3}

On some \UNIX{} systems (such as Linux), the following attributes may
also be available:
\member{st_blocks} (number of blocks allocated for file),
\member{st_blksize} (filesystem blocksize),
\member{st_rdev} (type of device if an inode device).
\member{st_flags} (user defined flags for file).

On other \UNIX{} systems (such as FreeBSD), the following attributes
may be available (but may be only filled out of root tries to
use them:
\member{st_gen} (file generation number),
\member{st_birthtime} (time of file creation).

On Mac OS systems, the following attributes may also be available:
\member{st_rsize},
\member{st_creator},
\member{st_type}.

On RISCOS systems, the following attributes are also available:
\member{st_ftype} (file type),
\member{st_attrs} (attributes),
\member{st_obtype} (object type).

For backward compatibility, the return value of \function{stat()} is
also accessible as a tuple of at least 10 integers giving the most
important (and portable) members of the \ctype{stat} structure, in the
order
\member{st_mode},
\member{st_ino},
\member{st_dev},
\member{st_nlink},
\member{st_uid},
\member{st_gid},
\member{st_size},
\member{st_atime},
\member{st_mtime},
\member{st_ctime}.
More items may be added at the end by some implementations.
The standard module \refmodule{stat}\refstmodindex{stat} defines
functions and constants that are useful for extracting information
from a \ctype{stat} structure.
(On Windows, some items are filled with dummy values.)

\note{The exact meaning and resolution of the \member{st_atime},
 \member{st_mtime}, and \member{st_ctime} members depends on the
 operating system and the file system.  For example, on Windows systems
 using the FAT or FAT32 file systems, \member{st_mtime} has 2-second
 resolution, and \member{st_atime} has only 1-day resolution.  See
 your operating system documentation for details.}

Availability: Macintosh, \UNIX, Windows.

\versionchanged
[Added access to values as attributes of the returned object]{2.2}
\versionchanged[Added st_gen, st_birthtime]{2.5}
\end{funcdesc}

\begin{funcdesc}{stat_float_times}{\optional{newvalue}}
Determine whether \class{stat_result} represents time stamps as float
objects.  If newval is True, future calls to stat() return floats, if
it is False, future calls return ints.  If newval is omitted, return
the current setting.

For compatibility with older Python versions, accessing
\class{stat_result} as a tuple always returns integers.

\versionchanged[Python now returns float values by default. Applications
which do not work correctly with floating point time stamps can use
this function to restore the old behaviour]{2.5}

The resolution of the timestamps (i.e. the smallest possible fraction)
depends on the system. Some systems only support second resolution;
on these systems, the fraction will always be zero.

It is recommended that this setting is only changed at program startup
time in the \var{__main__} module; libraries should never change this
setting. If an application uses a library that works incorrectly if
floating point time stamps are processed, this application should turn
the feature off until the library has been corrected.

\end{funcdesc}

\begin{funcdesc}{statvfs}{path}
Perform a \cfunction{statvfs()} system call on the given path.  The
return value is an object whose attributes describe the filesystem on
the given path, and correspond to the members of the
\ctype{statvfs} structure, namely:
\member{f_frsize},
\member{f_blocks},
\member{f_bfree},
\member{f_bavail},
\member{f_files},
\member{f_ffree},
\member{f_favail},
\member{f_flag},
\member{f_namemax}.
Availability: \UNIX.

For backward compatibility, the return value is also accessible as a
tuple whose values correspond to the attributes, in the order given above.
The standard module \refmodule{statvfs}\refstmodindex{statvfs}
defines constants that are useful for extracting information
from a \ctype{statvfs} structure when accessing it as a sequence; this
remains useful when writing code that needs to work with versions of
Python that don't support accessing the fields as attributes.

\versionchanged
[Added access to values as attributes of the returned object]{2.2}
\end{funcdesc}

\begin{funcdesc}{symlink}{src, dst}
Create a symbolic link pointing to \var{src} named \var{dst}.
Availability: \UNIX.
\end{funcdesc}

\begin{funcdesc}{tempnam}{\optional{dir\optional{, prefix}}}
Return a unique path name that is reasonable for creating a temporary
file.  This will be an absolute path that names a potential directory
entry in the directory \var{dir} or a common location for temporary
files if \var{dir} is omitted or \code{None}.  If given and not
\code{None}, \var{prefix} is used to provide a short prefix to the
filename.  Applications are responsible for properly creating and
managing files created using paths returned by \function{tempnam()};
no automatic cleanup is provided.
On \UNIX, the environment variable \envvar{TMPDIR} overrides
\var{dir}, while on Windows the \envvar{TMP} is used.  The specific
behavior of this function depends on the C library implementation;
some aspects are underspecified in system documentation.
\warning{Use of \function{tempnam()} is vulnerable to symlink attacks;
consider using \function{tmpfile()} (section \ref{os-newstreams})
instead.}  Availability: Macintosh, \UNIX, Windows.
\end{funcdesc}

\begin{funcdesc}{tmpnam}{}
Return a unique path name that is reasonable for creating a temporary
file.  This will be an absolute path that names a potential directory
entry in a common location for temporary files.  Applications are
responsible for properly creating and managing files created using
paths returned by \function{tmpnam()}; no automatic cleanup is
provided.
\warning{Use of \function{tmpnam()} is vulnerable to symlink attacks;
consider using \function{tmpfile()} (section \ref{os-newstreams})
instead.}  Availability: \UNIX, Windows.  This function probably
shouldn't be used on Windows, though: Microsoft's implementation of
\function{tmpnam()} always creates a name in the root directory of the
current drive, and that's generally a poor location for a temp file
(depending on privileges, you may not even be able to open a file
using this name).
\end{funcdesc}

\begin{datadesc}{TMP_MAX}
The maximum number of unique names that \function{tmpnam()} will
generate before reusing names.
\end{datadesc}

\begin{funcdesc}{unlink}{path}
Remove the file \var{path}.  This is the same function as
\function{remove()}; the \function{unlink()} name is its traditional
\UNIX{} name.
Availability: Macintosh, \UNIX, Windows.
\end{funcdesc}

\begin{funcdesc}{utime}{path, times}
Set the access and modified times of the file specified by \var{path}.
If \var{times} is \code{None}, then the file's access and modified
times are set to the current time.  Otherwise, \var{times} must be a
2-tuple of numbers, of the form \code{(\var{atime}, \var{mtime})}
which is used to set the access and modified times, respectively.
Whether a directory can be given for \var{path} depends on whether the
operating system implements directories as files (for example, Windows
does not).  Note that the exact times you set here may not be returned
by a subsequent \function{stat()} call, depending on the resolution
with which your operating system records access and modification times;
see \function{stat()}.
\versionchanged[Added support for \code{None} for \var{times}]{2.0}
Availability: Macintosh, \UNIX, Windows.
\end{funcdesc}

\begin{funcdesc}{walk}{top\optional{, topdown\code{=True}
                       \optional{, onerror\code{=None}}}}
\index{directory!walking}
\index{directory!traversal}
\function{walk()} generates the file names in a directory tree, by
walking the tree either top down or bottom up.
For each directory in the tree rooted at directory \var{top} (including
\var{top} itself), it yields a 3-tuple
\code{(\var{dirpath}, \var{dirnames}, \var{filenames})}.

\var{dirpath} is a string, the path to the directory.  \var{dirnames} is
a list of the names of the subdirectories in \var{dirpath}
(excluding \code{'.'} and \code{'..'}).  \var{filenames} is a list of
the names of the non-directory files in \var{dirpath}.  Note that the
names in the lists contain no path components.  To get a full
path (which begins with \var{top}) to a file or directory in
\var{dirpath}, do \code{os.path.join(\var{dirpath}, \var{name})}.

If optional argument \var{topdown} is true or not specified, the triple
for a directory is generated before the triples for any of its
subdirectories (directories are generated top down).  If \var{topdown} is
false, the triple for a directory is generated after the triples for all
of its subdirectories (directories are generated bottom up).

When \var{topdown} is true, the caller can modify the \var{dirnames} list
in-place (perhaps using \keyword{del} or slice assignment), and
\function{walk()} will only recurse into the subdirectories whose names
remain in \var{dirnames}; this can be used to prune the search,
impose a specific order of visiting, or even to inform \function{walk()}
about directories the caller creates or renames before it resumes
\function{walk()} again.  Modifying \var{dirnames} when \var{topdown} is
false is ineffective, because in bottom-up mode the directories in
\var{dirnames} are generated before \var{dirnames} itself is generated.

By default errors from the \code{os.listdir()} call are ignored.  If
optional argument \var{onerror} is specified, it should be a function;
it will be called with one argument, an os.error instance.  It can
report the error to continue with the walk, or raise the exception
to abort the walk.  Note that the filename is available as the
\code{filename} attribute of the exception object.

\begin{notice}
If you pass a relative pathname, don't change the current working
directory between resumptions of \function{walk()}.  \function{walk()}
never changes the current directory, and assumes that its caller
doesn't either.
\end{notice}

\begin{notice}
On systems that support symbolic links, links to subdirectories appear
in \var{dirnames} lists, but \function{walk()} will not visit them
(infinite loops are hard to avoid when following symbolic links).
To visit linked directories, you can identify them with
\code{os.path.islink(\var{path})}, and invoke \code{walk(\var{path})}
on each directly.
\end{notice}

This example displays the number of bytes taken by non-directory files
in each directory under the starting directory, except that it doesn't
look under any CVS subdirectory:

\begin{verbatim}
import os
from os.path import join, getsize
for root, dirs, files in os.walk('python/Lib/email'):
    print root, "consumes",
    print sum(getsize(join(root, name)) for name in files),
    print "bytes in", len(files), "non-directory files"
    if 'CVS' in dirs:
        dirs.remove('CVS')  # don't visit CVS directories
\end{verbatim}

In the next example, walking the tree bottom up is essential:
\function{rmdir()} doesn't allow deleting a directory before the
directory is empty:

\begin{verbatim}
# Delete everything reachable from the directory named in 'top',
# assuming there are no symbolic links.
# CAUTION:  This is dangerous!  For example, if top == '/', it
# could delete all your disk files.
import os
for root, dirs, files in os.walk(top, topdown=False):
    for name in files:
        os.remove(os.path.join(root, name))
    for name in dirs:
        os.rmdir(os.path.join(root, name))
\end{verbatim}

\versionadded{2.3}
\end{funcdesc}

\subsection{Process Management \label{os-process}}

These functions may be used to create and manage processes.

The various \function{exec*()} functions take a list of arguments for
the new program loaded into the process.  In each case, the first of
these arguments is passed to the new program as its own name rather
than as an argument a user may have typed on a command line.  For the
C programmer, this is the \code{argv[0]} passed to a program's
\cfunction{main()}.  For example, \samp{os.execv('/bin/echo', ['foo',
'bar'])} will only print \samp{bar} on standard output; \samp{foo}
will seem to be ignored.


\begin{funcdesc}{abort}{}
Generate a \constant{SIGABRT} signal to the current process.  On
\UNIX, the default behavior is to produce a core dump; on Windows, the
process immediately returns an exit code of \code{3}.  Be aware that
programs which use \function{signal.signal()} to register a handler
for \constant{SIGABRT} will behave differently.
Availability: Macintosh, \UNIX, Windows.
\end{funcdesc}

\begin{funcdesc}{execl}{path, arg0, arg1, \moreargs}
\funcline{execle}{path, arg0, arg1, \moreargs, env}
\funcline{execlp}{file, arg0, arg1, \moreargs}
\funcline{execlpe}{file, arg0, arg1, \moreargs, env}
\funcline{execv}{path, args}
\funcline{execve}{path, args, env}
\funcline{execvp}{file, args}
\funcline{execvpe}{file, args, env}
These functions all execute a new program, replacing the current
process; they do not return.  On \UNIX, the new executable is loaded
into the current process, and will have the same process ID as the
caller.  Errors will be reported as \exception{OSError} exceptions.

The \character{l} and \character{v} variants of the
\function{exec*()} functions differ in how command-line arguments are
passed.  The \character{l} variants are perhaps the easiest to work
with if the number of parameters is fixed when the code is written;
the individual parameters simply become additional parameters to the
\function{execl*()} functions.  The \character{v} variants are good
when the number of parameters is variable, with the arguments being
passed in a list or tuple as the \var{args} parameter.  In either
case, the arguments to the child process should start with the name of
the command being run, but this is not enforced.

The variants which include a \character{p} near the end
(\function{execlp()}, \function{execlpe()}, \function{execvp()},
and \function{execvpe()}) will use the \envvar{PATH} environment
variable to locate the program \var{file}.  When the environment is
being replaced (using one of the \function{exec*e()} variants,
discussed in the next paragraph), the
new environment is used as the source of the \envvar{PATH} variable.
The other variants, \function{execl()}, \function{execle()},
\function{execv()}, and \function{execve()}, will not use the
\envvar{PATH} variable to locate the executable; \var{path} must
contain an appropriate absolute or relative path.

For \function{execle()}, \function{execlpe()}, \function{execve()},
and \function{execvpe()} (note that these all end in \character{e}),
the \var{env} parameter must be a mapping which is used to define the
environment variables for the new process; the \function{execl()},
\function{execlp()}, \function{execv()}, and \function{execvp()}
all cause the new process to inherit the environment of the current
process.
Availability: Macintosh, \UNIX, Windows.
\end{funcdesc}

\begin{funcdesc}{_exit}{n}
Exit to the system with status \var{n}, without calling cleanup
handlers, flushing stdio buffers, etc.
Availability: Macintosh, \UNIX, Windows.

\begin{notice}
The standard way to exit is \code{sys.exit(\var{n})}.
\function{_exit()} should normally only be used in the child process
after a \function{fork()}.
\end{notice}
\end{funcdesc}

The following exit codes are a defined, and can be used with
\function{_exit()}, although they are not required.  These are
typically used for system programs written in Python, such as a
mail server's external command delivery program.
\note{Some of these may not be available on all \UNIX{} platforms,
since there is some variation.  These constants are defined where they
are defined by the underlying platform.}

\begin{datadesc}{EX_OK}
Exit code that means no error occurred.
Availability: Macintosh, \UNIX.
\versionadded{2.3}
\end{datadesc}

\begin{datadesc}{EX_USAGE}
Exit code that means the command was used incorrectly, such as when
the wrong number of arguments are given.
Availability: Macintosh, \UNIX.
\versionadded{2.3}
\end{datadesc}

\begin{datadesc}{EX_DATAERR}
Exit code that means the input data was incorrect.
Availability: Macintosh, \UNIX.
\versionadded{2.3}
\end{datadesc}

\begin{datadesc}{EX_NOINPUT}
Exit code that means an input file did not exist or was not readable.
Availability: Macintosh, \UNIX.
\versionadded{2.3}
\end{datadesc}

\begin{datadesc}{EX_NOUSER}
Exit code that means a specified user did not exist.
Availability: Macintosh, \UNIX.
\versionadded{2.3}
\end{datadesc}

\begin{datadesc}{EX_NOHOST}
Exit code that means a specified host did not exist.
Availability: Macintosh, \UNIX.
\versionadded{2.3}
\end{datadesc}

\begin{datadesc}{EX_UNAVAILABLE}
Exit code that means that a required service is unavailable.
Availability: Macintosh, \UNIX.
\versionadded{2.3}
\end{datadesc}

\begin{datadesc}{EX_SOFTWARE}
Exit code that means an internal software error was detected.
Availability: Macintosh, \UNIX.
\versionadded{2.3}
\end{datadesc}

\begin{datadesc}{EX_OSERR}
Exit code that means an operating system error was detected, such as
the inability to fork or create a pipe.
Availability: Macintosh, \UNIX.
\versionadded{2.3}
\end{datadesc}

\begin{datadesc}{EX_OSFILE}
Exit code that means some system file did not exist, could not be
opened, or had some other kind of error.
Availability: Macintosh, \UNIX.
\versionadded{2.3}
\end{datadesc}

\begin{datadesc}{EX_CANTCREAT}
Exit code that means a user specified output file could not be created.
Availability: Macintosh, \UNIX.
\versionadded{2.3}
\end{datadesc}

\begin{datadesc}{EX_IOERR}
Exit code that means that an error occurred while doing I/O on some file.
Availability: Macintosh, \UNIX.
\versionadded{2.3}
\end{datadesc}

\begin{datadesc}{EX_TEMPFAIL}
Exit code that means a temporary failure occurred.  This indicates
something that may not really be an error, such as a network
connection that couldn't be made during a retryable operation.
Availability: Macintosh, \UNIX.
\versionadded{2.3}
\end{datadesc}

\begin{datadesc}{EX_PROTOCOL}
Exit code that means that a protocol exchange was illegal, invalid, or
not understood.
Availability: Macintosh, \UNIX.
\versionadded{2.3}
\end{datadesc}

\begin{datadesc}{EX_NOPERM}
Exit code that means that there were insufficient permissions to
perform the operation (but not intended for file system problems).
Availability: Macintosh, \UNIX.
\versionadded{2.3}
\end{datadesc}

\begin{datadesc}{EX_CONFIG}
Exit code that means that some kind of configuration error occurred.
Availability: Macintosh, \UNIX.
\versionadded{2.3}
\end{datadesc}

\begin{datadesc}{EX_NOTFOUND}
Exit code that means something like ``an entry was not found''.
Availability: Macintosh, \UNIX.
\versionadded{2.3}
\end{datadesc}

\begin{funcdesc}{fork}{}
Fork a child process.  Return \code{0} in the child, the child's
process id in the parent.
Availability: Macintosh, \UNIX.
\end{funcdesc}

\begin{funcdesc}{forkpty}{}
Fork a child process, using a new pseudo-terminal as the child's
controlling terminal. Return a pair of \code{(\var{pid}, \var{fd})},
where \var{pid} is \code{0} in the child, the new child's process id
in the parent, and \var{fd} is the file descriptor of the master end
of the pseudo-terminal.  For a more portable approach, use the
\refmodule{pty} module.
Availability: Macintosh, Some flavors of \UNIX.
\end{funcdesc}

\begin{funcdesc}{kill}{pid, sig}
\index{process!killing}
\index{process!signalling}
Send signal \var{sig} to the process \var{pid}.  Constants for the
specific signals available on the host platform are defined in the
\refmodule{signal} module.
Availability: Macintosh, \UNIX.
\end{funcdesc}

\begin{funcdesc}{killpg}{pgid, sig}
\index{process!killing}
\index{process!signalling}
Send the signal \var{sig} to the process group \var{pgid}.
Availability: Macintosh, \UNIX.
\versionadded{2.3}
\end{funcdesc}

\begin{funcdesc}{nice}{increment}
Add \var{increment} to the process's ``niceness''.  Return the new
niceness.
Availability: Macintosh, \UNIX.
\end{funcdesc}

\begin{funcdesc}{plock}{op}
Lock program segments into memory.  The value of \var{op}
(defined in \code{<sys/lock.h>}) determines which segments are locked.
Availability: Macintosh, \UNIX.
\end{funcdesc}

\begin{funcdescni}{popen}{\unspecified}
\funclineni{popen2}{\unspecified}
\funclineni{popen3}{\unspecified}
\funclineni{popen4}{\unspecified}
Run child processes, returning opened pipes for communications.  These
functions are described in section \ref{os-newstreams}.
\end{funcdescni}

\begin{funcdesc}{spawnl}{mode, path, \moreargs}
\funcline{spawnle}{mode, path, \moreargs, env}
\funcline{spawnlp}{mode, file, \moreargs}
\funcline{spawnlpe}{mode, file, \moreargs, env}
\funcline{spawnv}{mode, path, args}
\funcline{spawnve}{mode, path, args, env}
\funcline{spawnvp}{mode, file, args}
\funcline{spawnvpe}{mode, file, args, env}
Execute the program \var{path} in a new process.  If \var{mode} is
\constant{P_NOWAIT}, this function returns the process ID of the new
process; if \var{mode} is \constant{P_WAIT}, returns the process's
exit code if it exits normally, or \code{-\var{signal}}, where
\var{signal} is the signal that killed the process.  On Windows, the
process ID will actually be the process handle, so can be used with
the \function{waitpid()} function.

The \character{l} and \character{v} variants of the
\function{spawn*()} functions differ in how command-line arguments are
passed.  The \character{l} variants are perhaps the easiest to work
with if the number of parameters is fixed when the code is written;
the individual parameters simply become additional parameters to the
\function{spawnl*()} functions.  The \character{v} variants are good
when the number of parameters is variable, with the arguments being
passed in a list or tuple as the \var{args} parameter.  In either
case, the arguments to the child process must start with the name of
the command being run.

The variants which include a second \character{p} near the end
(\function{spawnlp()}, \function{spawnlpe()}, \function{spawnvp()},
and \function{spawnvpe()}) will use the \envvar{PATH} environment
variable to locate the program \var{file}.  When the environment is
being replaced (using one of the \function{spawn*e()} variants,
discussed in the next paragraph), the new environment is used as the
source of the \envvar{PATH} variable.  The other variants,
\function{spawnl()}, \function{spawnle()}, \function{spawnv()}, and
\function{spawnve()}, will not use the \envvar{PATH} variable to
locate the executable; \var{path} must contain an appropriate absolute
or relative path.

For \function{spawnle()}, \function{spawnlpe()}, \function{spawnve()},
and \function{spawnvpe()} (note that these all end in \character{e}),
the \var{env} parameter must be a mapping which is used to define the
environment variables for the new process; the \function{spawnl()},
\function{spawnlp()}, \function{spawnv()}, and \function{spawnvp()}
all cause the new process to inherit the environment of the current
process.

As an example, the following calls to \function{spawnlp()} and
\function{spawnvpe()} are equivalent:

\begin{verbatim}
import os
os.spawnlp(os.P_WAIT, 'cp', 'cp', 'index.html', '/dev/null')

L = ['cp', 'index.html', '/dev/null']
os.spawnvpe(os.P_WAIT, 'cp', L, os.environ)
\end{verbatim}

Availability: \UNIX, Windows.  \function{spawnlp()},
\function{spawnlpe()}, \function{spawnvp()} and \function{spawnvpe()}
are not available on Windows.
\versionadded{1.6}
\end{funcdesc}

\begin{datadesc}{P_NOWAIT}
\dataline{P_NOWAITO}
Possible values for the \var{mode} parameter to the \function{spawn*()}
family of functions.  If either of these values is given, the
\function{spawn*()} functions will return as soon as the new process
has been created, with the process ID as the return value.
Availability: Macintosh, \UNIX, Windows.
\versionadded{1.6}
\end{datadesc}

\begin{datadesc}{P_WAIT}
Possible value for the \var{mode} parameter to the \function{spawn*()}
family of functions.  If this is given as \var{mode}, the
\function{spawn*()} functions will not return until the new process
has run to completion and will return the exit code of the process the
run is successful, or \code{-\var{signal}} if a signal kills the
process.
Availability: Macintosh, \UNIX, Windows.
\versionadded{1.6}
\end{datadesc}

\begin{datadesc}{P_DETACH}
\dataline{P_OVERLAY}
Possible values for the \var{mode} parameter to the
\function{spawn*()} family of functions.  These are less portable than
those listed above.
\constant{P_DETACH} is similar to \constant{P_NOWAIT}, but the new
process is detached from the console of the calling process.
If \constant{P_OVERLAY} is used, the current process will be replaced;
the \function{spawn*()} function will not return.
Availability: Windows.
\versionadded{1.6}
\end{datadesc}

\begin{funcdesc}{startfile}{path}
Start a file with its associated application.  This acts like
double-clicking the file in Windows Explorer, or giving the file name
as an argument to the \program{start} command from the interactive
command shell: the file is opened with whatever application (if any)
its extension is associated.

\function{startfile()} returns as soon as the associated application
is launched.  There is no option to wait for the application to close,
and no way to retrieve the application's exit status.  The \var{path}
parameter is relative to the current directory.  If you want to use an
absolute path, make sure the first character is not a slash
(\character{/}); the underlying Win32 \cfunction{ShellExecute()}
function doesn't work if it is.  Use the \function{os.path.normpath()}
function to ensure that the path is properly encoded for Win32.
Availability: Windows.
\versionadded{2.0}
\end{funcdesc}

\begin{funcdesc}{system}{command}
Execute the command (a string) in a subshell.  This is implemented by
calling the Standard C function \cfunction{system()}, and has the
same limitations.  Changes to \code{posix.environ}, \code{sys.stdin},
etc.\ are not reflected in the environment of the executed command.

On \UNIX, the return value is the exit status of the process encoded in the
format specified for \function{wait()}.  Note that \POSIX{} does not
specify the meaning of the return value of the C \cfunction{system()}
function, so the return value of the Python function is system-dependent.

On Windows, the return value is that returned by the system shell after
running \var{command}, given by the Windows environment variable
\envvar{COMSPEC}: on \program{command.com} systems (Windows 95, 98 and ME)
this is always \code{0}; on \program{cmd.exe} systems (Windows NT, 2000
and XP) this is the exit status of the command run; on systems using
a non-native shell, consult your shell documentation.

Availability: Macintosh, \UNIX, Windows.
\end{funcdesc}

\begin{funcdesc}{times}{}
Return a 5-tuple of floating point numbers indicating accumulated
(processor or other)
times, in seconds.  The items are: user time, system time, children's
user time, children's system time, and elapsed real time since a fixed
point in the past, in that order.  See the \UNIX{} manual page
\manpage{times}{2} or the corresponding Windows Platform API
documentation.
Availability: Macintosh, \UNIX, Windows.
\end{funcdesc}

\begin{funcdesc}{wait}{}
Wait for completion of a child process, and return a tuple containing
its pid and exit status indication: a 16-bit number, whose low byte is
the signal number that killed the process, and whose high byte is the
exit status (if the signal number is zero); the high bit of the low
byte is set if a core file was produced.
Availability: Macintosh, \UNIX.
\end{funcdesc}

\begin{funcdesc}{waitpid}{pid, options}
The details of this function differ on \UNIX{} and Windows.

On \UNIX:
Wait for completion of a child process given by process id \var{pid},
and return a tuple containing its process id and exit status
indication (encoded as for \function{wait()}).  The semantics of the
call are affected by the value of the integer \var{options}, which
should be \code{0} for normal operation.

If \var{pid} is greater than \code{0}, \function{waitpid()} requests
status information for that specific process.  If \var{pid} is
\code{0}, the request is for the status of any child in the process
group of the current process.  If \var{pid} is \code{-1}, the request
pertains to any child of the current process.  If \var{pid} is less
than \code{-1}, status is requested for any process in the process
group \code{-\var{pid}} (the absolute value of \var{pid}).

On Windows:
Wait for completion of a process given by process handle \var{pid},
and return a tuple containing \var{pid},
and its exit status shifted left by 8 bits (shifting makes cross-platform
use of the function easier).
A \var{pid} less than or equal to \code{0} has no special meaning on
Windows, and raises an exception.
The value of integer \var{options} has no effect.
\var{pid} can refer to any process whose id is known, not necessarily a
child process.
The \function{spawn()} functions called with \constant{P_NOWAIT}
return suitable process handles.
\end{funcdesc}

\begin{datadesc}{WNOHANG}
The option for \function{waitpid()} to return immediately if no child
process status is available immediately. The function returns
\code{(0, 0)} in this case.
Availability: Macintosh, \UNIX.
\end{datadesc}

\begin{datadesc}{WCONTINUED}
This option causes child processes to be reported if they have been
continued from a job control stop since their status was last
reported.
Availability: Some \UNIX{} systems.
\versionadded{2.3}
\end{datadesc}

\begin{datadesc}{WUNTRACED}
This option causes child processes to be reported if they have been
stopped but their current state has not been reported since they were
stopped.
Availability: Macintosh, \UNIX.
\versionadded{2.3}
\end{datadesc}

The following functions take a process status code as returned by
\function{system()}, \function{wait()}, or \function{waitpid()} as a
parameter.  They may be used to determine the disposition of a
process.

\begin{funcdesc}{WCOREDUMP}{status}
Returns \code{True} if a core dump was generated for the process,
otherwise it returns \code{False}.
Availability: Macintosh, \UNIX.
\versionadded{2.3}
\end{funcdesc}

\begin{funcdesc}{WIFCONTINUED}{status}
Returns \code{True} if the process has been continued from a job
control stop, otherwise it returns \code{False}.
Availability: \UNIX.
\versionadded{2.3}
\end{funcdesc}

\begin{funcdesc}{WIFSTOPPED}{status}
Returns \code{True} if the process has been stopped, otherwise it
returns \code{False}.
Availability: \UNIX.
\end{funcdesc}

\begin{funcdesc}{WIFSIGNALED}{status}
Returns \code{True} if the process exited due to a signal, otherwise
it returns \code{False}.
Availability: Macintosh, \UNIX.
\end{funcdesc}

\begin{funcdesc}{WIFEXITED}{status}
Returns \code{True} if the process exited using the \manpage{exit}{2}
system call, otherwise it returns \code{False}.
Availability: Macintosh, \UNIX.
\end{funcdesc}

\begin{funcdesc}{WEXITSTATUS}{status}
If \code{WIFEXITED(\var{status})} is true, return the integer
parameter to the \manpage{exit}{2} system call.  Otherwise, the return
value is meaningless.
Availability: Macintosh, \UNIX.
\end{funcdesc}

\begin{funcdesc}{WSTOPSIG}{status}
Return the signal which caused the process to stop.
Availability: Macintosh, \UNIX.
\end{funcdesc}

\begin{funcdesc}{WTERMSIG}{status}
Return the signal which caused the process to exit.
Availability: Macintosh, \UNIX.
\end{funcdesc}


\subsection{Miscellaneous System Information \label{os-path}}


\begin{funcdesc}{confstr}{name}
Return string-valued system configuration values.
\var{name} specifies the configuration value to retrieve; it may be a
string which is the name of a defined system value; these names are
specified in a number of standards (\POSIX, \UNIX{} 95, \UNIX{} 98, and
others).  Some platforms define additional names as well.  The names
known to the host operating system are given in the
\code{confstr_names} dictionary.  For configuration variables not
included in that mapping, passing an integer for \var{name} is also
accepted.
Availability: Macintosh, \UNIX.

If the configuration value specified by \var{name} isn't defined, the
empty string is returned.

If \var{name} is a string and is not known, \exception{ValueError} is
raised.  If a specific value for \var{name} is not supported by the
host system, even if it is included in \code{confstr_names}, an
\exception{OSError} is raised with \constant{errno.EINVAL} for the
error number.
\end{funcdesc}

\begin{datadesc}{confstr_names}
Dictionary mapping names accepted by \function{confstr()} to the
integer values defined for those names by the host operating system.
This can be used to determine the set of names known to the system.
Availability: Macintosh, \UNIX.
\end{datadesc}

\begin{funcdesc}{getloadavg}{}
Return the number of processes in the system run queue averaged over
the last 1, 5, and 15 minutes or raises OSError if the load average
was unobtainable.

\versionadded{2.3}
\end{funcdesc}

\begin{funcdesc}{sysconf}{name}
Return integer-valued system configuration values.
If the configuration value specified by \var{name} isn't defined,
\code{-1} is returned.  The comments regarding the \var{name}
parameter for \function{confstr()} apply here as well; the dictionary
that provides information on the known names is given by
\code{sysconf_names}.
Availability: Macintosh, \UNIX.
\end{funcdesc}

\begin{datadesc}{sysconf_names}
Dictionary mapping names accepted by \function{sysconf()} to the
integer values defined for those names by the host operating system.
This can be used to determine the set of names known to the system.
Availability: Macintosh, \UNIX.
\end{datadesc}


The follow data values are used to support path manipulation
operations.  These are defined for all platforms.

Higher-level operations on pathnames are defined in the
\refmodule{os.path} module.


\begin{datadesc}{curdir}
The constant string used by the operating system to refer to the current
directory.
For example: \code{'.'} for \POSIX{} or \code{':'} for Mac OS 9.
Also available via \module{os.path}.
\end{datadesc}

\begin{datadesc}{pardir}
The constant string used by the operating system to refer to the parent
directory.
For example: \code{'..'} for \POSIX{} or \code{'::'} for Mac OS 9.
Also available via \module{os.path}.
\end{datadesc}

\begin{datadesc}{sep}
The character used by the operating system to separate pathname components,
for example, \character{/} for \POSIX{} or \character{:} for
Mac OS 9.  Note that knowing this is not sufficient to be able to
parse or concatenate pathnames --- use \function{os.path.split()} and
\function{os.path.join()} --- but it is occasionally useful.
Also available via \module{os.path}.
\end{datadesc}

\begin{datadesc}{altsep}
An alternative character used by the operating system to separate pathname
components, or \code{None} if only one separator character exists.  This is
set to \character{/} on Windows systems where \code{sep} is a
backslash.
Also available via \module{os.path}.
\end{datadesc}

\begin{datadesc}{extsep}
The character which separates the base filename from the extension;
for example, the \character{.} in \file{os.py}.
Also available via \module{os.path}.
\versionadded{2.2}
\end{datadesc}

\begin{datadesc}{pathsep}
The character conventionally used by the operating system to separate
search path components (as in \envvar{PATH}), such as \character{:} for
\POSIX{} or \character{;} for Windows.
Also available via \module{os.path}.
\end{datadesc}

\begin{datadesc}{defpath}
The default search path used by \function{exec*p*()} and
\function{spawn*p*()} if the environment doesn't have a \code{'PATH'}
key.
Also available via \module{os.path}.
\end{datadesc}

\begin{datadesc}{linesep}
The string used to separate (or, rather, terminate) lines on the
current platform.  This may be a single character, such as \code{'\e
n'} for \POSIX{} or \code{'\e r'} for Mac OS, or multiple characters,
for example, \code{'\e r\e n'} for Windows.
\end{datadesc}

\begin{datadesc}{devnull}
The file path of the null device.
For example: \code{'/dev/null'} for \POSIX{} or \code{'Dev:Nul'} for
Mac OS 9.
Also available via \module{os.path}.
\versionadded{2.4}
\end{datadesc}


\subsection{Miscellaneous Functions \label{os-miscfunc}}

\begin{funcdesc}{urandom}{n}
Return a string of \var{n} random bytes suitable for cryptographic use.

This function returns random bytes from an OS-specific
randomness source.  The returned data should be unpredictable enough for
cryptographic applications, though its exact quality depends on the OS
implementation.  On a UNIX-like system this will query /dev/urandom, and
on Windows it will use CryptGenRandom.  If a randomness source is not
found, \exception{NotImplementedError} will be raised.
\versionadded{2.4}
\end{funcdesc}





\section{\module{time} ---
         Time access and conversions}

\declaremodule{builtin}{time}
\modulesynopsis{Time access and conversions.}


This module provides various time-related functions.
It is always available, but not all functions are available
on all platforms.

An explanation of some terminology and conventions is in order.

\begin{itemize}

\item
The \dfn{epoch}\index{epoch} is the point where the time starts.  On
January 1st of that year, at 0 hours, the ``time since the epoch'' is
zero.  For \UNIX{}, the epoch is 1970.  To find out what the epoch is,
look at \code{gmtime(0)}.

\item
The functions in this module do not handle dates and times before the
epoch or far in the future.  The cut-off point in the future is
determined by the C library; for \UNIX{}, it is typically in
2038\index{Year 2038}.

\item
\strong{Year 2000 (Y2K) issues}:\index{Year 2000}\index{Y2K}  Python
depends on the platform's C library, which generally doesn't have year
2000 issues, since all dates and times are represented internally as
seconds since the epoch.  Functions accepting a time tuple (see below)
generally require a 4-digit year.  For backward compatibility, 2-digit
years are supported if the module variable \code{accept2dyear} is a
non-zero integer; this variable is initialized to \code{1} unless the
environment variable \envvar{PYTHONY2K} is set to a non-empty string,
in which case it is initialized to \code{0}.  Thus, you can set
\envvar{PYTHONY2K} to a non-empty string in the environment to require 4-digit
years for all year input.  When 2-digit years are accepted, they are
converted according to the \POSIX{} or X/Open standard: values 69-99
are mapped to 1969-1999, and values 0--68 are mapped to 2000--2068.
Values 100--1899 are always illegal.  Note that this is new as of
Python 1.5.2(a2); earlier versions, up to Python 1.5.1 and 1.5.2a1,
would add 1900 to year values below 1900.

\item
UTC\index{UTC} is Coordinated Universal Time\index{Coordinated
Universal Time} (formerly known as Greenwich Mean
Time,\index{Greenwich Mean Time} or GMT).  The acronym UTC is not a
mistake but a compromise between English and French.

\item
DST is Daylight Saving Time,\index{Daylight Saving Time} an adjustment
of the timezone by (usually) one hour during part of the year.  DST
rules are magic (determined by local law) and can change from year to
year.  The C library has a table containing the local rules (often it
is read from a system file for flexibility) and is the only source of
True Wisdom in this respect.

\item
The precision of the various real-time functions may be less than
suggested by the units in which their value or argument is expressed.
E.g.\ on most \UNIX{} systems, the clock ``ticks'' only 50 or 100 times a
second, and on the Mac, times are only accurate to whole seconds.

\item
On the other hand, the precision of \function{time()} and
\function{sleep()} is better than their \UNIX{} equivalents: times are
expressed as floating point numbers, \function{time()} returns the
most accurate time available (using \UNIX{} \cfunction{gettimeofday()}
where available), and \function{sleep()} will accept a time with a
nonzero fraction (\UNIX{} \cfunction{select()} is used to implement
this, where available).

\item

The time tuple as returned by \function{gmtime()},
\function{localtime()}, and \function{strptime()}, and accepted by
\function{asctime()}, \function{mktime()} and \function{strftime()},
is a tuple of 9 integers:

\begin{tableiii}{r|l|l}{textrm}{Index}{Field}{Values}
  \lineiii{0}{year}{(e.g.\ 1993)}
  \lineiii{1}{month}{range [1,12]}
  \lineiii{2}{day}{range [1,31]}
  \lineiii{3}{hour}{range [0,23]}
  \lineiii{4}{minute}{range [0,59]}
  \lineiii{5}{second}{range [0,61]; see \strong{(1)} in \function{strftime()} description}
  \lineiii{6}{weekday}{range [0,6], monday is 0}
  \lineiii{7}{Julian day}{range [1,366]}
  \lineiii{8}{daylight savings flag}{0, 1 or -1; see below}
\end{tableiii}

Note that unlike the C structure, the month value is a
range of 1-12, not 0-11.  A year value will be handled as described
under ``Year 2000 (Y2K) issues'' above.  A \code{-1} argument as
daylight savings flag, passed to \function{mktime()} will usually
result in the correct daylight savings state to be filled in.

\end{itemize}

The module defines the following functions and data items:


\begin{datadesc}{accept2dyear}
Boolean value indicating whether two-digit year values will be
accepted.  This is true by default, but will be set to false if the
environment variable \envvar{PYTHONY2K} has been set to a non-empty
string.  It may also be modified at run time.
\end{datadesc}

\begin{datadesc}{altzone}
The offset of the local DST timezone, in seconds west of UTC, if one
is defined.  This is negative if the local DST timezone is east of UTC
(as in Western Europe, including the UK).  Only use this if
\code{daylight} is nonzero.
\end{datadesc}

\begin{funcdesc}{asctime}{tuple}
Convert a tuple representing a time as returned by \function{gmtime()}
or \function{localtime()} to a 24-character string of the following form:
\code{'Sun Jun 20 23:21:05 1993'}.  Note: unlike the C function of
the same name, there is no trailing newline.
\end{funcdesc}

\begin{funcdesc}{clock}{}
Return the current CPU time as a floating point number expressed in
seconds.  The precision, and in fact the very definiton of the meaning
of ``CPU time''\index{CPU time}, depends on that of the C function
of the same name, but in any case, this is the function to use for
benchmarking\index{benchmarking} Python or timing algorithms.
\end{funcdesc}

\begin{funcdesc}{ctime}{secs}
Convert a time expressed in seconds since the epoch to a string
representing local time.  \code{ctime(\var{secs})} is equivalent to
\code{asctime(localtime(\var{secs}))}.
\end{funcdesc}

\begin{datadesc}{daylight}
Nonzero if a DST timezone is defined.
\end{datadesc}

\begin{funcdesc}{gmtime}{secs}
Convert a time expressed in seconds since the epoch to a time tuple
in UTC in which the dst flag is always zero.  Fractions of a second are
ignored.  See above for a description of the tuple lay-out.
\end{funcdesc}

\begin{funcdesc}{localtime}{secs}
Like \function{gmtime()} but converts to local time.  The dst flag is
set to \code{1} when DST applies to the given time.
\end{funcdesc}

\begin{funcdesc}{mktime}{tuple}
This is the inverse function of \function{localtime()}.  Its argument
is the full 9-tuple (since the dst flag is needed --- pass \code{-1}
as the dst flag if it is unknown) which expresses the time in
\emph{local} time, not UTC.  It returns a floating point number, for
compatibility with \function{time()}.  If the input value cannot be
represented as a valid time, \exception{OverflowError} is raised.
\end{funcdesc}

\begin{funcdesc}{sleep}{secs}
Suspend execution for the given number of seconds.  The argument may
be a floating point number to indicate a more precise sleep time.
The actual suspension time may be less than that requested because any
caught signal will terminate the \function{sleep()} following
execution of that signal's catching routine.  Also, the suspension
time may be longer than requested by an arbitrary amount because of
the scheduling of other activity in the system.
\end{funcdesc}

\begin{funcdesc}{strftime}{format, tuple}
Convert a tuple representing a time as returned by \function{gmtime()}
or \function{localtime()} to a string as specified by the \var{format}
argument.  \var{format} must be a string.

The following directives can be embedded in the \var{format} string.
They are shown without the optional field width and precision
specification, and are replaced by the indicated characters in the
\function{strftime()} result:

\begin{tableiii}{c|p{24em}|c}{code}{Directive}{Meaning}{Notes}
  \lineiii{\%a}{Locale's abbreviated weekday name.}{}
  \lineiii{\%A}{Locale's full weekday name.}{}
  \lineiii{\%b}{Locale's abbreviated month name.}{}
  \lineiii{\%B}{Locale's full month name.}{}
  \lineiii{\%c}{Locale's appropriate date and time representation.}{}
  \lineiii{\%d}{Day of the month as a decimal number [01,31].}{}
  \lineiii{\%H}{Hour (24-hour clock) as a decimal number [00,23].}{}
  \lineiii{\%I}{Hour (12-hour clock) as a decimal number [01,12].}{}
  \lineiii{\%j}{Day of the year as a decimal number [001,366].}{}
  \lineiii{\%m}{Month as a decimal number [01,12].}{}
  \lineiii{\%M}{Minute as a decimal number [00,59].}{}
  \lineiii{\%p}{Locale's equivalent of either AM or PM.}{}
  \lineiii{\%S}{Second as a decimal number [00,61].}{(1)}
  \lineiii{\%U}{Week number of the year (Sunday as the first day of the
                week) as a decimal number [00,53].  All days in a new year
                preceding the first Sunday are considered to be in week 0.}{}
  \lineiii{\%w}{Weekday as a decimal number [0(Sunday),6].}{}
  \lineiii{\%W}{Week number of the year (Monday as the first day of the
                week) as a decimal number [00,53].  All days in a new year
                preceding the first Sunday are considered to be in week 0.}{}
  \lineiii{\%x}{Locale's appropriate date representation.}{}
  \lineiii{\%X}{Locale's appropriate time representation.}{}
  \lineiii{\%y}{Year without century as a decimal number [00,99].}{}
  \lineiii{\%Y}{Year with century as a decimal number.}{}
  \lineiii{\%Z}{Time zone name (or by no characters if no time zone exists).}{}
  \lineiii{\%\%}{A literal \character{\%} character.}{}
\end{tableiii}

\noindent
Notes:

\begin{description}
  \item[(1)]
    The range really is \code{0} to \code{61}; this accounts for leap
    seconds and the (very rare) double leap seconds.
\end{description}

Additional directives may be supported on certain platforms, but
only the ones listed here have a meaning standardized by ANSI C.

On some platforms, an optional field width and precision
specification can immediately follow the initial \character{\%} of a
directive in the following order; this is also not portable.
The field width is normally 2 except for \code{\%j} where it is 3.
\end{funcdesc}

\begin{funcdesc}{strptime}{string\optional{, format}}
Parse a string representing a time according to a format.  The return 
value is a tuple as returned by \function{gmtime()} or
\function{localtime()}.  The \var{format} parameter uses the same
directives as those used by \function{strftime()}; it defaults to
\code{"\%a \%b \%d \%H:\%M:\%S \%Y"} which matches the formatting
returned by \function{ctime()}.  The same platform caveats apply; see
the local \UNIX{} documentation for restrictions or additional
supported directives.  If \var{string} cannot be parsed according to
\var{format}, \exception{ValueError} is raised.

Availability: Most modern \UNIX{} systems.
\end{funcdesc}

\begin{funcdesc}{time}{}
Return the time as a floating point number expressed in seconds since
the epoch, in UTC.  Note that even though the time is always returned
as a floating point number, not all systems provide time with a better
precision than 1 second.
\end{funcdesc}

\begin{datadesc}{timezone}
The offset of the local (non-DST) timezone, in seconds west of UTC
(i.e. negative in most of Western Europe, positive in the US, zero in
the UK).
\end{datadesc}

\begin{datadesc}{tzname}
A tuple of two strings: the first is the name of the local non-DST
timezone, the second is the name of the local DST timezone.  If no DST
timezone is defined, the second string should not be used.
\end{datadesc}


\begin{seealso}
  \seemodule{locale}{Internationalization services.  The locale
                     settings can affect the return values for some of 
                     the functions in the \module{time} module.}
\end{seealso}

\section{\module{getpass}
         --- Portable password input}

\declaremodule{standard}{getpass}
\modulesynopsis{Portable reading of passwords and retrieval of the userid.}
\moduleauthor{Piers Lauder}{piers@cs.su.oz.au}
% Windows (& Mac?) support by Guido van Rossum.
\sectionauthor{Fred L. Drake, Jr.}{fdrake@acm.org}


The \module{getpass} module provides two functions:


\begin{funcdesc}{getpass}{\optional{prompt}}
  Prompt the user for a password without echoing.  The user is
  prompted using the string \var{prompt}, which defaults to
  \code{'Password: '}.
  Availability: Macintosh, \UNIX, Windows.
\end{funcdesc}


\begin{funcdesc}{getuser}{}
  Return the ``login name'' of the user.
  Availability: \UNIX, Windows.

  This function checks the environment variables \envvar{LOGNAME},
  \envvar{USER}, \envvar{LNAME} and \envvar{USERNAME}, in order, and
  returns the value of the first one which is set to a non-empty
  string.  If none are set, the login name from the password database
  is returned on systems which support the \refmodule{pwd} module,
  otherwise, an exception is raised.
\end{funcdesc}

\section{\module{getopt} ---
         Parser for command line options}

\declaremodule{standard}{getopt}
\modulesynopsis{Portable parser for command line options; support both
                short and long option names.}


This module helps scripts to parse the command line arguments in
\code{sys.argv}.
It supports the same conventions as the \UNIX{} \cfunction{getopt()}
function (including the special meanings of arguments of the form
`\code{-}' and `\code{-}\code{-}').
% That's to fool latex2html into leaving the two hyphens alone!
Long options similar to those supported by
GNU software may be used as well via an optional third argument.
This module provides a single function and an exception:

\begin{funcdesc}{getopt}{args, options\optional{, long_options}}
Parses command line options and parameter list.  \var{args} is the
argument list to be parsed, without the leading reference to the
running program. Typically, this means \samp{sys.argv[1:]}.
\var{options} is the string of option letters that the script wants to
recognize, with options that require an argument followed by a colon
(\character{:}; i.e., the same format that \UNIX{}
\cfunction{getopt()} uses).

\note{Unlike GNU \cfunction{getopt()}, after a non-option
argument, all further arguments are considered also non-options.
This is similar to the way non-GNU \UNIX{} systems work.}

\var{long_options}, if specified, must be a list of strings with the
names of the long options which should be supported.  The leading
\code{'-}\code{-'} characters should not be included in the option
name.  Long options which require an argument should be followed by an
equal sign (\character{=}).  To accept only long options,
\var{options} should be an empty string.  Long options on the command
line can be recognized so long as they provide a prefix of the option
name that matches exactly one of the accepted options.  For example,
if \var{long_options} is \code{['foo', 'frob']}, the option
\longprogramopt{fo} will match as \longprogramopt{foo}, but
\longprogramopt{f} will not match uniquely, so \exception{GetoptError}
will be raised.

The return value consists of two elements: the first is a list of
\code{(\var{option}, \var{value})} pairs; the second is the list of
program arguments left after the option list was stripped (this is a
trailing slice of \var{args}).  Each option-and-value pair returned
has the option as its first element, prefixed with a hyphen for short
options (e.g., \code{'-x'}) or two hyphens for long options (e.g.,
\code{'-}\code{-long-option'}), and the option argument as its second
element, or an empty string if the option has no argument.  The
options occur in the list in the same order in which they were found,
thus allowing multiple occurrences.  Long and short options may be
mixed.
\end{funcdesc}

\begin{funcdesc}{gnu_getopt}{args, options\optional{, long_options}}
This function works like \function{getopt()}, except that GNU style
scanning mode is used by default. This means that option and
non-option arguments may be intermixed. The \function{getopt()}
function stops processing options as soon as a non-option argument is
encountered.

If the first character of the option string is `+', or if the
environment variable POSIXLY_CORRECT is set, then option processing
stops as soon as a non-option argument is encountered.

\versionadded{2.3}
\end{funcdesc}

\begin{excdesc}{GetoptError}
This is raised when an unrecognized option is found in the argument
list or when an option requiring an argument is given none.
The argument to the exception is a string indicating the cause of the
error.  For long options, an argument given to an option which does
not require one will also cause this exception to be raised.  The
attributes \member{msg} and \member{opt} give the error message and
related option; if there is no specific option to which the exception
relates, \member{opt} is an empty string.

\versionchanged[Introduced \exception{GetoptError} as a synonym for
                \exception{error}]{1.6}
\end{excdesc}

\begin{excdesc}{error}
Alias for \exception{GetoptError}; for backward compatibility.
\end{excdesc}


An example using only \UNIX{} style options:

\begin{verbatim}
>>> import getopt
>>> args = '-a -b -cfoo -d bar a1 a2'.split()
>>> args
['-a', '-b', '-cfoo', '-d', 'bar', 'a1', 'a2']
>>> optlist, args = getopt.getopt(args, 'abc:d:')
>>> optlist
[('-a', ''), ('-b', ''), ('-c', 'foo'), ('-d', 'bar')]
>>> args
['a1', 'a2']
\end{verbatim}

Using long option names is equally easy:

\begin{verbatim}
>>> s = '--condition=foo --testing --output-file abc.def -x a1 a2'
>>> args = s.split()
>>> args
['--condition=foo', '--testing', '--output-file', 'abc.def', '-x', 'a1', 'a2']
>>> optlist, args = getopt.getopt(args, 'x', [
...     'condition=', 'output-file=', 'testing'])
>>> optlist
[('--condition', 'foo'), ('--testing', ''), ('--output-file', 'abc.def'), ('-x',
 '')]
>>> args
['a1', 'a2']
\end{verbatim}

In a script, typical usage is something like this:

\begin{verbatim}
import getopt, sys

def main():
    try:
        opts, args = getopt.getopt(sys.argv[1:], "ho:v", ["help", "output="])
    except getopt.GetoptError:
        # print help information and exit:
        usage()
        sys.exit(2)
    output = None
    verbose = False
    for o, a in opts:
        if o == "-v":
            verbose = True
        if o in ("-h", "--help"):
            usage()
            sys.exit()
        if o in ("-o", "--output"):
            output = a
    # ...

if __name__ == "__main__":
    main()
\end{verbatim}

\begin{seealso}
  \seemodule{optparse}{More object-oriented command line option parsing.}
\end{seealso}

\section{\module{tempfile} ---
         Generate temporary file names.}
\declaremodule{standard}{tempfile}

\modulesynopsis{Generate temporary file names.}

\indexii{temporary}{file name}
\indexii{temporary}{file}


This module generates temporary file names.  It is not \UNIX{} specific,
but it may require some help on non-\UNIX{} systems.

Note: the modules does not create temporary files, nor does it
automatically remove them when the current process exits or dies.

The module defines a single user-callable function:

\begin{funcdesc}{mktemp}{}
Return a unique temporary filename.  This is an absolute pathname of a
file that does not exist at the time the call is made.  No two calls
will return the same filename.
\end{funcdesc}

The module uses two global variables that tell it how to construct a
temporary name.  The caller may assign values to them; by default they
are initialized at the first call to \function{mktemp()}.

\begin{datadesc}{tempdir}
When set to a value other than \code{None}, this variable defines the
directory in which filenames returned by \function{mktemp()} reside.
The default is taken from the environment variable \envvar{TMPDIR}; if
this is not set, either \file{/usr/tmp} is used (on \UNIX{}), or the
current working directory (all other systems).  No check is made to
see whether its value is valid.
\end{datadesc}

\begin{datadesc}{template}
When set to a value other than \code{None}, this variable defines the
prefix of the final component of the filenames returned by
\function{mktemp()}.  A string of decimal digits is added to generate
unique filenames.  The default is either \file{@\var{pid}.} where
\var{pid} is the current process ID (on \UNIX{}), or \file{tmp} (all
other systems).
\end{datadesc}

\strong{Warning:} if a \UNIX{} process uses \code{mktemp()}, then
calls \function{fork()} and both parent and child continue to use
\function{mktemp()}, the processes will generate conflicting temporary
names.  To resolve this, the child process should assign \code{None}
to \code{template}, to force recomputing the default on the next call
to \function{mktemp()}.

\section{Standard Module \sectcode{errno}}
\stmodindex{errno}

\renewcommand{\indexsubitem}{(in module errno)}

This module makes available standard errno system symbols.
The value of each symbol is the corresponding integer value.
The names and descriptions are borrowed from linux/include/errno.h,
which should be pretty all-inclusive.  Of the following list, symbols
that are not used on the current platform are not defined by the
module.

Symbols available can include:
\begin{datadesc}{EPERM} Operation not permitted \end{datadesc}
\begin{datadesc}{ENOENT} No such file or directory \end{datadesc}
\begin{datadesc}{ESRCH} No such process \end{datadesc}
\begin{datadesc}{EINTR} Interrupted system call \end{datadesc}
\begin{datadesc}{EIO} I/O error \end{datadesc}
\begin{datadesc}{ENXIO} No such device or address \end{datadesc}
\begin{datadesc}{E2BIG} Arg list too long \end{datadesc}
\begin{datadesc}{ENOEXEC} Exec format error \end{datadesc}
\begin{datadesc}{EBADF} Bad file number \end{datadesc}
\begin{datadesc}{ECHILD} No child processes \end{datadesc}
\begin{datadesc}{EAGAIN} Try again \end{datadesc}
\begin{datadesc}{ENOMEM} Out of memory \end{datadesc}
\begin{datadesc}{EACCES} Permission denied \end{datadesc}
\begin{datadesc}{EFAULT} Bad address \end{datadesc}
\begin{datadesc}{ENOTBLK} Block device required \end{datadesc}
\begin{datadesc}{EBUSY} Device or resource busy \end{datadesc}
\begin{datadesc}{EEXIST} File exists \end{datadesc}
\begin{datadesc}{EXDEV} Cross-device link \end{datadesc}
\begin{datadesc}{ENODEV} No such device \end{datadesc}
\begin{datadesc}{ENOTDIR} Not a directory \end{datadesc}
\begin{datadesc}{EISDIR} Is a directory \end{datadesc}
\begin{datadesc}{EINVAL} Invalid argument \end{datadesc}
\begin{datadesc}{ENFILE} File table overflow \end{datadesc}
\begin{datadesc}{EMFILE} Too many open files \end{datadesc}
\begin{datadesc}{ENOTTY} Not a typewriter \end{datadesc}
\begin{datadesc}{ETXTBSY} Text file busy \end{datadesc}
\begin{datadesc}{EFBIG} File too large \end{datadesc}
\begin{datadesc}{ENOSPC} No space left on device \end{datadesc}
\begin{datadesc}{ESPIPE} Illegal seek \end{datadesc}
\begin{datadesc}{EROFS} Read-only file system \end{datadesc}
\begin{datadesc}{EMLINK} Too many links \end{datadesc}
\begin{datadesc}{EPIPE} Broken pipe \end{datadesc}
\begin{datadesc}{EDOM} Math argument out of domain of func \end{datadesc}
\begin{datadesc}{ERANGE} Math result not representable \end{datadesc}
\begin{datadesc}{EDEADLK} Resource deadlock would occur \end{datadesc}
\begin{datadesc}{ENAMETOOLONG} File name too long \end{datadesc}
\begin{datadesc}{ENOLCK} No record locks available \end{datadesc}
\begin{datadesc}{ENOSYS} Function not implemented \end{datadesc}
\begin{datadesc}{ENOTEMPTY} Directory not empty \end{datadesc}
\begin{datadesc}{ELOOP} Too many symbolic links encountered \end{datadesc}
\begin{datadesc}{EWOULDBLOCK} Operation would block \end{datadesc}
\begin{datadesc}{ENOMSG} No message of desired type \end{datadesc}
\begin{datadesc}{EIDRM} Identifier removed \end{datadesc}
\begin{datadesc}{ECHRNG} Channel number out of range \end{datadesc}
\begin{datadesc}{EL2NSYNC} Level 2 not synchronized \end{datadesc}
\begin{datadesc}{EL3HLT} Level 3 halted \end{datadesc}
\begin{datadesc}{EL3RST} Level 3 reset \end{datadesc}
\begin{datadesc}{ELNRNG} Link number out of range \end{datadesc}
\begin{datadesc}{EUNATCH} Protocol driver not attached \end{datadesc}
\begin{datadesc}{ENOCSI} No CSI structure available \end{datadesc}
\begin{datadesc}{EL2HLT} Level 2 halted \end{datadesc}
\begin{datadesc}{EBADE} Invalid exchange \end{datadesc}
\begin{datadesc}{EBADR} Invalid request descriptor \end{datadesc}
\begin{datadesc}{EXFULL} Exchange full \end{datadesc}
\begin{datadesc}{ENOANO} No anode \end{datadesc}
\begin{datadesc}{EBADRQC} Invalid request code \end{datadesc}
\begin{datadesc}{EBADSLT} Invalid slot \end{datadesc}
\begin{datadesc}{EDEADLOCK} File locking deadlock error \end{datadesc}
\begin{datadesc}{EBFONT} Bad font file format \end{datadesc}
\begin{datadesc}{ENOSTR} Device not a stream \end{datadesc}
\begin{datadesc}{ENODATA} No data available \end{datadesc}
\begin{datadesc}{ETIME} Timer expired \end{datadesc}
\begin{datadesc}{ENOSR} Out of streams resources \end{datadesc}
\begin{datadesc}{ENONET} Machine is not on the network \end{datadesc}
\begin{datadesc}{ENOPKG} Package not installed \end{datadesc}
\begin{datadesc}{EREMOTE} Object is remote \end{datadesc}
\begin{datadesc}{ENOLINK} Link has been severed \end{datadesc}
\begin{datadesc}{EADV} Advertise error \end{datadesc}
\begin{datadesc}{ESRMNT} Srmount error \end{datadesc}
\begin{datadesc}{ECOMM} Communication error on send \end{datadesc}
\begin{datadesc}{EPROTO} Protocol error \end{datadesc}
\begin{datadesc}{EMULTIHOP} Multihop attempted \end{datadesc}
\begin{datadesc}{EDOTDOT} RFS specific error \end{datadesc}
\begin{datadesc}{EBADMSG} Not a data message \end{datadesc}
\begin{datadesc}{EOVERFLOW} Value too large for defined data type \end{datadesc}
\begin{datadesc}{ENOTUNIQ} Name not unique on network \end{datadesc}
\begin{datadesc}{EBADFD} File descriptor in bad state \end{datadesc}
\begin{datadesc}{EREMCHG} Remote address changed \end{datadesc}
\begin{datadesc}{ELIBACC} Can not access a needed shared library \end{datadesc}
\begin{datadesc}{ELIBBAD} Accessing a corrupted shared library \end{datadesc}
\begin{datadesc}{ELIBSCN} .lib section in a.out corrupted \end{datadesc}
\begin{datadesc}{ELIBMAX} Attempting to link in too many shared libraries \end{datadesc}
\begin{datadesc}{ELIBEXEC} Cannot exec a shared library directly \end{datadesc}
\begin{datadesc}{EILSEQ} Illegal byte sequence \end{datadesc}
\begin{datadesc}{ERESTART} Interrupted system call should be restarted \end{datadesc}
\begin{datadesc}{ESTRPIPE} Streams pipe error \end{datadesc}
\begin{datadesc}{EUSERS} Too many users \end{datadesc}
\begin{datadesc}{ENOTSOCK} Socket operation on non-socket \end{datadesc}
\begin{datadesc}{EDESTADDRREQ} Destination address required \end{datadesc}
\begin{datadesc}{EMSGSIZE} Message too long \end{datadesc}
\begin{datadesc}{EPROTOTYPE} Protocol wrong type for socket \end{datadesc}
\begin{datadesc}{ENOPROTOOPT} Protocol not available \end{datadesc}
\begin{datadesc}{EPROTONOSUPPORT} Protocol not supported \end{datadesc}
\begin{datadesc}{ESOCKTNOSUPPORT} Socket type not supported \end{datadesc}
\begin{datadesc}{EOPNOTSUPP} Operation not supported on transport endpoint \end{datadesc}
\begin{datadesc}{EPFNOSUPPORT} Protocol family not supported \end{datadesc}
\begin{datadesc}{EAFNOSUPPORT} Address family not supported by protocol \end{datadesc}
\begin{datadesc}{EADDRINUSE} Address already in use \end{datadesc}
\begin{datadesc}{EADDRNOTAVAIL} Cannot assign requested address \end{datadesc}
\begin{datadesc}{ENETDOWN} Network is down \end{datadesc}
\begin{datadesc}{ENETUNREACH} Network is unreachable \end{datadesc}
\begin{datadesc}{ENETRESET} Network dropped connection because of reset \end{datadesc}
\begin{datadesc}{ECONNABORTED} Software caused connection abort \end{datadesc}
\begin{datadesc}{ECONNRESET} Connection reset by peer \end{datadesc}
\begin{datadesc}{ENOBUFS} No buffer space available \end{datadesc}
\begin{datadesc}{EISCONN} Transport endpoint is already connected \end{datadesc}
\begin{datadesc}{ENOTCONN} Transport endpoint is not connected \end{datadesc}
\begin{datadesc}{ESHUTDOWN} Cannot send after transport endpoint shutdown \end{datadesc}
\begin{datadesc}{ETOOMANYREFS} Too many references: cannot splice \end{datadesc}
\begin{datadesc}{ETIMEDOUT} Connection timed out \end{datadesc}
\begin{datadesc}{ECONNREFUSED} Connection refused \end{datadesc}
\begin{datadesc}{EHOSTDOWN} Host is down \end{datadesc}
\begin{datadesc}{EHOSTUNREACH} No route to host \end{datadesc}
\begin{datadesc}{EALREADY} Operation already in progress \end{datadesc}
\begin{datadesc}{EINPROGRESS} Operation now in progress \end{datadesc}
\begin{datadesc}{ESTALE} Stale NFS file handle \end{datadesc}
\begin{datadesc}{EUCLEAN} Structure needs cleaning \end{datadesc}
\begin{datadesc}{ENOTNAM} Not a XENIX named type file \end{datadesc}
\begin{datadesc}{ENAVAIL} No XENIX semaphores available \end{datadesc}
\begin{datadesc}{EISNAM} Is a named type file \end{datadesc}
\begin{datadesc}{EREMOTEIO} Remote I/O error \end{datadesc}
\begin{datadesc}{EDQUOT} Quota exceeded \end{datadesc}


\section{Standard Module \sectcode{glob}}
\stmodindex{glob}
\renewcommand{\indexsubitem}{(in module glob)}

The \code{glob} module finds all the pathnames matching a specified
pattern according to the rules used by the \UNIX{} shell.  No tilde
expansion is done, but \verb\*\, \verb\?\, and character ranges
expressed with \verb\[]\ will be correctly matched.  This is done by
using the \code{os.listdir()} and \code{fnmatch.fnmatch()} functions
in concert, and not by actually invoking a subshell.  (For tilde and
shell variable expansion, use \code{os.path.expanduser(}) and
\code{os.path.expandvars()}.)

\begin{funcdesc}{glob}{pathname}
Returns a possibly-empty list of path names that match \var{pathname},
which must be a string containing a path specification.
\var{pathname} can be either absolute (like
\file{/usr/src/Python1.4/Makefile}) or relative (like
\file{../../Tools/*.gif}), and can contain shell-style wildcards.
\end{funcdesc}

For example, consider a directory containing only the following files:
\file{1.gif}, \file{2.txt}, and \file{card.gif}.  \code{glob.glob()}
will produce the following results.  Notice how any leading components
of the path are preserved.

\begin{verbatim}
>>> import glob
>>> glob.glob('./[0-9].*')
['./1.gif', './2.txt']
>>> glob.glob('*.gif')
['1.gif', 'card.gif']
>>> glob.glob('?.gif')
['1.gif']
\end{verbatim}

\section{Standard Module \sectcode{fnmatch}}
\label{module-fnmatch}
\stmodindex{fnmatch}

This module provides support for Unix shell-style wildcards, which are
\emph{not} the same as Python's regular expressions (which are
documented in the \code{re} module).  The special characters used
in shell-style wildcards are:
\begin{itemize}
\item[\code{*}] matches everything
\item[\code{?}]	matches any single character
\item[\code{[}\var{seq}\code{]}] matches any character in \var{seq}
\item[\code{[!}\var{seq}\code{]}] matches any character not in \var{seq}
\end{itemize}

Note that the filename separator (\code{'/'} on Unix) is \emph{not}
special to this module.  See module \code{glob} for pathname expansion
(\code{glob} uses \code{fnmatch} to match filename segments).

\renewcommand{\indexsubitem}{(in module fnmatch)}

\begin{funcdesc}{fnmatch}{filename\, pattern}
Test whether the \var{filename} string matches the \var{pattern}
string, returning true or false.  If the operating system is
case-insensitive, then both parameters will be normalized to all
lower- or upper-case before the comparision is performed.  If you
require a case-sensitive comparision regardless of whether that's
standard for your operating system, use \code{fnmatchcase()} instead.
\end{funcdesc}

\begin{funcdesc}{fnmatchcase}{}
Test whether \var{filename} matches \var{pattern}, returning true or
false; the comparision is case-sensitive.
\end{funcdesc}

\begin{funcdesc}{translate}{pattern}
Translate a shell pattern into a corresponding regular expression,
returning a string describing the pattern.  It does not compile the
expression.  
\end{funcdesc}


\section{\module{shutil} ---
         High-level file operations}

\declaremodule{standard}{shutil}
\modulesynopsis{High-level file operations, including copying.}
\sectionauthor{Fred L. Drake, Jr.}{fdrake@acm.org}
% partly based on the docstrings


The \module{shutil} module offers a number of high-level operations on
files and collections of files.  In particular, functions are provided 
which support file copying and removal.
\index{file!copying}
\index{copying files}

\strong{Caveat:}  On MacOS, the resource fork and other metadata are
not used.  For file copies, this means that resources will be lost and 
file type and creator codes will not be correct.


\begin{funcdesc}{copyfile}{src, dst}
  Copy the contents of \var{src} to \var{dst}.  If \var{dst} exists,
  it will be replaced, otherwise it will be created.
\end{funcdesc}

\begin{funcdesc}{copyfileobj}{fsrc, fdst\optional{, length}}
  Copy the contents of the file-like object \var{fsrc} to the
  file-like object \var{fdst}.  The integer \var{length}, if given,
  is the buffer size. In particular, a negative \var{length} value
  means to copy the data without looping over the source data in
  chunks; by default the data is read in chunks to avoid uncontrolled
  memory consumption.
\end{funcdesc}

\begin{funcdesc}{copymode}{src, dst}
  Copy the permission bits from \var{src} to \var{dst}.  The file
  contents, owner, and group are unaffected.
\end{funcdesc}

\begin{funcdesc}{copystat}{src, dst}
  Copy the permission bits, last access time, and last modification
  time from \var{src} to \var{dst}.  The file contents, owner, and
  group are unaffected.
\end{funcdesc}

\begin{funcdesc}{copy}{src, dst}
  Copy the file \var{src} to the file or directory \var{dst}.  If
  \var{dst} is a directory, a file with the same basename as \var{src} 
  is created (or overwritten) in the directory specified.  Permission
  bits are copied.
\end{funcdesc}

\begin{funcdesc}{copy2}{src, dst}
  Similar to \function{copy()}, but last access time and last
  modification time are copied as well.  This is similar to the
  \UNIX{} command \program{cp} \programopt{-p}.
\end{funcdesc}

\begin{funcdesc}{copytree}{src, dst\optional{, symlinks}}
  Recursively copy an entire directory tree rooted at \var{src}.  The
  destination directory, named by \var{dst}, must not already exist;
  it will be created.  Individual files are copied using
  \function{copy2()}.  If \var{symlinks} is true, symbolic links in
  the source tree are represented as symbolic links in the new tree;
  if false or omitted, the contents of the linked files are copied to
  the new tree.  Errors are reported to standard output.

  The source code for this should be considered an example rather than 
  a tool.
\end{funcdesc}

\begin{funcdesc}{rmtree}{path\optional{, ignore_errors\optional{, onerror}}}
\index{directory!deleting}
  Delete an entire directory tree.  If \var{ignore_errors} is true,
  errors will be ignored; if false or omitted, errors are handled by
  calling a handler specified by \var{onerror} or raise an exception.

  If \var{onerror} is provided, it must be a callable that accepts
  three parameters: \var{function}, \var{path}, and \var{excinfo}.
  The first parameter, \var{function}, is the function which raised
  the exception; it will be \function{os.remove()} or
  \function{os.rmdir()}.  The second parameter, \var{path}, will be
  the path name passed to \var{function}.  The third parameter,
  \var{excinfo}, will be the exception information return by
  \function{sys.exc_info()}.  Exceptions raised by \var{onerror} will
  not be caught.
\end{funcdesc}


\subsection{Example \label{shutil-example}}

This example is the implementation of the \function{copytree()}
function, described above, with the docstring omitted.  It
demonstrates many of the other functions provided by this module.

\begin{verbatim}
def copytree(src, dst, symlinks=0):
    names = os.listdir(src)
    os.mkdir(dst)
    for name in names:
        srcname = os.path.join(src, name)
        dstname = os.path.join(dst, name)
        try:
            if symlinks and os.path.islink(srcname):
                linkto = os.readlink(srcname)
                os.symlink(linkto, dstname)
            elif os.path.isdir(srcname):
                copytree(srcname, dstname)
            else:
                copy2(srcname, dstname)
            # XXX What about devices, sockets etc.?
        except (IOError, os.error), why:
            print "Can't copy %s to %s: %s" % (`srcname`, `dstname`, str(why))
\end{verbatim}

\section{\module{locale} ---
         Internationalization services}

\declaremodule{standard}{locale}
\modulesynopsis{Internationalization services.}
\moduleauthor{Martin von L\"owis}{martin@v.loewis.de}
\sectionauthor{Martin von L\"owis}{martin@v.loewis.de}


The \module{locale} module opens access to the \POSIX{} locale
database and functionality. The \POSIX{} locale mechanism allows
programmers to deal with certain cultural issues in an application,
without requiring the programmer to know all the specifics of each
country where the software is executed.

The \module{locale} module is implemented on top of the
\module{_locale}\refbimodindex{_locale} module, which in turn uses an
ANSI C locale implementation if available.

The \module{locale} module defines the following exception and
functions:


\begin{excdesc}{Error}
  Exception raised when \function{setlocale()} fails.
\end{excdesc}

\begin{funcdesc}{setlocale}{category\optional{, locale}}
  If \var{locale} is specified, it may be a string, a tuple of the
  form \code{(\var{language code}, \var{encoding})}, or \code{None}.
  If it is a tuple, it is converted to a string using the locale
  aliasing engine.  If \var{locale} is given and not \code{None},
  \function{setlocale()} modifies the locale setting for the
  \var{category}.  The available categories are listed in the data
  description below.  The value is the name of a locale.  An empty
  string specifies the user's default settings. If the modification of
  the locale fails, the exception \exception{Error} is raised.  If
  successful, the new locale setting is returned.

  If \var{locale} is omitted or \code{None}, the current setting for
  \var{category} is returned.

  \function{setlocale()} is not thread safe on most systems.
  Applications typically start with a call of

\begin{verbatim}
import locale
locale.setlocale(locale.LC_ALL, '')
\end{verbatim}

  This sets the locale for all categories to the user's default
  setting (typically specified in the \envvar{LANG} environment
  variable).  If the locale is not changed thereafter, using
  multithreading should not cause problems.

  \versionchanged[Added support for tuple values of the \var{locale}
                  parameter]{2.0}
\end{funcdesc}

\begin{funcdesc}{localeconv}{}
  Returns the database of the local conventions as a dictionary.
  This dictionary has the following strings as keys:

  \begin{tableiii}{l|l|p{3in}}{constant}{Key}{Category}{Meaning}
    \lineiii{LC_NUMERIC}{\code{'decimal_point'}}
            {Decimal point character.}
    \lineiii{}{\code{'grouping'}}
            {Sequence of numbers specifying which relative positions
             the \code{'thousands_sep'} is expected.  If the sequence is
             terminated with \constant{CHAR_MAX}, no further grouping
             is performed. If the sequence terminates with a \code{0}, 
             the last group size is repeatedly used.}
    \lineiii{}{\code{'thousands_sep'}}
            {Character used between groups.}\hline
    \lineiii{LC_MONETARY}{\code{'int_curr_symbol'}}
            {International currency symbol.}
    \lineiii{}{\code{'currency_symbol'}}
            {Local currency symbol.}
    \lineiii{}{\code{'mon_decimal_point'}}
            {Decimal point used for monetary values.}
    \lineiii{}{\code{'mon_thousands_sep'}}
            {Group separator used for monetary values.}
    \lineiii{}{\code{'mon_grouping'}}
            {Equivalent to \code{'grouping'}, used for monetary
             values.}
    \lineiii{}{\code{'positive_sign'}}
            {Symbol used to annotate a positive monetary value.}
    \lineiii{}{\code{'negative_sign'}}
            {Symbol used to annotate a nnegative monetary value.}
    \lineiii{}{\code{'frac_digits'}}
            {Number of fractional digits used in local formatting
             of monetary values.}
    \lineiii{}{\code{'int_frac_digits'}}
            {Number of fractional digits used in international
             formatting of monetary values.}
  \end{tableiii}

  The possible values for \code{'p_sign_posn'} and
  \code{'n_sign_posn'} are given below.

  \begin{tableii}{c|l}{code}{Value}{Explanation}
    \lineii{0}{Currency and value are surrounded by parentheses.}
    \lineii{1}{The sign should precede the value and currency symbol.}
    \lineii{2}{The sign should follow the value and currency symbol.}
    \lineii{3}{The sign should immediately precede the value.}
    \lineii{4}{The sign should immediately follow the value.}
    \lineii{\constant{LC_MAX}}{Nothing is specified in this locale.}
  \end{tableii}
\end{funcdesc}

\begin{funcdesc}{nl_langinfo}{option}

Return some locale-specific information as a string. This function is
not available on all systems, and the set of possible options might
also vary across platforms. The possible argument values are numbers,
for which symbolic constants are available in the locale module.

\end{funcdesc}

\begin{funcdesc}{getdefaultlocale}{\optional{envvars}}
  Tries to determine the default locale settings and returns
  them as a tuple of the form \code{(\var{language code},
  \var{encoding})}.

  According to \POSIX, a program which has not called
  \code{setlocale(LC_ALL, '')} runs using the portable \code{'C'}
  locale.  Calling \code{setlocale(LC_ALL, '')} lets it use the
  default locale as defined by the \envvar{LANG} variable.  Since we
  do not want to interfere with the current locale setting we thus
  emulate the behavior in the way described above.

  To maintain compatibility with other platforms, not only the
  \envvar{LANG} variable is tested, but a list of variables given as
  envvars parameter.  The first found to be defined will be
  used.  \var{envvars} defaults to the search path used in GNU gettext;
  it must always contain the variable name \samp{LANG}.  The GNU
  gettext search path contains \code{'LANGUAGE'}, \code{'LC_ALL'},
  \code{'LC_CTYPE'}, and \code{'LANG'}, in that order.

  Except for the code \code{'C'}, the language code corresponds to
  \rfc{1766}.  \var{language code} and \var{encoding} may be
  \code{None} if their values cannot be determined.
  \versionadded{2.0}
\end{funcdesc}

\begin{funcdesc}{getlocale}{\optional{category}}
  Returns the current setting for the given locale category as
  sequence containing \var{language code}, \var{encoding}.
  \var{category} may be one of the \constant{LC_*} values except
  \constant{LC_ALL}.  It defaults to \constant{LC_CTYPE}.

  Except for the code \code{'C'}, the language code corresponds to
  \rfc{1766}.  \var{language code} and \var{encoding} may be
  \code{None} if their values cannot be determined.
  \versionadded{2.0}
\end{funcdesc}

\begin{funcdesc}{getpreferredencoding}{\optional{do_setlocale}}
  Return the encoding used for text data, according to user
  preferences.  User preferences are expressed differently on
  different systems, and might not be available programmatically on
  some systems, so this function only returns a guess.

  On some systems, it is necessary to invoke \function{setlocale}
  to obtain the user preferences, so this function is not thread-safe.
  If invoking setlocale is not necessary or desired, \var{do_setlocale}
  should be set to \code{False}.

  \versionadded{2.3}
\end{funcdesc}

\begin{funcdesc}{normalize}{localename}
  Returns a normalized locale code for the given locale name.  The
  returned locale code is formatted for use with
  \function{setlocale()}.  If normalization fails, the original name
  is returned unchanged.

  If the given encoding is not known, the function defaults to
  the default encoding for the locale code just like
  \function{setlocale()}.
  \versionadded{2.0}
\end{funcdesc}

\begin{funcdesc}{resetlocale}{\optional{category}}
  Sets the locale for \var{category} to the default setting.

  The default setting is determined by calling
  \function{getdefaultlocale()}.  \var{category} defaults to
  \constant{LC_ALL}.
  \versionadded{2.0}
\end{funcdesc}

\begin{funcdesc}{strcoll}{string1, string2}
  Compares two strings according to the current
  \constant{LC_COLLATE} setting. As any other compare function,
  returns a negative, or a positive value, or \code{0}, depending on
  whether \var{string1} collates before or after \var{string2} or is
  equal to it.
\end{funcdesc}

\begin{funcdesc}{strxfrm}{string}
  Transforms a string to one that can be used for the built-in
  function \function{cmp()}\bifuncindex{cmp}, and still returns
  locale-aware results.  This function can be used when the same
  string is compared repeatedly, e.g. when collating a sequence of
  strings.
\end{funcdesc}

\begin{funcdesc}{format}{format, val\optional{, grouping}}
  Formats a number \var{val} according to the current
  \constant{LC_NUMERIC} setting.  The format follows the conventions
  of the \code{\%} operator.  For floating point values, the decimal
  point is modified if appropriate.  If \var{grouping} is true, also
  takes the grouping into account.
\end{funcdesc}

\begin{funcdesc}{str}{float}
  Formats a floating point number using the same format as the
  built-in function \code{str(\var{float})}, but takes the decimal
  point into account.
\end{funcdesc}

\begin{funcdesc}{atof}{string}
  Converts a string to a floating point number, following the
  \constant{LC_NUMERIC} settings.
\end{funcdesc}

\begin{funcdesc}{atoi}{string}
  Converts a string to an integer, following the
  \constant{LC_NUMERIC} conventions.
\end{funcdesc}

\begin{datadesc}{LC_CTYPE}
\refstmodindex{string}
  Locale category for the character type functions.  Depending on the
  settings of this category, the functions of module
  \refmodule{string} dealing with case change their behaviour.
\end{datadesc}

\begin{datadesc}{LC_COLLATE}
  Locale category for sorting strings.  The functions
  \function{strcoll()} and \function{strxfrm()} of the
  \module{locale} module are affected.
\end{datadesc}

\begin{datadesc}{LC_TIME}
  Locale category for the formatting of time.  The function
  \function{time.strftime()} follows these conventions.
\end{datadesc}

\begin{datadesc}{LC_MONETARY}
  Locale category for formatting of monetary values.  The available
  options are available from the \function{localeconv()} function.
\end{datadesc}

\begin{datadesc}{LC_MESSAGES}
  Locale category for message display. Python currently does not
  support application specific locale-aware messages.  Messages
  displayed by the operating system, like those returned by
  \function{os.strerror()} might be affected by this category.
\end{datadesc}

\begin{datadesc}{LC_NUMERIC}
  Locale category for formatting numbers.  The functions
  \function{format()}, \function{atoi()}, \function{atof()} and
  \function{str()} of the \module{locale} module are affected by that
  category.  All other numeric formatting operations are not
  affected.
\end{datadesc}

\begin{datadesc}{LC_ALL}
  Combination of all locale settings.  If this flag is used when the
  locale is changed, setting the locale for all categories is
  attempted. If that fails for any category, no category is changed at
  all.  When the locale is retrieved using this flag, a string
  indicating the setting for all categories is returned. This string
  can be later used to restore the settings.
\end{datadesc}

\begin{datadesc}{CHAR_MAX}
  This is a symbolic constant used for different values returned by
  \function{localeconv()}.
\end{datadesc}

The \function{nl_langinfo} function accepts one of the following keys.
Most descriptions are taken from the corresponding description in the
GNU C library.

\begin{datadesc}{CODESET}
Return a string with the name of the character encoding used in the
selected locale.
\end{datadesc}

\begin{datadesc}{D_T_FMT}
Return a string that can be used as a format string for strftime(3) to
represent time and date in a locale-specific way.
\end{datadesc}

\begin{datadesc}{D_FMT}
Return a string that can be used as a format string for strftime(3) to
represent a date in a locale-specific way.
\end{datadesc}

\begin{datadesc}{T_FMT}
Return a string that can be used as a format string for strftime(3) to
represent a time in a locale-specific way.
\end{datadesc}

\begin{datadesc}{T_FMT_AMPM}
The return value can be used as a format string for `strftime' to
represent time in the am/pm format.
\end{datadesc}

\begin{datadesc}{DAY_1 ... DAY_7}
Return name of the n-th day of the week. \warning{This
follows the US convention of \constant{DAY_1} being Sunday, not the
international convention (ISO 8601) that Monday is the first day of
the week.}
\end{datadesc}

\begin{datadesc}{ABDAY_1 ... ABDAY_7}
Return abbreviated name of the n-th day of the week.
\end{datadesc}

\begin{datadesc}{MON_1 ... MON_12}
Return name of the n-th month.
\end{datadesc}

\begin{datadesc}{ABMON_1 ... ABMON_12}
Return abbreviated name of the n-th month.
\end{datadesc}

\begin{datadesc}{RADIXCHAR}
Return radix character (decimal dot, decimal comma, etc.)
\end{datadesc}

\begin{datadesc}{THOUSEP}
Return separator character for thousands (groups of three digits).
\end{datadesc}

\begin{datadesc}{YESEXPR}
Return a regular expression that can be used with the regex
function to recognize a positive response to a yes/no question.
\warning{The expression is in the syntax suitable for the
\cfunction{regex()} function from the C library, which might differ
from the syntax used in \refmodule{re}.}
\end{datadesc}

\begin{datadesc}{NOEXPR}
Return a regular expression that can be used with the regex(3)
function to recognize a negative response to a yes/no question.
\end{datadesc}

\begin{datadesc}{CRNCYSTR}
Return the currency symbol, preceded by "-" if the symbol should
appear before the value, "+" if the symbol should appear after the
value, or "." if the symbol should replace the radix character.
\end{datadesc}

\begin{datadesc}{ERA}
The return value represents the era used in the current locale.

Most locales do not define this value.  An example of a locale which
does define this value is the Japanese one.  In Japan, the traditional
representation of dates includes the name of the era corresponding to
the then-emperor's reign.

Normally it should not be necessary to use this value directly.
Specifying the \code{E} modifier in their format strings causes the
\function{strftime} function to use this information.  The format of the
returned string is not specified, and therefore you should not assume
knowledge of it on different systems.
\end{datadesc}

\begin{datadesc}{ERA_YEAR}
The return value gives the year in the relevant era of the locale.
\end{datadesc}

\begin{datadesc}{ERA_D_T_FMT}
This return value can be used as a format string for
\function{strftime} to represent dates and times in a locale-specific
era-based way.
\end{datadesc}

\begin{datadesc}{ERA_D_FMT}
This return value can be used as a format string for
\function{strftime} to represent time in a locale-specific era-based
way.
\end{datadesc}

\begin{datadesc}{ALT_DIGITS}
The return value is a representation of up to 100 values used to
represent the values 0 to 99.
\end{datadesc}

Example:

\begin{verbatim}
>>> import locale
>>> loc = locale.setlocale(locale.LC_ALL) # get current locale
>>> locale.setlocale(locale.LC_ALL, 'de_DE') # use German locale; name might vary with platform
>>> locale.strcoll('f\xe4n', 'foo') # compare a string containing an umlaut 
>>> locale.setlocale(locale.LC_ALL, '') # use user's preferred locale
>>> locale.setlocale(locale.LC_ALL, 'C') # use default (C) locale
>>> locale.setlocale(locale.LC_ALL, loc) # restore saved locale
\end{verbatim}


\subsection{Background, details, hints, tips and caveats}

The C standard defines the locale as a program-wide property that may
be relatively expensive to change.  On top of that, some
implementation are broken in such a way that frequent locale changes
may cause core dumps.  This makes the locale somewhat painful to use
correctly.

Initially, when a program is started, the locale is the \samp{C} locale, no
matter what the user's preferred locale is.  The program must
explicitly say that it wants the user's preferred locale settings by
calling \code{setlocale(LC_ALL, '')}.

It is generally a bad idea to call \function{setlocale()} in some library
routine, since as a side effect it affects the entire program.  Saving
and restoring it is almost as bad: it is expensive and affects other
threads that happen to run before the settings have been restored.

If, when coding a module for general use, you need a locale
independent version of an operation that is affected by the locale
(such as \function{string.lower()}, or certain formats used with
\function{time.strftime()}), you will have to find a way to do it
without using the standard library routine.  Even better is convincing
yourself that using locale settings is okay.  Only as a last resort
should you document that your module is not compatible with
non-\samp{C} locale settings.

The case conversion functions in the
\refmodule{string}\refstmodindex{string} module are affected by the
locale settings.  When a call to the \function{setlocale()} function
changes the \constant{LC_CTYPE} settings, the variables
\code{string.lowercase}, \code{string.uppercase} and
\code{string.letters} are recalculated.  Note that this code that uses
these variable through `\keyword{from} ... \keyword{import} ...',
e.g.\ \code{from string import letters}, is not affected by subsequent
\function{setlocale()} calls.

The only way to perform numeric operations according to the locale
is to use the special functions defined by this module:
\function{atof()}, \function{atoi()}, \function{format()},
\function{str()}.

\subsection{For extension writers and programs that embed Python
            \label{embedding-locale}}

Extension modules should never call \function{setlocale()}, except to
find out what the current locale is.  But since the return value can
only be used portably to restore it, that is not very useful (except
perhaps to find out whether or not the locale is \samp{C}).

When Python code uses the \module{locale} module to change the locale,
this also affects the embedding application.  If the embedding
application doesn't want this to happen, it should remove the
\module{_locale} extension module (which does all the work) from the
table of built-in modules in the \file{config.c} file, and make sure
that the \module{_locale} module is not accessible as a shared library.


\subsection{Access to message catalogs \label{locale-gettext}}

The locale module exposes the C library's gettext interface on systems
that provide this interface.  It consists of the functions
\function{gettext()}, \function{dgettext()}, \function{dcgettext()},
\function{textdomain()}, \function{bindtextdomain()}, and
\function{bind_textdomain_codeset()}.  These are similar to the same
functions in the \refmodule{gettext} module, but use the C library's
binary format for message catalogs, and the C library's search
algorithms for locating message catalogs. 

Python applications should normally find no need to invoke these
functions, and should use \refmodule{gettext} instead.  A known
exception to this rule are applications that link use additional C
libraries which internally invoke \cfunction{gettext()} or
\function{dcgettext()}.  For these applications, it may be necessary to
bind the text domain, so that the libraries can properly locate their
message catalogs.


\chapter{Optional Operating System Services}
\label{someos}

The modules described in this chapter provide interfaces to operating
system features that are available on selected operating systems only.
The interfaces are generally modeled after the \UNIX{} or \C{}
interfaces but they are available on some other systems as well
(e.g. Windows or NT).  Here's an overview:

\localmoduletable
		% Optional Operating System Services
\section{\module{signal} ---
         Set handlers for asynchronous events}

\declaremodule{builtin}{signal}
\modulesynopsis{Set handlers for asynchronous events.}


This module provides mechanisms to use signal handlers in Python.
Some general rules for working with signals and their handlers:

\begin{itemize}

\item
A handler for a particular signal, once set, remains installed until
it is explicitly reset (Python emulates the BSD style interface
regardless of the underlying implementation), with the exception of
the handler for \constant{SIGCHLD}, which follows the underlying
implementation.

\item
There is no way to ``block'' signals temporarily from critical
sections (since this is not supported by all \UNIX{} flavors).

\item
Although Python signal handlers are called asynchronously as far as
the Python user is concerned, they can only occur between the
``atomic'' instructions of the Python interpreter.  This means that
signals arriving during long calculations implemented purely in C
(such as regular expression matches on large bodies of text) may be
delayed for an arbitrary amount of time.

\item
When a signal arrives during an I/O operation, it is possible that the
I/O operation raises an exception after the signal handler returns.
This is dependent on the underlying \UNIX{} system's semantics regarding
interrupted system calls.

\item
Because the \C{} signal handler always returns, it makes little sense to
catch synchronous errors like \constant{SIGFPE} or \constant{SIGSEGV}.

\item
Python installs a small number of signal handlers by default:
\constant{SIGPIPE} is ignored (so write errors on pipes and sockets can be
reported as ordinary Python exceptions) and \constant{SIGINT} is translated
into a \exception{KeyboardInterrupt} exception.  All of these can be
overridden.

\item
Some care must be taken if both signals and threads are used in the
same program.  The fundamental thing to remember in using signals and
threads simultaneously is:\ always perform \function{signal()} operations
in the main thread of execution.  Any thread can perform an
\function{alarm()}, \function{getsignal()}, or \function{pause()};
only the main thread can set a new signal handler, and the main thread
will be the only one to receive signals (this is enforced by the
Python \module{signal} module, even if the underlying thread
implementation supports sending signals to individual threads).  This
means that signals can't be used as a means of inter-thread
communication.  Use locks instead.

\end{itemize}

The variables defined in the \module{signal} module are:

\begin{datadesc}{SIG_DFL}
  This is one of two standard signal handling options; it will simply
  perform the default function for the signal.  For example, on most
  systems the default action for \constant{SIGQUIT} is to dump core
  and exit, while the default action for \constant{SIGCLD} is to
  simply ignore it.
\end{datadesc}

\begin{datadesc}{SIG_IGN}
  This is another standard signal handler, which will simply ignore
  the given signal.
\end{datadesc}

\begin{datadesc}{SIG*}
  All the signal numbers are defined symbolically.  For example, the
  hangup signal is defined as \constant{signal.SIGHUP}; the variable names
  are identical to the names used in C programs, as found in
  \code{<signal.h>}.
  The \UNIX{} man page for `\cfunction{signal()}' lists the existing
  signals (on some systems this is \manpage{signal}{2}, on others the
  list is in \manpage{signal}{7}).
  Note that not all systems define the same set of signal names; only
  those names defined by the system are defined by this module.
\end{datadesc}

\begin{datadesc}{NSIG}
  One more than the number of the highest signal number.
\end{datadesc}

The \module{signal} module defines the following functions:

\begin{funcdesc}{alarm}{time}
  If \var{time} is non-zero, this function requests that a
  \constant{SIGALRM} signal be sent to the process in \var{time} seconds.
  Any previously scheduled alarm is canceled (only one alarm can
  be scheduled at any time).  The returned value is then the number of
  seconds before any previously set alarm was to have been delivered.
  If \var{time} is zero, no alarm id scheduled, and any scheduled
  alarm is canceled.  The return value is the number of seconds
  remaining before a previously scheduled alarm.  If the return value
  is zero, no alarm is currently scheduled.  (See the \UNIX{} man page
  \manpage{alarm}{2}.)
  Availability: \UNIX.
\end{funcdesc}

\begin{funcdesc}{getsignal}{signalnum}
  Return the current signal handler for the signal \var{signalnum}.
  The returned value may be a callable Python object, or one of the
  special values \constant{signal.SIG_IGN}, \constant{signal.SIG_DFL} or
  \constant{None}.  Here, \constant{signal.SIG_IGN} means that the
  signal was previously ignored, \constant{signal.SIG_DFL} means that the
  default way of handling the signal was previously in use, and
  \code{None} means that the previous signal handler was not installed
  from Python.
\end{funcdesc}

\begin{funcdesc}{pause}{}
  Cause the process to sleep until a signal is received; the
  appropriate handler will then be called.  Returns nothing.  Not on
  Windows. (See the \UNIX{} man page \manpage{signal}{2}.)
\end{funcdesc}

\begin{funcdesc}{signal}{signalnum, handler}
  Set the handler for signal \var{signalnum} to the function
  \var{handler}.  \var{handler} can be a callable Python object
  taking two arguments (see below), or
  one of the special values \constant{signal.SIG_IGN} or
  \constant{signal.SIG_DFL}.  The previous signal handler will be returned
  (see the description of \function{getsignal()} above).  (See the
  \UNIX{} man page \manpage{signal}{2}.)

  When threads are enabled, this function can only be called from the
  main thread; attempting to call it from other threads will cause a
  \exception{ValueError} exception to be raised.

  The \var{handler} is called with two arguments: the signal number
  and the current stack frame (\code{None} or a frame object;
  for a description of frame objects, see the reference manual section
  on the standard type hierarchy or see the attribute descriptions in
  the \refmodule{inspect} module).
\end{funcdesc}

\subsection{Example}
\nodename{Signal Example}

Here is a minimal example program. It uses the \function{alarm()}
function to limit the time spent waiting to open a file; this is
useful if the file is for a serial device that may not be turned on,
which would normally cause the \function{os.open()} to hang
indefinitely.  The solution is to set a 5-second alarm before opening
the file; if the operation takes too long, the alarm signal will be
sent, and the handler raises an exception.

\begin{verbatim}
import signal, os

def handler(signum, frame):
    print 'Signal handler called with signal', signum
    raise IOError, "Couldn't open device!"

# Set the signal handler and a 5-second alarm
signal.signal(signal.SIGALRM, handler)
signal.alarm(5)

# This open() may hang indefinitely
fd = os.open('/dev/ttyS0', os.O_RDWR)  

signal.alarm(0)          # Disable the alarm
\end{verbatim}

\section{\module{socket} ---
         Low-level networking interface}

\declaremodule{builtin}{socket}
\modulesynopsis{Low-level networking interface.}


This module provides access to the BSD \emph{socket} interface.
It is available on \UNIX{} systems that support this interface.

For an introduction to socket programming (in C), see the following
papers: \emph{An Introductory 4.3BSD Interprocess Communication
Tutorial}, by Stuart Sechrest and \emph{An Advanced 4.3BSD Interprocess
Communication Tutorial}, by Samuel J.  Leffler et al, both in the
\UNIX{} Programmer's Manual, Supplementary Documents 1 (sections PS1:7
and PS1:8).  The \UNIX{} manual pages for the various socket-related
system calls are also a valuable source of information on the details of
socket semantics.

The Python interface is a straightforward transliteration of the
\UNIX{} system call and library interface for sockets to Python's
object-oriented style: the \function{socket()} function returns a
\dfn{socket object}\obindex{socket} whose methods implement the
various socket system calls.  Parameter types are somewhat
higher-level than in the C interface: as with \method{read()} and
\method{write()} operations on Python files, buffer allocation on
receive operations is automatic, and buffer length is implicit on send
operations.

Socket addresses are represented as a single string for the
\constant{AF_UNIX} address family and as a pair
\code{(\var{host}, \var{port})} for the \constant{AF_INET} address
family, where \var{host} is a string representing
either a hostname in Internet domain notation like
\code{'daring.cwi.nl'} or an IP address like \code{'100.50.200.5'},
and \var{port} is an integral port number.  Other address families are
currently not supported.  The address format required by a particular
socket object is automatically selected based on the address family
specified when the socket object was created.

For IP addresses, two special forms are accepted instead of a host
address: the empty string represents \constant{INADDR_ANY}, and the string
\code{'<broadcast>'} represents \constant{INADDR_BROADCAST}.

All errors raise exceptions.  The normal exceptions for invalid
argument types and out-of-memory conditions can be raised; errors
related to socket or address semantics raise the error
\exception{socket.error}.

Non-blocking mode is supported through the
\method{setblocking()} method.

The module \module{socket} exports the following constants and functions:


\begin{excdesc}{error}
This exception is raised for socket- or address-related errors.
The accompanying value is either a string telling what went wrong or a
pair \code{(\var{errno}, \var{string})}
representing an error returned by a system
call, similar to the value accompanying \exception{os.error}.
See the module \refmodule{errno}\refbimodindex{errno}, which contains
names for the error codes defined by the underlying operating system.
\end{excdesc}

\begin{datadesc}{AF_UNIX}
\dataline{AF_INET}
These constants represent the address (and protocol) families,
used for the first argument to \function{socket()}.  If the
\constant{AF_UNIX} constant is not defined then this protocol is
unsupported.
\end{datadesc}

\begin{datadesc}{SOCK_STREAM}
\dataline{SOCK_DGRAM}
\dataline{SOCK_RAW}
\dataline{SOCK_RDM}
\dataline{SOCK_SEQPACKET}
These constants represent the socket types,
used for the second argument to \function{socket()}.
(Only \constant{SOCK_STREAM} and
\constant{SOCK_DGRAM} appear to be generally useful.)
\end{datadesc}

\begin{datadesc}{SO_*}
\dataline{SOMAXCONN}
\dataline{MSG_*}
\dataline{SOL_*}
\dataline{IPPROTO_*}
\dataline{IPPORT_*}
\dataline{INADDR_*}
\dataline{IP_*}
Many constants of these forms, documented in the \UNIX{} documentation on
sockets and/or the IP protocol, are also defined in the socket module.
They are generally used in arguments to the \method{setsockopt()} and
\method{getsockopt()} methods of socket objects.  In most cases, only
those symbols that are defined in the \UNIX{} header files are defined;
for a few symbols, default values are provided.
\end{datadesc}

\begin{funcdesc}{gethostbyname}{hostname}
Translate a host name to IP address format.  The IP address is
returned as a string, e.g.,  \code{'100.50.200.5'}.  If the host name
is an IP address itself it is returned unchanged.  See
\function{gethostbyname_ex()} for a more complete interface.
\end{funcdesc}

\begin{funcdesc}{gethostbyname_ex}{hostname}
Translate a host name to IP address format, extended interface.
Return a triple \code{(hostname, aliaslist, ipaddrlist)} where
\code{hostname} is the primary host name responding to the given
\var{ip_address}, \code{aliaslist} is a (possibly empty) list of
alternative host names for the same address, and \code{ipaddrlist} is
a list of IP addresses for the same interface on the same
host (often but not always a single address).
\end{funcdesc}

\begin{funcdesc}{gethostname}{}
Return a string containing the hostname of the machine where 
the Python interpreter is currently executing.  If you want to know the
current machine's IP address, use \code{gethostbyname(gethostname())}.
Note: \function{gethostname()} doesn't always return the fully qualified
domain name; use \code{gethostbyaddr(gethostname())}
(see below).
\end{funcdesc}

\begin{funcdesc}{gethostbyaddr}{ip_address}
Return a triple \code{(\var{hostname}, \var{aliaslist},
\var{ipaddrlist})} where \var{hostname} is the primary host name
responding to the given \var{ip_address}, \var{aliaslist} is a
(possibly empty) list of alternative host names for the same address,
and \var{ipaddrlist} is a list of IP addresses for the same interface
on the same host (most likely containing only a single address).
To find the fully qualified domain name, check \var{hostname} and the
items of \var{aliaslist} for an entry containing at least one period.
\end{funcdesc}

\begin{funcdesc}{getprotobyname}{protocolname}
Translate an Internet protocol name (e.g.\ \code{'icmp'}) to a constant
suitable for passing as the (optional) third argument to the
\function{socket()} function.  This is usually only needed for sockets
opened in ``raw'' mode (\constant{SOCK_RAW}); for the normal socket
modes, the correct protocol is chosen automatically if the protocol is
omitted or zero.
\end{funcdesc}

\begin{funcdesc}{getservbyname}{servicename, protocolname}
Translate an Internet service name and protocol name to a port number
for that service.  The protocol name should be \code{'tcp'} or
\code{'udp'}.
\end{funcdesc}

\begin{funcdesc}{socket}{family, type\optional{, proto}}
Create a new socket using the given address family, socket type and
protocol number.  The address family should be \constant{AF_INET} or
\constant{AF_UNIX}.  The socket type should be \constant{SOCK_STREAM},
\constant{SOCK_DGRAM} or perhaps one of the other \samp{SOCK_} constants.
The protocol number is usually zero and may be omitted in that case.
\end{funcdesc}

\begin{funcdesc}{fromfd}{fd, family, type\optional{, proto}}
Build a socket object from an existing file descriptor (an integer as
returned by a file object's \method{fileno()} method).  Address family,
socket type and protocol number are as for the \function{socket()} function
above.  The file descriptor should refer to a socket, but this is not
checked --- subsequent operations on the object may fail if the file
descriptor is invalid.  This function is rarely needed, but can be
used to get or set socket options on a socket passed to a program as
standard input or output (e.g.\ a server started by the \UNIX{} inet
daemon).
\end{funcdesc}

\begin{funcdesc}{ntohl}{x}
Convert 32-bit integers from network to host byte order.  On machines
where the host byte order is the same as network byte order, this is a
no-op; otherwise, it performs a 4-byte swap operation.
\end{funcdesc}

\begin{funcdesc}{ntohs}{x}
Convert 16-bit integers from network to host byte order.  On machines
where the host byte order is the same as network byte order, this is a
no-op; otherwise, it performs a 2-byte swap operation.
\end{funcdesc}

\begin{funcdesc}{htonl}{x}
Convert 32-bit integers from host to network byte order.  On machines
where the host byte order is the same as network byte order, this is a
no-op; otherwise, it performs a 4-byte swap operation.
\end{funcdesc}

\begin{funcdesc}{htons}{x}
Convert 16-bit integers from host to network byte order.  On machines
where the host byte order is the same as network byte order, this is a
no-op; otherwise, it performs a 2-byte swap operation.
\end{funcdesc}

\begin{datadesc}{SocketType}
This is a Python type object that represents the socket object type.
It is the same as \code{type(socket(...))}.
\end{datadesc}

\subsection{Socket Objects}

Socket objects have the following methods.  Except for
\method{makefile()} these correspond to \UNIX{} system calls
applicable to sockets.

\begin{methoddesc}[socket]{accept}{}
Accept a connection.
The socket must be bound to an address and listening for connections.
The return value is a pair \code{(\var{conn}, \var{address})}
where \var{conn} is a \emph{new} socket object usable to send and
receive data on the connection, and \var{address} is the address bound
to the socket on the other end of the connection.
\end{methoddesc}

\begin{methoddesc}[socket]{bind}{address}
Bind the socket to \var{address}.  The socket must not already be bound.
(The format of \var{address} depends on the address family --- see above.)
\end{methoddesc}

\begin{methoddesc}[socket]{close}{}
Close the socket.  All future operations on the socket object will fail.
The remote end will receive no more data (after queued data is flushed).
Sockets are automatically closed when they are garbage-collected.
\end{methoddesc}

\begin{methoddesc}[socket]{connect}{address}
Connect to a remote socket at \var{address}.
(The format of \var{address} depends on the address family --- see
above.)
\end{methoddesc}

\begin{methoddesc}[socket]{connect_ex}{address}
Like \code{connect(\var{address})}, but return an error indicator
instead of raising an exception for errors returned by the C-level
\cfunction{connect()} call (other problems, such as ``host not found,''
can still raise exceptions).  The error indicator is \code{0} if the
operation succeeded, otherwise the value of the \cdata{errno}
variable.  This is useful, e.g., for asynchronous connects.
\end{methoddesc}

\begin{methoddesc}[socket]{fileno}{}
Return the socket's file descriptor (a small integer).  This is useful
with \function{select.select()}.
\end{methoddesc}

\begin{methoddesc}[socket]{getpeername}{}
Return the remote address to which the socket is connected.  This is
useful to find out the port number of a remote IP socket, for instance.
(The format of the address returned depends on the address family ---
see above.)  On some systems this function is not supported.
\end{methoddesc}

\begin{methoddesc}[socket]{getsockname}{}
Return the socket's own address.  This is useful to find out the port
number of an IP socket, for instance.
(The format of the address returned depends on the address family ---
see above.)
\end{methoddesc}

\begin{methoddesc}[socket]{getsockopt}{level, optname\optional{, buflen}}
Return the value of the given socket option (see the \UNIX{} man page
\manpage{getsockopt}{2}).  The needed symbolic constants
(\constant{SO_*} etc.) are defined in this module.  If \var{buflen}
is absent, an integer option is assumed and its integer value
is returned by the function.  If \var{buflen} is present, it specifies
the maximum length of the buffer used to receive the option in, and
this buffer is returned as a string.  It is up to the caller to decode
the contents of the buffer (see the optional built-in module
\refmodule{struct} for a way to decode C structures encoded as strings).
\end{methoddesc}

\begin{methoddesc}[socket]{listen}{backlog}
Listen for connections made to the socket.  The \var{backlog} argument
specifies the maximum number of queued connections and should be at
least 1; the maximum value is system-dependent (usually 5).
\end{methoddesc}

\begin{methoddesc}[socket]{makefile}{\optional{mode\optional{, bufsize}}}
Return a \dfn{file object} associated with the socket.  (File objects
were described earlier in \ref{bltin-file-objects}, ``File Objects.'')
The file object references a \cfunction{dup()}ped version of the
socket file descriptor, so the file object and socket object may be
closed or garbage-collected independently.  The optional \var{mode}
and \var{bufsize} arguments are interpreted the same way as by the
built-in \function{open()} function.
\end{methoddesc}

\begin{methoddesc}[socket]{recv}{bufsize\optional{, flags}}
Receive data from the socket.  The return value is a string representing
the data received.  The maximum amount of data to be received
at once is specified by \var{bufsize}.  See the \UNIX{} manual page
\manpage{recv}{2} for the meaning of the optional argument
\var{flags}; it defaults to zero.
\end{methoddesc}

\begin{methoddesc}[socket]{recvfrom}{bufsize\optional{, flags}}
Receive data from the socket.  The return value is a pair
\code{(\var{string}, \var{address})} where \var{string} is a string
representing the data received and \var{address} is the address of the
socket sending the data.  The optional \var{flags} argument has the
same meaning as for \method{recv()} above.
(The format of \var{address} depends on the address family --- see above.)
\end{methoddesc}

\begin{methoddesc}[socket]{send}{string\optional{, flags}}
Send data to the socket.  The socket must be connected to a remote
socket.  The optional \var{flags} argument has the same meaning as for
\method{recv()} above.  Returns the number of bytes sent.
\end{methoddesc}

\begin{methoddesc}[socket]{sendto}{string\optional{, flags}, address}
Send data to the socket.  The socket should not be connected to a
remote socket, since the destination socket is specified by
\var{address}.  The optional \var{flags} argument has the same
meaning as for \method{recv()} above.  Return the number of bytes sent.
(The format of \var{address} depends on the address family --- see above.)
\end{methoddesc}

\begin{methoddesc}[socket]{setblocking}{flag}
Set blocking or non-blocking mode of the socket: if \var{flag} is 0,
the socket is set to non-blocking, else to blocking mode.  Initially
all sockets are in blocking mode.  In non-blocking mode, if a
\method{recv()} call doesn't find any data, or if a
\method{send()} call can't immediately dispose of the data, a
\exception{error} exception is raised; in blocking mode, the calls
block until they can proceed.
\end{methoddesc}

\begin{methoddesc}[socket]{setsockopt}{level, optname, value}
Set the value of the given socket option (see the \UNIX{} man page
\manpage{setsockopt}{2}).  The needed symbolic constants are defined in
the \module{socket} module (\code{SO_*} etc.).  The value can be an
integer or a string representing a buffer.  In the latter case it is
up to the caller to ensure that the string contains the proper bits
(see the optional built-in module
\refmodule{struct}\refbimodindex{struct} for a way to encode C
structures as strings). 
\end{methoddesc}

\begin{methoddesc}[socket]{shutdown}{how}
Shut down one or both halves of the connection.  If \var{how} is
\code{0}, further receives are disallowed.  If \var{how} is \code{1},
further sends are disallowed.  If \var{how} is \code{2}, further sends
and receives are disallowed.
\end{methoddesc}

Note that there are no methods \method{read()} or \method{write()};
use \method{recv()} and \method{send()} without \var{flags} argument
instead.

\subsection{Example}
\nodename{Socket Example}

Here are two minimal example programs using the TCP/IP protocol:\ a
server that echoes all data that it receives back (servicing only one
client), and a client using it.  Note that a server must perform the
sequence \function{socket()}, \method{bind()}, \method{listen()},
\method{accept()} (possibly repeating the \method{accept()} to service
more than one client), while a client only needs the sequence
\function{socket()}, \method{connect()}.  Also note that the server
does not \method{send()}/\method{recv()} on the 
socket it is listening on but on the new socket returned by
\method{accept()}.

\begin{verbatim}
# Echo server program
from socket import *
HOST = ''                 # Symbolic name meaning the local host
PORT = 50007              # Arbitrary non-privileged server
s = socket(AF_INET, SOCK_STREAM)
s.bind(HOST, PORT)
s.listen(1)
conn, addr = s.accept()
print 'Connected by', addr
while 1:
    data = conn.recv(1024)
    if not data: break
    conn.send(data)
conn.close()
\end{verbatim}

\begin{verbatim}
# Echo client program
from socket import *
HOST = 'daring.cwi.nl'    # The remote host
PORT = 50007              # The same port as used by the server
s = socket(AF_INET, SOCK_STREAM)
s.connect(HOST, PORT)
s.send('Hello, world')
data = s.recv(1024)
s.close()
print 'Received', `data`
\end{verbatim}

\begin{seealso}
\seemodule{SocketServer}{classes that simplify writing network servers}
\end{seealso}

\section{\module{select} ---
         Waiting for I/O completion}

\declaremodule{builtin}{select}
\modulesynopsis{Wait for I/O completion on multiple streams.}


This module provides access to the \cfunction{select()}
and \cfunction{poll()} functions
available in most operating systems.  Note that on Windows, it only
works for sockets; on other operating systems, it also works for other
file types (in particular, on \UNIX{}, it works on pipes).  It cannot
be used on regular files to determine whether a file has grown since
it was last read.

The module defines the following:

\begin{excdesc}{error}
The exception raised when an error occurs.  The accompanying value is
a pair containing the numeric error code from \cdata{errno} and the
corresponding string, as would be printed by the \C{} function
\cfunction{perror()}.
\end{excdesc}

\begin{funcdesc}{poll}{}
(Not supported by all operating systems.)  Returns a polling object, 
which supports registering and unregistering file descriptors, and
then polling them for I/O events; 
see section~\ref{poll-objects} below for the methods supported by 
polling objects.
\end{funcdesc}

\begin{funcdesc}{select}{iwtd, owtd, ewtd\optional{, timeout}}
This is a straightforward interface to the \UNIX{} \cfunction{select()}
system call.  The first three arguments are lists of `waitable
objects': either integers representing file descriptors or
objects with a parameterless method named \method{fileno()} returning
such an integer.  The three lists of waitable objects are for input,
output and `exceptional conditions', respectively.  Empty lists are
allowed, but acceptance of three empty lists is platform-dependent.
(It is known to work on \UNIX{} but not on Windows.)  The optional
\var{timeout} argument specifies a time-out as a floating point number
in seconds.  When the \var{timeout} argument is omitted the function
blocks until at least one file descriptor is ready.  A time-out value
of zero specifies a poll and never blocks.

The return value is a triple of lists of objects that are ready:
subsets of the first three arguments.  When the time-out is reached
without a file descriptor becoming ready, three empty lists are
returned.

Amongst the acceptable object types in the lists are Python file
objects (e.g. \code{sys.stdin}, or objects returned by
\function{open()} or \function{os.popen()}), socket objects
returned by \function{socket.socket()},%
\withsubitem{(in module socket)}{\ttindex{socket()}}
\withsubitem{(in module os)}{\ttindex{popen()}}.
You may also define a \dfn{wrapper} class yourself, as long as it has
an appropriate \method{fileno()} method (that really returns a file
descriptor, not just a random integer).
\strong{Note:}\index{WinSock}  File objects on Windows are not
acceptable, but sockets are.  On Windows, the underlying
\cfunction{select()} function is provided by the WinSock library, and
does not handle file desciptors that don't originate from WinSock.
\end{funcdesc}

\subsection{Polling Objects
            \label{poll-objects}}

The \cfunction{poll()} system call, supported on most Unix systems,
provides better scalability for network servers that service many,
many clients at the same time.
\cfunction{poll()} scales better because the system call only
requires listing the file descriptors of interest, while \cfunction{select()}
builds a bitmap, turns on bits for the fds of interest, and then
afterward the whole bitmap has to be linearly scanned again.
\cfunction{select()} is O(highest file descriptor), while
\cfunction{poll()} is O(number of file descriptors).

\begin{methoddesc}{register}{fd\optional{, eventmask}}
Register a file descriptor with the polling object.  Future calls to
the \method{poll()} method will then check whether the file descriptor
has any pending I/O events.  \var{fd} can be either an integer, or an
object with a \method{fileno()} method that returns an integer.  File
objects implement
\method{fileno()}, so they can also be used as the argument.

\var{eventmask} is an optional bitmask describing the type of events you
want to check for, and can be a combination of the constants
\constant{POLLIN}, \constant{POLLPRI}, and \constant{POLLOUT},
described in the table below.  If not specified, the default value
used will check for all 3 types of events.

\begin{tableii}{l|l}{constant}{Constant}{Meaning}
  \lineii{POLLIN}{There is data to read}
  \lineii{POLLPRI}{There is urgent data to read}
  \lineii{POLLOUT}{Ready for output: writing will not block}
  \lineii{POLLERR}{Error condition of some sort}
  \lineii{POLLHUP}{Hung up}
  \lineii{POLLNVAL}{Invalid request: descriptor not open}
\end{tableii}

Registering a file descriptor that's already registered is not an
error, and has the same effect as registering the descriptor exactly
once. 
 
\end{methoddesc}

\begin{methoddesc}{unregister}{fd}
Remove a file descriptor being tracked by a polling object.  Just like
the \method{register()} method, \var{fd} can be an integer or an
object with a \method{fileno()} method that returns an integer.

Attempting to remove a file descriptor that was never registered
causes a \exception{KeyError} exception to be raised.
\end{methoddesc}

\begin{methoddesc}{poll}{\optional{timeout}}
Polls the set of registered file descriptors, and returns a
possibly-empty list containing \code{(\var{fd}, \var{event})} 2-tuples
for the descriptors that have events or errors to report.
\var{fd} is the file descriptor, and \var{event} is a bitmask 
with bits set for the reported events for that descriptor
--- \constant{POLLIN} for waiting input, 
\constant{POLLOUT} to indicate that the descriptor can be written to, and
so forth.
An empty list indicates that the call timed out and no file
descriptors had any events to report.
If \var{timeout} is given, it specifies the length of time in
milliseconds which the system will wait for events before returning.
If \var{timeout} is omitted, negative, or \code{None}, the call will
block until there is an event for this poll object.
\end{methoddesc}



\section{\module{thread} ---
         Multiple threads of control}

\declaremodule{builtin}{thread}
\modulesynopsis{Create multiple threads of control within one interpreter.}


This module provides low-level primitives for working with multiple
threads (a.k.a.\ \dfn{light-weight processes} or \dfn{tasks}) --- multiple
threads of control sharing their global data space.  For
synchronization, simple locks (a.k.a.\ \dfn{mutexes} or \dfn{binary
semaphores}) are provided.
\index{light-weight processes}
\index{processes, light-weight}
\index{binary semaphores}
\index{semaphores, binary}

The module is optional.  It is supported on Windows NT and '95, SGI
IRIX, Solaris 2.x, as well as on systems that have a \POSIX{} thread
(a.k.a. ``pthread'') implementation.
\index{pthreads}
\indexii{threads}{\POSIX{}}

It defines the following constant and functions:

\begin{excdesc}{error}
Raised on thread-specific errors.
\end{excdesc}

\begin{datadesc}{LockType}
This is the type of lock objects.
\end{datadesc}

\begin{funcdesc}{start_new_thread}{function, args\optional{, kwargs}}
Start a new thread.  The thread executes the function \var{function}
with the argument list \var{args} (which must be a tuple).  The
optional \var{kwargs} argument specifies a dictionary of keyword
arguments.  When the
function returns, the thread silently exits.  When the function
terminates with an unhandled exception, a stack trace is printed and
then the thread exits (but other threads continue to run).
\end{funcdesc}

\begin{funcdesc}{exit}{}
Raise the \exception{SystemExit} exception.  When not caught, this
will cause the thread to exit silently.
\end{funcdesc}

\begin{funcdesc}{exit_thread}{}
\deprecated{1.5.2}{Use \function{exit()}.}
This is an obsolete synonym for \function{exit()}.
\end{funcdesc}

%\begin{funcdesc}{exit_prog}{status}
%Exit all threads and report the value of the integer argument
%\var{status} as the exit status of the entire program.
%\strong{Caveat:} code in pending \keyword{finally} clauses, in this thread
%or in other threads, is not executed.
%\end{funcdesc}

\begin{funcdesc}{allocate_lock}{}
Return a new lock object.  Methods of locks are described below.  The
lock is initially unlocked.
\end{funcdesc}

\begin{funcdesc}{get_ident}{}
Return the `thread identifier' of the current thread.  This is a
nonzero integer.  Its value has no direct meaning; it is intended as a
magic cookie to be used e.g. to index a dictionary of thread-specific
data.  Thread identifiers may be recycled when a thread exits and
another thread is created.
\end{funcdesc}


Lock objects have the following methods:

\begin{methoddesc}[lock]{acquire}{\optional{waitflag}}
Without the optional argument, this method acquires the lock
unconditionally, if necessary waiting until it is released by another
thread (only one thread at a time can acquire a lock --- that's their
reason for existence), and returns \code{None}.  If the integer
\var{waitflag} argument is present, the action depends on its
value: if it is zero, the lock is only acquired if it can be acquired
immediately without waiting, while if it is nonzero, the lock is
acquired unconditionally as before.  If an argument is present, the
return value is \code{1} if the lock is acquired successfully,
\code{0} if not.
\end{methoddesc}

\begin{methoddesc}[lock]{release}{}
Releases the lock.  The lock must have been acquired earlier, but not
necessarily by the same thread.
\end{methoddesc}

\begin{methoddesc}[lock]{locked}{}
Return the status of the lock:\ \code{1} if it has been acquired by
some thread, \code{0} if not.
\end{methoddesc}

\strong{Caveats:}

\begin{itemize}
\item
Threads interact strangely with interrupts: the
\exception{KeyboardInterrupt} exception will be received by an
arbitrary thread.  (When the \refmodule{signal}\refbimodindex{signal}
module is available, interrupts always go to the main thread.)

\item
Calling \function{sys.exit()} or raising the \exception{SystemExit}
exception is equivalent to calling \function{exit()}.

\item
Not all built-in functions that may block waiting for I/O allow other
threads to run.  (The most popular ones (\function{time.sleep()},
\method{\var{file}.read()}, \function{select.select()}) work as
expected.)

\item
It is not possible to interrupt the \method{acquire()} method on a lock
--- the \exception{KeyboardInterrupt} exception will happen after the
lock has been acquired.

\item
When the main thread exits, it is system defined whether the other
threads survive.  On SGI IRIX using the native thread implementation,
they survive.  On most other systems, they are killed without
executing \keyword{try} ... \keyword{finally} clauses or executing
object destructors.
\indexii{threads}{IRIX}

\item
When the main thread exits, it does not do any of its usual cleanup
(except that \keyword{try} ... \keyword{finally} clauses are honored),
and the standard I/O files are not flushed.

\end{itemize}

\section{\module{threading} ---
         Higher-level threading interface}

\declaremodule{standard}{threading}
\modulesynopsis{Higher-level threading interface.}


This module constructs higher-level threading interfaces on top of the 
lower level \refmodule{thread} module.

The \refmodule[dummythreading]{dummy_threading} module is provided for
situations where \module{threading} cannot be used because
\refmodule{thread} is missing.

This module defines the following functions and objects:

\begin{funcdesc}{activeCount}{}
Return the number of currently active \class{Thread} objects.
The returned count is equal to the length of the list returned by
\function{enumerate()}.
A function that returns the number of currently active threads.
\end{funcdesc}

\begin{funcdesc}{Condition}{}
A factory function that returns a new condition variable object.
A condition variable allows one or more threads to wait until they
are notified by another thread.
\end{funcdesc}

\begin{funcdesc}{currentThread}{}
Return the current \class{Thread} object, corresponding to the
caller's thread of control.  If the caller's thread of control was not
created through the
\module{threading} module, a dummy thread object with limited functionality
is returned.
\end{funcdesc}

\begin{funcdesc}{enumerate}{}
Return a list of all currently active \class{Thread} objects.
The list includes daemonic threads, dummy thread objects created
by \function{currentThread()}, and the main thread.  It excludes terminated
threads and threads that have not yet been started.
\end{funcdesc}

\begin{funcdesc}{Event}{}
A factory function that returns a new event object.  An event manages
a flag that can be set to true with the \method{set()} method and
reset to false with the \method{clear()} method.  The \method{wait()}
method blocks until the flag is true.
\end{funcdesc}

\begin{classdesc*}{local}{}
A class that represents thread-local data.  Thread-local data are data
whose values are thread specific.  To manage thread-local data, just
create an instance of \class{local} (or a subclass) and store
attributes on it:

\begin{verbatim}
mydata = threading.local()
mydata.x = 1
\end{verbatim}

The instance's values will be different for separate threads.

For more details and extensive examples, see the documentation string
of the \module{_threading_local} module.

\versionadded{2.4}
\end{classdesc*}

\begin{funcdesc}{Lock}{}
A factory function that returns a new primitive lock object.  Once
a thread has acquired it, subsequent attempts to acquire it block,
until it is released; any thread may release it.
\end{funcdesc}

\begin{funcdesc}{RLock}{}
A factory function that returns a new reentrant lock object.
A reentrant lock must be released by the thread that acquired it.
Once a thread has acquired a reentrant lock, the same thread may
acquire it again without blocking; the thread must release it once
for each time it has acquired it.
\end{funcdesc}

\begin{funcdesc}{Semaphore}{\optional{value}}
A factory function that returns a new semaphore object.  A
semaphore manages a counter representing the number of \method{release()}
calls minus the number of \method{acquire()} calls, plus an initial value.
The \method{acquire()} method blocks if necessary until it can return
without making the counter negative.  If not given, \var{value} defaults to
1. 
\end{funcdesc}

\begin{funcdesc}{BoundedSemaphore}{\optional{value}}
A factory function that returns a new bounded semaphore object.  A bounded
semaphore checks to make sure its current value doesn't exceed its initial
value.  If it does, \exception{ValueError} is raised. In most situations
semaphores are used to guard resources with limited capacity.  If the
semaphore is released too many times it's a sign of a bug.  If not given,
\var{value} defaults to 1. 
\end{funcdesc}

\begin{classdesc*}{Thread}{}
A class that represents a thread of control.  This class can be safely
subclassed in a limited fashion.
\end{classdesc*}

\begin{classdesc*}{Timer}{}
A thread that executes a function after a specified interval has passed.
\end{classdesc*}

\begin{funcdesc}{settrace}{func}
Set a trace function\index{trace function} for all threads started
from the \module{threading} module.  The \var{func} will be passed to 
\function{sys.settrace()} for each thread, before its \method{run()}
method is called.
\versionadded{2.3}
\end{funcdesc}

\begin{funcdesc}{setprofile}{func}
Set a profile function\index{profile function} for all threads started
from the \module{threading} module.  The \var{func} will be passed to 
\function{sys.setprofile()} for each thread, before its \method{run()}
method is called.
\versionadded{2.3}
\end{funcdesc}

\begin{funcdesc}{stack_size}{\optional{size}}
Return the thread stack size used when creating new threads.  The
optional \var{size} argument specifies the stack size to be used for
subsequently created threads, and must be 0 (use platform or
configured default) or a positive integer value of at least 32,768 (32kB).
If changing the thread stack size is unsupported, or the specified size
is invalid, a RuntimeWarning is issued and the stack size is unmodified.
32kB is currently the minimum supported stack size value, to guarantee
sufficient stack space for the interpreter itself.
Note that some platforms may have particular restrictions on values for
the stack size, such as requiring allocation in multiples of the system
memory page size - platform documentation should be referred to for more
information (4kB pages are common; using multiples of 4096 for the
stack size is the suggested approach in the absence of more specific
information).
Availability: Windows, systems with \POSIX{} threads.
\versionadded{2.5}
\end{funcdesc}

Detailed interfaces for the objects are documented below.  

The design of this module is loosely based on Java's threading model.
However, where Java makes locks and condition variables basic behavior
of every object, they are separate objects in Python.  Python's \class{Thread}
class supports a subset of the behavior of Java's Thread class;
currently, there are no priorities, no thread groups, and threads
cannot be destroyed, stopped, suspended, resumed, or interrupted.  The
static methods of Java's Thread class, when implemented, are mapped to
module-level functions.

All of the methods described below are executed atomically.


\subsection{Lock Objects \label{lock-objects}}

A primitive lock is a synchronization primitive that is not owned
by a particular thread when locked.  In Python, it is currently
the lowest level synchronization primitive available, implemented
directly by the \refmodule{thread} extension module.

A primitive lock is in one of two states, ``locked'' or ``unlocked''.
It is created in the unlocked state.  It has two basic methods,
\method{acquire()} and \method{release()}.  When the state is
unlocked, \method{acquire()} changes the state to locked and returns
immediately.  When the state is locked, \method{acquire()} blocks
until a call to \method{release()} in another thread changes it to
unlocked, then the \method{acquire()} call resets it to locked and
returns.  The \method{release()} method should only be called in the
locked state; it changes the state to unlocked and returns
immediately.  When more than one thread is blocked in
\method{acquire()} waiting for the state to turn to unlocked, only one
thread proceeds when a \method{release()} call resets the state to
unlocked; which one of the waiting threads proceeds is not defined,
and may vary across implementations.

All methods are executed atomically.

\begin{methoddesc}{acquire}{\optional{blocking\code{ = 1}}}
Acquire a lock, blocking or non-blocking.

When invoked without arguments, block until the lock is
unlocked, then set it to locked, and return true.  

When invoked with the \var{blocking} argument set to true, do the
same thing as when called without arguments, and return true.

When invoked with the \var{blocking} argument set to false, do not
block.  If a call without an argument would block, return false
immediately; otherwise, do the same thing as when called
without arguments, and return true.
\end{methoddesc}

\begin{methoddesc}{release}{}
Release a lock.

When the lock is locked, reset it to unlocked, and return.  If
any other threads are blocked waiting for the lock to become
unlocked, allow exactly one of them to proceed.

Do not call this method when the lock is unlocked.

There is no return value.
\end{methoddesc}


\subsection{RLock Objects \label{rlock-objects}}

A reentrant lock is a synchronization primitive that may be
acquired multiple times by the same thread.  Internally, it uses
the concepts of ``owning thread'' and ``recursion level'' in
addition to the locked/unlocked state used by primitive locks.  In
the locked state, some thread owns the lock; in the unlocked
state, no thread owns it.

To lock the lock, a thread calls its \method{acquire()} method; this
returns once the thread owns the lock.  To unlock the lock, a
thread calls its \method{release()} method.
\method{acquire()}/\method{release()} call pairs may be nested; only
the final \method{release()} (the \method{release()} of the outermost
pair) resets the lock to unlocked and allows another thread blocked in
\method{acquire()} to proceed.

\begin{methoddesc}{acquire}{\optional{blocking\code{ = 1}}}
Acquire a lock, blocking or non-blocking.

When invoked without arguments: if this thread already owns
the lock, increment the recursion level by one, and return
immediately.  Otherwise, if another thread owns the lock,
block until the lock is unlocked.  Once the lock is unlocked
(not owned by any thread), then grab ownership, set the
recursion level to one, and return.  If more than one thread
is blocked waiting until the lock is unlocked, only one at a
time will be able to grab ownership of the lock.  There is no
return value in this case.

When invoked with the \var{blocking} argument set to true, do the
same thing as when called without arguments, and return true.

When invoked with the \var{blocking} argument set to false, do not
block.  If a call without an argument would block, return false
immediately; otherwise, do the same thing as when called
without arguments, and return true.
\end{methoddesc}

\begin{methoddesc}{release}{}
Release a lock, decrementing the recursion level.  If after the
decrement it is zero, reset the lock to unlocked (not owned by any
thread), and if any other threads are blocked waiting for the lock to
become unlocked, allow exactly one of them to proceed.  If after the
decrement the recursion level is still nonzero, the lock remains
locked and owned by the calling thread.

Only call this method when the calling thread owns the lock.
Do not call this method when the lock is unlocked.

There is no return value.
\end{methoddesc}


\subsection{Condition Objects \label{condition-objects}}

A condition variable is always associated with some kind of lock;
this can be passed in or one will be created by default.  (Passing
one in is useful when several condition variables must share the
same lock.)

A condition variable has \method{acquire()} and \method{release()}
methods that call the corresponding methods of the associated lock.
It also has a \method{wait()} method, and \method{notify()} and
\method{notifyAll()} methods.  These three must only be called when
the calling thread has acquired the lock.

The \method{wait()} method releases the lock, and then blocks until it
is awakened by a \method{notify()} or \method{notifyAll()} call for
the same condition variable in another thread.  Once awakened, it
re-acquires the lock and returns.  It is also possible to specify a
timeout.

The \method{notify()} method wakes up one of the threads waiting for
the condition variable, if any are waiting.  The \method{notifyAll()}
method wakes up all threads waiting for the condition variable.

Note: the \method{notify()} and \method{notifyAll()} methods don't
release the lock; this means that the thread or threads awakened will
not return from their \method{wait()} call immediately, but only when
the thread that called \method{notify()} or \method{notifyAll()}
finally relinquishes ownership of the lock.

Tip: the typical programming style using condition variables uses the
lock to synchronize access to some shared state; threads that are
interested in a particular change of state call \method{wait()}
repeatedly until they see the desired state, while threads that modify
the state call \method{notify()} or \method{notifyAll()} when they
change the state in such a way that it could possibly be a desired
state for one of the waiters.  For example, the following code is a
generic producer-consumer situation with unlimited buffer capacity:

\begin{verbatim}
# Consume one item
cv.acquire()
while not an_item_is_available():
    cv.wait()
get_an_available_item()
cv.release()

# Produce one item
cv.acquire()
make_an_item_available()
cv.notify()
cv.release()
\end{verbatim}

To choose between \method{notify()} and \method{notifyAll()}, consider
whether one state change can be interesting for only one or several
waiting threads.  E.g. in a typical producer-consumer situation,
adding one item to the buffer only needs to wake up one consumer
thread.

\begin{classdesc}{Condition}{\optional{lock}}
If the \var{lock} argument is given and not \code{None}, it must be a
\class{Lock} or \class{RLock} object, and it is used as the underlying
lock.  Otherwise, a new \class{RLock} object is created and used as
the underlying lock.
\end{classdesc}

\begin{methoddesc}{acquire}{*args}
Acquire the underlying lock.
This method calls the corresponding method on the underlying
lock; the return value is whatever that method returns.
\end{methoddesc}

\begin{methoddesc}{release}{}
Release the underlying lock.
This method calls the corresponding method on the underlying
lock; there is no return value.
\end{methoddesc}

\begin{methoddesc}{wait}{\optional{timeout}}
Wait until notified or until a timeout occurs.
This must only be called when the calling thread has acquired the
lock.

This method releases the underlying lock, and then blocks until it is
awakened by a \method{notify()} or \method{notifyAll()} call for the
same condition variable in another thread, or until the optional
timeout occurs.  Once awakened or timed out, it re-acquires the lock
and returns.

When the \var{timeout} argument is present and not \code{None}, it
should be a floating point number specifying a timeout for the
operation in seconds (or fractions thereof).

When the underlying lock is an \class{RLock}, it is not released using
its \method{release()} method, since this may not actually unlock the
lock when it was acquired multiple times recursively.  Instead, an
internal interface of the \class{RLock} class is used, which really
unlocks it even when it has been recursively acquired several times.
Another internal interface is then used to restore the recursion level
when the lock is reacquired.
\end{methoddesc}

\begin{methoddesc}{notify}{}
Wake up a thread waiting on this condition, if any.
This must only be called when the calling thread has acquired the
lock.

This method wakes up one of the threads waiting for the condition
variable, if any are waiting; it is a no-op if no threads are waiting.

The current implementation wakes up exactly one thread, if any are
waiting.  However, it's not safe to rely on this behavior.  A future,
optimized implementation may occasionally wake up more than one
thread.

Note: the awakened thread does not actually return from its
\method{wait()} call until it can reacquire the lock.  Since
\method{notify()} does not release the lock, its caller should.
\end{methoddesc}

\begin{methoddesc}{notifyAll}{}
Wake up all threads waiting on this condition.  This method acts like
\method{notify()}, but wakes up all waiting threads instead of one.
\end{methoddesc}


\subsection{Semaphore Objects \label{semaphore-objects}}

This is one of the oldest synchronization primitives in the history of
computer science, invented by the early Dutch computer scientist
Edsger W. Dijkstra (he used \method{P()} and \method{V()} instead of
\method{acquire()} and \method{release()}).

A semaphore manages an internal counter which is decremented by each
\method{acquire()} call and incremented by each \method{release()}
call.  The counter can never go below zero; when \method{acquire()}
finds that it is zero, it blocks, waiting until some other thread
calls \method{release()}.

\begin{classdesc}{Semaphore}{\optional{value}}
The optional argument gives the initial value for the internal
counter; it defaults to \code{1}.
\end{classdesc}

\begin{methoddesc}{acquire}{\optional{blocking}}
Acquire a semaphore.

When invoked without arguments: if the internal counter is larger than
zero on entry, decrement it by one and return immediately.  If it is
zero on entry, block, waiting until some other thread has called
\method{release()} to make it larger than zero.  This is done with
proper interlocking so that if multiple \method{acquire()} calls are
blocked, \method{release()} will wake exactly one of them up.  The
implementation may pick one at random, so the order in which blocked
threads are awakened should not be relied on.  There is no return
value in this case.

When invoked with \var{blocking} set to true, do the same thing as
when called without arguments, and return true.

When invoked with \var{blocking} set to false, do not block.  If a
call without an argument would block, return false immediately;
otherwise, do the same thing as when called without arguments, and
return true.
\end{methoddesc}

\begin{methoddesc}{release}{}
Release a semaphore,
incrementing the internal counter by one.  When it was zero on
entry and another thread is waiting for it to become larger
than zero again, wake up that thread.
\end{methoddesc}


\subsubsection{\class{Semaphore} Example \label{semaphore-examples}}

Semaphores are often used to guard resources with limited capacity, for
example, a database server.  In any situation where the size of the resource
size is fixed, you should use a bounded semaphore.  Before spawning any
worker threads, your main thread would initialize the semaphore:

\begin{verbatim}
maxconnections = 5
...
pool_sema = BoundedSemaphore(value=maxconnections)
\end{verbatim}

Once spawned, worker threads call the semaphore's acquire and release
methods when they need to connect to the server:

\begin{verbatim}
pool_sema.acquire()
conn = connectdb()
... use connection ...
conn.close()
pool_sema.release()
\end{verbatim}

The use of a bounded semaphore reduces the chance that a programming error
which causes the semaphore to be released more than it's acquired will go
undetected.


\subsection{Event Objects \label{event-objects}}

This is one of the simplest mechanisms for communication between
threads: one thread signals an event and other threads wait for it.

An event object manages an internal flag that can be set to true with
the \method{set()} method and reset to false with the \method{clear()}
method.  The \method{wait()} method blocks until the flag is true.


\begin{classdesc}{Event}{}
The internal flag is initially false.
\end{classdesc}

\begin{methoddesc}{isSet}{}
Return true if and only if the internal flag is true.
\end{methoddesc}

\begin{methoddesc}{set}{}
Set the internal flag to true.
All threads waiting for it to become true are awakened.
Threads that call \method{wait()} once the flag is true will not block
at all.
\end{methoddesc}

\begin{methoddesc}{clear}{}
Reset the internal flag to false.
Subsequently, threads calling \method{wait()} will block until
\method{set()} is called to set the internal flag to true again.
\end{methoddesc}

\begin{methoddesc}{wait}{\optional{timeout}}
Block until the internal flag is true.
If the internal flag is true on entry, return immediately.  Otherwise,
block until another thread calls \method{set()} to set the flag to
true, or until the optional timeout occurs.

When the timeout argument is present and not \code{None}, it should be a
floating point number specifying a timeout for the operation in
seconds (or fractions thereof).
\end{methoddesc}


\subsection{Thread Objects \label{thread-objects}}

This class represents an activity that is run in a separate thread
of control.  There are two ways to specify the activity: by
passing a callable object to the constructor, or by overriding the
\method{run()} method in a subclass.  No other methods (except for the
constructor) should be overridden in a subclass.  In other words, 
\emph{only}  override the \method{__init__()} and \method{run()}
methods of this class.

Once a thread object is created, its activity must be started by
calling the thread's \method{start()} method.  This invokes the
\method{run()} method in a separate thread of control.

Once the thread's activity is started, the thread is considered
'alive' and 'active' (these concepts are almost, but not quite
exactly, the same; their definition is intentionally somewhat
vague).  It stops being alive and active when its \method{run()}
method terminates -- either normally, or by raising an unhandled
exception.  The \method{isAlive()} method tests whether the thread is
alive.

Other threads can call a thread's \method{join()} method.  This blocks
the calling thread until the thread whose \method{join()} method is
called is terminated.

A thread has a name.  The name can be passed to the constructor,
set with the \method{setName()} method, and retrieved with the
\method{getName()} method.

A thread can be flagged as a ``daemon thread''.  The significance
of this flag is that the entire Python program exits when only
daemon threads are left.  The initial value is inherited from the
creating thread.  The flag can be set with the \method{setDaemon()}
method and retrieved with the \method{isDaemon()} method.

There is a ``main thread'' object; this corresponds to the
initial thread of control in the Python program.  It is not a
daemon thread.

There is the possibility that ``dummy thread objects'' are
created.  These are thread objects corresponding to ``alien
threads''.  These are threads of control started outside the
threading module, such as directly from C code.  Dummy thread objects
have limited functionality; they are always considered alive,
active, and daemonic, and cannot be \method{join()}ed.  They are never 
deleted, since it is impossible to detect the termination of alien
threads.


\begin{classdesc}{Thread}{group=None, target=None, name=None,
                          args=(), kwargs=\{\}}
This constructor should always be called with keyword
arguments.  Arguments are:

\var{group} should be \code{None}; reserved for future extension when
a \class{ThreadGroup} class is implemented.

\var{target} is the callable object to be invoked by the
\method{run()} method.  Defaults to \code{None}, meaning nothing is
called.

\var{name} is the thread name.  By default, a unique name is
constructed of the form ``Thread-\var{N}'' where \var{N} is a small
decimal number.

\var{args} is the argument tuple for the target invocation.  Defaults
to \code{()}.

\var{kwargs} is a dictionary of keyword arguments for the target
invocation.  Defaults to \code{\{\}}.

If the subclass overrides the constructor, it must make sure
to invoke the base class constructor (\code{Thread.__init__()})
before doing anything else to the thread.
\end{classdesc}

\begin{methoddesc}{start}{}
Start the thread's activity.

This must be called at most once per thread object.  It
arranges for the object's \method{run()} method to be invoked in a
separate thread of control.
\end{methoddesc}

\begin{methoddesc}{run}{}
Method representing the thread's activity.

You may override this method in a subclass.  The standard
\method{run()} method invokes the callable object passed to the
object's constructor as the \var{target} argument, if any, with
sequential and keyword arguments taken from the \var{args} and
\var{kwargs} arguments, respectively.
\end{methoddesc}

\begin{methoddesc}{join}{\optional{timeout}}
Wait until the thread terminates.
This blocks the calling thread until the thread whose \method{join()}
method is called terminates -- either normally or through an
unhandled exception -- or until the optional timeout occurs.

When the \var{timeout} argument is present and not \code{None}, it
should be a floating point number specifying a timeout for the
operation in seconds (or fractions thereof). As \method{join()} always 
returns \code{None}, you must call \method{isAlive()} to decide whether 
a timeout happened.

When the \var{timeout} argument is not present or \code{None}, the
operation will block until the thread terminates.

A thread can be \method{join()}ed many times.

A thread cannot join itself because this would cause a
deadlock.

It is an error to attempt to \method{join()} a thread before it has
been started.
\end{methoddesc}

\begin{methoddesc}{getName}{}
Return the thread's name.
\end{methoddesc}

\begin{methoddesc}{setName}{name}
Set the thread's name.

The name is a string used for identification purposes only.
It has no semantics.  Multiple threads may be given the same
name.  The initial name is set by the constructor.
\end{methoddesc}

\begin{methoddesc}{isAlive}{}
Return whether the thread is alive.

Roughly, a thread is alive from the moment the \method{start()} method
returns until its \method{run()} method terminates.
\end{methoddesc}

\begin{methoddesc}{isDaemon}{}
Return the thread's daemon flag.
\end{methoddesc}

\begin{methoddesc}{setDaemon}{daemonic}
Set the thread's daemon flag to the Boolean value \var{daemonic}.
This must be called before \method{start()} is called.

The initial value is inherited from the creating thread.

The entire Python program exits when no active non-daemon
threads are left.
\end{methoddesc}


\subsection{Timer Objects \label{timer-objects}}

This class represents an action that should be run only after a
certain amount of time has passed --- a timer.  \class{Timer} is a
subclass of \class{Thread} and as such also functions as an example of
creating custom threads.

Timers are started, as with threads, by calling their \method{start()}
method.  The timer can be stopped (before its action has begun) by
calling the \method{cancel()} method.  The interval the timer will
wait before executing its action may not be exactly the same as the
interval specified by the user.

For example:
\begin{verbatim}
def hello():
    print "hello, world"

t = Timer(30.0, hello)
t.start() # after 30 seconds, "hello, world" will be printed
\end{verbatim}

\begin{classdesc}{Timer}{interval, function, args=[], kwargs=\{\}}
Create a timer that will run \var{function} with arguments \var{args} and 
keyword arguments \var{kwargs}, after \var{interval} seconds have passed.
\end{classdesc}

\begin{methoddesc}{cancel}{}
Stop the timer, and cancel the execution of the timer's action.  This
will only work if the timer is still in its waiting stage.
\end{methoddesc}

\subsection{Using locks, conditions, and semaphores in the \keyword{with}
statement \label{with-locks}}

All of the objects provided by this module that have \method{acquire()} and
\method{release()} methods can be used as context managers for a \keyword{with}
statement.  The \method{acquire()} method will be called when the block is
entered, and \method{release()} will be called when the block is exited.

Currently, \class{Lock}, \class{RLock}, \class{Condition}, \class{Semaphore},
and \class{BoundedSemaphore} objects may be used as \keyword{with}
statement context managers.  For example:

\begin{verbatim}
from __future__ import with_statement
import threading

some_rlock = threading.RLock()

with some_rlock:
    print "some_rlock is locked while this executes"
\end{verbatim}


\section{\module{Queue} ---
         A synchronized queue class.}
\declaremodule{standard}{Queue}

\modulesynopsis{A synchronized queue class.}



The \module{Queue} module implements a multi-producer, multi-consumer
FIFO queue.  It is especially useful in threads programming when
information must be exchanged safely between multiple threads.  The
\class{Queue} class in this module implements all the required locking
semantics.  It depends on the availability of thread support in
Python.

The \module{Queue} module defines the following class and exception:


\begin{classdesc}{Queue}{maxsize}
Constructor for the class.  \var{maxsize} is an integer that sets the
upperbound limit on the number of items that can be placed in the
queue.  Insertion will block once this size has been reached, until
queue items are consumed.  If \var{maxsize} is less than or equal to
zero, the queue size is infinite.
\end{classdesc}

\begin{excdesc}{Empty}
Exception raised when non-blocking get (e.g. \method{get_nowait()}) is
called on a \class{Queue} object which is empty, or for which the
emptyiness cannot be determined (i.e. because the appropriate locks
cannot be acquired).
\end{excdesc}

\subsection{Queue Objects}
\label{QueueObjects}

Class \class{Queue} implements queue objects and has the methods
described below.  This class can be derived from in order to implement
other queue organizations (e.g. stack) but the inheritable interface
is not described here.  See the source code for details.  The public
methods are:

\begin{methoddesc}{qsize}{}
Returns the approximate size of the queue.  Because of multithreading
semantics, this number is not reliable.
\end{methoddesc}

\begin{methoddesc}{empty}{}
Returns \code{1} if the queue is empty, \code{0} otherwise.  Because
of multithreading semantics, this is not reliable.
\end{methoddesc}

\begin{methoddesc}{full}{}
Returns \code{1} if the queue is full, \code{0} otherwise.  Because of
multithreading semantics, this is not reliable.
\end{methoddesc}

\begin{methoddesc}{put}{item}
Puts \var{item} into the queue.
\end{methoddesc}

\begin{methoddesc}{get}{}
Gets and returns an item from the queue, blocking if necessary until
one is available.
\end{methoddesc}

\begin{methoddesc}{get_nowait}{}
Gets and returns an item from the queue if one is immediately
available.  Raises an \exception{Empty} exception if the queue is
empty or if the queue's emptiness cannot be determined.
\end{methoddesc}

\section{\module{anydbm} ---
         Generic access to DBM-style databases}

\declaremodule{standard}{anydbm}
\modulesynopsis{Generic interface to DBM-style database modules.}


\module{anydbm} is a generic interface to variants of the DBM
database --- \refmodule{dbhash}\refstmodindex{dbhash} (requires
\refmodule{bsddb}\refbimodindex{bsddb}),
\refmodule{gdbm}\refbimodindex{gdbm}, or
\refmodule{dbm}\refbimodindex{dbm}.  If none of these modules is
installed, the slow-but-simple implementation in module
\refmodule{dumbdbm}\refstmodindex{dumbdbm} will be used.

\begin{funcdesc}{open}{filename\optional{, flag\optional{, mode}}}
Open the database file \var{filename} and return a corresponding object.

If the database file already exists, the \refmodule{whichdb} module is 
used to determine its type and the appropriate module is used; if it
does not exist, the first module listed above that can be imported is
used.

The optional \var{flag} argument can be
\code{'r'} to open an existing database for reading only,
\code{'w'} to open an existing database for reading and writing,
\code{'c'} to create the database if it doesn't exist, or
\code{'n'}, which will always create a new empty database.  If not
specified, the default value is \code{'r'}.

The optional \var{mode} argument is the \UNIX{} mode of the file, used
only when the database has to be created.  It defaults to octal
\code{0666} (and will be modified by the prevailing umask).
\end{funcdesc}

\begin{excdesc}{error}
A tuple containing the exceptions that can be raised by each of the
supported modules, with a unique exception \exception{anydbm.error} as
the first item --- the latter is used when \exception{anydbm.error} is
raised.
\end{excdesc}

The object returned by \function{open()} supports most of the same
functionality as dictionaries; keys and their corresponding values can
be stored, retrieved, and deleted, and the \method{has_key()} and
\method{keys()} methods are available.  Keys and values must always be
strings.


\begin{seealso}
  \seemodule{anydbm}{Generic interface to \code{dbm}-style databases.}
  \seemodule{dbhash}{BSD \code{db} database interface.}
  \seemodule{dbm}{Standard \UNIX{} database interface.}
  \seemodule{dumbdbm}{Portable implementation of the \code{dbm} interface.}
  \seemodule{gdbm}{GNU database interface, based on the \code{dbm} interface.}
  \seemodule{shelve}{General object persistence built on top of 
                     the Python \code{dbm} interface.}
  \seemodule{whichdb}{Utility module used to determine the type of an
                      existing database.}
\end{seealso}


\section{\module{dumbdbm} ---
         Portable DBM implementation}

\declaremodule{standard}{dumbdbm}
\modulesynopsis{Portable implementation of the simple DBM interface.}


A simple and slow database implemented entirely in Python.  This
should only be used when no other DBM-style database is available.


\begin{funcdesc}{open}{filename\optional{, flag\optional{, mode}}}
Open the database file \var{filename} and return a corresponding object.
The optional \var{flag} argument can be
\code{'r'} to open an existing database for reading only,
\code{'w'} to open an existing database for reading and writing,
\code{'c'} to create the database if it doesn't exist, or
\code{'n'}, which will always create a new empty database.  If not
specified, the default value is \code{'r'}.

The optional \var{mode} argument is the \UNIX{} mode of the file, used
only when the database has to be created.  It defaults to octal
\code{0666} (and will be modified by the prevailing umask).
\end{funcdesc}

\begin{excdesc}{error}
Raised for errors not reported as \exception{KeyError} errors.
\end{excdesc}


\begin{seealso}
  \seemodule{anydbm}{Generic interface to \code{dbm}-style databases.}
  \seemodule{whichdb}{Utility module used to determine the type of an
                      existing database.}
\end{seealso}

\section{Standard Module \module{whichdb}}
\declaremodule{standard}{whichdb}

\modulesynopsis{Guess which DBM-style module created a given database.}


The single function in this module attempts to guess which of the
several simple database modules available--dbm, gdbm, or
dbhash--should be used to open a given file.

\begin{funcdesc}{whichdb}{filename}
Returns one of the following values: \code{None} if the file can't be
opened because it's unreadable or doesn't exist; the empty string
(\code{""}) if the file's format can't be guessed; or a string
containing the required module name, such as \code{"dbm"} or
\code{"gdbm"}.
\end{funcdesc}


% XXX The module has been extended (by Jeremy) but this documentation
% hasn't been updated yet

\section{Built-in Module \module{zlib}}
\label{module-zlib}
\bimodindex{zlib}

For applications that require data compression, the functions in this
module allow compression and decompression, using the zlib library,
which is based on the GNU \program{gzip} program.  The zlib library
has its own home page at
\url{http://www.cdrom.com/pub/infozip/zlib/}.  Version 1.0.4 is the
most recent version as of December, 1997; use a later version if one
is available.

The available exception and functions in this module are:

\begin{excdesc}{error}
  Exception raised on compression and decompression errors.
\end{excdesc}


\begin{funcdesc}{adler32}{string\optional{, value}}
   Computes a Adler-32 checksum of \var{string}.  (An Adler-32
   checksum is almost as reliable as a CRC32 but can be computed much
   more quickly.)  If \var{value} is present, it is used as the
   starting value of the checksum; otherwise, a fixed default value is
   used.  This allows computing a running checksum over the
   concatenation of several input strings.  The algorithm is not
   cryptographically strong, and should not be used for
   authentication or digital signatures.
\end{funcdesc}

\begin{funcdesc}{compress}{string\optional{, level}}
Compresses the data in \var{string}, returning a string contained
compressed data.  \var{level} is an integer from \code{1} to \code{9}
controlling the level of compression; \code{1} is fastest and produces
the least compression, \code{9} is slowest and produces the most.  The
default value is \code{6}.  Raises the \exception{error}
exception if any error occurs.
\end{funcdesc}

\begin{funcdesc}{compressobj}{\optional{level}}
  Returns a compression object, to be used for compressing data streams
  that won't fit into memory at once.  \var{level} is an integer from
  \code{1} to \code{9} controlling the level of compression; \code{1} is
  fastest and produces the least compression, \code{9} is slowest and
  produces the most.  The default value is \code{6}.
\end{funcdesc}

\begin{funcdesc}{crc32}{string\optional{, value}}
  Computes a CRC (Cyclic Redundancy Check)%
  \index{Cyclic Redundancy Check}
  \index{checksum!Cyclic Redundancy Check}
  checksum of \var{string}. If
  \var{value} is present, it is used as the starting value of the
  checksum; otherwise, a fixed default value is used.  This allows
  computing a running checksum over the concatenation of several
  input strings.  The algorithm is not cryptographically strong, and
  should not be used for authentication or digital signatures.
\end{funcdesc}

\begin{funcdesc}{decompress}{string}
Decompresses the data in \var{string}, returning a string containing
the uncompressed data.  Raises the \exception{error} exception if any
error occurs.
\end{funcdesc}

\begin{funcdesc}{decompressobj}{\optional{wbits}}
Returns a compression object, to be used for decompressing data streams
that won't fit into memory at once.  The \var{wbits} parameter
controls the size of the window buffer; usually this can be left
alone.
\end{funcdesc}

Compression objects support the following methods:

\begin{methoddesc}[Compress]{compress}{string}
Compress \var{string}, returning a string containing compressed data
for at least part of the data in \var{string}.  This data should be
concatenated to the output produced by any preceding calls to the
\method{compress()} method.  Some input may be kept in internal buffers
for later processing.
\end{methoddesc}

\begin{methoddesc}[Compress]{flush}{}
All pending input is processed, and an string containing the remaining
compressed output is returned.  After calling \method{flush()}, the
\method{compress()} method cannot be called again; the only realistic
action is to delete the object.
\end{methoddesc}

Decompression objects support the following methods:

\begin{methoddesc}[Decompress]{decompress}{string}
Decompress \var{string}, returning a string containing the
uncompressed data corresponding to at least part of the data in
\var{string}.  This data should be concatenated to the output produced
by any preceding calls to the
\method{decompress()} method.  Some of the input data may be preserved
in internal buffers for later processing.
\end{methoddesc}

\begin{methoddesc}[Decompress]{flush}{}
All pending input is processed, and a string containing the remaining
uncompressed output is returned.  After calling \method{flush()}, the
\method{decompress()} method cannot be called again; the only realistic
action is to delete the object.
\end{methoddesc}

\begin{seealso}
\seemodule{gzip}{reading and writing \program{gzip}-format files}
\end{seealso}



\section{Built-in Module \sectcode{gzip}}
\label{module-gzip}
\bimodindex{gzip}

The data compression provided by the \code{zlib} module is compatible
with that used by the GNU compression program \file{gzip}.
Accordingly, the \code{gzip} module provides the \code{GzipFile} class
to read and write \file{gzip}-format files, automatically compressing
or decompressing the data so it looks like an ordinary file object.

\code{GzipFile} objects simulate most of the methods of a file
object, though it's not possible to use the \code{seek()} and
\code{tell()} methods to access the file randomly.

\setindexsubitem{(in module gzip)}
\begin{funcdesc}{open}{fileobj\optional{\, filename\optional{\, mode\, compresslevel}}}
  Returns a new \code{GzipFile} object on top of \var{fileobj}, which
  can be a regular file, a \code{StringIO} object, or any object which
  simulates a file.

  The \file{gzip} file format includes the original filename of the
  uncompressed file; when opening a \code{GzipFile} object for
  writing, it can be set by the \var{filename} argument.  The default
  value is an empty string.

  \var{mode} can be either \code{'r'} or \code{'w'} depending on
  whether the file will be read or written.  \var{compresslevel} is an
  integer from 1 to 9 controlling the level of compression; 1 is
  fastest and produces the least compression, and 9 is slowest and
  produces the most compression.  The default value of
  \var{compresslevel} is 9.

  Calling a \code{GzipFile} object's \code{close()} method does not
  close \var{fileobj}, since you might wish to append more material
  after the compressed data.  This also allows you to pass a
  \code{StringIO} object opened for writing as \var{fileobj}, and
  retrieve the resulting memory buffer using the \code{StringIO}
  object's \code{getvalue()} method.
\end{funcdesc}

\begin{seealso}
\seemodule{zlib}{the basic data compression module}
\end{seealso}



\chapter{Unix Specific Services}
\label{unix}

The modules described in this chapter provide interfaces to features
that are unique to the \UNIX{} operating system, or in some cases to
some or many variants of it.  Here's an overview:

\localmoduletable
			% UNIX Specific Services
\section{\module{posix} ---
         The most common \POSIX{} system calls}

\declaremodule{builtin}{posix}
  \platform{Unix}
\modulesynopsis{The most common \POSIX\ system calls (normally used
                via module \refmodule{os}).}


This module provides access to operating system functionality that is
standardized by the C Standard and the \POSIX{} standard (a thinly
disguised \UNIX{} interface).

\strong{Do not import this module directly.}  Instead, import the
module \refmodule{os}, which provides a \emph{portable} version of this
interface.  On \UNIX{}, the \refmodule{os} module provides a superset of
the \module{posix} interface.  On non-\UNIX{} operating systems the
\module{posix} module is not available, but a subset is always
available through the \refmodule{os} interface.  Once \refmodule{os} is
imported, there is \emph{no} performance penalty in using it instead
of \module{posix}.  In addition, \refmodule{os}\refstmodindex{os}
provides some additional functionality, such as automatically calling
\function{putenv()} when an entry in \code{os.environ} is changed.

The descriptions below are very terse; refer to the corresponding
\UNIX{} manual (or \POSIX{} documentation) entry for more information.
Arguments called \var{path} refer to a pathname given as a string.

Errors are reported as exceptions; the usual exceptions are given for
type errors, while errors reported by the system calls raise
\exception{error} (a synonym for the standard exception
\exception{OSError}), described below.


\subsection{Large File Support \label{posix-large-files}}
\sectionauthor{Steve Clift}{clift@mail.anacapa.net}
\index{large files}
\index{file!large files}


Several operating systems (including AIX, HPUX, Irix and Solaris)
provide support for files that are larger than 2 Gb from a C
programming model where \ctype{int} and \ctype{long} are 32-bit
values. This is typically accomplished by defining the relevant size
and offset types as 64-bit values. Such files are sometimes referred
to as \dfn{large files}.

Large file support is enabled in Python when the size of an
\ctype{off_t} is larger than a \ctype{long} and the \ctype{long long}
type is available and is at least as large as an \ctype{off_t}. Python
longs are then used to represent file sizes, offsets and other values
that can exceed the range of a Python int. It may be necessary to
configure and compile Python with certain compiler flags to enable
this mode. For example, it is enabled by default with recent versions
of Irix, but with Solaris 2.6 and 2.7 you need to do something like:

\begin{verbatim}
CFLAGS="`getconf LFS_CFLAGS`" OPT="-g -O2 $CFLAGS" \
        ./configure
\end{verbatim} % $ <-- bow to font-lock

On large-file-capable Linux systems, this might work:

\begin{verbatim}
CFLAGS='-D_LARGEFILE64_SOURCE -D_FILE_OFFSET_BITS=64' OPT="-g -O2 $CFLAGS" \
        ./configure
\end{verbatim} % $ <-- bow to font-lock


\subsection{Module Contents \label{posix-contents}}


Module \module{posix} defines the following data item:

\begin{datadesc}{environ}
A dictionary representing the string environment at the time the
interpreter was started. For example, \code{environ['HOME']} is the
pathname of your home directory, equivalent to
\code{getenv("HOME")} in C.

Modifying this dictionary does not affect the string environment
passed on by \function{execv()}, \function{popen()} or
\function{system()}; if you need to change the environment, pass
\code{environ} to \function{execve()} or add variable assignments and
export statements to the command string for \function{system()} or
\function{popen()}.

\note{The \refmodule{os} module provides an alternate
implementation of \code{environ} which updates the environment on
modification.  Note also that updating \code{os.environ} will render
this dictionary obsolete.  Use of the \refmodule{os} module version of
this is recommended over direct access to the \module{posix} module.}
\end{datadesc}

Additional contents of this module should only be accessed via the
\refmodule{os} module; refer to the documentation for that module for
further information.

\section{\module{os.path} ---
         Common pathname manipulations}
\declaremodule{standard}{os.path}

\modulesynopsis{Common pathname manipulations.}

This module implements some useful functions on pathnames.
\index{path!operations}

\warning{On Windows, many of these functions do not properly
support UNC pathnames.  \function{splitunc()} and \function{ismount()}
do handle them correctly.}


\begin{funcdesc}{abspath}{path}
Return a normalized absolutized version of the pathname \var{path}.
On most platforms, this is equivalent to
\code{normpath(join(os.getcwd(), \var{path}))}.
\versionadded{1.5.2}
\end{funcdesc}

\begin{funcdesc}{basename}{path}
Return the base name of pathname \var{path}.  This is the second half
of the pair returned by \code{split(\var{path})}.  Note that the
result of this function is different from the
\UNIX{} \program{basename} program; where \program{basename} for
\code{'/foo/bar/'} returns \code{'bar'}, the \function{basename()}
function returns an empty string (\code{''}).
\end{funcdesc}

\begin{funcdesc}{commonprefix}{list}
Return the longest path prefix (taken character-by-character) that is a
prefix of all paths in 
\var{list}.  If \var{list} is empty, return the empty string
(\code{''}).  Note that this may return invalid paths because it works a
character at a time.
\end{funcdesc}

\begin{funcdesc}{dirname}{path}
Return the directory name of pathname \var{path}.  This is the first
half of the pair returned by \code{split(\var{path})}.
\end{funcdesc}

\begin{funcdesc}{exists}{path}
Return \code{True} if \var{path} refers to an existing path.
Returns \code{False} for broken symbolic links.
\end{funcdesc}

\begin{funcdesc}{lexists}{path}
Return \code{True} if \var{path} refers to an existing path.
Returns \code{True} for broken symbolic links.  
Equivalent to \function{exists()} on platforms lacking
\function{os.lstat()}.
\versionadded{2.4}
\end{funcdesc}

\begin{funcdesc}{expanduser}{path}
On \UNIX, return the argument with an initial component of \samp{\~} or
\samp{\~\var{user}} replaced by that \var{user}'s home directory.
An initial \samp{\~} is replaced by the environment variable
\envvar{HOME} if it is set; otherwise the current user's home directory
is looked up in the password directory through the built-in module
\refmodule{pwd}\refbimodindex{pwd}.
An initial \samp{\~\var{user}} is looked up directly in the
password directory.

On Windows, only \samp{\~} is supported; it is replaced by the
environment variable \envvar{HOME} or by a combination of
\envvar{HOMEDRIVE} and \envvar{HOMEPATH}.

If the expansion fails or if the
path does not begin with a tilde, the path is returned unchanged.
\end{funcdesc}

\begin{funcdesc}{expandvars}{path}
Return the argument with environment variables expanded.  Substrings
of the form \samp{\$\var{name}} or \samp{\$\{\var{name}\}} are
replaced by the value of environment variable \var{name}.  Malformed
variable names and references to non-existing variables are left
unchanged.
\end{funcdesc}

\begin{funcdesc}{getatime}{path}
Return the time of last access of \var{path}.  The return
value is a number giving the number of seconds since the epoch (see the 
\refmodule{time} module).  Raise \exception{os.error} if the file does
not exist or is inaccessible.
\versionadded{1.5.2}
\versionchanged[If \function{os.stat_float_times()} returns True, the result is a floating point number]{2.3}
\end{funcdesc}

\begin{funcdesc}{getmtime}{path}
Return the time of last modification of \var{path}.  The return
value is a number giving the number of seconds since the epoch (see the 
\refmodule{time} module).  Raise \exception{os.error} if the file does
not exist or is inaccessible.
\versionadded{1.5.2}
\versionchanged[If \function{os.stat_float_times()} returns True, the result is a floating point number]{2.3}
\end{funcdesc}

\begin{funcdesc}{getctime}{path}
Return the system's ctime which, on some systems (like \UNIX) is the
time of the last change, and, on others (like Windows), is the
creation time for \var{path}.  The return
value is a number giving the number of seconds since the epoch (see the 
\refmodule{time} module).  Raise \exception{os.error} if the file does
not exist or is inaccessible.
\versionadded{2.3}
\end{funcdesc}

\begin{funcdesc}{getsize}{path}
Return the size, in bytes, of \var{path}.  Raise
\exception{os.error} if the file does not exist or is inaccessible.
\versionadded{1.5.2}
\end{funcdesc}

\begin{funcdesc}{isabs}{path}
Return \code{True} if \var{path} is an absolute pathname (begins with a
slash).
\end{funcdesc}

\begin{funcdesc}{isfile}{path}
Return \code{True} if \var{path} is an existing regular file.  This follows
symbolic links, so both \function{islink()} and \function{isfile()}
can be true for the same path.
\end{funcdesc}

\begin{funcdesc}{isdir}{path}
Return \code{True} if \var{path} is an existing directory.  This follows
symbolic links, so both \function{islink()} and \function{isdir()} can
be true for the same path.
\end{funcdesc}

\begin{funcdesc}{islink}{path}
Return \code{True} if \var{path} refers to a directory entry that is a
symbolic link.  Always \code{False} if symbolic links are not supported.
\end{funcdesc}

\begin{funcdesc}{ismount}{path}
Return \code{True} if pathname \var{path} is a \dfn{mount point}: a point in
a file system where a different file system has been mounted.  The
function checks whether \var{path}'s parent, \file{\var{path}/..}, is
on a different device than \var{path}, or whether \file{\var{path}/..}
and \var{path} point to the same i-node on the same device --- this
should detect mount points for all \UNIX{} and \POSIX{} variants.
\end{funcdesc}

\begin{funcdesc}{join}{path1\optional{, path2\optional{, ...}}}
Joins one or more path components intelligently.  If any component is
an absolute path, all previous components are thrown away, and joining
continues.  The return value is the concatenation of \var{path1}, and
optionally \var{path2}, etc., with exactly one directory separator
(\code{os.sep}) inserted between components, unless \var{path2} is
empty.  Note that on Windows, since there is a current directory for
each drive, \function{os.path.join("c:", "foo")} represents a path
relative to the current directory on drive \file{C:} (\file{c:foo}), not
\file{c:\textbackslash\textbackslash foo}.
\end{funcdesc}

\begin{funcdesc}{normcase}{path}
Normalize the case of a pathname.  On \UNIX, this returns the path
unchanged; on case-insensitive filesystems, it converts the path to
lowercase.  On Windows, it also converts forward slashes to backward
slashes.
\end{funcdesc}

\begin{funcdesc}{normpath}{path}
Normalize a pathname.  This collapses redundant separators and
up-level references so that \code{A//B}, \code{A/./B} and
\code{A/foo/../B} all become \code{A/B}.  It does not normalize the
case (use \function{normcase()} for that).  On Windows, it converts
forward slashes to backward slashes. It should be understood that this may
change the meaning of the path if it contains symbolic links! 
\end{funcdesc}

\begin{funcdesc}{realpath}{path}
Return the canonical path of the specified filename, eliminating any
symbolic links encountered in the path.
Availability:  \UNIX.
\versionadded{2.2}
\end{funcdesc}

\begin{funcdesc}{samefile}{path1, path2}
Return \code{True} if both pathname arguments refer to the same file or
directory (as indicated by device number and i-node number).
Raise an exception if a \function{os.stat()} call on either pathname
fails.
Availability:  Macintosh, \UNIX.
\end{funcdesc}

\begin{funcdesc}{sameopenfile}{fp1, fp2}
Return \code{True} if the file objects \var{fp1} and \var{fp2} refer to the
same file.  The two file objects may represent different file
descriptors.
Availability:  Macintosh, \UNIX.
\end{funcdesc}

\begin{funcdesc}{samestat}{stat1, stat2}
Return \code{True} if the stat tuples \var{stat1} and \var{stat2} refer to
the same file.  These structures may have been returned by
\function{fstat()}, \function{lstat()}, or \function{stat()}.  This
function implements the underlying comparison used by
\function{samefile()} and \function{sameopenfile()}.
Availability:  Macintosh, \UNIX.
\end{funcdesc}

\begin{funcdesc}{split}{path}
Split the pathname \var{path} into a pair, \code{(\var{head},
\var{tail})} where \var{tail} is the last pathname component and
\var{head} is everything leading up to that.  The \var{tail} part will
never contain a slash; if \var{path} ends in a slash, \var{tail} will
be empty.  If there is no slash in \var{path}, \var{head} will be
empty.  If \var{path} is empty, both \var{head} and \var{tail} are
empty.  Trailing slashes are stripped from \var{head} unless it is the
root (one or more slashes only).  In nearly all cases,
\code{join(\var{head}, \var{tail})} equals \var{path} (the only
exception being when there were multiple slashes separating \var{head}
from \var{tail}).
\end{funcdesc}

\begin{funcdesc}{splitdrive}{path}
Split the pathname \var{path} into a pair \code{(\var{drive},
\var{tail})} where \var{drive} is either a drive specification or the
empty string.  On systems which do not use drive specifications,
\var{drive} will always be the empty string.  In all cases,
\code{\var{drive} + \var{tail}} will be the same as \var{path}.
\versionadded{1.3}
\end{funcdesc}

\begin{funcdesc}{splitext}{path}
Split the pathname \var{path} into a pair \code{(\var{root}, \var{ext})} 
such that \code{\var{root} + \var{ext} == \var{path}},
and \var{ext} is empty or begins with a period and contains
at most one period.
\end{funcdesc}

\begin{funcdesc}{walk}{path, visit, arg}
Calls the function \var{visit} with arguments
\code{(\var{arg}, \var{dirname}, \var{names})} for each directory in the
directory tree rooted at \var{path} (including \var{path} itself, if it
is a directory).  The argument \var{dirname} specifies the visited
directory, the argument \var{names} lists the files in the directory
(gotten from \code{os.listdir(\var{dirname})}).
The \var{visit} function may modify \var{names} to
influence the set of directories visited below \var{dirname}, e.g. to
avoid visiting certain parts of the tree.  (The object referred to by
\var{names} must be modified in place, using \keyword{del} or slice
assignment.)

\begin{notice}
Symbolic links to directories are not treated as subdirectories, and
that \function{walk()} therefore will not visit them. To visit linked
directories you must identify them with
\code{os.path.islink(\var{file})} and
\code{os.path.isdir(\var{file})}, and invoke \function{walk()} as
necessary.
\end{notice}

\note{The newer \function{\refmodule{os}.walk()} generator supplies
      similar functionality and can be easier to use.}
\end{funcdesc}

\begin{datadesc}{supports_unicode_filenames}
True if arbitrary Unicode strings can be used as file names (within
limitations imposed by the file system), and if
\function{os.listdir()} returns Unicode strings for a Unicode
argument.
\versionadded{2.3}
\end{datadesc}

\section{\module{pwd} ---
         The password database}

\declaremodule{builtin}{pwd}
  \platform{Unix}
\modulesynopsis{The password database (\function{getpwnam()} and friends).}

This module provides access to the \UNIX{} user account and password
database.  It is available on all \UNIX{} versions.

Password database entries are reported as a tuple-like object, whose
attributes correspond to the members of the \code{passwd} structure
(Attribute field below, see \code{<pwd.h>}):

\begin{tableiii}{r|l|l}{textrm}{Index}{Attribute}{Meaning}
  \lineiii{0}{\code{pw_name}}{Login name}
  \lineiii{1}{\code{pw_passwd}}{Optional encrypted password}
  \lineiii{2}{\code{pw_uid}}{Numerical user ID}
  \lineiii{3}{\code{pw_gid}}{Numerical group ID}
  \lineiii{4}{\code{pw_gecos}}{User name or comment field}
  \lineiii{5}{\code{pw_dir}}{User home directory}
  \lineiii{6}{\code{pw_shell}}{User command interpreter}
\end{tableiii}

The uid and gid items are integers, all others are strings.
\exception{KeyError} is raised if the entry asked for cannot be found.

\note{In traditional \UNIX{} the field \code{pw_passwd} usually
contains a password encrypted with a DES derived algorithm (see module
\refmodule{crypt}\refbimodindex{crypt}).  However most modern unices 
use a so-called \emph{shadow password} system.  On those unices the
\var{pw_passwd} field only contains an asterisk (\code{'*'}) or the 
letter \character{x} where the encrypted password is stored in a file
\file{/etc/shadow} which is not world readable.  Whether the \var{pw_passwd}
field contains anything useful is system-dependent.}

It defines the following items:

\begin{funcdesc}{getpwuid}{uid}
Return the password database entry for the given numeric user ID.
\end{funcdesc}

\begin{funcdesc}{getpwnam}{name}
Return the password database entry for the given user name.
\end{funcdesc}

\begin{funcdesc}{getpwall}{}
Return a list of all available password database entries, in arbitrary order.
\end{funcdesc}


\begin{seealso}
  \seemodule{grp}{An interface to the group database, similar to this.}
\end{seealso}

\section{Built-in Module \module{grp}}
\declaremodule{builtin}{grp}


\modulesynopsis{The group database (\function{getgrnam()} and friends).}

This module provides access to the \UNIX{} group database.
It is available on all \UNIX{} versions.

Group database entries are reported as 4-tuples containing the
following items from the group database (see \file{<grp.h>}), in order:
\code{gr_name},
\code{gr_passwd},
\code{gr_gid},
\code{gr_mem}.
The gid is an integer, name and password are strings, and the member
list is a list of strings.
(Note that most users are not explicitly listed as members of the
group they are in according to the password database.)
A \code{KeyError} exception is raised if the entry asked for cannot be found.

It defines the following items:

\begin{funcdesc}{getgrgid}{gid}
Return the group database entry for the given numeric group ID.
\end{funcdesc}

\begin{funcdesc}{getgrnam}{name}
Return the group database entry for the given group name.
\end{funcdesc}

\begin{funcdesc}{getgrall}{}
Return a list of all available group entries, in arbitrary order.
\end{funcdesc}

\section{Built-in Module \sectcode{crypt}}
\label{module-crypt}
\bimodindex{crypt}

This module implements an interface to the \manpage{crypt}{3} routine,
which is a one-way hash function based upon a modified DES algorithm;
see the \UNIX{} man page for further details.  Possible uses include
allowing Python scripts to accept typed passwords from the user, or
attempting to crack \UNIX{} passwords with a dictionary.
\index{crypt(3)}

\setindexsubitem{(in module crypt)}
\begin{funcdesc}{crypt}{word\, salt} 
\var{word} will usually be a user's password.  \var{salt} is a
2-character string which will be used to select one of 4096 variations
of DES\indexii{cipher}{DES}.  The characters in \var{salt} must be
either \code{.}, \code{/}, or an alphanumeric character.  Returns the
hashed password as a string, which will be composed of characters from
the same alphabet as the salt.
\end{funcdesc}

The module and documentation were written by Steve Majewski.
\index{Majewski, Steve}

\section{Built-in Module \sectcode{dbm}}
\label{module-dbm}
\bimodindex{dbm}

The \code{dbm} module provides an interface to the \UNIX{}
\code{(n)dbm} library.  Dbm objects behave like mappings
(dictionaries), except that keys and values are always strings.
Printing a dbm object doesn't print the keys and values, and the
\code{items()} and \code{values()} methods are not supported.

See also the \code{gdbm} module, which provides a similar interface
using the GNU GDBM library.
\refbimodindex{gdbm}

The module defines the following constant and functions:

\renewcommand{\indexsubitem}{(in module dbm)}
\begin{excdesc}{error}
Raised on dbm-specific errors, such as I/O errors. \code{KeyError} is
raised for general mapping errors like specifying an incorrect key.
\end{excdesc}

\begin{funcdesc}{open}{filename\, \optional{flag\, \optional{mode}}}
Open a dbm database and return a dbm object.  The \var{filename}
argument is the name of the database file (without the \file{.dir} or
\file{.pag} extensions).

The optional \var{flag} argument can be
\code{'r'} (to open an existing database for reading only --- default),
\code{'w'} (to open an existing database for reading and writing),
\code{'c'} (which creates the database if it doesn't exist), or
\code{'n'} (which always creates a new empty database).

The optional \var{mode} argument is the \UNIX{} mode of the file, used
only when the database has to be created.  It defaults to octal
\code{0666}.
\end{funcdesc}

\section{Built-in Module \module{gdbm}}
\label{module-gdbm}
\bimodindex{gdbm}

% Note that if this section appears on the same page as the first
% paragraph of the dbm module section, makeindex will produce the
% warning:
%
% ## Warning (input = lib.idx, line = 1184; output = lib.ind, line = 852):
%    -- Conflicting entries: multiple encaps for the same page under same key.
%
% This is because the \bimodindex{gdbm} and \refbimodindex{gdbm}
% entries in the .idx file are slightly different (the \bimodindex{}
% version includes "|textbf" at the end to make the defining occurance 
% bold).  There doesn't appear to be anything that can be done about
% this; it's just a little annoying.  The warning can be ignored, but
% the index produced uses the non-bold version.

This module is quite similar to the \code{dbm} module, but uses \code{gdbm}
instead to provide some additional functionality.  Please note that
the file formats created by \code{gdbm} and \code{dbm} are incompatible.
\refbimodindex{dbm}

The \code{gdbm} module provides an interface to the GNU DBM
library.  \code{gdbm} objects behave like mappings
(dictionaries), except that keys and values are always strings.
Printing a \code{gdbm} object doesn't print the keys and values, and the
\code{items()} and \code{values()} methods are not supported.

The module defines the following constant and functions:

\begin{excdesc}{error}
Raised on \code{gdbm}-specific errors, such as I/O errors. \code{KeyError} is
raised for general mapping errors like specifying an incorrect key.
\end{excdesc}

\begin{funcdesc}{open}{filename, \optional{flag, \optional{mode}}}
Open a \code{gdbm} database and return a \code{gdbm} object.  The
\var{filename} argument is the name of the database file.

The optional \var{flag} argument can be
\code{'r'} (to open an existing database for reading only --- default),
\code{'w'} (to open an existing database for reading and writing),
\code{'c'} (which creates the database if it doesn't exist), or
\code{'n'} (which always creates a new empty database).

Appending \code{f} to the flag opens the database in fast mode;
altered data will not automatically be written to the disk after every
change.  This results in faster writes to the database, but may result
in an inconsistent database if the program crashes while the database
is still open.  Use the \code{sync()} method to force any unwritten
data to be written to the disk.

The optional \var{mode} argument is the \UNIX{} mode of the file, used
only when the database has to be created.  It defaults to octal
\code{0666}.
\end{funcdesc}

In addition to the dictionary-like methods, \code{gdbm} objects have the
following methods:

\begin{funcdesc}{firstkey}{}
It's possible to loop over every key in the database using this method
and the \code{nextkey()} method.  The traversal is ordered by \code{gdbm}'s
internal hash values, and won't be sorted by the key values.  This
method returns the starting key.
\end{funcdesc}

\begin{funcdesc}{nextkey}{key}
Returns the key that follows \var{key} in the traversal.  The
following code prints every key in the database \code{db}, without having to
create a list in memory that contains them all:
\begin{verbatim}
k=db.firstkey()
while k!=None:
    print k
    k=db.nextkey(k)
\end{verbatim}
\end{funcdesc}

\begin{funcdesc}{reorganize}{}
If you have carried out a lot of deletions and would like to shrink
the space used by the \code{gdbm} file, this routine will reorganize the
database.  \code{gdbm} will not shorten the length of a database file except
by using this reorganization; otherwise, deleted file space will be
kept and reused as new (key,value) pairs are added.
\end{funcdesc}

\begin{funcdesc}{sync}{}
When the database has been opened in fast mode, this method forces any
unwritten data to be written to the disk.
\end{funcdesc}


\section{Built-in Module \sectcode{termios}}
\label{module-termios}
\bimodindex{termios}
\indexii{Posix}{I/O control}
\indexii{tty}{I/O control}

\renewcommand{\indexsubitem}{(in module termios)}

This module provides an interface to the Posix calls for tty I/O
control.  For a complete description of these calls, see the Posix or
\UNIX{} manual pages.  It is only available for those \UNIX{} versions
that support Posix \code{termios} style tty I/O control (and then
only if configured at installation time).

All functions in this module take a file descriptor \var{fd} as their
first argument.  This must be an integer file descriptor, such as
returned by \code{sys.stdin.fileno()}.

This module should be used in conjunction with the \code{TERMIOS}
module, which defines the relevant symbolic constants (see the next
section).

The module defines the following functions:

\begin{funcdesc}{tcgetattr}{fd}
Return a list containing the tty attributes for file descriptor
\var{fd}, as follows: \code{[\var{iflag}, \var{oflag}, \var{cflag},
\var{lflag}, \var{ispeed}, \var{ospeed}, \var{cc}]} where \var{cc} is
a list of the tty special characters (each a string of length 1,
except the items with indices \code{VMIN} and \code{VTIME}, which are
integers when these fields are defined).  The interpretation of the
flags and the speeds as well as the indexing in the \var{cc} array
must be done using the symbolic constants defined in the
\code{TERMIOS} module.
\end{funcdesc}

\begin{funcdesc}{tcsetattr}{fd\, when\, attributes}
Set the tty attributes for file descriptor \var{fd} from the
\var{attributes}, which is a list like the one returned by
\code{tcgetattr()}.  The \var{when} argument determines when the
attributes are changed: \code{TERMIOS.TCSANOW} to change immediately,
\code{TERMIOS.TCSADRAIN} to change after transmitting all queued
output, or \code{TERMIOS.TCSAFLUSH} to change after transmitting all
queued output and discarding all queued input.
\end{funcdesc}

\begin{funcdesc}{tcsendbreak}{fd\, duration}
Send a break on file descriptor \var{fd}.  A zero \var{duration} sends
a break for 0.25--0.5 seconds; a nonzero \var{duration} has a system
dependent meaning.
\end{funcdesc}

\begin{funcdesc}{tcdrain}{fd}
Wait until all output written to file descriptor \var{fd} has been
transmitted.
\end{funcdesc}

\begin{funcdesc}{tcflush}{fd\, queue}
Discard queued data on file descriptor \var{fd}.  The \var{queue}
selector specifies which queue: \code{TERMIOS.TCIFLUSH} for the input
queue, \code{TERMIOS.TCOFLUSH} for the output queue, or
\code{TERMIOS.TCIOFLUSH} for both queues.
\end{funcdesc}

\begin{funcdesc}{tcflow}{fd\, action}
Suspend or resume input or output on file descriptor \var{fd}.  The
\var{action} argument can be \code{TERMIOS.TCOOFF} to suspend output,
\code{TERMIOS.TCOON} to restart output, \code{TERMIOS.TCIOFF} to
suspend input, or \code{TERMIOS.TCION} to restart input.
\end{funcdesc}

\subsection{Example}
\nodename{termios Example}

Here's a function that prompts for a password with echoing turned off.
Note the technique using a separate \code{termios.tcgetattr()} call
and a \code{try \ldots{} finally} statement to ensure that the old tty
attributes are restored exactly no matter what happens:

\bcode\begin{verbatim}
def getpass(prompt = "Password: "):
    import termios, TERMIOS, sys
    fd = sys.stdin.fileno()
    old = termios.tcgetattr(fd)
    new = termios.tcgetattr(fd)
    new[3] = new[3] & \~TERMIOS.ECHO          # lflags
    try:
        termios.tcsetattr(fd, TERMIOS.TCSADRAIN, new)
        passwd = raw_input(prompt)
    finally:
        termios.tcsetattr(fd, TERMIOS.TCSADRAIN, old)
    return passwd
\end{verbatim}\ecode
%
\section{Standard Module \sectcode{TERMIOS}}
\stmodindex{TERMIOS}
\indexii{Posix}{I/O control}
\indexii{tty}{I/O control}

\renewcommand{\indexsubitem}{(in module TERMIOS)}

This module defines the symbolic constants required to use the
\code{termios} module (see the previous section).  See the Posix or
\UNIX{} manual pages (or the source) for a list of those constants.
\refbimodindex{termios}

Note: this module resides in a system-dependent subdirectory of the
Python library directory.  You may have to generate it for your
particular system using the script \file{Tools/scripts/h2py.py}.

\section{\module{fcntl} ---
         The \function{fcntl()} and \function{ioctl()} system calls}

\declaremodule{builtin}{fcntl}
  \platform{Unix}
\modulesynopsis{The \function{fcntl()} and \function{ioctl()} system calls.}
\sectionauthor{Jaap Vermeulen}{}

\indexii{UNIX@\UNIX{}}{file control}
\indexii{UNIX@\UNIX{}}{I/O control}

This module performs file control and I/O control on file descriptors.
It is an interface to the \cfunction{fcntl()} and \cfunction{ioctl()}
\UNIX{} routines.

All functions in this module take a file descriptor \var{fd} as their
first argument.  This can be an integer file descriptor, such as
returned by \code{sys.stdin.fileno()}, or a file object, such as
\code{sys.stdin} itself.

The module defines the following functions:


\begin{funcdesc}{fcntl}{fd, op\optional{, arg}}
  Perform the requested operation on file descriptor \var{fd}.
  The operation is defined by \var{op} and is operating system
  dependent.  These codes are also found in the \module{fcntl}
  module. The argument \var{arg} is optional, and defaults to the
  integer value \code{0}.  When present, it can either be an integer
  value, or a string.  With the argument missing or an integer value,
  the return value of this function is the integer return value of the
  C \cfunction{fcntl()} call.  When the argument is a string it
  represents a binary structure, e.g.\ created by
  \function{struct.pack()}. The binary data is copied to a buffer
  whose address is passed to the C \cfunction{fcntl()} call.  The
  return value after a successful call is the contents of the buffer,
  converted to a string object.  The length of the returned string
  will be the same as the length of the \var{arg} argument.  This is
  limited to 1024 bytes.  If the information returned in the buffer by
  the operating system is larger than 1024 bytes, this is most likely
  to result in a segmentation violation or a more subtle data
  corruption.

  If the \cfunction{fcntl()} fails, an \exception{IOError} is
  raised.
\end{funcdesc}

\begin{funcdesc}{ioctl}{fd, op, arg}
  This function is identical to the \function{fcntl()} function, except
  that the operations are typically defined in the library module
  \module{IOCTL}.
\end{funcdesc}

\begin{funcdesc}{flock}{fd, op}
Perform the lock operation \var{op} on file descriptor \var{fd}.
See the \UNIX{} manual \manpage{flock}{3} for details.  (On some
systems, this function is emulated using \cfunction{fcntl()}.)
\end{funcdesc}

\begin{funcdesc}{lockf}{fd, operation,
    \optional{len, \optional{start, \optional{whence}}}}
This is essentially a wrapper around the \function{fcntl()} locking
calls.  \var{fd} is the file descriptor of the file to lock or unlock,
and \var{operation} is one of the following values:

\begin{itemize}
\item \constant{LOCK_UN} -- unlock
\item \constant{LOCK_SH} -- acquire a shared lock
\item \constant{LOCK_EX} -- acquire an exclusive lock
\end{itemize}

When \var{operation} is \constant{LOCK_SH} or \constant{LOCK_EX}, it
can also be bit-wise OR'd with \constant{LOCK_NB} to avoid blocking on
lock acquisition.  If \constant{LOCK_NB} is used and the lock cannot
be acquired, an \exception{IOError} will be raised and the exception
will have an \var{errno} attribute set to \constant{EACCES} or
\constant{EAGAIN} (depending on the operating system; for portability,
check for both values).

\var{length} is the number of bytes to lock, \var{start} is the byte
offset at which the lock starts, relative to \var{whence}, and
\var{whence} is as with \function{fileobj.seek()}, specifically:

\begin{itemize}
\item \constant{0} -- relative to the start of the file
      (\constant{SEEK_SET})
\item \constant{1} -- relative to the current buffer position
      (\constant{SEEK_CUR})
\item \constant{2} -- relative to the end of the file
      (\constant{SEEK_END})
\end{itemize}

The default for \var{start} is 0, which means to start at the
beginning of the file.  The default for \var{length} is 0 which means
to lock to the end of the file.  The default for \var{whence} is also
0.
\end{funcdesc}

Examples (all on a SVR4 compliant system):

\begin{verbatim}
import struct, fcntl

file = open(...)
rv = fcntl(file, fcntl.F_SETFL, os.O_NDELAY)

lockdata = struct.pack('hhllhh', fcntl.F_WRLCK, 0, 0, 0, 0, 0)
rv = fcntl.fcntl(file, fcntl.F_SETLKW, lockdata)
\end{verbatim}

Note that in the first example the return value variable \var{rv} will
hold an integer value; in the second example it will hold a string
value.  The structure lay-out for the \var{lockdata} variable is
system dependent --- therefore using the \function{flock()} call may be
better.

% Manual text and implementation by Jaap Vermeulen
\section{Standard Module \sectcode{posixfile}}
\label{module-posixfile}
\bimodindex{posixfile}
\indexii{\POSIX{}}{file object}

\emph{Note:} This module will become obsolete in a future release.
The locking operation that it provides is done better and more
portably by the \function{fcntl.lockf()} call.%
\withsubitem{(in module fcntl)}{\ttindex{lockf()}}

This module implements some additional functionality over the built-in
file objects.  In particular, it implements file locking, control over
the file flags, and an easy interface to duplicate the file object.
The module defines a new file object, the posixfile object.  It
has all the standard file object methods and adds the methods
described below.  This module only works for certain flavors of
\UNIX{}, since it uses \function{fcntl.fcntl()} for file locking.%
\withsubitem{(in module fcntl)}{\ttindex{fcntl()}}

To instantiate a posixfile object, use the \function{open()} function
in the \module{posixfile} module.  The resulting object looks and
feels roughly the same as a standard file object.

The \module{posixfile} module defines the following constants:


\begin{datadesc}{SEEK_SET}
Offset is calculated from the start of the file.
\end{datadesc}

\begin{datadesc}{SEEK_CUR}
Offset is calculated from the current position in the file.
\end{datadesc}

\begin{datadesc}{SEEK_END}
Offset is calculated from the end of the file.
\end{datadesc}

The \module{posixfile} module defines the following functions:


\begin{funcdesc}{open}{filename\optional{, mode\optional{, bufsize}}}
 Create a new posixfile object with the given filename and mode.  The
 \var{filename}, \var{mode} and \var{bufsize} arguments are
 interpreted the same way as by the built-in \function{open()}
 function.
\end{funcdesc}

\begin{funcdesc}{fileopen}{fileobject}
 Create a new posixfile object with the given standard file object.
 The resulting object has the same filename and mode as the original
 file object.
\end{funcdesc}

The posixfile object defines the following additional methods:

\setindexsubitem{(posixfile method)}
\begin{funcdesc}{lock}{fmt, \optional{len\optional{, start\optional{, whence}}}}
 Lock the specified section of the file that the file object is
 referring to.  The format is explained
 below in a table.  The \var{len} argument specifies the length of the
 section that should be locked. The default is \code{0}. \var{start}
 specifies the starting offset of the section, where the default is
 \code{0}.  The \var{whence} argument specifies where the offset is
 relative to. It accepts one of the constants \constant{SEEK_SET},
 \constant{SEEK_CUR} or \constant{SEEK_END}.  The default is
 \constant{SEEK_SET}.  For more information about the arguments refer
 to the \manpage{fcntl}{2} manual page on your system.
\end{funcdesc}

\begin{funcdesc}{flags}{\optional{flags}}
 Set the specified flags for the file that the file object is referring
 to.  The new flags are ORed with the old flags, unless specified
 otherwise.  The format is explained below in a table.  Without
 the \var{flags} argument
 a string indicating the current flags is returned (this is
 the same as the \samp{?} modifier).  For more information about the
 flags refer to the \manpage{fcntl}{2} manual page on your system.
\end{funcdesc}

\begin{funcdesc}{dup}{}
 Duplicate the file object and the underlying file pointer and file
 descriptor.  The resulting object behaves as if it were newly
 opened.
\end{funcdesc}

\begin{funcdesc}{dup2}{fd}
 Duplicate the file object and the underlying file pointer and file
 descriptor.  The new object will have the given file descriptor.
 Otherwise the resulting object behaves as if it were newly opened.
\end{funcdesc}

\begin{funcdesc}{file}{}
 Return the standard file object that the posixfile object is based
 on.  This is sometimes necessary for functions that insist on a
 standard file object.
\end{funcdesc}

All methods raise \exception{IOError} when the request fails.

Format characters for the \method{lock()} method have the following
meaning:

\begin{tableii}{|c|l|}{samp}{Format}{Meaning}
  \lineii{u}{unlock the specified region}
  \lineii{r}{request a read lock for the specified section}
  \lineii{w}{request a write lock for the specified section}
\end{tableii}

In addition the following modifiers can be added to the format:

\begin{tableiii}{|c|l|c|}{samp}{Modifier}{Meaning}{Notes}
  \lineiii{|}{wait until the lock has been granted}{}
  \lineiii{?}{return the first lock conflicting with the requested lock, or
              \code{None} if there is no conflict.}{(1)} 
\end{tableiii}

Note:

(1) The lock returned is in the format \code{(\var{mode}, \var{len},
\var{start}, \var{whence}, \var{pid})} where \var{mode} is a character
representing the type of lock ('r' or 'w').  This modifier prevents a
request from being granted; it is for query purposes only.

Format characters for the \method{flags()} method have the following
meanings:

\begin{tableii}{|c|l|}{samp}{Format}{Meaning}
  \lineii{a}{append only flag}
  \lineii{c}{close on exec flag}
  \lineii{n}{no delay flag (also called non-blocking flag)}
  \lineii{s}{synchronization flag}
\end{tableii}

In addition the following modifiers can be added to the format:

\begin{tableiii}{|c|l|c|}{samp}{Modifier}{Meaning}{Notes}
  \lineiii{!}{turn the specified flags 'off', instead of the default 'on'}{(1)}
  \lineiii{=}{replace the flags, instead of the default 'OR' operation}{(1)}
  \lineiii{?}{return a string in which the characters represent the flags that
  are set.}{(2)}
\end{tableiii}

Note:

(1) The \code{!} and \code{=} modifiers are mutually exclusive.

(2) This string represents the flags after they may have been altered
by the same call.

Examples:

\begin{verbatim}
import posixfile

file = posixfile.open('/tmp/test', 'w')
file.lock('w|')
...
file.lock('u')
file.close()
\end{verbatim}

\section{Built-in Module \module{resource}}
\label{module-resource}

\bimodindex{resource}
This module provides basic mechanisms for measuring and controlling
system resources utilized by a program.

Symbolic constants are used to specify particular system resources and
to request usage information about either the current process or its
children.

A single exception is defined for errors:


\begin{excdesc}{error}
  The functions described below may raise this error if the underlying
  system call failures unexpectedly.
\end{excdesc}

\subsection{Resource Limits}

Resources usage can be limited using the \function{setrlimit()} function
described below. Each resource is controlled by a pair of limits: a
soft limit and a hard limit. The soft limit is the current limit, and
may be lowered or raised by a process over time. The soft limit can
never exceed the hard limit. The hard limit can be lowered to any
value greater than the soft limit, but not raised. (Only processes with
the effective UID of the super-user can raise a hard limit.)

The specific resources that can be limited are system dependent. They
are described in the \manpage{getrlimit}{2} man page.  The resources
listed below are supported when the underlying operating system
supports them; resources which cannot be checked or controlled by the
operating system are not defined in this module for those platforms.

\begin{funcdesc}{getrlimit}{resource}
  Returns a tuple \code{(\var{soft}, \var{hard})} with the current
  soft and hard limits of \var{resource}. Raises \exception{ValueError} if
  an invalid resource is specified, or \exception{error} if the
  underyling system call fails unexpectedly.
\end{funcdesc}

\begin{funcdesc}{setrlimit}{resource, limits}
  Sets new limits of consumption of \var{resource}. The \var{limits}
  argument must be a tuple \code{(\var{soft}, \var{hard})} of two
  integers describing the new limits. A value of \code{-1} can be used to
  specify the maximum possible upper limit.

  Raises \exception{ValueError} if an invalid resource is specified,
  if the new soft limit exceeds the hard limit, or if a process tries
  to raise its hard limit (unless the process has an effective UID of
  super-user).  Can also raise \exception{error} if the underyling
  system call fails.
\end{funcdesc}

These symbols define resources whose consumption can be controlled
using the \function{setrlimit()} and \function{getrlimit()} functions
described below. The values of these symbols are exactly the constants
used by \C{} programs.

The \UNIX{} man page for \manpage{getrlimit}{2} lists the available
resources.  Note that not all systems use the same symbol or same
value to denote the same resource.

\begin{datadesc}{RLIMIT_CORE}
  The maximum size (in bytes) of a core file that the current process
  can create.  This may result in the creation of a partial core file
  if a larger core would be required to contain the entire process
  image.
\end{datadesc}

\begin{datadesc}{RLIMIT_CPU}
  The maximum amount of CPU time (in seconds) that a process can
  use. If this limit is exceeded, a \constant{SIGXCPU} signal is sent to
  the process. (See the \module{signal} module documentation for
  information about how to catch this signal and do something useful,
  e.g. flush open files to disk.)
\end{datadesc}

\begin{datadesc}{RLIMIT_FSIZE}
  The maximum size of a file which the process may create.  This only
  affects the stack of the main thread in a multi-threaded process.
\end{datadesc}

\begin{datadesc}{RLIMIT_DATA}
  The maximum size (in bytes) of the process's heap.
\end{datadesc}

\begin{datadesc}{RLIMIT_STACK}
  The maximum size (in bytes) of the call stack for the current
  process.
\end{datadesc}

\begin{datadesc}{RLIMIT_RSS}
  The maximum resident set size that should be made available to the
  process.
\end{datadesc}

\begin{datadesc}{RLIMIT_NPROC}
  The maximum number of processes the current process may create.
\end{datadesc}

\begin{datadesc}{RLIMIT_NOFILE}
  The maximum number of open file descriptors for the current
  process.
\end{datadesc}

\begin{datadesc}{RLIMIT_OFILE}
  The BSD name for \constant{RLIMIT_NOFILE}.
\end{datadesc}

\begin{datadesc}{RLIMIT_MEMLOC}
  The maximm address space which may be locked in memory.
\end{datadesc}

\begin{datadesc}{RLIMIT_VMEM}
  The largest area of mapped memory which the process may occupy.
\end{datadesc}

\begin{datadesc}{RLIMIT_AS}
  The maximum area (in bytes) of address space which may be taken by
  the process.
\end{datadesc}

\subsection{Resource Usage}

These functiona are used to retrieve resource usage information:

\begin{funcdesc}{getrusage}{who}
  This function returns a large tuple that describes the resources
  consumed by either the current process or its children, as specified
  by the \var{who} parameter.  The \var{who} parameter should be
  specified using one of the \code{RUSAGE_*} constants described
  below.

  The elements of the return value each
  describe how a particular system resource has been used, e.g. amount
  of time spent running is user mode or number of times the process was
  swapped out of main memory. Some values are dependent on the clock
  tick internal, e.g. the amount of memory the process is using.

  The first two elements of the return value are floating point values
  representing the amount of time spent executing in user mode and the
  amount of time spent executing in system mode, respectively. The
  remaining values are integers. Consult the \manpage{getrusage}{2}
  man page for detailed information about these values. A brief
  summary is presented here:

\begin{tableii}{|r|l|}{code}{Offset}{Resource}
  \lineii{0}{time in user mode (float)}
  \lineii{1}{time in system mode (float)}
  \lineii{2}{maximum resident set size}
  \lineii{3}{shared memory size}
  \lineii{4}{unshared memory size}
  \lineii{5}{unshared stack size}
  \lineii{6}{page faults not requiring I/O}
  \lineii{7}{page faults requiring I/O}
  \lineii{8}{number of swap outs}
  \lineii{9}{block input operations}
  \lineii{10}{block output operations}
  \lineii{11}{messages sent}
  \lineii{12}{messages received}
  \lineii{13}{signals received}
  \lineii{14}{voluntary context switches}
  \lineii{15}{involuntary context switches}
\end{tableii}

  This function will raise a \exception{ValueError} if an invalid
  \var{who} parameter is specified. It may also raise
  \exception{error} exception in unusual circumstances.
\end{funcdesc}

\begin{funcdesc}{getpagesize}{}
  Returns the number of bytes in a system page. (This need not be the
  same as the hardware page size.) This function is useful for
  determining the number of bytes of memory a process is using. The
  third element of the tuple returned by \function{getrusage()} describes
  memory usage in pages; multiplying by page size produces number of
  bytes. 
\end{funcdesc}

The following \code{RUSAGE_*} symbols are passed to the
\function{getrusage()} function to specify which processes information
should be provided for.

\begin{datadesc}{RUSAGE_SELF}
  \constant{RUSAGE_SELF} should be used to
  request information pertaining only to the process itself.
\end{datadesc}

\begin{datadesc}{RUSAGE_CHILDREN}
  Pass to \function{getrusage()} to request resource information for
  child processes of the calling process.
\end{datadesc}

\begin{datadesc}{RUSAGE_BOTH}
  Pass to \function{getrusage()} to request resources consumed by both
  the current process and child processes.  May not be available on all
  systems.
\end{datadesc}

\section{Built-in Module \sectcode{syslog}}
\bimodindex{syslog}

This module provides an interface to the Unix \code{syslog} library
routines.  Refer to the \UNIX{} manual pages for a detailed description
of the \code{syslog} facility.

The module defines the following functions:

\begin{funcdesc}{syslog}{\optional{priority\,} message}
Send the string \var{message} to the system logger.
A trailing newline is added if necessary.
Each message is tagged with a priority composed of a \var{facility} and
a \var{level}.
The optional \var{priority} argument, which defaults to
\code{(LOG_USER | LOG_INFO)}, determines the message priority.
\end{funcdesc}

\begin{funcdesc}{openlog}{ident\, \optional{logopt\, \optional{facility}}}
Logging options other than the defaults can be set by explicitly opening
the log file with \code{openlog()} prior to calling \code{syslog()}.
The defaults are (usually) \var{ident} = \samp{syslog}, \var{logopt} = 0,
\var{facility} = \code{LOG_USER}.
The \var{ident} argument is a string which is prepended to every message.
The optional \var{logopt} argument is a bit field - see below for possible
values to combine.
The optional \var{facility} argument sets the default facility for messages
which do not have a facility explicitly encoded.
\end{funcdesc}

\begin{funcdesc}{closelog}{}
Close the log file.
\end{funcdesc}

\begin{funcdesc}{setlogmask}{maskpri}
This function set the priority mask to \var{maskpri} and returns the
previous mask value.
Calls to \code{syslog} with a priority level not set in \var{maskpri}
are ignored.
The default is to log all priorities.
The function \code{LOG_MASK(\var{pri})} calculates the mask for the
individual priority \var{pri}.
The function \code{LOG_UPTO(\var{pri})} calculates the mask for all priorities
up to and including \var{pri}.
\end{funcdesc}

The module defines the following constants:

\begin{description}

\item[Priority levels (high to low):]

\code{LOG_EMERG}, \code{LOG_ALERT}, \code{LOG_CRIT}, \code{LOG_ERR},
\code{LOG_WARNING}, \code{LOG_NOTICE}, \code{LOG_INFO}, \code{LOG_DEBUG}.

\item[Facilities:]

\code{LOG_KERN}, \code{LOG_USER}, \code{LOG_MAIL}, \code{LOG_DAEMON},
\code{LOG_AUTH}, \code{LOG_LPR}, \code{LOG_NEWS}, \code{LOG_UUCP},
\code{LOG_CRON} and \code{LOG_LOCAL0} to \code{LOG_LOCAL7}.

\item[Log options:]

\code{LOG_PID}, \code{LOG_CONS}, \code{LOG_NDELAY}, \code{LOG_NOWAIT}
and \code{LOG_PERROR} if defined in \file{syslog.h}.

\end{description}

\section{\module{stat} ---
         Interpreting \function{stat()} results}

\declaremodule{standard}{stat}
\modulesynopsis{Utilities for interpreting the results of
  \function{os.stat()}, \function{os.lstat()} and \function{os.fstat()}.}
\sectionauthor{Skip Montanaro}{skip@automatrix.com}


The \module{stat} module defines constants and functions for
interpreting the results of \function{os.stat()},
\function{os.fstat()} and \function{os.lstat()} (if they exist).  For
complete details about the \cfunction{stat()}, \cfunction{fstat()} and
\cfunction{lstat()} calls, consult the documentation for your system.

The \module{stat} module defines the following functions to test for
specific file types:


\begin{funcdesc}{S_ISDIR}{mode}
Return non-zero if the mode is from a directory.
\end{funcdesc}

\begin{funcdesc}{S_ISCHR}{mode}
Return non-zero if the mode is from a character special device file.
\end{funcdesc}

\begin{funcdesc}{S_ISBLK}{mode}
Return non-zero if the mode is from a block special device file.
\end{funcdesc}

\begin{funcdesc}{S_ISREG}{mode}
Return non-zero if the mode is from a regular file.
\end{funcdesc}

\begin{funcdesc}{S_ISFIFO}{mode}
Return non-zero if the mode is from a FIFO (named pipe).
\end{funcdesc}

\begin{funcdesc}{S_ISLNK}{mode}
Return non-zero if the mode is from a symbolic link.
\end{funcdesc}

\begin{funcdesc}{S_ISSOCK}{mode}
Return non-zero if the mode is from a socket.
\end{funcdesc}

Two additional functions are defined for more general manipulation of
the file's mode:

\begin{funcdesc}{S_IMODE}{mode}
Return the portion of the file's mode that can be set by
\function{os.chmod()}---that is, the file's permission bits, plus the
sticky bit, set-group-id, and set-user-id bits (on systems that support
them).
\end{funcdesc}

\begin{funcdesc}{S_IFMT}{mode}
Return the portion of the file's mode that describes the file type (used
by the \function{S_IS*()} functions above).
\end{funcdesc}

Normally, you would use the \function{os.path.is*()} functions for
testing the type of a file; the functions here are useful when you are
doing multiple tests of the same file and wish to avoid the overhead of
the \cfunction{stat()} system call for each test.  These are also
useful when checking for information about a file that isn't handled
by \refmodule{os.path}, like the tests for block and character
devices.

All the variables below are simply symbolic indexes into the 10-tuple
returned by \function{os.stat()}, \function{os.fstat()} or
\function{os.lstat()}.

\begin{datadesc}{ST_MODE}
Inode protection mode.
\end{datadesc}

\begin{datadesc}{ST_INO}
Inode number.
\end{datadesc}

\begin{datadesc}{ST_DEV}
Device inode resides on.
\end{datadesc}

\begin{datadesc}{ST_NLINK}
Number of links to the inode.
\end{datadesc}

\begin{datadesc}{ST_UID}
User id of the owner.
\end{datadesc}

\begin{datadesc}{ST_GID}
Group id of the owner.
\end{datadesc}

\begin{datadesc}{ST_SIZE}
Size in bytes of a plain file; amount of data waiting on some special
files.
\end{datadesc}

\begin{datadesc}{ST_ATIME}
Time of last access.
\end{datadesc}

\begin{datadesc}{ST_MTIME}
Time of last modification.
\end{datadesc}

\begin{datadesc}{ST_CTIME}
Time of last status change (see manual pages for details).
\end{datadesc}

The interpretation of ``file size'' changes according to the file
type.  For plain files this is the size of the file in bytes.  For
FIFOs and sockets under most Unixes (including Linux in particular),
the ``size'' is the number of bytes waiting to be read at the time of
the call to \function{os.stat()}, \function{os.fstat()}, or
\function{os.lstat()}; this can sometimes be useful, especially for
polling one of these special files after a non-blocking open.  The
meaning of the size field for other character and block devices varies
more, depending on the implementation of the underlying system call.

Example:

\begin{verbatim}
import os, sys
from stat import *

def walktree(dir, callback):
    '''recursively descend the directory rooted at dir,
       calling the callback function for each regular file'''

    for f in os.listdir(dir):
        pathname = '%s/%s' % (dir, f)
        mode = os.stat(pathname)[ST_MODE]
        if S_ISDIR(mode):
            # It's a directory, recurse into it
            walktree(pathname, callback)
        elif S_ISREG(mode):
            # It's a file, call the callback function
            callback(pathname)
        else:
            # Unknown file type, print a message
            print 'Skipping %s' % pathname

def visitfile(file):
    print 'visiting', file

if __name__ == '__main__':
    walktree(sys.argv[1], visitfile)
\end{verbatim}

\section{\module{popen2} ---
         Subprocesses with accessible I/O streams}

\declaremodule{standard}{popen2}
  \platform{Unix, Windows}
\modulesynopsis{Subprocesses with accessible standard I/O streams.}
\sectionauthor{Drew Csillag}{drew_csillag@geocities.com}


This module allows you to spawn processes and connect to their
input/output/error pipes and obtain their return codes under
\UNIX{} and Windows.

Note that starting with Python 2.0, this functionality is available
using functions from the \refmodule{os} module which have the same
names as the factory functions here, but the order of the return
values is more intuitive in the \refmodule{os} module variants.

The primary interface offered by this module is a trio of factory
functions.  For each of these, if \var{bufsize} is specified, 
it specifies the buffer size for the I/O pipes.  \var{mode}, if
provided, should be the string \code{'b'} or \code{'t'}; on Windows
this is needed to determine whether the file objects should be opened
in binary or text mode.  The default value for \var{mode} is
\code{'t'}.

On \UNIX, \var{cmd} may be a sequence, in which case arguments will be passed
directly to the program without shell intervention (as with
\function{os.spawnv()}). If \var{cmd} is a string it will be passed to the
shell (as with \function{os.system()}).

The only way to retrieve the return codes for the child processes is
by using the \method{poll()} or \method{wait()} methods on the
\class{Popen3} and \class{Popen4} classes; these are only available on
\UNIX.  This information is not available when using the
\function{popen2()}, \function{popen3()}, and \function{popen4()}
functions, or the equivalent functions in the \refmodule{os} module.
(Note that the tuples returned by the \refmodule{os} module's functions
are in a different order from the ones returned by the \module{popen2}
module.)

\begin{funcdesc}{popen2}{cmd\optional{, bufsize\optional{, mode}}}
Executes \var{cmd} as a sub-process.  Returns the file objects
\code{(\var{child_stdout}, \var{child_stdin})}.
\end{funcdesc}

\begin{funcdesc}{popen3}{cmd\optional{, bufsize\optional{, mode}}}
Executes \var{cmd} as a sub-process.  Returns the file objects
\code{(\var{child_stdout}, \var{child_stdin}, \var{child_stderr})}.
\end{funcdesc}

\begin{funcdesc}{popen4}{cmd\optional{, bufsize\optional{, mode}}}
Executes \var{cmd} as a sub-process.  Returns the file objects
\code{(\var{child_stdout_and_stderr}, \var{child_stdin})}.
\versionadded{2.0}
\end{funcdesc}


On \UNIX, a class defining the objects returned by the factory
functions is also available.  These are not used for the Windows
implementation, and are not available on that platform.

\begin{classdesc}{Popen3}{cmd\optional{, capturestderr\optional{, bufsize}}}
This class represents a child process.  Normally, \class{Popen3}
instances are created using the \function{popen2()} and
\function{popen3()} factory functions described above.

If not using one of the helper functions to create \class{Popen3}
objects, the parameter \var{cmd} is the shell command to execute in a
sub-process.  The \var{capturestderr} flag, if true, specifies that
the object should capture standard error output of the child process.
The default is false.  If the \var{bufsize} parameter is specified, it
specifies the size of the I/O buffers to/from the child process.
\end{classdesc}

\begin{classdesc}{Popen4}{cmd\optional{, bufsize}}
Similar to \class{Popen3}, but always captures standard error into the
same file object as standard output.  These are typically created
using \function{popen4()}.
\versionadded{2.0}
\end{classdesc}

\subsection{Popen3 and Popen4 Objects \label{popen3-objects}}

Instances of the \class{Popen3} and \class{Popen4} classes have the
following methods:

\begin{methoddesc}{poll}{}
Returns \code{-1} if child process hasn't completed yet, or its return 
code otherwise.
\end{methoddesc}

\begin{methoddesc}{wait}{}
Waits for and returns the status code of the child process.  The
status code encodes both the return code of the process and
information about whether it exited using the \cfunction{exit()}
system call or died due to a signal.  Functions to help interpret the
status code are defined in the \refmodule{os} module; see section
\ref{os-process} for the \function{W\var{*}()} family of functions.
\end{methoddesc}


The following attributes are also available: 

\begin{memberdesc}{fromchild}
A file object that provides output from the child process.  For
\class{Popen4} instances, this will provide both the standard output
and standard error streams.
\end{memberdesc}

\begin{memberdesc}{tochild}
A file object that provides input to the child process.
\end{memberdesc}

\begin{memberdesc}{childerr}
A file object that provides error output from the child process, if
\var{capturestderr} was true for the constructor, otherwise
\code{None}.  This will always be \code{None} for \class{Popen4}
instances.
\end{memberdesc}

\begin{memberdesc}{pid}
The process ID of the child process.
\end{memberdesc}


\subsection{Flow Control Issues \label{popen2-flow-control}}

Any time you are working with any form of inter-process communication,
control flow needs to be carefully thought out.  This remains the case
with the file objects provided by this module (or the \refmodule{os}
module equivalents).

% Example explanation and suggested work-arounds substantially stolen
% from Martin von L�wis:
% http://mail.python.org/pipermail/python-dev/2000-September/009460.html

When reading output from a child process that writes a lot of data to
standard error while the parent is reading from the child's standard
output, a deadlock can occur.  A similar situation can occur with other
combinations of reads and writes.  The essential factors are that more
than \constant{_PC_PIPE_BUF} bytes are being written by one process in
a blocking fashion, while the other process is reading from the other
process, also in a blocking fashion.

There are several ways to deal with this situation.

The simplest application change, in many cases, will be to follow this
model in the parent process:

\begin{verbatim}
import popen2

r, w, e = popen2.popen3('python slave.py')
e.readlines()
r.readlines()
r.close()
e.close()
w.close()
\end{verbatim}

with code like this in the child:

\begin{verbatim}
import os
import sys

# note that each of these print statements
# writes a single long string

print >>sys.stderr, 400 * 'this is a test\n'
os.close(sys.stderr.fileno())
print >>sys.stdout, 400 * 'this is another test\n'
\end{verbatim}

In particular, note that \code{sys.stderr} must be closed after
writing all data, or \method{readlines()} won't return.  Also note
that \function{os.close()} must be used, as \code{sys.stderr.close()}
won't close \code{stderr} (otherwise assigning to \code{sys.stderr}
will silently close it, so no further errors can be printed).

Applications which need to support a more general approach should
integrate I/O over pipes with their \function{select()} loops, or use
separate threads to read each of the individual files provided by
whichever \function{popen*()} function or \class{Popen*} class was
used.

\section{Standard module \sectcode{commands}}	% If implemented in Python
\label{module-commands}
\stmodindex{commands}

The \code{commands} module contains wrapper functions for \code{os.popen()} 
which take a system command as a string and return any output generated by 
the command, and optionally, the exit status.

The \code{commands} module is only usable on systems which support 
\code{popen()} (currently \UNIX{}).

The \code{commands} module defines the following functions:

\renewcommand{\indexsubitem}{(in module commands)}
\begin{funcdesc}{getstatusoutput}{cmd}
Execute the string \var{cmd} in a shell with \code{os.popen()} and return
a 2-tuple (status, output).  \var{cmd} is actually run as
\code{\{ cmd ; \} 2>\&1}, so that the returned output will contain output
or error messages. A trailing newline is stripped from the output.
The exit status for the  command can be interpreted according to the
rules for the \C{} function \code{wait()}.  
\end{funcdesc}

\begin{funcdesc}{getoutput}{cmd}
Like \code{getstatusoutput()}, except the exit status is ignored and
the return value is a string containing the command's output.  
\end{funcdesc}

\begin{funcdesc}{getstatus}{file}
Return the output of \samp{ls -ld \var{file}} as a string.  This
function uses the \code{getoutput()} function, and properly escapes
backslashes and dollar signs in the argument.
\end{funcdesc}

Example:

\begin{verbatim}
>>> import commands
>>> commands.getstatusoutput('ls /bin/ls')
(0, '/bin/ls')
>>> commands.getstatusoutput('cat /bin/junk')
(256, 'cat: /bin/junk: No such file or directory')
>>> commands.getstatusoutput('/bin/junk')
(256, 'sh: /bin/junk: not found')
>>> commands.getoutput('ls /bin/ls')
'/bin/ls'
>>> commands.getstatus('/bin/ls')
'-rwxr-xr-x  1 root        13352 Oct 14  1994 /bin/ls'
\end{verbatim}


\chapter{The Python Debugger}

\declaremodule{standard}{pdb}
\modulesynopsis{The Python debugger for interactive interpreters.}


The module \module{pdb} defines an interactive source code
debugger\index{debugging} for Python programs.  It supports setting
(conditional) breakpoints and single stepping at the source line
level, inspection of stack frames, source code listing, and evaluation
of arbitrary Python code in the context of any stack frame.  It also
supports post-mortem debugging and can be called under program
control.

The debugger is extensible --- it is actually defined as the class
\class{Pdb}\withsubitem{(class in pdb)}{\ttindex{Pdb}}.
This is currently undocumented but easily understood by reading the
source.  The extension interface uses the modules
\module{bdb}\refstmodindex{bdb} (undocumented) and
\refmodule{cmd}\refstmodindex{cmd}.

A primitive windowing version of the debugger also exists --- this is
module \module{wdb}\refstmodindex{wdb}, which requires
\module{stdwin}\refbimodindex{stdwin}.

The debugger's prompt is \samp{(Pdb) }.
Typical usage to run a program under control of the debugger is:

\begin{verbatim}
>>> import pdb
>>> import mymodule
>>> pdb.run('mymodule.test()')
> <string>(0)?()
(Pdb) continue
> <string>(1)?()
(Pdb) continue
NameError: 'spam'
> <string>(1)?()
(Pdb) 
\end{verbatim}

\file{pdb.py} can also be invoked as
a script to debug other scripts.  For example:

\begin{verbatim}
python /usr/local/lib/python1.5/pdb.py myscript.py
\end{verbatim}

Typical usage to inspect a crashed program is:

\begin{verbatim}
>>> import pdb
>>> import mymodule
>>> mymodule.test()
Traceback (innermost last):
  File "<stdin>", line 1, in ?
  File "./mymodule.py", line 4, in test
    test2()
  File "./mymodule.py", line 3, in test2
    print spam
NameError: spam
>>> pdb.pm()
> ./mymodule.py(3)test2()
-> print spam
(Pdb) 
\end{verbatim}

The module defines the following functions; each enters the debugger
in a slightly different way:

\begin{funcdesc}{run}{statement\optional{, globals\optional{, locals}}}
Execute the \var{statement} (given as a string) under debugger
control.  The debugger prompt appears before any code is executed; you
can set breakpoints and type \samp{continue}, or you can step through
the statement using \samp{step} or \samp{next} (all these commands are
explained below).  The optional \var{globals} and \var{locals}
arguments specify the environment in which the code is executed; by
default the dictionary of the module \module{__main__} is used.  (See
the explanation of the \keyword{exec} statement or the
\function{eval()} built-in function.)
\end{funcdesc}

\begin{funcdesc}{runeval}{expression\optional{, globals\optional{, locals}}}
Evaluate the \var{expression} (given as a a string) under debugger
control.  When \function{runeval()} returns, it returns the value of the
expression.  Otherwise this function is similar to
\function{run()}.
\end{funcdesc}

\begin{funcdesc}{runcall}{function\optional{, argument, ...}}
Call the \var{function} (a function or method object, not a string)
with the given arguments.  When \function{runcall()} returns, it returns
whatever the function call returned.  The debugger prompt appears as
soon as the function is entered.
\end{funcdesc}

\begin{funcdesc}{set_trace}{}
Enter the debugger at the calling stack frame.  This is useful to
hard-code a breakpoint at a given point in a program, even if the code
is not otherwise being debugged (e.g. when an assertion fails).
\end{funcdesc}

\begin{funcdesc}{post_mortem}{traceback}
Enter post-mortem debugging of the given \var{traceback} object.
\end{funcdesc}

\begin{funcdesc}{pm}{}
Enter post-mortem debugging of the traceback found in
\code{sys.last_traceback}.
\end{funcdesc}


\section{Debugger Commands \label{debugger-commands}}

The debugger recognizes the following commands.  Most commands can be
abbreviated to one or two letters; e.g. \samp{h(elp)} means that
either \samp{h} or \samp{help} can be used to enter the help
command (but not \samp{he} or \samp{hel}, nor \samp{H} or
\samp{Help} or \samp{HELP}).  Arguments to commands must be
separated by whitespace (spaces or tabs).  Optional arguments are
enclosed in square brackets (\samp{[]}) in the command syntax; the
square brackets must not be typed.  Alternatives in the command syntax
are separated by a vertical bar (\samp{|}).

Entering a blank line repeats the last command entered.  Exception: if
the last command was a \samp{list} command, the next 11 lines are
listed.

Commands that the debugger doesn't recognize are assumed to be Python
statements and are executed in the context of the program being
debugged.  Python statements can also be prefixed with an exclamation
point (\samp{!}).  This is a powerful way to inspect the program
being debugged; it is even possible to change a variable or call a
function.  When an
exception occurs in such a statement, the exception name is printed
but the debugger's state is not changed.

Multiple commands may be entered on a single line, separated by
\samp{;;}.  (A single \samp{;} is not used as it is
the separator for multiple commands in a line that is passed to
the Python parser.)
No intelligence is applied to separating the commands;
the input is split at the first \samp{;;} pair, even if it is in
the middle of a quoted string.

The debugger supports aliases.  Aliases can have parameters which
allows one a certain level of adaptability to the context under
examination.

If a file \file{.pdbrc}
\indexii{.pdbrc}{file}\indexiii{debugger}{configuration}{file}
exists in the user's home directory or in the current directory, it is
read in and executed as if it had been typed at the debugger prompt.
This is particularly useful for aliases.  If both files exist, the one
in the home directory is read first and aliases defined there can be
overridden by the local file.

\begin{description}

\item[h(elp) \optional{\var{command}}]

Without argument, print the list of available commands.  With a
\var{command} as argument, print help about that command.  \samp{help
pdb} displays the full documentation file; if the environment variable
\envvar{PAGER} is defined, the file is piped through that command
instead.  Since the \var{command} argument must be an identifier,
\samp{help exec} must be entered to get help on the \samp{!} command.

\item[w(here)]

Print a stack trace, with the most recent frame at the bottom.  An
arrow indicates the current frame, which determines the context of
most commands.

\item[d(own)]

Move the current frame one level down in the stack trace
(to an newer frame).

\item[u(p)]

Move the current frame one level up in the stack trace
(to a older frame).

\item[b(reak) \optional{\optional{\var{filename}:}\var{lineno}\code{\Large{|}}\var{function}\optional{, \var{condition}}}]

With a \var{lineno} argument, set a break there in the current
file.  With a \var{function} argument, set a break at the first
executable statement within that function.
The line number may be prefixed with a filename and a colon,
to specify a breakpoint in another file (probably one that
hasn't been loaded yet).  The file is searched on \code{sys.path}.
Note that each breakpoint is assigned a number to which all the other
breakpoint commands refer.

If a second argument is present, it is an expression which must
evaluate to true before the breakpoint is honored.

Without argument, list all breaks, including for each breakpoint,
the number of times that breakpoint has been hit, the current
ignore count, and the associated condition if any.

\item[tbreak \optional{\optional{\var{filename}:}\var{lineno}\code{\Large{|}}\var{function}\optional{, \var{condition}}}]

Temporary breakpoint, which is removed automatically when it is
first hit.  The arguments are the same as break.

\item[cl(ear) \optional{\var{bpnumber} \optional{\var{bpnumber ...}}}]

With a space separated list of breakpoint numbers, clear those
breakpoints.  Without argument, clear all breaks (but first
ask confirmation).

\item[disable \optional{\var{bpnumber} \optional{\var{bpnumber ...}}}]

Disables the breakpoints given as a space separated list of
breakpoint numbers.  Disabling a breakpoint means it cannot cause
the program to stop execution, but unlike clearing a breakpoint, it
remains in the list of breakpoints and can be (re-)enabled.

\item[enable \optional{\var{bpnumber} \optional{\var{bpnumber ...}}}]

Enables the breakpoints specified.

\item[ignore \var{bpnumber} \optional{\var{count}}]

Sets the ignore count for the given breakpoint number.  If
count is omitted, the ignore count is set to 0.  A breakpoint
becomes active when the ignore count is zero.  When non-zero,
the count is decremented each time the breakpoint is reached
and the breakpoint is not disabled and any associated condition
evaluates to true.

\item[condition \var{bpnumber} \optional{\var{condition}}]

Condition is an expression which must evaluate to true before
the breakpoint is honored.  If condition is absent, any existing
condition is removed; i.e., the breakpoint is made unconditional.

\item[s(tep)]

Execute the current line, stop at the first possible occasion
(either in a function that is called or on the next line in the
current function).

\item[n(ext)]

Continue execution until the next line in the current function
is reached or it returns.  (The difference between \samp{next} and
\samp{step} is that \samp{step} stops inside a called function, while
\samp{next} executes called functions at (nearly) full speed, only
stopping at the next line in the current function.)

\item[r(eturn)]

Continue execution until the current function returns.

\item[c(ont(inue))]

Continue execution, only stop when a breakpoint is encountered.

\item[l(ist) \optional{\var{first\optional{, last}}}]

List source code for the current file.  Without arguments, list 11
lines around the current line or continue the previous listing.  With
one argument, list 11 lines around at that line.  With two arguments,
list the given range; if the second argument is less than the first,
it is interpreted as a count.

\item[a(rgs)]

Print the argument list of the current function.

\item[p \var{expression}]

Evaluate the \var{expression} in the current context and print its
value.  (Note: \samp{print} can also be used, but is not a debugger
command --- this executes the Python \keyword{print} statement.)

\item[alias \optional{\var{name} \optional{command}}]

Creates an alias called \var{name} that executes \var{command}.  The
command must \emph{not} be enclosed in quotes.  Replaceable parameters
can be indicated by \samp{\%1}, \samp{\%2}, and so on, while \samp{\%*} is
replaced by all the parameters.  If no command is given, the current
alias for \var{name} is shown. If no arguments are given, all
aliases are listed.

Aliases may be nested and can contain anything that can be
legally typed at the pdb prompt.  Note that internal pdb commands
\emph{can} be overridden by aliases.  Such a command is
then hidden until the alias is removed.  Aliasing is recursively
applied to the first word of the command line; all other words
in the line are left alone.

As an example, here are two useful aliases (especially when placed
in the \file{.pdbrc} file):

\begin{verbatim}
#Print instance variables (usage "pi classInst")
alias pi for k in %1.__dict__.keys(): print "%1.",k,"=",%1.__dict__[k]
#Print instance variables in self
alias ps pi self
\end{verbatim}
		
\item[unalias \var{name}]

Deletes the specified alias.

\item[\optional{!}\var{statement}]

Execute the (one-line) \var{statement} in the context of
the current stack frame.
The exclamation point can be omitted unless the first word
of the statement resembles a debugger command.
To set a global variable, you can prefix the assignment
command with a \samp{global} command on the same line, e.g.:

\begin{verbatim}
(Pdb) global list_options; list_options = ['-l']
(Pdb)
\end{verbatim}

\item[q(uit)]

Quit from the debugger.
The program being executed is aborted.

\end{description}

\section{How It Works}

Some changes were made to the interpreter:

\begin{itemize}
\item \code{sys.settrace(\var{func})} sets the global trace function
\item there can also a local trace function (see later)
\end{itemize}

Trace functions have three arguments: \var{frame}, \var{event}, and
\var{arg}. \var{frame} is the current stack frame.  \var{event} is a
string: \code{'call'}, \code{'line'}, \code{'return'} or
\code{'exception'}.  \var{arg} depends on the event type.

The global trace function is invoked (with \var{event} set to
\code{'call'}) whenever a new local scope is entered; it should return
a reference to the local trace function to be used that scope, or
\code{None} if the scope shouldn't be traced.

The local trace function should return a reference to itself (or to
another function for further tracing in that scope), or \code{None} to
turn off tracing in that scope.

Instance methods are accepted (and very useful!) as trace functions.

The events have the following meaning:

\begin{description}

\item[\code{'call'}]
A function is called (or some other code block entered).  The global
trace function is called; arg is the argument list to the function;
the return value specifies the local trace function.

\item[\code{'line'}]
The interpreter is about to execute a new line of code (sometimes
multiple line events on one line exist).  The local trace function is
called; arg in None; the return value specifies the new local trace
function.

\item[\code{'return'}]
A function (or other code block) is about to return.  The local trace
function is called; arg is the value that will be returned.  The trace
function's return value is ignored.

\item[\code{'exception'}]
An exception has occurred.  The local trace function is called; arg is
a triple (exception, value, traceback); the return value specifies the
new local trace function

\end{description}

Note that as an exception is propagated down the chain of callers, an
\code{'exception'} event is generated at each level.

For more information on code and frame objects, refer to the
\citetitle[../ref/ref.html]{Python Reference Manual}.
			% The Python Debugger

\chapter{The Python Profilers \label{profile}}

\sectionauthor{James Roskind}{}

Copyright \copyright{} 1994, by InfoSeek Corporation, all rights reserved.
\index{InfoSeek Corporation}

Written by James Roskind.\footnote{
  Updated and converted to \LaTeX\ by Guido van Rossum.
  Further updated by Armin Rigo to integrate the documentation for the new
  \module{cProfile} module of Python 2.5.}

Permission to use, copy, modify, and distribute this Python software
and its associated documentation for any purpose (subject to the
restriction in the following sentence) without fee is hereby granted,
provided that the above copyright notice appears in all copies, and
that both that copyright notice and this permission notice appear in
supporting documentation, and that the name of InfoSeek not be used in
advertising or publicity pertaining to distribution of the software
without specific, written prior permission.  This permission is
explicitly restricted to the copying and modification of the software
to remain in Python, compiled Python, or other languages (such as C)
wherein the modified or derived code is exclusively imported into a
Python module.

INFOSEEK CORPORATION DISCLAIMS ALL WARRANTIES WITH REGARD TO THIS
SOFTWARE, INCLUDING ALL IMPLIED WARRANTIES OF MERCHANTABILITY AND
FITNESS. IN NO EVENT SHALL INFOSEEK CORPORATION BE LIABLE FOR ANY
SPECIAL, INDIRECT OR CONSEQUENTIAL DAMAGES OR ANY DAMAGES WHATSOEVER
RESULTING FROM LOSS OF USE, DATA OR PROFITS, WHETHER IN AN ACTION OF
CONTRACT, NEGLIGENCE OR OTHER TORTIOUS ACTION, ARISING OUT OF OR IN
CONNECTION WITH THE USE OR PERFORMANCE OF THIS SOFTWARE.


The profiler was written after only programming in Python for 3 weeks.
As a result, it is probably clumsy code, but I don't know for sure yet
'cause I'm a beginner :-).  I did work hard to make the code run fast,
so that profiling would be a reasonable thing to do.  I tried not to
repeat code fragments, but I'm sure I did some stuff in really awkward
ways at times.  Please send suggestions for improvements to:
\email{jar@netscape.com}.  I won't promise \emph{any} support.  ...but
I'd appreciate the feedback.


\section{Introduction to the profilers}
\nodename{Profiler Introduction}

A \dfn{profiler} is a program that describes the run time performance
of a program, providing a variety of statistics.  This documentation
describes the profiler functionality provided in the modules
\module{profile} and \module{pstats}.  This profiler provides
\dfn{deterministic profiling} of any Python programs.  It also
provides a series of report generation tools to allow users to rapidly
examine the results of a profile operation.
\index{deterministic profiling}
\index{profiling, deterministic}

The Python standard library provides three different profilers:

\begin{enumerate}
\item \module{profile}, a pure Python module, described in the sequel.
  Copyright \copyright{} 1994, by InfoSeek Corporation.
  \versionchanged[also reports the time spent in calls to built-in
  functions and methods]{2.4}

\item \module{cProfile}, a module written in C, with a reasonable
  overhead that makes it suitable for profiling long-running programs.
  Based on \module{lsprof}, contributed by Brett Rosen and Ted Czotter.
  \versionadded{2.5}

\item \module{hotshot}, a C module focusing on minimizing the overhead
  while profiling, at the expense of long data post-processing times.
  \versionchanged[the results should be more meaningful than in the
  past: the timing core contained a critical bug]{2.5}
\end{enumerate}

The \module{profile} and \module{cProfile} modules export the same
interface, so they are mostly interchangeables; \module{cProfile} has a
much lower overhead but is not so far as well-tested and might not be
available on all systems.  \module{cProfile} is really a compatibility
layer on top of the internal \module{_lsprof} module.  The
\module{hotshot} module is reserved to specialized usages.

%\section{How Is This Profiler Different From The Old Profiler?}
%\nodename{Profiler Changes}
%
%(This section is of historical importance only; the old profiler
%discussed here was last seen in Python 1.1.)
%
%The big changes from old profiling module are that you get more
%information, and you pay less CPU time.  It's not a trade-off, it's a
%trade-up.
%
%To be specific:
%
%\begin{description}
%
%\item[Bugs removed:]
%Local stack frame is no longer molested, execution time is now charged
%to correct functions.
%
%\item[Accuracy increased:]
%Profiler execution time is no longer charged to user's code,
%calibration for platform is supported, file reads are not done \emph{by}
%profiler \emph{during} profiling (and charged to user's code!).
%
%\item[Speed increased:]
%Overhead CPU cost was reduced by more than a factor of two (perhaps a
%factor of five), lightweight profiler module is all that must be
%loaded, and the report generating module (\module{pstats}) is not needed
%during profiling.
%
%\item[Recursive functions support:]
%Cumulative times in recursive functions are correctly calculated;
%recursive entries are counted.
%
%\item[Large growth in report generating UI:]
%Distinct profiles runs can be added together forming a comprehensive
%report; functions that import statistics take arbitrary lists of
%files; sorting criteria is now based on keywords (instead of 4 integer
%options); reports shows what functions were profiled as well as what
%profile file was referenced; output format has been improved.
%
%\end{description}


\section{Instant User's Manual \label{profile-instant}}

This section is provided for users that ``don't want to read the
manual.'' It provides a very brief overview, and allows a user to
rapidly perform profiling on an existing application.

To profile an application with a main entry point of \function{foo()},
you would add the following to your module:

\begin{verbatim}
import cProfile
cProfile.run('foo()')
\end{verbatim}

(Use \module{profile} instead of \module{cProfile} if the latter is not
available on your system.)

The above action would cause \function{foo()} to be run, and a series of
informative lines (the profile) to be printed.  The above approach is
most useful when working with the interpreter.  If you would like to
save the results of a profile into a file for later examination, you
can supply a file name as the second argument to the \function{run()}
function:

\begin{verbatim}
import cProfile
cProfile.run('foo()', 'fooprof')
\end{verbatim}

The file \file{cProfile.py} can also be invoked as
a script to profile another script.  For example:

\begin{verbatim}
python -m cProfile myscript.py
\end{verbatim}

\file{cProfile.py} accepts two optional arguments on the command line:

\begin{verbatim}
cProfile.py [-o output_file] [-s sort_order]
\end{verbatim}

\programopt{-s} only applies to standard output (\programopt{-o} is
not supplied).  Look in the \class{Stats} documentation for valid sort
values.

When you wish to review the profile, you should use the methods in the
\module{pstats} module.  Typically you would load the statistics data as
follows:

\begin{verbatim}
import pstats
p = pstats.Stats('fooprof')
\end{verbatim}

The class \class{Stats} (the above code just created an instance of
this class) has a variety of methods for manipulating and printing the
data that was just read into \code{p}.  When you ran
\function{cProfile.run()} above, what was printed was the result of three
method calls:

\begin{verbatim}
p.strip_dirs().sort_stats(-1).print_stats()
\end{verbatim}

The first method removed the extraneous path from all the module
names. The second method sorted all the entries according to the
standard module/line/name string that is printed.
%(this is to comply with the semantics of the old profiler).
The third method printed out
all the statistics.  You might try the following sort calls:

\begin{verbatim}
p.sort_stats('name')
p.print_stats()
\end{verbatim}

The first call will actually sort the list by function name, and the
second call will print out the statistics.  The following are some
interesting calls to experiment with:

\begin{verbatim}
p.sort_stats('cumulative').print_stats(10)
\end{verbatim}

This sorts the profile by cumulative time in a function, and then only
prints the ten most significant lines.  If you want to understand what
algorithms are taking time, the above line is what you would use.

If you were looking to see what functions were looping a lot, and
taking a lot of time, you would do:

\begin{verbatim}
p.sort_stats('time').print_stats(10)
\end{verbatim}

to sort according to time spent within each function, and then print
the statistics for the top ten functions.

You might also try:

\begin{verbatim}
p.sort_stats('file').print_stats('__init__')
\end{verbatim}

This will sort all the statistics by file name, and then print out
statistics for only the class init methods (since they are spelled
with \code{__init__} in them).  As one final example, you could try:

\begin{verbatim}
p.sort_stats('time', 'cum').print_stats(.5, 'init')
\end{verbatim}

This line sorts statistics with a primary key of time, and a secondary
key of cumulative time, and then prints out some of the statistics.
To be specific, the list is first culled down to 50\% (re: \samp{.5})
of its original size, then only lines containing \code{init} are
maintained, and that sub-sub-list is printed.

If you wondered what functions called the above functions, you could
now (\code{p} is still sorted according to the last criteria) do:

\begin{verbatim}
p.print_callers(.5, 'init')
\end{verbatim}

and you would get a list of callers for each of the listed functions.

If you want more functionality, you're going to have to read the
manual, or guess what the following functions do:

\begin{verbatim}
p.print_callees()
p.add('fooprof')
\end{verbatim}

Invoked as a script, the \module{pstats} module is a statistics
browser for reading and examining profile dumps.  It has a simple
line-oriented interface (implemented using \refmodule{cmd}) and
interactive help.

\section{What Is Deterministic Profiling?}
\nodename{Deterministic Profiling}

\dfn{Deterministic profiling} is meant to reflect the fact that all
\emph{function call}, \emph{function return}, and \emph{exception} events
are monitored, and precise timings are made for the intervals between
these events (during which time the user's code is executing).  In
contrast, \dfn{statistical profiling} (which is not done by this
module) randomly samples the effective instruction pointer, and
deduces where time is being spent.  The latter technique traditionally
involves less overhead (as the code does not need to be instrumented),
but provides only relative indications of where time is being spent.

In Python, since there is an interpreter active during execution, the
presence of instrumented code is not required to do deterministic
profiling.  Python automatically provides a \dfn{hook} (optional
callback) for each event.  In addition, the interpreted nature of
Python tends to add so much overhead to execution, that deterministic
profiling tends to only add small processing overhead in typical
applications.  The result is that deterministic profiling is not that
expensive, yet provides extensive run time statistics about the
execution of a Python program.

Call count statistics can be used to identify bugs in code (surprising
counts), and to identify possible inline-expansion points (high call
counts).  Internal time statistics can be used to identify ``hot
loops'' that should be carefully optimized.  Cumulative time
statistics should be used to identify high level errors in the
selection of algorithms.  Note that the unusual handling of cumulative
times in this profiler allows statistics for recursive implementations
of algorithms to be directly compared to iterative implementations.


\section{Reference Manual -- \module{profile} and \module{cProfile}}

\declaremodule{standard}{profile}
\declaremodule{standard}{cProfile}
\modulesynopsis{Python profiler}



The primary entry point for the profiler is the global function
\function{profile.run()} (resp. \function{cProfile.run()}).
It is typically used to create any profile
information.  The reports are formatted and printed using methods of
the class \class{pstats.Stats}.  The following is a description of all
of these standard entry points and functions.  For a more in-depth
view of some of the code, consider reading the later section on
Profiler Extensions, which includes discussion of how to derive
``better'' profilers from the classes presented, or reading the source
code for these modules.

\begin{funcdesc}{run}{command\optional{, filename}}

This function takes a single argument that has can be passed to the
\keyword{exec} statement, and an optional file name.  In all cases this
routine attempts to \keyword{exec} its first argument, and gather profiling
statistics from the execution. If no file name is present, then this
function automatically prints a simple profiling report, sorted by the
standard name string (file/line/function-name) that is presented in
each line.  The following is a typical output from such a call:

\begin{verbatim}
      2706 function calls (2004 primitive calls) in 4.504 CPU seconds

Ordered by: standard name

ncalls  tottime  percall  cumtime  percall filename:lineno(function)
     2    0.006    0.003    0.953    0.477 pobject.py:75(save_objects)
  43/3    0.533    0.012    0.749    0.250 pobject.py:99(evaluate)
 ...
\end{verbatim}

The first line indicates that 2706 calls were
monitored.  Of those calls, 2004 were \dfn{primitive}.  We define
\dfn{primitive} to mean that the call was not induced via recursion.
The next line: \code{Ordered by:\ standard name}, indicates that
the text string in the far right column was used to sort the output.
The column headings include:

\begin{description}

\item[ncalls ]
for the number of calls,

\item[tottime ]
for the total time spent in the given function (and excluding time
made in calls to sub-functions),

\item[percall ]
is the quotient of \code{tottime} divided by \code{ncalls}

\item[cumtime ]
is the total time spent in this and all subfunctions (from invocation
till exit). This figure is accurate \emph{even} for recursive
functions.

\item[percall ]
is the quotient of \code{cumtime} divided by primitive calls

\item[filename:lineno(function) ]
provides the respective data of each function

\end{description}

When there are two numbers in the first column (for example,
\samp{43/3}), then the latter is the number of primitive calls, and
the former is the actual number of calls.  Note that when the function
does not recurse, these two values are the same, and only the single
figure is printed.

\end{funcdesc}

\begin{funcdesc}{runctx}{command, globals, locals\optional{, filename}}
This function is similar to \function{run()}, with added
arguments to supply the globals and locals dictionaries for the
\var{command} string.
\end{funcdesc}

Analysis of the profiler data is done using this class from the
\module{pstats} module:

% now switch modules....
% (This \stmodindex use may be hard to change ;-( )
\stmodindex{pstats}

\begin{classdesc}{Stats}{filename\optional{, \moreargs}}
This class constructor creates an instance of a ``statistics object''
from a \var{filename} (or set of filenames).  \class{Stats} objects are
manipulated by methods, in order to print useful reports.

The file selected by the above constructor must have been created by
the corresponding version of \module{profile} or \module{cProfile}.
To be specific, there is
\emph{no} file compatibility guaranteed with future versions of this
profiler, and there is no compatibility with files produced by other
profilers.
%(such as the old system profiler).

If several files are provided, all the statistics for identical
functions will be coalesced, so that an overall view of several
processes can be considered in a single report.  If additional files
need to be combined with data in an existing \class{Stats} object, the
\method{add()} method can be used.
\end{classdesc}


\subsection{The \class{Stats} Class \label{profile-stats}}

\class{Stats} objects have the following methods:

\begin{methoddesc}[Stats]{strip_dirs}{}
This method for the \class{Stats} class removes all leading path
information from file names.  It is very useful in reducing the size
of the printout to fit within (close to) 80 columns.  This method
modifies the object, and the stripped information is lost.  After
performing a strip operation, the object is considered to have its
entries in a ``random'' order, as it was just after object
initialization and loading.  If \method{strip_dirs()} causes two
function names to be indistinguishable (they are on the same
line of the same filename, and have the same function name), then the
statistics for these two entries are accumulated into a single entry.
\end{methoddesc}


\begin{methoddesc}[Stats]{add}{filename\optional{, \moreargs}}
This method of the \class{Stats} class accumulates additional
profiling information into the current profiling object.  Its
arguments should refer to filenames created by the corresponding
version of \function{profile.run()} or \function{cProfile.run()}.
Statistics for identically named
(re: file, line, name) functions are automatically accumulated into
single function statistics.
\end{methoddesc}

\begin{methoddesc}[Stats]{dump_stats}{filename}
Save the data loaded into the \class{Stats} object to a file named
\var{filename}.  The file is created if it does not exist, and is
overwritten if it already exists.  This is equivalent to the method of
the same name on the \class{profile.Profile} and
\class{cProfile.Profile} classes.
\versionadded{2.3}
\end{methoddesc}

\begin{methoddesc}[Stats]{sort_stats}{key\optional{, \moreargs}}
This method modifies the \class{Stats} object by sorting it according
to the supplied criteria.  The argument is typically a string
identifying the basis of a sort (example: \code{'time'} or
\code{'name'}).

When more than one key is provided, then additional keys are used as
secondary criteria when there is equality in all keys selected
before them.  For example, \code{sort_stats('name', 'file')} will sort
all the entries according to their function name, and resolve all ties
(identical function names) by sorting by file name.

Abbreviations can be used for any key names, as long as the
abbreviation is unambiguous.  The following are the keys currently
defined:

\begin{tableii}{l|l}{code}{Valid Arg}{Meaning}
  \lineii{'calls'}{call count}
  \lineii{'cumulative'}{cumulative time}
  \lineii{'file'}{file name}
  \lineii{'module'}{file name}
  \lineii{'pcalls'}{primitive call count}
  \lineii{'line'}{line number}
  \lineii{'name'}{function name}
  \lineii{'nfl'}{name/file/line}
  \lineii{'stdname'}{standard name}
  \lineii{'time'}{internal time}
\end{tableii}

Note that all sorts on statistics are in descending order (placing
most time consuming items first), where as name, file, and line number
searches are in ascending order (alphabetical). The subtle
distinction between \code{'nfl'} and \code{'stdname'} is that the
standard name is a sort of the name as printed, which means that the
embedded line numbers get compared in an odd way.  For example, lines
3, 20, and 40 would (if the file names were the same) appear in the
string order 20, 3 and 40.  In contrast, \code{'nfl'} does a numeric
compare of the line numbers.  In fact, \code{sort_stats('nfl')} is the
same as \code{sort_stats('name', 'file', 'line')}.

%For compatibility with the old profiler,
For backward-compatibility reasons, the numeric arguments
\code{-1}, \code{0}, \code{1}, and \code{2} are permitted.  They are
interpreted as \code{'stdname'}, \code{'calls'}, \code{'time'}, and
\code{'cumulative'} respectively.  If this old style format (numeric)
is used, only one sort key (the numeric key) will be used, and
additional arguments will be silently ignored.
\end{methoddesc}


\begin{methoddesc}[Stats]{reverse_order}{}
This method for the \class{Stats} class reverses the ordering of the basic
list within the object.  %This method is provided primarily for
%compatibility with the old profiler.
Note that by default ascending vs descending order is properly selected
based on the sort key of choice.
\end{methoddesc}

\begin{methoddesc}[Stats]{print_stats}{\optional{restriction, \moreargs}}
This method for the \class{Stats} class prints out a report as described
in the \function{profile.run()} definition.

The order of the printing is based on the last \method{sort_stats()}
operation done on the object (subject to caveats in \method{add()} and
\method{strip_dirs()}).

The arguments provided (if any) can be used to limit the list down to
the significant entries.  Initially, the list is taken to be the
complete set of profiled functions.  Each restriction is either an
integer (to select a count of lines), or a decimal fraction between
0.0 and 1.0 inclusive (to select a percentage of lines), or a regular
expression (to pattern match the standard name that is printed; as of
Python 1.5b1, this uses the Perl-style regular expression syntax
defined by the \refmodule{re} module).  If several restrictions are
provided, then they are applied sequentially.  For example:

\begin{verbatim}
print_stats(.1, 'foo:')
\end{verbatim}

would first limit the printing to first 10\% of list, and then only
print functions that were part of filename \file{.*foo:}.  In
contrast, the command:

\begin{verbatim}
print_stats('foo:', .1)
\end{verbatim}

would limit the list to all functions having file names \file{.*foo:},
and then proceed to only print the first 10\% of them.
\end{methoddesc}


\begin{methoddesc}[Stats]{print_callers}{\optional{restriction, \moreargs}}
This method for the \class{Stats} class prints a list of all functions
that called each function in the profiled database.  The ordering is
identical to that provided by \method{print_stats()}, and the definition
of the restricting argument is also identical.  Each caller is reported on
its own line.  The format differs slightly depending on the profiler that
produced the stats:

\begin{itemize}
\item With \module{profile}, a number is shown in parentheses after each
  caller to show how many times this specific call was made.  For
  convenience, a second non-parenthesized number repeats the cumulative
  time spent in the function at the right.

\item With \module{cProfile}, each caller is preceeded by three numbers:
  the number of times this specific call was made, and the total and
  cumulative times spent in the current function while it was invoked by
  this specific caller.
\end{itemize}
\end{methoddesc}

\begin{methoddesc}[Stats]{print_callees}{\optional{restriction, \moreargs}}
This method for the \class{Stats} class prints a list of all function
that were called by the indicated function.  Aside from this reversal
of direction of calls (re: called vs was called by), the arguments and
ordering are identical to the \method{print_callers()} method.
\end{methoddesc}


\section{Limitations \label{profile-limits}}

One limitation has to do with accuracy of timing information.
There is a fundamental problem with deterministic profilers involving
accuracy.  The most obvious restriction is that the underlying ``clock''
is only ticking at a rate (typically) of about .001 seconds.  Hence no
measurements will be more accurate than the underlying clock.  If
enough measurements are taken, then the ``error'' will tend to average
out. Unfortunately, removing this first error induces a second source
of error.

The second problem is that it ``takes a while'' from when an event is
dispatched until the profiler's call to get the time actually
\emph{gets} the state of the clock.  Similarly, there is a certain lag
when exiting the profiler event handler from the time that the clock's
value was obtained (and then squirreled away), until the user's code
is once again executing.  As a result, functions that are called many
times, or call many functions, will typically accumulate this error.
The error that accumulates in this fashion is typically less than the
accuracy of the clock (less than one clock tick), but it
\emph{can} accumulate and become very significant.

The problem is more important with \module{profile} than with the
lower-overhead \module{cProfile}.  For this reason, \module{profile}
provides a means of calibrating itself for a given platform so that
this error can be probabilistically (on the average) removed.
After the profiler is calibrated, it will be more accurate (in a least
square sense), but it will sometimes produce negative numbers (when
call counts are exceptionally low, and the gods of probability work
against you :-). )  Do \emph{not} be alarmed by negative numbers in
the profile.  They should \emph{only} appear if you have calibrated
your profiler, and the results are actually better than without
calibration.


\section{Calibration \label{profile-calibration}}

The profiler of the \module{profile} module subtracts a constant from each
event handling time to compensate for the overhead of calling the time
function, and socking away the results.  By default, the constant is 0.
The following procedure can
be used to obtain a better constant for a given platform (see discussion
in section Limitations above).

\begin{verbatim}
import profile
pr = profile.Profile()
for i in range(5):
    print pr.calibrate(10000)
\end{verbatim}

The method executes the number of Python calls given by the argument,
directly and again under the profiler, measuring the time for both.
It then computes the hidden overhead per profiler event, and returns
that as a float.  For example, on an 800 MHz Pentium running
Windows 2000, and using Python's time.clock() as the timer,
the magical number is about 12.5e-6.

The object of this exercise is to get a fairly consistent result.
If your computer is \emph{very} fast, or your timer function has poor
resolution, you might have to pass 100000, or even 1000000, to get
consistent results.

When you have a consistent answer,
there are three ways you can use it:\footnote{Prior to Python 2.2, it
  was necessary to edit the profiler source code to embed the bias as
  a literal number.  You still can, but that method is no longer
  described, because no longer needed.}

\begin{verbatim}
import profile

# 1. Apply computed bias to all Profile instances created hereafter.
profile.Profile.bias = your_computed_bias

# 2. Apply computed bias to a specific Profile instance.
pr = profile.Profile()
pr.bias = your_computed_bias

# 3. Specify computed bias in instance constructor.
pr = profile.Profile(bias=your_computed_bias)
\end{verbatim}

If you have a choice, you are better off choosing a smaller constant, and
then your results will ``less often'' show up as negative in profile
statistics.


\section{Extensions --- Deriving Better Profilers}
\nodename{Profiler Extensions}

The \class{Profile} class of both modules, \module{profile} and
\module{cProfile}, were written so that
derived classes could be developed to extend the profiler.  The details
are not described here, as doing this successfully requires an expert
understanding of how the \class{Profile} class works internally.  Study
the source code of the module carefully if you want to
pursue this.

If all you want to do is change how current time is determined (for
example, to force use of wall-clock time or elapsed process time),
pass the timing function you want to the \class{Profile} class
constructor:

\begin{verbatim}
pr = profile.Profile(your_time_func)
\end{verbatim}

The resulting profiler will then call \function{your_time_func()}.

\begin{description}
\item[\class{profile.Profile}]
\function{your_time_func()} should return a single number, or a list of
numbers whose sum is the current time (like what \function{os.times()}
returns).  If the function returns a single time number, or the list of
returned numbers has length 2, then you will get an especially fast
version of the dispatch routine.

Be warned that you should calibrate the profiler class for the
timer function that you choose.  For most machines, a timer that
returns a lone integer value will provide the best results in terms of
low overhead during profiling.  (\function{os.times()} is
\emph{pretty} bad, as it returns a tuple of floating point values).  If
you want to substitute a better timer in the cleanest fashion,
derive a class and hardwire a replacement dispatch method that best
handles your timer call, along with the appropriate calibration
constant.

\item[\class{cProfile.Profile}]
\function{your_time_func()} should return a single number.  If it returns
plain integers, you can also invoke the class constructor with a second
argument specifying the real duration of one unit of time.  For example,
if \function{your_integer_time_func()} returns times measured in thousands
of seconds, you would constuct the \class{Profile} instance as follows:

\begin{verbatim}
pr = profile.Profile(your_integer_time_func, 0.001)
\end{verbatim}

As the \module{cProfile.Profile} class cannot be calibrated, custom
timer functions should be used with care and should be as fast as
possible.  For the best results with a custom timer, it might be
necessary to hard-code it in the C source of the internal
\module{_lsprof} module.

\end{description}
		% The Python Profiler

\chapter{Internet Protocols and Support \label{internet}}

\index{WWW}
\index{Internet}
\index{World-Wide Web}

The modules described in this chapter implement Internet protocols and 
support for related technology.  They are all implemented in Python.
Some of these modules require the presence of the system-dependent
module \module{socket}\refbimodindex{socket}, which is currently only
fully supported on \UNIX{} and Windows NT.  Here is an overview:

\localmoduletable
		% Internet Protocols
\section{Standard Module \sectcode{cgi}}
\label{module-cgi}
\stmodindex{cgi}
\indexii{WWW}{server}
\indexii{CGI}{protocol}
\indexii{HTTP}{protocol}
\indexii{MIME}{headers}
\index{URL}

\renewcommand{\indexsubitem}{(in module cgi)}

Support module for CGI (Common Gateway Interface) scripts.

This module defines a number of utilities for use by CGI scripts
written in Python.

\subsection{Introduction}
\nodename{Introduction to the CGI module}

A CGI script is invoked by an HTTP server, usually to process user
input submitted through an HTML \code{<FORM>} or \code{<ISINPUT>} element.

Most often, CGI scripts live in the server's special \code{cgi-bin}
directory.  The HTTP server places all sorts of information about the
request (such as the client's hostname, the requested URL, the query
string, and lots of other goodies) in the script's shell environment,
executes the script, and sends the script's output back to the client.

The script's input is connected to the client too, and sometimes the
form data is read this way; at other times the form data is passed via
the ``query string'' part of the URL.  This module (\code{cgi.py}) is intended
to take care of the different cases and provide a simpler interface to
the Python script.  It also provides a number of utilities that help
in debugging scripts, and the latest addition is support for file
uploads from a form (if your browser supports it -- Grail 0.3 and
Netscape 2.0 do).

The output of a CGI script should consist of two sections, separated
by a blank line.  The first section contains a number of headers,
telling the client what kind of data is following.  Python code to
generate a minimal header section looks like this:

\bcode\begin{verbatim}
print "Content-type: text/html"     # HTML is following
print                               # blank line, end of headers
\end{verbatim}\ecode
%
The second section is usually HTML, which allows the client software
to display nicely formatted text with header, in-line images, etc.
Here's Python code that prints a simple piece of HTML:

\bcode\begin{verbatim}
print "<TITLE>CGI script output</TITLE>"
print "<H1>This is my first CGI script</H1>"
print "Hello, world!"
\end{verbatim}\ecode
%
(It may not be fully legal HTML according to the letter of the
standard, but any browser will understand it.)

\subsection{Using the cgi module}
\nodename{Using the cgi module}

Begin by writing \code{import cgi}.  Don't use \code{from cgi import *} -- the
module defines all sorts of names for its own use or for backward 
compatibility that you don't want in your namespace.

It's best to use the \code{FieldStorage} class.  The other classes define in this 
module are provided mostly for backward compatibility.  Instantiate it 
exactly once, without arguments.  This reads the form contents from 
standard input or the environment (depending on the value of various 
environment variables set according to the CGI standard).  Since it may 
consume standard input, it should be instantiated only once.

The \code{FieldStorage} instance can be accessed as if it were a Python 
dictionary.  For instance, the following code (which assumes that the 
\code{Content-type} header and blank line have already been printed) checks that 
the fields \code{name} and \code{addr} are both set to a non-empty string:

\bcode\begin{verbatim}
form = cgi.FieldStorage()
form_ok = 0
if form.has_key("name") and form.has_key("addr"):
    if form["name"].value != "" and form["addr"].value != "":
        form_ok = 1
if not form_ok:
    print "<H1>Error</H1>"
    print "Please fill in the name and addr fields."
    return
...further form processing here...
\end{verbatim}\ecode
%
Here the fields, accessed through \code{form[key]}, are themselves instances
of \code{FieldStorage} (or \code{MiniFieldStorage}, depending on the form encoding).

If the submitted form data contains more than one field with the same
name, the object retrieved by \code{form[key]} is not a \code{(Mini)FieldStorage}
instance but a list of such instances.  If you expect this possibility
(i.e., when your HTML form comtains multiple fields with the same
name), use the \code{type()} function to determine whether you have a single
instance or a list of instances.  For example, here's code that
concatenates any number of username fields, separated by commas:

\bcode\begin{verbatim}
username = form["username"]
if type(username) is type([]):
    # Multiple username fields specified
    usernames = ""
    for item in username:
        if usernames:
            # Next item -- insert comma
            usernames = usernames + "," + item.value
        else:
            # First item -- don't insert comma
            usernames = item.value
else:
    # Single username field specified
    usernames = username.value
\end{verbatim}\ecode
%
If a field represents an uploaded file, the value attribute reads the 
entire file in memory as a string.  This may not be what you want.  You can 
test for an uploaded file by testing either the filename attribute or the 
file attribute.  You can then read the data at leasure from the file 
attribute:

\bcode\begin{verbatim}
fileitem = form["userfile"]
if fileitem.file:
    # It's an uploaded file; count lines
    linecount = 0
    while 1:
        line = fileitem.file.readline()
        if not line: break
        linecount = linecount + 1
\end{verbatim}\ecode
%
The file upload draft standard entertains the possibility of uploading
multiple files from one field (using a recursive \code{multipart/*}
encoding).  When this occurs, the item will be a dictionary-like
FieldStorage item.  This can be determined by testing its type
attribute, which should have the value \code{multipart/form-data} (or
perhaps another string beginning with \code{multipart/}  It this case, it
can be iterated over recursively just like the top-level form object.

When a form is submitted in the ``old'' format (as the query string or as a 
single data part of type \code{application/x-www-form-urlencoded}), the items 
will actually be instances of the class \code{MiniFieldStorage}.  In this case,
the list, file and filename attributes are always \code{None}.


\subsection{Old classes}

These classes, present in earlier versions of the \code{cgi} module, are still 
supported for backward compatibility.  New applications should use the
FieldStorage class.

\code{SvFormContentDict}: single value form content as dictionary; assumes each 
field name occurs in the form only once.

\code{FormContentDict}: multiple value form content as dictionary (the form
items are lists of values).  Useful if your form contains multiple
fields with the same name.

Other classes (\code{FormContent}, \code{InterpFormContentDict}) are present for
backwards compatibility with really old applications only.  If you still 
use these and would be inconvenienced when they disappeared from a next 
version of this module, drop me a note.


\subsection{Functions}
\nodename{Functions in cgi module}

These are useful if you want more control, or if you want to employ
some of the algorithms implemented in this module in other
circumstances.

\begin{funcdesc}{parse}{fp}: Parse a query in the environment or from a file (default \code{sys.stdin}).
\end{funcdesc}

\begin{funcdesc}{parse_qs}{qs}: parse a query string given as a string argument (data of type 
\code{application/x-www-form-urlencoded}).
\end{funcdesc}

\begin{funcdesc}{parse_multipart}{fp\, pdict}: parse input of type \code{multipart/form-data} (for 
file uploads).  Arguments are \code{fp} for the input file and 
    \code{pdict} for the dictionary containing other parameters of \code{content-type} header

    Returns a dictionary just like \code{parse_qs()}: keys are the field names, each 
    value is a list of values for that field.  This is easy to use but not 
    much good if you are expecting megabytes to be uploaded -- in that case, 
    use the \code{FieldStorage} class instead which is much more flexible.  Note 
    that \code{content-type} is the raw, unparsed contents of the \code{content-type} 
    header.

    Note that this does not parse nested multipart parts -- use \code{FieldStorage} for 
    that.
\end{funcdesc}

\begin{funcdesc}{parse_header}{string}: parse a header like \code{Content-type} into a main
content-type and a dictionary of parameters.
\end{funcdesc}

\begin{funcdesc}{test}{}: robust test CGI script, usable as main program.
    Writes minimal HTTP headers and formats all information provided to
    the script in HTML form.
\end{funcdesc}

\begin{funcdesc}{print_environ}{}: format the shell environment in HTML.
\end{funcdesc}

\begin{funcdesc}{print_form}{form}: format a form in HTML.
\end{funcdesc}

\begin{funcdesc}{print_directory}{}: format the current directory in HTML.
\end{funcdesc}

\begin{funcdesc}{print_environ_usage}{}: print a list of useful (used by CGI) environment variables in
HTML.
\end{funcdesc}

\begin{funcdesc}{escape}{}: convert the characters ``\code{\&}'', ``\code{<}'' and ``\code{>}'' to HTML-safe
sequences.  Use this if you need to display text that might contain
such characters in HTML.  To translate URLs for inclusion in the HREF
attribute of an \code{<A>} tag, use \code{urllib.quote()}.
\end{funcdesc}


\subsection{Caring about security}

There's one important rule: if you invoke an external program (e.g.
via the \code{os.system()} or \code{os.popen()} functions), make very sure you don't
pass arbitrary strings received from the client to the shell.  This is
a well-known security hole whereby clever hackers anywhere on the web
can exploit a gullible CGI script to invoke arbitrary shell commands.
Even parts of the URL or field names cannot be trusted, since the
request doesn't have to come from your form!

To be on the safe side, if you must pass a string gotten from a form
to a shell command, you should make sure the string contains only
alphanumeric characters, dashes, underscores, and periods.


\subsection{Installing your CGI script on a Unix system}

Read the documentation for your HTTP server and check with your local
system administrator to find the directory where CGI scripts should be
installed; usually this is in a directory \code{cgi-bin} in the server tree.

Make sure that your script is readable and executable by ``others''; the
Unix file mode should be 755 (use \code{chmod 755 filename}).  Make sure
that the first line of the script contains \code{\#!} starting in column 1
followed by the pathname of the Python interpreter, for instance:

\bcode\begin{verbatim}
#!/usr/local/bin/python
\end{verbatim}\ecode
%
Make sure the Python interpreter exists and is executable by ``others''.

Make sure that any files your script needs to read or write are
readable or writable, respectively, by ``others'' -- their mode should
be 644 for readable and 666 for writable.  This is because, for
security reasons, the HTTP server executes your script as user
``nobody'', without any special privileges.  It can only read (write,
execute) files that everybody can read (write, execute).  The current
directory at execution time is also different (it is usually the
server's cgi-bin directory) and the set of environment variables is
also different from what you get at login.  in particular, don't count
on the shell's search path for executables (\code{\$PATH}) or the Python
module search path (\code{\$PYTHONPATH}) to be set to anything interesting.

If you need to load modules from a directory which is not on Python's
default module search path, you can change the path in your script,
before importing other modules, e.g.:

\bcode\begin{verbatim}
import sys
sys.path.insert(0, "/usr/home/joe/lib/python")
sys.path.insert(0, "/usr/local/lib/python")
\end{verbatim}\ecode
%
(This way, the directory inserted last will be searched first!)

Instructions for non-Unix systems will vary; check your HTTP server's
documentation (it will usually have a section on CGI scripts).


\subsection{Testing your CGI script}

Unfortunately, a CGI script will generally not run when you try it
from the command line, and a script that works perfectly from the
command line may fail mysteriously when run from the server.  There's
one reason why you should still test your script from the command
line: if it contains a syntax error, the python interpreter won't
execute it at all, and the HTTP server will most likely send a cryptic
error to the client.

Assuming your script has no syntax errors, yet it does not work, you
have no choice but to read the next section:


\subsection{Debugging CGI scripts}

First of all, check for trivial installation errors -- reading the
section above on installing your CGI script carefully can save you a
lot of time.  If you wonder whether you have understood the
installation procedure correctly, try installing a copy of this module
file (\code{cgi.py}) as a CGI script.  When invoked as a script, the file
will dump its environment and the contents of the form in HTML form.
Give it the right mode etc, and send it a request.  If it's installed
in the standard \code{cgi-bin} directory, it should be possible to send it a
request by entering a URL into your browser of the form:

\bcode\begin{verbatim}
http://yourhostname/cgi-bin/cgi.py?name=Joe+Blow&addr=At+Home
\end{verbatim}\ecode
%
If this gives an error of type 404, the server cannot find the script
-- perhaps you need to install it in a different directory.  If it
gives another error (e.g.  500), there's an installation problem that
you should fix before trying to go any further.  If you get a nicely
formatted listing of the environment and form content (in this
example, the fields should be listed as ``addr'' with value ``At Home''
and ``name'' with value ``Joe Blow''), the \code{cgi.py} script has been
installed correctly.  If you follow the same procedure for your own
script, you should now be able to debug it.

The next step could be to call the \code{cgi} module's test() function from
your script: replace its main code with the single statement

\bcode\begin{verbatim}
cgi.test()
\end{verbatim}\ecode
%
This should produce the same results as those gotten from installing
the \code{cgi.py} file itself.

When an ordinary Python script raises an unhandled exception
(e.g. because of a typo in a module name, a file that can't be opened,
etc.), the Python interpreter prints a nice traceback and exits.
While the Python interpreter will still do this when your CGI script
raises an exception, most likely the traceback will end up in one of
the HTTP server's log file, or be discarded altogether.

Fortunately, once you have managed to get your script to execute
*some* code, it is easy to catch exceptions and cause a traceback to
be printed.  The \code{test()} function below in this module is an example.
Here are the rules:

\begin{enumerate}
	\item Import the traceback module (before entering the
	   try-except!)
	
	\item Make sure you finish printing the headers and the blank
	   line early
	
	\item Assign \code{sys.stderr} to \code{sys.stdout}
	
	\item Wrap all remaining code in a try-except statement
	
	\item In the except clause, call \code{traceback.print_exc()}
\end{enumerate}

For example:

\bcode\begin{verbatim}
import sys
import traceback
print "Content-type: text/html"
print
sys.stderr = sys.stdout
try:
    ...your code here...
except:
    print "\n\n<PRE>"
    traceback.print_exc()
\end{verbatim}\ecode
%
Notes: The assignment to \code{sys.stderr} is needed because the traceback
prints to \code{sys.stderr}.  The \code{print "$\backslash$n$\backslash$n<PRE>"} statement is necessary to
disable the word wrapping in HTML.

If you suspect that there may be a problem in importing the traceback
module, you can use an even more robust approach (which only uses
built-in modules):

\bcode\begin{verbatim}
import sys
sys.stderr = sys.stdout
print "Content-type: text/plain"
print
...your code here...
\end{verbatim}\ecode
%
This relies on the Python interpreter to print the traceback.  The
content type of the output is set to plain text, which disables all
HTML processing.  If your script works, the raw HTML will be displayed
by your client.  If it raises an exception, most likely after the
first two lines have been printed, a traceback will be displayed.
Because no HTML interpretation is going on, the traceback will
readable.


\subsection{Common problems and solutions}

\begin{itemize}
\item Most HTTP servers buffer the output from CGI scripts until the
script is completed.  This means that it is not possible to display a
progress report on the client's display while the script is running.

\item Check the installation instructions above.

\item Check the HTTP server's log files.  (\code{tail -f logfile} in a separate
window may be useful!)

\item Always check a script for syntax errors first, by doing something
like \code{python script.py}.

\item When using any of the debugging techniques, don't forget to add
\code{import sys} to the top of the script.

\item When invoking external programs, make sure they can be found.
Usually, this means using absolute path names -- \code{\$PATH} is usually not
set to a very useful value in a CGI script.

\item When reading or writing external files, make sure they can be read
or written by every user on the system.

\item Don't try to give a CGI script a set-uid mode.  This doesn't work on
most systems, and is a security liability as well.
\end{itemize}


\section{\module{urllib} ---
         Open an arbitrary resource by URL}

\declaremodule{standard}{urllib}
\modulesynopsis{Open an arbitrary network resource by URL (requires sockets).}

\index{WWW}
\index{World-Wide Web}
\index{URL}


This module provides a high-level interface for fetching data across
the World-Wide Web.  In particular, the \function{urlopen()} function
is similar to the built-in function \function{open()}, but accepts
Universal Resource Locators (URLs) instead of filenames.  Some
restrictions apply --- it can only open URLs for reading, and no seek
operations are available.

It defines the following public functions:

\begin{funcdesc}{urlopen}{url\optional{, data}}
Open a network object denoted by a URL for reading.  If the URL does
not have a scheme identifier, or if it has \file{file:} as its scheme
identifier, this opens a local file; otherwise it opens a socket to a
server somewhere on the network.  If the connection cannot be made, or
if the server returns an error code, the \exception{IOError} exception
is raised.  If all went well, a file-like object is returned.  This
supports the following methods: \method{read()}, \method{readline()},
\method{readlines()}, \method{fileno()}, \method{close()},
\method{info()} and \method{geturl()}.

Except for the \method{info()} and \method{geturl()} methods,
these methods have the same interface as for
file objects --- see section \ref{bltin-file-objects} in this
manual.  (It is not a built-in file object, however, so it can't be
used at those few places where a true built-in file object is
required.)

The \method{info()} method returns an instance of the class
\class{mimetools.Message} containing meta-information associated
with the URL.  When the method is HTTP, these headers are those
returned by the server at the head of the retrieved HTML page
(including Content-Length and Content-Type).  When the method is FTP,
a Content-Length header will be present if (as is now usual) the
server passed back a file length in response to the FTP retrieval
request.  When the method is local-file, returned headers will include
a Date representing the file's last-modified time, a Content-Length
giving file size, and a Content-Type containing a guess at the file's
type. See also the description of the
\refmodule{mimetools}\refstmodindex{mimetools} module.

The \method{geturl()} method returns the real URL of the page.  In
some cases, the HTTP server redirects a client to another URL.  The
\function{urlopen()} function handles this transparently, but in some
cases the caller needs to know which URL the client was redirected
to.  The \method{geturl()} method can be used to get at this
redirected URL.

If the \var{url} uses the \file{http:} scheme identifier, the optional
\var{data} argument may be given to specify a \code{POST} request
(normally the request type is \code{GET}).  The \var{data} argument
must in standard \file{application/x-www-form-urlencoded} format;
see the \function{urlencode()} function below.

The \function{urlopen()} function works transparently with proxies.
In a \UNIX{} or Windows environment, set the \envvar{http_proxy},
\envvar{ftp_proxy} or \envvar{gopher_proxy} environment variables to a
URL that identifies the proxy server before starting the Python
interpreter.  For example (the \character{\%} is the command prompt):

\begin{verbatim}
% http_proxy="http://www.someproxy.com:3128"
% export http_proxy
% python
...
\end{verbatim}

In a Macintosh environment, \function{urlopen()} will retrieve proxy
information from Internet\index{Internet Config} Config.

The \function{urlopen()} function works transparently with proxies.
In a \UNIX{} or Windows environment, set the \envvar{http_proxy},
\envvar{ftp_proxy} or \envvar{gopher_proxy} environment variables to a
URL that identifies the proxy server before starting the Python
interpreter, e.g.:

\begin{verbatim}
% http_proxy="http://www.someproxy.com:3128"
% export http_proxy
% python
...
\end{verbatim}

In a Macintosh environment, \function{urlopen()} will retrieve proxy
information from Internet Config.
\end{funcdesc}

\begin{funcdesc}{urlretrieve}{url\optional{, filename\optional{, hook}}}
Copy a network object denoted by a URL to a local file, if necessary.
If the URL points to a local file, or a valid cached copy of the
object exists, the object is not copied.  Return a tuple
\code{(\var{filename}, \var{headers})} where \var{filename} is the
local file name under which the object can be found, and \var{headers}
is either \code{None} (for a local object) or whatever the
\method{info()} method of the object returned by \function{urlopen()}
returned (for a remote object, possibly cached).  Exceptions are the
same as for \function{urlopen()}.

The second argument, if present, specifies the file location to copy
to (if absent, the location will be a tempfile with a generated name).
The third argument, if present, is a hook function that will be called
once on establishment of the network connection and once after each
block read thereafter.  The hook will be passed three arguments; a
count of blocks transferred so far, a block size in bytes, and the
total size of the file.  The third argument may be \code{-1} on older
FTP servers which do not return a file size in response to a retrieval 
request.

If the \var{url} uses the \file{http:} scheme identifier, the optional
\var{data} argument may be given to specify a \code{POST} request
(normally the request type is \code{GET}).  The \var{data} argument
must in standard \file{application/x-www-form-urlencoded} format;
see the \function{urlencode()} function below.
\end{funcdesc}

\begin{funcdesc}{urlcleanup}{}
Clear the cache that may have been built up by previous calls to
\function{urlretrieve()}.
\end{funcdesc}

\begin{funcdesc}{quote}{string\optional{, safe}}
Replace special characters in \var{string} using the \samp{\%xx} escape.
Letters, digits, and the characters \character{_,.-} are never quoted.
The optional \var{safe} parameter specifies additional characters
that should not be quoted --- its default value is \code{'/'}.

Example: \code{quote('/\~{}connolly/')} yields \code{'/\%7econnolly/'}.
\end{funcdesc}

\begin{funcdesc}{quote_plus}{string\optional{, safe}}
Like \function{quote()}, but also replaces spaces by plus signs, as
required for quoting HTML form values.  Plus signs in the original
string are escaped unless they are included in \var{safe}.
\end{funcdesc}

\begin{funcdesc}{unquote}{string}
Replace \samp{\%xx} escapes by their single-character equivalent.

Example: \code{unquote('/\%7Econnolly/')} yields \code{'/\~{}connolly/'}.
\end{funcdesc}

\begin{funcdesc}{unquote_plus}{string}
Like \function{unquote()}, but also replaces plus signs by spaces, as
required for unquoting HTML form values.
\end{funcdesc}

\begin{funcdesc}{urlencode}{dict}
Convert a dictionary to a ``url-encoded'' string, suitable to pass to
\function{urlopen()} above as the optional \var{data} argument.  This
is useful to pass a dictionary of form fields to a \code{POST}
request.  The resulting string is a series of
\code{\var{key}=\var{value}} pairs separated by \character{\&}
characters, where both \var{key} and \var{value} are quoted using
\function{quote_plus()} above.
\end{funcdesc}

The public functions \function{urlopen()} and \function{urlretrieve()}
create an instance of the \class{FancyURLopener} class and use it to perform
their requested actions.  To override this functionality, programmers can
create a subclass of \class{URLopener} or \class{FancyURLopener}, then
assign that class to the \var{urllib._urlopener} variable before calling the
desired function.  For example, applications may want to specify a different
\code{user-agent} header than \class{URLopener} defines.  This can be
accomplished with the following code:

\begin{verbatim}
class AppURLopener(urllib.FancyURLopener):
    def __init__(self, *args):
        apply(urllib.FancyURLopener.__init__, (self,) + args)
        self.version = "App/1.7"

urllib._urlopener = AppURLopener()
\end{verbatim}

\begin{classdesc}{URLopener}{\optional{proxies\optional{, **x509}}}
Base class for opening and reading URLs.  Unless you need to support
opening objects using schemes other than \file{http:}, \file{ftp:},
\file{gopher:} or \file{file:}, you probably want to use
\class{FancyURLopener}.

By default, the \class{URLopener} class sends a
\code{user-agent} header of \samp{urllib/\var{VVV}}, where
\var{VVV} is the \module{urllib} version number.  Applications can
define their own \code{user-agent} header by subclassing
\class{URLopener} or \class{FancyURLopener} and setting the instance
attribute \var{version} to an appropriate string value before the
\method{open()} method is called.

Additional keyword parameters, collected in \var{x509}, are used for
authentication with the \file{https:} scheme.  The keywords
\var{key_file} and \var{cert_file} are supported; both are needed to
actually retrieve a resource at an \file{https:} URL.
\end{classdesc}

\begin{classdesc}{FancyURLopener}{...}
\class{FancyURLopener} subclasses \class{URLopener} providing default
handling for the following HTTP response codes: 301, 302 or 401.  For
301 and 302 response codes, the \code{location} header is used to
fetch the actual URL.  For 401 response codes (authentication
required), basic HTTP authentication is performed.

The parameters to the constructor are the same as those for
\class{URLopener}.
\end{classdesc}

Restrictions:

\begin{itemize}

\item
Currently, only the following protocols are supported: HTTP, (versions
0.9 and 1.0), Gopher (but not Gopher-+), FTP, and local files.
\indexii{HTTP}{protocol}
\indexii{Gopher}{protocol}
\indexii{FTP}{protocol}

\item
The caching feature of \function{urlretrieve()} has been disabled
until I find the time to hack proper processing of Expiration time
headers.

\item
There should be a function to query whether a particular URL is in
the cache.

\item
For backward compatibility, if a URL appears to point to a local file
but the file can't be opened, the URL is re-interpreted using the FTP
protocol.  This can sometimes cause confusing error messages.

\item
The \function{urlopen()} and \function{urlretrieve()} functions can
cause arbitrarily long delays while waiting for a network connection
to be set up.  This means that it is difficult to build an interactive
web client using these functions without using threads.

\item
The data returned by \function{urlopen()} or \function{urlretrieve()}
is the raw data returned by the server.  This may be binary data
(e.g. an image), plain text or (for example) HTML\index{HTML}.  The
HTTP\indexii{HTTP}{protocol} protocol provides type information in the
reply header, which can be inspected by looking at the
\code{content-type} header.  For the Gopher\indexii{Gopher}{protocol}
protocol, type information is encoded in the URL; there is currently
no easy way to extract it.  If the returned data is HTML, you can use
the module \refmodule{htmllib}\refstmodindex{htmllib} to parse it.

\item
Although the \module{urllib} module contains (undocumented) routines
to parse and unparse URL strings, the recommended interface for URL
manipulation is in module \refmodule{urlparse}\refstmodindex{urlparse}.

\end{itemize}


\subsection{URLopener Objects \label{urlopener-objs}}
\sectionauthor{Skip Montanaro}{skip@mojam.com}

\class{URLopener} and \class{FancyURLopener} objects have the
following methods.

\begin{methoddesc}{open}{fullurl\optional{, data}}
Open \var{fullurl} using the appropriate protocol.  This method sets 
up cache and proxy information, then calls the appropriate open method with
its input arguments.  If the scheme is not recognized,
\method{open_unknown()} is called.  The \var{data} argument 
has the same meaning as the \var{data} argument of \function{urlopen()}.
\end{methoddesc}

\begin{methoddesc}{open_unknown}{fullurl\optional{, data}}
Overridable interface to open unknown URL types.
\end{methoddesc}

\begin{methoddesc}{retrieve}{url\optional{, filename\optional{, reporthook}}}
Retrieves the contents of \var{url} and places it in \var{filename}.  The
return value is a tuple consisting of a local filename and either a
\class{mimetools.Message} object containing the response headers (for remote
URLs) or None (for local URLs).  The caller must then open and read the
contents of \var{filename}.  If \var{filename} is not given and the URL
refers to a local file, the input filename is returned.  If the URL is
non-local and \var{filename} is not given, the filename is the output of
\function{tempfile.mktemp()} with a suffix that matches the suffix of the last
path component of the input URL.  If \var{reporthook} is given, it must be
a function accepting three numeric parameters.  It will be called after each
chunk of data is read from the network.  \var{reporthook} is ignored for
local URLs.

If the \var{url} uses the \file{http:} scheme identifier, the optional
\var{data} argument may be given to specify a \code{POST} request
(normally the request type is \code{GET}).  The \var{data} argument
must in standard \file{application/x-www-form-urlencoded} format;
see the \function{urlencode()} function below.
\end{methoddesc}


\subsection{Examples}
\nodename{Urllib Examples}

Here is an example session that uses the \samp{GET} method to retrieve
a URL containing parameters:

\begin{verbatim}
>>> import urllib
>>> params = urllib.urlencode({'spam': 1, 'eggs': 2, 'bacon': 0})
>>> f = urllib.urlopen("http://www.musi-cal.com/cgi-bin/query?%s" % params)
>>> print f.read()
\end{verbatim}

The following example uses the \samp{POST} method instead:

\begin{verbatim}
>>> import urllib
>>> params = urllib.urlencode({'spam': 1, 'eggs': 2, 'bacon': 0})
>>> f = urllib.urlopen("http://www.musi-cal.com/cgi-bin/query", params)
>>> print f.read()
\end{verbatim}

\section{\module{httplib} ---
         HTTP protocol client}

\declaremodule{standard}{httplib}
\modulesynopsis{HTTP and HTTPS protocol client (requires sockets).}

\indexii{HTTP}{protocol}
\index{HTTP!\module{httplib} (standard module)}

This module defines classes which implement the client side of the
HTTP and HTTPS protocols.  It is normally not used directly --- the
module \refmodule{urllib}\refstmodindex{urllib} uses it to handle URLs
that use HTTP and HTTPS.

\begin{notice}
  HTTPS support is only available if the \refmodule{socket} module was
  compiled with SSL support.
\end{notice}

\begin{notice}
  The public interface for this module changed substantially in Python
  2.0.  The \class{HTTP} class is retained only for backward
  compatibility with 1.5.2.  It should not be used in new code.  Refer
  to the online docstrings for usage.
\end{notice}

The module provides the following classes:

\begin{classdesc}{HTTPConnection}{host\optional{, port}}
An \class{HTTPConnection} instance represents one transaction with an HTTP
server.  It should be instantiated passing it a host and optional port number.
If no port number is passed, the port is extracted from the host string if it
has the form \code{\var{host}:\var{port}}, else the default HTTP port (80) is
used.  For example, the following calls all create instances that connect to
the server at the same host and port:

\begin{verbatim}
>>> h1 = httplib.HTTPConnection('www.cwi.nl')
>>> h2 = httplib.HTTPConnection('www.cwi.nl:80')
>>> h3 = httplib.HTTPConnection('www.cwi.nl', 80)
\end{verbatim}
\versionadded{2.0}
\end{classdesc}

\begin{classdesc}{HTTPSConnection}{host\optional{, port, key_file, cert_file}}
A subclass of \class{HTTPConnection} that uses SSL for communication with
secure servers.  Default port is \code{443}.
\var{key_file} is
the name of a PEM formatted file that contains your private
key. \var{cert_file} is a PEM formatted certificate chain file.

\warning{This does not do any certificate verification!}

\versionadded{2.0}
\end{classdesc}

\begin{classdesc}{HTTPResponse}{sock\optional{, debuglevel=0}\optional{, strict=0}}
Class whose instances are returned upon successful connection.  Not
instantiated directly by user.
\versionadded{2.0}
\end{classdesc}

The following exceptions are raised as appropriate:

\begin{excdesc}{HTTPException}
The base class of the other exceptions in this module.  It is a
subclass of \exception{Exception}.
\versionadded{2.0}
\end{excdesc}

\begin{excdesc}{NotConnected}
A subclass of \exception{HTTPException}.
\versionadded{2.0}
\end{excdesc}

\begin{excdesc}{InvalidURL}
A subclass of \exception{HTTPException}, raised if a port is given and is
either non-numeric or empty.
\versionadded{2.3}
\end{excdesc}

\begin{excdesc}{UnknownProtocol}
A subclass of \exception{HTTPException}.
\versionadded{2.0}
\end{excdesc}

\begin{excdesc}{UnknownTransferEncoding}
A subclass of \exception{HTTPException}.
\versionadded{2.0}
\end{excdesc}

\begin{excdesc}{UnimplementedFileMode}
A subclass of \exception{HTTPException}.
\versionadded{2.0}
\end{excdesc}

\begin{excdesc}{IncompleteRead}
A subclass of \exception{HTTPException}.
\versionadded{2.0}
\end{excdesc}

\begin{excdesc}{ImproperConnectionState}
A subclass of \exception{HTTPException}.
\versionadded{2.0}
\end{excdesc}

\begin{excdesc}{CannotSendRequest}
A subclass of \exception{ImproperConnectionState}.
\versionadded{2.0}
\end{excdesc}

\begin{excdesc}{CannotSendHeader}
A subclass of \exception{ImproperConnectionState}.
\versionadded{2.0}
\end{excdesc}

\begin{excdesc}{ResponseNotReady}
A subclass of \exception{ImproperConnectionState}.
\versionadded{2.0}
\end{excdesc}

\begin{excdesc}{BadStatusLine}
A subclass of \exception{HTTPException}.  Raised if a server responds with a
HTTP status code that we don't understand.
\versionadded{2.0}
\end{excdesc}

The constants defined in this module are:

\begin{datadesc}{HTTP_PORT}
  The default port for the HTTP protocol (always \code{80}).
\end{datadesc}

\begin{datadesc}{HTTPS_PORT}
  The default port for the HTTPS protocol (always \code{443}).
\end{datadesc}

and also the following constants for integer status codes:

\begin{tableiii}{l|c|l}{constant}{Constant}{Value}{Definition}
  \lineiii{CONTINUE}{\code{100}}
    {HTTP/1.1, \ulink{RFC 2616, Section 10.1.1}
      {http://www.w3.org/Protocols/rfc2616/rfc2616-sec10.html#sec10.1.1}}
  \lineiii{SWITCHING_PROTOCOLS}{\code{101}}
    {HTTP/1.1, \ulink{RFC 2616, Section 10.1.2}
      {http://www.w3.org/Protocols/rfc2616/rfc2616-sec10.html#sec10.1.2}}
  \lineiii{PROCESSING}{\code{102}}
    {WEBDAV, \ulink{RFC 2518, Section 10.1}
      {http://www.webdav.org/specs/rfc2518.html#STATUS_102}}

  \lineiii{OK}{\code{200}}
    {HTTP/1.1, \ulink{RFC 2616, Section 10.2.1}
      {http://www.w3.org/Protocols/rfc2616/rfc2616-sec10.html#sec10.2.1}}
  \lineiii{CREATED}{\code{201}}
    {HTTP/1.1, \ulink{RFC 2616, Section 10.2.2}
      {http://www.w3.org/Protocols/rfc2616/rfc2616-sec10.html#sec10.2.2}}
  \lineiii{ACCEPTED}{\code{202}}
    {HTTP/1.1, \ulink{RFC 2616, Section 10.2.3}
      {http://www.w3.org/Protocols/rfc2616/rfc2616-sec10.html#sec10.2.3}}
  \lineiii{NON_AUTHORITATIVE_INFORMATION}{\code{203}}
    {HTTP/1.1, \ulink{RFC 2616, Section 10.2.4}
      {http://www.w3.org/Protocols/rfc2616/rfc2616-sec10.html#sec10.2.4}}
  \lineiii{NO_CONTENT}{\code{204}}
    {HTTP/1.1, \ulink{RFC 2616, Section 10.2.5}
      {http://www.w3.org/Protocols/rfc2616/rfc2616-sec10.html#sec10.2.5}}
  \lineiii{RESET_CONTENT}{\code{205}}
    {HTTP/1.1, \ulink{RFC 2616, Section 10.2.6}
      {http://www.w3.org/Protocols/rfc2616/rfc2616-sec10.html#sec10.2.6}}
  \lineiii{PARTIAL_CONTENT}{\code{206}}
    {HTTP/1.1, \ulink{RFC 2616, Section 10.2.7}
      {http://www.w3.org/Protocols/rfc2616/rfc2616-sec10.html#sec10.2.7}}
  \lineiii{MULTI_STATUS}{\code{207}}
    {WEBDAV \ulink{RFC 2518, Section 10.2}
      {http://www.webdav.org/specs/rfc2518.html#STATUS_207}}
  \lineiii{IM_USED}{\code{226}}
    {Delta encoding in HTTP, \rfc{3229}, Section 10.4.1}

  \lineiii{MULTIPLE_CHOICES}{\code{300}}
    {HTTP/1.1, \ulink{RFC 2616, Section 10.3.1}
      {http://www.w3.org/Protocols/rfc2616/rfc2616-sec10.html#sec10.3.1}}
  \lineiii{MOVED_PERMANENTLY}{\code{301}}
    {HTTP/1.1, \ulink{RFC 2616, Section 10.3.2}
      {http://www.w3.org/Protocols/rfc2616/rfc2616-sec10.html#sec10.3.2}}
  \lineiii{FOUND}{\code{302}}
    {HTTP/1.1, \ulink{RFC 2616, Section 10.3.3}
      {http://www.w3.org/Protocols/rfc2616/rfc2616-sec10.html#sec10.3.3}}
  \lineiii{SEE_OTHER}{\code{303}}
    {HTTP/1.1, \ulink{RFC 2616, Section 10.3.4}
      {http://www.w3.org/Protocols/rfc2616/rfc2616-sec10.html#sec10.3.4}}
  \lineiii{NOT_MODIFIED}{\code{304}}
    {HTTP/1.1, \ulink{RFC 2616, Section 10.3.5}
      {http://www.w3.org/Protocols/rfc2616/rfc2616-sec10.html#sec10.3.5}}
  \lineiii{USE_PROXY}{\code{305}}
    {HTTP/1.1, \ulink{RFC 2616, Section 10.3.6}
      {http://www.w3.org/Protocols/rfc2616/rfc2616-sec10.html#sec10.3.6}}
  \lineiii{TEMPORARY_REDIRECT}{\code{307}}
    {HTTP/1.1, \ulink{RFC 2616, Section 10.3.8}
      {http://www.w3.org/Protocols/rfc2616/rfc2616-sec10.html#sec10.3.8}}

  \lineiii{BAD_REQUEST}{\code{400}}
    {HTTP/1.1, \ulink{RFC 2616, Section 10.4.1}
      {http://www.w3.org/Protocols/rfc2616/rfc2616-sec10.html#sec10.4.1}}
  \lineiii{UNAUTHORIZED}{\code{401}}
    {HTTP/1.1, \ulink{RFC 2616, Section 10.4.2}
      {http://www.w3.org/Protocols/rfc2616/rfc2616-sec10.html#sec10.4.2}}
  \lineiii{PAYMENT_REQUIRED}{\code{402}}
    {HTTP/1.1, \ulink{RFC 2616, Section 10.4.3}
      {http://www.w3.org/Protocols/rfc2616/rfc2616-sec10.html#sec10.4.3}}
  \lineiii{FORBIDDEN}{\code{403}}
    {HTTP/1.1, \ulink{RFC 2616, Section 10.4.4}
      {http://www.w3.org/Protocols/rfc2616/rfc2616-sec10.html#sec10.4.4}}
  \lineiii{NOT_FOUND}{\code{404}}
    {HTTP/1.1, \ulink{RFC 2616, Section 10.4.5}
      {http://www.w3.org/Protocols/rfc2616/rfc2616-sec10.html#sec10.4.5}}
  \lineiii{METHOD_NOT_ALLOWED}{\code{405}}
    {HTTP/1.1, \ulink{RFC 2616, Section 10.4.6}
      {http://www.w3.org/Protocols/rfc2616/rfc2616-sec10.html#sec10.4.6}}
  \lineiii{NOT_ACCEPTABLE}{\code{406}}
    {HTTP/1.1, \ulink{RFC 2616, Section 10.4.7}
      {http://www.w3.org/Protocols/rfc2616/rfc2616-sec10.html#sec10.4.7}}
  \lineiii{PROXY_AUTHENTICATION_REQUIRED}
    {\code{407}}{HTTP/1.1, \ulink{RFC 2616, Section 10.4.8}
      {http://www.w3.org/Protocols/rfc2616/rfc2616-sec10.html#sec10.4.8}}
  \lineiii{REQUEST_TIMEOUT}{\code{408}}
    {HTTP/1.1, \ulink{RFC 2616, Section 10.4.9}
      {http://www.w3.org/Protocols/rfc2616/rfc2616-sec10.html#sec10.4.9}}
  \lineiii{CONFLICT}{\code{409}}
    {HTTP/1.1, \ulink{RFC 2616, Section 10.4.10}
      {http://www.w3.org/Protocols/rfc2616/rfc2616-sec10.html#sec10.4.10}}
  \lineiii{GONE}{\code{410}}
    {HTTP/1.1, \ulink{RFC 2616, Section 10.4.11}
      {http://www.w3.org/Protocols/rfc2616/rfc2616-sec10.html#sec10.4.11}}
  \lineiii{LENGTH_REQUIRED}{\code{411}}
    {HTTP/1.1, \ulink{RFC 2616, Section 10.4.12}
      {http://www.w3.org/Protocols/rfc2616/rfc2616-sec10.html#sec10.4.12}}
  \lineiii{PRECONDITION_FAILED}{\code{412}}
    {HTTP/1.1, \ulink{RFC 2616, Section 10.4.13}
      {http://www.w3.org/Protocols/rfc2616/rfc2616-sec10.html#sec10.4.13}}
  \lineiii{REQUEST_ENTITY_TOO_LARGE}
    {\code{413}}{HTTP/1.1, \ulink{RFC 2616, Section 10.4.14}
      {http://www.w3.org/Protocols/rfc2616/rfc2616-sec10.html#sec10.4.14}}
  \lineiii{REQUEST_URI_TOO_LONG}{\code{414}}
    {HTTP/1.1, \ulink{RFC 2616, Section 10.4.15}
      {http://www.w3.org/Protocols/rfc2616/rfc2616-sec10.html#sec10.4.15}}
  \lineiii{UNSUPPORTED_MEDIA_TYPE}{\code{415}}
    {HTTP/1.1, \ulink{RFC 2616, Section 10.4.16}
      {http://www.w3.org/Protocols/rfc2616/rfc2616-sec10.html#sec10.4.16}}
  \lineiii{REQUESTED_RANGE_NOT_SATISFIABLE}{\code{416}}
    {HTTP/1.1, \ulink{RFC 2616, Section 10.4.17}
      {http://www.w3.org/Protocols/rfc2616/rfc2616-sec10.html#sec10.4.17}}
  \lineiii{EXPECTATION_FAILED}{\code{417}}
    {HTTP/1.1, \ulink{RFC 2616, Section 10.4.18}
      {http://www.w3.org/Protocols/rfc2616/rfc2616-sec10.html#sec10.4.18}}
  \lineiii{UNPROCESSABLE_ENTITY}{\code{422}}
    {WEBDAV, \ulink{RFC 2518, Section 10.3}
      {http://www.webdav.org/specs/rfc2518.html#STATUS_422}}
  \lineiii{LOCKED}{\code{423}}
    {WEBDAV \ulink{RFC 2518, Section 10.4}
      {http://www.webdav.org/specs/rfc2518.html#STATUS_423}}
  \lineiii{FAILED_DEPENDENCY}{\code{424}}
    {WEBDAV, \ulink{RFC 2518, Section 10.5}
      {http://www.webdav.org/specs/rfc2518.html#STATUS_424}}
  \lineiii{UPGRADE_REQUIRED}{\code{426}}
    {HTTP Upgrade to TLS, \rfc{2817}, Section 6}

  \lineiii{INTERNAL_SERVER_ERROR}{\code{500}}
    {HTTP/1.1, \ulink{RFC 2616, Section 10.5.1}
      {http://www.w3.org/Protocols/rfc2616/rfc2616-sec10.html#sec10.5.1}}
  \lineiii{NOT_IMPLEMENTED}{\code{501}}
    {HTTP/1.1, \ulink{RFC 2616, Section 10.5.2}
      {http://www.w3.org/Protocols/rfc2616/rfc2616-sec10.html#sec10.5.2}}
  \lineiii{BAD_GATEWAY}{\code{502}}
    {HTTP/1.1 \ulink{RFC 2616, Section 10.5.3}
      {http://www.w3.org/Protocols/rfc2616/rfc2616-sec10.html#sec10.5.3}}
  \lineiii{SERVICE_UNAVAILABLE}{\code{503}}
    {HTTP/1.1, \ulink{RFC 2616, Section 10.5.4}
      {http://www.w3.org/Protocols/rfc2616/rfc2616-sec10.html#sec10.5.4}}
  \lineiii{GATEWAY_TIMEOUT}{\code{504}}
    {HTTP/1.1 \ulink{RFC 2616, Section 10.5.5}
      {http://www.w3.org/Protocols/rfc2616/rfc2616-sec10.html#sec10.5.5}}
  \lineiii{HTTP_VERSION_NOT_SUPPORTED}{\code{505}}
    {HTTP/1.1, \ulink{RFC 2616, Section 10.5.6}
      {http://www.w3.org/Protocols/rfc2616/rfc2616-sec10.html#sec10.5.6}}
  \lineiii{INSUFFICIENT_STORAGE}{\code{507}}
    {WEBDAV, \ulink{RFC 2518, Section 10.6}
      {http://www.webdav.org/specs/rfc2518.html#STATUS_507}}
  \lineiii{NOT_EXTENDED}{\code{510}}
    {An HTTP Extension Framework, \rfc{2774}, Section 7}
\end{tableiii}

\subsection{HTTPConnection Objects \label{httpconnection-objects}}

\class{HTTPConnection} instances have the following methods:

\begin{methoddesc}{request}{method, url\optional{, body\optional{, headers}}}
This will send a request to the server using the HTTP request method
\var{method} and the selector \var{url}.  If the \var{body} argument is
present, it should be a string of data to send after the headers are finished.
The header Content-Length is automatically set to the correct value.
The \var{headers} argument should be a mapping of extra HTTP headers to send
with the request.
\end{methoddesc}

\begin{methoddesc}{getresponse}{}
Should be called after a request is sent to get the response from the server.
Returns an \class{HTTPResponse} instance.
\note{Note that you must have read the whole response before you can send a new
request to the server.}
\end{methoddesc}

\begin{methoddesc}{set_debuglevel}{level}
Set the debugging level (the amount of debugging output printed).
The default debug level is \code{0}, meaning no debugging output is
printed.
\end{methoddesc}

\begin{methoddesc}{connect}{}
Connect to the server specified when the object was created.
\end{methoddesc}

\begin{methoddesc}{close}{}
Close the connection to the server.
\end{methoddesc}

As an alternative to using the \method{request()} method described above,
you can also send your request step by step, by using the four functions
below.

\begin{methoddesc}{putrequest}{request, selector\optional{,
skip\_host\optional{, skip_accept_encoding}}}
This should be the first call after the connection to the server has
been made.  It sends a line to the server consisting of the
\var{request} string, the \var{selector} string, and the HTTP version
(\code{HTTP/1.1}).  To disable automatic sending of \code{Host:} or
\code{Accept-Encoding:} headers (for example to accept additional
content encodings), specify \var{skip_host} or \var{skip_accept_encoding}
with non-False values.
\versionchanged[\var{skip_accept_encoding} argument added]{2.4}
\end{methoddesc}

\begin{methoddesc}{putheader}{header, argument\optional{, ...}}
Send an \rfc{822}-style header to the server.  It sends a line to the
server consisting of the header, a colon and a space, and the first
argument.  If more arguments are given, continuation lines are sent,
each consisting of a tab and an argument.
\end{methoddesc}

\begin{methoddesc}{endheaders}{}
Send a blank line to the server, signalling the end of the headers.
\end{methoddesc}

\begin{methoddesc}{send}{data}
Send data to the server.  This should be used directly only after the
\method{endheaders()} method has been called and before
\method{getresponse()} is called.
\end{methoddesc}

\subsection{HTTPResponse Objects \label{httpresponse-objects}}

\class{HTTPResponse} instances have the following methods and attributes:

\begin{methoddesc}{read}{\optional{amt}}
Reads and returns the response body, or up to the next \var{amt} bytes.
\end{methoddesc}

\begin{methoddesc}{getheader}{name\optional{, default}}
Get the contents of the header \var{name}, or \var{default} if there is no
matching header.
\end{methoddesc}

\begin{methoddesc}{getheaders}{}
Return a list of (header, value) tuples. \versionadded{2.4}
\end{methoddesc}

\begin{datadesc}{msg}
  A \class{mimetools.Message} instance containing the response headers.
\end{datadesc}

\begin{datadesc}{version}
  HTTP protocol version used by server.  10 for HTTP/1.0, 11 for HTTP/1.1.
\end{datadesc}

\begin{datadesc}{status}
  Status code returned by server.
\end{datadesc}

\begin{datadesc}{reason}
  Reason phrase returned by server.
\end{datadesc}


\subsection{Examples \label{httplib-examples}}

Here is an example session that uses the \samp{GET} method:

\begin{verbatim}
>>> import httplib
>>> conn = httplib.HTTPConnection("www.python.org")
>>> conn.request("GET", "/index.html")
>>> r1 = conn.getresponse()
>>> print r1.status, r1.reason
200 OK
>>> data1 = r1.read()
>>> conn.request("GET", "/parrot.spam")
>>> r2 = conn.getresponse()
>>> print r2.status, r2.reason
404 Not Found
>>> data2 = r2.read()
>>> conn.close()
\end{verbatim}

Here is an example session that shows how to \samp{POST} requests:

\begin{verbatim}
>>> import httplib, urllib
>>> params = urllib.urlencode({'spam': 1, 'eggs': 2, 'bacon': 0})
>>> headers = {"Content-type": "application/x-www-form-urlencoded",
...            "Accept": "text/plain"}
>>> conn = httplib.HTTPConnection("musi-cal.mojam.com:80")
>>> conn.request("POST", "/cgi-bin/query", params, headers)
>>> response = conn.getresponse()
>>> print response.status, response.reason
200 OK
>>> data = response.read()
>>> conn.close()
\end{verbatim}

\section{Standard Module \module{ftplib}}
\label{module-ftplib}
\stmodindex{ftplib}
\indexii{FTP}{protocol}


This module defines the class \class{FTP} and a few related items.
The \class{FTP} class implements the client side of the FTP protocol.
You can use this to write Python programs that perform a variety of
automated FTP jobs, such as mirroring other ftp servers.  It is also
used by the module \module{urllib} to handle URLs that use FTP.  For
more information on FTP (File Transfer Protocol), see Internet
\rfc{959}.

Here's a sample session using the \module{ftplib} module:

\begin{verbatim}
>>> from ftplib import FTP
>>> ftp = FTP('ftp.cwi.nl')   # connect to host, default port
>>> ftp.login()               # user anonymous, passwd user@hostname
>>> ftp.retrlines('LIST')     # list directory contents
total 24418
drwxrwsr-x   5 ftp-usr  pdmaint     1536 Mar 20 09:48 .
dr-xr-srwt 105 ftp-usr  pdmaint     1536 Mar 21 14:32 ..
-rw-r--r--   1 ftp-usr  pdmaint     5305 Mar 20 09:48 INDEX
 .
 .
 .
>>> ftp.quit()
\end{verbatim}

The module defines the following items:

\begin{classdesc}{FTP}{\optional{host\optional{, user\optional{,
                       passwd\optional{, acct}}}}}
Return a new instance of the \class{FTP} class.  When
\var{host} is given, the method call \code{connect(\var{host})} is
made.  When \var{user} is given, additionally the method call
\code{login(\var{user}, \var{passwd}, \var{acct})} is made (where
\var{passwd} and \var{acct} default to the empty string when not given).
\end{classdesc}

\begin{datadesc}{all_errors}
The set of all exceptions (as a tuple) that methods of \class{FTP}
instances may raise as a result of problems with the FTP connection
(as opposed to programming errors made by the caller).  This set
includes the four exceptions listed below as well as
\exception{socket.error} and \exception{IOError}.
\end{datadesc}

\begin{excdesc}{error_reply}
Exception raised when an unexpected reply is received from the server.
\end{excdesc}

\begin{excdesc}{error_temp}
Exception raised when an error code in the range 400--499 is received.
\end{excdesc}

\begin{excdesc}{error_perm}
Exception raised when an error code in the range 500--599 is received.
\end{excdesc}

\begin{excdesc}{error_proto}
Exception raised when a reply is received from the server that does
not begin with a digit in the range 1--5.
\end{excdesc}


\subsection{FTP Objects}
\label{ftp-objects}

\class{FTP} instances have the following methods:

\begin{methoddesc}{set_debuglevel}{level}
Set the instance's debugging level.  This controls the amount of
debugging output printed.  The default, \code{0}, produces no
debugging output.  A value of \code{1} produces a moderate amount of
debugging output, generally a single line per request.  A value of
\code{2} or higher produces the maximum amount of debugging output,
logging each line sent and received on the control connection.
\end{methoddesc}

\begin{methoddesc}{connect}{host\optional{, port}}
Connect to the given host and port.  The default port number is \code{21}, as
specified by the FTP protocol specification.  It is rarely needed to
specify a different port number.  This function should be called only
once for each instance; it should not be called at all if a host was
given when the instance was created.  All other methods can only be
used after a connection has been made.
\end{methoddesc}

\begin{methoddesc}{getwelcome}{}
Return the welcome message sent by the server in reply to the initial
connection.  (This message sometimes contains disclaimers or help
information that may be relevant to the user.)
\end{methoddesc}

\begin{methoddesc}{login}{\optional{user\optional{, passwd\optional{, acct}}}}
Log in as the given \var{user}.  The \var{passwd} and \var{acct}
parameters are optional and default to the empty string.  If no
\var{user} is specified, it defaults to \code{'anonymous'}.  If
\var{user} is \code{anonymous}, the default \var{passwd} is
\samp{\var{realuser}@\var{host}} where \var{realuser} is the real user
name (glanced from the \envvar{LOGNAME} or \envvar{USER} environment
variable) and \var{host} is the hostname as returned by
\function{socket.gethostname()}.  This function should be called only
once for each instance, after a connection has been established; it
should not be called at all if a host and user were given when the
instance was created.  Most FTP commands are only allowed after the
client has logged in.
\end{methoddesc}

\begin{methoddesc}{abort}{}
Abort a file transfer that is in progress.  Using this does not always
work, but it's worth a try.
\end{methoddesc}

\begin{methoddesc}{sendcmd}{command}
Send a simple command string to the server and return the response
string.
\end{methoddesc}

\begin{methoddesc}{voidcmd}{command}
Send a simple command string to the server and handle the response.
Return nothing if a response code in the range 200--299 is received.
Raise an exception otherwise.
\end{methoddesc}

\begin{methoddesc}{retrbinary}{command, callback\optional{, maxblocksize}}
Retrieve a file in binary transfer mode.  \var{command} should be an
appropriate \samp{RETR} command, i.e.\ \code{'RETR \var{filename}'}.
The \var{callback} function is called for each block of data received,
with a single string argument giving the data block.
The optional \var{maxblocksize} argument specifies the maximum chunk size to
read on the low-level socket object created to do the actual transfer
(which will also be the largest size of the data blocks passed to
\var{callback}).  A reasonable default is chosen.
\end{methoddesc}

\begin{methoddesc}{retrlines}{command\optional{, callback}}
Retrieve a file or directory listing in \ASCII{} transfer mode.
\var{command} should be an appropriate \samp{RETR} command (see
\method{retrbinary()} or a \samp{LIST} command (usually just the string
\code{'LIST'}).  The \var{callback} function is called for each line,
with the trailing CRLF stripped.  The default \var{callback} prints
the line to \code{sys.stdout}.
\end{methoddesc}

\begin{methoddesc}{storbinary}{command, file, blocksize}
Store a file in binary transfer mode.  \var{command} should be an
appropriate \samp{STOR} command, i.e.\ \code{"STOR \var{filename}"}.
\var{file} is an open file object which is read until \EOF{} using its
\method{read()} method in blocks of size \var{blocksize} to provide the
data to be stored.
\end{methoddesc}

\begin{methoddesc}{storlines}{command, file}
Store a file in \ASCII{} transfer mode.  \var{command} should be an
appropriate \samp{STOR} command (see \method{storbinary()}).  Lines are
read until \EOF{} from the open file object \var{file} using its
\method{readline()} method to privide the data to be stored.
\end{methoddesc}

\begin{methoddesc}{transfercmd}{cmd}
Initiate a transfer over the data connection.  If the transfer is
active, send a \samp{PORT} command and the transfer command specified
by \var{cmd}, and accept the connection.  If the server is passive,
send a \samp{PASV} command, connect to it, and start the transfer
command.  Either way, return the socket for the connection.
\end{methoddesc}

\begin{methoddesc}{ntransfercmd}{cmd}
Like \method{transfercmd()}, but returns a tuple of the data
connection and the expected size of the data.  If the expected size
could not be computed, \code{None} will be returned as the expected
size.
\end{methoddesc}

\begin{methoddesc}{nlst}{argument\optional{, \ldots}}
Return a list of files as returned by the \samp{NLST} command.  The
optional \var{argument} is a directory to list (default is the current
server directory).  Multiple arguments can be used to pass
non-standard options to the \samp{NLST} command.
\end{methoddesc}

\begin{methoddesc}{dir}{argument\optional{, \ldots}}
Return a directory listing as returned by the \samp{LIST} command, as
a list of lines.  The optional \var{argument} is a directory to list
(default is the current server directory).  Multiple arguments can be
used to pass non-standard options to the \samp{LIST} command.  If the
last argument is a function, it is used as a \var{callback} function
as for \method{retrlines()}.
\end{methoddesc}

\begin{methoddesc}{rename}{fromname, toname}
Rename file \var{fromname} on the server to \var{toname}.
\end{methoddesc}

\begin{methoddesc}{delete}{filename}
Remove the file named \var{filename} from the server.  If successful,
returns the text of the response, otherwise raises
\exception{error_perm} on permission errors or \exception{error_reply} 
on other errors.
\end{methoddesc}

\begin{methoddesc}{cwd}{pathname}
Set the current directory on the server.
\end{methoddesc}

\begin{methoddesc}{mkd}{pathname}
Create a new directory on the server.
\end{methoddesc}

\begin{methoddesc}{pwd}{}
Return the pathname of the current directory on the server.
\end{methoddesc}

\begin{methoddesc}{rmd}{dirname}
Remove the directory named \var{dirname} on the server.
\end{methoddesc}

\begin{methoddesc}{size}{filename}
Request the size of the file named \var{filename} on the server.  On
success, the size of the file is returned as an integer, otherwise
\code{None} is returned.  Note that the \samp{SIZE} command is not 
standardized, but is supported by many common server implementations.
\end{methoddesc}

\begin{methoddesc}{quit}{}
Send a \samp{QUIT} command to the server and close the connection.
This is the ``polite'' way to close a connection, but it may raise an
exception of the server reponds with an error to the \samp{QUIT}
command.
\end{methoddesc}

\begin{methoddesc}{close}{}
Close the connection unilaterally.  This should not be applied to an
already closed connection (e.g.\ after a successful call to
\method{quit()}.
\end{methoddesc}

\section{\module{gopherlib} ---
         Gopher protocol client}

\declaremodule{standard}{gopherlib}
\modulesynopsis{Gopher protocol client (requires sockets).}

\deprecated{2.5}{The \code{gopher} protocol is not in active use
                 anymore.}

\indexii{Gopher}{protocol}

This module provides a minimal implementation of client side of the
Gopher protocol.  It is used by the module \refmodule{urllib} to
handle URLs that use the Gopher protocol.

The module defines the following functions:

\begin{funcdesc}{send_selector}{selector, host\optional{, port}}
Send a \var{selector} string to the gopher server at \var{host} and
\var{port} (default \code{70}).  Returns an open file object from
which the returned document can be read.
\end{funcdesc}

\begin{funcdesc}{send_query}{selector, query, host\optional{, port}}
Send a \var{selector} string and a \var{query} string to a gopher
server at \var{host} and \var{port} (default \code{70}).  Returns an
open file object from which the returned document can be read.
\end{funcdesc}

Note that the data returned by the Gopher server can be of any type,
depending on the first character of the selector string.  If the data
is text (first character of the selector is \samp{0}), lines are
terminated by CRLF, and the data is terminated by a line consisting of
a single \samp{.}, and a leading \samp{.} should be stripped from
lines that begin with \samp{..}.  Directory listings (first character
of the selector is \samp{1}) are transferred using the same protocol.

\section{\module{poplib} ---
         POP3 protocol client}

\declaremodule{standard}{poplib}
\modulesynopsis{POP3 protocol client (requires sockets).}

%By Andrew T. Csillag
%Even though I put it into LaTeX, I cannot really claim that I wrote
%it since I just stole most of it from the poplib.py source code and
%the imaplib ``chapter''.
%Revised by ESR, January 2000

\indexii{POP3}{protocol}

This module defines a class, \class{POP3}, which encapsulates a
connection to an POP3 server and implements the protocol as defined in
\rfc{1725}.  The \class{POP3} class supports both the minimal and
optional command sets.

Note that POP3, though widely supported, is obsolescent.  The
implementation quality of POP3 servers varies widely, and too many are
quite poor. If your mailserver supports IMAP, you would be better off
using the \code{\refmodule{imaplib}.\class{IMAP4}} class, as IMAP
servers tend to be better implemented.

A single class is provided by the \module{poplib} module:

\begin{classdesc}{POP3}{host\optional{, port}}
This class implements the actual POP3 protocol.  The connection is
created when the instance is initialized.
If \var{port} is omitted, the standard POP3 port (110) is used.
\end{classdesc}

One exception is defined as an attribute of the \module{poplib} module:

\begin{excdesc}{error_proto}
Exception raised on any errors.  The reason for the exception is
passed to the constructor as a string.
\end{excdesc}

\begin{seealso}
  \seemodule{imaplib}{The standard Python IMAP module.}
  \seetitle[http://www.tuxedo.org/\~{}esr/fetchmail/fetchmail-FAQ.html]
        {Frequently Asked Questions About Fetchmail}
        {The FAQ for the \program{fetchmail} POP/IMAP client collects
         information on POP3 server variations and RFC noncompliance
         that may be useful if you need to write an application based
         on the POP protocol.}
\end{seealso}


\subsection{POP3 Objects \label{pop3-objects}}

All POP3 commands are represented by methods of the same name,
in lower-case; most return the response text sent by the server.

An \class{POP3} instance has the following methods:


\begin{methoddesc}{set_debuglevel}{level}
Set the instance's debugging level.  This controls the amount of
debugging output printed.  The default, \code{0}, produces no
debugging output.  A value of \code{1} produces a moderate amount of
debugging output, generally a single line per request.  A value of
\code{2} or higher produces the maximum amount of debugging output,
logging each line sent and received on the control connection.
\end{methoddesc}

\begin{methoddesc}{getwelcome}{}
Returns the greeting string sent by the POP3 server.
\end{methoddesc}

\begin{methoddesc}{user}{username}
Send user command, response should indicate that a password is required.
\end{methoddesc}

\begin{methoddesc}{pass_}{password}
Send password, response includes message count and mailbox size.
Note: the mailbox on the server is locked until \method{quit()} is
called.
\end{methoddesc}

\begin{methoddesc}{apop}{user, secret}
Use the more secure APOP authentication to log into the POP3 server.
\end{methoddesc}

\begin{methoddesc}{rpop}{user}
Use RPOP authentication (similar to UNIX r-commands) to log into POP3 server.
\end{methoddesc}

\begin{methoddesc}{stat}{}
Get mailbox status.  The result is a tuple of 2 integers:
\code{(\var{message count}, \var{mailbox size})}.
\end{methoddesc}

\begin{methoddesc}{list}{\optional{which}}
Request message list, result is in the form
\code{(\var{response}, ['mesg_num octets', ...])}.  If \var{which} is
set, it is the message to list.
\end{methoddesc}

\begin{methoddesc}{retr}{which}
Retrieve whole message number \var{which}, and set its seen flag.
Result is in form  \code{(\var{response}, ['line', ...], \var{octets})}.
\end{methoddesc}

\begin{methoddesc}{dele}{which}
Flag message number \var{which} for deletion.  On most servers
deletions are not actually performed until QUIT (the major exception is
Eudora QPOP, which deliberately violates the RFCs by doing pending
deletes on any disconnect).
\end{methoddesc}

\begin{methoddesc}{rset}{}
Remove any deletion marks for the mailbox.
\end{methoddesc}

\begin{methoddesc}{noop}{}
Do nothing.  Might be used as a keep-alive.
\end{methoddesc}

\begin{methoddesc}{quit}{}
Signoff:  commit changes, unlock mailbox, drop connection.
\end{methoddesc}

\begin{methoddesc}{top}{which, howmuch}
Retrieves the message header plus \var{howmuch} lines of the message
after the header of message number \var{which}. Result is in form
\code{(\var{response}, ['line', ...], \var{octets})}.

The POP3 TOP command this method uses, unlike the RETR command,
doesn't set the message's seen flag; unfortunately, TOP is poorly
specified in the RFCs and is frequently broken in off-brand servers.
Test this method by hand against the POP3 servers you will use before
trusting it.
\end{methoddesc}

\begin{methoddesc}{uidl}{\optional{which}}
Return message digest (unique id) list.
If \var{which} is specified, result contains the unique id for that
message in the form \code{'\var{response}\ \var{mesgnum}\ \var{uid}},
otherwise result is list \code{(\var{response}, ['mesgnum uid', ...],
\var{octets})}.
\end{methoddesc}


\subsection{POP3 Example \label{pop3-example}}

Here is a minimal example (without error checking) that opens a
mailbox and retrieves and prints all messages:

\begin{verbatim}
import getpass, poplib

M = poplib.POP3('localhost')
M.user(getpass.getuser())
M.pass_(getpass.getpass())
numMessages = len(M.list()[1])
for i in range(numMessages):
    for j in M.retr(i+1)[1]:
        print j
\end{verbatim}

At the end of the module, there is a test section that contains a more
extensive example of usage.

\section{\module{imaplib} ---
         IMAP4 protocol client}

\declaremodule{standard}{imaplib}
\modulesynopsis{IMAP4 protocol client (requires sockets).}
\moduleauthor{Piers Lauder}{piers@communitysolutions.com.au}
\sectionauthor{Piers Lauder}{piers@communitysolutions.com.au}

% Based on HTML documentation by Piers Lauder <piers@communitysolutions.com.au>;
% converted by Fred L. Drake, Jr. <fdrake@acm.org>.
% Revised by ESR, January 2000.
% Changes for IMAP4_SSL by Tino Lange <Tino.Lange@isg.de>, March 2002 
% Changes for IMAP4_stream by Piers Lauder <piers@communitysolutions.com.au>, November 2002 

\indexii{IMAP4}{protocol}
\indexii{IMAP4_SSL}{protocol}
\indexii{IMAP4_stream}{protocol}

This module defines three classes, \class{IMAP4}, \class{IMAP4_SSL} and \class{IMAP4_stream}, which encapsulate a
connection to an IMAP4 server and implement a large subset of the
IMAP4rev1 client protocol as defined in \rfc{2060}. It is backward
compatible with IMAP4 (\rfc{1730}) servers, but note that the
\samp{STATUS} command is not supported in IMAP4.

Three classes are provided by the \module{imaplib} module, \class{IMAP4} is the base class:

\begin{classdesc}{IMAP4}{\optional{host\optional{, port}}}
This class implements the actual IMAP4 protocol.  The connection is
created and protocol version (IMAP4 or IMAP4rev1) is determined when
the instance is initialized.
If \var{host} is not specified, \code{''} (the local host) is used.
If \var{port} is omitted, the standard IMAP4 port (143) is used.
\end{classdesc}

Three exceptions are defined as attributes of the \class{IMAP4} class:

\begin{excdesc}{IMAP4.error}
Exception raised on any errors.  The reason for the exception is
passed to the constructor as a string.
\end{excdesc}

\begin{excdesc}{IMAP4.abort}
IMAP4 server errors cause this exception to be raised.  This is a
sub-class of \exception{IMAP4.error}.  Note that closing the instance
and instantiating a new one will usually allow recovery from this
exception.
\end{excdesc}

\begin{excdesc}{IMAP4.readonly}
This exception is raised when a writable mailbox has its status changed by the server.  This is a
sub-class of \exception{IMAP4.error}.  Some other client now has write permission,
and the mailbox will need to be re-opened to re-obtain write permission.
\end{excdesc}

There's also a subclass for secure connections:

\begin{classdesc}{IMAP4_SSL}{\optional{host\optional{, port\optional{, keyfile\optional{, certfile}}}}}
This is a subclass derived from \class{IMAP4} that connects over an SSL encrypted socket 
(to use this class you need a socket module that was compiled with SSL support).
If \var{host} is not specified, \code{''} (the local host) is used.
If \var{port} is omitted, the standard IMAP4-over-SSL port (993) is used.
\var{keyfile} and \var{certfile} are also optional - they can contain a PEM formatted
private key and certificate chain file for the SSL connection. 
\end{classdesc}

The second subclass allows for connections created by a child process:

\begin{classdesc}{IMAP4_stream}{command}
This is a subclass derived from \class{IMAP4} that connects
to the \code{stdin/stdout} file descriptors created by passing \var{command} to \code{os.popen2()}.
\versionadded{2.3}
\end{classdesc}

The following utility functions are defined:

\begin{funcdesc}{Internaldate2tuple}{datestr}
  Converts an IMAP4 INTERNALDATE string to Coordinated Universal
  Time. Returns a \refmodule{time} module tuple.
\end{funcdesc}

\begin{funcdesc}{Int2AP}{num}
  Converts an integer into a string representation using characters
  from the set [\code{A} .. \code{P}].
\end{funcdesc}

\begin{funcdesc}{ParseFlags}{flagstr}
  Converts an IMAP4 \samp{FLAGS} response to a tuple of individual
  flags.
\end{funcdesc}

\begin{funcdesc}{Time2Internaldate}{date_time}
  Converts a \refmodule{time} module tuple to an IMAP4
  \samp{INTERNALDATE} representation.  Returns a string in the form:
  \code{"DD-Mmm-YYYY HH:MM:SS +HHMM"} (including double-quotes).
\end{funcdesc}


Note that IMAP4 message numbers change as the mailbox changes; in
particular, after an \samp{EXPUNGE} command performs deletions the
remaining messages are renumbered. So it is highly advisable to use
UIDs instead, with the UID command.

At the end of the module, there is a test section that contains a more
extensive example of usage.

\begin{seealso}
  \seetext{Documents describing the protocol, and sources and binaries 
           for servers implementing it, can all be found at the
           University of Washington's \emph{IMAP Information Center}
           (\url{http://www.cac.washington.edu/imap/}).}
\end{seealso}


\subsection{IMAP4 Objects \label{imap4-objects}}

All IMAP4rev1 commands are represented by methods of the same name,
either upper-case or lower-case.

All arguments to commands are converted to strings, except for
\samp{AUTHENTICATE}, and the last argument to \samp{APPEND} which is
passed as an IMAP4 literal.  If necessary (the string contains IMAP4
protocol-sensitive characters and isn't enclosed with either
parentheses or double quotes) each string is quoted. However, the
\var{password} argument to the \samp{LOGIN} command is always quoted.
If you want to avoid having an argument string quoted
(eg: the \var{flags} argument to \samp{STORE}) then enclose the string in
parentheses (eg: \code{r'(\e Deleted)'}).

Each command returns a tuple: \code{(\var{type}, [\var{data},
...])} where \var{type} is usually \code{'OK'} or \code{'NO'},
and \var{data} is either the text from the command response, or
mandated results from the command. Each \var{data}
is either a string, or a tuple. If a tuple, then the first part
is the header of the response, and the second part contains
the data (ie: 'literal' value).

An \class{IMAP4} instance has the following methods:


\begin{methoddesc}{append}{mailbox, flags, date_time, message}
  Append \var{message} to named mailbox. 
\end{methoddesc}

\begin{methoddesc}{authenticate}{mechanism, authobject}
  Authenticate command --- requires response processing.

  \var{mechanism} specifies which authentication mechanism is to
  be used - it should appear in the instance variable \code{capabilities} in the
  form \code{AUTH=mechanism}.

  \var{authobject} must be a callable object:

\begin{verbatim}
data = authobject(response)
\end{verbatim}

  It will be called to process server continuation responses.
  It should return \code{data} that will be encoded and sent to server.
  It should return \code{None} if the client abort response \samp{*} should
  be sent instead.
\end{methoddesc}

\begin{methoddesc}{check}{}
  Checkpoint mailbox on server. 
\end{methoddesc}

\begin{methoddesc}{close}{}
  Close currently selected mailbox. Deleted messages are removed from
  writable mailbox. This is the recommended command before
  \samp{LOGOUT}.
\end{methoddesc}

\begin{methoddesc}{copy}{message_set, new_mailbox}
  Copy \var{message_set} messages onto end of \var{new_mailbox}. 
\end{methoddesc}

\begin{methoddesc}{create}{mailbox}
  Create new mailbox named \var{mailbox}.
\end{methoddesc}

\begin{methoddesc}{delete}{mailbox}
  Delete old mailbox named \var{mailbox}.
\end{methoddesc}

\begin{methoddesc}{deleteacl}{mailbox, who}
  Delete the ACLs (remove any rights) set for who on mailbox.
\versionadded{2.4}
\end{methoddesc}

\begin{methoddesc}{expunge}{}
  Permanently remove deleted items from selected mailbox. Generates an
  \samp{EXPUNGE} response for each deleted message. Returned data
  contains a list of \samp{EXPUNGE} message numbers in order
  received.
\end{methoddesc}

\begin{methoddesc}{fetch}{message_set, message_parts}
  Fetch (parts of) messages.  \var{message_parts} should be
  a string of message part names enclosed within parentheses,
  eg: \samp{"(UID BODY[TEXT])"}.  Returned data are tuples
  of message part envelope and data.
\end{methoddesc}

\begin{methoddesc}{getacl}{mailbox}
  Get the \samp{ACL}s for \var{mailbox}.
  The method is non-standard, but is supported by the \samp{Cyrus} server.
\end{methoddesc}

\begin{methoddesc}{getquota}{root}
  Get the \samp{quota} \var{root}'s resource usage and limits.
  This method is part of the IMAP4 QUOTA extension defined in rfc2087.
\versionadded{2.3}
\end{methoddesc}

\begin{methoddesc}{getquotaroot}{mailbox}
  Get the list of \samp{quota} \samp{roots} for the named \var{mailbox}.
  This method is part of the IMAP4 QUOTA extension defined in rfc2087.
\versionadded{2.3}
\end{methoddesc}

\begin{methoddesc}{list}{\optional{directory\optional{, pattern}}}
  List mailbox names in \var{directory} matching
  \var{pattern}.  \var{directory} defaults to the top-level mail
  folder, and \var{pattern} defaults to match anything.  Returned data
  contains a list of \samp{LIST} responses.
\end{methoddesc}

\begin{methoddesc}{login}{user, password}
  Identify the client using a plaintext password.
  The \var{password} will be quoted.
\end{methoddesc}

\begin{methoddesc}{login_cram_md5}{user, password}
  Force use of \samp{CRAM-MD5} authentication when identifying the client to protect the password.
  Will only work if the server \samp{CAPABILITY} response includes the phrase \samp{AUTH=CRAM-MD5}.
\versionadded{2.3}
\end{methoddesc}

\begin{methoddesc}{logout}{}
  Shutdown connection to server. Returns server \samp{BYE} response.
\end{methoddesc}

\begin{methoddesc}{lsub}{\optional{directory\optional{, pattern}}}
  List subscribed mailbox names in directory matching pattern.
  \var{directory} defaults to the top level directory and
  \var{pattern} defaults to match any mailbox.
  Returned data are tuples of message part envelope and data.
\end{methoddesc}

\begin{methoddes}{myrights}{mailbox}
  Show my ACLs for a mailbox (i.e. the rights that I have on mailbox).
\versionadded{2.4}
\end{methoddesc}

\begin{methoddesc}{namespace}{}
  Returns IMAP namespaces as defined in RFC2342.
\versionadded{2.3}
\end{methoddesc}

\begin{methoddesc}{noop}{}
  Send \samp{NOOP} to server.
\end{methoddesc}

\begin{methoddesc}{open}{host, port}
  Opens socket to \var{port} at \var{host}.
  The connection objects established by this method
  will be used in the \code{read}, \code{readline}, \code{send}, and \code{shutdown} methods.
  You may override this method.
\end{methoddesc}

\begin{methoddesc}{partial}{message_num, message_part, start, length}
  Fetch truncated part of a message.
  Returned data is a tuple of message part envelope and data.
\end{methoddesc}

\begin{methoddesc}{proxyauth}{user}
  Assume authentication as \var{user}.
  Allows an authorised administrator to proxy into any user's mailbox.
\versionadded{2.3}
\end{methoddesc}

\begin{methoddesc}{read}{size}
  Reads \var{size} bytes from the remote server.
  You may override this method.
\end{methoddesc}

\begin{methoddesc}{readline}{}
  Reads one line from the remote server.
  You may override this method.
\end{methoddesc}

\begin{methoddesc}{recent}{}
  Prompt server for an update. Returned data is \code{None} if no new
  messages, else value of \samp{RECENT} response.
\end{methoddesc}

\begin{methoddesc}{rename}{oldmailbox, newmailbox}
  Rename mailbox named \var{oldmailbox} to \var{newmailbox}.
\end{methoddesc}

\begin{methoddesc}{response}{code}
  Return data for response \var{code} if received, or
  \code{None}. Returns the given code, instead of the usual type.
\end{methoddesc}

\begin{methoddesc}{search}{charset, criterion\optional{, ...}}
  Search mailbox for matching messages.  Returned data contains a space
  separated list of matching message numbers.  \var{charset} may be
  \code{None}, in which case no \samp{CHARSET} will be specified in the
  request to the server.  The IMAP protocol requires that at least one
  criterion be specified; an exception will be raised when the server
  returns an error.

  Example:

\begin{verbatim}
# M is a connected IMAP4 instance...
msgnums = M.search(None, 'FROM', '"LDJ"')

# or:
msgnums = M.search(None, '(FROM "LDJ")')
\end{verbatim}
\end{methoddesc}

\begin{methoddesc}{select}{\optional{mailbox\optional{, readonly}}}
  Select a mailbox. Returned data is the count of messages in
  \var{mailbox} (\samp{EXISTS} response).  The default \var{mailbox}
  is \code{'INBOX'}.  If the \var{readonly} flag is set, modifications
  to the mailbox are not allowed.
\end{methoddesc}

\begin{methoddesc}{send}{data}
  Sends \code{data} to the remote server.
  You may override this method.
\end{methoddesc}

\begin{methoddesc}{setacl}{mailbox, who, what}
  Set an \samp{ACL} for \var{mailbox}.
  The method is non-standard, but is supported by the \samp{Cyrus} server.
\end{methoddesc}

\begin{methoddesc}{setquota}{root, limits}
  Set the \samp{quota} \var{root}'s resource \var{limits}.
  This method is part of the IMAP4 QUOTA extension defined in rfc2087.
\versionadded{2.3}
\end{methoddesc}

\begin{methoddesc}{shutdown}{}
  Close connection established in \code{open}.
  You may override this method.
\end{methoddesc}

\begin{methoddesc}{socket}{}
  Returns socket instance used to connect to server.
\end{methoddesc}

\begin{methoddesc}{sort}{sort_criteria, charset, search_criterion\optional{, ...}}
  The \code{sort} command is a variant of \code{search} with sorting semantics for
  the results.  Returned data contains a space
  separated list of matching message numbers.

  Sort has two arguments before the \var{search_criterion}
  argument(s); a parenthesized list of \var{sort_criteria}, and the searching \var{charset}.
  Note that unlike \code{search}, the searching \var{charset} argument is mandatory.
  There is also a \code{uid sort} command which corresponds to \code{sort} the way
  that \code{uid search} corresponds to \code{search}.
  The \code{sort} command first searches the mailbox for messages that
  match the given searching criteria using the charset argument for
  the interpretation of strings in the searching criteria.  It then
  returns the numbers of matching messages.

  This is an \samp{IMAP4rev1} extension command.
\end{methoddesc}

\begin{methoddesc}{status}{mailbox, names}
  Request named status conditions for \var{mailbox}. 
\end{methoddesc}

\begin{methoddesc}{store}{message_set, command, flag_list}
  Alters flag dispositions for messages in mailbox.
\end{methoddesc}

\begin{methoddesc}{subscribe}{mailbox}
  Subscribe to new mailbox.
\end{methoddesc}

\begin{methoddesc}{thread}{threading_algorithm, charset, search_criterion\optional{, ...}}
  The \code{thread} command is a variant of \code{search} with threading semantics for
  the results.  Returned data contains a space
  separated list of thread members.

  Thread members consist of zero or more messages numbers, delimited by spaces,
  indicating successive parent and child.

  Thread has two arguments before the \var{search_criterion}
  argument(s); a \var{threading_algorithm}, and the searching \var{charset}.
  Note that unlike \code{search}, the searching \var{charset} argument is mandatory.
  There is also a \code{uid thread} command which corresponds to \code{thread} the way
  that \code{uid search} corresponds to \code{search}.
  The \code{thread} command first searches the mailbox for messages that
  match the given searching criteria using the charset argument for
  the interpretation of strings in the searching criteria. It thren
  returns the matching messages threaded according to the specified
  threading algorithm.

  This is an \samp{IMAP4rev1} extension command. \versionadded{2.4}
\end{methoddesc}

\begin{methoddesc}{uid}{command, arg\optional{, ...}}
  Execute command args with messages identified by UID, rather than
  message number.  Returns response appropriate to command.  At least
  one argument must be supplied; if none are provided, the server will
  return an error and an exception will be raised.
\end{methoddesc}

\begin{methoddesc}{unsubscribe}{mailbox}
  Unsubscribe from old mailbox.
\end{methoddesc}

\begin{methoddesc}{xatom}{name\optional{, arg\optional{, ...}}}
  Allow simple extension commands notified by server in
  \samp{CAPABILITY} response.
\end{methoddesc}


Instances of \class{IMAP4_SSL} have just one additional method:

\begin{methoddesc}{ssl}{}
  Returns SSLObject instance used for the secure connection with the server.
\end{methoddesc}


The following attributes are defined on instances of \class{IMAP4}:


\begin{memberdesc}{PROTOCOL_VERSION}
The most recent supported protocol in the
\samp{CAPABILITY} response from the server.
\end{memberdesc}

\begin{memberdesc}{debug}
Integer value to control debugging output.  The initialize value is
taken from the module variable \code{Debug}.  Values greater than
three trace each command.
\end{memberdesc}


\subsection{IMAP4 Example \label{imap4-example}}

Here is a minimal example (without error checking) that opens a
mailbox and retrieves and prints all messages:

\begin{verbatim}
import getpass, imaplib

M = imaplib.IMAP4()
M.login(getpass.getuser(), getpass.getpass())
M.select()
typ, data = M.search(None, 'ALL')
for num in data[0].split():
    typ, data = M.fetch(num, '(RFC822)')
    print 'Message %s\n%s\n' % (num, data[0][1])
M.logout()
\end{verbatim}

\section{Standard Module \module{nntplib}}
\label{module-nntplib}
\stmodindex{nntplib}
\indexii{NNTP}{protocol}
\index{Network News Transfer Protocol}


This module defines the class \class{NNTP} which implements the client
side of the NNTP protocol.  It can be used to implement a news reader
or poster, or automated news processors.  For more information on NNTP
(Network News Transfer Protocol), see Internet \rfc{977}.

Here are two small examples of how it can be used.  To list some
statistics about a newsgroup and print the subjects of the last 10
articles:

\begin{verbatim}
>>> s = NNTP('news.cwi.nl')
>>> resp, count, first, last, name = s.group('comp.lang.python')
>>> print 'Group', name, 'has', count, 'articles, range', first, 'to', last
Group comp.lang.python has 59 articles, range 3742 to 3803
>>> resp, subs = s.xhdr('subject', first + '-' + last)
>>> for id, sub in subs[-10:]: print id, sub
... 
3792 Re: Removing elements from a list while iterating...
3793 Re: Who likes Info files?
3794 Emacs and doc strings
3795 a few questions about the Mac implementation
3796 Re: executable python scripts
3797 Re: executable python scripts
3798 Re: a few questions about the Mac implementation 
3799 Re: PROPOSAL: A Generic Python Object Interface for Python C Modules
3802 Re: executable python scripts 
3803 Re: \POSIX{} wait and SIGCHLD
>>> s.quit()
'205 news.cwi.nl closing connection.  Goodbye.'
\end{verbatim}

To post an article from a file (this assumes that the article has
valid headers):

\begin{verbatim}
>>> s = NNTP('news.cwi.nl')
>>> f = open('/tmp/article')
>>> s.post(f)
'240 Article posted successfully.'
>>> s.quit()
'205 news.cwi.nl closing connection.  Goodbye.'
\end{verbatim}
%
The module itself defines the following items:

\begin{classdesc}{NNTP}{host\optional{, port}}
Return a new instance of the \class{NNTP} class, representing a
connection to the NNTP server running on host \var{host}, listening at
port \var{port}.  The default \var{port} is 119.
\end{classdesc}

\begin{excdesc}{error_reply}
Exception raised when an unexpected reply is received from the server.
\end{excdesc}

\begin{excdesc}{error_temp}
Exception raised when an error code in the range 400--499 is received.
\end{excdesc}

\begin{excdesc}{error_perm}
Exception raised when an error code in the range 500--599 is received.
\end{excdesc}

\begin{excdesc}{error_proto}
Exception raised when a reply is received from the server that does
not begin with a digit in the range 1--5.
\end{excdesc}


\subsection{NNTP Objects}
\label{nntp-objects}

NNTP instances have the following methods.  The \var{response} that is
returned as the first item in the return tuple of almost all methods
is the server's response: a string beginning with a three-digit code.
If the server's response indicates an error, the method raises one of
the above exceptions.


\begin{methoddesc}{getwelcome}{}
Return the welcome message sent by the server in reply to the initial
connection.  (This message sometimes contains disclaimers or help
information that may be relevant to the user.)
\end{methoddesc}

\begin{methoddesc}{set_debuglevel}{level}
Set the instance's debugging level.  This controls the amount of
debugging output printed.  The default, \code{0}, produces no debugging
output.  A value of \code{1} produces a moderate amount of debugging
output, generally a single line per request or response.  A value of
\code{2} or higher produces the maximum amount of debugging output,
logging each line sent and received on the connection (including
message text).
\end{methoddesc}

\begin{methoddesc}{newgroups}{date, time}
Send a \samp{NEWGROUPS} command.  The \var{date} argument should be a
string of the form \code{"\var{yy}\var{mm}\var{dd}"} indicating the
date, and \var{time} should be a string of the form
\code{"\var{hh}\var{mm}\var{ss}"} indicating the time.  Return a pair
\code{(\var{response}, \var{groups})} where \var{groups} is a list of
group names that are new since the given date and time.
\end{methoddesc}

\begin{methoddesc}{newnews}{group, date, time}
Send a \samp{NEWNEWS} command.  Here, \var{group} is a group name or
\code{'*'}, and \var{date} and \var{time} have the same meaning as for
\method{newgroups()}.  Return a pair \code{(\var{response},
\var{articles})} where \var{articles} is a list of article ids.
\end{methoddesc}

\begin{methoddesc}{list}{}
Send a \samp{LIST} command.  Return a pair \code{(\var{response},
\var{list})} where \var{list} is a list of tuples.  Each tuple has the
form \code{(\var{group}, \var{last}, \var{first}, \var{flag})}, where
\var{group} is a group name, \var{last} and \var{first} are the last
and first article numbers (as strings), and \var{flag} is \code{'y'}
if posting is allowed, \code{'n'} if not, and \code{'m'} if the
newsgroup is moderated.  (Note the ordering: \var{last}, \var{first}.)
\end{methoddesc}

\begin{methoddesc}{group}{name}
Send a \samp{GROUP} command, where \var{name} is the group name.
Return a tuple \code{(}\var{response}\code{,} \var{count}\code{,}
\var{first}\code{,} \var{last}\code{,} \var{name}\code{)} where
\var{count} is the (estimated) number of articles in the group,
\var{first} is the first article number in the group, \var{last} is
the last article number in the group, and \var{name} is the group
name.  The numbers are returned as strings.
\end{methoddesc}

\begin{methoddesc}{help}{}
Send a \samp{HELP} command.  Return a pair \code{(\var{response},
\var{list})} where \var{list} is a list of help strings.
\end{methoddesc}

\begin{methoddesc}{stat}{id}
Send a \samp{STAT} command, where \var{id} is the message id (enclosed
in \character{<} and \character{>}) or an article number (as a string).
Return a triple \code{(\var{response}, \var{number}, \var{id})} where
\var{number} is the article number (as a string) and \var{id} is the
article id  (enclosed in \character{<} and \character{>}).
\end{methoddesc}

\begin{methoddesc}{next}{}
Send a \samp{NEXT} command.  Return as for \method{stat()}.
\end{methoddesc}

\begin{methoddesc}{last}{}
Send a \samp{LAST} command.  Return as for \method{stat()}.
\end{methoddesc}

\begin{methoddesc}{head}{id}
Send a \samp{HEAD} command, where \var{id} has the same meaning as for
\method{stat()}.  Return a tuple
\code{(\var{response}, \var{number}, \var{id}, \var{list})}
where the first three are the same as for \method{stat()},
and \var{list} is a list of the article's headers (an uninterpreted
list of lines, without trailing newlines).
\end{methoddesc}

\begin{methoddesc}{body}{id}
Send a \samp{BODY} command, where \var{id} has the same meaning as for
\method{stat()}.  Return as for \method{head()}.
\end{methoddesc}

\begin{methoddesc}{article}{id}
Send a \samp{ARTICLE} command, where \var{id} has the same meaning as
for \method{stat()}.  Return as for \method{head()}.
\end{methoddesc}

\begin{methoddesc}{slave}{}
Send a \samp{SLAVE} command.  Return the server's \var{response}.
\end{methoddesc}

\begin{methoddesc}{xhdr}{header, string}
Send an \samp{XHDR} command.  This command is not defined in the RFC
but is a common extension.  The \var{header} argument is a header
keyword, e.g. \code{'subject'}.  The \var{string} argument should have
the form \code{"\var{first}-\var{last}"} where \var{first} and
\var{last} are the first and last article numbers to search.  Return a
pair \code{(\var{response}, \var{list})}, where \var{list} is a list of
pairs \code{(\var{id}, \var{text})}, where \var{id} is an article id
(as a string) and \var{text} is the text of the requested header for
that article.
\end{methoddesc}

\begin{methoddesc}{post}{file}
Post an article using the \samp{POST} command.  The \var{file}
argument is an open file object which is read until EOF using its
\method{readline()} method.  It should be a well-formed news article,
including the required headers.  The \method{post()} method
automatically escapes lines beginning with \samp{.}.
\end{methoddesc}

\begin{methoddesc}{ihave}{id, file}
Send an \samp{IHAVE} command.  If the response is not an error, treat
\var{file} exactly as for the \method{post()} method.
\end{methoddesc}

\begin{methoddesc}{date}{}
Return a triple \code{(\var{response}, \var{date}, \var{time})},
containing the current date and time in a form suitable for the
\method{newnews()} and \method{newgroups()} methods.
This is an optional NNTP extension, and may not be supported by all
servers.
\end{methoddesc}

\begin{methoddesc}{xgtitle}{name}
Process an \samp{XGTITLE} command, returning a pair \code{(\var{response},
\var{list})}, where \var{list} is a list of tuples containing
\code{(\var{name}, \var{title})}.
% XXX huh?  Should that be name, description?
This is an optional NNTP extension, and may not be supported by all
servers.
\end{methoddesc}

\begin{methoddesc}{xover}{start, end}
Return a pair \code{(\var{resp}, \var{list})}.  \var{list} is a list
of tuples, one for each article in the range delimited by the \var{start}
and \var{end} article numbers.  Each tuple is of the form
\code{(}\var{article number}\code{,} \var{subject}\code{,}
\var{poster}\code{,} \var{date}\code{,} \var{id}\code{,}
\var{references}\code{,} \var{size}\code{,} \var{lines}\code{)}.
This is an optional NNTP extension, and may not be supported by all
servers.
\end{methoddesc}

\begin{methoddesc}{xpath}{id}
Return a pair \code{(\var{resp}, \var{path})}, where \var{path} is the
directory path to the article with message ID \var{id}.  This is an
optional NNTP extension, and may not be supported by all servers.
\end{methoddesc}

\begin{methoddesc}{quit}{}
Send a \samp{QUIT} command and close the connection.  Once this method
has been called, no other methods of the NNTP object should be called.
\end{methoddesc}

\section{\module{smtplib} ---
         SMTP protocol client.}
\declaremodule{standard}{smtplib}
\sectionauthor{Eric S. Raymond}{esr@snark.thyrsus.com}

\modulesynopsis{SMTP protocol client (requires sockets).}

\indexii{SMTP}{protocol}
\index{Simple Mail Transfer Protocol}

The \module{smtplib} module defines an SMTP client session object that
can be used to send mail to any Internet machine with an SMTP or ESMTP
listener daemon.  For details of SMTP and ESMTP operation, consult
\rfc{821} (\emph{Simple Mail Transfer Protocol}) and \rfc{1869}
(\emph{SMTP Service Extensions}).

\begin{classdesc}{SMTP}{\optional{host\optional{, port}}}
A \class{SMTP} instance encapsulates an SMTP connection.  It has
methods that support a full repertoire of SMTP and ESMTP
operations. If the optional host and port parameters are given, the
SMTP \method{connect()} method is called with those parameters during
initialization.

For normal use, you should only require the initialization/connect,
\method{sendmail()}, and \method{quit()} methods  An example is
included below.
\end{classdesc}


\subsection{SMTP Objects}
\label{SMTP-objects}

An \class{SMTP} instance has the following methods:

\begin{methoddesc}{set_debuglevel}{level}
Set the debug output level.  A true value for \var{level} results in
debug messages for connection and for all messages sent to and
received from the server.
\end{methoddesc}

\begin{methoddesc}{connect}{\optional{host\optional{, port}}}
Connect to a host on a given port.  The defaults are to connect to the 
local host at the standard SMTP port (25).

If the hostname ends with a colon (\character{:}) followed by a
number, that suffix will be stripped off and the number interpreted as 
the port number to use.

Note:  This method is automatically invoked by the constructor if a
host is specified during instantiation.
\end{methoddesc}

\begin{methoddesc}{docmd}{cmd, \optional{, argstring}}
Send a command \var{cmd} to the server.  The optional argument
\var{argstring} is simply concatenated to the command, separated by a
space.

This returns a 2-tuple composed of a numeric response code and the
actual response line (multiline responses are joined into one long
line.)

In normal operation it should not be necessary to call this method
explicitly.  It is used to implement other methods and may be useful
for testing private extensions.
\end{methoddesc}

\begin{methoddesc}{helo}{\optional{hostname}}
Identify yourself to the SMTP server using \samp{HELO}.  The hostname
argument defaults to the fully qualified domain name of the local
host.

In normal operation it should not be necessary to call this method
explicitly.  It will be implicitly called by the \method{sendmail()}
when necessary.
\end{methoddesc}

\begin{methoddesc}{ehlo}{\optional{hostname}}
Identify yourself to an ESMTP server using \samp{HELO}.  The hostname
argument defaults to the fully qualified domain name of the local
host.  Examine the response for ESMTP option and store them for use by
\method{has_option()}.

Unless you wish to use \method{has_option()} before sending
mail, it should not be necessary to call this method explicitly.  It
will be implicitly called by \method{sendmail()} when necessary.
\end{methoddesc}

\begin{methoddesc}{has_option}{name}
Return \code{1} if \var{name} is in the set of ESMTP options returned
by the server, \code{0} otherwise.  Case is ignored.
\end{methoddesc}

\begin{methoddesc}{verify}{address}
Check the validity of an address on this server using SMTP \samp{VRFY}.
Returns a tuple consisting of code 250 and a full \rfc{822} address
(including human name) if the user address is valid. Otherwise returns
an SMTP error code of 400 or greater and an error string.

Note: many sites disable SMTP \samp{VRFY} in order to foil spammers.
\end{methoddesc}

\begin{methoddesc}{sendmail}{from_addr, to_addrs, msg\optional{, options}}
Send mail.  The required arguments are an \rfc{822} from-address
string, a list of \rfc{822} to-address strings, and a message string.
The caller may pass a list of ESMTP options to be used in \samp{MAIL
FROM} commands as \var{options}.

If there has been no previous \samp{EHLO} or \samp{HELO} command this
session, this method tries ESMTP \samp{EHLO} first. If the server does
ESMTP, message size and each of the specified options will be passed
to it (if the option is in the feature set the server advertises).  If
\samp{EHLO} fails, \samp{HELO} will be tried and ESMTP options
suppressed.

This method will return normally if the mail is accepted for at least 
one recipient. Otherwise it will throw an exception (either
\exception{SMTPSenderRefused}, \exception{SMTPRecipientsRefused}, or
\exception{SMTPDataError}).  That is, if this method does not throw an
exception, then someone should get your mail.  If this method does not
throw an exception, it returns a dictionary, with one entry for each
recipient that was refused.
\end{methoddesc}

\begin{methoddesc}{quit}{}
Terminate the SMTP session and close the connection.
\end{methoddesc}

Low-level methods corresponding to the standard SMTP/ESMTP commands
\samp{HELP}, \samp{RSET}, \samp{NOOP}, \samp{MAIL}, \samp{RCPT}, and
\samp{DATA} are also supported.  Normally these do not need to be
called directly, so they are not documented here.  For details,
consult the module code.


\subsection{SMTP Example \label{SMTP-example}}

% really need a little description here...

\begin{verbatim}
import rfc822, string, sys
import smtplib

def prompt(prompt):
    sys.stdout.write(prompt + ": ")
    return string.strip(sys.stdin.readline())

fromaddr = prompt("From")
toaddrs  = string.splitfields(prompt("To"), ',')
print "Enter message, end with ^D:"
msg = ''
while 1:
    line = sys.stdin.readline()
    if not line:
        break
    msg = msg + line
print "Message length is " + `len(msg)`

server = smtplib.SMTP('localhost')
server.set_debuglevel(1)
server.sendmail(fromaddr, toaddrs, msg)
server.quit()
\end{verbatim}


\begin{seealso}
\seetext{\rfc{821}, \emph{Simple Mail Transfer Protocol}.  Available
online at \url{http://info.internet.isi.edu/in-notes/rfc/files/rfc821.txt}.}

\seetext{\rfc{1869}, \emph{SMTP Service Extensions}.  Available online 
at \url{http://info.internet.isi.edu/in-notes/rfc/files/rfc1869.txt}.}
\end{seealso}

\section{\module{urlparse} ---
         Parse URLs into components}
\declaremodule{standard}{urlparse}

\modulesynopsis{Parse URLs into components.}

\index{WWW}
\index{World Wide Web}
\index{URL}
\indexii{URL}{parsing}
\indexii{relative}{URL}


This module defines a standard interface to break Uniform Resource
Locator (URL) strings up in components (addressing scheme, network
location, path etc.), to combine the components back into a URL
string, and to convert a ``relative URL'' to an absolute URL given a
``base URL.''

The module has been designed to match the Internet RFC on Relative
Uniform Resource Locators (and discovered a bug in an earlier
draft!). It supports the following URL schemes:
\code{file}, \code{ftp}, \code{gopher}, \code{hdl}, \code{http}, 
\code{https}, \code{imap}, \code{mailto}, \code{mms}, \code{news}, 
\code{nntp}, \code{prospero}, \code{rsync}, \code{rtsp}, \code{rtspu}, 
\code{sftp}, \code{shttp}, \code{sip}, \code{snews}, \code{svn}, 
\code{svn+ssh}, \code{telnet}, \code{wais}.

The \module{urlparse} module defines the following functions:

\begin{funcdesc}{urlparse}{urlstring\optional{, default_scheme\optional{, allow_fragments}}}
Parse a URL into 6 components, returning a 6-tuple: (addressing
scheme, network location, path, parameters, query, fragment
identifier).  This corresponds to the general structure of a URL:
\code{\var{scheme}://\var{netloc}/\var{path};\var{parameters}?\var{query}\#\var{fragment}}.
Each tuple item is a string, possibly empty.
The components are not broken up in smaller parts (e.g. the network
location is a single string), and \% escapes are not expanded.
The delimiters as shown above are not part of the tuple items,
except for a leading slash in the \var{path} component, which is
retained if present.

Example:

\begin{verbatim}
urlparse('http://www.cwi.nl:80/%7Eguido/Python.html')
\end{verbatim}

yields the tuple

\begin{verbatim}
('http', 'www.cwi.nl:80', '/%7Eguido/Python.html', '', '', '')
\end{verbatim}

If the \var{default_scheme} argument is specified, it gives the
default addressing scheme, to be used only if the URL string does not
specify one.  The default value for this argument is the empty string.

If the \var{allow_fragments} argument is zero, fragment identifiers
are not allowed, even if the URL's addressing scheme normally does
support them.  The default value for this argument is \code{1}.
\end{funcdesc}

\begin{funcdesc}{urlunparse}{tuple}
Construct a URL string from a tuple as returned by \code{urlparse()}.
This may result in a slightly different, but equivalent URL, if the
URL that was parsed originally had redundant delimiters, e.g. a ? with
an empty query (the draft states that these are equivalent).
\end{funcdesc}

\begin{funcdesc}{urlsplit}{urlstring\optional{,
                           default_scheme\optional{, allow_fragments}}}
This is similar to \function{urlparse()}, but does not split the
params from the URL.  This should generally be used instead of
\function{urlparse()} if the more recent URL syntax allowing
parameters to be applied to each segment of the \var{path} portion of
the URL (see \rfc{2396}) is wanted.  A separate function is needed to
separate the path segments and parameters.  This function returns a
5-tuple: (addressing scheme, network location, path, query, fragment
identifier).
\versionadded{2.2}
\end{funcdesc}

\begin{funcdesc}{urlunsplit}{tuple}
Combine the elements of a tuple as returned by \function{urlsplit()}
into a complete URL as a string.
\versionadded{2.2}
\end{funcdesc}

\begin{funcdesc}{urljoin}{base, url\optional{, allow_fragments}}
Construct a full (``absolute'') URL by combining a ``base URL''
(\var{base}) with a ``relative URL'' (\var{url}).  Informally, this
uses components of the base URL, in particular the addressing scheme,
the network location and (part of) the path, to provide missing
components in the relative URL.

Example:

\begin{verbatim}
urljoin('http://www.cwi.nl/%7Eguido/Python.html', 'FAQ.html')
\end{verbatim}

yields the string

\begin{verbatim}
'http://www.cwi.nl/%7Eguido/FAQ.html'
\end{verbatim}

The \var{allow_fragments} argument has the same meaning as for
\code{urlparse()}.
\end{funcdesc}

\begin{funcdesc}{urldefrag}{url}
If \var{url} contains a fragment identifier, returns a modified
version of \var{url} with no fragment identifier, and the fragment
identifier as a separate string.  If there is no fragment identifier
in \var{url}, returns \var{url} unmodified and an empty string.
\end{funcdesc}


\begin{seealso}
  \seerfc{1738}{Uniform Resource Locators (URL)}{
        This specifies the formal syntax and semantics of absolute
        URLs.}
  \seerfc{1808}{Relative Uniform Resource Locators}{
        This Request For Comments includes the rules for joining an
        absolute and a relative URL, including a fair number of
        ``Abnormal Examples'' which govern the treatment of border
        cases.}
  \seerfc{2396}{Uniform Resource Identifiers (URI): Generic Syntax}{
        Document describing the generic syntactic requirements for
        both Uniform Resource Names (URNs) and Uniform Resource
        Locators (URLs).}
\end{seealso}

\section{\module{SocketServer} ---
         A framework for network servers.}
\declaremodule{standard}{SocketServer}

\modulesynopsis{A framework for network servers.}


The \module{SocketServer} module simplifies the task of writing network
servers.

There are four basic server classes: \class{TCPServer} uses the
Internet TCP protocol, which provides for continuous streams of data
between the client and server.  \class{UDPServer} uses datagrams, which
are discrete packets of information that may arrive out of order or be
lost while in transit.  The more infrequently used
\class{UnixStreamServer} and \class{UnixDatagramServer} classes are
similar, but use \UNIX{} domain sockets; they're not available on
non-\UNIX{} platforms.  For more details on network programming, consult
a book such as W. Richard Steven's \emph{UNIX Network Programming}
or Ralph Davis's \emph{Win32 Network Programming}.

These four classes process requests \dfn{synchronously}; each request
must be completed before the next request can be started.  This isn't
suitable if each request takes a long time to complete, because it
requires a lot of computation, or because it returns a lot of data
which the client is slow to process.  The solution is to create a
separate process or thread to handle each request; the
\class{ForkingMixIn} and \class{ThreadingMixIn} mix-in classes can be
used to support asynchronous behaviour.

Creating a server requires several steps.  First, you must create a
request handler class by subclassing the \class{BaseRequestHandler}
class and overriding its \method{handle()} method; this method will
process incoming requests.  Second, you must instantiate one of the
server classes, passing it the server's address and the request
handler class.  Finally, call the \method{handle_request()} or
\method{serve_forever()} method of the server object to process one or
many requests.

Server classes have the same external methods and attributes, no
matter what network protocol they use:

\setindexsubitem{(SocketServer protocol)}

%XXX should data and methods be intermingled, or separate?
% how should the distinction between class and instance variables be
% drawn?

\begin{funcdesc}{fileno}{}
Return an integer file descriptor for the socket on which the server
is listening.  This function is most commonly passed to
\function{select.select()}, to allow monitoring multiple servers in the
same process.
\end{funcdesc}

\begin{funcdesc}{handle_request}{}
Process a single request.  This function calls the following methods
in order: \method{get_request()}, \method{verify_request()}, and
\method{process_request()}.  If the user-provided \method{handle()}
method of the handler class raises an exception, the server's
\method{handle_error()} method will be called.
\end{funcdesc}

\begin{funcdesc}{serve_forever}{}
Handle an infinite number of requests.  This simply calls
\method{handle_request()} inside an infinite loop.
\end{funcdesc}

\begin{datadesc}{address_family}
The family of protocols to which the server's socket belongs.
\constant{socket.AF_INET} and \constant{socket.AF_UNIX} are two
possible values.
\end{datadesc}

\begin{datadesc}{RequestHandlerClass}
The user-provided request handler class; an instance of this class is
created for each request.
\end{datadesc}

\begin{datadesc}{server_address}
The address on which the server is listening.  The format of addresses
varies depending on the protocol family; see the documentation for the
socket module for details.  For Internet protocols, this is a tuple
containing a string giving the address, and an integer port number:
\code{('127.0.0.1', 80)}, for example.
\end{datadesc}

\begin{datadesc}{socket}
The socket object on which the server will listen for incoming requests.
\end{datadesc}

% XXX should class variables be covered before instance variables, or
% vice versa?

The server classes support the following class variables:

\begin{datadesc}{request_queue_size}
The size of the request queue.  If it takes a long time to process a
single request, any requests that arrive while the server is busy are
placed into a queue, up to \member{request_queue_size} requests.  Once
the queue is full, further requests from clients will get a
``Connection denied'' error.  The default value is usually 5, but this
can be overridden by subclasses.
\end{datadesc}

\begin{datadesc}{socket_type}
The type of socket used by the server; \constant{socket.SOCK_STREAM}
and \constant{socket.SOCK_DGRAM} are two possible values.
\end{datadesc}

There are various server methods that can be overridden by subclasses
of base server classes like \class{TCPServer}; these methods aren't
useful to external users of the server object.

% should the default implementations of these be documented, or should
% it be assumed that the user will look at SocketServer.py?

\begin{funcdesc}{finish_request}{}
Actually processes the request by instantiating
\member{RequestHandlerClass} and calling its \method{handle()} method.
\end{funcdesc}

\begin{funcdesc}{get_request}{}
Must accept a request from the socket, and return a 2-tuple containing
the \emph{new} socket object to be used to communicate with the
client, and the client's address.
\end{funcdesc}

\begin{funcdesc}{handle_error}{request, client_address}
This function is called if the \member{RequestHandlerClass}'s
\method{handle()} method raises an exception.  The default action is
to print the traceback to standard output and continue handling
further requests.
\end{funcdesc}

\begin{funcdesc}{process_request}{request, client_address}
Calls \method{finish_request()} to create an instance of the
\member{RequestHandlerClass}.  If desired, this function can create a
new process or thread to handle the request; the \class{ForkingMixIn}
and \class{ThreadingMixIn} classes do this.
\end{funcdesc}

% Is there any point in documenting the following two functions?
% What would the purpose of overriding them be: initializing server
% instance variables, adding new network families?

\begin{funcdesc}{server_activate}{}
Called by the server's constructor to activate the server.
May be overridden.
\end{funcdesc}

\begin{funcdesc}{server_bind}{}
Called by the server's constructor to bind the socket to the desired
address.  May be overridden.
\end{funcdesc}

\begin{funcdesc}{verify_request}{request, client_address}
Must return a Boolean value; if the value is true, the request will be
processed, and if it's false, the request will be denied.
This function can be overridden to implement access controls for a server.
The default implementation always return true.
\end{funcdesc}

The request handler class must define a new \method{handle()} method,
and can override any of the following methods.  A new instance is
created for each request.

\begin{funcdesc}{finish}{}
Called after the \method{handle()} method to perform any clean-up
actions required.  The default implementation does nothing.  If
\method{setup()} or \method{handle()} raise an exception, this
function will not be called.
\end{funcdesc}

\begin{funcdesc}{handle}{}
This function must do all the work required to service a request.
Several instance attributes are available to it; the request is
available as \member{self.request}; the client address as
\member{self.client_address}; and the server instance as
\member{self.server}, in case it needs access to per-server
information.

The type of \member{self.request} is different for datagram or stream
services.  For stream services, \member{self.request} is a socket
object; for datagram services, \member{self.request} is a string.
However, this can be hidden by using the mix-in request handler
classes
\class{StreamRequestHandler} or \class{DatagramRequestHandler}, which
override the \method{setup()} and \method{finish()} methods, and
provides \member{self.rfile} and \member{self.wfile} attributes.
\member{self.rfile} and \member{self.wfile} can be read or written,
respectively, to get the request data or return data to the client.
\end{funcdesc}

\begin{funcdesc}{setup}{}
Called before the \method{handle()} method to perform any
initialization actions required.  The default implementation does
nothing.
\end{funcdesc}

\section{\module{BaseHTTPServer} ---
         Basic HTTP server}

\declaremodule{standard}{BaseHTTPServer}
\modulesynopsis{Basic HTTP server (base class for
                \class{SimpleHTTPServer} and \class{CGIHTTPServer}).}


\indexii{WWW}{server}
\indexii{HTTP}{protocol}
\index{URL}
\index{httpd}

This module defines two classes for implementing HTTP servers
(Web servers). Usually, this module isn't used directly, but is used
as a basis for building functioning Web servers. See the
\refmodule{SimpleHTTPServer}\refstmodindex{SimpleHTTPServer} and
\refmodule{CGIHTTPServer}\refstmodindex{CGIHTTPServer} modules.

The first class, \class{HTTPServer}, is a
\class{SocketServer.TCPServer} subclass.  It creates and listens at the
HTTP socket, dispatching the requests to a handler.  Code to create and
run the server looks like this:

\begin{verbatim}
def run(server_class=BaseHTTPServer.HTTPServer,
        handler_class=BaseHTTPServer.BaseHTTPRequestHandler):
    server_address = ('', 8000)
    httpd = server_class(server_address, handler_class)
    httpd.serve_forever()
\end{verbatim}

\begin{classdesc}{HTTPServer}{server_address, RequestHandlerClass}
This class builds on the \class{TCPServer} class by
storing the server address as instance
variables named \member{server_name} and \member{server_port}. The
server is accessible by the handler, typically through the handler's
\member{server} instance variable.
\end{classdesc}

\begin{classdesc}{BaseHTTPRequestHandler}{request, client_address, server}
This class is used
to handle the HTTP requests that arrive at the server. By itself,
it cannot respond to any actual HTTP requests; it must be subclassed
to handle each request method (e.g. GET or POST).
\class{BaseHTTPRequestHandler} provides a number of class and instance
variables, and methods for use by subclasses.

The handler will parse the request and the headers, then call a
method specific to the request type. The method name is constructed
from the request. For example, for the request method \samp{SPAM}, the
\method{do_SPAM()} method will be called with no arguments. All of
the relevant information is stored in instance variables of the
handler.  Subclasses should not need to override or extend the
\method{__init__()} method.
\end{classdesc}


\class{BaseHTTPRequestHandler} has the following instance variables:

\begin{memberdesc}{client_address}
Contains a tuple of the form \code{(\var{host}, \var{port})} referring
to the client's address.
\end{memberdesc}

\begin{memberdesc}{command}
Contains the command (request type). For example, \code{'GET'}.
\end{memberdesc}

\begin{memberdesc}{path}
Contains the request path.
\end{memberdesc}

\begin{memberdesc}{request_version}
Contains the version string from the request. For example,
\code{'HTTP/1.0'}.
\end{memberdesc}

\begin{memberdesc}{headers}
Holds an instance of the class specified by the \member{MessageClass}
class variable. This instance parses and manages the headers in
the HTTP request.
\end{memberdesc}

\begin{memberdesc}{rfile}
Contains an input stream, positioned at the start of the optional
input data.
\end{memberdesc}

\begin{memberdesc}{wfile}
Contains the output stream for writing a response back to the client.
Proper adherence to the HTTP protocol must be used when writing
to this stream.
\end{memberdesc}


\class{BaseHTTPRequestHandler} has the following class variables:

\begin{memberdesc}{server_version}
Specifies the server software version.  You may want to override
this.
The format is multiple whitespace-separated strings,
where each string is of the form name[/version].
For example, \code{'BaseHTTP/0.2'}.
\end{memberdesc}

\begin{memberdesc}{sys_version}
Contains the Python system version, in a form usable by the
\member{version_string} method and the \member{server_version} class
variable. For example, \code{'Python/1.4'}.
\end{memberdesc}

\begin{memberdesc}{error_message_format}
Specifies a format string for building an error response to the
client. It uses parenthesized, keyed format specifiers, so the
format operand must be a dictionary. The \var{code} key should
be an integer, specifying the numeric HTTP error code value.
\var{message} should be a string containing a (detailed) error
message of what occurred, and \var{explain} should be an
explanation of the error code number. Default \var{message}
and \var{explain} values can found in the \var{responses}
class variable.
\end{memberdesc}

\begin{memberdesc}{protocol_version}
This specifies the HTTP protocol version used in responses.  If set
to \code{'HTTP/1.1'}, the server will permit HTTP persistent
connections; however, your server \emph{must} then include an
accurate \code{Content-Length} header (using \method{send_header()})
in all of its responses to clients.  For backwards compatibility,
the setting defaults to \code{'HTTP/1.0'}.
\end{memberdesc}

\begin{memberdesc}{MessageClass}
Specifies a \class{rfc822.Message}-like class to parse HTTP
headers. Typically, this is not overridden, and it defaults to
\class{mimetools.Message}.
\withsubitem{(in module mimetools)}{\ttindex{Message}}
\end{memberdesc}

\begin{memberdesc}{responses}
This variable contains a mapping of error code integers to two-element
tuples containing a short and long message. For example,
\code{\{\var{code}: (\var{shortmessage}, \var{longmessage})\}}. The
\var{shortmessage} is usually used as the \var{message} key in an
error response, and \var{longmessage} as the \var{explain} key
(see the \member{error_message_format} class variable).
\end{memberdesc}


A \class{BaseHTTPRequestHandler} instance has the following methods:

\begin{methoddesc}{handle}{}
Calls \method{handle_one_request()} once (or, if persistent connections
are enabled, multiple times) to handle incoming HTTP requests.
You should never need to override it; instead, implement appropriate
\method{do_*()} methods.
\end{methoddesc}

\begin{methoddesc}{handle_one_request}{}
This method will parse and dispatch
the request to the appropriate \method{do_*()} method.  You should
never need to override it.
\end{methoddesc}

\begin{methoddesc}{send_error}{code\optional{, message}}
Sends and logs a complete error reply to the client. The numeric
\var{code} specifies the HTTP error code, with \var{message} as
optional, more specific text. A complete set of headers is sent,
followed by text composed using the \member{error_message_format}
class variable.
\end{methoddesc}

\begin{methoddesc}{send_response}{code\optional{, message}}
Sends a response header and logs the accepted request. The HTTP
response line is sent, followed by \emph{Server} and \emph{Date}
headers. The values for these two headers are picked up from the
\method{version_string()} and \method{date_time_string()} methods,
respectively.
\end{methoddesc}

\begin{methoddesc}{send_header}{keyword, value}
Writes a specific HTTP header to the output stream. \var{keyword}
should specify the header keyword, with \var{value} specifying
its value.
\end{methoddesc}

\begin{methoddesc}{end_headers}{}
Sends a blank line, indicating the end of the HTTP headers in
the response.
\end{methoddesc}

\begin{methoddesc}{log_request}{\optional{code\optional{, size}}}
Logs an accepted (successful) request. \var{code} should specify
the numeric HTTP code associated with the response. If a size of
the response is available, then it should be passed as the
\var{size} parameter.
\end{methoddesc}

\begin{methoddesc}{log_error}{...}
Logs an error when a request cannot be fulfilled. By default,
it passes the message to \method{log_message()}, so it takes the
same arguments (\var{format} and additional values).
\end{methoddesc}

\begin{methoddesc}{log_message}{format, ...}
Logs an arbitrary message to \code{sys.stderr}. This is typically
overridden to create custom error logging mechanisms. The
\var{format} argument is a standard printf-style format string,
where the additional arguments to \method{log_message()} are applied
as inputs to the formatting. The client address and current date
and time are prefixed to every message logged.
\end{methoddesc}

\begin{methoddesc}{version_string}{}
Returns the server software's version string. This is a combination
of the \member{server_version} and \member{sys_version} class variables.
\end{methoddesc}

\begin{methoddesc}{date_time_string}{}
Returns the current date and time, formatted for a message header.
\end{methoddesc}

\begin{methoddesc}{log_data_time_string}{}
Returns the current date and time, formatted for logging.
\end{methoddesc}

\begin{methoddesc}{address_string}{}
Returns the client address, formatted for logging. A name lookup
is performed on the client's IP address.
\end{methoddesc}


\begin{seealso}
  \seemodule{CGIHTTPServer}{Extended request handler that supports CGI
                            scripts.}

  \seemodule{SimpleHTTPServer}{Basic request handler that limits response
                               to files actually under the document root.}
\end{seealso}


\chapter{Internet Data Handling \label{netdata}}

This chapter describes modules which support handling data formats
commonly used on the internet.  Some, like SGML and XML, may be useful 
for other applications as well.

\localmoduletable

\section{\module{sgmllib} ---
         Simple SGML parser}

\declaremodule{standard}{sgmllib}
\modulesynopsis{Only as much of an SGML parser as needed to parse HTML.}

\index{SGML}

This module defines a class \class{SGMLParser} which serves as the
basis for parsing text files formatted in SGML (Standard Generalized
Mark-up Language).  In fact, it does not provide a full SGML parser
--- it only parses SGML insofar as it is used by HTML, and the module
only exists as a base for the \refmodule{htmllib}\refstmodindex{htmllib}
module.


\begin{classdesc}{SGMLParser}{}
The \class{SGMLParser} class is instantiated without arguments.
The parser is hardcoded to recognize the following
constructs:

\begin{itemize}
\item
Opening and closing tags of the form
\samp{<\var{tag} \var{attr}="\var{value}" ...>} and
\samp{</\var{tag}>}, respectively.

\item
Numeric character references of the form \samp{\&\#\var{name};}.

\item
Entity references of the form \samp{\&\var{name};}.

\item
SGML comments of the form \samp{<!--\var{text}-->}.  Note that
spaces, tabs, and newlines are allowed between the trailing
\samp{>} and the immediately preceeding \samp{--}.

\end{itemize}
\end{classdesc}

\class{SGMLParser} instances have the following interface methods:


\begin{methoddesc}{reset}{}
Reset the instance.  Loses all unprocessed data.  This is called
implicitly at instantiation time.
\end{methoddesc}

\begin{methoddesc}{setnomoretags}{}
Stop processing tags.  Treat all following input as literal input
(CDATA).  (This is only provided so the HTML tag
\code{<PLAINTEXT>} can be implemented.)
\end{methoddesc}

\begin{methoddesc}{setliteral}{}
Enter literal mode (CDATA mode).
\end{methoddesc}

\begin{methoddesc}{feed}{data}
Feed some text to the parser.  It is processed insofar as it consists
of complete elements; incomplete data is buffered until more data is
fed or \method{close()} is called.
\end{methoddesc}

\begin{methoddesc}{close}{}
Force processing of all buffered data as if it were followed by an
end-of-file mark.  This method may be redefined by a derived class to
define additional processing at the end of the input, but the
redefined version should always call \method{close()}.
\end{methoddesc}

\begin{methoddesc}{get_starttag_text}{}
Return the text of the most recently opened start tag.  This should
not normally be needed for structured processing, but may be useful in
dealing with HTML ``as deployed'' or for re-generating input with
minimal changes (whitespace between attributes can be preserved,
etc.).
\end{methoddesc}

\begin{methoddesc}{handle_starttag}{tag, method, attributes}
This method is called to handle start tags for which either a
\method{start_\var{tag}()} or \method{do_\var{tag}()} method has been
defined.  The \var{tag} argument is the name of the tag converted to
lower case, and the \var{method} argument is the bound method which
should be used to support semantic interpretation of the start tag.
The \var{attributes} argument is a list of \code{(\var{name},
\var{value})} pairs containing the attributes found inside the tag's
\code{<>} brackets.  The \var{name} has been translated to lower case
and double quotes and backslashes in the \var{value} have been interpreted.
For instance, for the tag \code{<A HREF="http://www.cwi.nl/">}, this
method would be called as \samp{unknown_starttag('a', [('href',
'http://www.cwi.nl/')])}.  The base implementation simply calls
\var{method} with \var{attributes} as the only argument.
\end{methoddesc}

\begin{methoddesc}{handle_endtag}{tag, method}
This method is called to handle endtags for which an
\method{end_\var{tag}()} method has been defined.  The
\var{tag} argument is the name of the tag converted to lower case, and
the \var{method} argument is the bound method which should be used to
support semantic interpretation of the end tag.  If no
\method{end_\var{tag}()} method is defined for the closing element,
this handler is not called.  The base implementation simply calls
\var{method}.
\end{methoddesc}

\begin{methoddesc}{handle_data}{data}
This method is called to process arbitrary data.  It is intended to be
overridden by a derived class; the base class implementation does
nothing.
\end{methoddesc}

\begin{methoddesc}{handle_charref}{ref}
This method is called to process a character reference of the form
\samp{\&\#\var{ref};}.  In the base implementation, \var{ref} must
be a decimal number in the
range 0-255.  It translates the character to \ASCII{} and calls the
method \method{handle_data()} with the character as argument.  If
\var{ref} is invalid or out of range, the method
\code{unknown_charref(\var{ref})} is called to handle the error.  A
subclass must override this method to provide support for named
character entities.
\end{methoddesc}

\begin{methoddesc}{handle_entityref}{ref}
This method is called to process a general entity reference of the
form \samp{\&\var{ref};} where \var{ref} is an general entity
reference.  It looks for \var{ref} in the instance (or class)
variable \member{entitydefs} which should be a mapping from entity
names to corresponding translations.  If a translation is found, it
calls the method \method{handle_data()} with the translation;
otherwise, it calls the method \code{unknown_entityref(\var{ref})}.
The default \member{entitydefs} defines translations for
\code{\&amp;}, \code{\&apos}, \code{\&gt;}, \code{\&lt;}, and
\code{\&quot;}.
\end{methoddesc}

\begin{methoddesc}{handle_comment}{comment}
This method is called when a comment is encountered.  The
\var{comment} argument is a string containing the text between the
\samp{<!--} and \samp{-->} delimiters, but not the delimiters
themselves.  For example, the comment \samp{<!--text-->} will
cause this method to be called with the argument \code{'text'}.  The
default method does nothing.
\end{methoddesc}

\begin{methoddesc}{report_unbalanced}{tag}
This method is called when an end tag is found which does not
correspond to any open element.
\end{methoddesc}

\begin{methoddesc}{unknown_starttag}{tag, attributes}
This method is called to process an unknown start tag.  It is intended
to be overridden by a derived class; the base class implementation
does nothing.
\end{methoddesc}

\begin{methoddesc}{unknown_endtag}{tag}
This method is called to process an unknown end tag.  It is intended
to be overridden by a derived class; the base class implementation
does nothing.
\end{methoddesc}

\begin{methoddesc}{unknown_charref}{ref}
This method is called to process unresolvable numeric character
references.  Refer to \method{handle_charref()} to determine what is
handled by default.  It is intended to be overridden by a derived
class; the base class implementation does nothing.
\end{methoddesc}

\begin{methoddesc}{unknown_entityref}{ref}
This method is called to process an unknown entity reference.  It is
intended to be overridden by a derived class; the base class
implementation does nothing.
\end{methoddesc}

Apart from overriding or extending the methods listed above, derived
classes may also define methods of the following form to define
processing of specific tags.  Tag names in the input stream are case
independent; the \var{tag} occurring in method names must be in lower
case:

\begin{methoddescni}{start_\var{tag}}{attributes}
This method is called to process an opening tag \var{tag}.  It has
preference over \method{do_\var{tag}()}.  The
\var{attributes} argument has the same meaning as described for
\method{handle_starttag()} above.
\end{methoddescni}

\begin{methoddescni}{do_\var{tag}}{attributes}
This method is called to process an opening tag \var{tag} that does
not come with a matching closing tag.  The \var{attributes} argument
has the same meaning as described for \method{handle_starttag()} above.
\end{methoddescni}

\begin{methoddescni}{end_\var{tag}}{}
This method is called to process a closing tag \var{tag}.
\end{methoddescni}

Note that the parser maintains a stack of open elements for which no
end tag has been found yet.  Only tags processed by
\method{start_\var{tag}()} are pushed on this stack.  Definition of an
\method{end_\var{tag}()} method is optional for these tags.  For tags
processed by \method{do_\var{tag}()} or by \method{unknown_tag()}, no
\method{end_\var{tag}()} method must be defined; if defined, it will
not be used.  If both \method{start_\var{tag}()} and
\method{do_\var{tag}()} methods exist for a tag, the
\method{start_\var{tag}()} method takes precedence.

\section{\module{htmllib} ---
         A parser for HTML documents}

\declaremodule{standard}{htmllib}
\modulesynopsis{A parser for HTML documents.}

\index{HTML}
\index{hypertext}


This module defines a class which can serve as a base for parsing text
files formatted in the HyperText Mark-up Language (HTML).  The class
is not directly concerned with I/O --- it must be provided with input
in string form via a method, and makes calls to methods of a
``formatter'' object in order to produce output.  The
\class{HTMLParser} class is designed to be used as a base class for
other classes in order to add functionality, and allows most of its
methods to be extended or overridden.  In turn, this class is derived
from and extends the \class{SGMLParser} class defined in module
\refmodule{sgmllib}\refstmodindex{sgmllib}.  The \class{HTMLParser}
implementation supports the HTML 2.0 language as described in
\rfc{1866}.  Two implementations of formatter objects are provided in
the \refmodule{formatter}\refstmodindex{formatter}\ module; refer to the
documentation for that module for information on the formatter
interface.
\withsubitem{(in module sgmllib)}{\ttindex{SGMLParser}}

The following is a summary of the interface defined by
\class{sgmllib.SGMLParser}:

\begin{itemize}

\item
The interface to feed data to an instance is through the \method{feed()}
method, which takes a string argument.  This can be called with as
little or as much text at a time as desired; \samp{p.feed(a);
p.feed(b)} has the same effect as \samp{p.feed(a+b)}.  When the data
contains complete HTML tags, these are processed immediately;
incomplete elements are saved in a buffer.  To force processing of all
unprocessed data, call the \method{close()} method.

For example, to parse the entire contents of a file, use:
\begin{verbatim}
parser.feed(open('myfile.html').read())
parser.close()
\end{verbatim}

\item
The interface to define semantics for HTML tags is very simple: derive
a class and define methods called \method{start_\var{tag}()},
\method{end_\var{tag}()}, or \method{do_\var{tag}()}.  The parser will
call these at appropriate moments: \method{start_\var{tag}} or
\method{do_\var{tag}()} is called when an opening tag of the form
\code{<\var{tag} ...>} is encountered; \method{end_\var{tag}()} is called
when a closing tag of the form \code{<\var{tag}>} is encountered.  If
an opening tag requires a corresponding closing tag, like \code{<H1>}
... \code{</H1>}, the class should define the \method{start_\var{tag}()}
method; if a tag requires no closing tag, like \code{<P>}, the class
should define the \method{do_\var{tag}()} method.

\end{itemize}

The module defines a single class:

\begin{classdesc}{HTMLParser}{formatter}
This is the basic HTML parser class.  It supports all entity names
required by the HTML 2.0 specification (\rfc{1866}).  It also defines
handlers for all HTML 2.0 and many HTML 3.0 and 3.2 elements.
\end{classdesc}


\begin{seealso}
  \seemodule{formatter}{Interface definition for transforming an
                        abstract flow of formatting events into
                        specific output events on writer objects.}
  \seemodule{HTMLParser}{Alternate HTML parser that offers a slightly
                         lower-level view of the input, but is
                         designed to work with XHTML, and does not
                         implement some of the SGML syntax not used in
                         ``HTML as deployed'' and which isn't legal
                         for XHTML.}
  \seemodule{htmlentitydefs}{Definition of replacement text for HTML
                             2.0 entities.}
  \seemodule{sgmllib}{Base class for \class{HTMLParser}.}
\end{seealso}


\subsection{HTMLParser Objects \label{html-parser-objects}}

In addition to tag methods, the \class{HTMLParser} class provides some
additional methods and instance variables for use within tag methods.

\begin{memberdesc}{formatter}
This is the formatter instance associated with the parser.
\end{memberdesc}

\begin{memberdesc}{nofill}
Boolean flag which should be true when whitespace should not be
collapsed, or false when it should be.  In general, this should only
be true when character data is to be treated as ``preformatted'' text,
as within a \code{<PRE>} element.  The default value is false.  This
affects the operation of \method{handle_data()} and \method{save_end()}.
\end{memberdesc}


\begin{methoddesc}{anchor_bgn}{href, name, type}
This method is called at the start of an anchor region.  The arguments
correspond to the attributes of the \code{<A>} tag with the same
names.  The default implementation maintains a list of hyperlinks
(defined by the \code{HREF} attribute for \code{<A>} tags) within the
document.  The list of hyperlinks is available as the data attribute
\member{anchorlist}.
\end{methoddesc}

\begin{methoddesc}{anchor_end}{}
This method is called at the end of an anchor region.  The default
implementation adds a textual footnote marker using an index into the
list of hyperlinks created by \method{anchor_bgn()}.
\end{methoddesc}

\begin{methoddesc}{handle_image}{source, alt\optional{, ismap\optional{, align\optional{, width\optional{, height}}}}}
This method is called to handle images.  The default implementation
simply passes the \var{alt} value to the \method{handle_data()}
method.
\end{methoddesc}

\begin{methoddesc}{save_bgn}{}
Begins saving character data in a buffer instead of sending it to the
formatter object.  Retrieve the stored data via \method{save_end()}.
Use of the \method{save_bgn()} / \method{save_end()} pair may not be
nested.
\end{methoddesc}

\begin{methoddesc}{save_end}{}
Ends buffering character data and returns all data saved since the
preceding call to \method{save_bgn()}.  If the \member{nofill} flag is
false, whitespace is collapsed to single spaces.  A call to this
method without a preceding call to \method{save_bgn()} will raise a
\exception{TypeError} exception.
\end{methoddesc}



\section{\module{htmlentitydefs} ---
         Definitions of HTML general entities}

\declaremodule{standard}{htmlentitydefs}
\modulesynopsis{Definitions of HTML general entities.}
\sectionauthor{Fred L. Drake, Jr.}{fdrake@acm.org}

This module defines three dictionaries, \code{name2codepoint},
\code{codepoint2name}, and \code{entitydefs}. \code{entitydefs} is
used by the \refmodule{htmllib} module to provide the
\member{entitydefs} member of the \class{HTMLParser} class.  The
definition provided here contains all the entities defined by XHTML 1.0 
that can be handled using simple textual substitution in the Latin-1
character set (ISO-8859-1).


\begin{datadesc}{entitydefs}
  A dictionary mapping XHTML 1.0 entity definitions to their
  replacement text in ISO Latin-1.

\end{datadesc}

\begin{datadesc}{name2codepoint}
  A dictionary that maps HTML entity names to the Unicode codepoints.
  \versionadded{2.3}
\end{datadesc}

\begin{datadesc}{codepoint2name}
  A dictionary that maps Unicode codepoints to HTML entity names.
  \versionadded{2.3}
\end{datadesc}

\section{Standard Module \sectcode{xmllib}}
% Author: Sjoerd Mullender
\label{module-xmllib}
\stmodindex{xmllib}
\index{XML}

This module defines a class \code{XMLParser} which serves as the basis 
for parsing text files formatted in XML (eXtended Markup Language).

The \code{XMLParser} class must be instantiated without arguments.  It 
has the following interface methods:

\setindexsubitem{(XMLParser method)}

\begin{funcdesc}{reset}{}
Reset the instance.  Loses all unprocessed data.  This is called
implicitly at the instantiation time.
\end{funcdesc}

\begin{funcdesc}{setnomoretags}{}
Stop processing tags.  Treat all following input as literal input
(CDATA).
\end{funcdesc}

\begin{funcdesc}{setliteral}{}
Enter literal mode (CDATA mode).
\end{funcdesc}

\begin{funcdesc}{feed}{data}
Feed some text to the parser.  It is processed insofar as it consists
of complete elements; incomplete data is buffered until more data is
fed or \code{close()} is called.
\end{funcdesc}

\begin{funcdesc}{close}{}
Force processing of all buffered data as if it were followed by an
end-of-file mark.  This method may be redefined by a derived class to
define additional processing at the end of the input, but the
redefined version should always call \code{XMLParser.close()}.
\end{funcdesc}

\begin{funcdesc}{translate_references}{data}
Translate all entity and character references in \code{data} and
returns the translated string.
\end{funcdesc}

\begin{funcdesc}{handle_xml}{encoding\, standalone}
This method is called when the \code{<?xml ...?>} tag is processed.
The arguments are the values of the encoding and standalone attributes 
in the tag.  Both encoding and standalone are optional.  The values
passed to \code{handle_xml} default to \code{None} and the string
\code{'no'} respectively.
\end{funcdesc}

\begin{funcdesc}{handle_doctype}{tag\, data}
This method is called when the \code{<!DOCTYPE...>} tag is processed.
The arguments are the name of the root element and the uninterpreted
contents of the tag, starting after the white space after the name of
the root element.
\end{funcdesc}

\begin{funcdesc}{handle_starttag}{tag\, method\, attributes}
This method is called to handle start tags for which a
\code{start_\var{tag}()} method has been defined.  The \code{tag}
argument is the name of the tag, and the \code{method} argument is the
bound method which should be used to support semantic interpretation
of the start tag.  The \var{attributes} argument is a dictionary of
attributes, the key being the \var{name} and the value being the
\var{value} of the attribute found inside the tag's \code{<>} brackets.
Character and entity references in the \var{value} have
been interpreted.  For instance, for the tag
\code{<A HREF="http://www.cwi.nl/">}, this method would be called as
\code{handle_starttag('A', self.start_A, \{'HREF': 'http://www.cwi.nl/'\})}.
The base implementation simply calls \code{method} with \code{attributes}
as the only argument.
\end{funcdesc}

\begin{funcdesc}{handle_endtag}{tag\, method}
This method is called to handle endtags for which an
\code{end_\var{tag}()} method has been defined.  The \code{tag}
argument is the name of the tag, and the
\code{method} argument is the bound method which should be used to
support semantic interpretation of the end tag.  If no
\code{end_\var{tag}()} method is defined for the closing element, this
handler is not called.  The base implementation simply calls
\code{method}.
\end{funcdesc}

\begin{funcdesc}{handle_data}{data}
This method is called to process arbitrary data.  It is intended to be
overridden by a derived class; the base class implementation does
nothing.
\end{funcdesc}

\begin{funcdesc}{handle_charref}{ref}
This method is called to process a character reference of the form
\samp{\&\#\var{ref};}.  \var{ref} can either be a decimal number,
or a hexadecimal number when preceded by \code{x}.
In the base implementation, \var{ref} must be a number in the
range 0-255.  It translates the character to \ASCII{} and calls the
method \code{handle_data()} with the character as argument.  If
\var{ref} is invalid or out of range, the method
\code{unknown_charref(\var{ref})} is called to handle the error.  A
subclass must override this method to provide support for character
references outside of the \ASCII{} range.
\end{funcdesc}

\begin{funcdesc}{handle_entityref}{ref}
This method is called to process a general entity reference of the form
\samp{\&\var{ref};} where \var{ref} is an general entity
reference.  It looks for \var{ref} in the instance (or class)
variable \code{entitydefs} which should be a mapping from entity names
to corresponding translations.
If a translation is found, it calls the method \code{handle_data()}
with the translation; otherwise, it calls the method
\code{unknown_entityref(\var{ref})}.  The default \code{entitydefs}
defines translations for \code{\&amp;}, \code{\&apos}, \code{\&gt;},
\code{\&lt;}, and \code{\&quot;}.
\end{funcdesc}

\begin{funcdesc}{handle_comment}{comment}
This method is called when a comment is encountered.  The
\code{comment} argument is a string containing the text between the
\samp{<!--} and \samp{-->} delimiters, but not the delimiters
themselves.  For example, the comment \samp{<!--text-->} will
cause this method to be called with the argument \code{'text'}.  The
default method does nothing.
\end{funcdesc}

\begin{funcdesc}{handle_cdata}{data}
This method is called when a CDATA element is encountered.  The
\code{data} argument is a string containing the text between the
\samp{<![CDATA[} and \samp{]]>} delimiters, but not the delimiters
themselves.  For example, the entity \samp{<![CDATA[text]]>} will
cause this method to be called with the argument \code{'text'}.  The
default method does nothing.
\end{funcdesc}

\begin{funcdesc}{handle_proc}{name\, data}
This method is called when a processing instruction (PI) is encountered.  The
\code{name} is the PI target, and the \code{data} argument is a
string containing the text between the PI target and the closing delimiter,
but not the delimiter itself.  For example, the instruction
\samp{<?XML text?>} will cause this method to be called with the
arguments \code{'XML'} and \code{'text'}.  The default method does
nothing.  Note that if a document starts with a \code{<?xml ...?>}
tag, \code{handle_xml} is called to handle it.
\end{funcdesc}

\begin{funcdesc}{handle_special}{data}
This method is called when a declaration is encountered.  The
\code{data} argument is a string containing the text between the
\samp{<!} and \samp{>} delimiters, but not the delimiters
themselves.  For example, the entity \samp{<!ENTITY text>} will
cause this method to be called with the argument \code{'ENTITY text'}.  The
default method does nothing.  Note that \code{<!DOCTYPE ...>} is
handled separately if it is located at the start of the document.
\end{funcdesc}

\begin{funcdesc}{syntax_error}{message}
This method is called when a syntax error is encountered.  The
\code{message} is a description of what was wrong.  The default method 
raises a \code{RuntimeError} exception.  If this method is overridden, 
it is permissable for it to return.  This method is only called when
the error can be recovered from.  Unrecoverable errors raise a
\code{RuntimeError} without first calling \code{syntax_error}.
\end{funcdesc}

\begin{funcdesc}{unknown_starttag}{tag\, attributes}
This method is called to process an unknown start tag.  It is intended
to be overridden by a derived class; the base class implementation
does nothing.
\end{funcdesc}

\begin{funcdesc}{unknown_endtag}{tag}
This method is called to process an unknown end tag.  It is intended
to be overridden by a derived class; the base class implementation
does nothing.
\end{funcdesc}

\begin{funcdesc}{unknown_charref}{ref}
This method is called to process unresolvable numeric character
references.  It is intended to be overridden by a derived class; the
base class implementation does nothing.
\end{funcdesc}

\begin{funcdesc}{unknown_entityref}{ref}
This method is called to process an unknown entity reference.  It is
intended to be overridden by a derived class; the base class
implementation does nothing.
\end{funcdesc}

Apart from overriding or extending the methods listed above, derived
classes may also define methods and variables of the following form to
define processing of specific tags.  Tag names in the input stream are
case dependent; the \var{tag} occurring in method names must be in the
correct case:

\begin{funcdescni}{start_\var{tag}}{attributes}
This method is called to process an opening tag \var{tag}.  The
\var{attributes} argument has the same meaning as described for
\code{handle_starttag()} above.  In fact, the base implementation of
\code{handle_starttag()} calls this method.
\end{funcdescni}

\begin{funcdescni}{end_\var{tag}}{}
This method is called to process a closing tag \var{tag}.
\end{funcdescni}

\begin{datadescni}{\var{tag}_attributes}
If a class or instance variable \code{\var{tag}_attributes} exists, it 
should be a list or a dictionary.  If a list, the elements of the list 
are the valid attributes for the element \var{tag}; if a dictionary,
the keys are the valid attributes for the element \var{tag}, and the
values the default values of the attributes, or \code{None} if there
is no default.
In addition to the attributes that were present in the tag, the
attribute dictionary that is passed to \code{handle_starttag()} and
\code{unknown_starttag()} contains values for all attributes that have a
default value.
\end{datadescni}

\section{Standard Module \sectcode{formatter}}
\label{module-formatter}
\stmodindex{formatter}


This module supports two interface definitions, each with mulitple
implementations.  The \emph{formatter} interface is used by the
\class{HTMLParser} class of the \module{htmllib} module, and the
\emph{writer} interface is required by the formatter interface.
\withsubitem{(im module htmllib)}{\ttindex{HTMLParser}}

Formatter objects transform an abstract flow of formatting events into
specific output events on writer objects.  Formatters manage several
stack structures to allow various properties of a writer object to be
changed and restored; writers need not be able to handle relative
changes nor any sort of ``change back'' operation.  Specific writer
properties which may be controlled via formatter objects are
horizontal alignment, font, and left margin indentations.  A mechanism
is provided which supports providing arbitrary, non-exclusive style
settings to a writer as well.  Additional interfaces facilitate
formatting events which are not reversible, such as paragraph
separation.

Writer objects encapsulate device interfaces.  Abstract devices, such
as file formats, are supported as well as physical devices.  The
provided implementations all work with abstract devices.  The
interface makes available mechanisms for setting the properties which
formatter objects manage and inserting data into the output.


\subsection{The Formatter Interface}

Interfaces to create formatters are dependent on the specific
formatter class being instantiated.  The interfaces described below
are the required interfaces which all formatters must support once
initialized.

One data element is defined at the module level:


\begin{datadesc}{AS_IS}
Value which can be used in the font specification passed to the
\code{push_font()} method described below, or as the new value to any
other \code{push_\var{property}()} method.  Pushing the \code{AS_IS}
value allows the corresponding \code{pop_\var{property}()} method to
be called without having to track whether the property was changed.
\end{datadesc}

The following attributes are defined for formatter instance objects:


\begin{memberdesc}[formatter]{writer}
The writer instance with which the formatter interacts.
\end{memberdesc}


\begin{methoddesc}[formatter]{end_paragraph}{blanklines}
Close any open paragraphs and insert at least \var{blanklines}
before the next paragraph.
\end{methoddesc}

\begin{methoddesc}[formatter]{add_line_break}{}
Add a hard line break if one does not already exist.  This does not
break the logical paragraph.
\end{methoddesc}

\begin{methoddesc}[formatter]{add_hor_rule}{*args, **kw}
Insert a horizontal rule in the output.  A hard break is inserted if
there is data in the current paragraph, but the logical paragraph is
not broken.  The arguments and keywords are passed on to the writer's
\method{send_line_break()} method.
\end{methoddesc}

\begin{methoddesc}[formatter]{add_flowing_data}{data}
Provide data which should be formatted with collapsed whitespaces.
Whitespace from preceeding and successive calls to
\method{add_flowing_data()} is considered as well when the whitespace
collapse is performed.  The data which is passed to this method is
expected to be word-wrapped by the output device.  Note that any
word-wrapping still must be performed by the writer object due to the
need to rely on device and font information.
\end{methoddesc}

\begin{methoddesc}[formatter]{add_literal_data}{data}
Provide data which should be passed to the writer unchanged.
Whitespace, including newline and tab characters, are considered legal
in the value of \var{data}.  
\end{methoddesc}

\begin{methoddesc}[formatter]{add_label_data}{format, counter}
Insert a label which should be placed to the left of the current left
margin.  This should be used for constructing bulleted or numbered
lists.  If the \var{format} value is a string, it is interpreted as a
format specification for \var{counter}, which should be an integer.
The result of this formatting becomes the value of the label; if
\var{format} is not a string it is used as the label value directly.
The label value is passed as the only argument to the writer's
\method{send_label_data()} method.  Interpretation of non-string label
values is dependent on the associated writer.

Format specifications are strings which, in combination with a counter
value, are used to compute label values.  Each character in the format
string is copied to the label value, with some characters recognized
to indicate a transform on the counter value.  Specifically, the
character \character{1} represents the counter value formatter as an
arabic number, the characters \character{A} and \character{a}
represent alphabetic representations of the counter value in upper and
lower case, respectively, and \character{I} and \character{i}
represent the counter value in Roman numerals, in upper and lower
case.  Note that the alphabetic and roman transforms require that the
counter value be greater than zero.
\end{methoddesc}

\begin{methoddesc}[formatter]{flush_softspace}{}
Send any pending whitespace buffered from a previous call to
\method{add_flowing_data()} to the associated writer object.  This
should be called before any direct manipulation of the writer object.
\end{methoddesc}

\begin{methoddesc}[formatter]{push_alignment}{align}
Push a new alignment setting onto the alignment stack.  This may be
\constant{AS_IS} if no change is desired.  If the alignment value is
changed from the previous setting, the writer's \method{new_alignment()}
method is called with the \var{align} value.
\end{methoddesc}

\begin{methoddesc}[formatter]{pop_alignment}{}
Restore the previous alignment.
\end{methoddesc}

\begin{methoddesc}[formatter]{push_font}{\code{(}size, italic, bold, teletype\code{)}}
Change some or all font properties of the writer object.  Properties
which are not set to \constant{AS_IS} are set to the values passed in
while others are maintained at their current settings.  The writer's
\method{new_font()} method is called with the fully resolved font
specification.
\end{methoddesc}

\begin{methoddesc}[formatter]{pop_font}{}
Restore the previous font.
\end{methoddesc}

\begin{methoddesc}[formatter]{push_margin}{margin}
Increase the number of left margin indentations by one, associating
the logical tag \var{margin} with the new indentation.  The initial
margin level is \code{0}.  Changed values of the logical tag must be
true values; false values other than \constant{AS_IS} are not
sufficient to change the margin.
\end{methoddesc}

\begin{methoddesc}[formatter]{pop_margin}{}
Restore the previous margin.
\end{methoddesc}

\begin{methoddesc}[formatter]{push_style}{*styles}
Push any number of arbitrary style specifications.  All styles are
pushed onto the styles stack in order.  A tuple representing the
entire stack, including \constant{AS_IS} values, is passed to the
writer's \method{new_styles()} method.
\end{methoddesc}

\begin{methoddesc}[formatter]{pop_style}{\optional{n\code{ = 1}}}
Pop the last \var{n} style specifications passed to
\method{push_style()}.  A tuple representing the revised stack,
including \constant{AS_IS} values, is passed to the writer's
\method{new_styles()} method.
\end{methoddesc}

\begin{methoddesc}[formatter]{set_spacing}{spacing}
Set the spacing style for the writer.
\end{methoddesc}

\begin{methoddesc}[formatter]{assert_line_data}{\optional{flag\code{ = 1}}}
Inform the formatter that data has been added to the current paragraph
out-of-band.  This should be used when the writer has been manipulated
directly.  The optional \var{flag} argument can be set to false if
the writer manipulations produced a hard line break at the end of the
output.
\end{methoddesc}


\subsection{Formatter Implementations}

Two implementations of formatter objects are provided by this module.
Most applications may use one of these classes without modification or
subclassing.

\begin{classdesc}{NullFormatter}{\optional{writer}}
A formatter which does nothing.  If \var{writer} is omitted, a
\class{NullWriter} instance is created.  No methods of the writer are
called by \class{NullFormatter} instances.  Implementations should
inherit from this class if implementing a writer interface but don't
need to inherit any implementation.
\end{classdesc}

\begin{classdesc}{AbstractFormatter}{writer}
The standard formatter.  This implementation has demonstrated wide
applicability to many writers, and may be used directly in most
circumstances.  It has been used to implement a full-featured
world-wide web browser.
\end{classdesc}



\subsection{The Writer Interface}

Interfaces to create writers are dependent on the specific writer
class being instantiated.  The interfaces described below are the
required interfaces which all writers must support once initialized.
Note that while most applications can use the
\class{AbstractFormatter} class as a formatter, the writer must
typically be provided by the application.


\begin{methoddesc}[writer]{flush}{}
Flush any buffered output or device control events.
\end{methoddesc}

\begin{methoddesc}[writer]{new_alignment}{align}
Set the alignment style.  The \var{align} value can be any object,
but by convention is a string or \code{None}, where \code{None}
indicates that the writer's ``preferred'' alignment should be used.
Conventional \var{align} values are \code{'left'}, \code{'center'},
\code{'right'}, and \code{'justify'}.
\end{methoddesc}

\begin{methoddesc}[writer]{new_font}{font}
Set the font style.  The value of \var{font} will be \code{None},
indicating that the device's default font should be used, or a tuple
of the form \code{(}\var{size}, \var{italic}, \var{bold},
\var{teletype}\code{)}.  Size will be a string indicating the size of
font that should be used; specific strings and their interpretation
must be defined by the application.  The \var{italic}, \var{bold}, and
\var{teletype} values are boolean indicators specifying which of those
font attributes should be used.
\end{methoddesc}

\begin{methoddesc}[writer]{new_margin}{margin, level}
Set the margin level to the integer \var{level} and the logical tag
to \var{margin}.  Interpretation of the logical tag is at the
writer's discretion; the only restriction on the value of the logical
tag is that it not be a false value for non-zero values of
\var{level}.
\end{methoddesc}

\begin{methoddesc}[writer]{new_spacing}{spacing}
Set the spacing style to \var{spacing}.
\end{methoddesc}

\begin{methoddesc}[writer]{new_styles}{styles}
Set additional styles.  The \var{styles} value is a tuple of
arbitrary values; the value \constant{AS_IS} should be ignored.  The
\var{styles} tuple may be interpreted either as a set or as a stack
depending on the requirements of the application and writer
implementation.
\end{methoddesc}

\begin{methoddesc}[writer]{send_line_break}{}
Break the current line.
\end{methoddesc}

\begin{methoddesc}[writer]{send_paragraph}{blankline}
Produce a paragraph separation of at least \var{blankline} blank
lines, or the equivelent.  The \var{blankline} value will be an
integer.
\end{methoddesc}

\begin{methoddesc}[writer]{send_hor_rule}{*args, **kw}
Display a horizontal rule on the output device.  The arguments to this
method are entirely application- and writer-specific, and should be
interpreted with care.  The method implementation may assume that a
line break has already been issued via \method{send_line_break()}.
\end{methoddesc}

\begin{methoddesc}[writer]{send_flowing_data}{data}
Output character data which may be word-wrapped and re-flowed as
needed.  Within any sequence of calls to this method, the writer may
assume that spans of multiple whitespace characters have been
collapsed to single space characters.
\end{methoddesc}

\begin{methoddesc}[writer]{send_literal_data}{data}
Output character data which has already been formatted
for display.  Generally, this should be interpreted to mean that line
breaks indicated by newline characters should be preserved and no new
line breaks should be introduced.  The data may contain embedded
newline and tab characters, unlike data provided to the
\method{send_formatted_data()} interface.
\end{methoddesc}

\begin{methoddesc}[writer]{send_label_data}{data}
Set \var{data} to the left of the current left margin, if possible.
The value of \var{data} is not restricted; treatment of non-string
values is entirely application- and writer-dependent.  This method
will only be called at the beginning of a line.
\end{methoddesc}


\subsection{Writer Implementations}

Three implementations of the writer object interface are provided as
examples by this module.  Most applications will need to derive new
writer classes from the \class{NullWriter} class.

\begin{classdesc}{NullWriter}{}
A writer which only provides the interface definition; no actions are
taken on any methods.  This should be the base class for all writers
which do not need to inherit any implementation methods.
\end{classdesc}

\begin{classdesc}{AbstractWriter}{}
A writer which can be used in debugging formatters, but not much
else.  Each method simply announces itself by printing its name and
arguments on standard output.
\end{classdesc}

\begin{classdesc}{DumbWriter}{\optional{file\optional{, maxcol\code{ = 72}}}}
Simple writer class which writes output on the file object passed in
as \var{file} or, if \var{file} is omitted, on standard output.  The
output is simply word-wrapped to the number of columns specified by
\var{maxcol}.  This class is suitable for reflowing a sequence of
paragraphs.
\end{classdesc}

\section{\module{rfc822} ---
         Parse RFC 2822 mail headers}

\declaremodule{standard}{rfc822}
\modulesynopsis{Parse \rfc{2822} style mail messages.}

This module defines a class, \class{Message}, which represents an
``email message'' as defined by the Internet standard
\rfc{2822}.\footnote{This module originally conformed to \rfc{822},
hence the name.  Since then, \rfc{2822} has been released as an
update to \rfc{822}.  This module should be considered
\rfc{2822}-conformant, especially in cases where the
syntax or semantics have changed since \rfc{822}.}  Such messages
consist of a collection of message headers, and a message body.  This
module also defines a helper class
\class{AddressList} for parsing \rfc{2822} addresses.  Please refer to
the RFC for information on the specific syntax of \rfc{2822} messages.

The \refmodule{mailbox}\refstmodindex{mailbox} module provides classes 
to read mailboxes produced by various end-user mail programs.

\begin{classdesc}{Message}{file\optional{, seekable}}
A \class{Message} instance is instantiated with an input object as
parameter.  Message relies only on the input object having a
\method{readline()} method; in particular, ordinary file objects
qualify.  Instantiation reads headers from the input object up to a
delimiter line (normally a blank line) and stores them in the
instance.  The message body, following the headers, is not consumed.

This class can work with any input object that supports a
\method{readline()} method.  If the input object has seek and tell
capability, the \method{rewindbody()} method will work; also, illegal
lines will be pushed back onto the input stream.  If the input object
lacks seek but has an \method{unread()} method that can push back a
line of input, \class{Message} will use that to push back illegal
lines.  Thus this class can be used to parse messages coming from a
buffered stream.

The optional \var{seekable} argument is provided as a workaround for
certain stdio libraries in which \cfunction{tell()} discards buffered
data before discovering that the \cfunction{lseek()} system call
doesn't work.  For maximum portability, you should set the seekable
argument to zero to prevent that initial \method{tell()} when passing
in an unseekable object such as a a file object created from a socket
object.

Input lines as read from the file may either be terminated by CR-LF or
by a single linefeed; a terminating CR-LF is replaced by a single
linefeed before the line is stored.

All header matching is done independent of upper or lower case;
e.g.\ \code{\var{m}['From']}, \code{\var{m}['from']} and
\code{\var{m}['FROM']} all yield the same result.
\end{classdesc}

\begin{classdesc}{AddressList}{field}
You may instantiate the \class{AddressList} helper class using a single
string parameter, a comma-separated list of \rfc{2822} addresses to be
parsed.  (The parameter \code{None} yields an empty list.)
\end{classdesc}

\begin{funcdesc}{quote}{str}
Return a new string with backslashes in \var{str} replaced by two
backslashes and double quotes replaced by backslash-double quote.
\end{funcdesc}

\begin{funcdesc}{unquote}{str}
Return a new string which is an \emph{unquoted} version of \var{str}.
If \var{str} ends and begins with double quotes, they are stripped
off.  Likewise if \var{str} ends and begins with angle brackets, they
are stripped off.
\end{funcdesc}

\begin{funcdesc}{parseaddr}{address}
Parse \var{address}, which should be the value of some address-containing
field such as \code{To:} or \code{Cc:}, into its constituent
``realname'' and ``email address'' parts.  Returns a tuple of that
information, unless the parse fails, in which case a 2-tuple of
\code{(None, None)} is returned.
\end{funcdesc}

\begin{funcdesc}{dump_address_pair}{pair}
The inverse of \method{parseaddr()}, this takes a 2-tuple of the form
\code{(realname, email_address)} and returns the string value suitable
for a \code{To:} or \code{Cc:} header.  If the first element of
\var{pair} is false, then the second element is returned unmodified.
\end{funcdesc}

\begin{funcdesc}{parsedate}{date}
Attempts to parse a date according to the rules in \rfc{2822}.
however, some mailers don't follow that format as specified, so
\function{parsedate()} tries to guess correctly in such cases. 
\var{date} is a string containing an \rfc{2822} date, such as 
\code{'Mon, 20 Nov 1995 19:12:08 -0500'}.  If it succeeds in parsing
the date, \function{parsedate()} returns a 9-tuple that can be passed
directly to \function{time.mktime()}; otherwise \code{None} will be
returned.  Note that fields 6, 7, and 8 of the result tuple are not
usable.
\end{funcdesc}

\begin{funcdesc}{parsedate_tz}{date}
Performs the same function as \function{parsedate()}, but returns
either \code{None} or a 10-tuple; the first 9 elements make up a tuple
that can be passed directly to \function{time.mktime()}, and the tenth
is the offset of the date's timezone from UTC (which is the official
term for Greenwich Mean Time).  (Note that the sign of the timezone
offset is the opposite of the sign of the \code{time.timezone}
variable for the same timezone; the latter variable follows the
\POSIX{} standard while this module follows \rfc{2822}.)  If the input
string has no timezone, the last element of the tuple returned is
\code{None}.  Note that fields 6, 7, and 8 of the result tuple are not
usable.
\end{funcdesc}

\begin{funcdesc}{mktime_tz}{tuple}
Turn a 10-tuple as returned by \function{parsedate_tz()} into a UTC
timestamp.  It the timezone item in the tuple is \code{None}, assume
local time.  Minor deficiency: this first interprets the first 8
elements as a local time and then compensates for the timezone
difference; this may yield a slight error around daylight savings time
switch dates.  Not enough to worry about for common use.
\end{funcdesc}


\begin{seealso}
  \seemodule{mailbox}{Classes to read various mailbox formats produced 
                      by end-user mail programs.}
  \seemodule{mimetools}{Subclass of rfc.Message that handles MIME encoded
		        messages.} 
\end{seealso}


\subsection{Message Objects \label{message-objects}}

A \class{Message} instance has the following methods:

\begin{methoddesc}{rewindbody}{}
Seek to the start of the message body.  This only works if the file
object is seekable.
\end{methoddesc}

\begin{methoddesc}{isheader}{line}
Returns a line's canonicalized fieldname (the dictionary key that will
be used to index it) if the line is a legal \rfc{2822} header; otherwise
returns None (implying that parsing should stop here and the line be
pushed back on the input stream).  It is sometimes useful to override
this method in a subclass.
\end{methoddesc}

\begin{methoddesc}{islast}{line}
Return true if the given line is a delimiter on which Message should
stop.  The delimiter line is consumed, and the file object's read
location positioned immediately after it.  By default this method just
checks that the line is blank, but you can override it in a subclass.
\end{methoddesc}

\begin{methoddesc}{iscomment}{line}
Return true if the given line should be ignored entirely, just skipped.
By default this is a stub that always returns false, but you can
override it in a subclass.
\end{methoddesc}

\begin{methoddesc}{getallmatchingheaders}{name}
Return a list of lines consisting of all headers matching
\var{name}, if any.  Each physical line, whether it is a continuation
line or not, is a separate list item.  Return the empty list if no
header matches \var{name}.
\end{methoddesc}

\begin{methoddesc}{getfirstmatchingheader}{name}
Return a list of lines comprising the first header matching
\var{name}, and its continuation line(s), if any.  Return
\code{None} if there is no header matching \var{name}.
\end{methoddesc}

\begin{methoddesc}{getrawheader}{name}
Return a single string consisting of the text after the colon in the
first header matching \var{name}.  This includes leading whitespace,
the trailing linefeed, and internal linefeeds and whitespace if there
any continuation line(s) were present.  Return \code{None} if there is
no header matching \var{name}.
\end{methoddesc}

\begin{methoddesc}{getheader}{name\optional{, default}}
Like \code{getrawheader(\var{name})}, but strip leading and trailing
whitespace.  Internal whitespace is not stripped.  The optional
\var{default} argument can be used to specify a different default to
be returned when there is no header matching \var{name}.
\end{methoddesc}

\begin{methoddesc}{get}{name\optional{, default}}
An alias for \method{getheader()}, to make the interface more compatible 
with regular dictionaries.
\end{methoddesc}

\begin{methoddesc}{getaddr}{name}
Return a pair \code{(\var{full name}, \var{email address})} parsed
from the string returned by \code{getheader(\var{name})}.  If no
header matching \var{name} exists, return \code{(None, None)};
otherwise both the full name and the address are (possibly empty)
strings.

Example: If \var{m}'s first \code{From} header contains the string
\code{'jack@cwi.nl (Jack Jansen)'}, then
\code{m.getaddr('From')} will yield the pair
\code{('Jack Jansen', 'jack@cwi.nl')}.
If the header contained
\code{'Jack Jansen <jack@cwi.nl>'} instead, it would yield the
exact same result.
\end{methoddesc}

\begin{methoddesc}{getaddrlist}{name}
This is similar to \code{getaddr(\var{list})}, but parses a header
containing a list of email addresses (e.g.\ a \code{To} header) and
returns a list of \code{(\var{full name}, \var{email address})} pairs
(even if there was only one address in the header).  If there is no
header matching \var{name}, return an empty list.

If multiple headers exist that match the named header (e.g. if there
are several \code{Cc} headers), all are parsed for addresses.  Any
continuation lines the named headers contain are also parsed.
\end{methoddesc}

\begin{methoddesc}{getdate}{name}
Retrieve a header using \method{getheader()} and parse it into a 9-tuple
compatible with \function{time.mktime()}; note that fields 6, 7, and 8 
are not usable.  If there is no header matching
\var{name}, or it is unparsable, return \code{None}.

Date parsing appears to be a black art, and not all mailers adhere to
the standard.  While it has been tested and found correct on a large
collection of email from many sources, it is still possible that this
function may occasionally yield an incorrect result.
\end{methoddesc}

\begin{methoddesc}{getdate_tz}{name}
Retrieve a header using \method{getheader()} and parse it into a
10-tuple; the first 9 elements will make a tuple compatible with
\function{time.mktime()}, and the 10th is a number giving the offset
of the date's timezone from UTC.  Note that fields 6, 7, and 8 
are not usable.  Similarly to \method{getdate()}, if
there is no header matching \var{name}, or it is unparsable, return
\code{None}. 
\end{methoddesc}

\class{Message} instances also support a limited mapping interface.
In particular: \code{\var{m}[name]} is like
\code{\var{m}.getheader(name)} but raises \exception{KeyError} if
there is no matching header; and \code{len(\var{m})},
\code{\var{m}.get(name\optional{, deafult})},
\code{\var{m}.has_key(name)}, \code{\var{m}.keys()},
\code{\var{m}.values()} \code{\var{m}.items()}, and
\code{\var{m}.setdefault(name\optional{, default})} act as expected,
with the one difference that \method{get()} and \method{setdefault()}
use an empty string as the default value.  \class{Message} instances
also support the mapping writable interface \code{\var{m}[name] =
value} and \code{del \var{m}[name]}.  \class{Message} objects do not
support the \method{clear()}, \method{copy()}, \method{popitem()}, or
\method{update()} methods of the mapping interface.  (Support for
\method{get()} and \method{setdefault()} was only added in Python
2.2.)

Finally, \class{Message} instances have two public instance variables:

\begin{memberdesc}{headers}
A list containing the entire set of header lines, in the order in
which they were read (except that setitem calls may disturb this
order). Each line contains a trailing newline.  The
blank line terminating the headers is not contained in the list.
\end{memberdesc}

\begin{memberdesc}{fp}
The file or file-like object passed at instantiation time.  This can
be used to read the message content.
\end{memberdesc}


\subsection{AddressList Objects \label{addresslist-objects}}

An \class{AddressList} instance has the following methods:

\begin{methoddesc}{__len__}{}
Return the number of addresses in the address list.
\end{methoddesc}

\begin{methoddesc}{__str__}{}
Return a canonicalized string representation of the address list.
Addresses are rendered in "name" <host@domain> form, comma-separated.
\end{methoddesc}

\begin{methoddesc}{__add__}{alist}
Return a new \class{AddressList} instance that contains all addresses
in both \class{AddressList} operands, with duplicates removed (set
union).
\end{methoddesc}

\begin{methoddesc}{__iadd__}{alist}
In-place version of \method{__add__()}; turns this \class{AddressList}
instance into the union of itself and the right-hand instance,
\var{alist}.
\end{methoddesc}

\begin{methoddesc}{__sub__}{alist}
Return a new \class{AddressList} instance that contains every address
in the left-hand \class{AddressList} operand that is not present in
the right-hand address operand (set difference).
\end{methoddesc}

\begin{methoddesc}{__isub__}{alist}
In-place version of \method{__sub__()}, removing addresses in this
list which are also in \var{alist}.
\end{methoddesc}


Finally, \class{AddressList} instances have one public instance variable:

\begin{memberdesc}{addresslist}
A list of tuple string pairs, one per address.  In each member, the
first is the canonicalized name part, the second is the
actual route-address (\character{@}-separated username-host.domain
pair).
\end{memberdesc}

\section{\module{mimetools} ---
         Tools for parsing MIME messages}

\declaremodule{standard}{mimetools}
\modulesynopsis{Tools for parsing MIME-style message bodies.}

\deprecated{2.3}{The \refmodule{email} package should be used in
                 preference to the \module{mimetools} module.  This
                 module is present only to maintain backward
                 compatibility.}

This module defines a subclass of the
\refmodule{rfc822}\refstmodindex{rfc822} module's
\class{Message} class and a number of utility functions that are
useful for the manipulation for MIME multipart or encoded message.

It defines the following items:

\begin{classdesc}{Message}{fp\optional{, seekable}}
Return a new instance of the \class{Message} class.  This is a
subclass of the \class{rfc822.Message} class, with some additional
methods (see below).  The \var{seekable} argument has the same meaning
as for \class{rfc822.Message}.
\end{classdesc}

\begin{funcdesc}{choose_boundary}{}
Return a unique string that has a high likelihood of being usable as a
part boundary.  The string has the form
\code{'\var{hostipaddr}.\var{uid}.\var{pid}.\var{timestamp}.\var{random}'}.
\end{funcdesc}

\begin{funcdesc}{decode}{input, output, encoding}
Read data encoded using the allowed MIME \var{encoding} from open file
object \var{input} and write the decoded data to open file object
\var{output}.  Valid values for \var{encoding} include
\code{'base64'}, \code{'quoted-printable'}, \code{'uuencode'},
\code{'x-uuencode'}, \code{'uue'}, \code{'x-uue'}, \code{'7bit'}, and 
\code{'8bit'}.  Decoding messages encoded in \code{'7bit'} or \code{'8bit'}
has no effect.  The input is simply copied to the output.
\end{funcdesc}

\begin{funcdesc}{encode}{input, output, encoding}
Read data from open file object \var{input} and write it encoded using
the allowed MIME \var{encoding} to open file object \var{output}.
Valid values for \var{encoding} are the same as for \method{decode()}.
\end{funcdesc}

\begin{funcdesc}{copyliteral}{input, output}
Read lines from open file \var{input} until \EOF{} and write them to
open file \var{output}.
\end{funcdesc}

\begin{funcdesc}{copybinary}{input, output}
Read blocks until \EOF{} from open file \var{input} and write them to
open file \var{output}.  The block size is currently fixed at 8192.
\end{funcdesc}


\begin{seealso}
  \seemodule{email}{Comprehensive email handling package; supercedes
                    the \module{mimetools} module.}
  \seemodule{rfc822}{Provides the base class for
                     \class{mimetools.Message}.}
  \seemodule{multifile}{Support for reading files which contain
                        distinct parts, such as MIME data.}
  \seeurl{http://www.cs.uu.nl/wais/html/na-dir/mail/mime-faq/.html}{
          The MIME Frequently Asked Questions document.  For an
          overview of MIME, see the answer to question 1.1 in Part 1
          of this document.}
\end{seealso}


\subsection{Additional Methods of Message Objects
            \label{mimetools-message-objects}}

The \class{Message} class defines the following methods in
addition to the \class{rfc822.Message} methods:

\begin{methoddesc}{getplist}{}
Return the parameter list of the \mailheader{Content-Type} header.
This is a list of strings.  For parameters of the form
\samp{\var{key}=\var{value}}, \var{key} is converted to lower case but
\var{value} is not.  For example, if the message contains the header
\samp{Content-type: text/html; spam=1; Spam=2; Spam} then
\method{getplist()} will return the Python list \code{['spam=1',
'spam=2', 'Spam']}.
\end{methoddesc}

\begin{methoddesc}{getparam}{name}
Return the \var{value} of the first parameter (as returned by
\method{getplist()} of the form \samp{\var{name}=\var{value}} for the
given \var{name}.  If \var{value} is surrounded by quotes of the form
`\code{<}...\code{>}' or `\code{"}...\code{"}', these are removed.
\end{methoddesc}

\begin{methoddesc}{getencoding}{}
Return the encoding specified in the
\mailheader{Content-Transfer-Encoding} message header.  If no such
header exists, return \code{'7bit'}.  The encoding is converted to
lower case.
\end{methoddesc}

\begin{methoddesc}{gettype}{}
Return the message type (of the form \samp{\var{type}/\var{subtype}})
as specified in the \mailheader{Content-Type} header.  If no such
header exists, return \code{'text/plain'}.  The type is converted to
lower case.
\end{methoddesc}

\begin{methoddesc}{getmaintype}{}
Return the main type as specified in the \mailheader{Content-Type}
header.  If no such header exists, return \code{'text'}.  The main
type is converted to lower case.
\end{methoddesc}

\begin{methoddesc}{getsubtype}{}
Return the subtype as specified in the \mailheader{Content-Type}
header.  If no such header exists, return \code{'plain'}.  The subtype
is converted to lower case.
\end{methoddesc}

\section{\module{multifile} ---
         Support for files containing distinct parts}

\declaremodule{standard}{multifile}
\modulesynopsis{Support for reading files which contain distinct
                parts, such as some MIME data.}
\sectionauthor{Eric S. Raymond}{esr@snark.thyrsus.com}


The \class{MultiFile} object enables you to treat sections of a text
file as file-like input objects, with \code{''} being returned by
\method{readline()} when a given delimiter pattern is encountered.  The
defaults of this class are designed to make it useful for parsing
MIME multipart messages, but by subclassing it and overriding methods 
it can be easily adapted for more general use.

\begin{classdesc}{MultiFile}{fp\optional{, seekable}}
Create a multi-file.  You must instantiate this class with an input
object argument for the \class{MultiFile} instance to get lines from,
such as as a file object returned by \function{open()}.

\class{MultiFile} only ever looks at the input object's
\method{readline()}, \method{seek()} and \method{tell()} methods, and
the latter two are only needed if you want random access to the
individual MIME parts. To use \class{MultiFile} on a non-seekable
stream object, set the optional \var{seekable} argument to false; this
will prevent using the input object's \method{seek()} and
\method{tell()} methods.
\end{classdesc}

It will be useful to know that in \class{MultiFile}'s view of the world, text
is composed of three kinds of lines: data, section-dividers, and
end-markers.  MultiFile is designed to support parsing of
messages that may have multiple nested message parts, each with its
own pattern for section-divider and end-marker lines.


\subsection{MultiFile Objects \label{MultiFile-objects}}

A \class{MultiFile} instance has the following methods:

\begin{methoddesc}{push}{str}
Push a boundary string.  When an appropriately decorated version of
this boundary is found as an input line, it will be interpreted as a
section-divider or end-marker.  All subsequent
reads will return the empty string to indicate end-of-file, until a
call to \method{pop()} removes the boundary a or \method{next()} call
reenables it.

It is possible to push more than one boundary.  Encountering the
most-recently-pushed boundary will return EOF; encountering any other
boundary will raise an error.
\end{methoddesc}

\begin{methoddesc}{readline}{str}
Read a line.  If the line is data (not a section-divider or end-marker
or real EOF) return it.  If the line matches the most-recently-stacked
boundary, return \code{''} and set \code{self.last} to 1 or 0 according as
the match is or is not an end-marker.  If the line matches any other
stacked boundary, raise an error.  On encountering end-of-file on the
underlying stream object, the method raises \exception{Error} unless
all boundaries have been popped.
\end{methoddesc}

\begin{methoddesc}{readlines}{str}
Return all lines remaining in this part as a list of strings.
\end{methoddesc}

\begin{methoddesc}{read}{}
Read all lines, up to the next section.  Return them as a single
(multiline) string.  Note that this doesn't take a size argument!
\end{methoddesc}

\begin{methoddesc}{next}{}
Skip lines to the next section (that is, read lines until a
section-divider or end-marker has been consumed).  Return true if
there is such a section, false if an end-marker is seen.  Re-enable
the most-recently-pushed boundary.
\end{methoddesc}

\begin{methoddesc}{pop}{}
Pop a section boundary.  This boundary will no longer be interpreted
as EOF.
\end{methoddesc}

\begin{methoddesc}{seek}{pos\optional{, whence}}
Seek.  Seek indices are relative to the start of the current section.
The \var{pos} and \var{whence} arguments are interpreted as for a file
seek.
\end{methoddesc}

\begin{methoddesc}{tell}{}
Return the file position relative to the start of the current section.
\end{methoddesc}

\begin{methoddesc}{is_data}{str}
Return true if \var{str} is data and false if it might be a section
boundary.  As written, it tests for a prefix other than \code{'-}\code{-'} at
start of line (which all MIME boundaries have) but it is declared so
it can be overridden in derived classes.

Note that this test is used intended as a fast guard for the real
boundary tests; if it always returns false it will merely slow
processing, not cause it to fail.
\end{methoddesc}

\begin{methoddesc}{section_divider}{str}
Turn a boundary into a section-divider line.  By default, this
method prepends \code{'-}\code{-'} (which MIME section boundaries have) but
it is declared so it can be overridden in derived classes.  This
method need not append LF or CR-LF, as comparison with the result
ignores trailing whitespace. 
\end{methoddesc}

\begin{methoddesc}{end_marker}{str}
Turn a boundary string into an end-marker line.  By default, this
method prepends \code{'-}\code{-'} and appends \code{'-}\code{-'} (like a
MIME-multipart end-of-message marker) but it is declared so it can be
be overridden in derived classes.  This method need not append LF or
CR-LF, as comparison with the result ignores trailing whitespace.
\end{methoddesc}

Finally, \class{MultiFile} instances have two public instance variables:

\begin{memberdesc}{level}
Nesting depth of the current part.
\end{memberdesc}

\begin{memberdesc}{last}
True if the last end-of-file was for an end-of-message marker. 
\end{memberdesc}


\subsection{\class{MultiFile} Example \label{multifile-example}}

% This is almost unreadable; should be re-written when someone gets time.

\begin{verbatim}
fp = MultiFile(sys.stdin, 0)
fp.push(outer_boundary)
message1 = fp.readlines()
# We should now be either at real EOF or stopped on a message
# boundary. Re-enable the outer boundary.
fp.next()
# Read another message with the same delimiter
message2 = fp.readlines()
# Re-enable that delimiter again
fp.next()
# Now look for a message subpart with a different boundary
fp.push(inner_boundary)
sub_header = fp.readlines()
# If no exception has been thrown, we're looking at the start of
# the message subpart.  Reset and grab the subpart
fp.next()
sub_body = fp.readlines()
# Got it.  Now pop the inner boundary to re-enable the outer one.
fp.pop()
# Read to next outer boundary
message3 = fp.readlines()
\end{verbatim}

\section{\module{binhex} ---
         Encode and decode binhex4 files}

\declaremodule{standard}{binhex}
\modulesynopsis{Encode and decode files in binhex4 format.}


This module encodes and decodes files in binhex4 format, a format
allowing representation of Macintosh files in \ASCII.  On the Macintosh,
both forks of a file and the finder information are encoded (or
decoded), on other platforms only the data fork is handled.

The \module{binhex} module defines the following functions:

\begin{funcdesc}{binhex}{input, output}
Convert a binary file with filename \var{input} to binhex file
\var{output}. The \var{output} parameter can either be a filename or a
file-like object (any object supporting a \method{write()} and
\method{close()} method).
\end{funcdesc}

\begin{funcdesc}{hexbin}{input\optional{, output}}
Decode a binhex file \var{input}. \var{input} may be a filename or a
file-like object supporting \method{read()} and \method{close()} methods.
The resulting file is written to a file named \var{output}, unless the
argument is omitted in which case the output filename is read from the
binhex file.
\end{funcdesc}

The following exception is also defined:

\begin{excdesc}{Error}
Exception raised when something can't be encoded using the binhex
format (for example, a filename is too long to fit in the filename
field), or when input is not properly encoded binhex data.
\end{excdesc}


\begin{seealso}
  \seemodule{binascii}{Support module containing \ASCII-to-binary
                       and binary-to-\ASCII{} conversions.}
\end{seealso}


\subsection{Notes \label{binhex-notes}}

There is an alternative, more powerful interface to the coder and
decoder, see the source for details.

If you code or decode textfiles on non-Macintosh platforms they will
still use the Macintosh newline convention (carriage-return as end of
line).

As of this writing, \function{hexbin()} appears to not work in all
cases.

\section{Standard Module \module{uu}}
\label{module-uu}
\stmodindex{uu}

This module encodes and decodes files in uuencode format, allowing
arbitrary binary data to be transferred over ascii-only connections.
Wherever a file argument is expected, the methods accept a file-like
object.  For backwards compatibility, a string containing a pathname
is also accepted, and the corresponding file will be opened for
reading and writing; the pathname \code{'-'} is understood to mean the
standard input or output.  However, this interface is deprecated; it's
better for the caller to open the file itself, and be sure that, when
required, the mode is \code{'rb'} or \code{'wb'} on Windows or DOS.

This code was contributed by Lance Ellinghouse, and modified by Jack
Jansen.
\index{Jansen, Jack}
\index{Ellinghouse, Lance}

The \module{uu} module defines the following functions:

\begin{funcdesc}{encode}{in_file, out_file\optional{, name\optional{, mode}}}
Uuencode file \var{in_file} into file \var{out_file}.  The uuencoded
file will have the header specifying \var{name} and \var{mode} as the
defaults for the results of decoding the file. The default defaults
are taken from \var{in_file}, or \code{'-'} and \code{0666}
respectively. 
\end{funcdesc}

\begin{funcdesc}{decode}{in_file\optional{, out_file\optional{, mode}}}
This call decodes uuencoded file \var{in_file} placing the result on
file \var{out_file}. If \var{out_file} is a pathname the \var{mode} is
also set. Defaults for \var{out_file} and \var{mode} are taken from
the uuencode header.
\end{funcdesc}

\section{Standard Module \sectcode{binhex}}
\label{module-binhex}
\stmodindex{binhex}

This module encodes and decodes files in binhex4 format, a format
allowing representation of Macintosh files in ASCII. On the macintosh,
both forks of a file and the finder information are encoded (or
decoded), on other platforms only the data fork is handled.

The \code{binhex} module defines the following functions:

\setindexsubitem{(in module binhex)}

\begin{funcdesc}{binhex}{input\, output}
Convert a binary file with filename \var{input} to binhex file
\var{output}. The \var{output} parameter can either be a filename or a
file-like object (any object supporting a \var{write} and \var{close}
method).
\end{funcdesc}

\begin{funcdesc}{hexbin}{input\optional{\, output}}
Decode a binhex file \var{input}. \var{input} may be a filename or a
file-like object supporting \var{read} and \var{close} methods.
The resulting file is written to a file named \var{output}, unless the
argument is empty in which case the output filename is read from the
binhex file.
\end{funcdesc}

\subsection{Notes}
There is an alternative, more powerful interface to the coder and
decoder, see the source for details.

If you code or decode textfiles on non-Macintosh platforms they will
still use the macintosh newline convention (carriage-return as end of
line).

As of this writing, \var{hexbin} appears to not work in all cases.

\section{Standard Module \sectcode{uu}}
\stmodindex{uu}

This module encodes and decodes files in uuencode format, allowing
arbitrary binary data to be transferred over ascii-only connections.
Wherever a file argument is expected, the methods accept a file-like
object.  For backwards compatibility, a string containing a pathname
is also accepted, and the corresponding file will be opened for
reading and writing; the pathname \code{'-'} is understood to mean the
standard input or output.  However, this interface is deprecated; it's
better for the caller to open the file itself, and be sure that, when
required, the mode is \code{'rb'} or \code{'wb'} on Windows or DOS.

This code was contributed by Lance Ellinghouse, and modified by Jack
Jansen.

The \code{uu} module defines the following functions:

\setindexsubitem{(in module uu)}

\begin{funcdesc}{encode}{in_file\, out_file\optional{\, name\, mode}}
Uuencode file \var{in_file} into file \var{out_file}.  The uuencoded
file will have the header specifying \var{name} and \var{mode} as the
defaults for the results of decoding the file. The default defaults
are taken from \var{in_file}, or \code{'-'} and \code{0666}
respectively. 
\end{funcdesc}

\begin{funcdesc}{decode}{in_file\optional{\, out_file\, mode}}
This call decodes uuencoded file \var{in_file} placing the result on
file \var{out_file}. If \var{out_file} is a pathname the \var{mode} is
also set. Defaults for \var{out_file} and \var{mode} are taken from
the uuencode header.
\end{funcdesc}

\section{Built-in Module \sectcode{binascii}}	% If implemented in C
\bimodindex{binascii}

The binascii module contains a number of methods to convert between
binary and various ascii-encoded binary representations. Normally, you
will not use these modules directly but use wrapper modules like
\var{uu} or \var{hexbin} in stead, this module solely exists because
bit-manipuation of large amounts of data is slow in python.

The \code{binascii} module defines the following functions:

\setindexsubitem{(in module binascii)}

\begin{funcdesc}{a2b_uu}{string}
Convert a single line of uuencoded data back to binary and return the
binary data. Lines normally contain 45 (binary) bytes, except for the
last line. Line data may be followed by whitespace.
\end{funcdesc}

\begin{funcdesc}{b2a_uu}{data}
Convert binary data to a line of ascii characters, the return value is
the converted line, including a newline char. The length of \var{data}
should be at most 45.
\end{funcdesc}

\begin{funcdesc}{a2b_base64}{string}
Convert a block of base64 data back to binary and return the
binary data. More than one line may be passed at a time.
\end{funcdesc}

\begin{funcdesc}{b2a_base64}{data}
Convert binary data to a line of ascii characters in base64 coding.
The return value is the converted line, including a newline char.
The length of \var{data} should be at most 57 to adhere to the base64
standard.
\end{funcdesc}

\begin{funcdesc}{a2b_hqx}{string}
Convert binhex4 formatted ascii data to binary, without doing
rle-decompression. The string should contain a complete number of
binary bytes, or (in case of the last portion of the binhex4 data)
have the remaining bits zero.
\end{funcdesc}

\begin{funcdesc}{rledecode_hqx}{data}
Perform RLE-decompression on the data, as per the binhex4
standard. The algorithm uses \code{0x90} after a byte as a repeat
indicator, followed by a count. A count of \code{0} specifies a byte
value of \code{0x90}. The routine returns the decompressed data,
unless data input data ends in an orphaned repeat indicator, in which
case the \var{Incomplete} exception is raised.
\end{funcdesc}

\begin{funcdesc}{rlecode_hqx}{data}
Perform binhex4 style RLE-compression on \var{data} and return the
result.
\end{funcdesc}

\begin{funcdesc}{b2a_hqx}{data}
Perform hexbin4 binary-to-ascii translation and return the resulting
string. The argument should already be rle-coded, and have a length
divisible by 3 (except possibly the last fragment).
\end{funcdesc}

\begin{funcdesc}{crc_hqx}{data, crc}
Compute the binhex4 crc value of \var{data}, starting with an initial
\var{crc} and returning the result.
\end{funcdesc}
 
\begin{excdesc}{Error}
Exception raised on errors. These are usually programming errors.
\end{excdesc}

\begin{excdesc}{Incomplete}
Exception raised on incomplete data. These are usually not programming
errors, but handled by reading a little more data and trying again.
\end{excdesc}

\section{Standard Module \module{xdrlib}}
\declaremodule{standard}{xdrlib}

\modulesynopsis{The External Data Representation Standard as described
in \rfc{1014}.}

\index{XDR}
\index{External Data Representation}


The \module{xdrlib} module supports the External Data Representation
Standard as described in \rfc{1014}, written by Sun Microsystems,
Inc. June 1987.  It supports most of the data types described in the
RFC.

The \module{xdrlib} module defines two classes, one for packing
variables into XDR representation, and another for unpacking from XDR
representation.  There are also two exception classes.

\begin{classdesc}{Packer}{}
\class{Packer} is the class for packing data into XDR representation.
The \class{Packer} class is instantiated with no arguments.
\end{classdesc}

\begin{classdesc}{Unpacker}{data}
\code{Unpacker} is the complementary class which unpacks XDR data
values from a string buffer.  The input buffer is given as
\var{data}.
\end{classdesc}


\subsection{Packer Objects}
\label{xdr-packer-objects}

\class{Packer} instances have the following methods:

\begin{methoddesc}[Packer]{get_buffer}{}
Returns the current pack buffer as a string.
\end{methoddesc}

\begin{methoddesc}[Packer]{reset}{}
Resets the pack buffer to the empty string.
\end{methoddesc}

In general, you can pack any of the most common XDR data types by
calling the appropriate \code{pack_\var{type}()} method.  Each method
takes a single argument, the value to pack.  The following simple data
type packing methods are supported: \method{pack_uint()},
\method{pack_int()}, \method{pack_enum()}, \method{pack_bool()},
\method{pack_uhyper()}, and \method{pack_hyper()}.

\begin{methoddesc}[Packer]{pack_float}{value}
Packs the single-precision floating point number \var{value}.
\end{methoddesc}

\begin{methoddesc}[Packer]{pack_double}{value}
Packs the double-precision floating point number \var{value}.
\end{methoddesc}

The following methods support packing strings, bytes, and opaque data:

\begin{methoddesc}[Packer]{pack_fstring}{n, s}
Packs a fixed length string, \var{s}.  \var{n} is the length of the
string but it is \emph{not} packed into the data buffer.  The string
is padded with null bytes if necessary to guaranteed 4 byte alignment.
\end{methoddesc}

\begin{methoddesc}[Packer]{pack_fopaque}{n, data}
Packs a fixed length opaque data stream, similarly to
\method{pack_fstring()}.
\end{methoddesc}

\begin{methoddesc}[Packer]{pack_string}{s}
Packs a variable length string, \var{s}.  The length of the string is
first packed as an unsigned integer, then the string data is packed
with \method{pack_fstring()}.
\end{methoddesc}

\begin{methoddesc}[Packer]{pack_opaque}{data}
Packs a variable length opaque data string, similarly to
\method{pack_string()}.
\end{methoddesc}

\begin{methoddesc}[Packer]{pack_bytes}{bytes}
Packs a variable length byte stream, similarly to \method{pack_string()}.
\end{methoddesc}

The following methods support packing arrays and lists:

\begin{methoddesc}[Packer]{pack_list}{list, pack_item}
Packs a \var{list} of homogeneous items.  This method is useful for
lists with an indeterminate size; i.e. the size is not available until
the entire list has been walked.  For each item in the list, an
unsigned integer \code{1} is packed first, followed by the data value
from the list.  \var{pack_item} is the function that is called to pack
the individual item.  At the end of the list, an unsigned integer
\code{0} is packed.
\end{methoddesc}

\begin{methoddesc}[Packer]{pack_farray}{n, array, pack_item}
Packs a fixed length list (\var{array}) of homogeneous items.  \var{n}
is the length of the list; it is \emph{not} packed into the buffer,
but a \exception{ValueError} exception is raised if
\code{len(\var{array})} is not equal to \var{n}.  As above,
\var{pack_item} is the function used to pack each element.
\end{methoddesc}

\begin{methoddesc}[Packer]{pack_array}{list, pack_item}
Packs a variable length \var{list} of homogeneous items.  First, the
length of the list is packed as an unsigned integer, then each element
is packed as in \method{pack_farray()} above.
\end{methoddesc}


\subsection{Unpacker Objects}
\label{xdr-unpacker-objects}

The \class{Unpacker} class offers the following methods:

\begin{methoddesc}[Unpacker]{reset}{data}
Resets the string buffer with the given \var{data}.
\end{methoddesc}

\begin{methoddesc}[Unpacker]{get_position}{}
Returns the current unpack position in the data buffer.
\end{methoddesc}

\begin{methoddesc}[Unpacker]{set_position}{position}
Sets the data buffer unpack position to \var{position}.  You should be
careful about using \method{get_position()} and \method{set_position()}.
\end{methoddesc}

\begin{methoddesc}[Unpacker]{get_buffer}{}
Returns the current unpack data buffer as a string.
\end{methoddesc}

\begin{methoddesc}[Unpacker]{done}{}
Indicates unpack completion.  Raises an \exception{Error} exception
if all of the data has not been unpacked.
\end{methoddesc}

In addition, every data type that can be packed with a \class{Packer},
can be unpacked with an \class{Unpacker}.  Unpacking methods are of the
form \code{unpack_\var{type}()}, and take no arguments.  They return the
unpacked object.

\begin{methoddesc}[Unpacker]{unpack_float}{}
Unpacks a single-precision floating point number.
\end{methoddesc}

\begin{methoddesc}[Unpacker]{unpack_double}{}
Unpacks a double-precision floating point number, similarly to
\method{unpack_float()}.
\end{methoddesc}

In addition, the following methods unpack strings, bytes, and opaque
data:

\begin{methoddesc}[Unpacker]{unpack_fstring}{n}
Unpacks and returns a fixed length string.  \var{n} is the number of
characters expected.  Padding with null bytes to guaranteed 4 byte
alignment is assumed.
\end{methoddesc}

\begin{methoddesc}[Unpacker]{unpack_fopaque}{n}
Unpacks and returns a fixed length opaque data stream, similarly to
\method{unpack_fstring()}.
\end{methoddesc}

\begin{methoddesc}[Unpacker]{unpack_string}{}
Unpacks and returns a variable length string.  The length of the
string is first unpacked as an unsigned integer, then the string data
is unpacked with \method{unpack_fstring()}.
\end{methoddesc}

\begin{methoddesc}[Unpacker]{unpack_opaque}{}
Unpacks and returns a variable length opaque data string, similarly to
\method{unpack_string()}.
\end{methoddesc}

\begin{methoddesc}[Unpacker]{unpack_bytes}{}
Unpacks and returns a variable length byte stream, similarly to
\method{unpack_string()}.
\end{methoddesc}

The following methods support unpacking arrays and lists:

\begin{methoddesc}[Unpacker]{unpack_list}{unpack_item}
Unpacks and returns a list of homogeneous items.  The list is unpacked
one element at a time
by first unpacking an unsigned integer flag.  If the flag is \code{1},
then the item is unpacked and appended to the list.  A flag of
\code{0} indicates the end of the list.  \var{unpack_item} is the
function that is called to unpack the items.
\end{methoddesc}

\begin{methoddesc}[Unpacker]{unpack_farray}{n, unpack_item}
Unpacks and returns (as a list) a fixed length array of homogeneous
items.  \var{n} is number of list elements to expect in the buffer.
As above, \var{unpack_item} is the function used to unpack each element.
\end{methoddesc}

\begin{methoddesc}[Unpacker]{unpack_array}{unpack_item}
Unpacks and returns a variable length \var{list} of homogeneous items.
First, the length of the list is unpacked as an unsigned integer, then
each element is unpacked as in \method{unpack_farray()} above.
\end{methoddesc}


\subsection{Exceptions}
\nodename{xdr-exceptions}

Exceptions in this module are coded as class instances:

\begin{excdesc}{Error}
The base exception class.  \exception{Error} has a single public data
member \member{msg} containing the description of the error.
\end{excdesc}

\begin{excdesc}{ConversionError}
Class derived from \exception{Error}.  Contains no additional instance
variables.
\end{excdesc}

Here is an example of how you would catch one of these exceptions:

\begin{verbatim}
import xdrlib
p = xdrlib.Packer()
try:
    p.pack_double(8.01)
except xdrlib.ConversionError, instance:
    print 'packing the double failed:', instance.msg
\end{verbatim}

\section{Standard Module \sectcode{mailcap}}
\label{module-mailcap}
\stmodindex{mailcap}
\renewcommand{\indexsubitem}{(in module mailcap)}

Mailcap files are used to configure how MIME-aware applications such
as mail readers and Web browsers react to files with different MIME
types. (The name ``mailcap'' is derived from the phrase ``mail
capability''.)  For example, a mailcap file might contain a line like
\samp{video/mpeg; xmpeg \%s}.  Then, if the user encounters an email
message or Web document with the MIME type video/mpeg, \code{\%s} will be
replaced by a filename (usually one belonging to a temporary file) and
the xmpeg program can be automatically started to view the file.

The mailcap format is documented in RFC 1524, ``A User Agent
Configuration Mechanism For Multimedia Mail Format Information'', but
is not an Internet standard.  However, mailcap files are supported on
most Unix systems.

\begin{funcdesc}{findmatch}{caps\, MIMEtype\, key\, filename\, plist}
Return a 2-tuple; the first element is a string containing the command
line to be executed
(which can be passed to \code{os.system()}), and the second element is
the mailcap entry for a given MIME type.  If no matching MIME
type can be found, \code{(None, None)} is returned.

\var{key} is the name of the field desired, which represents the type of
activity to be performed; the default value is 'view', since in the
most common case you simply want to view the body of the MIME-typed
data.  Other possible values might be 'compose' and 'edit', if you
wanted to create a new body of the given MIME type or alter the
existing body data.  See RFC1524 for a complete list of these fields.

\var{filename} is the filename to be substituted for \%s in the
command line; the default value is
\file{/dev/null} which is almost certainly not what you want, so
usually you'll override it by specifying a filename.

\var{plist} can be a list containing named parameters; the default
value is simply an empty list.  Each entry in the list must be a
string containing the parameter name, an equals sign (\code{=}), and the
parameter's value.  Mailcap entries can contain 
named parameters like \code{\%\{foo\}}, which will be replaced by the
value of the parameter named 'foo'.  For example, if the command line
\samp{showpartial \%\{id\} \%\{number\} \%\{total\}}
was in a mailcap file, and \var{plist} was set to \code{['id=1',
'number=2', 'total=3']}, the resulting command line would be 
\code{"showpartial 1 2 3"}.  

In a mailcap file, the "test" field can optionally be specified to
test some external condition (e.g., the machine architecture, or the
window system in use) to determine whether or not the mailcap line
applies.  \code{findmatch()} will automatically check such conditions
and skip the entry if the check fails.
\end{funcdesc}

\begin{funcdesc}{getcaps}{}
Returns a dictionary mapping MIME types to a list of mailcap file
entries. This dictionary must be passed to the \code{findmatch()}
function.  An entry is stored as a list of dictionaries, but it
shouldn't be necessary to know the details of this representation.

The information is derived from all of the mailcap files found on the
system. Settings in the user's mailcap file \file{\$HOME/.mailcap}
will override settings in the system mailcap files
\file{/etc/mailcap}, \file{/usr/etc/mailcap}, and
\file{/usr/local/etc/mailcap}.
\end{funcdesc}

An example usage:
\bcode\begin{verbatim}
>>> import mailcap
>>> d=mailcap.getcaps()
>>> mailcap.findmatch(d, 'video/mpeg', filename='/tmp/tmp1223')
('xmpeg /tmp/tmp1223', {'view': 'xmpeg %s'})
\end{verbatim}\ecode

\section{\module{mimetypes} ---
         Map filenames to MIME types}

\declaremodule{standard}{mimetypes}
\modulesynopsis{Mapping of filename extensions to MIME types.}
\sectionauthor{Fred L. Drake, Jr.}{fdrake@acm.org}


\indexii{MIME}{content type}

The \module{mimetypes} module converts between a filename or URL and
the MIME type associated with the filename extension.  Conversions are
provided from filename to MIME type and from MIME type to filename
extension; encodings are not supported for the latter conversion.

The module provides one class and a number of convenience functions.
The functions are the normal interface to this module, but some
applications may be interested in the class as well.

The functions described below provide the primary interface for this
module.  If the module has not been initialized, they will call
\function{init()} if they rely on the information \function{init()}
sets up.


\begin{funcdesc}{guess_type}{filename\optional{, strict}}
Guess the type of a file based on its filename or URL, given by
\var{filename}.  The return value is a tuple \code{(\var{type},
\var{encoding})} where \var{type} is \code{None} if the type can't be
guessed (missing or unknown suffix) or a string of the form
\code{'\var{type}/\var{subtype}'}, usable for a MIME
\mailheader{content-type} header\indexii{MIME}{headers}.

\var{encoding} is \code{None} for no encoding or the name of the
program used to encode (e.g. \program{compress} or \program{gzip}).
The encoding is suitable for use as a \mailheader{Content-Encoding}
header, \emph{not} as a \mailheader{Content-Transfer-Encoding} header.
The mappings are table driven.  Encoding suffixes are case sensitive;
type suffixes are first tried case sensitively, then case
insensitively.

Optional \var{strict} is a flag specifying whether the list of known
MIME types is limited to only the official types \ulink{registered
with IANA}{http://www.isi.edu/in-notes/iana/assignments/media-types}
are recognized.  When \var{strict} is true (the default), only the
IANA types are supported; when \var{strict} is false, some additional
non-standard but commonly used MIME types are also recognized.
\end{funcdesc}

\begin{funcdesc}{guess_all_extensions}{type\optional{, strict}}
Guess the extensions for a file based on its MIME type, given by
\var{type}.
The return value is a list of strings giving all possible filename extensions,
including the leading dot (\character{.}).  The extensions are not guaranteed
to have been associated with any particular data stream, but would be mapped
to the MIME type \var{type} by \function{guess_type()}.

Optional \var{strict} has the same meaning as with the
\function{guess_type()} function.
\end{funcdesc}


\begin{funcdesc}{guess_extension}{type\optional{, strict}}
Guess the extension for a file based on its MIME type, given by
\var{type}.
The return value is a string giving a filename extension, including the
leading dot (\character{.}).  The extension is not guaranteed to have been
associated with any particular data stream, but would be mapped to the 
MIME type \var{type} by \function{guess_type()}.  If no extension can
be guessed for \var{type}, \code{None} is returned.

Optional \var{strict} has the same meaning as with the
\function{guess_type()} function.
\end{funcdesc}


Some additional functions and data items are available for controlling
the behavior of the module.


\begin{funcdesc}{init}{\optional{files}}
Initialize the internal data structures.  If given, \var{files} must
be a sequence of file names which should be used to augment the
default type map.  If omitted, the file names to use are taken from
\constant{knownfiles}.  Each file named in \var{files} or
\constant{knownfiles} takes precedence over those named before it.
Calling \function{init()} repeatedly is allowed.
\end{funcdesc}

\begin{funcdesc}{read_mime_types}{filename}
Load the type map given in the file \var{filename}, if it exists.  The 
type map is returned as a dictionary mapping filename extensions,
including the leading dot (\character{.}), to strings of the form
\code{'\var{type}/\var{subtype}'}.  If the file \var{filename} does
not exist or cannot be read, \code{None} is returned.
\end{funcdesc}


\begin{funcdesc}{add_type}{type, ext\optional{, strict}}
Add a mapping from the mimetype \var{type} to the extension \var{ext}.
When the extension is already known, the new type will replace the old
one. When the type is already known the extension will be added
to the list of known extensions.

When \var{strict} is the mapping will added to the official
MIME types, otherwise to the non-standard ones.
\end{funcdesc}


\begin{datadesc}{inited}
Flag indicating whether or not the global data structures have been
initialized.  This is set to true by \function{init()}.
\end{datadesc}

\begin{datadesc}{knownfiles}
List of type map file names commonly installed.  These files are
typically named \file{mime.types} and are installed in different
locations by different packages.\index{file!mime.types}
\end{datadesc}

\begin{datadesc}{suffix_map}
Dictionary mapping suffixes to suffixes.  This is used to allow
recognition of encoded files for which the encoding and the type are
indicated by the same extension.  For example, the \file{.tgz}
extension is mapped to \file{.tar.gz} to allow the encoding and type
to be recognized separately.
\end{datadesc}

\begin{datadesc}{encodings_map}
Dictionary mapping filename extensions to encoding types.
\end{datadesc}

\begin{datadesc}{types_map}
Dictionary mapping filename extensions to MIME types.
\end{datadesc}

\begin{datadesc}{common_types}
Dictionary mapping filename extensions to non-standard, but commonly
found MIME types.
\end{datadesc}


The \class{MimeTypes} class may be useful for applications which may
want more than one MIME-type database:

\begin{classdesc}{MimeTypes}{\optional{filenames}}
  This class represents a MIME-types database.  By default, it
  provides access to the same database as the rest of this module.
  The initial database is a copy of that provided by the module, and
  may be extended by loading additional \file{mime.types}-style files
  into the database using the \method{read()} or \method{readfp()}
  methods.  The mapping dictionaries may also be cleared before
  loading additional data if the default data is not desired.

  The optional \var{filenames} parameter can be used to cause
  additional files to be loaded ``on top'' of the default database.

  \versionadded{2.2}
\end{classdesc}

An example usage of the module:

\begin{verbatim}
>>> import mimetypes
>>> mimetypes.init()
>>> mimetypes.knownfiles
['/etc/mime.types', '/etc/httpd/mime.types', ... ]
>>> mimetypes.suffix_map['.tgz']
'.tar.gz'
>>> mimetypes.encodings_map['.gz']
'gzip'
>>> mimetypes.types_map['.tgz']
'application/x-tar-gz'
\end{verbatim}

\subsection{MimeTypes Objects \label{mimetypes-objects}}

\class{MimeTypes} instances provide an interface which is very like
that of the \refmodule{mimetypes} module.

\begin{memberdesc}[MimeTypes]{suffix_map}
  Dictionary mapping suffixes to suffixes.  This is used to allow
  recognition of encoded files for which the encoding and the type are
  indicated by the same extension.  For example, the \file{.tgz}
  extension is mapped to \file{.tar.gz} to allow the encoding and type
  to be recognized separately.  This is initially a copy of the global
  \code{suffix_map} defined in the module.
\end{memberdesc}

\begin{memberdesc}[MimeTypes]{encodings_map}
  Dictionary mapping filename extensions to encoding types.  This is
  initially a copy of the global \code{encodings_map} defined in the
  module.
\end{memberdesc}

\begin{memberdesc}[MimeTypes]{types_map}
  Dictionary mapping filename extensions to MIME types.  This is
  initially a copy of the global \code{types_map} defined in the
  module.
\end{memberdesc}

\begin{memberdesc}[MimeTypes]{common_types}
  Dictionary mapping filename extensions to non-standard, but commonly
  found MIME types.  This is initially a copy of the global
  \code{common_types} defined in the module.
\end{memberdesc}

\begin{methoddesc}[MimeTypes]{guess_extension}{type\optional{, strict}}
  Similar to the \function{guess_extension()} function, using the
  tables stored as part of the object.
\end{methoddesc}

\begin{methoddesc}[MimeTypes]{guess_type}{url\optional{, strict}}
  Similar to the \function{guess_type()} function, using the tables
  stored as part of the object.
\end{methoddesc}

\begin{methoddesc}[MimeTypes]{read}{path}
  Load MIME information from a file named \var{path}.  This uses
  \method{readfp()} to parse the file.
\end{methoddesc}

\begin{methoddesc}[MimeTypes]{readfp}{file}
  Load MIME type information from an open file.  The file must have
  the format of the standard \file{mime.types} files.
\end{methoddesc}

\section{\module{base64} ---
         Encode and decode MIME base64 data}

\declaremodule{standard}{base64}
\modulesynopsis{Encode and decode files using the MIME base64 data.}


\indexii{base64}{encoding}
\index{MIME!base64 encoding}

This module performs base64 encoding and decoding of arbitrary binary
strings into text strings that can be safely emailed or posted.  The
encoding scheme is defined in \rfc{1421} (``Privacy Enhancement for
Internet Electronic Mail: Part I: Message Encryption and
Authentication Procedures'', section 4.3.2.4, ``Step 4: Printable
Encoding'') and is used for MIME email and
various other Internet-related applications; it is not the same as the
output produced by the \program{uuencode} program.  For example, the
string \code{'www.python.org'} is encoded as the string
\code{'d3d3LnB5dGhvbi5vcmc=\e n'}.  


\begin{funcdesc}{decode}{input, output}
Decode the contents of the \var{input} file and write the resulting
binary data to the \var{output} file.
\var{input} and \var{output} must either be file objects or objects that
mimic the file object interface. \var{input} will be read until
\code{\var{input}.read()} returns an empty string.
\end{funcdesc}

\begin{funcdesc}{decodestring}{s}
Decode the string \var{s}, which must contain one or more lines of
base64 encoded data, and return a string containing the resulting
binary data.
\end{funcdesc}

\begin{funcdesc}{encode}{input, output}
Encode the contents of the \var{input} file and write the resulting
base64 encoded data to the \var{output} file.
\var{input} and \var{output} must either be file objects or objects that
mimic the file object interface. \var{input} will be read until
\code{\var{input}.read()} returns an empty string.
\end{funcdesc}

\begin{funcdesc}{encodestring}{s}
Encode the string \var{s}, which can contain arbitrary binary data,
and return a string containing one or more lines of
base64 encoded data.
\end{funcdesc}


\begin{seealso}
  \seemodule{binascii}{support module containing \ASCII{}-to-binary
                       and binary-to-\ASCII{} conversions}
\end{seealso}

\section{Standard Module \sectcode{quopri}}
\label{module-quopri}
\stmodindex{quopri}

This module performs quoted-printable transport encoding and decoding,
as defined in \rfc{1521}: ``MIME (Multipurpose Internet Mail Extensions)
Part One''.  The quoted-printable encoding is designed for data where
there are relatively few nonprintable characters; the base-64 encoding
scheme available via the \code{base64} module is more compact if there
are many such characters, as when sending a graphics file.
\indexii{quoted printable}{encoding}
\index{MIME!quoted-printable encoding}

\setindexsubitem{(in module quopri)}

\begin{funcdesc}{decode}{input, output}
Decode the contents of the \var{input} file and write the resulting
decoded binary data to the \var{output} file.
\var{input} and \var{output} must either be file objects or objects that
mimic the file object interface. \var{input} will be read until
\code{\var{input}.read()} returns an empty string.
\end{funcdesc}

\begin{funcdesc}{encode}{input, output, quotetabs}
Encode the contents of the \var{input} file and write the resulting
quoted-printable data to the \var{output} file.
\var{input} and \var{output} must either be file objects or objects that
mimic the file object interface. \var{input} will be read until
\code{\var{input}.read()} returns an empty string.
\end{funcdesc}




\section{\module{mailbox} ---
          Manipulate mailboxes in various formats}

\declaremodule{}{mailbox}
\moduleauthor{Gregory K.~Johnson}{gkj@gregorykjohnson.com}
\sectionauthor{Gregory K.~Johnson}{gkj@gregorykjohnson.com}
\modulesynopsis{Manipulate mailboxes in various formats}


This module defines two classes, \class{Mailbox} and \class{Message}, for
accessing and manipulating on-disk mailboxes and the messages they contain.
\class{Mailbox} offers a dictionary-like mapping from keys to messages.
\class{Message} extends the \module{email.Message} module's \class{Message}
class with format-specific state and behavior. Supported mailbox formats are
Maildir, mbox, MH, Babyl, and MMDF.

\begin{seealso}
    \seemodule{email}{Represent and manipulate messages.}
\end{seealso}

\subsection{\class{Mailbox} objects}
\label{mailbox-objects}

\begin{classdesc*}{Mailbox}
A mailbox, which may be inspected and modified.
\end{classdesc*}

The \class{Mailbox} interface is dictionary-like, with small keys
corresponding to messages. Keys are issued by the \class{Mailbox} instance
with which they will be used and are only meaningful to that \class{Mailbox}
instance. A key continues to identify a message even if the corresponding
message is modified, such as by replacing it with another message. Messages may
be added to a \class{Mailbox} instance using the set-like method
\method{add()} and removed using a \code{del} statement or the set-like methods
\method{remove()} and \method{discard()}.

\class{Mailbox} interface semantics differ from dictionary semantics in some
noteworthy ways. Each time a message is requested, a new representation
(typically a \class{Message} instance) is generated, based upon the current
state of the mailbox. Similarly, when a message is added to a \class{Mailbox}
instance, the provided message representation's contents are copied. In neither
case is a reference to the message representation kept by the \class{Mailbox}
instance.

The default \class{Mailbox} iterator iterates over message representations, not
keys as the default dictionary iterator does. Moreover, modification of a
mailbox during iteration is safe and well-defined. Messages added to the
mailbox after an iterator is created will not be seen by the iterator. Messages
removed from the mailbox before the iterator yields them will be silently
skipped, though using a key from an iterator may result in a
\exception{KeyError} exception if the corresponding message is subsequently
removed.

\class{Mailbox} itself is intended to define an interface and to be inherited
from by format-specific subclasses but is not intended to be instantiated.
Instead, you should instantiate a subclass.

\class{Mailbox} instances have the following methods:

\begin{methoddesc}{add}{message}
Add \var{message} to the mailbox and return the key that has been assigned to
it.

Parameter \var{message} may be a \class{Message} instance, an
\class{email.Message.Message} instance, a string, or a file-like object (which
should be open in text mode). If \var{message} is an instance of the
appropriate format-specific \class{Message} subclass (e.g., if it's an
\class{mboxMessage} instance and this is an \class{mbox} instance), its
format-specific information is used. Otherwise, reasonable defaults for
format-specific information are used.
\end{methoddesc}

\begin{methoddesc}{remove}{key}
\methodline{__delitem__}{key}
\methodline{discard}{key}
Delete the message corresponding to \var{key} from the mailbox.

If no such message exists, a \exception{KeyError} exception is raised if the
method was called as \method{remove()} or \method{__delitem__()} but no
exception is raised if the method was called as \method{discard()}. The
behavior of \method{discard()} may be preferred if the underlying mailbox
format supports concurrent modification by other processes.
\end{methoddesc}

\begin{methoddesc}{__setitem__}{key, message}
Replace the message corresponding to \var{key} with \var{message}. Raise a
\exception{KeyError} exception if no message already corresponds to \var{key}.

As with \method{add()}, parameter \var{message} may be a \class{Message}
instance, an \class{email.Message.Message} instance, a string, or a file-like
object (which should be open in text mode). If \var{message} is an instance of
the appropriate format-specific \class{Message} subclass (e.g., if it's an
\class{mboxMessage} instance and this is an \class{mbox} instance), its
format-specific information is used. Otherwise, the format-specific information
of the message that currently corresponds to \var{key} is left unchanged. 
\end{methoddesc}

\begin{methoddesc}{iterkeys}{}
\methodline{keys}{}
Return an iterator over all keys if called as \method{iterkeys()} or return a
list of keys if called as \method{keys()}.
\end{methoddesc}

\begin{methoddesc}{itervalues}{}
\methodline{__iter__}{}
\methodline{values}{}
Return an iterator over representations of all messages if called as
\method{itervalues()} or \method{__iter__()} or return a list of such
representations if called as \method{values()}. The messages are represented as
instances of the appropriate format-specific \class{Message} subclass unless a
custom message factory was specified when the \class{Mailbox} instance was
initialized. \note{The behavior of \method{__iter__()} is unlike that of
dictionaries, which iterate over keys.}
\end{methoddesc}

\begin{methoddesc}{iteritems}{}
\methodline{items}{}
Return an iterator over (\var{key}, \var{message}) pairs, where \var{key} is a
key and \var{message} is a message representation, if called as
\method{iteritems()} or return a list of such pairs if called as
\method{items()}. The messages are represented as instances of the appropriate
format-specific \class{Message} subclass unless a custom message factory was
specified when the \class{Mailbox} instance was initialized.
\end{methoddesc}

\begin{methoddesc}{get}{key\optional{, default=None}}
\methodline{__getitem__}{key}
Return a representation of the message corresponding to \var{key}. If no such
message exists, \var{default} is returned if the method was called as
\method{get()} and a \exception{KeyError} exception is raised if the method was
called as \method{__getitem__()}. The message is represented as an instance of
the appropriate format-specific \class{Message} subclass unless a custom
message factory was specified when the \class{Mailbox} instance was
initialized.
\end{methoddesc}

\begin{methoddesc}{get_message}{key}
Return a representation of the message corresponding to \var{key} as an
instance of the appropriate format-specific \class{Message} subclass, or raise
a \exception{KeyError} exception if no such message exists.
\end{methoddesc}

\begin{methoddesc}{get_string}{key}
Return a string representation of the message corresponding to \var{key}, or
raise a \exception{KeyError} exception if no such message exists.
\end{methoddesc}

\begin{methoddesc}{get_file}{key}
Return a file-like representation of the message corresponding to \var{key},
or raise a \exception{KeyError} exception if no such message exists. The
file-like object behaves as if open in binary mode. This file should be closed
once it is no longer needed.

\note{Unlike other representations of messages, file-like representations are
not necessarily independent of the \class{Mailbox} instance that created them
or of the underlying mailbox. More specific documentation is provided by each
subclass.}
\end{methoddesc}

\begin{methoddesc}{has_key}{key}
\methodline{__contains__}{key}
Return \code{True} if \var{key} corresponds to a message, \code{False}
otherwise.
\end{methoddesc}

\begin{methoddesc}{__len__}{}
Return a count of messages in the mailbox.
\end{methoddesc}

\begin{methoddesc}{clear}{}
Delete all messages from the mailbox.
\end{methoddesc}

\begin{methoddesc}{pop}{key\optional{, default}}
Return a representation of the message corresponding to \var{key} and delete
the message. If no such message exists, return \var{default} if it was supplied
or else raise a \exception{KeyError} exception. The message is represented as
an instance of the appropriate format-specific \class{Message} subclass unless
a custom message factory was specified when the \class{Mailbox} instance was
initialized.
\end{methoddesc}

\begin{methoddesc}{popitem}{}
Return an arbitrary (\var{key}, \var{message}) pair, where \var{key} is a key
and \var{message} is a message representation, and delete the corresponding
message. If the mailbox is empty, raise a \exception{KeyError} exception. The
message is represented as an instance of the appropriate format-specific
\class{Message} subclass unless a custom message factory was specified when the
\class{Mailbox} instance was initialized.
\end{methoddesc}

\begin{methoddesc}{update}{arg}
Parameter \var{arg} should be a \var{key}-to-\var{message} mapping or an
iterable of (\var{key}, \var{message}) pairs. Updates the mailbox so that, for
each given \var{key} and \var{message}, the message corresponding to \var{key}
is set to \var{message} as if by using \method{__setitem__()}. As with
\method{__setitem__()}, each \var{key} must already correspond to a message in
the mailbox or else a \exception{KeyError} exception will be raised, so in
general it is incorrect for \var{arg} to be a \class{Mailbox} instance.
\note{Unlike with dictionaries, keyword arguments are not supported.}
\end{methoddesc}

\begin{methoddesc}{flush}{}
Write any pending changes to the filesystem. For some \class{Mailbox}
subclasses, changes are always written immediately and this method does
nothing.
\end{methoddesc}

\begin{methoddesc}{lock}{}
Acquire an exclusive advisory lock on the mailbox so that other processes know
not to modify it. An \exception{ExternalClashError} is raised if the lock is
not available. The particular locking mechanisms used depend upon the mailbox
format.
\end{methoddesc}

\begin{methoddesc}{unlock}{}
Release the lock on the mailbox, if any.
\end{methoddesc}

\begin{methoddesc}{close}{}
Flush the mailbox, unlock it if necessary, and close any open files. For some
\class{Mailbox} subclasses, this method does nothing.
\end{methoddesc}


\subsubsection{\class{Maildir}}
\label{mailbox-maildir}

\begin{classdesc}{Maildir}{dirname\optional{, factory=rfc822.Message\optional{,
create=True}}}
A subclass of \class{Mailbox} for mailboxes in Maildir format. Parameter
\var{factory} is a callable object that accepts a file-like message
representation (which behaves as if opened in binary mode) and returns a custom
representation. If \var{factory} is \code{None}, \class{MaildirMessage} is used
as the default message representation. If \var{create} is \code{True}, the
mailbox is created if it does not exist.

It is for historical reasons that \var{factory} defaults to
\class{rfc822.Message} and that \var{dirname} is named as such rather than
\var{path}. For a \class{Maildir} instance that behaves like instances of other
\class{Mailbox} subclasses, set \var{factory} to \code{None}.
\end{classdesc}

Maildir is a directory-based mailbox format invented for the qmail mail
transfer agent and now widely supported by other programs. Messages in a
Maildir mailbox are stored in separate files within a common directory
structure. This design allows Maildir mailboxes to be accessed and modified by
multiple unrelated programs without data corruption, so file locking is
unnecessary.

Maildir mailboxes contain three subdirectories, namely: \file{tmp}, \file{new},
and \file{cur}. Messages are created momentarily in the \file{tmp} subdirectory
and then moved to the \file{new} subdirectory to finalize delivery. A mail user
agent may subsequently move the message to the \file{cur} subdirectory and
store information about the state of the message in a special "info" section
appended to its file name.

Folders of the style introduced by the Courier mail transfer agent are also
supported. Any subdirectory of the main mailbox is considered a folder if
\character{.} is the first character in its name. Folder names are represented
by \class{Maildir} without the leading \character{.}. Each folder is itself a
Maildir mailbox but should not contain other folders. Instead, a logical
nesting is indicated using \character{.} to delimit levels, e.g.,
"Archived.2005.07".

\begin{notice}
The Maildir specification requires the use of a colon (\character{:}) in
certain message file names. However, some operating systems do not permit this
character in file names, If you wish to use a Maildir-like format on such an
operating system, you should specify another character to use instead. The
exclamation point (\character{!}) is a popular choice. For example:
\begin{verbatim}
import mailbox
mailbox.Maildir.colon = '!'
\end{verbatim}
The \member{colon} attribute may also be set on a per-instance basis.
\end{notice}

\class{Maildir} instances have all of the methods of \class{Mailbox} in
addition to the following:

\begin{methoddesc}{list_folders}{}
Return a list of the names of all folders.
\end{methoddesc}

\begin{methoddesc}{get_folder}{folder}
Return a \class{Maildir} instance representing the folder whose name is
\var{folder}. A \exception{NoSuchMailboxError} exception is raised if the
folder does not exist.
\end{methoddesc}

\begin{methoddesc}{add_folder}{folder}
Create a folder whose name is \var{folder} and return a \class{Maildir}
instance representing it.
\end{methoddesc}

\begin{methoddesc}{remove_folder}{folder}
Delete the folder whose name is \var{folder}. If the folder contains any
messages, a \exception{NotEmptyError} exception will be raised and the folder
will not be deleted.
\end{methoddesc}

\begin{methoddesc}{clean}{}
Delete temporary files from the mailbox that have not been accessed in the
last 36 hours. The Maildir specification says that mail-reading programs
should do this occasionally.
\end{methoddesc}

Some \class{Mailbox} methods implemented by \class{Maildir} deserve special
remarks:

\begin{methoddesc}{add}{message}
\methodline[Maildir]{__setitem__}{key, message}
\methodline[Maildir]{update}{arg}
\warning{These methods generate unique file names based upon the current
process ID. When using multiple threads, undetected name clashes may occur and
cause corruption of the mailbox unless threads are coordinated to avoid using
these methods to manipulate the same mailbox simultaneously.}
\end{methoddesc}

\begin{methoddesc}{flush}{}
All changes to Maildir mailboxes are immediately applied, so this method does
nothing.
\end{methoddesc}

\begin{methoddesc}{lock}{}
\methodline{unlock}{}
Maildir mailboxes do not support (or require) locking, so these methods do
nothing. 
\end{methoddesc}

\begin{methoddesc}{close}{}
\class{Maildir} instances do not keep any open files and the underlying
mailboxes do not support locking, so this method does nothing.
\end{methoddesc}

\begin{methoddesc}{get_file}{key}
Depending upon the host platform, it may not be possible to modify or remove
the underlying message while the returned file remains open.
\end{methoddesc}

\begin{seealso}
    \seelink{http://www.qmail.org/man/man5/maildir.html}{maildir man page from
    qmail}{The original specification of the format.}
    \seelink{http://cr.yp.to/proto/maildir.html}{Using maildir format}{Notes
    on Maildir by its inventor. Includes an updated name-creation scheme and
    details on "info" semantics.}
    \seelink{http://www.courier-mta.org/?maildir.html}{maildir man page from
    Courier}{Another specification of the format. Describes a common extension
    for supporting folders.}
\end{seealso}

\subsubsection{\class{mbox}}
\label{mailbox-mbox}

\begin{classdesc}{mbox}{path\optional{, factory=None\optional{, create=True}}}
A subclass of \class{Mailbox} for mailboxes in mbox format. Parameter
\var{factory} is a callable object that accepts a file-like message
representation (which behaves as if opened in binary mode) and returns a custom
representation. If \var{factory} is \code{None}, \class{mboxMessage} is used as
the default message representation. If \var{create} is \code{True}, the mailbox
is created if it does not exist.
\end{classdesc}

The mbox format is the classic format for storing mail on \UNIX{} systems. All
messages in an mbox mailbox are stored in a single file with the beginning of
each message indicated by a line whose first five characters are "From~".

Several variations of the mbox format exist to address perceived shortcomings
in the original. In the interest of compatibility, \class{mbox} implements the
original format, which is sometimes referred to as \dfn{mboxo}. This means that
the \mailheader{Content-Length} header, if present, is ignored and that any
occurrences of "From~" at the beginning of a line in a message body are
transformed to ">From~" when storing the message, although occurences of
">From~" are not transformed to "From~" when reading the message.

Some \class{Mailbox} methods implemented by \class{mbox} deserve special
remarks:

\begin{methoddesc}{get_file}{key}
Using the file after calling \method{flush()} or \method{close()} on the
\class{mbox} instance may yield unpredictable results or raise an exception.
\end{methoddesc}

\begin{methoddesc}{lock}{}
\methodline{unlock}{}
Three locking mechanisms are used---dot locking and, if available, the
\cfunction{flock()} and \cfunction{lockf()} system calls.
\end{methoddesc}

\begin{seealso}
    \seelink{http://www.qmail.org/man/man5/mbox.html}{mbox man page from
    qmail}{A specification of the format and its variations.}
    \seelink{http://www.tin.org/bin/man.cgi?section=5\&topic=mbox}{mbox man
    page from tin}{Another specification of the format, with details on
    locking.}
    \seelink{http://home.netscape.com/eng/mozilla/2.0/relnotes/demo/content-length.html}
    {Configuring Netscape Mail on \UNIX{}: Why The Content-Length Format is
    Bad}{An argument for using the original mbox format rather than a
    variation.}
    \seelink{http://homepages.tesco.net./\tilde{}J.deBoynePollard/FGA/mail-mbox-formats.html}
    {"mbox" is a family of several mutually incompatible mailbox formats}{A
    history of mbox variations.}
\end{seealso}

\subsubsection{\class{MH}}
\label{mailbox-mh}

\begin{classdesc}{MH}{path\optional{, factory=None\optional{, create=True}}}
A subclass of \class{Mailbox} for mailboxes in MH format. Parameter
\var{factory} is a callable object that accepts a file-like message
representation (which behaves as if opened in binary mode) and returns a custom
representation. If \var{factory} is \code{None}, \class{MHMessage} is used as
the default message representation. If \var{create} is \code{True}, the mailbox
is created if it does not exist.
\end{classdesc}

MH is a directory-based mailbox format invented for the MH Message Handling
System, a mail user agent. Each message in an MH mailbox resides in its own
file. An MH mailbox may contain other MH mailboxes (called \dfn{folders}) in
addition to messages. Folders may be nested indefinitely. MH mailboxes also
support \dfn{sequences}, which are named lists used to logically group messages
without moving them to sub-folders. Sequences are defined in a file called
\file{.mh_sequences} in each folder.

The \class{MH} class manipulates MH mailboxes, but it does not attempt to
emulate all of \program{mh}'s behaviors. In particular, it does not modify and
is not affected by the \file{context} or \file{.mh_profile} files that are used
by \program{mh} to store its state and configuration.

\class{MH} instances have all of the methods of \class{Mailbox} in addition to
the following:

\begin{methoddesc}{list_folders}{}
Return a list of the names of all folders.
\end{methoddesc}

\begin{methoddesc}{get_folder}{folder}
Return an \class{MH} instance representing the folder whose name is
\var{folder}. A \exception{NoSuchMailboxError} exception is raised if the
folder does not exist.
\end{methoddesc}

\begin{methoddesc}{add_folder}{folder}
Create a folder whose name is \var{folder} and return an \class{MH} instance
representing it.
\end{methoddesc}

\begin{methoddesc}{remove_folder}{folder}
Delete the folder whose name is \var{folder}. If the folder contains any
messages, a \exception{NotEmptyError} exception will be raised and the folder
will not be deleted.
\end{methoddesc}

\begin{methoddesc}{get_sequences}{}
Return a dictionary of sequence names mapped to key lists. If there are no
sequences, the empty dictionary is returned.
\end{methoddesc}

\begin{methoddesc}{set_sequences}{sequences}
Re-define the sequences that exist in the mailbox based upon \var{sequences}, a
dictionary of names mapped to key lists, like returned by
\method{get_sequences()}.
\end{methoddesc}

\begin{methoddesc}{pack}{}
Rename messages in the mailbox as necessary to eliminate gaps in numbering.
Entries in the sequences list are updated correspondingly. \note{Already-issued
keys are invalidated by this operation and should not be subsequently used.}
\end{methoddesc}

Some \class{Mailbox} methods implemented by \class{MH} deserve special remarks:

\begin{methoddesc}{remove}{key}
\methodline{__delitem__}{key}
\methodline{discard}{key}
These methods immediately delete the message. The MH convention of marking a
message for deletion by prepending a comma to its name is not used.
\end{methoddesc}

\begin{methoddesc}{lock}{}
\methodline{unlock}{}
Three locking mechanisms are used---dot locking and, if available, the
\cfunction{flock()} and \cfunction{lockf()} system calls. For MH mailboxes,
locking the mailbox means locking the \file{.mh_sequences} file and, only for
the duration of any operations that affect them, locking individual message
files.
\end{methoddesc}

\begin{methoddesc}{get_file}{key}
Depending upon the host platform, it may not be possible to remove the
underlying message while the returned file remains open.
\end{methoddesc}

\begin{methoddesc}{flush}{}
All changes to MH mailboxes are immediately applied, so this method does
nothing.
\end{methoddesc}

\begin{methoddesc}{close}{}
\class{MH} instances do not keep any open files, so this method is equivelant
to \method{unlock()}.
\end{methoddesc}

\begin{seealso}
\seelink{http://www.nongnu.org/nmh/}{nmh - Message Handling System}{Home page
of \program{nmh}, an updated version of the original \program{mh}.}
\seelink{http://www.ics.uci.edu/\tilde{}mh/book/}{MH \& nmh: Email for Users \&
Programmers}{A GPL-licensed book on \program{mh} and \program{nmh}, with some
information on the mailbox format.}
\end{seealso}

\subsubsection{\class{Babyl}}
\label{mailbox-babyl}

\begin{classdesc}{Babyl}{path\optional{, factory=None\optional{, create=True}}}
A subclass of \class{Mailbox} for mailboxes in Babyl format. Parameter
\var{factory} is a callable object that accepts a file-like message
representation (which behaves as if opened in binary mode) and returns a custom
representation. If \var{factory} is \code{None}, \class{BabylMessage} is used
as the default message representation. If \var{create} is \code{True}, the
mailbox is created if it does not exist.
\end{classdesc}

Babyl is a single-file mailbox format used by the Rmail mail user agent
included with Emacs. The beginning of a message is indicated by a line
containing the two characters Control-Underscore
(\character{\textbackslash037}) and Control-L (\character{\textbackslash014}).
The end of a message is indicated by the start of the next message or, in the
case of the last message, a line containing a Control-Underscore
(\character{\textbackslash037}) character.

Messages in a Babyl mailbox have two sets of headers, original headers and
so-called visible headers. Visible headers are typically a subset of the
original headers that have been reformatted or abridged to be more attractive.
Each message in a Babyl mailbox also has an accompanying list of \dfn{labels},
or short strings that record extra information about the message, and a list of
all user-defined labels found in the mailbox is kept in the Babyl options
section.

\class{Babyl} instances have all of the methods of \class{Mailbox} in addition
to the following:

\begin{methoddesc}{get_labels}{}
Return a list of the names of all user-defined labels used in the mailbox.
\note{The actual messages are inspected to determine which labels exist in the
mailbox rather than consulting the list of labels in the Babyl options section,
but the Babyl section is updated whenever the mailbox is modified.}
\end{methoddesc}

Some \class{Mailbox} methods implemented by \class{Babyl} deserve special
remarks:

\begin{methoddesc}{get_file}{key}
In Babyl mailboxes, the headers of a message are not stored contiguously with
the body of the message. To generate a file-like representation, the headers
and body are copied together into a \class{StringIO} instance (from the
\module{StringIO} module), which has an API identical to that of a file. As a
result, the file-like object is truly independent of the underlying mailbox but
does not save memory compared to a string representation.
\end{methoddesc}

\begin{methoddesc}{lock}{}
\methodline{unlock}{}
Three locking mechanisms are used---dot locking and, if available, the
\cfunction{flock()} and \cfunction{lockf()} system calls.
\end{methoddesc}

\begin{seealso}
\seelink{http://quimby.gnus.org/notes/BABYL}{Format of Version 5 Babyl Files}{A
specification of the Babyl format.}
\seelink{http://www.gnu.org/software/emacs/manual/html_node/Rmail.html}{Reading
Mail with Rmail}{The Rmail manual, with some information on Babyl semantics.}
\end{seealso}

\subsubsection{\class{MMDF}}
\label{mailbox-mmdf}

\begin{classdesc}{MMDF}{path\optional{, factory=None\optional{, create=True}}}
A subclass of \class{Mailbox} for mailboxes in MMDF format. Parameter
\var{factory} is a callable object that accepts a file-like message
representation (which behaves as if opened in binary mode) and returns a custom
representation. If \var{factory} is \code{None}, \class{MMDFMessage} is used as
the default message representation. If \var{create} is \code{True}, the mailbox
is created if it does not exist.
\end{classdesc}

MMDF is a single-file mailbox format invented for the Multichannel Memorandum
Distribution Facility, a mail transfer agent. Each message is in the same form
as an mbox message but is bracketed before and after by lines containing four
Control-A (\character{\textbackslash001}) characters. As with the mbox format,
the beginning of each message is indicated by a line whose first five
characters are "From~", but additional occurrences of "From~" are not
transformed to ">From~" when storing messages because the extra message
separator lines prevent mistaking such occurrences for the starts of subsequent
messages.

Some \class{Mailbox} methods implemented by \class{MMDF} deserve special
remarks:

\begin{methoddesc}{get_file}{key}
Using the file after calling \method{flush()} or \method{close()} on the
\class{MMDF} instance may yield unpredictable results or raise an exception.
\end{methoddesc}

\begin{methoddesc}{lock}{}
\methodline{unlock}{}
Three locking mechanisms are used---dot locking and, if available, the
\cfunction{flock()} and \cfunction{lockf()} system calls.
\end{methoddesc}

\begin{seealso}
\seelink{http://www.tin.org/bin/man.cgi?section=5\&topic=mmdf}{mmdf man page
from tin}{A specification of MMDF format from the documentation of tin, a
newsreader.}
\seelink{http://en.wikipedia.org/wiki/MMDF}{MMDF}{A Wikipedia article
describing the Multichannel Memorandum Distribution Facility.}
\end{seealso}

\subsection{\class{Message} objects}
\label{mailbox-message-objects}

\begin{classdesc}{Message}{\optional{message}}
A subclass of the \module{email.Message} module's \class{Message}. Subclasses
of \class{mailbox.Message} add mailbox-format-specific state and behavior.

If \var{message} is omitted, the new instance is created in a default, empty
state. If \var{message} is an \class{email.Message.Message} instance, its
contents are copied; furthermore, any format-specific information is converted
insofar as possible if \var{message} is a \class{Message} instance. If
\var{message} is a string or a file, it should contain an \rfc{2822}-compliant
message, which is read and parsed.
\end{classdesc}

The format-specific state and behaviors offered by subclasses vary, but in
general it is only the properties that are not specific to a particular mailbox
that are supported (although presumably the properties are specific to a
particular mailbox format). For example, file offsets for single-file mailbox
formats and file names for directory-based mailbox formats are not retained,
because they are only applicable to the original mailbox. But state such as
whether a message has been read by the user or marked as important is retained,
because it applies to the message itself.

There is no requirement that \class{Message} instances be used to represent
messages retrieved using \class{Mailbox} instances. In some situations, the
time and memory required to generate \class{Message} representations might not
not acceptable. For such situations, \class{Mailbox} instances also offer
string and file-like representations, and a custom message factory may be
specified when a \class{Mailbox} instance is initialized. 

\subsubsection{\class{MaildirMessage}}
\label{mailbox-maildirmessage}

\begin{classdesc}{MaildirMessage}{\optional{message}}
A message with Maildir-specific behaviors. Parameter \var{message}
has the same meaning as with the \class{Message} constructor.
\end{classdesc}

Typically, a mail user agent application moves all of the messages in the
\file{new} subdirectory to the \file{cur} subdirectory after the first time the
user opens and closes the mailbox, recording that the messages are old whether
or not they've actually been read. Each message in \file{cur} has an "info"
section added to its file name to store information about its state. (Some mail
readers may also add an "info" section to messages in \file{new}.) The "info"
section may take one of two forms: it may contain "2," followed by a list of
standardized flags (e.g., "2,FR") or it may contain "1," followed by so-called
experimental information. Standard flags for Maildir messages are as follows:

\begin{tableiii}{l|l|l}{textrm}{Flag}{Meaning}{Explanation}
\lineiii{D}{Draft}{Under composition}
\lineiii{F}{Flagged}{Marked as important}
\lineiii{P}{Passed}{Forwarded, resent, or bounced}
\lineiii{R}{Replied}{Replied to}
\lineiii{S}{Seen}{Read}
\lineiii{T}{Trashed}{Marked for subsequent deletion}
\end{tableiii}

\class{MaildirMessage} instances offer the following methods:

\begin{methoddesc}{get_subdir}{}
Return either "new" (if the message should be stored in the \file{new}
subdirectory) or "cur" (if the message should be stored in the \file{cur}
subdirectory). \note{A message is typically moved from \file{new} to \file{cur}
after its mailbox has been accessed, whether or not the message is has been
read. A message \code{msg} has been read if \code{"S" not in msg.get_flags()}
is \code{True}.}
\end{methoddesc}

\begin{methoddesc}{set_subdir}{subdir}
Set the subdirectory the message should be stored in. Parameter \var{subdir}
must be either "new" or "cur".
\end{methoddesc}

\begin{methoddesc}{get_flags}{}
Return a string specifying the flags that are currently set. If the message
complies with the standard Maildir format, the result is the concatenation in
alphabetical order of zero or one occurrence of each of \character{D},
\character{F}, \character{P}, \character{R}, \character{S}, and \character{T}.
The empty string is returned if no flags are set or if "info" contains
experimental semantics.
\end{methoddesc}

\begin{methoddesc}{set_flags}{flags}
Set the flags specified by \var{flags} and unset all others.
\end{methoddesc}

\begin{methoddesc}{add_flag}{flag}
Set the flag(s) specified by \var{flag} without changing other flags. To add
more than one flag at a time, \var{flag} may be a string of more than one
character. The current "info" is overwritten whether or not it contains
experimental information rather than
flags.
\end{methoddesc}

\begin{methoddesc}{remove_flag}{flag}
Unset the flag(s) specified by \var{flag} without changing other flags. To
remove more than one flag at a time, \var{flag} maybe a string of more than one
character. If "info" contains experimental information rather than flags, the
current "info" is not modified.
\end{methoddesc}

\begin{methoddesc}{get_date}{}
Return the delivery date of the message as a floating-point number representing
seconds since the epoch.
\end{methoddesc}

\begin{methoddesc}{set_date}{date}
Set the delivery date of the message to \var{date}, a floating-point number
representing seconds since the epoch.
\end{methoddesc}

\begin{methoddesc}{get_info}{}
Return a string containing the "info" for a message. This is useful for
accessing and modifying "info" that is experimental (i.e., not a list of
flags).
\end{methoddesc}

\begin{methoddesc}{set_info}{info}
Set "info" to \var{info}, which should be a string.
\end{methoddesc}

When a \class{MaildirMessage} instance is created based upon an
\class{mboxMessage} or \class{MMDFMessage} instance, the \mailheader{Status}
and \mailheader{X-Status} headers are omitted and the following conversions
take place:

\begin{tableii}{l|l}{textrm}
    {Resulting state}{\class{mboxMessage} or \class{MMDFMessage} state}
\lineii{"cur" subdirectory}{O flag}
\lineii{F flag}{F flag}
\lineii{R flag}{A flag}
\lineii{S flag}{R flag}
\lineii{T flag}{D flag}
\end{tableii}

When a \class{MaildirMessage} instance is created based upon an
\class{MHMessage} instance, the following conversions take place:

\begin{tableii}{l|l}{textrm}
    {Resulting state}{\class{MHMessage} state}
\lineii{"cur" subdirectory}{"unseen" sequence}
\lineii{"cur" subdirectory and S flag}{no "unseen" sequence}
\lineii{F flag}{"flagged" sequence}
\lineii{R flag}{"replied" sequence}
\end{tableii}

When a \class{MaildirMessage} instance is created based upon a
\class{BabylMessage} instance, the following conversions take place:

\begin{tableii}{l|l}{textrm}
    {Resulting state}{\class{BabylMessage} state}
\lineii{"cur" subdirectory}{"unseen" label}
\lineii{"cur" subdirectory and S flag}{no "unseen" label}
\lineii{P flag}{"forwarded" or "resent" label}
\lineii{R flag}{"answered" label}
\lineii{T flag}{"deleted" label}
\end{tableii}

\subsubsection{\class{mboxMessage}}
\label{mailbox-mboxmessage}

\begin{classdesc}{mboxMessage}{\optional{message}}
A message with mbox-specific behaviors. Parameter \var{message} has the same
meaning as with the \class{Message} constructor.
\end{classdesc}

Messages in an mbox mailbox are stored together in a single file. The sender's
envelope address and the time of delivery are typically stored in a line
beginning with "From~" that is used to indicate the start of a message, though
there is considerable variation in the exact format of this data among mbox
implementations. Flags that indicate the state of the message, such as whether
it has been read or marked as important, are typically stored in
\mailheader{Status} and \mailheader{X-Status} headers.

Conventional flags for mbox messages are as follows:

\begin{tableiii}{l|l|l}{textrm}{Flag}{Meaning}{Explanation}
\lineiii{R}{Read}{Read}
\lineiii{O}{Old}{Previously detected by MUA}
\lineiii{D}{Deleted}{Marked for subsequent deletion}
\lineiii{F}{Flagged}{Marked as important}
\lineiii{A}{Answered}{Replied to}
\end{tableiii}

The "R" and "O" flags are stored in the \mailheader{Status} header, and the
"D", "F", and "A" flags are stored in the \mailheader{X-Status} header. The
flags and headers typically appear in the order mentioned.

\class{mboxMessage} instances offer the following methods:

\begin{methoddesc}{get_from}{}
Return a string representing the "From~" line that marks the start of the
message in an mbox mailbox. The leading "From~" and the trailing newline are
excluded.
\end{methoddesc}

\begin{methoddesc}{set_from}{from_\optional{, time_=None}}
Set the "From~" line to \var{from_}, which should be specified without a
leading "From~" or trailing newline. For convenience, \var{time_} may be
specified and will be formatted appropriately and appended to \var{from_}. If
\var{time_} is specified, it should be a \class{struct_time} instance, a tuple
suitable for passing to \method{time.strftime()}, or \code{True} (to use
\method{time.gmtime()}).
\end{methoddesc}

\begin{methoddesc}{get_flags}{}
Return a string specifying the flags that are currently set. If the message
complies with the conventional format, the result is the concatenation in the
following order of zero or one occurrence of each of \character{R},
\character{O}, \character{D}, \character{F}, and \character{A}.
\end{methoddesc}

\begin{methoddesc}{set_flags}{flags}
Set the flags specified by \var{flags} and unset all others. Parameter
\var{flags} should be the concatenation in any order of zero or more
occurrences of each of \character{R}, \character{O}, \character{D},
\character{F}, and \character{A}.
\end{methoddesc}

\begin{methoddesc}{add_flag}{flag}
Set the flag(s) specified by \var{flag} without changing other flags. To add
more than one flag at a time, \var{flag} may be a string of more than one
character.
\end{methoddesc}

\begin{methoddesc}{remove_flag}{flag}
Unset the flag(s) specified by \var{flag} without changing other flags. To
remove more than one flag at a time, \var{flag} maybe a string of more than one
character.
\end{methoddesc}

When an \class{mboxMessage} instance is created based upon a
\class{MaildirMessage} instance, a "From~" line is generated based upon the
\class{MaildirMessage} instance's delivery date, and the following conversions
take place:

\begin{tableii}{l|l}{textrm}
    {Resulting state}{\class{MaildirMessage} state}
\lineii{R flag}{S flag}
\lineii{O flag}{"cur" subdirectory}
\lineii{D flag}{T flag}
\lineii{F flag}{F flag}
\lineii{A flag}{R flag}
\end{tableii}

When an \class{mboxMessage} instance is created based upon an \class{MHMessage}
instance, the following conversions take place:

\begin{tableii}{l|l}{textrm}
    {Resulting state}{\class{MHMessage} state}
\lineii{R flag and O flag}{no "unseen" sequence}
\lineii{O flag}{"unseen" sequence}
\lineii{F flag}{"flagged" sequence}
\lineii{A flag}{"replied" sequence}
\end{tableii}

When an \class{mboxMessage} instance is created based upon a
\class{BabylMessage} instance, the following conversions take place:

\begin{tableii}{l|l}{textrm}
    {Resulting state}{\class{BabylMessage} state}
\lineii{R flag and O flag}{no "unseen" label}
\lineii{O flag}{"unseen" label}
\lineii{D flag}{"deleted" label}
\lineii{A flag}{"answered" label}
\end{tableii}

When a \class{Message} instance is created based upon an \class{MMDFMessage}
instance, the "From~" line is copied and all flags directly correspond:

\begin{tableii}{l|l}{textrm}
    {Resulting state}{\class{MMDFMessage} state}
\lineii{R flag}{R flag}
\lineii{O flag}{O flag}
\lineii{D flag}{D flag}
\lineii{F flag}{F flag}
\lineii{A flag}{A flag}
\end{tableii}

\subsubsection{\class{MHMessage}}
\label{mailbox-mhmessage}

\begin{classdesc}{MHMessage}{\optional{message}}
A message with MH-specific behaviors. Parameter \var{message} has the same
meaning as with the \class{Message} constructor.
\end{classdesc}

MH messages do not support marks or flags in the traditional sense, but they do
support sequences, which are logical groupings of arbitrary messages. Some mail
reading programs (although not the standard \program{mh} and \program{nmh}) use
sequences in much the same way flags are used with other formats, as follows:

\begin{tableii}{l|l}{textrm}{Sequence}{Explanation}
\lineii{unseen}{Not read, but previously detected by MUA}
\lineii{replied}{Replied to}
\lineii{flagged}{Marked as important}
\end{tableii}

\class{MHMessage} instances offer the following methods:

\begin{methoddesc}{get_sequences}{}
Return a list of the names of sequences that include this message.
\end{methoddesc}

\begin{methoddesc}{set_sequences}{sequences}
Set the list of sequences that include this message.
\end{methoddesc}

\begin{methoddesc}{add_sequence}{sequence}
Add \var{sequence} to the list of sequences that include this message.
\end{methoddesc}

\begin{methoddesc}{remove_sequence}{sequence}
Remove \var{sequence} from the list of sequences that include this message.
\end{methoddesc}

When an \class{MHMessage} instance is created based upon a
\class{MaildirMessage} instance, the following conversions take place:

\begin{tableii}{l|l}{textrm}
    {Resulting state}{\class{MaildirMessage} state}
\lineii{"unseen" sequence}{no S flag}
\lineii{"replied" sequence}{R flag}
\lineii{"flagged" sequence}{F flag}
\end{tableii}

When an \class{MHMessage} instance is created based upon an \class{mboxMessage}
or \class{MMDFMessage} instance, the \mailheader{Status} and
\mailheader{X-Status} headers are omitted and the following conversions take
place:

\begin{tableii}{l|l}{textrm}
    {Resulting state}{\class{mboxMessage} or \class{MMDFMessage} state}
\lineii{"unseen" sequence}{no R flag}
\lineii{"replied" sequence}{A flag}
\lineii{"flagged" sequence}{F flag}
\end{tableii}

When an \class{MHMessage} instance is created based upon a \class{BabylMessage}
instance, the following conversions take place:

\begin{tableii}{l|l}{textrm}
    {Resulting state}{\class{BabylMessage} state}
\lineii{"unseen" sequence}{"unseen" label}
\lineii{"replied" sequence}{"answered" label}
\end{tableii}

\subsubsection{\class{BabylMessage}}
\label{mailbox-babylmessage}

\begin{classdesc}{BabylMessage}{\optional{message}}
A message with Babyl-specific behaviors. Parameter \var{message} has the same
meaning as with the \class{Message} constructor.
\end{classdesc}

Certain message labels, called \dfn{attributes}, are defined by convention to
have special meanings. The attributes are as follows:

\begin{tableii}{l|l}{textrm}{Label}{Explanation}
\lineii{unseen}{Not read, but previously detected by MUA}
\lineii{deleted}{Marked for subsequent deletion}
\lineii{filed}{Copied to another file or mailbox}
\lineii{answered}{Replied to}
\lineii{forwarded}{Forwarded}
\lineii{edited}{Modified by the user}
\lineii{resent}{Resent}
\end{tableii}

By default, Rmail displays only
visible headers. The \class{BabylMessage} class, though, uses the original
headers because they are more complete. Visible headers may be accessed
explicitly if desired.

\class{BabylMessage} instances offer the following methods:

\begin{methoddesc}{get_labels}{}
Return a list of labels on the message.
\end{methoddesc}

\begin{methoddesc}{set_labels}{labels}
Set the list of labels on the message to \var{labels}.
\end{methoddesc}

\begin{methoddesc}{add_label}{label}
Add \var{label} to the list of labels on the message.
\end{methoddesc}

\begin{methoddesc}{remove_label}{label}
Remove \var{label} from the list of labels on the message.
\end{methoddesc}

\begin{methoddesc}{get_visible}{}
Return an \class{Message} instance whose headers are the message's visible
headers and whose body is empty.
\end{methoddesc}

\begin{methoddesc}{set_visible}{visible}
Set the message's visible headers to be the same as the headers in
\var{message}. Parameter \var{visible} should be a \class{Message} instance, an
\class{email.Message.Message} instance, a string, or a file-like object (which
should be open in text mode).
\end{methoddesc}

\begin{methoddesc}{update_visible}{}
When a \class{BabylMessage} instance's original headers are modified, the
visible headers are not automatically modified to correspond. This method
updates the visible headers as follows: each visible header with a
corresponding original header is set to the value of the original header, each
visible header without a corresponding original header is removed, and any of
\mailheader{Date}, \mailheader{From}, \mailheader{Reply-To}, \mailheader{To},
\mailheader{CC}, and \mailheader{Subject} that are present in the original
headers but not the visible headers are added to the visible headers.
\end{methoddesc}

When a \class{BabylMessage} instance is created based upon a
\class{MaildirMessage} instance, the following conversions take place:

\begin{tableii}{l|l}{textrm}
    {Resulting state}{\class{MaildirMessage} state}
\lineii{"unseen" label}{no S flag}
\lineii{"deleted" label}{T flag}
\lineii{"answered" label}{R flag}
\lineii{"forwarded" label}{P flag}
\end{tableii}

When a \class{BabylMessage} instance is created based upon an
\class{mboxMessage} or \class{MMDFMessage} instance, the \mailheader{Status}
and \mailheader{X-Status} headers are omitted and the following conversions
take place:

\begin{tableii}{l|l}{textrm}
    {Resulting state}{\class{mboxMessage} or \class{MMDFMessage} state}
\lineii{"unseen" label}{no R flag}
\lineii{"deleted" label}{D flag}
\lineii{"answered" label}{A flag}
\end{tableii}

When a \class{BabylMessage} instance is created based upon an \class{MHMessage}
instance, the following conversions take place:

\begin{tableii}{l|l}{textrm}
    {Resulting state}{\class{MHMessage} state}
\lineii{"unseen" label}{"unseen" sequence}
\lineii{"answered" label}{"replied" sequence}
\end{tableii}

\subsubsection{\class{MMDFMessage}}
\label{mailbox-mmdfmessage}

\begin{classdesc}{MMDFMessage}{\optional{message}}
A message with MMDF-specific behaviors. Parameter \var{message} has the same
meaning as with the \class{Message} constructor.
\end{classdesc}

As with message in an mbox mailbox, MMDF messages are stored with the sender's
address and the delivery date in an initial line beginning with "From ".
Likewise, flags that indicate the state of the message are typically stored in
\mailheader{Status} and \mailheader{X-Status} headers.

Conventional flags for MMDF messages are identical to those of mbox message and
are as follows:

\begin{tableiii}{l|l|l}{textrm}{Flag}{Meaning}{Explanation}
\lineiii{R}{Read}{Read}
\lineiii{O}{Old}{Previously detected by MUA}
\lineiii{D}{Deleted}{Marked for subsequent deletion}
\lineiii{F}{Flagged}{Marked as important}
\lineiii{A}{Answered}{Replied to}
\end{tableiii}

The "R" and "O" flags are stored in the \mailheader{Status} header, and the
"D", "F", and "A" flags are stored in the \mailheader{X-Status} header. The
flags and headers typically appear in the order mentioned.

\class{MMDFMessage} instances offer the following methods, which are identical
to those offered by \class{mboxMessage}:

\begin{methoddesc}{get_from}{}
Return a string representing the "From~" line that marks the start of the
message in an mbox mailbox. The leading "From~" and the trailing newline are
excluded.
\end{methoddesc}

\begin{methoddesc}{set_from}{from_\optional{, time_=None}}
Set the "From~" line to \var{from_}, which should be specified without a
leading "From~" or trailing newline. For convenience, \var{time_} may be
specified and will be formatted appropriately and appended to \var{from_}. If
\var{time_} is specified, it should be a \class{struct_time} instance, a tuple
suitable for passing to \method{time.strftime()}, or \code{True} (to use
\method{time.gmtime()}).
\end{methoddesc}

\begin{methoddesc}{get_flags}{}
Return a string specifying the flags that are currently set. If the message
complies with the conventional format, the result is the concatenation in the
following order of zero or one occurrence of each of \character{R},
\character{O}, \character{D}, \character{F}, and \character{A}.
\end{methoddesc}

\begin{methoddesc}{set_flags}{flags}
Set the flags specified by \var{flags} and unset all others. Parameter
\var{flags} should be the concatenation in any order of zero or more
occurrences of each of \character{R}, \character{O}, \character{D},
\character{F}, and \character{A}.
\end{methoddesc}

\begin{methoddesc}{add_flag}{flag}
Set the flag(s) specified by \var{flag} without changing other flags. To add
more than one flag at a time, \var{flag} may be a string of more than one
character.
\end{methoddesc}

\begin{methoddesc}{remove_flag}{flag}
Unset the flag(s) specified by \var{flag} without changing other flags. To
remove more than one flag at a time, \var{flag} maybe a string of more than one
character.
\end{methoddesc}

When an \class{MMDFMessage} instance is created based upon a
\class{MaildirMessage} instance, a "From~" line is generated based upon the
\class{MaildirMessage} instance's delivery date, and the following conversions
take place:

\begin{tableii}{l|l}{textrm}
    {Resulting state}{\class{MaildirMessage} state}
\lineii{R flag}{S flag}
\lineii{O flag}{"cur" subdirectory}
\lineii{D flag}{T flag}
\lineii{F flag}{F flag}
\lineii{A flag}{R flag}
\end{tableii}

When an \class{MMDFMessage} instance is created based upon an \class{MHMessage}
instance, the following conversions take place:

\begin{tableii}{l|l}{textrm}
    {Resulting state}{\class{MHMessage} state}
\lineii{R flag and O flag}{no "unseen" sequence}
\lineii{O flag}{"unseen" sequence}
\lineii{F flag}{"flagged" sequence}
\lineii{A flag}{"replied" sequence}
\end{tableii}

When an \class{MMDFMessage} instance is created based upon a
\class{BabylMessage} instance, the following conversions take place:

\begin{tableii}{l|l}{textrm}
    {Resulting state}{\class{BabylMessage} state}
\lineii{R flag and O flag}{no "unseen" label}
\lineii{O flag}{"unseen" label}
\lineii{D flag}{"deleted" label}
\lineii{A flag}{"answered" label}
\end{tableii}

When an \class{MMDFMessage} instance is created based upon an
\class{mboxMessage} instance, the "From~" line is copied and all flags directly
correspond:

\begin{tableii}{l|l}{textrm}
    {Resulting state}{\class{mboxMessage} state}
\lineii{R flag}{R flag}
\lineii{O flag}{O flag}
\lineii{D flag}{D flag}
\lineii{F flag}{F flag}
\lineii{A flag}{A flag}
\end{tableii}

\subsection{Exceptions}
\label{mailbox-deprecated}

The following exception classes are defined in the \module{mailbox} module:

\begin{classdesc}{Error}{}
The based class for all other module-specific exceptions.
\end{classdesc}

\begin{classdesc}{NoSuchMailboxError}{}
Raised when a mailbox is expected but is not found, such as when instantiating
a \class{Mailbox} subclass with a path that does not exist (and with the
\var{create} parameter set to \code{False}), or when opening a folder that does
not exist.
\end{classdesc}

\begin{classdesc}{NotEmptyErrorError}{}
Raised when a mailbox is not empty but is expected to be, such as when deleting
a folder that contains messages.
\end{classdesc}

\begin{classdesc}{ExternalClashError}{}
Raised when some mailbox-related condition beyond the control of the program
causes it to be unable to proceed, such as when failing to acquire a lock that
another program already holds a lock, or when a uniquely-generated file name
already exists.
\end{classdesc}

\begin{classdesc}{FormatError}{}
Raised when the data in a file cannot be parsed, such as when an \class{MH}
instance attempts to read a corrupted \file{.mh_sequences} file.
\end{classdesc}

\subsection{Deprecated classes and methods}
\label{mailbox-deprecated}

Older versions of the \module{mailbox} module do not support modification of
mailboxes, such as adding or removing message, and do not provide classes to
represent format-specific message properties. For backward compatibility, the
older mailbox classes are still available, but the newer classes should be used
in preference to them.

Older mailbox objects support only iteration and provide a single public
method:

\begin{methoddesc}{next}{}
Return the next message in the mailbox, created with the optional \var{factory}
argument passed into the mailbox object's constructor. By default this is an
\class{rfc822.Message} object (see the \refmodule{rfc822} module).  Depending
on the mailbox implementation the \var{fp} attribute of this object may be a
true file object or a class instance simulating a file object, taking care of
things like message boundaries if multiple mail messages are contained in a
single file, etc.  If no more messages are available, this method returns
\code{None}.
\end{methoddesc}

Most of the older mailbox classes have names that differ from the current
mailbox class names, except for \class{Maildir}. For this reason, the new
\class{Maildir} class defines a \method{next()} method and its constructor
differs slightly from those of the other new mailbox classes.

The older mailbox classes whose names are not the same as their newer
counterparts are as follows:

\begin{classdesc}{UnixMailbox}{fp\optional{, factory}}
Access to a classic \UNIX-style mailbox, where all messages are
contained in a single file and separated by \samp{From }
(a.k.a.\ \samp{From_}) lines.  The file object \var{fp} points to the
mailbox file.  The optional \var{factory} parameter is a callable that
should create new message objects.  \var{factory} is called with one
argument, \var{fp} by the \method{next()} method of the mailbox
object.  The default is the \class{rfc822.Message} class (see the
\refmodule{rfc822} module -- and the note below).

\begin{notice}
  For reasons of this module's internal implementation, you will
  probably want to open the \var{fp} object in binary mode.  This is
  especially important on Windows.
\end{notice}

For maximum portability, messages in a \UNIX-style mailbox are
separated by any line that begins exactly with the string \code{'From
'} (note the trailing space) if preceded by exactly two newlines.
Because of the wide-range of variations in practice, nothing else on
the From_ line should be considered.  However, the current
implementation doesn't check for the leading two newlines.  This is
usually fine for most applications.

The \class{UnixMailbox} class implements a more strict version of
From_ line checking, using a regular expression that usually correctly
matched From_ delimiters.  It considers delimiter line to be separated
by \samp{From \var{name} \var{time}} lines.  For maximum portability,
use the \class{PortableUnixMailbox} class instead.  This class is
identical to \class{UnixMailbox} except that individual messages are
separated by only \samp{From } lines.

For more information, see
\citetitle[http://home.netscape.com/eng/mozilla/2.0/relnotes/demo/content-length.html]{Configuring
Netscape Mail on \UNIX: Why the Content-Length Format is Bad}.
\end{classdesc}

\begin{classdesc}{PortableUnixMailbox}{fp\optional{, factory}}
A less-strict version of \class{UnixMailbox}, which considers only the
\samp{From } at the beginning of the line separating messages.  The
``\var{name} \var{time}'' portion of the From line is ignored, to
protect against some variations that are observed in practice.  This
works since lines in the message which begin with \code{'From '} are
quoted by mail handling software at delivery-time.
\end{classdesc}

\begin{classdesc}{MmdfMailbox}{fp\optional{, factory}}
Access an MMDF-style mailbox, where all messages are contained
in a single file and separated by lines consisting of 4 control-A
characters.  The file object \var{fp} points to the mailbox file.
Optional \var{factory} is as with the \class{UnixMailbox} class.
\end{classdesc}

\begin{classdesc}{MHMailbox}{dirname\optional{, factory}}
Access an MH mailbox, a directory with each message in a separate
file with a numeric name.
The name of the mailbox directory is passed in \var{dirname}.
\var{factory} is as with the \class{UnixMailbox} class.
\end{classdesc}

\begin{classdesc}{BabylMailbox}{fp\optional{, factory}}
Access a Babyl mailbox, which is similar to an MMDF mailbox.  In
Babyl format, each message has two sets of headers, the
\emph{original} headers and the \emph{visible} headers.  The original
headers appear before a line containing only \code{'*** EOOH ***'}
(End-Of-Original-Headers) and the visible headers appear after the
\code{EOOH} line.  Babyl-compliant mail readers will show you only the
visible headers, and \class{BabylMailbox} objects will return messages
containing only the visible headers.  You'll have to do your own
parsing of the mailbox file to get at the original headers.  Mail
messages start with the EOOH line and end with a line containing only
\code{'\e{}037\e{}014'}.  \var{factory} is as with the
\class{UnixMailbox} class.
\end{classdesc}

If you wish to use the older mailbox classes with the \module{email} module
rather than the deprecated \module{rfc822} module, you can do so as follows:

\begin{verbatim}
import email
import email.Errors
import mailbox

def msgfactory(fp):
    try:
        return email.message_from_file(fp)
    except email.Errors.MessageParseError:
        # Don't return None since that will
        # stop the mailbox iterator
        return ''

mbox = mailbox.UnixMailbox(fp, msgfactory)
\end{verbatim}

Alternatively, if you know your mailbox contains only well-formed MIME
messages, you can simplify this to:

\begin{verbatim}
import email
import mailbox

mbox = mailbox.UnixMailbox(fp, email.message_from_file)
\end{verbatim}

\subsection{Examples}
\label{mailbox-examples}

A simple example of printing the subjects of all messages in a mailbox that
seem interesting:

\begin{verbatim}
import mailbox
for message in mailbox.mbox('~/mbox'):
    subject = message['subject']       # Could possibly be None.
    if subject and 'python' in subject.lower():
        print subject
\end{verbatim}

To copy all mail from a Babyl mailbox to an MH mailbox, converting all
of the format-specific information that can be converted:

\begin{verbatim}
import mailbox
destination = mailbox.MH('~/Mail')
for message in mailbox.Babyl('~/RMAIL'):
    destination.add(MHMessage(message))
\end{verbatim}

An example of sorting mail from numerous mailing lists, being careful to avoid
mail corruption due to concurrent modification by other programs, mail loss due
to interruption of the program, or premature termination due to malformed
messages in the mailbox:

\begin{verbatim}
import mailbox
import email.Errors
list_names = ('python-list', 'python-dev', 'python-bugs')
boxes = dict((name, mailbox.mbox('~/email/%s' % name)) for name in list_names)
inbox = mailbox.Maildir('~/Maildir', None)
for key in inbox.iterkeys():
    try:
        message = inbox[key]
    except email.Errors.MessageParseError:
        continue                # The message is malformed. Just leave it.
    for name in list_names:
        list_id = message['list-id']
        if list_id and name in list_id:
            box = boxes[name]
            box.lock()
            box.add(message)
            box.flush()         # Write copy to disk before removing original.
            box.unlock()
            inbox.discard(key)
            break               # Found destination, so stop looking.
for box in boxes.itervalues():
    box.close()
\end{verbatim}

\section{Standard Module \sectcode{mimify}}
\label{module-mimify}
\stmodindex{mimify}

The mimify module defines two functions to convert mail messages to
and from MIME format.  The mail message can be either a simple message
or a so-called multipart message.  Each part is treated separately.
Mimifying (a part of) a message entails encoding the message as
quoted-printable if it contains any characters that cannot be
represented using 7-bit ASCII.  Unmimifying (a part of) a message
entails undoing the quoted-printable encoding.  Mimify and unmimify
are especially useful when a message has to be edited before being
sent.  Typical use would be:

\begin{verbatim}
unmimify message
edit message
mimify message
send message
\end{verbatim}

The modules defines the following user-callable functions and
user-settable variables:

\begin{funcdesc}{mimify}{infile, outfile}
Copy the message in \var{infile} to \var{outfile}, converting parts to
quoted-printable and adding MIME mail headers when necessary.
\var{infile} and \var{outfile} can be file objects (actually, any
object that has a \code{readline} method (for \var{infile}) or a
\code{write} method (for \var{outfile})) or strings naming the files.
If \var{infile} and \var{outfile} are both strings, they may have the
same value.
\end{funcdesc}

\begin{funcdesc}{unmimify}{infile, outfile, decode_base64 = 0} 
Copy the message in \var{infile} to \var{outfile}, decoding all
quoted-printable parts.  \var{infile} and \var{outfile} can be file
objects (actually, any object that has a \code{readline} method (for
\var{infile}) or a \code{write} method (for \var{outfile})) or strings
naming the files.  If \var{infile} and \var{outfile} are both strings,
they may have the same value.
If the \var{decode_base64} argument is provided and tests true, any
parts that are coded in the base64 encoding are decoded as well.
\end{funcdesc}

\begin{funcdesc}{mime_decode_header}{line}
Return a decoded version of the encoded header line in \var{line}.
\end{funcdesc}

\begin{funcdesc}{mime_encode_header}{line}
Return a MIME-encoded version of the header line in \var{line}.
\end{funcdesc}

\begin{datadesc}{MAXLEN}
By default, a part will be encoded as quoted-printable when it
contains any non-ASCII characters (i.e., characters with the 8th bit
set), or if there are any lines longer than \code{MAXLEN} characters
(default value 200).  
\end{datadesc}

\begin{datadesc}{CHARSET}
When not specified in the mail headers, a character set must be filled
in.  The string used is stored in \code{CHARSET}, and the default
value is ISO-8859-1 (also known as Latin1 (latin-one)).
\end{datadesc}

This module can also be used from the command line.  Usage is as
follows:
\begin{verbatim}
mimify.py -e [-l length] [infile [outfile]]
mimify.py -d [-b] [infile [outfile]]
\end{verbatim}
to encode (mimify) and decode (unmimify) respectively.  \var{infile}
defaults to standard input, \var{outfile} defaults to standard output.
The same file can be specified for input and output.

If the \code{-l} option is given when encoding, if there are any lines
longer than the specified \var{length}, the containing part will be
encoded.

If the \code{-b} option is given when decoding, any base64 parts will
be decoded as well.


% Module and documentation by Eric S. Raymond, 21 Dec 1998 
\section{Standard Module \module{netrc}}
\stmodindex{netrc}
\label{module-netrc}

The \code{netrc} class parses and encapsulates the netrc file format
used by Unix's ftp(1) and other FTP clientd

\begin{classdesc}{netrc}{\optional{file}}
A \class{netrc} instance or subclass instance enapsulates data from 
a netrc file.  The initialization argument, if present, specifies the file
to parse.  If no argument is given, the file .netrc in the user's home
directory will be read.  Parse errors will throw a SyntaxError
exception with associated diagnostic information including the file
name, line number, and terminating token.
\end{classdesc}

\subsection{netrc Objects}
\label{netrc-objects}

A \class{netrc} instance has the following methods:

\begin{methoddesc}{authenticators}{}
Return a 3-tuple (login, account, password) of authenticators for the
given host.  If the netrc file did not contain an entry for the given
host, return the tuple associated with the `default' entry.  If
neither matching host nor default entry is available, return None.
\end{methoddesc}

\begin{methoddesc}{__repr__}{host}
Dump the class data as a string in the format of a netrc file.
(This discards comments and may reorder the entries.)
\end{methoddesc}

Instances of \class{netrc} have public instance variables:

\begin{memberdesc}{hosts}
Dictionmary mapping host names to login/account/password tuples.  The
`default' entry, if any, is represented as a pseudo-host by that name.
\end{memberdesc}

\begin{memberdesc}{macros}
Dictionary mapping macro names to string lists.
\end{memberdesc}





\chapter{Restricted Execution}
\label{restricted}

In general, Python programs have complete access to the underlying
operating system throug the various functions and classes, For
example, a Python program can open any file for reading and writing by
using the \code{open()} built-in function (provided the underlying OS
gives you permission!).  This is exactly what you want for most
applications.

There exists a class of applications for which this ``openness'' is
inappropriate.  Take Grail: a web browser that accepts ``applets'',
snippets of Python code, from anywhere on the Internet for execution
on the local system.  This can be used to improve the user interface
of forms, for instance.  Since the originator of the code is unknown,
it is obvious that it cannot be trusted with the full resources of the
local machine.

\emph{Restricted execution} is the basic framework in Python that allows
for the segregation of trusted and untrusted code.  It is based on the
notion that trusted Python code (a \emph{supervisor}) can create a
``padded cell' (or environment) with limited permissions, and run the
untrusted code within this cell.  The untrusted code cannot break out
of its cell, and can only interact with sensitive system resources
through interfaces defined and managed by the trusted code.  The term
``restricted execution'' is favored over ``safe-Python''
since true safety is hard to define, and is determined by the way the
restricted environment is created.  Note that the restricted
environments can be nested, with inner cells creating subcells of
lesser, but never greater, privilege.

An interesting aspect of Python's restricted execution model is that
the interfaces presented to untrusted code usually have the same names
as those presented to trusted code.  Therefore no special interfaces
need to be learned to write code designed to run in a restricted
environment.  And because the exact nature of the padded cell is
determined by the supervisor, different restrictions can be imposed,
depending on the application.  For example, it might be deemed
``safe'' for untrusted code to read any file within a specified
directory, but never to write a file.  In this case, the supervisor
may redefine the built-in
\code{open()} function so that it raises an exception whenever the
\var{mode} parameter is \code{'w'}.  It might also perform a
\code{chroot()}-like operation on the \var{filename} parameter, such
that root is always relative to some safe ``sandbox'' area of the
filesystem.  In this case, the untrusted code would still see an
built-in \code{open()} function in its environment, with the same
calling interface.  The semantics would be identical too, with
\code{IOError}s being raised when the supervisor determined that an
unallowable parameter is being used.

The Python run-time determines whether a particular code block is
executing in restricted execution mode based on the identity of the
\code{__builtins__} object in its global variables: if this is (the
dictionary of) the standard \code{__builtin__} module, the code is
deemed to be unrestricted, else it is deemed to be restricted.

Python code executing in restricted mode faces a number of limitations
that are designed to prevent it from escaping from the padded cell.
For instance, the function object attribute \code{func_globals} and the
class and instance object attribute \code{__dict__} are unavailable.

Two modules provide the framework for setting up restricted execution
environments:

\begin{description}

\item[rexec]
--- Basic restricted execution framework.

\item[Bastion]
--- Providing restricted access to objects.

\end{description}

\section{\module{rexec} ---
         Restricted execution framework}

\declaremodule{standard}{rexec}
\modulesynopsis{Basic restricted execution framework.}
\versionchanged[Disabled module]{2.3}

\begin{notice}[warning]
  The documentation has been left in place to help in reading old code
  that uses the module.
\end{notice}

This module contains the \class{RExec} class, which supports
\method{r_eval()}, \method{r_execfile()}, \method{r_exec()}, and
\method{r_import()} methods, which are restricted versions of the standard
Python functions \method{eval()}, \method{execfile()} and
the \keyword{exec} and \keyword{import} statements.
Code executed in this restricted environment will
only have access to modules and functions that are deemed safe; you
can subclass \class{RExec} to add or remove capabilities as desired.

\begin{notice}[warning]
  While the \module{rexec} module is designed to perform as described
  below, it does have a few known vulnerabilities which could be
  exploited by carefully written code.  Thus it should not be relied
  upon in situations requiring ``production ready'' security.  In such
  situations, execution via sub-processes or very careful
  ``cleansing'' of both code and data to be processed may be
  necessary.  Alternatively, help in patching known \module{rexec}
  vulnerabilities would be welcomed.
\end{notice}

\begin{notice}
  The \class{RExec} class can prevent code from performing unsafe
  operations like reading or writing disk files, or using TCP/IP
  sockets.  However, it does not protect against code using extremely
  large amounts of memory or processor time.
\end{notice}

\begin{classdesc}{RExec}{\optional{hooks\optional{, verbose}}}
Returns an instance of the \class{RExec} class.  

\var{hooks} is an instance of the \class{RHooks} class or a subclass of it.
If it is omitted or \code{None}, the default \class{RHooks} class is
instantiated.
Whenever the \module{rexec} module searches for a module (even a
built-in one) or reads a module's code, it doesn't actually go out to
the file system itself.  Rather, it calls methods of an \class{RHooks}
instance that was passed to or created by its constructor.  (Actually,
the \class{RExec} object doesn't make these calls --- they are made by
a module loader object that's part of the \class{RExec} object.  This
allows another level of flexibility, which can be useful when changing
the mechanics of \keyword{import} within the restricted environment.)

By providing an alternate \class{RHooks} object, we can control the
file system accesses made to import a module, without changing the
actual algorithm that controls the order in which those accesses are
made.  For instance, we could substitute an \class{RHooks} object that
passes all filesystem requests to a file server elsewhere, via some
RPC mechanism such as ILU.  Grail's applet loader uses this to support
importing applets from a URL for a directory.

If \var{verbose} is true, additional debugging output may be sent to
standard output.
\end{classdesc}

It is important to be aware that code running in a restricted
environment can still call the \function{sys.exit()} function.  To
disallow restricted code from exiting the interpreter, always protect
calls that cause restricted code to run with a
\keyword{try}/\keyword{except} statement that catches the
\exception{SystemExit} exception.  Removing the \function{sys.exit()}
function from the restricted environment is not sufficient --- the
restricted code could still use \code{raise SystemExit}.  Removing
\exception{SystemExit} is not a reasonable option; some library code
makes use of this and would break were it not available.


\begin{seealso}
  \seetitle[http://grail.sourceforge.net/]{Grail Home Page}{Grail is a
            Web browser written entirely in Python.  It uses the
            \module{rexec} module as a foundation for supporting
            Python applets, and can be used as an example usage of
            this module.}
\end{seealso}


\subsection{RExec Objects \label{rexec-objects}}

\class{RExec} instances support the following methods:

\begin{methoddesc}{r_eval}{code}
\var{code} must either be a string containing a Python expression, or
a compiled code object, which will be evaluated in the restricted
environment's \module{__main__} module.  The value of the expression or
code object will be returned.
\end{methoddesc}

\begin{methoddesc}{r_exec}{code}
\var{code} must either be a string containing one or more lines of
Python code, or a compiled code object, which will be executed in the
restricted environment's \module{__main__} module.
\end{methoddesc}

\begin{methoddesc}{r_execfile}{filename}
Execute the Python code contained in the file \var{filename} in the
restricted environment's \module{__main__} module.
\end{methoddesc}

Methods whose names begin with \samp{s_} are similar to the functions
beginning with \samp{r_}, but the code will be granted access to
restricted versions of the standard I/O streams \code{sys.stdin},
\code{sys.stderr}, and \code{sys.stdout}.

\begin{methoddesc}{s_eval}{code}
\var{code} must be a string containing a Python expression, which will
be evaluated in the restricted environment.  
\end{methoddesc}

\begin{methoddesc}{s_exec}{code}
\var{code} must be a string containing one or more lines of Python code,
which will be executed in the restricted environment.  
\end{methoddesc}

\begin{methoddesc}{s_execfile}{code}
Execute the Python code contained in the file \var{filename} in the
restricted environment.
\end{methoddesc}

\class{RExec} objects must also support various methods which will be
implicitly called by code executing in the restricted environment.
Overriding these methods in a subclass is used to change the policies
enforced by a restricted environment.

\begin{methoddesc}{r_import}{modulename\optional{, globals\optional{,
                             locals\optional{, fromlist}}}}
Import the module \var{modulename}, raising an \exception{ImportError}
exception if the module is considered unsafe.
\end{methoddesc}

\begin{methoddesc}{r_open}{filename\optional{, mode\optional{, bufsize}}}
Method called when \function{open()} is called in the restricted
environment.  The arguments are identical to those of \function{open()},
and a file object (or a class instance compatible with file objects)
should be returned.  \class{RExec}'s default behaviour is allow opening
any file for reading, but forbidding any attempt to write a file.  See
the example below for an implementation of a less restrictive
\method{r_open()}.
\end{methoddesc}

\begin{methoddesc}{r_reload}{module}
Reload the module object \var{module}, re-parsing and re-initializing it.  
\end{methoddesc}

\begin{methoddesc}{r_unload}{module}
Unload the module object \var{module} (remove it from the
restricted environment's \code{sys.modules} dictionary).
\end{methoddesc}

And their equivalents with access to restricted standard I/O streams:

\begin{methoddesc}{s_import}{modulename\optional{, globals\optional{,
                             locals\optional{, fromlist}}}}
Import the module \var{modulename}, raising an \exception{ImportError}
exception if the module is considered unsafe.
\end{methoddesc}

\begin{methoddesc}{s_reload}{module}
Reload the module object \var{module}, re-parsing and re-initializing it.  
\end{methoddesc}

\begin{methoddesc}{s_unload}{module}
Unload the module object \var{module}.   
% XXX what are the semantics of this?  
\end{methoddesc}


\subsection{Defining restricted environments \label{rexec-extension}}

The \class{RExec} class has the following class attributes, which are
used by the \method{__init__()} method.  Changing them on an existing
instance won't have any effect; instead, create a subclass of
\class{RExec} and assign them new values in the class definition.
Instances of the new class will then use those new values.  All these
attributes are tuples of strings.

\begin{memberdesc}{nok_builtin_names}
Contains the names of built-in functions which will \emph{not} be
available to programs running in the restricted environment.  The
value for \class{RExec} is \code{('open', 'reload', '__import__')}.
(This gives the exceptions, because by far the majority of built-in
functions are harmless.  A subclass that wants to override this
variable should probably start with the value from the base class and
concatenate additional forbidden functions --- when new dangerous
built-in functions are added to Python, they will also be added to
this module.)
\end{memberdesc}

\begin{memberdesc}{ok_builtin_modules}
Contains the names of built-in modules which can be safely imported.
The value for \class{RExec} is \code{('audioop', 'array', 'binascii',
'cmath', 'errno', 'imageop', 'marshal', 'math', 'md5', 'operator',
'parser', 'regex', 'rotor', 'select', 'sha', '_sre', 'strop',
'struct', 'time')}.  A similar remark about overriding this variable
applies --- use the value from the base class as a starting point.
\end{memberdesc}

\begin{memberdesc}{ok_path}
Contains the directories which will be searched when an \keyword{import}
is performed in the restricted environment.  
The value for \class{RExec} is the same as \code{sys.path} (at the time
the module is loaded) for unrestricted code.
\end{memberdesc}

\begin{memberdesc}{ok_posix_names}
% Should this be called ok_os_names?
Contains the names of the functions in the \refmodule{os} module which will be
available to programs running in the restricted environment.  The
value for \class{RExec} is \code{('error', 'fstat', 'listdir',
'lstat', 'readlink', 'stat', 'times', 'uname', 'getpid', 'getppid',
'getcwd', 'getuid', 'getgid', 'geteuid', 'getegid')}.
\end{memberdesc}

\begin{memberdesc}{ok_sys_names}
Contains the names of the functions and variables in the \refmodule{sys}
module which will be available to programs running in the restricted
environment.  The value for \class{RExec} is \code{('ps1', 'ps2',
'copyright', 'version', 'platform', 'exit', 'maxint')}.
\end{memberdesc}

\begin{memberdesc}{ok_file_types}
Contains the file types from which modules are allowed to be loaded.
Each file type is an integer constant defined in the \refmodule{imp} module.
The meaningful values are \constant{PY_SOURCE}, \constant{PY_COMPILED}, and
\constant{C_EXTENSION}.  The value for \class{RExec} is \code{(C_EXTENSION,
PY_SOURCE)}.  Adding \constant{PY_COMPILED} in subclasses is not recommended;
an attacker could exit the restricted execution mode by putting a forged
byte-compiled file (\file{.pyc}) anywhere in your file system, for example
by writing it to \file{/tmp} or uploading it to the \file{/incoming}
directory of your public FTP server.
\end{memberdesc}


\subsection{An example}

Let us say that we want a slightly more relaxed policy than the
standard \class{RExec} class.  For example, if we're willing to allow
files in \file{/tmp} to be written, we can subclass the \class{RExec}
class:

\begin{verbatim}
class TmpWriterRExec(rexec.RExec):
    def r_open(self, file, mode='r', buf=-1):
        if mode in ('r', 'rb'):
            pass
        elif mode in ('w', 'wb', 'a', 'ab'):
            # check filename : must begin with /tmp/
            if file[:5]!='/tmp/': 
                raise IOError, "can't write outside /tmp"
            elif (string.find(file, '/../') >= 0 or
                 file[:3] == '../' or file[-3:] == '/..'):
                raise IOError, "'..' in filename forbidden"
        else: raise IOError, "Illegal open() mode"
        return open(file, mode, buf)
\end{verbatim}
%
Notice that the above code will occasionally forbid a perfectly valid
filename; for example, code in the restricted environment won't be
able to open a file called \file{/tmp/foo/../bar}.  To fix this, the
\method{r_open()} method would have to simplify the filename to
\file{/tmp/bar}, which would require splitting apart the filename and
performing various operations on it.  In cases where security is at
stake, it may be preferable to write simple code which is sometimes
overly restrictive, instead of more general code that is also more
complex and may harbor a subtle security hole.

\section{Standard Module \sectcode{Bastion}}
\stmodindex{Bastion}
\renewcommand{\indexsubitem}{(in module Bastion)}

% I'm concerned that the word 'bastion' won't be understood by people
% for whom English is a second language, making the module name
% somewhat mysterious.  Thus, the brief definition... --amk

According to the dictionary, a bastion is ``a fortified area or
position'', or ``something that is considered a stronghold.''  It's a
suitable name for this module, which provides a way to forbid access
to certain attributes of an object.  It must always be used with the
\code{rexec} module, in order to allow restricted-mode programs access
to certain safe attributes of an object, while denying access to
other, unsafe attributes.

% I've punted on the issue of documenting keyword arguments for now.

\begin{funcdesc}{Bastion}{object\optional{\, filter\, name\, class}}
Protect the class instance \var{object}, returning a bastion for the
object.  Any attempt to access one of the object's attributes will
have to be approved by the \var{filter} function; if the access is
denied an AttributeError exception will be raised.

If present, \var{filter} must be a function that accepts a string
containing an attribute name, and returns true if access to that
attribute will be permitted; if \var{filter} returns false, the access
is denied.  The default filter denies access to any function beginning
with an underscore (\code{_}).  The bastion's string representation
will be \code{<Bastion for \var{name}>} if a value for
\var{name} is provided; otherwise, \code{repr(\var{object})} will be used.

\var{class}, if present, would be a subclass of \code{BastionClass};
see the code in \file{bastion.py} for the details.  Overriding the
default \code{BastionClass} will rarely be required.  

\end{funcdesc}


\chapter{Multimedia Services}
\label{mmedia}

The modules described in this chapter implement various algorithms or
interfaces that are mainly useful for multimedia applications.  They
are available at the discretion of the installation.  Here's an overview:

\begin{description}

\item[audioop]
--- Manipulate raw audio data.

\item[imageop]
--- Manipulate raw image data.

\item[aifc]
--- Read and write audio files in AIFF or AIFC format.

\item[jpeg]
--- Read and write image files in compressed JPEG format.

\item[rgbimg]
--- Read and write image files in ``SGI RGB'' format (the module is
\emph{not} SGI specific though)!

\item[imghdr]
--- Determine the type of image contained in a file or byte stream.

\end{description}
			% Multimedia Services
\section{Built-in Module \sectcode{audioop}}
\label{module-audioop}
\bimodindex{audioop}

The \code{audioop} module contains some useful operations on sound fragments.
It operates on sound fragments consisting of signed integer samples
8, 16 or 32 bits wide, stored in Python strings.  This is the same
format as used by the \code{al} and \code{sunaudiodev} modules.  All
scalar items are integers, unless specified otherwise.

A few of the more complicated operations only take 16-bit samples,
otherwise the sample size (in bytes) is always a parameter of the operation.

The module defines the following variables and functions:

\renewcommand{\indexsubitem}{(in module audioop)}
\begin{excdesc}{error}
This exception is raised on all errors, such as unknown number of bytes
per sample, etc.
\end{excdesc}

\begin{funcdesc}{add}{fragment1\, fragment2\, width}
Return a fragment which is the addition of the two samples passed as
parameters.  \var{width} is the sample width in bytes, either
\code{1}, \code{2} or \code{4}.  Both fragments should have the same
length.
\end{funcdesc}

\begin{funcdesc}{adpcm2lin}{adpcmfragment\, width\, state}
Decode an Intel/DVI ADPCM coded fragment to a linear fragment.  See
the description of \code{lin2adpcm} for details on ADPCM coding.
Return a tuple \code{(\var{sample}, \var{newstate})} where the sample
has the width specified in \var{width}.
\end{funcdesc}

\begin{funcdesc}{adpcm32lin}{adpcmfragment\, width\, state}
Decode an alternative 3-bit ADPCM code.  See \code{lin2adpcm3} for
details.
\end{funcdesc}

\begin{funcdesc}{avg}{fragment\, width}
Return the average over all samples in the fragment.
\end{funcdesc}

\begin{funcdesc}{avgpp}{fragment\, width}
Return the average peak-peak value over all samples in the fragment.
No filtering is done, so the usefulness of this routine is
questionable.
\end{funcdesc}

\begin{funcdesc}{bias}{fragment\, width\, bias}
Return a fragment that is the original fragment with a bias added to
each sample.
\end{funcdesc}

\begin{funcdesc}{cross}{fragment\, width}
Return the number of zero crossings in the fragment passed as an
argument.
\end{funcdesc}

\begin{funcdesc}{findfactor}{fragment\, reference}
Return a factor \var{F} such that
\code{rms(add(fragment, mul(reference, -F)))} is minimal, i.e.,
return the factor with which you should multiply \var{reference} to
make it match as well as possible to \var{fragment}.  The fragments
should both contain 2-byte samples.

The time taken by this routine is proportional to \code{len(fragment)}. 
\end{funcdesc}

\begin{funcdesc}{findfit}{fragment\, reference}
This routine (which only accepts 2-byte sample fragments)

Try to match \var{reference} as well as possible to a portion of
\var{fragment} (which should be the longer fragment).  This is
(conceptually) done by taking slices out of \var{fragment}, using
\code{findfactor} to compute the best match, and minimizing the
result.  The fragments should both contain 2-byte samples.  Return a
tuple \code{(\var{offset}, \var{factor})} where \var{offset} is the
(integer) offset into \var{fragment} where the optimal match started
and \var{factor} is the (floating-point) factor as per
\code{findfactor}.
\end{funcdesc}

\begin{funcdesc}{findmax}{fragment\, length}
Search \var{fragment} for a slice of length \var{length} samples (not
bytes!)\ with maximum energy, i.e., return \var{i} for which
\code{rms(fragment[i*2:(i+length)*2])} is maximal.  The fragments
should both contain 2-byte samples.

The routine takes time proportional to \code{len(fragment)}.
\end{funcdesc}

\begin{funcdesc}{getsample}{fragment\, width\, index}
Return the value of sample \var{index} from the fragment.
\end{funcdesc}

\begin{funcdesc}{lin2lin}{fragment\, width\, newwidth}
Convert samples between 1-, 2- and 4-byte formats.
\end{funcdesc}

\begin{funcdesc}{lin2adpcm}{fragment\, width\, state}
Convert samples to 4 bit Intel/DVI ADPCM encoding.  ADPCM coding is an
adaptive coding scheme, whereby each 4 bit number is the difference
between one sample and the next, divided by a (varying) step.  The
Intel/DVI ADPCM algorithm has been selected for use by the IMA, so it
may well become a standard.

\code{State} is a tuple containing the state of the coder.  The coder
returns a tuple \code{(\var{adpcmfrag}, \var{newstate})}, and the
\var{newstate} should be passed to the next call of lin2adpcm.  In the
initial call \code{None} can be passed as the state.  \var{adpcmfrag}
is the ADPCM coded fragment packed 2 4-bit values per byte.
\end{funcdesc}

\begin{funcdesc}{lin2adpcm3}{fragment\, width\, state}
This is an alternative ADPCM coder that uses only 3 bits per sample.
It is not compatible with the Intel/DVI ADPCM coder and its output is
not packed (due to laziness on the side of the author).  Its use is
discouraged.
\end{funcdesc}

\begin{funcdesc}{lin2ulaw}{fragment\, width}
Convert samples in the audio fragment to U-LAW encoding and return
this as a Python string.  U-LAW is an audio encoding format whereby
you get a dynamic range of about 14 bits using only 8 bit samples.  It
is used by the Sun audio hardware, among others.
\end{funcdesc}

\begin{funcdesc}{minmax}{fragment\, width}
Return a tuple consisting of the minimum and maximum values of all
samples in the sound fragment.
\end{funcdesc}

\begin{funcdesc}{max}{fragment\, width}
Return the maximum of the {\em absolute value} of all samples in a
fragment.
\end{funcdesc}

\begin{funcdesc}{maxpp}{fragment\, width}
Return the maximum peak-peak value in the sound fragment.
\end{funcdesc}

\begin{funcdesc}{mul}{fragment\, width\, factor}
Return a fragment that has all samples in the original framgent
multiplied by the floating-point value \var{factor}.  Overflow is
silently ignored.
\end{funcdesc}

\begin{funcdesc}{ratecv}{fragment\, width\, nchannels\, inrate\, outrate\, state\optional{\, weightA\, weightB}}
Convert the frame rate of the input fragment.

\code{State} is a tuple containing the state of the converter.  The
converter returns a tupl \code{(\var{newfragment}, \var{newstate})},
and \var{newstate} should be passed to the next call of ratecv.

The \code{weightA} and \code{weightB} arguments are parameters for a
simple digital filter and default to 1 and 0 respectively.
\end{funcdesc}

\begin{funcdesc}{reverse}{fragment\, width}
Reverse the samples in a fragment and returns the modified fragment.
\end{funcdesc}

\begin{funcdesc}{rms}{fragment\, width}
Return the root-mean-square of the fragment, i.e.
\iftexi
the square root of the quotient of the sum of all squared sample value,
divided by the sumber of samples.
\else
% in eqn: sqrt { sum S sub i sup 2  over n }
\begin{displaymath}
\catcode`_=8
\sqrt{\frac{\sum{{S_{i}}^{2}}}{n}}
\end{displaymath}
\fi
This is a measure of the power in an audio signal.
\end{funcdesc}

\begin{funcdesc}{tomono}{fragment\, width\, lfactor\, rfactor} 
Convert a stereo fragment to a mono fragment.  The left channel is
multiplied by \var{lfactor} and the right channel by \var{rfactor}
before adding the two channels to give a mono signal.
\end{funcdesc}

\begin{funcdesc}{tostereo}{fragment\, width\, lfactor\, rfactor}
Generate a stereo fragment from a mono fragment.  Each pair of samples
in the stereo fragment are computed from the mono sample, whereby left
channel samples are multiplied by \var{lfactor} and right channel
samples by \var{rfactor}.
\end{funcdesc}

\begin{funcdesc}{ulaw2lin}{fragment\, width}
Convert sound fragments in ULAW encoding to linearly encoded sound
fragments.  ULAW encoding always uses 8 bits samples, so \var{width}
refers only to the sample width of the output fragment here.
\end{funcdesc}

Note that operations such as \code{mul} or \code{max} make no
distinction between mono and stereo fragments, i.e.\ all samples are
treated equal.  If this is a problem the stereo fragment should be split
into two mono fragments first and recombined later.  Here is an example
of how to do that:
\bcode\begin{verbatim}
def mul_stereo(sample, width, lfactor, rfactor):
    lsample = audioop.tomono(sample, width, 1, 0)
    rsample = audioop.tomono(sample, width, 0, 1)
    lsample = audioop.mul(sample, width, lfactor)
    rsample = audioop.mul(sample, width, rfactor)
    lsample = audioop.tostereo(lsample, width, 1, 0)
    rsample = audioop.tostereo(rsample, width, 0, 1)
    return audioop.add(lsample, rsample, width)
\end{verbatim}\ecode
%
If you use the ADPCM coder to build network packets and you want your
protocol to be stateless (i.e.\ to be able to tolerate packet loss)
you should not only transmit the data but also the state.  Note that
you should send the \var{initial} state (the one you passed to
\code{lin2adpcm}) along to the decoder, not the final state (as returned by
the coder).  If you want to use \code{struct} to store the state in
binary you can code the first element (the predicted value) in 16 bits
and the second (the delta index) in 8.

The ADPCM coders have never been tried against other ADPCM coders,
only against themselves.  It could well be that I misinterpreted the
standards in which case they will not be interoperable with the
respective standards.

The \code{find...} routines might look a bit funny at first sight.
They are primarily meant to do echo cancellation.  A reasonably
fast way to do this is to pick the most energetic piece of the output
sample, locate that in the input sample and subtract the whole output
sample from the input sample:
\bcode\begin{verbatim}
def echocancel(outputdata, inputdata):
    pos = audioop.findmax(outputdata, 800)    # one tenth second
    out_test = outputdata[pos*2:]
    in_test = inputdata[pos*2:]
    ipos, factor = audioop.findfit(in_test, out_test)
    # Optional (for better cancellation):
    # factor = audioop.findfactor(in_test[ipos*2:ipos*2+len(out_test)], 
    #              out_test)
    prefill = '\0'*(pos+ipos)*2
    postfill = '\0'*(len(inputdata)-len(prefill)-len(outputdata))
    outputdata = prefill + audioop.mul(outputdata,2,-factor) + postfill
    return audioop.add(inputdata, outputdata, 2)
\end{verbatim}\ecode

\section{\module{imageop} ---
         Manipulate raw image data}

\declaremodule{builtin}{imageop}
\modulesynopsis{Manipulate raw image data.}


The \module{imageop} module contains some useful operations on images.
It operates on images consisting of 8 or 32 bit pixels stored in
Python strings.  This is the same format as used by
\function{gl.lrectwrite()} and the \refmodule{imgfile} module.

The module defines the following variables and functions:

\begin{excdesc}{error}
This exception is raised on all errors, such as unknown number of bits
per pixel, etc.
\end{excdesc}


\begin{funcdesc}{crop}{image, psize, width, height, x0, y0, x1, y1}
Return the selected part of \var{image}, which should by
\var{width} by \var{height} in size and consist of pixels of
\var{psize} bytes. \var{x0}, \var{y0}, \var{x1} and \var{y1} are like
the \function{gl.lrectread()} parameters, i.e.\ the boundary is
included in the new image.  The new boundaries need not be inside the
picture.  Pixels that fall outside the old image will have their value
set to zero.  If \var{x0} is bigger than \var{x1} the new image is
mirrored.  The same holds for the y coordinates.
\end{funcdesc}

\begin{funcdesc}{scale}{image, psize, width, height, newwidth, newheight}
Return \var{image} scaled to size \var{newwidth} by \var{newheight}.
No interpolation is done, scaling is done by simple-minded pixel
duplication or removal.  Therefore, computer-generated images or
dithered images will not look nice after scaling.
\end{funcdesc}

\begin{funcdesc}{tovideo}{image, psize, width, height}
Run a vertical low-pass filter over an image.  It does so by computing
each destination pixel as the average of two vertically-aligned source
pixels.  The main use of this routine is to forestall excessive
flicker if the image is displayed on a video device that uses
interlacing, hence the name.
\end{funcdesc}

\begin{funcdesc}{grey2mono}{image, width, height, threshold}
Convert a 8-bit deep greyscale image to a 1-bit deep image by
thresholding all the pixels.  The resulting image is tightly packed and
is probably only useful as an argument to \function{mono2grey()}.
\end{funcdesc}

\begin{funcdesc}{dither2mono}{image, width, height}
Convert an 8-bit greyscale image to a 1-bit monochrome image using a
(simple-minded) dithering algorithm.
\end{funcdesc}

\begin{funcdesc}{mono2grey}{image, width, height, p0, p1}
Convert a 1-bit monochrome image to an 8 bit greyscale or color image.
All pixels that are zero-valued on input get value \var{p0} on output
and all one-value input pixels get value \var{p1} on output.  To
convert a monochrome black-and-white image to greyscale pass the
values \code{0} and \code{255} respectively.
\end{funcdesc}

\begin{funcdesc}{grey2grey4}{image, width, height}
Convert an 8-bit greyscale image to a 4-bit greyscale image without
dithering.
\end{funcdesc}

\begin{funcdesc}{grey2grey2}{image, width, height}
Convert an 8-bit greyscale image to a 2-bit greyscale image without
dithering.
\end{funcdesc}

\begin{funcdesc}{dither2grey2}{image, width, height}
Convert an 8-bit greyscale image to a 2-bit greyscale image with
dithering.  As for \function{dither2mono()}, the dithering algorithm
is currently very simple.
\end{funcdesc}

\begin{funcdesc}{grey42grey}{image, width, height}
Convert a 4-bit greyscale image to an 8-bit greyscale image.
\end{funcdesc}

\begin{funcdesc}{grey22grey}{image, width, height}
Convert a 2-bit greyscale image to an 8-bit greyscale image.
\end{funcdesc}

\section{Standard Module \sectcode{aifc}}
\label{module-aifc}
\stmodindex{aifc}

This module provides support for reading and writing AIFF and AIFF-C
files.  AIFF is Audio Interchange File Format, a format for storing
digital audio samples in a file.  AIFF-C is a newer version of the
format that includes the ability to compress the audio data.

Audio files have a number of parameters that describe the audio data.
The sampling rate or frame rate is the number of times per second the
sound is sampled.  The number of channels indicate if the audio is
mono, stereo, or quadro.  Each frame consists of one sample per
channel.  The sample size is the size in bytes of each sample.  Thus a
frame consists of \var{nchannels}*\var{samplesize} bytes, and a
second's worth of audio consists of
\var{nchannels}*\var{samplesize}*\var{framerate} bytes.

For example, CD quality audio has a sample size of two bytes (16
bits), uses two channels (stereo) and has a frame rate of 44,100
frames/second.  This gives a frame size of 4 bytes (2*2), and a
second's worth occupies 2*2*44100 bytes, i.e.\ 176,400 bytes.

Module \code{aifc} defines the following function:

\setindexsubitem{(in module aifc)}
\begin{funcdesc}{open}{file, mode}
Open an AIFF or AIFF-C file and return an object instance with
methods that are described below.  The argument file is either a
string naming a file or a file object.  The mode is either the string
\code{'r'} when the file must be opened for reading, or \code{'w'}
when the file must be opened for writing.  When used for writing, the
file object should be seekable, unless you know ahead of time how many
samples you are going to write in total and use
\code{writeframesraw()} and \code{setnframes()}.
\end{funcdesc}

Objects returned by \code{aifc.open()} when a file is opened for
reading have the following methods:

\setindexsubitem{(aifc object method)}
\begin{funcdesc}{getnchannels}{}
Return the number of audio channels (1 for mono, 2 for stereo).
\end{funcdesc}

\begin{funcdesc}{getsampwidth}{}
Return the size in bytes of individual samples.
\end{funcdesc}

\begin{funcdesc}{getframerate}{}
Return the sampling rate (number of audio frames per second).
\end{funcdesc}

\begin{funcdesc}{getnframes}{}
Return the number of audio frames in the file.
\end{funcdesc}

\begin{funcdesc}{getcomptype}{}
Return a four-character string describing the type of compression used
in the audio file.  For AIFF files, the returned value is
\code{'NONE'}.
\end{funcdesc}

\begin{funcdesc}{getcompname}{}
Return a human-readable description of the type of compression used in
the audio file.  For AIFF files, the returned value is \code{'not
compressed'}.
\end{funcdesc}

\begin{funcdesc}{getparams}{}
Return a tuple consisting of all of the above values in the above
order.
\end{funcdesc}

\begin{funcdesc}{getmarkers}{}
Return a list of markers in the audio file.  A marker consists of a
tuple of three elements.  The first is the mark ID (an integer), the
second is the mark position in frames from the beginning of the data
(an integer), the third is the name of the mark (a string).
\end{funcdesc}

\begin{funcdesc}{getmark}{id}
Return the tuple as described in \code{getmarkers} for the mark with
the given id.
\end{funcdesc}

\begin{funcdesc}{readframes}{nframes}
Read and return the next \var{nframes} frames from the audio file.  The
returned data is a string containing for each frame the uncompressed
samples of all channels.
\end{funcdesc}

\begin{funcdesc}{rewind}{}
Rewind the read pointer.  The next \code{readframes} will start from
the beginning.
\end{funcdesc}

\begin{funcdesc}{setpos}{pos}
Seek to the specified frame number.
\end{funcdesc}

\begin{funcdesc}{tell}{}
Return the current frame number.
\end{funcdesc}

\begin{funcdesc}{close}{}
Close the AIFF file.  After calling this method, the object can no
longer be used.
\end{funcdesc}

Objects returned by \code{aifc.open()} when a file is opened for
writing have all the above methods, except for \code{readframes} and
\code{setpos}.  In addition the following methods exist.  The
\code{get} methods can only be called after the corresponding
\code{set} methods have been called.  Before the first
\code{writeframes()} or \code{writeframesraw()}, all parameters except
for the number of frames must be filled in.

\begin{funcdesc}{aiff}{}
Create an AIFF file.  The default is that an AIFF-C file is created,
unless the name of the file ends in \code{'.aiff'} in which case the
default is an AIFF file.
\end{funcdesc}

\begin{funcdesc}{aifc}{}
Create an AIFF-C file.  The default is that an AIFF-C file is created,
unless the name of the file ends in \code{'.aiff'} in which case the
default is an AIFF file.
\end{funcdesc}

\begin{funcdesc}{setnchannels}{nchannels}
Specify the number of channels in the audio file.
\end{funcdesc}

\begin{funcdesc}{setsampwidth}{width}
Specify the size in bytes of audio samples.
\end{funcdesc}

\begin{funcdesc}{setframerate}{rate}
Specify the sampling frequency in frames per second.
\end{funcdesc}

\begin{funcdesc}{setnframes}{nframes}
Specify the number of frames that are to be written to the audio file.
If this parameter is not set, or not set correctly, the file needs to
support seeking.
\end{funcdesc}

\begin{funcdesc}{setcomptype}{type, name}
Specify the compression type.  If not specified, the audio data will
not be compressed.  In AIFF files, compression is not possible.  The
name parameter should be a human-readable description of the
compression type, the type parameter should be a four-character
string.  Currently the following compression types are supported:
NONE, ULAW, ALAW, G722.
\end{funcdesc}

\begin{funcdesc}{setparams}{nchannels, sampwidth, framerate, comptype, compname}
Set all the above parameters at once.  The argument is a tuple
consisting of the various parameters.  This means that it is possible
to use the result of a \code{getparams()} call as argument to
\code{setparams()}.
\end{funcdesc}

\begin{funcdesc}{setmark}{id, pos, name}
Add a mark with the given id (larger than 0), and the given name at
the given position.  This method can be called at any time before
\code{close()}.
\end{funcdesc}

\begin{funcdesc}{tell}{}
Return the current write position in the output file.  Useful in
combination with \code{setmark()}.
\end{funcdesc}

\begin{funcdesc}{writeframes}{data}
Write data to the output file.  This method can only be called after
the audio file parameters have been set.
\end{funcdesc}

\begin{funcdesc}{writeframesraw}{data}
Like \code{writeframes()}, except that the header of the audio file is
not updated.
\end{funcdesc}

\begin{funcdesc}{close}{}
Close the AIFF file.  The header of the file is updated to reflect the
actual size of the audio data. After calling this method, the object
can no longer be used.
\end{funcdesc}

\section{Built-in Module \module{jpeg}}
\label{module-jpeg}
\bimodindex{jpeg}

The module \module{jpeg} provides access to the jpeg compressor and
decompressor written by the Independent JPEG Group%
\index{Independent JPEG Group}%
. JPEG is a (draft?)
standard for compressing pictures.  For details on JPEG or the
Independent JPEG Group software refer to the JPEG standard or the
documentation provided with the software.

The \module{jpeg} module defines an exception and some functions.

\begin{excdesc}{error}
Exception raised by \function{compress()} and \function{decompress()}
in case of errors.
\end{excdesc}

\begin{funcdesc}{compress}{data, w, h, b}
Treat data as a pixmap of width \var{w} and height \var{h}, with
\var{b} bytes per pixel.  The data is in SGI GL order, so the first
pixel is in the lower-left corner. This means that \function{gl.lrectread()}
return data can immediately be passed to \function{compress()}.
Currently only 1 byte and 4 byte pixels are allowed, the former being
treated as greyscale and the latter as RGB color.
\function{compress()} returns a string that contains the compressed
picture, in JFIF\index{JFIF} format.
\end{funcdesc}

\begin{funcdesc}{decompress}{data}
Data is a string containing a picture in JFIF\index{JFIF} format. It
returns a tuple \code{(\var{data}, \var{width}, \var{height},
\var{bytesperpixel})}.  Again, the data is suitable to pass to
\function{gl.lrectwrite()}.
\end{funcdesc}

\begin{funcdesc}{setoption}{name, value}
Set various options.  Subsequent \function{compress()} and
\function{decompress()} calls will use these options.  The following
options are available:

\begin{tableii}{l|p{3in}}{code}{Option}{Effect}
  \lineii{'forcegray'}{%
    Force output to be grayscale, even if input is RGB.}
  \lineii{'quality'}{%
    Set the quality of the compressed image to a value between
    \code{0} and \code{100} (default is \code{75}).  This only affects
    compression.}
  \lineii{'optimize'}{%
    Perform Huffman table optimization.  Takes longer, but results in
    smaller compressed image.  This only affects compression.}
  \lineii{'smooth'}{%
    Perform inter-block smoothing on uncompressed image.  Only useful
    for low-quality images.  This only affects decompression.}
\end{tableii}
\end{funcdesc}

\section{\module{rgbimg} ---
         Read and write image files in ``SGI RGB'' format.}
\declaremodule{builtin}{rgbimg}

\modulesynopsis{Read and write image files in ``SGI RGB'' format (the module is
\emph{not} SGI specific though!).}


The \module{rgbimg} module allows Python programs to access SGI imglib image
files (also known as \file{.rgb} files).  The module is far from
complete, but is provided anyway since the functionality that there is
enough in some cases.  Currently, colormap files are not supported.

The module defines the following variables and functions:

\begin{excdesc}{error}
This exception is raised on all errors, such as unsupported file type, etc.
\end{excdesc}

\begin{funcdesc}{sizeofimage}{file}
This function returns a tuple \code{(\var{x}, \var{y})} where
\var{x} and \var{y} are the size of the image in pixels.
Only 4 byte RGBA pixels, 3 byte RGB pixels, and 1 byte greyscale pixels
are currently supported.
\end{funcdesc}

\begin{funcdesc}{longimagedata}{file}
This function reads and decodes the image on the specified file, and
returns it as a Python string. The string has 4 byte RGBA pixels.
The bottom left pixel is the first in
the string. This format is suitable to pass to \function{gl.lrectwrite()},
for instance.
\end{funcdesc}

\begin{funcdesc}{longstoimage}{data, x, y, z, file}
This function writes the RGBA data in \var{data} to image
file \var{file}. \var{x} and \var{y} give the size of the image.
\var{z} is 1 if the saved image should be 1 byte greyscale, 3 if the
saved image should be 3 byte RGB data, or 4 if the saved images should
be 4 byte RGBA data.  The input data always contains 4 bytes per pixel.
These are the formats returned by \function{gl.lrectread()}.
\end{funcdesc}

\begin{funcdesc}{ttob}{flag}
This function sets a global flag which defines whether the scan lines
of the image are read or written from bottom to top (flag is zero,
compatible with SGI GL) or from top to bottom(flag is one,
compatible with X).  The default is zero.
\end{funcdesc}

\section{Standard module \sectcode{imghdr}}
\label{module-imghdr}
\stmodindex{imghdr}

The \code{imghdr} module determines the type of image contained in a
file or byte stream.

The \code{imghdr} module defines the following function:

\renewcommand{\indexsubitem}{(in module imghdr)}

\begin{funcdesc}{what}{filename\optional{\, h}}
Tests the image data contained in the file named by \var{filename},
and returns a string describing the image type.  If optional \var{h}
is provided, the \var{filename} is ignored and \var{h} is assumed to
contain the byte stream to test.
\end{funcdesc}

The following image types are recognized, as listed below with the
return value from \code{what}:

\begin{enumerate}
\item[``rgb''] SGI ImgLib Files

\item[``gif''] GIF 87a and 89a Files

\item[``pbm''] Portable Bitmap Files

\item[``pgm''] Portable Graymap Files

\item[``ppm''] Portable Pixmap Files

\item[``tiff''] TIFF Files

\item[``rast''] Sun Raster Files

\item[``xbm''] X Bitmap Files

\item[``jpeg''] JPEG data in JIFF format
\end{enumerate}

You can extend the list of file types \code{imghdr} can recognize by
appending to this variable:

\begin{datadesc}{tests}
A list of functions performing the individual tests.  Each function
takes two arguments: the byte-stream and an open file-like object.
When \code{what()} is called with a byte-stream, the file-like
object will be \code{None}.

The test function should return a string describing the image type if
the test succeeded, or \code{None} if it failed.
\end{datadesc}

Example:

\bcode\begin{verbatim}
>>> import imghdr
>>> imghdr.what('/tmp/bass.gif')
'gif'
\end{verbatim}\ecode


\chapter{Cryptographic Services}
\label{crypto}
\index{cryptography}

The modules described in this chapter implement various algorithms of
a cryptographic nature.  They are available at the discretion of the
installation.  Here's an overview:

\localmoduletable

Hardcore cypherpunks will probably find the cryptographic modules
written by Andrew Kuchling of further interest; the package adds
built-in modules for DES and IDEA encryption, provides a Python module
for reading and decrypting PGP files, and then some.  These modules
are not distributed with Python but available separately.  See the URL
\url{http://starship.python.net/crew/amk/python/crypto.html} or
send email to \email{amk1@bigfoot.com} for more information.
\index{PGP}
\index{Pretty Good Privacy}
\indexii{DES}{cipher}
\indexii{IDEA}{cipher}
\index{cryptography}
\index{Kuchling, Andrew}
		% Cryptographic Services
\section{Built-in Module \sectcode{md5}}
\bimodindex{md5}

This module implements the interface to RSA's MD5 message digest
algorithm (see also Internet RFC 1321).  Its use is quite
straightforward:\ use the \code{md5.new()} to create an md5 object.
You can now feed this object with arbitrary strings using the
\code{update()} method, and at any point you can ask it for the
\dfn{digest} (a strong kind of 128-bit checksum,
a.k.a. ``fingerprint'') of the contatenation of the strings fed to it
so far using the \code{digest()} method.

For example, to obtain the digest of the string {\tt"Nobody inspects
the spammish repetition"}:

\bcode\begin{verbatim}
>>> import md5
>>> m = md5.new()
>>> m.update("Nobody inspects")
>>> m.update(" the spammish repetition")
>>> m.digest()
'\273d\234\203\335\036\245\311\331\336\311\241\215\360\377\351'
\end{verbatim}\ecode

More condensed:

\bcode\begin{verbatim}
>>> md5.new("Nobody inspects the spammish repetition").digest()
'\273d\234\203\335\036\245\311\331\336\311\241\215\360\377\351'
\end{verbatim}\ecode

\renewcommand{\indexsubitem}{(in module md5)}

\begin{funcdesc}{new}{\optional{arg}}
Return a new md5 object.  If \var{arg} is present, the method call
\code{update(\var{arg})} is made.
\end{funcdesc}

\begin{funcdesc}{md5}{\optional{arg}}
For backward compatibility reasons, this is an alternative name for the
\code{new()} function.
\end{funcdesc}

An md5 object has the following methods:

\renewcommand{\indexsubitem}{(md5 method)}
\begin{funcdesc}{update}{arg}
Update the md5 object with the string \var{arg}.  Repeated calls are
equivalent to a single call with the concatenation of all the
arguments, i.e.\ \code{m.update(a); m.update(b)} is equivalent to
\code{m.update(a+b)}.
\end{funcdesc}

\begin{funcdesc}{digest}{}
Return the digest of the strings passed to the \code{update()}
method so far.  This is an 16-byte string which may contain
non-\ASCII{} characters, including null bytes.
\end{funcdesc}

\begin{funcdesc}{copy}{}
Return a copy (``clone'') of the md5 object.  This can be used to
efficiently compute the digests of strings that share a common initial
substring.
\end{funcdesc}

\section{Built-in Module \sectcode{mpz}}
\label{module-mpz}
\bimodindex{mpz}

This is an optional module.  It is only available when Python is
configured to include it, which requires that the GNU MP software is
installed.

This module implements the interface to part of the GNU MP library,
which defines arbitrary precision integer and rational number
arithmetic routines.  Only the interfaces to the \emph{integer}
(\samp{mpz_{\rm \ldots}}) routines are provided. If not stated
otherwise, the description in the GNU MP documentation can be applied.

In general, \dfn{mpz}-numbers can be used just like other standard
Python numbers, e.g.\ you can use the built-in operators like \code{+},
\code{*}, etc., as well as the standard built-in functions like
\code{abs}, \code{int}, \ldots, \code{divmod}, \code{pow}.
\strong{Please note:} the \emph{bitwise-xor} operation has been implemented as
a bunch of \emph{and}s, \emph{invert}s and \emph{or}s, because the library
lacks an \code{mpz_xor} function, and I didn't need one.

You create an mpz-number by calling the function called \code{mpz} (see
below for an exact description). An mpz-number is printed like this:
\code{mpz(\var{value})}.

\setindexsubitem{(in module mpz)}
\begin{funcdesc}{mpz}{value}
  Create a new mpz-number. \var{value} can be an integer, a long,
  another mpz-number, or even a string. If it is a string, it is
  interpreted as an array of radix-256 digits, least significant digit
  first, resulting in a positive number. See also the \code{binary}
  method, described below.
\end{funcdesc}

A number of \emph{extra} functions are defined in this module. Non
mpz-arguments are converted to mpz-values first, and the functions
return mpz-numbers.

\begin{funcdesc}{powm}{base, exponent, modulus}
  Return \code{pow(\var{base}, \var{exponent}) \%{} \var{modulus}}. If
  \code{\var{exponent} == 0}, return \code{mpz(1)}. In contrast to the
  \C-library function, this version can handle negative exponents.
\end{funcdesc}

\begin{funcdesc}{gcd}{op1, op2}
  Return the greatest common divisor of \var{op1} and \var{op2}.
\end{funcdesc}

\begin{funcdesc}{gcdext}{a, b}
  Return a tuple \code{(\var{g}, \var{s}, \var{t})}, such that
  \code{\var{a}*\var{s} + \var{b}*\var{t} == \var{g} == gcd(\var{a}, \var{b})}.
\end{funcdesc}

\begin{funcdesc}{sqrt}{op}
  Return the square root of \var{op}. The result is rounded towards zero.
\end{funcdesc}

\begin{funcdesc}{sqrtrem}{op}
  Return a tuple \code{(\var{root}, \var{remainder})}, such that
  \code{\var{root}*\var{root} + \var{remainder} == \var{op}}.
\end{funcdesc}

\begin{funcdesc}{divm}{numerator, denominator, modulus}
  Returns a number \var{q}. such that
  \code{\var{q} * \var{denominator} \%{} \var{modulus} == \var{numerator}}.
  One could also implement this function in Python, using \code{gcdext}.
\end{funcdesc}

An mpz-number has one method:

\setindexsubitem{(mpz method)}
\begin{funcdesc}{binary}{}
  Convert this mpz-number to a binary string, where the number has been
  stored as an array of radix-256 digits, least significant digit first.

  The mpz-number must have a value greater than or equal to zero,
  otherwise a \code{ValueError}-exception will be raised.
\end{funcdesc}

\section{Built-in Module \sectcode{rotor}}
\bimodindex{rotor}

This module implements a rotor-based encryption algorithm, contributed by
Lance Ellinghouse.  The design is derived from the Enigma device, a machine
used during World War II to encipher messages.  A rotor is simply a
permutation.  For example, if the character `A' is the origin of the rotor,
then a given rotor might map `A' to `L', `B' to `Z', `C' to `G', and so on.
To encrypt, we choose several different rotors, and set the origins of the
rotors to known positions; their initial position is the ciphering key.  To
encipher a character, we permute the original character by the first rotor,
and then apply the second rotor's permutation to the result. We continue
until we've applied all the rotors; the resulting character is our
ciphertext.  We then change the origin of the final rotor by one position,
from `A' to `B'; if the final rotor has made a complete revolution, then we
rotate the next-to-last rotor by one position, and apply the same procedure
recursively.  In other words, after enciphering one character, we advance
the rotors in the same fashion as a car's odometer. Decoding works in the
same way, except we reverse the permutations and apply them in the opposite
order.
\index{Ellinghouse, Lance}
\indexii{Enigma}{cipher}

The available functions in this module are:

\renewcommand{\indexsubitem}{(in module rotor)}
\begin{funcdesc}{newrotor}{key\optional{\, numrotors}}
Return a rotor object. \var{key} is a string containing the encryption key
for the object; it can contain arbitrary binary data. The key will be used
to randomly generate the rotor permutations and their initial positions.
\var{numrotors} is the number of rotor permutations in the returned object;
if it is omitted, a default value of 6 will be used.
\end{funcdesc}

Rotor objects have the following methods:

\renewcommand{\indexsubitem}{(rotor method)}
\begin{funcdesc}{setkey}{key}
Sets the rotor's key to \var{key}.
\end{funcdesc}

\begin{funcdesc}{encrypt}{plaintext}
Reset the rotor object to its initial state and encrypt \var{plaintext},
returning a string containing the ciphertext.  The ciphertext is always the
same length as the original plaintext.
\end{funcdesc}

\begin{funcdesc}{encryptmore}{plaintext}
Encrypt \var{plaintext} without resetting the rotor object, and return a
string containing the ciphertext.
\end{funcdesc}

\begin{funcdesc}{decrypt}{ciphertext}
Reset the rotor object to its initial state and decrypt \var{ciphertext},
returning a string containing the ciphertext.  The plaintext string will
always be the same length as the ciphertext.
\end{funcdesc}

\begin{funcdesc}{decryptmore}{ciphertext}
Decrypt \var{ciphertext} without resetting the rotor object, and return a
string containing the ciphertext.
\end{funcdesc}

An example usage:
\bcode\begin{verbatim}
>>> import rotor
>>> rt = rotor.newrotor('key', 12)
>>> rt.encrypt('bar')
'\2534\363'
>>> rt.encryptmore('bar')
'\357\375$'
>>> rt.encrypt('bar')
'\2534\363'
>>> rt.decrypt('\2534\363')
'bar'
>>> rt.decryptmore('\357\375$')
'bar'
>>> rt.decrypt('\357\375$')
'l(\315'
>>> del rt
\end{verbatim}\ecode

The module's code is not an exact simulation of the original Enigma device;
it implements the rotor encryption scheme differently from the original. The
most important difference is that in the original Enigma, there were only 5
or 6 different rotors in existence, and they were applied twice to each
character; the cipher key was the order in which they were placed in the
machine.  The Python rotor module uses the supplied key to initialize a
random number generator; the rotor permutations and their initial positions
are then randomly generated.  The original device only enciphered the
letters of the alphabet, while this module can handle any 8-bit binary data;
it also produces binary output.  This module can also operate with an
arbitrary number of rotors.

The original Enigma cipher was broken in 1944. % XXX: Is this right?
The version implemented here is probably a good deal more difficult to crack
(especially if you use many rotors), but it won't be impossible for
a truly skilful and determined attacker to break the cipher.  So if you want
to keep the NSA out of your files, this rotor cipher may well be unsafe, but
for discouraging casual snooping through your files, it will probably be
just fine, and may be somewhat safer than using the Unix \file{crypt}
command.
\index{National Security Agency}\index{crypt(1)}
% XXX How were Unix commands represented in the docs?



%\chapter{Amoeba Specific Services}

\section{Built-in Module \sectcode{amoeba}}

\bimodindex{amoeba}
This module provides some object types and operations useful for
Amoeba applications.  It is only available on systems that support
Amoeba operations.  RPC errors and other Amoeba errors are reported as
the exception \code{amoeba.error = 'amoeba.error'}.

The module \code{amoeba} defines the following items:

\renewcommand{\indexsubitem}{(in module amoeba)}
\begin{funcdesc}{name_append}{path\, cap}
Stores a capability in the Amoeba directory tree.
Arguments are the pathname (a string) and the capability (a capability
object as returned by
\code{name_lookup()}).
\end{funcdesc}

\begin{funcdesc}{name_delete}{path}
Deletes a capability from the Amoeba directory tree.
Argument is the pathname.
\end{funcdesc}

\begin{funcdesc}{name_lookup}{path}
Looks up a capability.
Argument is the pathname.
Returns a
\dfn{capability}
object, to which various interesting operations apply, described below.
\end{funcdesc}

\begin{funcdesc}{name_replace}{path\, cap}
Replaces a capability in the Amoeba directory tree.
Arguments are the pathname and the new capability.
(This differs from
\code{name_append()}
in the behavior when the pathname already exists:
\code{name_append()}
finds this an error while
\code{name_replace()}
allows it, as its name suggests.)
\end{funcdesc}

\begin{datadesc}{capv}
A table representing the capability environment at the time the
interpreter was started.
(Alas, modifying this table does not affect the capability environment
of the interpreter.)
For example,
\code{amoeba.capv['ROOT']}
is the capability of your root directory, similar to
\code{getcap("ROOT")}
in C.
\end{datadesc}

\begin{excdesc}{error}
The exception raised when an Amoeba function returns an error.
The value accompanying this exception is a pair containing the numeric
error code and the corresponding string, as returned by the C function
\code{err_why()}.
\end{excdesc}

\begin{funcdesc}{timeout}{msecs}
Sets the transaction timeout, in milliseconds.
Returns the previous timeout.
Initially, the timeout is set to 2 seconds by the Python interpreter.
\end{funcdesc}

\subsection{Capability Operations}

Capabilities are written in a convenient \ASCII{} format, also used by the
Amoeba utilities
{\it c2a}(U)
and
{\it a2c}(U).
For example:

\bcode\begin{verbatim}
>>> amoeba.name_lookup('/profile/cap')
aa:1c:95:52:6a:fa/14(ff)/8e:ba:5b:8:11:1a
>>> 
\end{verbatim}\ecode
%
The following methods are defined for capability objects.

\renewcommand{\indexsubitem}{(capability method)}
\begin{funcdesc}{dir_list}{}
Returns a list of the names of the entries in an Amoeba directory.
\end{funcdesc}

\begin{funcdesc}{b_read}{offset\, maxsize}
Reads (at most)
\var{maxsize}
bytes from a bullet file at offset
\var{offset.}
The data is returned as a string.
EOF is reported as an empty string.
\end{funcdesc}

\begin{funcdesc}{b_size}{}
Returns the size of a bullet file.
\end{funcdesc}

\begin{funcdesc}{dir_append}{}
\funcline{dir_delete}{}\ 
\funcline{dir_lookup}{}\ 
\funcline{dir_replace}{}
Like the corresponding
\samp{name_}*
functions, but with a path relative to the capability.
(For paths beginning with a slash the capability is ignored, since this
is the defined semantics for Amoeba.)
\end{funcdesc}

\begin{funcdesc}{std_info}{}
Returns the standard info string of the object.
\end{funcdesc}

\begin{funcdesc}{tod_gettime}{}
Returns the time (in seconds since the Epoch, in UCT, as for POSIX) from
a time server.
\end{funcdesc}

\begin{funcdesc}{tod_settime}{t}
Sets the time kept by a time server.
\end{funcdesc}
		% AMOEBA ONLY

%\chapter{Standard Windowing Interface}

The modules in this chapter are available only on those systems where
the STDWIN library is available.  STDWIN runs on \UNIX{} under X11 and
on the Macintosh.  See CWI report CS-R8817.

\strong{Warning:} Using STDWIN is not recommended for new
applications.  It has never been ported to Microsoft Windows or
Windows NT, and for X11 or the Macintosh it lacks important
functionality --- in particular, it has no tools for the construction
of dialogs.  For most platforms, alternative, native solutions exist
(though none are currently documented in this manual): Tkinter for
\UNIX{} under X11, native Xt with Motif or Athena widgets for \UNIX{}
under X11, Win32 for Windows and Windows NT, and a collection of
native toolkit interfaces for the Macintosh.

\section{\module{stdwin} ---
         Platform-independent GUI System}
\declaremodule{builtin}{stdwin}

\modulesynopsis{Older GUI system for X11 and Macintosh}


This module defines several new object types and functions that
provide access to the functionality of STDWIN.

On \UNIX{} running X11, it can only be used if the \envvar{DISPLAY}
environment variable is set or an explicit \samp{-display
\var{displayname}} argument is passed to the Python interpreter.

Functions have names that usually resemble their C STDWIN counterparts
with the initial `w' dropped.
Points are represented by pairs of integers; rectangles
by pairs of points.
For a complete description of STDWIN please refer to the documentation
of STDWIN for C programmers (aforementioned CWI report).

\subsection{Functions Defined in Module \module{stdwin}}
\nodename{STDWIN Functions}

The following functions are defined in the \module{stdwin} module:

\begin{funcdesc}{open}{title}
Open a new window whose initial title is given by the string argument.
Return a window object; window object methods are described below.%
\footnote{The Python version of STDWIN does not support draw procedures; all
	drawing requests are reported as draw events.}
\end{funcdesc}

\begin{funcdesc}{getevent}{}
Wait for and return the next event.
An event is returned as a triple: the first element is the event
type, a small integer; the second element is the window object to which
the event applies, or
\code{None}
if it applies to no window in particular;
the third element is type-dependent.
Names for event types and command codes are defined in the standard
module \module{stdwinevent}.
\end{funcdesc}

\begin{funcdesc}{pollevent}{}
Return the next event, if one is immediately available.
If no event is available, return \code{()}.
\end{funcdesc}

\begin{funcdesc}{getactive}{}
Return the window that is currently active, or \code{None} if no
window is currently active.  (This can be emulated by monitoring
WE_ACTIVATE and WE_DEACTIVATE events.)
\end{funcdesc}

\begin{funcdesc}{listfontnames}{pattern}
Return the list of font names in the system that match the pattern (a
string).  The pattern should normally be \code{'*'}; returns all
available fonts.  If the underlying window system is X11, other
patterns follow the standard X11 font selection syntax (as used e.g.
in resource definitions), i.e. the wildcard character \code{'*'}
matches any sequence of characters (including none) and \code{'?'}
matches any single character.
On the Macintosh this function currently returns an empty list.
\end{funcdesc}

\begin{funcdesc}{setdefscrollbars}{hflag, vflag}
Set the flags controlling whether subsequently opened windows will
have horizontal and/or vertical scroll bars.
\end{funcdesc}

\begin{funcdesc}{setdefwinpos}{h, v}
Set the default window position for windows opened subsequently.
\end{funcdesc}

\begin{funcdesc}{setdefwinsize}{width, height}
Set the default window size for windows opened subsequently.
\end{funcdesc}

\begin{funcdesc}{getdefscrollbars}{}
Return the flags controlling whether subsequently opened windows will
have horizontal and/or vertical scroll bars.
\end{funcdesc}

\begin{funcdesc}{getdefwinpos}{}
Return the default window position for windows opened subsequently.
\end{funcdesc}

\begin{funcdesc}{getdefwinsize}{}
Return the default window size for windows opened subsequently.
\end{funcdesc}

\begin{funcdesc}{getscrsize}{}
Return the screen size in pixels.
\end{funcdesc}

\begin{funcdesc}{getscrmm}{}
Return the screen size in millimeters.
\end{funcdesc}

\begin{funcdesc}{fetchcolor}{colorname}
Return the pixel value corresponding to the given color name.
Return the default foreground color for unknown color names.
Hint: the following code tests whether you are on a machine that
supports more than two colors:
\begin{verbatim}
if stdwin.fetchcolor('black') <> \
          stdwin.fetchcolor('red') <> \
          stdwin.fetchcolor('white'):
    print 'color machine'
else:
    print 'monochrome machine'
\end{verbatim}
\end{funcdesc}

\begin{funcdesc}{setfgcolor}{pixel}
Set the default foreground color.
This will become the default foreground color of windows opened
subsequently, including dialogs.
\end{funcdesc}

\begin{funcdesc}{setbgcolor}{pixel}
Set the default background color.
This will become the default background color of windows opened
subsequently, including dialogs.
\end{funcdesc}

\begin{funcdesc}{getfgcolor}{}
Return the pixel value of the current default foreground color.
\end{funcdesc}

\begin{funcdesc}{getbgcolor}{}
Return the pixel value of the current default background color.
\end{funcdesc}

\begin{funcdesc}{setfont}{fontname}
Set the current default font.
This will become the default font for windows opened subsequently,
and is also used by the text measuring functions \function{textwidth()},
\function{textbreak()}, \function{lineheight()} and
\function{baseline()} below.  This accepts two more optional
parameters, size and style:  Size is the font size (in `points').
Style is a single character specifying the style, as follows:
\code{'b'} = bold,
\code{'i'} = italic,
\code{'o'} = bold + italic,
\code{'u'} = underline;
default style is roman.
Size and style are ignored under X11 but used on the Macintosh.
(Sorry for all this complexity --- a more uniform interface is being designed.)
\end{funcdesc}

\begin{funcdesc}{menucreate}{title}
Create a menu object referring to a global menu (a menu that appears in
all windows).
Methods of menu objects are described below.
Note: normally, menus are created locally; see the window method
\method{menucreate()} below.
\strong{Warning:} the menu only appears in a window as long as the object
returned by this call exists.
\end{funcdesc}

\begin{funcdesc}{newbitmap}{width, height}
Create a new bitmap object of the given dimensions.
Methods of bitmap objects are described below.
Not available on the Macintosh.
\end{funcdesc}

\begin{funcdesc}{fleep}{}
Cause a beep or bell (or perhaps a `visual bell' or flash, hence the
name).
\end{funcdesc}

\begin{funcdesc}{message}{string}
Display a dialog box containing the string.
The user must click OK before the function returns.
\end{funcdesc}

\begin{funcdesc}{askync}{prompt, default}
Display a dialog that prompts the user to answer a question with yes or
no.  Return 0 for no, 1 for yes.  If the user hits the Return key, the
default (which must be 0 or 1) is returned.  If the user cancels the
dialog, \exception{KeyboardInterrupt} is raised.
\end{funcdesc}

\begin{funcdesc}{askstr}{prompt, default}
Display a dialog that prompts the user for a string.
If the user hits the Return key, the default string is returned.
If the user cancels the dialog, \exception{KeyboardInterrupt} is
raised.
\end{funcdesc}

\begin{funcdesc}{askfile}{prompt, default, new}
Ask the user to specify a filename.  If \var{new} is zero it must be
an existing file; otherwise, it must be a new file.  If the user
cancels the dialog, \exception{KeyboardInterrupt} is raised.
\end{funcdesc}

\begin{funcdesc}{setcutbuffer}{i, string}
Store the string in the system's cut buffer number \var{i}, where it
can be found (for pasting) by other applications.  On X11, there are 8
cut buffers (numbered 0..7).  Cut buffer number 0 is the `clipboard'
on the Macintosh.
\end{funcdesc}

\begin{funcdesc}{getcutbuffer}{i}
Return the contents of the system's cut buffer number \var{i}.
\end{funcdesc}

\begin{funcdesc}{rotatecutbuffers}{n}
On X11, rotate the 8 cut buffers by \var{n}.  Ignored on the
Macintosh.
\end{funcdesc}

\begin{funcdesc}{getselection}{i}
Return X11 selection number \var{i.}  Selections are not cut buffers.
Selection numbers are defined in module \module{stdwinevents}.
Selection \constant{WS_PRIMARY} is the \dfn{primary} selection (used
by \program{xterm}, for instance); selection \constant{WS_SECONDARY}
is the \dfn{secondary} selection; selection \constant{WS_CLIPBOARD} is
the \dfn{clipboard} selection (used by \program{xclipboard}).  On the
Macintosh, this always returns an empty string.
\end{funcdesc}

\begin{funcdesc}{resetselection}{i}
Reset selection number \var{i}, if this process owns it.  (See window
method \method{setselection()}).
\end{funcdesc}

\begin{funcdesc}{baseline}{}
Return the baseline of the current font (defined by STDWIN as the
vertical distance between the baseline and the top of the
characters).
\end{funcdesc}

\begin{funcdesc}{lineheight}{}
Return the total line height of the current font.
\end{funcdesc}

\begin{funcdesc}{textbreak}{str, width}
Return the number of characters of the string that fit into a space of
\var{width}
bits wide when drawn in the curent font.
\end{funcdesc}

\begin{funcdesc}{textwidth}{str}
Return the width in bits of the string when drawn in the current font.
\end{funcdesc}

\begin{funcdesc}{connectionnumber}{}
\funcline{fileno}{}
(X11 under \UNIX{} only) Return the ``connection number'' used by the
underlying X11 implementation.  (This is normally the file number of
the socket.)  Both functions return the same value;
\method{connectionnumber()} is named after the corresponding function in
X11 and STDWIN, while \method{fileno()} makes it possible to use the
\module{stdwin} module as a ``file'' object parameter to
\function{select.select()}.  Note that if \constant{select()} implies that
input is possible on \module{stdwin}, this does not guarantee that an
event is ready --- it may be some internal communication going on
between the X server and the client library.  Thus, you should call
\function{stdwin.pollevent()} until it returns \code{None} to check for
events if you don't want your program to block.  Because of internal
buffering in X11, it is also possible that \function{stdwin.pollevent()}
returns an event while \function{select()} does not find \module{stdwin} to
be ready, so you should read any pending events with
\function{stdwin.pollevent()} until it returns \code{None} before entering
a blocking \function{select()} call.
\withsubitem{(in module select)}{\ttindex{select()}}
\end{funcdesc}

\subsection{Window Objects}
\nodename{STDWIN Window Objects}

Window objects are created by \function{stdwin.open()}.  They are closed
by their \method{close()} method or when they are garbage-collected.
Window objects have the following methods:

\begin{methoddesc}[window]{begindrawing}{}
Return a drawing object, whose methods (described below) allow drawing
in the window.
\end{methoddesc}

\begin{methoddesc}[window]{change}{rect}
Invalidate the given rectangle; this may cause a draw event.
\end{methoddesc}

\begin{methoddesc}[window]{gettitle}{}
Returns the window's title string.
\end{methoddesc}

\begin{methoddesc}[window]{getdocsize}{}
\begin{sloppypar}
Return a pair of integers giving the size of the document as set by
\method{setdocsize()}.
\end{sloppypar}
\end{methoddesc}

\begin{methoddesc}[window]{getorigin}{}
Return a pair of integers giving the origin of the window with respect
to the document.
\end{methoddesc}

\begin{methoddesc}[window]{gettitle}{}
Return the window's title string.
\end{methoddesc}

\begin{methoddesc}[window]{getwinsize}{}
Return a pair of integers giving the size of the window.
\end{methoddesc}

\begin{methoddesc}[window]{getwinpos}{}
Return a pair of integers giving the position of the window's upper
left corner (relative to the upper left corner of the screen).
\end{methoddesc}

\begin{methoddesc}[window]{menucreate}{title}
Create a menu object referring to a local menu (a menu that appears
only in this window).
Methods of menu objects are described below.
\strong{Warning:} the menu only appears as long as the object
returned by this call exists.
\end{methoddesc}

\begin{methoddesc}[window]{scroll}{rect, point}
Scroll the given rectangle by the vector given by the point.
\end{methoddesc}

\begin{methoddesc}[window]{setdocsize}{point}
Set the size of the drawing document.
\end{methoddesc}

\begin{methoddesc}[window]{setorigin}{point}
Move the origin of the window (its upper left corner)
to the given point in the document.
\end{methoddesc}

\begin{methoddesc}[window]{setselection}{i, str}
Attempt to set X11 selection number \var{i} to the string \var{str}.
(See \module{stdwin} function \function{getselection()} for the
meaning of \var{i}.)  Return true if it succeeds.
If  succeeds, the window ``owns'' the selection until
(a) another application takes ownership of the selection; or
(b) the window is deleted; or
(c) the application clears ownership by calling
\function{stdwin.resetselection(\var{i})}.  When another application
takes ownership of the selection, a \constant{WE_LOST_SEL} event is
received for no particular window and with the selection number as
detail.  Ignored on the Macintosh.
\end{methoddesc}

\begin{methoddesc}[window]{settimer}{dsecs}
Schedule a timer event for the window in \code{\var{dsecs}/10}
seconds.
\end{methoddesc}

\begin{methoddesc}[window]{settitle}{title}
Set the window's title string.
\end{methoddesc}

\begin{methoddesc}[window]{setwincursor}{name}
\begin{sloppypar}
Set the window cursor to a cursor of the given name.  It raises
\exception{RuntimeError} if no cursor of the given name exists.
Suitable names include
\code{'ibeam'},
\code{'arrow'},
\code{'cross'},
\code{'watch'}
and
\code{'plus'}.
On X11, there are many more (see \code{<X11/cursorfont.h>}).
\end{sloppypar}
\end{methoddesc}

\begin{methoddesc}[window]{setwinpos}{h, v}
Set the the position of the window's upper left corner (relative to
the upper left corner of the screen).
\end{methoddesc}

\begin{methoddesc}[window]{setwinsize}{width, height}
Set the window's size.
\end{methoddesc}

\begin{methoddesc}[window]{show}{rect}
Try to ensure that the given rectangle of the document is visible in
the window.
\end{methoddesc}

\begin{methoddesc}[window]{textcreate}{rect}
Create a text-edit object in the document at the given rectangle.
Methods of text-edit objects are described below.
\end{methoddesc}

\begin{methoddesc}[window]{setactive}{}
Attempt to make this window the active window.  If successful, this
will generate a WE_ACTIVATE event (and a WE_DEACTIVATE event in case
another window in this application became inactive).
\end{methoddesc}

\begin{methoddesc}[window]{close}{}
Discard the window object.  It should not be used again.
\end{methoddesc}

\subsection{Drawing Objects}

Drawing objects are created exclusively by the window method
\method{begindrawing()}.  Only one drawing object can exist at any
given time; the drawing object must be deleted to finish drawing.  No
drawing object may exist when \function{stdwin.getevent()} is called.
Drawing objects have the following methods:

\begin{methoddesc}[drawing]{box}{rect}
Draw a box just inside a rectangle.
\end{methoddesc}

\begin{methoddesc}[drawing]{circle}{center, radius}
Draw a circle with given center point and radius.
\end{methoddesc}

\begin{methoddesc}[drawing]{elarc}{center, (rh, rv), (a1, a2)}
Draw an elliptical arc with given center point.
\code{(\var{rh}, \var{rv})}
gives the half sizes of the horizontal and vertical radii.
\code{(\var{a1}, \var{a2})}
gives the angles (in degrees) of the begin and end points.
0 degrees is at 3 o'clock, 90 degrees is at 12 o'clock.
\end{methoddesc}

\begin{methoddesc}[drawing]{erase}{rect}
Erase a rectangle.
\end{methoddesc}

\begin{methoddesc}[drawing]{fillcircle}{center, radius}
Draw a filled circle with given center point and radius.
\end{methoddesc}

\begin{methoddesc}[drawing]{fillelarc}{center, (rh, rv), (a1, a2)}
Draw a filled elliptical arc; arguments as for \method{elarc()}.
\end{methoddesc}

\begin{methoddesc}[drawing]{fillpoly}{points}
Draw a filled polygon given by a list (or tuple) of points.
\end{methoddesc}

\begin{methoddesc}[drawing]{invert}{rect}
Invert a rectangle.
\end{methoddesc}

\begin{methoddesc}[drawing]{line}{p1, p2}
Draw a line from point
\var{p1}
to
\var{p2}.
\end{methoddesc}

\begin{methoddesc}[drawing]{paint}{rect}
Fill a rectangle.
\end{methoddesc}

\begin{methoddesc}[drawing]{poly}{points}
Draw the lines connecting the given list (or tuple) of points.
\end{methoddesc}

\begin{methoddesc}[drawing]{shade}{rect, percent}
Fill a rectangle with a shading pattern that is about
\var{percent}
percent filled.
\end{methoddesc}

\begin{methoddesc}[drawing]{text}{p, str}
Draw a string starting at point p (the point specifies the
top left coordinate of the string).
\end{methoddesc}

\begin{methoddesc}[drawing]{xorcircle}{center, radius}
\funcline{xorelarc}{center, (rh, rv), (a1, a2)}
\funcline{xorline}{p1, p2}
\funcline{xorpoly}{points}
Draw a circle, an elliptical arc, a line or a polygon, respectively,
in XOR mode.
\end{methoddesc}

\begin{methoddesc}[drawing]{setfgcolor}{}
\funcline{setbgcolor}{}
\funcline{getfgcolor}{}
\funcline{getbgcolor}{}
These functions are similar to the corresponding functions described
above for the \module{stdwin}
module, but affect or return the colors currently used for drawing
instead of the global default colors.
When a drawing object is created, its colors are set to the window's
default colors, which are in turn initialized from the global default
colors when the window is created.
\end{methoddesc}

\begin{methoddesc}[drawing]{setfont}{}
\funcline{baseline}{}
\funcline{lineheight}{}
\funcline{textbreak}{}
\funcline{textwidth}{}
These functions are similar to the corresponding functions described
above for the \module{stdwin}
module, but affect or use the current drawing font instead of
the global default font.
When a drawing object is created, its font is set to the window's
default font, which is in turn initialized from the global default
font when the window is created.
\end{methoddesc}

\begin{methoddesc}[drawing]{bitmap}{point, bitmap, mask}
Draw the \var{bitmap} with its top left corner at \var{point}.
If the optional \var{mask} argument is present, it should be either
the same object as \var{bitmap}, to draw only those bits that are set
in the bitmap, in the foreground color, or \code{None}, to draw all
bits (ones are drawn in the foreground color, zeros in the background
color).
Not available on the Macintosh.
\end{methoddesc}

\begin{methoddesc}[drawing]{cliprect}{rect}
Set the ``clipping region'' to a rectangle.
The clipping region limits the effect of all drawing operations, until
it is changed again or until the drawing object is closed.  When a
drawing object is created the clipping region is set to the entire
window.  When an object to be drawn falls partly outside the clipping
region, the set of pixels drawn is the intersection of the clipping
region and the set of pixels that would be drawn by the same operation
in the absence of a clipping region.
\end{methoddesc}

\begin{methoddesc}[drawing]{noclip}{}
Reset the clipping region to the entire window.
\end{methoddesc}

\begin{methoddesc}[drawing]{close}{}
\funcline{enddrawing}{}
Discard the drawing object.  It should not be used again.
\end{methoddesc}

\subsection{Menu Objects}

A menu object represents a menu.
The menu is destroyed when the menu object is deleted.
The following methods are defined:


\begin{methoddesc}[menu]{additem}{text, shortcut}
Add a menu item with given text.
The shortcut must be a string of length 1, or omitted (to specify no
shortcut).
\end{methoddesc}

\begin{methoddesc}[menu]{setitem}{i, text}
Set the text of item number \var{i}.
\end{methoddesc}

\begin{methoddesc}[menu]{enable}{i, flag}
Enable or disables item \var{i}.
\end{methoddesc}

\begin{methoddesc}[menu]{check}{i, flag}
Set or clear the \dfn{check mark} for item \var{i}.
\end{methoddesc}

\begin{methoddesc}[menu]{close}{}
Discard the menu object.  It should not be used again.
\end{methoddesc}

\subsection{Bitmap Objects}

A bitmap represents a rectangular array of bits.
The top left bit has coordinate (0, 0).
A bitmap can be drawn with the \method{bitmap()} method of a drawing object.
Bitmaps are currently not available on the Macintosh.

The following methods are defined:


\begin{methoddesc}[bitmap]{getsize}{}
Return a tuple representing the width and height of the bitmap.
(This returns the values that have been passed to the
\function{newbitmap()} function.)
\end{methoddesc}

\begin{methoddesc}[bitmap]{setbit}{point, bit}
Set the value of the bit indicated by \var{point} to \var{bit}.
\end{methoddesc}

\begin{methoddesc}[bitmap]{getbit}{point}
Return the value of the bit indicated by \var{point}.
\end{methoddesc}

\begin{methoddesc}[bitmap]{close}{}
Discard the bitmap object.  It should not be used again.
\end{methoddesc}

\subsection{Text-edit Objects}

A text-edit object represents a text-edit block.
For semantics, see the STDWIN documentation for \C{} programmers.
The following methods exist:


\begin{methoddesc}[text-edit]{arrow}{code}
Pass an arrow event to the text-edit block.
The \var{code} must be one of \constant{WC_LEFT}, \constant{WC_RIGHT}, 
\constant{WC_UP} or \constant{WC_DOWN} (see module
\module{stdwinevents}).
\end{methoddesc}

\begin{methoddesc}[text-edit]{draw}{rect}
Pass a draw event to the text-edit block.
The rectangle specifies the redraw area.
\end{methoddesc}

\begin{methoddesc}[text-edit]{event}{type, window, detail}
Pass an event gotten from
\function{stdwin.getevent()}
to the text-edit block.
Return true if the event was handled.
\end{methoddesc}

\begin{methoddesc}[text-edit]{getfocus}{}
Return 2 integers representing the start and end positions of the
focus, usable as slice indices on the string returned by
\method{gettext()}.
\end{methoddesc}

\begin{methoddesc}[text-edit]{getfocustext}{}
Return the text in the focus.
\end{methoddesc}

\begin{methoddesc}[text-edit]{getrect}{}
Return a rectangle giving the actual position of the text-edit block.
(The bottom coordinate may differ from the initial position because
the block automatically shrinks or grows to fit.)
\end{methoddesc}

\begin{methoddesc}[text-edit]{gettext}{}
Return the entire text buffer.
\end{methoddesc}

\begin{methoddesc}[text-edit]{move}{rect}
Specify a new position for the text-edit block in the document.
\end{methoddesc}

\begin{methoddesc}[text-edit]{replace}{str}
Replace the text in the focus by the given string.
The new focus is an insert point at the end of the string.
\end{methoddesc}

\begin{methoddesc}[text-edit]{setfocus}{i, j}
Specify the new focus.
Out-of-bounds values are silently clipped.
\end{methoddesc}

\begin{methoddesc}[text-edit]{settext}{str}
Replace the entire text buffer by the given string and set the focus
to \code{(0, 0)}.
\end{methoddesc}

\begin{methoddesc}[text-edit]{setview}{rect}
Set the view rectangle to \var{rect}.  If \var{rect} is \code{None},
viewing mode is reset.  In viewing mode, all output from the text-edit
object is clipped to the viewing rectangle.  This may be useful to
implement your own scrolling text subwindow.
\end{methoddesc}

\begin{methoddesc}[text-edit]{close}{}
Discard the text-edit object.  It should not be used again.
\end{methoddesc}

\subsection{Example}
\nodename{STDWIN Example}

Here is a minimal example of using STDWIN in Python.
It creates a window and draws the string ``Hello world'' in the top
left corner of the window.
The window will be correctly redrawn when covered and re-exposed.
The program quits when the close icon or menu item is requested.

\begin{verbatim}
import stdwin
from stdwinevents import *

def main():
    mywin = stdwin.open('Hello')
    #
    while 1:
        (type, win, detail) = stdwin.getevent()
        if type == WE_DRAW:
            draw = win.begindrawing()
            draw.text((0, 0), 'Hello, world')
            del draw
        elif type == WE_CLOSE:
            break

main()
\end{verbatim}

\section{\module{stdwinevents} ---
         Constants for use with \module{stdwin}}
\declaremodule{standard}{stdwinevents}

\modulesynopsis{Constant definitions for use with \module{stdwin}}


This module defines constants used by STDWIN for event types
(\constant{WE_ACTIVATE} etc.), command codes (\constant{WC_LEFT} etc.)
and selection types (\constant{WS_PRIMARY} etc.).
Read the file for details.
Suggested usage is

\begin{verbatim}
>>> from stdwinevents import *
>>> 
\end{verbatim}

\section{\module{rect} ---
         Functions for use with \module{stdwin}}
\declaremodule{standard}{rect}

\modulesynopsis{Geometry-related utility function for use with \module{stdwin}}


This module contains useful operations on rectangles.
A rectangle is defined as in module \module{stdwin}:
a pair of points, where a point is a pair of integers.
For example, the rectangle

\begin{verbatim}
(10, 20), (90, 80)
\end{verbatim}

is a rectangle whose left, top, right and bottom edges are 10, 20, 90
and 80, respectively.  Note that the positive vertical axis points
down (as in \module{stdwin}).

The module defines the following objects:

\begin{excdesc}{error}
The exception raised by functions in this module when they detect an
error.  The exception argument is a string describing the problem in
more detail.
\end{excdesc}

\begin{datadesc}{empty}
The rectangle returned when some operations return an empty result.
This makes it possible to quickly check whether a result is empty:

\begin{verbatim}
>>> import rect
>>> r1 = (10, 20), (90, 80)
>>> r2 = (0, 0), (10, 20)
>>> r3 = rect.intersect([r1, r2])
>>> if r3 is rect.empty: print 'Empty intersection'
Empty intersection
>>> 
\end{verbatim}
\end{datadesc}

\begin{funcdesc}{is_empty}{r}
Returns true if the given rectangle is empty.
A rectangle
\code{(\var{left}, \var{top}), (\var{right}, \var{bottom})}
is empty if
\begin{math}\var{left} \geq \var{right}\end{math} or
\begin{math}\var{top} \geq \var{bottom}\end{math}.
\end{funcdesc}

\begin{funcdesc}{intersect}{list}
Returns the intersection of all rectangles in the list argument.
It may also be called with a tuple argument.  Raises
\exception{rect.error} if the list is empty.  Returns
\constant{rect.empty} if the intersection of the rectangles is empty.
\end{funcdesc}

\begin{funcdesc}{union}{list}
Returns the smallest rectangle that contains all non-empty rectangles in
the list argument.  It may also be called with a tuple argument or
with two or more rectangles as arguments.  Returns
\constant{rect.empty} if the list is empty or all its rectangles are
empty.
\end{funcdesc}

\begin{funcdesc}{pointinrect}{point, rect}
Returns true if the point is inside the rectangle.  By definition, a
point \code{(\var{h}, \var{v})} is inside a rectangle
\code{(\var{left}, \var{top}), (\var{right}, \var{bottom})} if
\begin{math}\var{left} \leq \var{h} < \var{right}\end{math} and
\begin{math}\var{top} \leq \var{v} < \var{bottom}\end{math}.
\end{funcdesc}

\begin{funcdesc}{inset}{rect, (dh, dv)}
Returns a rectangle that lies inside the \var{rect} argument by
\var{dh} pixels horizontally and \var{dv} pixels vertically.  If
\var{dh} or \var{dv} is negative, the result lies outside \var{rect}.
\end{funcdesc}

\begin{funcdesc}{rect2geom}{rect}
Converts a rectangle to geometry representation:
\code{(\var{left}, \var{top}), (\var{width}, \var{height})}.
\end{funcdesc}

\begin{funcdesc}{geom2rect}{geom}
Converts a rectangle given in geometry representation back to the
standard rectangle representation
\code{(\var{left}, \var{top}), (\var{right}, \var{bottom})}.
\end{funcdesc}
		% STDWIN ONLY

\chapter{SGI IRIX Specific Services}
\label{sgi}

The modules described in this chapter provide interfaces to features
that are unique to SGI's IRIX operating system (versions 4 and 5).

\localmoduletable
			% SGI IRIX ONLY
\section{\module{al} ---
         Audio functions on the SGI}

\declaremodule{builtin}{al}
  \platform{IRIX}
\modulesynopsis{Audio functions on the SGI.}


This module provides access to the audio facilities of the SGI Indy
and Indigo workstations.  See section 3A of the IRIX man pages for
details.  You'll need to read those man pages to understand what these
functions do!  Some of the functions are not available in IRIX
releases before 4.0.5.  Again, see the manual to check whether a
specific function is available on your platform.

All functions and methods defined in this module are equivalent to
the C functions with \samp{AL} prefixed to their name.

Symbolic constants from the C header file \code{<audio.h>} are
defined in the standard module \module{AL}\refstmodindex{AL}, see
below.

\strong{Warning:} the current version of the audio library may dump core
when bad argument values are passed rather than returning an error
status.  Unfortunately, since the precise circumstances under which
this may happen are undocumented and hard to check, the Python
interface can provide no protection against this kind of problems.
(One example is specifying an excessive queue size --- there is no
documented upper limit.)

The module defines the following functions:


\begin{funcdesc}{openport}{name, direction\optional{, config}}
The name and direction arguments are strings.  The optional
\var{config} argument is a configuration object as returned by
\function{newconfig()}.  The return value is an \dfn{audio port
object}; methods of audio port objects are described below.
\end{funcdesc}

\begin{funcdesc}{newconfig}{}
The return value is a new \dfn{audio configuration object}; methods of
audio configuration objects are described below.
\end{funcdesc}

\begin{funcdesc}{queryparams}{device}
The device argument is an integer.  The return value is a list of
integers containing the data returned by \cfunction{ALqueryparams()}.
\end{funcdesc}

\begin{funcdesc}{getparams}{device, list}
The \var{device} argument is an integer.  The list argument is a list
such as returned by \function{queryparams()}; it is modified in place
(!).
\end{funcdesc}

\begin{funcdesc}{setparams}{device, list}
The \var{device} argument is an integer.  The \var{list} argument is a
list such as returned by \function{queryparams()}.
\end{funcdesc}


\subsection{Configuration Objects \label{al-config-objects}}

Configuration objects (returned by \function{newconfig()} have the
following methods:

\begin{methoddesc}[audio configuration]{getqueuesize}{}
Return the queue size.
\end{methoddesc}

\begin{methoddesc}[audio configuration]{setqueuesize}{size}
Set the queue size.
\end{methoddesc}

\begin{methoddesc}[audio configuration]{getwidth}{}
Get the sample width.
\end{methoddesc}

\begin{methoddesc}[audio configuration]{setwidth}{width}
Set the sample width.
\end{methoddesc}

\begin{methoddesc}[audio configuration]{getchannels}{}
Get the channel count.
\end{methoddesc}

\begin{methoddesc}[audio configuration]{setchannels}{nchannels}
Set the channel count.
\end{methoddesc}

\begin{methoddesc}[audio configuration]{getsampfmt}{}
Get the sample format.
\end{methoddesc}

\begin{methoddesc}[audio configuration]{setsampfmt}{sampfmt}
Set the sample format.
\end{methoddesc}

\begin{methoddesc}[audio configuration]{getfloatmax}{}
Get the maximum value for floating sample formats.
\end{methoddesc}

\begin{methoddesc}[audio configuration]{setfloatmax}{floatmax}
Set the maximum value for floating sample formats.
\end{methoddesc}


\subsection{Port Objects \label{al-port-objects}}

Port objects, as returned by \function{openport()}, have the following
methods:

\begin{methoddesc}[audio port]{closeport}{}
Close the port.
\end{methoddesc}

\begin{methoddesc}[audio port]{getfd}{}
Return the file descriptor as an int.
\end{methoddesc}

\begin{methoddesc}[audio port]{getfilled}{}
Return the number of filled samples.
\end{methoddesc}

\begin{methoddesc}[audio port]{getfillable}{}
Return the number of fillable samples.
\end{methoddesc}

\begin{methoddesc}[audio port]{readsamps}{nsamples}
Read a number of samples from the queue, blocking if necessary.
Return the data as a string containing the raw data, (e.g., 2 bytes per
sample in big-endian byte order (high byte, low byte) if you have set
the sample width to 2 bytes).
\end{methoddesc}

\begin{methoddesc}[audio port]{writesamps}{samples}
Write samples into the queue, blocking if necessary.  The samples are
encoded as described for the \method{readsamps()} return value.
\end{methoddesc}

\begin{methoddesc}[audio port]{getfillpoint}{}
Return the `fill point'.
\end{methoddesc}

\begin{methoddesc}[audio port]{setfillpoint}{fillpoint}
Set the `fill point'.
\end{methoddesc}

\begin{methoddesc}[audio port]{getconfig}{}
Return a configuration object containing the current configuration of
the port.
\end{methoddesc}

\begin{methoddesc}[audio port]{setconfig}{config}
Set the configuration from the argument, a configuration object.
\end{methoddesc}

\begin{methoddesc}[audio port]{getstatus}{list}
Get status information on last error.
\end{methoddesc}


\section{\module{AL} ---
         Constants used with the \module{al} module}

\declaremodule[al-constants]{standard}{AL}
  \platform{IRIX}
\modulesynopsis{Constants used with the \module{al} module.}


This module defines symbolic constants needed to use the built-in
module \module{al} (see above); they are equivalent to those defined
in the C header file \code{<audio.h>} except that the name prefix
\samp{AL_} is omitted.  Read the module source for a complete list of
the defined names.  Suggested use:

\begin{verbatim}
import al
from AL import *
\end{verbatim}

\section{\module{cd} ---
         CD-ROM access on SGI systems}

\declaremodule{builtin}{cd}
  \platform{IRIX}
\modulesynopsis{Interface to the CD-ROM on Silicon Graphics systems.}


This module provides an interface to the Silicon Graphics CD library.
It is available only on Silicon Graphics systems.

The way the library works is as follows.  A program opens the CD-ROM
device with \function{open()} and creates a parser to parse the data
from the CD with \function{createparser()}.  The object returned by
\function{open()} can be used to read data from the CD, but also to get
status information for the CD-ROM device, and to get information about
the CD, such as the table of contents.  Data from the CD is passed to
the parser, which parses the frames, and calls any callback
functions that have previously been added.

An audio CD is divided into \dfn{tracks} or \dfn{programs} (the terms
are used interchangeably).  Tracks can be subdivided into
\dfn{indices}.  An audio CD contains a \dfn{table of contents} which
gives the starts of the tracks on the CD.  Index 0 is usually the
pause before the start of a track.  The start of the track as given by
the table of contents is normally the start of index 1.

Positions on a CD can be represented in two ways.  Either a frame
number or a tuple of three values, minutes, seconds and frames.  Most
functions use the latter representation.  Positions can be both
relative to the beginning of the CD, and to the beginning of the
track.

Module \module{cd} defines the following functions and constants:


\begin{funcdesc}{createparser}{}
Create and return an opaque parser object.  The methods of the parser
object are described below.
\end{funcdesc}

\begin{funcdesc}{msftoframe}{minutes, seconds, frames}
Converts a \code{(\var{minutes}, \var{seconds}, \var{frames})} triple
representing time in absolute time code into the corresponding CD
frame number.
\end{funcdesc}

\begin{funcdesc}{open}{\optional{device\optional{, mode}}}
Open the CD-ROM device.  The return value is an opaque player object;
methods of the player object are described below.  The device is the
name of the SCSI device file, e.g. \code{'/dev/scsi/sc0d4l0'}, or
\code{None}.  If omitted or \code{None}, the hardware inventory is
consulted to locate a CD-ROM drive.  The \var{mode}, if not omitted,
should be the string \code{'r'}.
\end{funcdesc}

The module defines the following variables:

\begin{excdesc}{error}
Exception raised on various errors.
\end{excdesc}

\begin{datadesc}{DATASIZE}
The size of one frame's worth of audio data.  This is the size of the
audio data as passed to the callback of type \code{audio}.
\end{datadesc}

\begin{datadesc}{BLOCKSIZE}
The size of one uninterpreted frame of audio data.
\end{datadesc}

The following variables are states as returned by
\function{getstatus()}:

\begin{datadesc}{READY}
The drive is ready for operation loaded with an audio CD.
\end{datadesc}

\begin{datadesc}{NODISC}
The drive does not have a CD loaded.
\end{datadesc}

\begin{datadesc}{CDROM}
The drive is loaded with a CD-ROM.  Subsequent play or read operations
will return I/O errors.
\end{datadesc}

\begin{datadesc}{ERROR}
An error occurred while trying to read the disc or its table of
contents.
\end{datadesc}

\begin{datadesc}{PLAYING}
The drive is in CD player mode playing an audio CD through its audio
jacks.
\end{datadesc}

\begin{datadesc}{PAUSED}
The drive is in CD layer mode with play paused.
\end{datadesc}

\begin{datadesc}{STILL}
The equivalent of \constant{PAUSED} on older (non 3301) model Toshiba
CD-ROM drives.  Such drives have never been shipped by SGI.
\end{datadesc}

\begin{datadesc}{audio}
\dataline{pnum}
\dataline{index}
\dataline{ptime}
\dataline{atime}
\dataline{catalog}
\dataline{ident}
\dataline{control}
Integer constants describing the various types of parser callbacks
that can be set by the \method{addcallback()} method of CD parser
objects (see below).
\end{datadesc}


\subsection{Player Objects}
\label{player-objects}

Player objects (returned by \function{open()}) have the following
methods:

\begin{methoddesc}[CD player]{allowremoval}{}
Unlocks the eject button on the CD-ROM drive permitting the user to
eject the caddy if desired.
\end{methoddesc}

\begin{methoddesc}[CD player]{bestreadsize}{}
Returns the best value to use for the \var{num_frames} parameter of
the \method{readda()} method.  Best is defined as the value that
permits a continuous flow of data from the CD-ROM drive.
\end{methoddesc}

\begin{methoddesc}[CD player]{close}{}
Frees the resources associated with the player object.  After calling
\method{close()}, the methods of the object should no longer be used.
\end{methoddesc}

\begin{methoddesc}[CD player]{eject}{}
Ejects the caddy from the CD-ROM drive.
\end{methoddesc}

\begin{methoddesc}[CD player]{getstatus}{}
Returns information pertaining to the current state of the CD-ROM
drive.  The returned information is a tuple with the following values:
\var{state}, \var{track}, \var{rtime}, \var{atime}, \var{ttime},
\var{first}, \var{last}, \var{scsi_audio}, \var{cur_block}.
\var{rtime} is the time relative to the start of the current track;
\var{atime} is the time relative to the beginning of the disc;
\var{ttime} is the total time on the disc.  For more information on
the meaning of the values, see the man page \manpage{CDgetstatus}{3dm}.
The value of \var{state} is one of the following: \constant{ERROR},
\constant{NODISC}, \constant{READY}, \constant{PLAYING},
\constant{PAUSED}, \constant{STILL}, or \constant{CDROM}.
\end{methoddesc}

\begin{methoddesc}[CD player]{gettrackinfo}{track}
Returns information about the specified track.  The returned
information is a tuple consisting of two elements, the start time of
the track and the duration of the track.
\end{methoddesc}

\begin{methoddesc}[CD player]{msftoblock}{min, sec, frame}
Converts a minutes, seconds, frames triple representing a time in
absolute time code into the corresponding logical block number for the
given CD-ROM drive.  You should use \function{msftoframe()} rather than
\method{msftoblock()} for comparing times.  The logical block number
differs from the frame number by an offset required by certain CD-ROM
drives.
\end{methoddesc}

\begin{methoddesc}[CD player]{play}{start, play}
Starts playback of an audio CD in the CD-ROM drive at the specified
track.  The audio output appears on the CD-ROM drive's headphone and
audio jacks (if fitted).  Play stops at the end of the disc.
\var{start} is the number of the track at which to start playing the
CD; if \var{play} is 0, the CD will be set to an initial paused
state.  The method \method{togglepause()} can then be used to commence
play.
\end{methoddesc}

\begin{methoddesc}[CD player]{playabs}{minutes, seconds, frames, play}
Like \method{play()}, except that the start is given in minutes,
seconds, and frames instead of a track number.
\end{methoddesc}

\begin{methoddesc}[CD player]{playtrack}{start, play}
Like \method{play()}, except that playing stops at the end of the
track.
\end{methoddesc}

\begin{methoddesc}[CD player]{playtrackabs}{track, minutes, seconds, frames, play}
Like \method{play()}, except that playing begins at the specified
absolute time and ends at the end of the specified track.
\end{methoddesc}

\begin{methoddesc}[CD player]{preventremoval}{}
Locks the eject button on the CD-ROM drive thus preventing the user
from arbitrarily ejecting the caddy.
\end{methoddesc}

\begin{methoddesc}[CD player]{readda}{num_frames}
Reads the specified number of frames from an audio CD mounted in the
CD-ROM drive.  The return value is a string representing the audio
frames.  This string can be passed unaltered to the
\method{parseframe()} method of the parser object.
\end{methoddesc}

\begin{methoddesc}[CD player]{seek}{minutes, seconds, frames}
Sets the pointer that indicates the starting point of the next read of
digital audio data from a CD-ROM.  The pointer is set to an absolute
time code location specified in \var{minutes}, \var{seconds}, and
\var{frames}.  The return value is the logical block number to which
the pointer has been set.
\end{methoddesc}

\begin{methoddesc}[CD player]{seekblock}{block}
Sets the pointer that indicates the starting point of the next read of
digital audio data from a CD-ROM.  The pointer is set to the specified
logical block number.  The return value is the logical block number to
which the pointer has been set.
\end{methoddesc}

\begin{methoddesc}[CD player]{seektrack}{track}
Sets the pointer that indicates the starting point of the next read of
digital audio data from a CD-ROM.  The pointer is set to the specified
track.  The return value is the logical block number to which the
pointer has been set.
\end{methoddesc}

\begin{methoddesc}[CD player]{stop}{}
Stops the current playing operation.
\end{methoddesc}

\begin{methoddesc}[CD player]{togglepause}{}
Pauses the CD if it is playing, and makes it play if it is paused.
\end{methoddesc}


\subsection{Parser Objects}
\label{cd-parser-objects}

Parser objects (returned by \function{createparser()}) have the
following methods:

\begin{methoddesc}[CD parser]{addcallback}{type, func, arg}
Adds a callback for the parser.  The parser has callbacks for eight
different types of data in the digital audio data stream.  Constants
for these types are defined at the \module{cd} module level (see above).
The callback is called as follows: \code{\var{func}(\var{arg}, type,
data)}, where \var{arg} is the user supplied argument, \var{type} is
the particular type of callback, and \var{data} is the data returned
for this \var{type} of callback.  The type of the data depends on the
\var{type} of callback as follows:

\begin{tableii}{l|p{4in}}{code}{Type}{Value}
  \lineii{audio}{String which can be passed unmodified to
\function{al.writesamps()}.}
  \lineii{pnum}{Integer giving the program (track) number.}
  \lineii{index}{Integer giving the index number.}
  \lineii{ptime}{Tuple consisting of the program time in minutes,
seconds, and frames.}
  \lineii{atime}{Tuple consisting of the absolute time in minutes,
seconds, and frames.}
  \lineii{catalog}{String of 13 characters, giving the catalog number
of the CD.}
  \lineii{ident}{String of 12 characters, giving the ISRC
identification number of the recording.  The string consists of two
characters country code, three characters owner code, two characters
giving the year, and five characters giving a serial number.}
  \lineii{control}{Integer giving the control bits from the CD
subcode data}
\end{tableii}
\end{methoddesc}

\begin{methoddesc}[CD parser]{deleteparser}{}
Deletes the parser and frees the memory it was using.  The object
should not be used after this call.  This call is done automatically
when the last reference to the object is removed.
\end{methoddesc}

\begin{methoddesc}[CD parser]{parseframe}{frame}
Parses one or more frames of digital audio data from a CD such as
returned by \method{readda()}.  It determines which subcodes are
present in the data.  If these subcodes have changed since the last
frame, then \method{parseframe()} executes a callback of the
appropriate type passing to it the subcode data found in the frame.
Unlike the \C{} function, more than one frame of digital audio data
can be passed to this method.
\end{methoddesc}

\begin{methoddesc}[CD parser]{removecallback}{type}
Removes the callback for the given \var{type}.
\end{methoddesc}

\begin{methoddesc}[CD parser]{resetparser}{}
Resets the fields of the parser used for tracking subcodes to an
initial state.  \method{resetparser()} should be called after the disc
has been changed.
\end{methoddesc}

\section{Built-in Module \sectcode{fl}}
\label{module-fl}
\bimodindex{fl}

This module provides an interface to the FORMS Library by Mark
Overmars.  The source for the library can be retrieved by anonymous
ftp from host \samp{ftp.cs.ruu.nl}, directory \file{SGI/FORMS}.  It
was last tested with version 2.0b.

Most functions are literal translations of their C equivalents,
dropping the initial \samp{fl_} from their name.  Constants used by
the library are defined in module \code{FL} described below.

The creation of objects is a little different in Python than in C:
instead of the `current form' maintained by the library to which new
FORMS objects are added, all functions that add a FORMS object to a
form are methods of the Python object representing the form.
Consequently, there are no Python equivalents for the C functions
\code{fl_addto_form} and \code{fl_end_form}, and the equivalent of
\code{fl_bgn_form} is called \code{fl.make_form}.

Watch out for the somewhat confusing terminology: FORMS uses the word
\dfn{object} for the buttons, sliders etc. that you can place in a form.
In Python, `object' means any value.  The Python interface to FORMS
introduces two new Python object types: form objects (representing an
entire form) and FORMS objects (representing one button, slider etc.).
Hopefully this isn't too confusing...

There are no `free objects' in the Python interface to FORMS, nor is
there an easy way to add object classes written in Python.  The FORMS
interface to GL event handling is available, though, so you can mix
FORMS with pure GL windows.

\strong{Please note:} importing \code{fl} implies a call to the GL function
\code{foreground()} and to the FORMS routine \code{fl_init()}.

\subsection{Functions Defined in Module \sectcode{fl}}
\nodename{FL Functions}

Module \code{fl} defines the following functions.  For more information
about what they do, see the description of the equivalent C function
in the FORMS documentation:

\setindexsubitem{(in module fl)}
\begin{funcdesc}{make_form}{type\, width\, height}
Create a form with given type, width and height.  This returns a
\dfn{form} object, whose methods are described below.
\end{funcdesc}

\begin{funcdesc}{do_forms}{}
The standard FORMS main loop.  Returns a Python object representing
the FORMS object needing interaction, or the special value
\code{FL.EVENT}.
\end{funcdesc}

\begin{funcdesc}{check_forms}{}
Check for FORMS events.  Returns what \code{do_forms} above returns,
or \code{None} if there is no event that immediately needs
interaction.
\end{funcdesc}

\begin{funcdesc}{set_event_call_back}{function}
Set the event callback function.
\end{funcdesc}

\begin{funcdesc}{set_graphics_mode}{rgbmode\, doublebuffering}
Set the graphics modes.
\end{funcdesc}

\begin{funcdesc}{get_rgbmode}{}
Return the current rgb mode.  This is the value of the C global
variable \code{fl_rgbmode}.
\end{funcdesc}

\begin{funcdesc}{show_message}{str1\, str2\, str3}
Show a dialog box with a three-line message and an OK button.
\end{funcdesc}

\begin{funcdesc}{show_question}{str1\, str2\, str3}
Show a dialog box with a three-line message and YES and NO buttons.
It returns \code{1} if the user pressed YES, \code{0} if NO.
\end{funcdesc}

\begin{funcdesc}{show_choice}{str1, str2, str3, but1\optional{, but2, but3}}
Show a dialog box with a three-line message and up to three buttons.
It returns the number of the button clicked by the user
(\code{1}, \code{2} or \code{3}).
\end{funcdesc}

\begin{funcdesc}{show_input}{prompt\, default}
Show a dialog box with a one-line prompt message and text field in
which the user can enter a string.  The second argument is the default
input string.  It returns the string value as edited by the user.
\end{funcdesc}

\begin{funcdesc}{show_file_selector}{message\, directory\, pattern\, default}
Show a dialog box in which the user can select a file.  It returns
the absolute filename selected by the user, or \code{None} if the user
presses Cancel.
\end{funcdesc}

\begin{funcdesc}{get_directory}{}
\funcline{get_pattern}{}
\funcline{get_filename}{}
These functions return the directory, pattern and filename (the tail
part only) selected by the user in the last \code{show_file_selector}
call.
\end{funcdesc}

\begin{funcdesc}{qdevice}{dev}
\funcline{unqdevice}{dev}
\funcline{isqueued}{dev}
\funcline{qtest}{}
\funcline{qread}{}
%\funcline{blkqread}{?}
\funcline{qreset}{}
\funcline{qenter}{dev\, val}
\funcline{get_mouse}{}
\funcline{tie}{button\, valuator1\, valuator2}
These functions are the FORMS interfaces to the corresponding GL
functions.  Use these if you want to handle some GL events yourself
when using \code{fl.do_events}.  When a GL event is detected that
FORMS cannot handle, \code{fl.do_forms()} returns the special value
\code{FL.EVENT} and you should call \code{fl.qread()} to read the
event from the queue.  Don't use the equivalent GL functions!
\end{funcdesc}

\begin{funcdesc}{color}{}
\funcline{mapcolor}{}
\funcline{getmcolor}{}
See the description in the FORMS documentation of \code{fl_color},
\code{fl_mapcolor} and \code{fl_getmcolor}.
\end{funcdesc}

\subsection{Form Objects}

Form objects (returned by \code{fl.make_form()} above) have the
following methods.  Each method corresponds to a C function whose name
is prefixed with \samp{fl_}; and whose first argument is a form
pointer; please refer to the official FORMS documentation for
descriptions.

All the \samp{add_{\rm \ldots}} functions return a Python object representing
the FORMS object.  Methods of FORMS objects are described below.  Most
kinds of FORMS object also have some methods specific to that kind;
these methods are listed here.

\begin{flushleft}
\setindexsubitem{(form object method)}
\begin{funcdesc}{show_form}{placement\, bordertype\, name}
  Show the form.
\end{funcdesc}

\begin{funcdesc}{hide_form}{}
  Hide the form.
\end{funcdesc}

\begin{funcdesc}{redraw_form}{}
  Redraw the form.
\end{funcdesc}

\begin{funcdesc}{set_form_position}{x\, y}
Set the form's position.
\end{funcdesc}

\begin{funcdesc}{freeze_form}{}
Freeze the form.
\end{funcdesc}

\begin{funcdesc}{unfreeze_form}{}
  Unfreeze the form.
\end{funcdesc}

\begin{funcdesc}{activate_form}{}
  Activate the form.
\end{funcdesc}

\begin{funcdesc}{deactivate_form}{}
  Deactivate the form.
\end{funcdesc}

\begin{funcdesc}{bgn_group}{}
  Begin a new group of objects; return a group object.
\end{funcdesc}

\begin{funcdesc}{end_group}{}
  End the current group of objects.
\end{funcdesc}

\begin{funcdesc}{find_first}{}
  Find the first object in the form.
\end{funcdesc}

\begin{funcdesc}{find_last}{}
  Find the last object in the form.
\end{funcdesc}

%---

\begin{funcdesc}{add_box}{type\, x\, y\, w\, h\, name}
Add a box object to the form.
No extra methods.
\end{funcdesc}

\begin{funcdesc}{add_text}{type\, x\, y\, w\, h\, name}
Add a text object to the form.
No extra methods.
\end{funcdesc}

%\begin{funcdesc}{add_bitmap}{type\, x\, y\, w\, h\, name}
%Add a bitmap object to the form.
%\end{funcdesc}

\begin{funcdesc}{add_clock}{type\, x\, y\, w\, h\, name}
Add a clock object to the form. \\
Method:
\code{get_clock}.
\end{funcdesc}

%---

\begin{funcdesc}{add_button}{type\, x\, y\, w\, h\,  name}
Add a button object to the form. \\
Methods:
\code{get_button},
\code{set_button}.
\end{funcdesc}

\begin{funcdesc}{add_lightbutton}{type\, x\, y\, w\, h\, name}
Add a lightbutton object to the form. \\
Methods:
\code{get_button},
\code{set_button}.
\end{funcdesc}

\begin{funcdesc}{add_roundbutton}{type\, x\, y\, w\, h\, name}
Add a roundbutton object to the form. \\
Methods:
\code{get_button},
\code{set_button}.
\end{funcdesc}

%---

\begin{funcdesc}{add_slider}{type\, x\, y\, w\, h\, name}
Add a slider object to the form. \\
Methods:
\code{set_slider_value},
\code{get_slider_value},
\code{set_slider_bounds},
\code{get_slider_bounds},
\code{set_slider_return},
\code{set_slider_size},
\code{set_slider_precision},
\code{set_slider_step}.
\end{funcdesc}

\begin{funcdesc}{add_valslider}{type\, x\, y\, w\, h\, name}
Add a valslider object to the form. \\
Methods:
\code{set_slider_value},
\code{get_slider_value},
\code{set_slider_bounds},
\code{get_slider_bounds},
\code{set_slider_return},
\code{set_slider_size},
\code{set_slider_precision},
\code{set_slider_step}.
\end{funcdesc}

\begin{funcdesc}{add_dial}{type\, x\, y\, w\, h\, name}
Add a dial object to the form. \\
Methods:
\code{set_dial_value},
\code{get_dial_value},
\code{set_dial_bounds},
\code{get_dial_bounds}.
\end{funcdesc}

\begin{funcdesc}{add_positioner}{type\, x\, y\, w\, h\, name}
Add a positioner object to the form. \\
Methods:
\code{set_positioner_xvalue},
\code{set_positioner_yvalue},
\code{set_positioner_xbounds},
\code{set_positioner_ybounds},
\code{get_positioner_xvalue},
\code{get_positioner_yvalue},
\code{get_positioner_xbounds},
\code{get_positioner_ybounds}.
\end{funcdesc}

\begin{funcdesc}{add_counter}{type\, x\, y\, w\, h\, name}
Add a counter object to the form. \\
Methods:
\code{set_counter_value},
\code{get_counter_value},
\code{set_counter_bounds},
\code{set_counter_step},
\code{set_counter_precision},
\code{set_counter_return}.
\end{funcdesc}

%---

\begin{funcdesc}{add_input}{type\, x\, y\, w\, h\, name}
Add a input object to the form. \\
Methods:
\code{set_input},
\code{get_input},
\code{set_input_color},
\code{set_input_return}.
\end{funcdesc}

%---

\begin{funcdesc}{add_menu}{type\, x\, y\, w\, h\, name}
Add a menu object to the form. \\
Methods:
\code{set_menu},
\code{get_menu},
\code{addto_menu}.
\end{funcdesc}

\begin{funcdesc}{add_choice}{type\, x\, y\, w\, h\, name}
Add a choice object to the form. \\
Methods:
\code{set_choice},
\code{get_choice},
\code{clear_choice},
\code{addto_choice},
\code{replace_choice},
\code{delete_choice},
\code{get_choice_text},
\code{set_choice_fontsize},
\code{set_choice_fontstyle}.
\end{funcdesc}

\begin{funcdesc}{add_browser}{type\, x\, y\, w\, h\, name}
Add a browser object to the form. \\
Methods:
\code{set_browser_topline},
\code{clear_browser},
\code{add_browser_line},
\code{addto_browser},
\code{insert_browser_line},
\code{delete_browser_line},
\code{replace_browser_line},
\code{get_browser_line},
\code{load_browser},
\code{get_browser_maxline},
\code{select_browser_line},
\code{deselect_browser_line},
\code{deselect_browser},
\code{isselected_browser_line},
\code{get_browser},
\code{set_browser_fontsize},
\code{set_browser_fontstyle},
\code{set_browser_specialkey}.
\end{funcdesc}

%---

\begin{funcdesc}{add_timer}{type\, x\, y\, w\, h\, name}
Add a timer object to the form. \\
Methods:
\code{set_timer},
\code{get_timer}.
\end{funcdesc}
\end{flushleft}

Form objects have the following data attributes; see the FORMS
documentation:

\begin{tableiii}{|l|c|l|}{code}{Name}{Type}{Meaning}
  \lineiii{window}{int (read-only)}{GL window id}
  \lineiii{w}{float}{form width}
  \lineiii{h}{float}{form height}
  \lineiii{x}{float}{form x origin}
  \lineiii{y}{float}{form y origin}
  \lineiii{deactivated}{int}{nonzero if form is deactivated}
  \lineiii{visible}{int}{nonzero if form is visible}
  \lineiii{frozen}{int}{nonzero if form is frozen}
  \lineiii{doublebuf}{int}{nonzero if double buffering on}
\end{tableiii}

\subsection{FORMS Objects}

Besides methods specific to particular kinds of FORMS objects, all
FORMS objects also have the following methods:

\setindexsubitem{(FORMS object method)}
\begin{funcdesc}{set_call_back}{function\, argument}
Set the object's callback function and argument.  When the object
needs interaction, the callback function will be called with two
arguments: the object, and the callback argument.  (FORMS objects
without a callback function are returned by \code{fl.do_forms()} or
\code{fl.check_forms()} when they need interaction.)  Call this method
without arguments to remove the callback function.
\end{funcdesc}

\begin{funcdesc}{delete_object}{}
  Delete the object.
\end{funcdesc}

\begin{funcdesc}{show_object}{}
  Show the object.
\end{funcdesc}

\begin{funcdesc}{hide_object}{}
  Hide the object.
\end{funcdesc}

\begin{funcdesc}{redraw_object}{}
  Redraw the object.
\end{funcdesc}

\begin{funcdesc}{freeze_object}{}
  Freeze the object.
\end{funcdesc}

\begin{funcdesc}{unfreeze_object}{}
  Unfreeze the object.
\end{funcdesc}

%\begin{funcdesc}{handle_object}{} XXX
%\end{funcdesc}

%\begin{funcdesc}{handle_object_direct}{} XXX
%\end{funcdesc}

FORMS objects have these data attributes; see the FORMS documentation:

\begin{tableiii}{|l|c|l|}{code}{Name}{Type}{Meaning}
  \lineiii{objclass}{int (read-only)}{object class}
  \lineiii{type}{int (read-only)}{object type}
  \lineiii{boxtype}{int}{box type}
  \lineiii{x}{float}{x origin}
  \lineiii{y}{float}{y origin}
  \lineiii{w}{float}{width}
  \lineiii{h}{float}{height}
  \lineiii{col1}{int}{primary color}
  \lineiii{col2}{int}{secondary color}
  \lineiii{align}{int}{alignment}
  \lineiii{lcol}{int}{label color}
  \lineiii{lsize}{float}{label font size}
  \lineiii{label}{string}{label string}
  \lineiii{lstyle}{int}{label style}
  \lineiii{pushed}{int (read-only)}{(see FORMS docs)}
  \lineiii{focus}{int (read-only)}{(see FORMS docs)}
  \lineiii{belowmouse}{int (read-only)}{(see FORMS docs)}
  \lineiii{frozen}{int (read-only)}{(see FORMS docs)}
  \lineiii{active}{int (read-only)}{(see FORMS docs)}
  \lineiii{input}{int (read-only)}{(see FORMS docs)}
  \lineiii{visible}{int (read-only)}{(see FORMS docs)}
  \lineiii{radio}{int (read-only)}{(see FORMS docs)}
  \lineiii{automatic}{int (read-only)}{(see FORMS docs)}
\end{tableiii}

\section{Standard Module \sectcode{FL}}
\nodename{FL (uppercase)}
\label{module-FL}
\stmodindex{FL}

This module defines symbolic constants needed to use the built-in
module \code{fl} (see above); they are equivalent to those defined in
the C header file \file{<forms.h>} except that the name prefix
\samp{FL_} is omitted.  Read the module source for a complete list of
the defined names.  Suggested use:

\begin{verbatim}
import fl
from FL import *
\end{verbatim}
%
\section{Standard Module \sectcode{flp}}
\label{module-flp}
\stmodindex{flp}

This module defines functions that can read form definitions created
by the `form designer' (\code{fdesign}) program that comes with the
FORMS library (see module \code{fl} above).

For now, see the file \file{flp.doc} in the Python library source
directory for a description.

XXX A complete description should be inserted here!

\section{\module{fm} ---
         \emph{Font Manager} interface}

\declaremodule{builtin}{fm}
  \platform{IRIX}
\modulesynopsis{\emph{Font Manager} interface for SGI workstations.}


This module provides access to the IRIS \emph{Font Manager} library.
\index{Font Manager, IRIS}
\index{IRIS Font Manager}
It is available only on Silicon Graphics machines.
See also: \emph{4Sight User's Guide}, section 1, chapter 5: ``Using
the IRIS Font Manager.''

This is not yet a full interface to the IRIS Font Manager.
Among the unsupported features are: matrix operations; cache
operations; character operations (use string operations instead); some
details of font info; individual glyph metrics; and printer matching.

It supports the following operations:

\begin{funcdesc}{init}{}
Initialization function.
Calls \cfunction{fminit()}.
It is normally not necessary to call this function, since it is called
automatically the first time the \module{fm} module is imported.
\end{funcdesc}

\begin{funcdesc}{findfont}{fontname}
Return a font handle object.
Calls \code{fmfindfont(\var{fontname})}.
\end{funcdesc}

\begin{funcdesc}{enumerate}{}
Returns a list of available font names.
This is an interface to \cfunction{fmenumerate()}.
\end{funcdesc}

\begin{funcdesc}{prstr}{string}
Render a string using the current font (see the \function{setfont()} font
handle method below).
Calls \code{fmprstr(\var{string})}.
\end{funcdesc}

\begin{funcdesc}{setpath}{string}
Sets the font search path.
Calls \code{fmsetpath(\var{string})}.
(XXX Does not work!?!)
\end{funcdesc}

\begin{funcdesc}{fontpath}{}
Returns the current font search path.
\end{funcdesc}

Font handle objects support the following operations:

\setindexsubitem{(font handle method)}
\begin{funcdesc}{scalefont}{factor}
Returns a handle for a scaled version of this font.
Calls \code{fmscalefont(\var{fh}, \var{factor})}.
\end{funcdesc}

\begin{funcdesc}{setfont}{}
Makes this font the current font.
Note: the effect is undone silently when the font handle object is
deleted.
Calls \code{fmsetfont(\var{fh})}.
\end{funcdesc}

\begin{funcdesc}{getfontname}{}
Returns this font's name.
Calls \code{fmgetfontname(\var{fh})}.
\end{funcdesc}

\begin{funcdesc}{getcomment}{}
Returns the comment string associated with this font.
Raises an exception if there is none.
Calls \code{fmgetcomment(\var{fh})}.
\end{funcdesc}

\begin{funcdesc}{getfontinfo}{}
Returns a tuple giving some pertinent data about this font.
This is an interface to \code{fmgetfontinfo()}.
The returned tuple contains the following numbers:
\code{(}\var{printermatched}, \var{fixed_width}, \var{xorig},
\var{yorig}, \var{xsize}, \var{ysize}, \var{height},
\var{nglyphs}\code{)}.
\end{funcdesc}

\begin{funcdesc}{getstrwidth}{string}
Returns the width, in pixels, of \var{string} when drawn in this font.
Calls \code{fmgetstrwidth(\var{fh}, \var{string})}.
\end{funcdesc}

\section{Built-in Module \sectcode{gl}}
\label{module-gl}
\bimodindex{gl}

This module provides access to the Silicon Graphics
\emph{Graphics Library}.
It is available only on Silicon Graphics machines.

\strong{Warning:}
Some illegal calls to the GL library cause the Python interpreter to dump
core.
In particular, the use of most GL calls is unsafe before the first
window is opened.

The module is too large to document here in its entirety, but the
following should help you to get started.
The parameter conventions for the C functions are translated to Python as
follows:

\begin{itemize}
\item
All (short, long, unsigned) int values are represented by Python
integers.
\item
All float and double values are represented by Python floating point
numbers.
In most cases, Python integers are also allowed.
\item
All arrays are represented by one-dimensional Python lists.
In most cases, tuples are also allowed.
\item
\begin{sloppypar}
All string and character arguments are represented by Python strings,
for instance,
\code{winopen('Hi There!')}
and
\code{rotate(900, 'z')}.
\end{sloppypar}
\item
All (short, long, unsigned) integer arguments or return values that are
only used to specify the length of an array argument are omitted.
For example, the C call

\bcode\begin{verbatim}
lmdef(deftype, index, np, props)
\end{verbatim}\ecode
%
is translated to Python as

\bcode\begin{verbatim}
lmdef(deftype, index, props)
\end{verbatim}\ecode
%
\item
Output arguments are omitted from the argument list; they are
transmitted as function return values instead.
If more than one value must be returned, the return value is a tuple.
If the C function has both a regular return value (that is not omitted
because of the previous rule) and an output argument, the return value
comes first in the tuple.
Examples: the C call

\bcode\begin{verbatim}
getmcolor(i, &red, &green, &blue)
\end{verbatim}\ecode
%
is translated to Python as

\bcode\begin{verbatim}
red, green, blue = getmcolor(i)
\end{verbatim}\ecode
%
\end{itemize}

The following functions are non-standard or have special argument
conventions:

\renewcommand{\indexsubitem}{(in module gl)}
\begin{funcdesc}{varray}{argument}
%JHXXX the argument-argument added
Equivalent to but faster than a number of
\code{v3d()}
calls.
The \var{argument} is a list (or tuple) of points.
Each point must be a tuple of coordinates
\code{(\var{x}, \var{y}, \var{z})} or \code{(\var{x}, \var{y})}.
The points may be 2- or 3-dimensional but must all have the
same dimension.
Float and int values may be mixed however.
The points are always converted to 3D double precision points
by assuming \code{\var{z} = 0.0} if necessary (as indicated in the man page),
and for each point
\code{v3d()}
is called.
\end{funcdesc}

\begin{funcdesc}{nvarray}{}
Equivalent to but faster than a number of
\code{n3f}
and
\code{v3f}
calls.
The argument is an array (list or tuple) of pairs of normals and points.
Each pair is a tuple of a point and a normal for that point.
Each point or normal must be a tuple of coordinates
\code{(\var{x}, \var{y}, \var{z})}.
Three coordinates must be given.
Float and int values may be mixed.
For each pair,
\code{n3f()}
is called for the normal, and then
\code{v3f()}
is called for the point.
\end{funcdesc}

\begin{funcdesc}{vnarray}{}
Similar to 
\code{nvarray()}
but the pairs have the point first and the normal second.
\end{funcdesc}

\begin{funcdesc}{nurbssurface}{s_k\, t_k\, ctl\, s_ord\, t_ord\, type}
% XXX s_k[], t_k[], ctl[][]
Defines a nurbs surface.
The dimensions of
\code{\var{ctl}[][]}
are computed as follows:
\code{[len(\var{s_k}) - \var{s_ord}]},
\code{[len(\var{t_k}) - \var{t_ord}]}.
\end{funcdesc}

\begin{funcdesc}{nurbscurve}{knots\, ctlpoints\, order\, type}
Defines a nurbs curve.
The length of ctlpoints is
\code{len(\var{knots}) - \var{order}}.
\end{funcdesc}

\begin{funcdesc}{pwlcurve}{points\, type}
Defines a piecewise-linear curve.
\var{points}
is a list of points.
\var{type}
must be
\code{N_ST}.
\end{funcdesc}

\begin{funcdesc}{pick}{n}
\funcline{select}{n}
The only argument to these functions specifies the desired size of the
pick or select buffer.
\end{funcdesc}

\begin{funcdesc}{endpick}{}
\funcline{endselect}{}
These functions have no arguments.
They return a list of integers representing the used part of the
pick/select buffer.
No method is provided to detect buffer overrun.
\end{funcdesc}

Here is a tiny but complete example GL program in Python:

\bcode\begin{verbatim}
import gl, GL, time

def main():
    gl.foreground()
    gl.prefposition(500, 900, 500, 900)
    w = gl.winopen('CrissCross')
    gl.ortho2(0.0, 400.0, 0.0, 400.0)
    gl.color(GL.WHITE)
    gl.clear()
    gl.color(GL.RED)
    gl.bgnline()
    gl.v2f(0.0, 0.0)
    gl.v2f(400.0, 400.0)
    gl.endline()
    gl.bgnline()
    gl.v2f(400.0, 0.0)
    gl.v2f(0.0, 400.0)
    gl.endline()
    time.sleep(5)

main()
\end{verbatim}\ecode
%
\section{Standard Modules \sectcode{GL} and \sectcode{DEVICE}}
\nodename{GL and DEVICE}
\stmodindex{GL}
\stmodindex{DEVICE}

These modules define the constants used by the Silicon Graphics
\emph{Graphics Library}
that C programmers find in the header files
\file{<gl/gl.h>}
and
\file{<gl/device.h>}.
Read the module source files for details.

\section{Built-in Module \module{imgfile}}
\label{module-imgfile}
\bimodindex{imgfile}

The \module{imgfile} module allows Python programs to access SGI imglib image
files (also known as \file{.rgb} files).  The module is far from
complete, but is provided anyway since the functionality that there is
is enough in some cases.  Currently, colormap files are not supported.

The module defines the following variables and functions:

\begin{excdesc}{error}
This exception is raised on all errors, such as unsupported file type, etc.
\end{excdesc}

\begin{funcdesc}{getsizes}{file}
This function returns a tuple \code{(\var{x}, \var{y}, \var{z})} where
\var{x} and \var{y} are the size of the image in pixels and
\var{z} is the number of
bytes per pixel. Only 3 byte RGB pixels and 1 byte greyscale pixels
are currently supported.
\end{funcdesc}

\begin{funcdesc}{read}{file}
This function reads and decodes the image on the specified file, and
returns it as a Python string. The string has either 1 byte greyscale
pixels or 4 byte RGBA pixels. The bottom left pixel is the first in
the string. This format is suitable to pass to \function{gl.lrectwrite()},
for instance.
\end{funcdesc}

\begin{funcdesc}{readscaled}{file, x, y, filter\optional{, blur}}
This function is identical to read but it returns an image that is
scaled to the given \var{x} and \var{y} sizes. If the \var{filter} and
\var{blur} parameters are omitted scaling is done by
simply dropping or duplicating pixels, so the result will be less than
perfect, especially for computer-generated images.

Alternatively, you can specify a filter to use to smoothen the image
after scaling. The filter forms supported are \code{'impulse'},
\code{'box'}, \code{'triangle'}, \code{'quadratic'} and
\code{'gaussian'}. If a filter is specified \var{blur} is an optional
parameter specifying the blurriness of the filter. It defaults to \code{1.0}.

\code{readscaled} makes no
attempt to keep the aspect ratio correct, so that is the users'
responsibility.
\end{funcdesc}

\begin{funcdesc}{ttob}{flag}
This function sets a global flag which defines whether the scan lines
of the image are read or written from bottom to top (flag is zero,
compatible with SGI GL) or from top to bottom(flag is one,
compatible with X).  The default is zero.
\end{funcdesc}

\begin{funcdesc}{write}{file, data, x, y, z}
This function writes the RGB or greyscale data in \var{data} to image
file \var{file}. \var{x} and \var{y} give the size of the image,
\var{z} is 1 for 1 byte greyscale images or 3 for RGB images (which are
stored as 4 byte values of which only the lower three bytes are used).
These are the formats returned by \code{gl.lrectread}.
\end{funcdesc}

%\section{\module{panel} ---
         None}
\declaremodule{standard}{panel}

\modulesynopsis{None}


\strong{Please note:} The FORMS library, to which the
\code{fl}\refbimodindex{fl} module described above interfaces, is a
simpler and more accessible user interface library for use with GL
than the \code{panel} module (besides also being by a Dutch author).

This module should be used instead of the built-in module
\code{pnl}\refbimodindex{pnl}
to interface with the
\emph{Panel Library}.

The module is too large to document here in its entirety.
One interesting function:

\begin{funcdesc}{defpanellist}{filename}
Parses a panel description file containing S-expressions written by the
\emph{Panel Editor}
that accompanies the Panel Library and creates the described panels.
It returns a list of panel objects.
\end{funcdesc}

\strong{Warning:}
the Python interpreter will dump core if you don't create a GL window
before calling
\code{panel.mkpanel()}
or
\code{panel.defpanellist()}.

\section{\module{panelparser} ---
         None}
\declaremodule{standard}{panelparser}

\modulesynopsis{None}


This module defines a self-contained parser for S-expressions as output
by the Panel Editor (which is written in Scheme so it can't help writing
S-expressions).
The relevant function is
\code{panelparser.parse_file(\var{file})}
which has a file object (not a filename!) as argument and returns a list
of parsed S-expressions.
Each S-expression is converted into a Python list, with atoms converted
to Python strings and sub-expressions (recursively) to Python lists.
For more details, read the module file.
% XXXXJH should be funcdesc, I think

\section{\module{pnl} ---
         None}
\declaremodule{builtin}{pnl}

\modulesynopsis{None}


This module provides access to the
\emph{Panel Library}
built by NASA Ames\index{NASA} (to get it, send e-mail to
\code{panel-request@nas.nasa.gov}).
All access to it should be done through the standard module
\code{panel}\refstmodindex{panel},
which transparantly exports most functions from
\code{pnl}
but redefines
\code{pnl.dopanel()}.

\strong{Warning:}
the Python interpreter will dump core if you don't create a GL window
before calling
\code{pnl.mkpanel()}.

The module is too large to document here in its entirety.


\chapter{SunOS Specific Services}
\label{sunos}

The modules described in this chapter provide interfaces to features
that are unique to SunOS 5 (also known as Solaris version 2).
			% SUNOS ONLY
\section{\module{sunaudiodev} ---
         Access to Sun audio hardware.}
\declaremodule{builtin}{sunaudiodev}

\modulesynopsis{Access to Sun audio hardware.}


This module allows you to access the Sun audio interface. The Sun
audio hardware is capable of recording and playing back audio data
in u-LAW\index{u-LAW} format with a sample rate of 8K per second. A
full description can be found in the \manpage{audio}{7I} manual page.

The module defines the following variables and functions:

\begin{excdesc}{error}
This exception is raised on all errors. The argument is a string
describing what went wrong.
\end{excdesc}

\begin{funcdesc}{open}{mode}
This function opens the audio device and returns a Sun audio device
object. This object can then be used to do I/O on. The \var{mode} parameter
is one of \code{'r'} for record-only access, \code{'w'} for play-only
access, \code{'rw'} for both and \code{'control'} for access to the
control device. Since only one process is allowed to have the recorder
or player open at the same time it is a good idea to open the device
only for the activity needed. See \manpage{audio}{7I} for details.

As per the manpage, this module first looks in the environment
variable \code{AUDIODEV} for the base audio device filename.  If not
found, it falls back to \file{/dev/audio}.  The control device is
calculated by appending ``ctl'' to the base audio device.
\end{funcdesc}


\subsection{Audio Device Objects}
\label{audio-device-objects}

The audio device objects are returned by \function{open()} define the
following methods (except \code{control} objects which only provide
\method{getinfo()}, \method{setinfo()}, \method{fileno()}, and
\method{drain()}):

\begin{methoddesc}[audio device]{close}{}
This method explicitly closes the device. It is useful in situations
where deleting the object does not immediately close it since there
are other references to it. A closed device should not be used again.
\end{methoddesc}

\begin{methoddesc}[audio device]{fileno}{}
Returns the file descriptor associated with the device.  This can be
used to set up \code{SIGPOLL} notification, as described below.
\end{methoddesc}

\begin{methoddesc}[audio device]{drain}{}
This method waits until all pending output is processed and then returns.
Calling this method is often not necessary: destroying the object will
automatically close the audio device and this will do an implicit drain.
\end{methoddesc}

\begin{methoddesc}[audio device]{flush}{}
This method discards all pending output. It can be used avoid the
slow response to a user's stop request (due to buffering of up to one
second of sound).
\end{methoddesc}

\begin{methoddesc}[audio device]{getinfo}{}
This method retrieves status information like input and output volume,
etc. and returns it in the form of
an audio status object. This object has no methods but it contains a
number of attributes describing the current device status. The names
and meanings of the attributes are described in
\file{/usr/include/sun/audioio.h} and in the \manpage{audio}{7I}
manual page.  Member names
are slightly different from their \C{} counterparts: a status object is
only a single structure. Members of the \cdata{play} substructure have
\samp{o_} prepended to their name and members of the \cdata{record}
structure have \samp{i_}. So, the \C{} member \cdata{play.sample_rate} is
accessed as \member{o_sample_rate}, \cdata{record.gain} as \member{i_gain}
and \cdata{monitor_gain} plainly as \member{monitor_gain}.
\end{methoddesc}

\begin{methoddesc}[audio device]{ibufcount}{}
This method returns the number of samples that are buffered on the
recording side, i.e.\ the program will not block on a
\function{read()} call of so many samples.
\end{methoddesc}

\begin{methoddesc}[audio device]{obufcount}{}
This method returns the number of samples buffered on the playback
side. Unfortunately, this number cannot be used to determine a number
of samples that can be written without blocking since the kernel
output queue length seems to be variable.
\end{methoddesc}

\begin{methoddesc}[audio device]{read}{size}
This method reads \var{size} samples from the audio input and returns
them as a Python string. The function blocks until enough data is available.
\end{methoddesc}

\begin{methoddesc}[audio device]{setinfo}{status}
This method sets the audio device status parameters. The \var{status}
parameter is an device status object as returned by \function{getinfo()} and
possibly modified by the program.
\end{methoddesc}

\begin{methoddesc}[audio device]{write}{samples}
Write is passed a Python string containing audio samples to be played.
If there is enough buffer space free it will immediately return,
otherwise it will block.
\end{methoddesc}

There is a companion module,
\module{SUNAUDIODEV}\refstmodindex{SUNAUDIODEV}, which defines useful
symbolic constants like \constant{MIN_GAIN}, \constant{MAX_GAIN},
\constant{SPEAKER}, etc. The names of the constants are the same names
as used in the \C{} include file \code{<sun/audioio.h>}, with the
leading string \samp{AUDIO_} stripped.

The audio device supports asynchronous notification of various events,
through the SIGPOLL signal.  Here's an example of how you might enable 
this in Python:

\begin{verbatim}
def handle_sigpoll(signum, frame):
    print 'I got a SIGPOLL update'
pp
import fcntl, signal, STROPTS

signal.signal(signal.SIGPOLL, handle_sigpoll)
fcntl.ioctl(audio_obj.fileno(), STROPTS.I_SETSIG, STROPTS.S_MSG)
\end{verbatim}


\chapter{Undocumented Modules \label{undoc}}

Here's a quick listing of modules that are currently undocumented, but
that should be documented.  Feel free to contribute documentation for
them!  (Send via email to \email{python-docs@python.org}.)

The idea and original contents for this chapter were taken
from a posting by Fredrik Lundh; the specific contents of this chapter
have been substantially revised.


\section{Frameworks}

Frameworks tend to be harder to document, but are well worth the
effort spent.

\begin{description}
\item[\module{Tkinter}]
--- Interface to Tcl/Tk for graphical user interfaces; Fredrik Lundh
is working on this one!  See
\citetitle[http://www.pythonware.com/library.htm]{An Introduction to
Tkinter} at \url{http://www.pythonware.com/library.htm} for on-line
reference material.

\item[\module{Tkdnd}]
--- Drag-and-drop support for \module{Tkinter}.

\item[\module{turtle}]
--- Turtle graphics in a Tk window.

\item[\module{test}]
--- Regression testing framework.  This is used for the Python
regression test, but is useful for other Python libraries as well.
This is a package rather than a single module.
\end{description}


\section{Miscellaneous useful utilities}

Some of these are very old and/or not very robust; marked with ``hmm.''

\begin{description}
\item[\module{bdb}]
--- A generic Python debugger base class (used by pdb).

\item[\module{ihooks}]
--- Import hook support (for \refmodule{rexec}; may become obsolete).

\item[\module{smtpd}]
--- An SMTP daemon implementation which meets the minimum requirements
for \rfc{821} conformance.
\end{description}


\section{Platform specific modules}

These modules are used to implement the \refmodule{os.path} module,
and are not documented beyond this mention.  There's little need to
document these.

\begin{description}
\item[\module{dospath}]
--- Implementation of \module{os.path} on MS-DOS.

\item[\module{ntpath}]
--- Implementation on \module{os.path} on Win32, Win64, WinCE, and
OS/2 platforms.

\item[\module{posixpath}]
--- Implementation on \module{os.path} on \POSIX.
\end{description}


\section{Multimedia}

\begin{description}
\item[\module{audiodev}]
--- Platform-independent API for playing audio data.

\item[\module{sunaudio}]
--- Interpret Sun audio headers (may become obsolete or a tool/demo).

\item[\module{toaiff}]
--- Convert "arbitrary" sound files to AIFF files; should probably
become a tool or demo.  Requires the external program \program{sox}.
\end{description}


\section{Obsolete \label{obsolete-modules}}

These modules are not normally available for import; additional work
must be done to make them available.

Those which are written in Python will be installed into the directory 
\file{lib-old/} installed as part of the standard library.  To use
these, the directory must be added to \code{sys.path}, possibly using
\envvar{PYTHONPATH}.

Obsolete extension modules written in C are not built by default.
Under \UNIX, these must be enabled by uncommenting the appropriate
lines in \file{Modules/Setup} in the build tree and either rebuilding
Python if the modules are statically linked, or building and
installing the shared object if using dynamically-loaded extensions.

% XXX need Windows instructions!

\begin{description}
\item[\module{addpack}]
--- Alternate approach to packages.  Use the built-in package support
instead.

\item[\module{cmp}]
--- File comparison function.  Use the newer \refmodule{filecmp} instead.

\item[\module{cmpcache}]
--- Caching version of the obsolete \module{cmp} module.  Use the
newer \refmodule{filecmp} instead.

\item[\module{codehack}]
--- Extract function name or line number from a function
code object (these are now accessible as attributes:
\member{co.co_name}, \member{func.func_name},
\member{co.co_firstlineno}).

\item[\module{dircmp}]
--- Class to build directory diff tools on (may become a demo or tool).
\deprecated{2.0}{The \refmodule{filecmp} module replaces
\module{dircmp}.}

\item[\module{dump}]
--- Print python code that reconstructs a variable.

\item[\module{fmt}]
--- Text formatting abstractions (too slow).

\item[\module{lockfile}]
--- Wrapper around FCNTL file locking (use
\function{fcntl.lockf()}/\function{flock()} instead; see \refmodule{fcntl}).

\item[\module{newdir}]
--- New \function{dir()} function (the standard \function{dir()} is
now just as good).

\item[\module{Para}]
--- Helper for \module{fmt}.

\item[\module{poly}]
--- Polynomials.

\item[\module{regex}]
--- Emacs-style regular expression support; may still be used in some
old code (extension module).  Refer to the
\citetitle[http://www.python.org/doc/1.6/lib/module-regex.html]{Python
1.6 Documentation} for documentation.

\item[\module{regsub}]
--- Regular expression based string replacement utilities, for use
with \module{regex} (extension module).  Refer to the
\citetitle[http://www.python.org/doc/1.6/lib/module-regsub.html]{Python
1.6 Documentation} for documentation.

\item[\module{tb}]
--- Print tracebacks, with a dump of local variables (use
\function{pdb.pm()} or \refmodule{traceback} instead).

\item[\module{timing}]
--- Measure time intervals to high resolution (use
\function{time.clock()} instead).  (This is an extension module.)

\item[\module{tzparse}]
--- Parse a timezone specification (unfinished; may disappear in the
future, and does not work when the \envvar{TZ} environment variable is
not set).

\item[\module{util}]
--- Useful functions that don't fit elsewhere.

\item[\module{whatsound}]
--- Recognize sound files; use \refmodule{sndhdr} instead.

\item[\module{zmod}]
--- Compute properties of mathematical ``fields.''
\end{description}


The following modules are obsolete, but are likely to re-surface as
tools or scripts:

\begin{description}
\item[\module{find}]
--- Find files matching pattern in directory tree.

\item[\module{grep}]
--- \program{grep} implementation in Python.

\item[\module{packmail}]
--- Create a self-unpacking \UNIX{} shell archive.
\end{description}


The following modules were documented in previous versions of this
manual, but are now considered obsolete.  The source for the
documentation is still available as part of the documentation source
archive.

\begin{description}
\item[\module{ni}]
--- Import modules in ``packages.''  Basic package support is now
built in.  The built-in support is very similar to what is provided in
this module.

\item[\module{rand}]
--- Old interface to the random number generator.

\item[\module{soundex}]
--- Algorithm for collapsing names which sound similar to a shared
key.  The specific algorithm doesn't seem to match any published
algorithm.  (This is an extension module.)
\end{description}


\section{SGI-specific Extension modules}

The following are SGI specific, and may be out of touch with the
current version of reality.

\begin{description}
\item[\module{cl}]
--- Interface to the SGI compression library.

\item[\module{sv}]
--- Interface to the ``simple video'' board on SGI Indigo
(obsolete hardware).
\end{description}


%
%  The ugly "%begin{latexonly}" pseudo-environments are really just to
%  keep LaTeX2HTML quiet during the \renewcommand{} macros; they're
%  not really valuable.
%

%begin{latexonly}
\renewcommand{\indexname}{Module Index}
%end{latexonly}
\input{modlib.ind}		% Module Index

%begin{latexonly}
\renewcommand{\indexname}{Index}
%end{latexonly}
\documentclass{manual}

% NOTE: this file controls which chapters/sections of the library
% manual are actually printed.  It is easy to customize your manual
% by commenting out sections that you're not interested in.

\title{Python Library Reference}

\input{boilerplate}

\makeindex			% tell \index to actually write the
				% .idx file
\makemodindex			% ... and the module index as well.


\begin{document}

\maketitle

\ifhtml
\chapter*{Front Matter\label{front}}
\fi

\input{copyright}

\begin{abstract}

\noindent
Python is an extensible, interpreted, object-oriented programming
language.  It supports a wide range of applications, from simple text
processing scripts to interactive WWW browsers.

While the \emph{Python Reference Manual} describes the exact syntax and
semantics of the language, it does not describe the standard library
that is distributed with the language, and which greatly enhances its
immediate usability.  This library contains built-in modules (written
in C) that provide access to system functionality such as file I/O
that would otherwise be inaccessible to Python programmers, as well as
modules written in Python that provide standardized solutions for many
problems that occur in everyday programming.  Some of these modules
are explicitly designed to encourage and enhance the portability of
Python programs.

This library reference manual documents Python's standard library, as
well as many optional library modules (which may or may not be
available, depending on whether the underlying platform supports them
and on the configuration choices made at compile time).  It also
documents the standard types of the language and its built-in
functions and exceptions, many of which are not or incompletely
documented in the Reference Manual.

This manual assumes basic knowledge about the Python language.  For an
informal introduction to Python, see the \emph{Python Tutorial}; the
\emph{Python Reference Manual} remains the highest authority on
syntactic and semantic questions.  Finally, the manual entitled
\emph{Extending and Embedding the Python Interpreter} describes how to
add new extensions to Python and how to embed it in other applications.

\end{abstract}

\tableofcontents

				% Chapter title:

\input{libintro}		% Introduction

\input{libobjs}			% Built-in Types, Exceptions and Functions
\input{libstdtypes}
\input{libexcs}
\input{libfuncs}

\input{libpython}		% Python Services
\input{libsys}
\input{libtypes}
\input{libuserdict}
\input{liboperator}
\input{libtraceback}
\input{libpickle}
\input{libcopyreg}		% really copy_reg
\input{libshelve}
\input{libcopy}
\input{libmarshal}
\input{libimp}
%\input{libni}
\input{libparser}
\input{libsymbol}
\input{libtoken}
\input{libkeyword}
\input{libcode}
\input{libpprint}
\input{libpycompile}		% really py_compile
\input{libcompileall}
\input{libdis}
\input{libsite}
\input{libuser}
\input{libbltin}		% really __builtin__
\input{libmain}			% really __main__

\input{libstrings}		% String Services
\input{libstring}
\input{libre}
\input{libregex}
\input{libregsub}
\input{libstruct}
\input{libstringio}
%\input{libsoundex}

\input{libmisc}			% Miscellaneous Services
\input{libmath}
\input{libcmath}
\input{libwhrandom}
\input{librandom}
%\input{librand}
\input{libbisect}
\input{libarray}
\input{libfileinput}
\input{libcalendar}
\input{libcmd}
\input{libshlex}

\input{liballos}		% Generic Operating System Services
\input{libos}
\input{libtime}
\input{libgetpass}
\input{libgetopt}
\input{libtempfile}
\input{liberrno}
\input{libglob}
\input{libfnmatch}
\input{libshutil}
\input{liblocale}

\input{libsomeos}		% Optional Operating System Services
\input{libsignal}
\input{libsocket}
\input{libselect}
\input{libthread}
\input{libthreading}
\input{libqueue}
\input{libanydbm}
\input{libwhichdb}
\input{libzlib}
\input{libgzip}

\input{libunix}			% UNIX Specific Services
\input{libposix}
\input{libposixpath}
\input{libpwd}
\input{libgrp}
\input{libcrypt}
\input{libdbm}
\input{libgdbm}
\input{libtermios}
\input{libfcntl}
\input{libposixfile}
\input{libresource}
\input{libsyslog}
\input{libstat}
\input{libpopen2}
\input{libcommands}

\input{libpdb}			% The Python Debugger

\input{libprofile}		% The Python Profiler

\input{internet}		% Internet Protocols
\input{libcgi}
\input{liburllib}
\input{libhttplib}
\input{libftplib}
\input{libgopherlib}
\input{libpoplib}
\input{libimaplib}
\input{libnntplib}
\input{libsmtplib}
\input{liburlparse}
\input{libsocksvr}
\input{libbasehttp}

\input{netdata}
\input{libsgmllib}
\input{libhtmllib}
\input{libxmllib}
\input{libformatter}
\input{librfc822}
\input{libmimetools}
\input{libmultifile}
\input{libbinhex}
\input{libuu}
\input{libbinascii}
\input{libxdrlib}
\input{libmailcap}
\input{libmimetypes}
\input{libbase64}
\input{libquopri}
\input{libmailbox}
\input{libmimify}
\input{libnetrc}

\input{librestricted}
\input{librexec}
\input{libbastion}

\input{libmm}			% Multimedia Services
\input{libaudioop}
\input{libimageop}
\input{libaifc}
\input{libjpeg}
\input{librgbimg}
\input{libimghdr}

\input{libcrypto}		% Cryptographic Services
\input{libmd5}
\input{libmpz}
\input{librotor}

%\input{libamoeba}		% AMOEBA ONLY

%\input{libstdwin}		% STDWIN ONLY

\input{libsgi}			% SGI IRIX ONLY
\input{libal}
\input{libcd}
\input{libfl}
\input{libfm}
\input{libgl}
\input{libimgfile}
%\input{libpanel}

\input{libsun}			% SUNOS ONLY
\input{libsunaudio}

\input{libundoc}

%
%  The ugly "%begin{latexonly}" pseudo-environments are really just to
%  keep LaTeX2HTML quiet during the \renewcommand{} macros; they're
%  not really valuable.
%

%begin{latexonly}
\renewcommand{\indexname}{Module Index}
%end{latexonly}
\input{modlib.ind}		% Module Index

%begin{latexonly}
\renewcommand{\indexname}{Index}
%end{latexonly}
\input{lib.ind}			% Index

\end{document}
			% Index

\end{document}
			% Index

\end{document}
			% Index

\end{document}
