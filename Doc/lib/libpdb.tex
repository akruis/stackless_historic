\chapter{The Python Debugger \label{debugger}}

\declaremodule{standard}{pdb}
\modulesynopsis{The Python debugger for interactive interpreters.}


The module \module{pdb} defines an interactive source code
debugger\index{debugging} for Python programs.  It supports setting
(conditional) breakpoints and single stepping at the source line
level, inspection of stack frames, source code listing, and evaluation
of arbitrary Python code in the context of any stack frame.  It also
supports post-mortem debugging and can be called under program
control.

The debugger is extensible --- it is actually defined as the class
\class{Pdb}\withsubitem{(class in pdb)}{\ttindex{Pdb}}.
This is currently undocumented but easily understood by reading the
source.  The extension interface uses the modules
\module{bdb}\refstmodindex{bdb} (undocumented) and
\refmodule{cmd}\refstmodindex{cmd}.

The debugger's prompt is \samp{(Pdb) }.
Typical usage to run a program under control of the debugger is:

\begin{verbatim}
>>> import pdb
>>> import mymodule
>>> pdb.run('mymodule.test()')
> <string>(0)?()
(Pdb) continue
> <string>(1)?()
(Pdb) continue
NameError: 'spam'
> <string>(1)?()
(Pdb) 
\end{verbatim}

\file{pdb.py} can also be invoked as
a script to debug other scripts.  For example:

\begin{verbatim}
python -m pdb myscript.py
\end{verbatim}

When invoked as a script, pdb will automatically enter post-mortem debugging
if the program being debugged exits abnormally. After post-mortem debugging
(or after normal exit of the program), pdb will restart the program.
Automatic restarting preserves pdb's state (such as breakpoints) and in most
cases is more useful than quitting the debugger upon program's exit.
\versionadded[Restarting post-mortem behavior added]{2.4}

Typical usage to inspect a crashed program is:

\begin{verbatim}
>>> import pdb
>>> import mymodule
>>> mymodule.test()
Traceback (most recent call last):
  File "<stdin>", line 1, in ?
  File "./mymodule.py", line 4, in test
    test2()
  File "./mymodule.py", line 3, in test2
    print spam
NameError: spam
>>> pdb.pm()
> ./mymodule.py(3)test2()
-> print spam
(Pdb) 
\end{verbatim}

The module defines the following functions; each enters the debugger
in a slightly different way:

\begin{funcdesc}{run}{statement\optional{, globals\optional{, locals}}}
Execute the \var{statement} (given as a string) under debugger
control.  The debugger prompt appears before any code is executed; you
can set breakpoints and type \samp{continue}, or you can step through
the statement using \samp{step} or \samp{next} (all these commands are
explained below).  The optional \var{globals} and \var{locals}
arguments specify the environment in which the code is executed; by
default the dictionary of the module \refmodule[main]{__main__} is
used.  (See the explanation of the \keyword{exec} statement or the
\function{eval()} built-in function.)
\end{funcdesc}

\begin{funcdesc}{runeval}{expression\optional{, globals\optional{, locals}}}
Evaluate the \var{expression} (given as a string) under debugger
control.  When \function{runeval()} returns, it returns the value of the
expression.  Otherwise this function is similar to
\function{run()}.
\end{funcdesc}

\begin{funcdesc}{runcall}{function\optional{, argument, ...}}
Call the \var{function} (a function or method object, not a string)
with the given arguments.  When \function{runcall()} returns, it returns
whatever the function call returned.  The debugger prompt appears as
soon as the function is entered.
\end{funcdesc}

\begin{funcdesc}{set_trace}{}
Enter the debugger at the calling stack frame.  This is useful to
hard-code a breakpoint at a given point in a program, even if the code
is not otherwise being debugged (e.g. when an assertion fails).
\end{funcdesc}

\begin{funcdesc}{post_mortem}{traceback}
Enter post-mortem debugging of the given \var{traceback} object.
\end{funcdesc}

\begin{funcdesc}{pm}{}
Enter post-mortem debugging of the traceback found in
\code{sys.last_traceback}.
\end{funcdesc}


\section{Debugger Commands \label{debugger-commands}}

The debugger recognizes the following commands.  Most commands can be
abbreviated to one or two letters; e.g. \samp{h(elp)} means that
either \samp{h} or \samp{help} can be used to enter the help
command (but not \samp{he} or \samp{hel}, nor \samp{H} or
\samp{Help} or \samp{HELP}).  Arguments to commands must be
separated by whitespace (spaces or tabs).  Optional arguments are
enclosed in square brackets (\samp{[]}) in the command syntax; the
square brackets must not be typed.  Alternatives in the command syntax
are separated by a vertical bar (\samp{|}).

Entering a blank line repeats the last command entered.  Exception: if
the last command was a \samp{list} command, the next 11 lines are
listed.

Commands that the debugger doesn't recognize are assumed to be Python
statements and are executed in the context of the program being
debugged.  Python statements can also be prefixed with an exclamation
point (\samp{!}).  This is a powerful way to inspect the program
being debugged; it is even possible to change a variable or call a
function.  When an
exception occurs in such a statement, the exception name is printed
but the debugger's state is not changed.

Multiple commands may be entered on a single line, separated by
\samp{;;}.  (A single \samp{;} is not used as it is
the separator for multiple commands in a line that is passed to
the Python parser.)
No intelligence is applied to separating the commands;
the input is split at the first \samp{;;} pair, even if it is in
the middle of a quoted string.

The debugger supports aliases.  Aliases can have parameters which
allows one a certain level of adaptability to the context under
examination.

If a file \file{.pdbrc}
\indexii{.pdbrc}{file}\indexiii{debugger}{configuration}{file}
exists in the user's home directory or in the current directory, it is
read in and executed as if it had been typed at the debugger prompt.
This is particularly useful for aliases.  If both files exist, the one
in the home directory is read first and aliases defined there can be
overridden by the local file.

\begin{description}

\item[h(elp) \optional{\var{command}}]

Without argument, print the list of available commands.  With a
\var{command} as argument, print help about that command.  \samp{help
pdb} displays the full documentation file; if the environment variable
\envvar{PAGER} is defined, the file is piped through that command
instead.  Since the \var{command} argument must be an identifier,
\samp{help exec} must be entered to get help on the \samp{!} command.

\item[w(here)]

Print a stack trace, with the most recent frame at the bottom.  An
arrow indicates the current frame, which determines the context of
most commands.

\item[d(own)]

Move the current frame one level down in the stack trace
(to an newer frame).

\item[u(p)]

Move the current frame one level up in the stack trace
(to a older frame).

\item[b(reak) \optional{\optional{\var{filename}:}\var{lineno}\code{\Large{|}}\var{function}\optional{, \var{condition}}}]

With a \var{lineno} argument, set a break there in the current
file.  With a \var{function} argument, set a break at the first
executable statement within that function.
The line number may be prefixed with a filename and a colon,
to specify a breakpoint in another file (probably one that
hasn't been loaded yet).  The file is searched on \code{sys.path}.
Note that each breakpoint is assigned a number to which all the other
breakpoint commands refer.

If a second argument is present, it is an expression which must
evaluate to true before the breakpoint is honored.

Without argument, list all breaks, including for each breakpoint,
the number of times that breakpoint has been hit, the current
ignore count, and the associated condition if any.

\item[tbreak \optional{\optional{\var{filename}:}\var{lineno}\code{\Large{|}}\var{function}\optional{, \var{condition}}}]

Temporary breakpoint, which is removed automatically when it is
first hit.  The arguments are the same as break.

\item[cl(ear) \optional{\var{bpnumber} \optional{\var{bpnumber ...}}}]

With a space separated list of breakpoint numbers, clear those
breakpoints.  Without argument, clear all breaks (but first
ask confirmation).

\item[disable \optional{\var{bpnumber} \optional{\var{bpnumber ...}}}]

Disables the breakpoints given as a space separated list of
breakpoint numbers.  Disabling a breakpoint means it cannot cause
the program to stop execution, but unlike clearing a breakpoint, it
remains in the list of breakpoints and can be (re-)enabled.

\item[enable \optional{\var{bpnumber} \optional{\var{bpnumber ...}}}]

Enables the breakpoints specified.

\item[ignore \var{bpnumber} \optional{\var{count}}]

Sets the ignore count for the given breakpoint number.  If
count is omitted, the ignore count is set to 0.  A breakpoint
becomes active when the ignore count is zero.  When non-zero,
the count is decremented each time the breakpoint is reached
and the breakpoint is not disabled and any associated condition
evaluates to true.

\item[condition \var{bpnumber} \optional{\var{condition}}]

Condition is an expression which must evaluate to true before
the breakpoint is honored.  If condition is absent, any existing
condition is removed; i.e., the breakpoint is made unconditional.

\item[commands \optional{\var{bpnumber}}]

Specify a list of commands for breakpoint number \var{bpnumber}.  The
commands themselves appear on the following lines.  Type a line
containing just 'end' to terminate the commands. An example:

\begin{verbatim}
(Pdb) commands 1
(com) print some_variable
(com) end
(Pdb)
\end{verbatim}

To remove all commands from a breakpoint, type commands and
follow it immediately with  end; that is, give no commands.

With no \var{bpnumber} argument, commands refers to the last
breakpoint set.

You can use breakpoint commands to start your program up again.
Simply use the continue command, or step, or any other
command that resumes execution.

Specifying any command resuming execution (currently continue,
step, next, return, jump, quit and their abbreviations) terminates
the command list (as if that command was immediately followed by end).
This is because any time you resume execution
(even with a simple next or step), you may encounter·
another breakpoint--which could have its own command list, leading to
ambiguities about which list to execute.

   If you use the 'silent' command in the command list, the
usual message about stopping at a breakpoint is not printed.  This may
be desirable for breakpoints that are to print a specific message and
then continue.  If none of the other commands print anything, you
see no sign that the breakpoint was reached.

\versionadded{2.5}

\item[s(tep)]

Execute the current line, stop at the first possible occasion
(either in a function that is called or on the next line in the
current function).

\item[n(ext)]

Continue execution until the next line in the current function
is reached or it returns.  (The difference between \samp{next} and
\samp{step} is that \samp{step} stops inside a called function, while
\samp{next} executes called functions at (nearly) full speed, only
stopping at the next line in the current function.)

\item[r(eturn)]

Continue execution until the current function returns.

\item[c(ont(inue))]

Continue execution, only stop when a breakpoint is encountered.

\item[j(ump) \var{lineno}]

Set the next line that will be executed.  Only available in the
bottom-most frame.  This lets you jump back and execute code
again, or jump forward to skip code that you don't want to run.

It should be noted that not all jumps are allowed --- for instance it
is not possible to jump into the middle of a \keyword{for} loop or out
of a \keyword{finally} clause.

\item[l(ist) \optional{\var{first}\optional{, \var{last}}}]

List source code for the current file.  Without arguments, list 11
lines around the current line or continue the previous listing.  With
one argument, list 11 lines around at that line.  With two arguments,
list the given range; if the second argument is less than the first,
it is interpreted as a count.

\item[a(rgs)]

Print the argument list of the current function.

\item[p \var{expression}]

Evaluate the \var{expression} in the current context and print its
value.  \note{\samp{print} can also be used, but is not a debugger
command --- this executes the Python \keyword{print} statement.}

\item[pp \var{expression}]

Like the \samp{p} command, except the value of the expression is
pretty-printed using the \module{pprint} module.

\item[alias \optional{\var{name} \optional{command}}]

Creates an alias called \var{name} that executes \var{command}.  The
command must \emph{not} be enclosed in quotes.  Replaceable parameters
can be indicated by \samp{\%1}, \samp{\%2}, and so on, while \samp{\%*} is
replaced by all the parameters.  If no command is given, the current
alias for \var{name} is shown. If no arguments are given, all
aliases are listed.

Aliases may be nested and can contain anything that can be
legally typed at the pdb prompt.  Note that internal pdb commands
\emph{can} be overridden by aliases.  Such a command is
then hidden until the alias is removed.  Aliasing is recursively
applied to the first word of the command line; all other words
in the line are left alone.

As an example, here are two useful aliases (especially when placed
in the \file{.pdbrc} file):

\begin{verbatim}
#Print instance variables (usage "pi classInst")
alias pi for k in %1.__dict__.keys(): print "%1.",k,"=",%1.__dict__[k]
#Print instance variables in self
alias ps pi self
\end{verbatim}
                
\item[unalias \var{name}]

Deletes the specified alias.

\item[\optional{!}\var{statement}]

Execute the (one-line) \var{statement} in the context of
the current stack frame.
The exclamation point can be omitted unless the first word
of the statement resembles a debugger command.
To set a global variable, you can prefix the assignment
command with a \samp{global} command on the same line, e.g.:

\begin{verbatim}
(Pdb) global list_options; list_options = ['-l']
(Pdb)
\end{verbatim}

\item[q(uit)]

Quit from the debugger.
The program being executed is aborted.

\end{description}

\section{How It Works \label{debugger-hooks}}

Some changes were made to the interpreter:

\begin{itemize}
\item \code{sys.settrace(\var{func})} sets the global trace function
\item there can also a local trace function (see later)
\end{itemize}

Trace functions have three arguments: \var{frame}, \var{event}, and
\var{arg}. \var{frame} is the current stack frame.  \var{event} is a
string: \code{'call'}, \code{'line'}, \code{'return'}, \code{'exception'},
 \code{'c_call'}, \code{'c_return'}, or \code{'c_exception'}. \var{arg}
 depends on the event type.

The global trace function is invoked (with \var{event} set to
\code{'call'}) whenever a new local scope is entered; it should return
a reference to the local trace function to be used that scope, or
\code{None} if the scope shouldn't be traced.

The local trace function should return a reference to itself (or to
another function for further tracing in that scope), or \code{None} to
turn off tracing in that scope.

Instance methods are accepted (and very useful!) as trace functions.

The events have the following meaning:

\begin{description}

\item[\code{'call'}]
A function is called (or some other code block entered).  The global
trace function is called; \var{arg} is \code{None};
the return value specifies the local trace function.

\item[\code{'line'}]
The interpreter is about to execute a new line of code (sometimes
multiple line events on one line exist).  The local trace function is
called; \var{arg} is \code{None}; the return value specifies the new
local trace function.

\item[\code{'return'}]
A function (or other code block) is about to return.  The local trace
function is called; \var{arg} is the value that will be returned.  The
trace function's return value is ignored.

\item[\code{'exception'}]
An exception has occurred.  The local trace function is called;
\var{arg} is a triple \code{(\var{exception}, \var{value},
\var{traceback})}; the return value specifies the new local trace
function.

\item[\code{'c_call'}]
A C function is about to be called.  This may be an extension function
or a builtin.  \var{arg} is the C function object.

\item[\code{'c_return'}]
A C function has returned. \var{arg} is \code{None}.

\item[\code{'c_exception'}]
A C function has thrown an exception.  \var{arg} is \code{None}.

\end{description}

Note that as an exception is propagated down the chain of callers, an
\code{'exception'} event is generated at each level.

For more information on code and frame objects, refer to the
\citetitle[../ref/ref.html]{Python Reference Manual}.
