\section{\module{cgitb} ---
         Traceback manager for CGI scripts}

\declaremodule{standard}{cgitb}
\modulesynopsis{Configurable traceback handler for CGI scripts.}
\moduleauthor{Ka Ping Yee}{ping@lfw.org}
\sectionauthor{Fred L. Drake, Jr.}{fdrake@acm.org}

\index{CGI!tracebacks}
\index{traceback!in CGI scripts}

The \module{cgitb} module provides a special exception handler for CGI
scripts.  Using this module, an exception raised and left uncaught in
a CGI script can be presented as colorized HTML in the user's Web
browser.  The options to the \function{enable()} function can be used
to control whether the colorized traceback is presented and whether
the traceback is logged to a file for later analysis.


\begin{funcdesc}{enable}{\optional{display\optional{, logdir\optional{,
                         context}}}}
  This function causes the \module{cgitb} module to take over the
  interpreters default handling for exceptions that propogate out of
  the top level of a script.  The causes \module{cgitb} to set the
  value of \code{\refmodule{sys}.excepthook}.

  The optional argument \var{display} defaults to true and can be set
  to false to suppress sending the traceback to the browser.
  \var{logdir}, if given, should be a directory to cause tracebacks to
  be written to files there; by default tracebacks will not be written
  to files.  \var{context} is the number of lines of context around
  the offending line in a traceback display; this defaults to
  \code{5}.
\end{funcdesc}

\begin{funcdesc}{handler}{\optional{info}}
  This function handles an exception using the default settings.  This
  can be used when you've caught an exception and want to report it
  using \module{cgitb}, but need to continue processing.
\end{funcdesc}
