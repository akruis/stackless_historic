\section{\module{gettext} ---
         Multilingual internationalization services}

\declaremodule{standard}{gettext}
\modulesynopsis{Multilingual internationalization services.}
\moduleauthor{Barry A. Warsaw}{bwarsaw@beopen.com}
\sectionauthor{Barry A. Warsaw}{bwarsaw@beopen.com}


The \module{gettext} module provides internationalization (I18N) and
localization (L10N) services for your Python modules and applications.
It supports both the GNU \program{gettext} message catalog API and a
higher level, class-based API that may be more appropriate for Python
files.  The interface described below allows you to write your
module and application messages in one natural language, and provide a
catalog of translated messages for running under different natural
languages.

Some hints on localizing your Python modules and applications are also
given.

\subsection{GNU \program{gettext} API}

The \module{gettext} module defines the following API, which is very
similar to the GNU \program{gettext} API.  If you use this API you
will affect the translation of your entire application globally.  Often
this is what you want if your application is monolingual, with the choice
of language dependent on the locale of your user.  If you are
localizing a Python module, or if your application needs to switch
languages on the fly, you probably want to use the class-based API
instead.

\begin{funcdesc}{bindtextdomain}{domain, localedir\code{=None}}
Bind the \var{domain} to the locale directory
\var{localedir}.  More concretely, \module{gettext} will look for
binary \file{.mo} files for the given domain using the path (on Unix):
\file{\var{localedir}/\var{language}/LC_MESSAGES/\var{domain}.mo},
where \var{languages} is searched for in the environment variables
\code{LANGUAGE}, \code{LC_ALL}, \code{LC_MESSAGES}, and \code{LANG}
respectively.

If \var{localedir} is \code{None}, then the current binding for
\var{domain} is returned\footnote{The default locale directory is system
dependent; e.g. on standard RedHat Linux it is
\file{/usr/share/locale}, but on Solaris it is 
\file{/usr/lib/locale}.  The \module{gettext} module does not try to
support these system dependent defaults; instead its default is
\file{\code{sys.prefix}/share/locale}.  For this reason, it is always
best to call \code{gettext.bindtextdomain()} with an explicit absolute
path at the start of your application.}.
\end{funcdesc}

\begin{funcdesc}{textdomain}{domain\code{=None}}
Change or query the current global domain.  If \var{domain} is
\code{None}, then the current global domain is returned, otherwise the
global domain is set to \var{domain}, which is returned.
\end{funcdesc}

\begin{funcdesc}{gettext}{message}
Return the localized translation of \var{message}, based on the
current global domain, language, and locale directory.  This function
is usually aliased as \function{_} in the local namespace (see
examples below).
\end{funcdesc}

\begin{funcdesc}{dgettext}{domain, message}
Like \function{gettext()}, but look the message up in the specified
\var{domain}.
\end{funcdesc}

Note that GNU \program{gettext} also defines a \function{dcgettext()}
method, but this was deemed not useful and so it is currently
unimplemented.

Here's an example of typical usage for this API:

\begin{verbatim}
import gettext
gettext.bindtextdomain('myapplication', '/path/to/my/language/directory')
gettext.textdomain('myapplication')
_ = gettext.gettext
# ...
print _('This is a translatable string.')
\end{verbatim}

\subsection{Class-based API}

The class-based API of the \module{gettext} module gives you more
flexibility and greater convenience than the GNU \program{gettext}
API.  It is the recommended way of localizing your Python applications and
modules.  \module{gettext} defines a ``translations'' class which
implements the parsing of GNU \file{.mo} format files, and has methods
for returning either standard 8-bit strings or Unicode strings.
Translations instances can also install themselves in the built-in
namespace as the function \function{_()}.

\begin{funcdesc}{find}{domain, localedir\code{=None}, languages\code{=None}}
This function implements the standard \file{.mo} file search
algorithm.  It takes a \var{domain}, identical to what
\function{textdomain()} takes, and optionally a \var{localedir} (as in
\function{bindtextdomain()}), and a list of languages.  All arguments
are strings.

If \var{localedir} is not given, then the default system locale
directory is used\footnote{See the footnote for
\function{bindtextdomain()} above.}.  If \var{languages} is not given,
then the following environment variables are searched: \code{LANGUAGE},
\code{LC_ALL}, \code{LC_MESSAGES}, and \code{LANG}.  The first one
returning a non-empty value is used for the \var{languages} variable.
The environment variables can contain a colon separated list of
languages, which will be split.

\function{find()} then expands and normalizes the languages, and then
iterates through them, searching for an existing file built of these
components:

\file{\var{localedir}/\var{language}/LC_MESSAGES/\var{domain}.mo}

The first such file name that exists is returned by \function{find()}.
If no such file is found, then \code{None} is returned.
\end{funcdesc}

\begin{funcdesc}{translation}{domain, localedir\code{=None},
languages\code{=None}, class_\code{=None}}
Return a \class{Translations} instance based on the \var{domain},
\var{localedir}, and \var{languages}, which are first passed to
\function{find()} to get the
associated \file{.mo} file path.  Instances with
identical \file{.mo} file names are cached.  The actual class instantiated
is either \var{class_} if provided, otherwise
\class{GNUTranslations}.  The class's constructor must take a single
file object argument.  If no \file{.mo} file is found, this
function raises \exception{IOError}.
\end{funcdesc}

\begin{funcdesc}{install}{domain, localedir\code{=None}, unicode\code{=0}}
This installs the function \function{_} in Python's builtin namespace,
based on \var{domain}, and \var{localedir} which are passed to the
function \function{translation()}.  The \var{unicode} flag is passed to
the resulting translation object's \method{install} method.

As seen below, you usually mark the strings in your application that are
candidates for translation, by wrapping them in a call to the function
\function{_()}, e.g.

\begin{verbatim}
print _('This string will be translated.')
\end{verbatim}

For convenience, you want the \function{_()} function to be installed in
Python's builtin namespace, so it is easily accessible in all modules
of your application.  
\end{funcdesc}

\subsubsection{The \class{NullTranslations} class}
Translation classes are what actually implement the translation of
original source file message strings to translated message strings.
The base class used by all translation classes is
\class{NullTranslations}; this provides the basic interface you can use
to write your own specialized translation classes.  Here are the
methods of \class{NullTranslations}:

\begin{methoddesc}[NullTranslations]{__init__}{fp\code{=None}}
Takes an optional file object \var{fp}, which is ignored by the base
class.  Initializes ``protected'' instance variables \var{_info} and
\var{_charset} which are set by derived classes.  It then calls
\code{self._parse(fp)} if \var{fp} is not \code{None}.
\end{methoddesc}

\begin{methoddesc}[NullTranslations]{_parse}{fp}
No-op'd in the base class, this method takes file object \var{fp}, and
reads the data from the file, initializing its message catalog.  If
you have an unsupported message catalog file format, you should
override this method to parse your format.
\end{methoddesc}

\begin{methoddesc}[NullTranslations]{gettext}{message}
Return the translated message.  Overridden in derived classes.
\end{methoddesc}

\begin{methoddesc}[NullTranslations]{ugettext}{message}
Return the translated message as a Unicode string.  Overridden in
derived classes.
\end{methoddesc}

\begin{methoddesc}[NullTranslations]{info}{}
Return the ``protected'' \var{_info} variable.
\end{methoddesc}

\begin{methoddesc}[NullTranslations]{charset}{}
Return the ``protected'' \var{_charset} variable.
\end{methoddesc}

\begin{methoddesc}[NullTranslations]{install}{unicode\code{=0}}
If the \var{unicode} flag is false, this method installs
\code{self.gettext} into the built-in namespace, binding it to
\function{_}.  If \var{unicode} is true, it binds \code{self.ugettext}
instead.

Note that this is only one way, albeit the most convenient way, to
make the \function{_} function available to your application.  Because it
affects the entire application globally, and specifically the built-in
namespace, localized modules should never install \function{_}.
Instead, they should use this code to make \function{_} available to
their module:

\begin{verbatim}
import gettext
t = gettext.translation('mymodule', ...)
_ = t.gettext
\end{verbatim}

This puts \function{_} only in the module's global namespace and so
only affects calls within this module.
\end{methoddesc}

\subsubsection{The \class{GNUTranslations} class}

The \module{gettext} module provides one additional class derived from
\class{NullTranslations}: \class{GNUTranslations}.  This class
overrides \method{_parse()} to enable reading GNU \program{gettext}
format \file{.mo} files in both big-endian and little-endian format.

It also parses optional meta-data out of the translation catalog.  It
is convention with GNU \program{gettext} to include meta-data as the
translation for the empty string.  This meta-data is in RFC822-style
\code{key: value} pairs.  If the key \code{Content-Type:} is found,
then the \code{charset} property is used to initialize the
``protected'' \code{_charset} instance variable.  The entire set of
key/value pairs are placed into a dictionary and set as the
``protected'' \code{_info} instance variable.

If the \file{.mo} file's magic number is invalid, or if other problems
occur while reading the file, instantiating a \class{GNUTranslations} class
can raise \exception{IOError}.

The other usefully overridden method is \method{ugettext()}, which
returns a Unicode string by passing both the translated message string
and the value of the ``protected'' \code{_charset} variable to the
builtin \function{unicode()} function.

\subsubsection{Solaris \file{.mo} file support}

The Solaris operating system defines its own binary
\file{.mo} file format, but since no documentation can be found on
this format, it is not supported at this time.

\subsubsection{The Catalog constructor}

GNOME uses a version of the \module{gettext} module by James
Henstridge, but this version has a slightly different API.  Its
documented usage was:

\begin{verbatim}
import gettext
cat = gettext.Catalog(domain, localedir)
_ = cat.gettext
print _('hello world')
\end{verbatim}

For compatibility with this older module, the function
\function{Catalog()} is an alias for the the \function{translation()}
function described above.

One difference between this module and Henstridge's: his catalog
objects supported access through a mapping API, but this appears to be
unused and so is not currently supported.

\subsection{Internationalizing your programs and modules}
Internationalization (I18N) refers to the operation by which a program
is made aware of multiple languages.  Localization (L10N) refers to
the adaptation of your program, once internationalized, to the local
language and cultural habits.  In order to provide multilingual
messages for your Python programs, you need to take the following
steps:

\begin{enumerate}
    \item prepare your program or module by specially marking
          translatable strings
    \item run a suite of tools over your marked files to generate raw
          messages catalogs
    \item create language specific translations of the message catalogs
    \item use the \module{gettext} module so that message strings are
          properly translated
\end{enumerate}

In order to prepare your code for I18N, you need to look at all the
strings in your files.  Any string that needs to be translated
should be marked by wrapping it in \code{_('...')} -- i.e. a call to
the function \function{_()}.  For example:

\begin{verbatim}
filename = 'mylog.txt'
message = _('writing a log message')
fp = open(filename, 'w')
fp.write(message)
fp.close()
\end{verbatim}

In this example, the string ``\code{writing a log message}'' is marked as
a candidate for translation, while the strings ``\code{mylog.txt}'' and
``\code{w}'' are not.

The GNU \program{gettext} package provides a tool, called
\program{xgettext}, that scans C and C++ source code looking for these
specially marked strings.  \program{xgettext} generates what are
called \file{.pot} files, essentially structured human readable files
which contain every marked string in the source code.  These
\file{.pot} files are copied and handed over to human translators who write
language-specific versions for every supported natural language.

For I18N Python programs however, \program{xgettext} won't work; it
doesn't understand the myriad of string types support by Python.  The
standard Python distribution provides a tool called
\program{pygettext} that does though (found in the \file{Tools/i18n}
directory)\footnote{Fran\c cois Pinard has written a program called
\program{xpot} which does a similar job.  It is distributed separately
from the Python distribution.}.  This is a command line script that
supports a similar interface as \program{xgettext}; see its
documentation for details.  Once you've used \program{pygettext} to
create your \file{.pot} files, you can use the standard GNU
\program{gettext} tools to generate your machine-readable \file{.mo}
files, which are readable by the \class{GNUTranslations} class.

How you use the \module{gettext} module in your code depends on
whether you are internationalizing your entire application or a single
module.

\subsubsection{Localizing your module}

If you are localizing your module, you must take care not to make
global changes, e.g. to the built-in namespace.  You should not use
the GNU \program{gettext} API but instead the class-based API.  

Let's say your module is called ``spam'' and the module's various
natural language translation \file{.mo} files reside in
\file{/usr/share/locale} in GNU
\program{gettext} format.  Here's what you would put at the top of
your module:

\begin{verbatim}
import gettext
t = gettext.translation('spam', '/usr/share/locale')
_ = t.gettext
\end{verbatim}

If your translators were providing you with Unicode strings in their
\file{.po} files, you'd instead do:

\begin{verbatim}
import gettext
t = gettext.translation('spam', '/usr/share/locale')
_ = t.ugettext
\end{verbatim}

\subsubsection{Localizing your application}

If you are localizing your application, you can install the \function{_()}
function globally into the built-in namespace, usually in the main driver file
of your application.  This will let all your application-specific
files just use \code{_('...')} without having to explicitly install it in
each file.

In the simple case then, you need only add the following bit of code
to the main driver file of your application:

\begin{verbatim}
import gettext
gettext.install('myapplication')
\end{verbatim}

If you need to set the locale directory or the \code{unicode} flag,
you can pass these into the \function{install()} function:

\begin{verbatim}
import gettext
gettext.install('myapplication', '/usr/share/locale', unicode=1)
\end{verbatim}

\subsubsection{Changing languages on the fly}

If your program needs to support many languages at the same time, you
may want to create multiple translation instances and then switch
between them explicitly, like so:

\begin{verbatim}
import gettext

lang1 = gettext.translation(languages=['en'])
lang2 = gettext.translation(languages=['fr'])
lang3 = gettext.translation(languages=['de'])

# start by using language1
lang1.install()

# ... time goes by, user selects language 2
lang2.install()

# ... more time goes by, user selects language 3
lang3.install()
\end{verbatim}

\subsubsection{Deferred translations}

In most coding situations, strings are translated were they are coded.
Occasionally however, you need to mark strings for translation, but
defer actual translation until later.  A classic example is:

\begin{verbatim}
animals = ['mollusk',
           'albatross',
	   'rat',
	   'penguin',
	   'python',
	   ]
# ...
for a in animals:
    print a
\end{verbatim}

Here, you want to mark the strings in the \code{animals} list as being
translatable, but you don't actually want to translate them until they
are printed.

Here is one way you can handle this situation:

\begin{verbatim}
def _(message): return message

animals = [_('mollusk'),
           _('albatross'),
	   _('rat'),
	   _('penguin'),
	   _('python'),
	   ]

del _

# ...
for a in animals:
    print _(a)
\end{verbatim}

This works because the dummy definition of \function{_()} simply returns
the string unchanged.  And this dummy definition will temporarily
override any definition of \function{_()} in the built-in namespace
(until the \code{del} command).
Take care, though if you have a previous definition of \function{_} in
the local namespace.

Note that the second use of \function{_()} will not identify ``a'' as
being translatable to the \program{pygettext} program, since it is not
a string.

Another way to handle this is with the following example:

\begin{verbatim}
def N_(message): return message

animals = [N_('mollusk'),
           N_('albatross'),
	   N_('rat'),
	   N_('penguin'),
	   N_('python'),
	   ]

# ...
for a in animals:
    print _(a)
\end{verbatim}

In this case, you are marking translatable strings with the function
\function{N_()}\footnote{The choice of \function{N_()} here is totally
arbitrary; it could have just as easily been
\function{MarkThisStringForTranslation()}.},
which won't conflict with any definition of
\function{_()}.  However, you will need to teach your message extraction
program to look for translatable strings marked with \function{N_()}.
\program{pygettext} and \program{xpot} both support this through the
use of command line switches.

\subsection{Acknowledgements}

The following people contributed code, feedback, design suggestions,
previous implementations, and valuable experience to the creation of
this module:

\begin{itemize}
    \item Peter Funk
    \item James Henstridge
    \item Mark-Andre Lemburg
    \item Martin von L\"owis
    \item Fran\c cois Pinard
    \item Barry Warsaw
\end{itemize}
