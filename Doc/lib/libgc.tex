\section{\module{gc} ---
         Garbage Collector interface}

\declaremodule{extension}{gc}
\moduleauthor{Neil Schemenauer}{nascheme@enme.ucalgary.ca}
\sectionauthor{Neil Schemenauer}{nascheme@enme.ucalgary.ca}

This module provides an interface to the optional garbage collector.
It provides the ability to disable the collector, tune the collection
frequency, and set debugging options.  It also provides access to
unreachable objects that the collector found but cannot free.  Since
the collector supplements the reference counting already used in
Python, you can disable the collector if you are sure your program
does not create reference cycles.  The collector can be disabled by
calling \code{gc.set_threshold(0)}.  To debug a leaking program call
\code{gc.set_debug(gc.DEBUG_LEAK)}.

The \module{gc} module provides the following functions:

\begin{funcdesc}{collect}{}
Run a full collection.  All generations are examined and the
number of unreachable objects found is returned.
\end{funcdesc}

\begin{funcdesc}{set_debug}{flags}
Set the garbage collection debugging flags.
Debugging information will be written to \code{sys.stderr}.  See below
for a list of debugging flags which can be combined using bit
operations to control debugging.
\end{funcdesc}

\begin{funcdesc}{get_debug}{}
Return the debugging flags currently set.
\end{funcdesc}

\begin{funcdesc}{set_threshold}{threshold0\optional{,
                                threshold1\optional{, threshold2}}}
Set the garbage collection thresholds (the collection frequency).
Setting \var{threshold0} to zero disables collection.

The GC classifies objects into three generations depending on how many
collection sweeps they have survived.  New objects are placed in the
youngest generation (generation \code{0}).  If an object survives a
collection it is moved into the next older generation.  Since
generation \code{2} is the oldest generation, objects in that
generation remain there after a collection.  In order to decide when
to run, the collector keeps track of the number object allocations and
deallocations since the last collection.  When the number of
allocations minus the number of deallocations exceeds
\var{threshold0}, collection starts.  Initially only generation
\code{0} is examined.  If generation \code{0} has been examined more
than \var{threshold1} times since generation \code{1} has been
examined, then generation \code{1} is examined as well.  Similarly,
\var{threshold2} controls the number of collections of generation
\code{1} before collecting generation \code{2}.
\end{funcdesc}

\begin{funcdesc}{get_threshold}{}
Return the current collection thresholds as a tuple of
\code{(\var{threshold0}, \var{threshold1}, \var{threshold2})}.
\end{funcdesc}


The following variable is provided for read-only access:

\begin{datadesc}{garbage}
A list of objects which the collector found to be unreachable
but could not be freed (uncollectable objects).  Objects that have
\method{__del__()} methods and create part of a reference cycle cause
the entire reference cycle to be uncollectable.  
\end{datadesc}


The following constants are provided for use with
\function{set_debug()}:

\begin{datadesc}{DEBUG_STATS}
Print statistics during collection.  This information can
be useful when tuning the collection frequency.
\end{datadesc}

\begin{datadesc}{DEBUG_COLLECTABLE}
Print information on collectable objects found.
\end{datadesc}

\begin{datadesc}{DEBUG_UNCOLLECTABLE}
Print information of uncollectable objects found (objects which are
not reachable but cannot be freed by the collector).  These objects
will be added to the \code{garbage} list.
\end{datadesc}

\begin{datadesc}{DEBUG_INSTANCES}
When \constant{DEBUG_COLLECTABLE} or \constant{DEBUG_UNCOLLECTABLE} is
set, print information about instance objects found.
\end{datadesc}

\begin{datadesc}{DEBUG_OBJECTS}
When \constant{DEBUG_COLLECTABLE} or \constant{DEBUG_UNCOLLECTABLE} is
set, print information about objects other than instance objects found.
\end{datadesc}

\begin{datadesc}{DEBUG_LEAK}
The debugging flags necessary for the collector to print
information about a leaking program (equal to \code{DEBUG_COLLECTABLE |
DEBUG_UNCOLLECTABLE | DEBUG_INSTANCES | DEBUG_OBJECTS}).  
\end{datadesc}
