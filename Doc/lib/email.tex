% Copyright (C) 2001 Python Software Foundation
% Author: barry@zope.com (Barry Warsaw)

\section{\module{email} ---
	 An email and MIME handling package}

\declaremodule{standard}{email}
\modulesynopsis{Package supporting the parsing, manipulating, and
    generating email messages, including MIME documents.}
\moduleauthor{Barry A. Warsaw}{barry@zope.com}

\versionadded{2.2}

The \module{email} package is a library for managing email messages,
including MIME and other \rfc{2822}-based message documents.  It
subsumes most of the functionality in several older standard modules
such as \module{rfc822}, \module{mimetools}, \module{multifile}, and
other non-standard packages such as \module{mimecntl}.

The primary distinguishing feature of the \module{email} package is
that it splits the parsing and generating of email messages from the
internal \emph{object model} representation of email.  Applications
using the \module{email} package deal primarily with objects; you can
add sub-objects to messages, remove sub-objects from messages,
completely re-arrange the contents, etc.  There is a separate parser
and a separate generator which handles the transformation from flat
text to the object module, and then back to flat text again.  There
are also handy subclasses for some common MIME object types, and a few
miscellaneous utilities that help with such common tasks as extracting
and parsing message field values, creating RFC-compliant dates, etc.

The following sections describe the functionality of the
\module{email} package.  The ordering follows a progression that
should be common in applications: an email message is read as flat
text from a file or other source, the text is parsed to produce an
object model representation of the email message, this model is
manipulated, and finally the model is rendered back into
flat text.

It is perfectly feasible to create the object model out of whole cloth
-- i.e. completely from scratch.  From there, a similar progression can
be taken as above.  

Also included are detailed specifications of all the classes and
modules that the \module{email} package provides, the exception
classes you might encounter while using the \module{email} package,
some auxiliary utilities, and a few examples.  For users of the older
\module{mimelib} package, from which the \module{email} package is
descendent, a section on differences and porting is provided.

\subsection{Representing an email message}

The primary object in the \module{email} package is the
\class{Message} class, provided in the \refmodule{email.Message}
module.  \class{Message} is the base class for the \module{email}
object model.  It provides the core functionality for setting and
querying header fields, and for accessing message bodies.

Conceptually, a \class{Message} object consists of \emph{headers} and
\emph{payloads}.  Headers are \rfc{2822} style field name and
values where the field name and value are separated by a colon.  The
colon is not part of either the field name or the field value.

Headers are stored and returned in case-preserving form but are
matched case-insensitively.  There may also be a single
\emph{Unix-From} header, also known as the envelope header or the
\mailheader{From_} header.  The payload is either a string in the case of
simple message objects, a list of \class{Message} objects for
multipart MIME documents, or a single \class{Message} instance for
\mimetype{message/rfc822} type objects.

\class{Message} objects provide a mapping style interface for
accessing the message headers, and an explicit interface for accessing
both the headers and the payload.  It provides convenience methods for
generating a flat text representation of the message object tree, for
accessing commonly used header parameters, and for recursively walking
over the object tree.

\subsection{Parsing email messages}
Message object trees can be created in one of two ways: they can be
created from whole cloth by instantiating \class{Message} objects and
stringing them together via \method{add_payload()} and
\method{set_payload()} calls, or they can be created by parsing a flat text
representation of the email message.

The \module{email} package provides a standard parser that understands
most email document structures, including MIME documents.  You can
pass the parser a string or a file object, and the parser will return
to you the root \class{Message} instance of the object tree.  For
simple, non-MIME messages the payload of this root object will likely
be a string (e.g. containing the text of the message).  For MIME
messages, the root object will return 1 from its
\method{is_multipart()} method, and the subparts can be accessed via
the \method{get_payload()} and \method{walk()} methods.

Note that the parser can be extended in limited ways, and of course
you can implement your own parser completely from scratch.  There is
no magical connection between the \module{email} package's bundled
parser and the
\class{Message} class, so your custom parser can create message object
trees in any way it find necessary.  The \module{email} package's
parser is described in detail in the \refmodule{email.Parser} module
documentation.

\subsection{Generating MIME documents}
One of the most common tasks is to generate the flat text of the email
message represented by a message object tree.  You will need to do
this if you want to send your message via the \refmodule{smtplib}
module or the \refmodule{nntplib} module, or print the message on the
console.  Taking a message object tree and producing a flat text
document is the job of the \refmodule{email.Generator} module.

Again, as with the \refmodule{email.Parser} module, you aren't limited
to the functionality of the bundled generator; you could write one
from scratch yourself.  However the bundled generator knows how to
generate most email in a standards-compliant way, should handle MIME
and non-MIME email messages just fine, and is designed so that the
transformation from flat text, to an object tree via the
\class{Parser} class,
and back to flat text, be idempotent (the input is identical to the
output).

\subsection{Creating email and MIME objects from scratch}

Ordinarily, you get a message object tree by passing some text to a
parser, which parses the text and returns the root of the message
object tree.  However you can also build a complete object tree from
scratch, or even individual \class{Message} objects by hand.  In fact,
you can also take an existing tree and add new \class{Message}
objects, move them around, etc.  This makes a very convenient
interface for slicing-and-dicing MIME messages.

You can create a new object tree by creating \class{Message}
instances, adding payloads and all the appropriate headers manually.
For MIME messages though, the \module{email} package provides some
convenient classes to make things easier.  Each of these classes
should be imported from a module with the same name as the class, from
within the \module{email} package.  E.g.:

\begin{verbatim}
import email.MIMEImage.MIMEImage
\end{verbatim}

or

\begin{verbatim}
from email.MIMEText import MIMEText
\end{verbatim}

Here are the classes:

\begin{classdesc}{MIMEBase}{_maintype, _subtype, **_params}
This is the base class for all the MIME-specific subclasses of
\class{Message}.  Ordinarily you won't create instances specifically
of \class{MIMEBase}, although you could.  \class{MIMEBase} is provided
primarily as a convenient base class for more specific MIME-aware
subclasses.

\var{_maintype} is the \code{Content-Type:} major type (e.g. \code{text} or
\code{image}), and \var{_subtype} is the \code{Content-Type:} minor type
(e.g. \code{plain} or \code{gif}).  \var{_params} is a parameter
key/value dictionary and is passed directly to
\method{Message.add_header()}.

The \class{MIMEBase} class always adds a \mailheader{Content-Type} header
(based on \var{_maintype}, \var{_subtype}, and \var{_params}), and a
\mailheader{MIME-Version} header (always set to \code{1.0}).
\end{classdesc}

\begin{classdesc}{MIMEImage}{_imagedata\optional{, _subtype\optional{,
    _encoder\optional{, **_params}}}}

A subclass of \class{MIMEBase}, the \class{MIMEImage} class is used to
create MIME message objects of major type \mimetype{image}.
\var{_imagedata} is a string containing the raw image data.  If this
data can be decoded by the standard Python module \refmodule{imghdr},
then the subtype will be automatically included in the
\mailheader{Content-Type} header.  Otherwise you can explicitly specify the
image subtype via the \var{_subtype} parameter.  If the minor type could
not be guessed and \var{_subtype} was not given, then \exception{TypeError}
is raised.

Optional \var{_encoder} is a callable (i.e. function) which will
perform the actual encoding of the image data for transport.  This
callable takes one argument, which is the \class{MIMEImage} instance.
It should use \method{get_payload()} and \method{set_payload()} to
change the payload to encoded form.  It should also add any
\mailheader{Content-Transfer-Encoding} or other headers to the message
object as necessary.  The default encoding is \emph{Base64}.  See the
\refmodule{email.Encoders} module for a list of the built-in encoders.

\var{_params} are passed straight through to the \class{MIMEBase}
constructor.
\end{classdesc}

\begin{classdesc}{MIMEText}{_text\optional{, _subtype\optional{,
    _charset\optional{, _encoder}}}}
A subclass of \class{MIMEBase}, the \class{MIMEText} class is used to
create MIME objects of major type \mimetype{text}.  \var{_text} is the string
for the payload.  \var{_subtype} is the minor type and defaults to
\mimetype{plain}.  \var{_charset} is the character set of the text and is
passed as a parameter to the \class{MIMEBase} constructor; it defaults
to \code{us-ascii}.  No guessing or encoding is performed on the text
data, but a newline is appended to \var{_text} if it doesn't already
end with a newline.

The \var{_encoding} argument is as with the \class{MIMEImage} class
constructor, except that the default encoding for \class{MIMEText}
objects is one that doesn't actually modify the payload, but does set
the \mailheader{Content-Transfer-Encoding} header to \code{7bit} or
\code{8bit} as appropriate.
\end{classdesc}

\begin{classdesc}{MIMEMessage}{_msg\optional{, _subtype}}
A subclass of \class{MIMEBase}, the \class{MIMEMessage} class is used to
create MIME objects of main type \mimetype{message}.  \var{_msg} is used as
the payload, and must be an instance of class \class{Message} (or a
subclass thereof), otherwise a \exception{TypeError} is raised.

Optional \var{_subtype} sets the subtype of the message; it defaults
to \mimetype{rfc822}.
\end{classdesc}

\subsection{Encoders, Exceptions, Utilities, and Iterators}

The \module{email} package provides various encoders for safe
transport of binary payloads in \class{MIMEImage} and \class{MIMEText}
instances.  See the \refmodule{email.Encoders} module for more
details.

All of the class exceptions that the \module{email} package can raise
are available in the \refmodule{email.Errors} module.

Some miscellaneous utility functions are available in the
\refmodule{email.Utils} module.

Iterating over a message object tree is easy with the
\method{Message.walk()} method; some additional helper iterators are
available in the \refmodule{email.Iterators} module.

\subsection{Differences from \module{mimelib}}

The \module{email} package was originally prototyped as a separate
library called \module{mimelib}.  Changes have been made so that
method names are more consistent, and some methods or modules have
either been added or removed.  The semantics of some of the methods
have also changed.  For the most part, any functionality available in
\module{mimelib} is still available in the \module{email} package,
albeit often in a different way.

Here is a brief description of the differences between the
\module{mimelib} and the \module{email} packages, along with hints on
how to port your applications.

Of course, the most visible difference between the two packages is
that the package name has been changed to \module{email}.  In
addition, the top-level package has the following differences:

\begin{itemize}
\item \function{messageFromString()} has been renamed to
      \function{message_from_string()}.
\item \function{messageFromFile()} has been renamed to
      \function{message_from_file()}.
\end{itemize}

The \class{Message} class has the following differences:

\begin{itemize}
\item The method \method{asString()} was renamed to \method{as_string()}.
\item The method \method{ismultipart()} was renamed to
      \method{is_multipart()}.
\item The \method{get_payload()} method has grown a \var{decode}
      optional argument.
\item The method \method{getall()} was renamed to \method{get_all()}.
\item The method \method{addheader()} was renamed to \method{add_header()}.
\item The method \method{gettype()} was renamed to \method{get_type()}.
\item The method\method{getmaintype()} was renamed to
      \method{get_main_type()}.
\item The method \method{getsubtype()} was renamed to
      \method{get_subtype()}.
\item The method \method{getparams()} was renamed to
      \method{get_params()}.
      Also, whereas \method{getparams()} returned a list of strings,
      \method{get_params()} returns a list of 2-tuples, effectively
      the key/value pairs of the parameters, split on the \samp{=}
      sign.
\item The method \method{getparam()} was renamed to \method{get_param()}.
\item The method \method{getcharsets()} was renamed to
      \method{get_charsets()}.
\item The method \method{getfilename()} was renamed to
      \method{get_filename()}.
\item The method \method{getboundary()} was renamed to
      \method{get_boundary()}.
\item The method \method{setboundary()} was renamed to
      \method{set_boundary()}.
\item The method \method{getdecodedpayload()} was removed.  To get
      similar functionality, pass the value 1 to the \var{decode} flag
      of the {get_payload()} method.
\item The method \method{getpayloadastext()} was removed.  Similar
      functionality
      is supported by the \class{DecodedGenerator} class in the
      \refmodule{email.Generator} module.
\item The method \method{getbodyastext()} was removed.  You can get
      similar functionality by creating an iterator with
      \function{typed_subpart_iterator()} in the
      \refmodule{email.Iterators} module.
\end{itemize}

The \class{Parser} class has no differences in its public interface.
It does have some additional smarts to recognize
\mimetype{message/delivery-status} type messages, which it represents as
a \class{Message} instance containing separate \class{Message}
subparts for each header block in the delivery status
notification\footnote{Delivery Status Notifications (DSN) are defined
in \rfc{1894}}.

The \class{Generator} class has no differences in its public
interface.  There is a new class in the \refmodule{email.Generator}
module though, called \class{DecodedGenerator} which provides most of
the functionality previously available in the
\method{Message.getpayloadastext()} method.

The following modules and classes have been changed:

\begin{itemize}
\item The \class{MIMEBase} class constructor arguments \var{_major}
      and \var{_minor} have changed to \var{_maintype} and
      \var{_subtype} respectively.
\item The \code{Image} class/module has been renamed to
      \code{MIMEImage}.  The \var{_minor} argument has been renamed to
      \var{_subtype}.
\item The \code{Text} class/module has been renamed to
      \code{MIMEText}.  The \var{_minor} argument has been renamed to
      \var{_subtype}.
\item The \code{MessageRFC822} class/module has been renamed to
      \code{MIMEMessage}.  Note that an earlier version of
      \module{mimelib} called this class/module \code{RFC822}, but
      that clashed with the Python standard library module
      \refmodule{rfc822} on some case-insensitive file systems.

      Also, the \class{MIMEMessage} class now represents any kind of
      MIME message with main type \mimetype{message}.  It takes an
      optional argument \var{_subtype} which is used to set the MIME
      subtype.  \var{_subtype} defaults to \mimetype{rfc822}.
\end{itemize}

\module{mimelib} provided some utility functions in its
\module{address} and \module{date} modules.  All of these functions
have been moved to the \refmodule{email.Utils} module.

The \code{MsgReader} class/module has been removed.  Its functionality
is most closely supported in the \function{body_line_iterator()}
function in the \refmodule{email.Iterators} module.

\subsection{Examples}

Coming soon...

