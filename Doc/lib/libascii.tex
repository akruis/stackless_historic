\section{\module{curses.ascii} ---
         Utilities for ASCII characters}

\declaremodule{standard}{curses.ascii}
\modulesynopsis{Constants and set-membership functions for
                \ASCII{} characters.}
\moduleauthor{Eric S. Raymond}{esr@thyrsus.com}
\sectionauthor{Eric S. Raymond}{esr@thyrsus.com}

\versionadded{2.0}

The \module{curses.ascii} module supplies name constants for
\ASCII{} characters and functions to test membership in various
\ASCII{} character classes.  The constants supplied are names for
control characters as follows:

\begin{tableii}{l|l}{constant}{Name}{Meaning}
  \lineii{NUL}{}
  \lineii{SOH}{Start of heading, console interrupt}
  \lineii{STX}{Start of text}
  \lineii{ETX}{End of text}
  \lineii{EOT}{End of transmission}
  \lineii{ENQ}{Enquiry, goes with \constant{ACK} flow control}
  \lineii{ACK}{Acknowledgement}
  \lineii{BEL}{Bell}
  \lineii{BS}{Backspace}
  \lineii{TAB}{Tab}
  \lineii{HT}{Alias for \constant{TAB}: ``Horizontal tab''}
  \lineii{LF}{Line feed}
  \lineii{NL}{Alias for \constant{LF}: ``New line''}
  \lineii{VT}{Vertical tab}
  \lineii{FF}{Form feed}
  \lineii{CR}{Carriage return}
  \lineii{SO}{Shift-out, begin alternate character set}
  \lineii{SI}{Shift-in, resume default character set}
  \lineii{DLE}{Data-link escape}
  \lineii{DC1}{XON, for flow control}
  \lineii{DC2}{Device control 2, block-mode flow control}
  \lineii{DC3}{XOFF, for flow control}
  \lineii{DC4}{Device control 4}
  \lineii{NAK}{Negative acknowledgement}
  \lineii{SYN}{Synchronous idle}
  \lineii{ETB}{End transmission block}
  \lineii{CAN}{Cancel}
  \lineii{EM}{End of medium}
  \lineii{SUB}{Substitute}
  \lineii{ESC}{Escape}
  \lineii{FS}{File separator}
  \lineii{GS}{Group separator}
  \lineii{RS}{Record separator, block-mode terminator}
  \lineii{US}{Unit separator}
  \lineii{SP}{Space}
  \lineii{DEL}{Delete}
\end{tableii}

Note that many of these have little practical use in modern usage.

The module supplies the following functions, patterned on those in the
standard C library:


\begin{funcdesc}{isalnum}{c}
Checks for an \ASCII{} alphanumeric character; it is equivalent to
\samp{isalpha(\var{c}) or isdigit(\var{c})}.
\end{funcdesc}

\begin{funcdesc}{isalpha}{c}
Checks for an \ASCII{} alphabetic character; it is equivalent to
\samp{isupper(\var{c}) or islower(\var{c})}.
\end{funcdesc}

\begin{funcdesc}{isascii}{c}
Checks for a character value that fits in the 7-bit \ASCII{} set.
\end{funcdesc}

\begin{funcdesc}{isblank}{c}
Checks for an \ASCII{} whitespace character.
\end{funcdesc}

\begin{funcdesc}{iscntrl}{c}
Checks for an \ASCII{} control character (in the range 0x00 to 0x1f).
\end{funcdesc}

\begin{funcdesc}{isdigit}{c}
Checks for an \ASCII{} decimal digit, \character{0} through
\character{9}.  This is equivalent to \samp{\var{c} in string.digits}.
\end{funcdesc}

\begin{funcdesc}{isgraph}{c}
Checks for \ASCII{} any printable character except space.
\end{funcdesc}

\begin{funcdesc}{islower}{c}
Checks for an \ASCII{} lower-case character.
\end{funcdesc}

\begin{funcdesc}{isprint}{c}
Checks for any \ASCII{} printable character including space.
\end{funcdesc}

\begin{funcdesc}{ispunct}{c}
Checks for any printable \ASCII{} character which is not a space or an
alphanumeric character.
\end{funcdesc}

\begin{funcdesc}{isspace}{c}
Checks for \ASCII{} white-space characters; space, tab, line feed,
carriage return, form feed, horizontal tab, vertical tab.
\end{funcdesc}

\begin{funcdesc}{isupper}{c}
Checks for an \ASCII{} uppercase letter.
\end{funcdesc}

\begin{funcdesc}{isxdigit}{c}
Checks for an \ASCII{} hexadecimal digit.  This is equivalent to
\samp{\var{c} in string.hexdigits}.
\end{funcdesc}

\begin{funcdesc}{isctrl}{c}
Checks for an \ASCII{} control character (ordinal values 0 to 31).
\end{funcdesc}

\begin{funcdesc}{ismeta}{c}
Checks for a non-\ASCII{} character (ordinal values 0x80 and above).
\end{funcdesc}

These functions accept either integers or strings; when the argument
is a string, it is first converted using the built-in function
\function{ord()}.

Note that all these functions check ordinal bit values derived from the 
first character of the string you pass in; they do not actually know
anything about the host machine's character encoding.  For functions 
that know about the character encoding (and handle
internationalization properly) see the \refmodule{string} module.

The following two functions take either a single-character string or
integer byte value; they return a value of the same type.

\begin{funcdesc}{ascii}{c}
Return the ASCII value corresponding to the low 7 bits of \var{c}.
\end{funcdesc}

\begin{funcdesc}{ctrl}{c}
Return the control character corresponding to the given character
(the character bit value is bitwise-anded with 0x1f).
\end{funcdesc}

\begin{funcdesc}{alt}{c}
Return the 8-bit character corresponding to the given ASCII character
(the character bit value is bitwise-ored with 0x80).
\end{funcdesc}

The following function takes either a single-character string or
integer value; it returns a string.

\begin{funcdesc}{unctrl}{c}
Return a string representation of the \ASCII{} character \var{c}.  If
\var{c} is printable, this string is the character itself.  If the
character is a control character (0x00-0x1f) the string consists of a
caret (\character{\^}) followed by the corresponding uppercase letter.
If the character is an \ASCII{} delete (0x7f) the string is
\code{'\^{}?'}.  If the character has its meta bit (0x80) set, the meta
bit is stripped, the preceding rules applied, and
\character{!} prepended to the result.
\end{funcdesc}

\begin{datadesc}{controlnames}
A 33-element string array that contains the \ASCII{} mnemonics for the
thirty-two \ASCII{} control characters from 0 (NUL) to 0x1f (US), in
order, plus the mnemonic \samp{SP} for the space character.
\end{datadesc}
