\chapter{Optional Operating System Services}
\label{someos}

The modules described in this chapter provide interfaces to operating
system features that are available on selected operating systems only.
The interfaces are generally modelled after the \UNIX{} or \C{}
interfaces but they are available on some other systems as well
(e.g. Windows or NT).  Here's an overview:

\begin{description}

\item[signal]
--- Set handlers for asynchronous events.

\item[socket]
--- Low-level networking interface.

\item[select]
--- Wait for I/O completion on multiple streams.

\item[thread]
--- Create multiple threads of control within one namespace.

\item[Queue]
--- A stynchronized queue class.

\item[anydbm]
--- Generic interface to DBM-style database modules.

\item[whichdb]
--- Guess which DBM-style module created a given database.

\item[zlib]
\item[gzip]
--- Compression and decompression compatible with the
\program{gzip} program (\module{zlib} is the low-level interface,
\module{gzip} the high-level one).

\end{description}
