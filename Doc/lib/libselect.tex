\section{\module{select} ---
         Wait for I/O completion on multiple streams.}
\declaremodule{builtin}{select}

\modulesynopsis{Wait for I/O completion on multiple streams.}


This module provides access to the function \cfunction{select()}
available in most \UNIX{} versions.  It defines the following:

\begin{excdesc}{error}
The exception raised when an error occurs.  The accompanying value is
a pair containing the numeric error code from \cdata{errno} and the
corresponding string, as would be printed by the \C{} function
\cfunction{perror()}.
\end{excdesc}

\begin{funcdesc}{select}{iwtd, owtd, ewtd\optional{, timeout}}
This is a straightforward interface to the \UNIX{} \cfunction{select()}
system call.  The first three arguments are lists of `waitable
objects': either integers representing \UNIX{} file descriptors or
objects with a parameterless method named \method{fileno()} returning
such an integer.  The three lists of waitable objects are for input,
output and `exceptional conditions', respectively.  Empty lists are
allowed.  The optional \var{timeout} argument specifies a time-out as a
floating point number in seconds.  When the \var{timeout} argument
is omitted the function blocks until at least one file descriptor is
ready.  A time-out value of zero specifies a poll and never blocks.

The return value is a triple of lists of objects that are ready:
subsets of the first three arguments.  When the time-out is reached
without a file descriptor becoming ready, three empty lists are
returned.

Amongst the acceptable object types in the lists are Python file
objects (e.g. \code{sys.stdin}, or objects returned by
\function{open()} or \function{os.popen()}), socket objects
returned by \function{socket.socket()},%
\withsubitem{(in module socket)}{\ttindex{socket()}}
\withsubitem{(in module posix)}{\ttindex{popen()}}
\withsubitem{(in module os)}{\ttindex{popen()}}
and the module \module{stdwin}\refbimodindex{stdwin} which happens to
define a function \function{fileno()}%
\withsubitem{(in module stdwin)}{\ttindex{fileno()}}
for just this purpose.  You may
also define a \dfn{wrapper} class yourself, as long as it has an
appropriate \method{fileno()} method (that really returns a \UNIX{}
file descriptor, not just a random integer).
\end{funcdesc}
