\section{\module{zipimport} ---
         Import modules from Zip archives}

\declaremodule{standard}{zipimport}
\modulesynopsis{support for importing Python modules from ZIP archives.}
\moduleauthor{Just van Rossum}{just@letterror.com}

\versionadded{2.3}

This module adds the ability to import Python modules (\file{*.py},
\file{*.py[co]}) and packages from ZIP-format archives. It is usually
not needed to use the \module{zipimport} module explicitly; it is
automatically used by the builtin \keyword{import} mechanism for
\code{sys.path} items that are paths to ZIP archives.

Typically, \code{sys.path} is a list of directory names as strings.  This
module also allows an item of \code{sys.path} to be a string naming a ZIP
file archive. The ZIP archive can contain a subdirectory structure to
support package imports, and a path within the archive can be specified to
only import from a subdirectory.  For example, the path
\file{/tmp/example.zip/lib/} would only import from the
\file{lib/} subdirectory within the archive.

Any files may be present in the ZIP archive, but only files \file{.py} and
\file{.py[co]} are available for import.  ZIP import of dynamic modules
(\file{.pyd}, \file{.so}) is disallowed. Note that if an archive only
contains \file{.py} files, Python will not attempt to modify the archive
by adding the corresponding \file{.pyc} or \file{.pyo} file, meaning that
if a ZIP archive doesn't contain \file{.pyc} files, importing may be rather
slow.

Using the built-in \function{reload()} function will
fail if called on a module loaded from a ZIP archive; it is unlikely that
\function{reload()} would be needed, since this would imply that the ZIP
has been altered during runtime.

The available attributes of this module are:

\begin{excdesc}{ZipImportError}
  Exception raised by zipimporter objects. It's a subclass of
  \exception{ImportError}, so it can be caught as \exception{ImportError},
  too.
\end{excdesc}

\begin{classdesc*}{zipimporter}
  The class for importing ZIP files.  See
  ``\citetitle{zipimporter Objects}'' (section \ref{zipimporter-objects})
  for constructor details.
\end{classdesc*}


\begin{seealso}
  \seetitle[http://www.pkware.com/business_and_developers/developer/appnote/]
           {PKZIP Application Note}{Documentation on the ZIP file format by
            Phil Katz, the creator of the format and algorithms used.}

  \seepep{0273}{Import Modules from Zip Archives}{Written by James C.
          Ahlstrom, who also provided an implementation. Python 2.3
          follows the specification in PEP 273, but uses an
          implementation written by Just van Rossum that uses the import
          hooks described in PEP 302.}

  \seepep{0302}{New Import Hooks}{The PEP to add the import hooks that help
          this module work.}
\end{seealso}


\subsection{zipimporter Objects \label{zipimporter-objects}}

\begin{classdesc}{zipimporter}{archivepath} 
  Create a new zipimporter instance. \var{archivepath} must be a path to
  a zipfile.  \exception{ZipImportError} is raised if \var{archivepath}
  doesn't point to a valid ZIP archive.
\end{classdesc}

\begin{methoddesc}{find_module}{fullname\optional{, path}}
  Search for a module specified by \var{fullname}. \var{fullname} must be
  the fully qualified (dotted) module name. It returns the zipimporter
  instance itself if the module was found, or \constant{None} if it wasn't.
  The optional \var{path} argument is ignored---it's there for 
  compatibility with the importer protocol.
\end{methoddesc}

\begin{methoddesc}{get_code}{fullname}
  Return the code object for the specified module. Raise
  \exception{ZipImportError} if the module couldn't be found.
\end{methoddesc}

\begin{methoddesc}{get_data}{pathname}
  Return the data associated with \var{pathname}. Raise \exception{IOError}
  if the file wasn't found.
\end{methoddesc}

\begin{methoddesc}{get_source}{fullname}
  Return the source code for the specified module. Raise
  \exception{ZipImportError} if the module couldn't be found, return
  \constant{None} if the archive does contain the module, but has
  no source for it.
\end{methoddesc}

\begin{methoddesc}{is_package}{fullname}
  Return True if the module specified by \var{fullname} is a package.
  Raise \exception{ZipImportError} if the module couldn't be found.
\end{methoddesc}

\begin{methoddesc}{load_module}{fullname}
  Load the module specified by \var{fullname}. \var{fullname} must be the
  fully qualified (dotted) module name. It returns the imported
  module, or raises \exception{ZipImportError} if it wasn't found.
\end{methoddesc}

\subsection{Examples}
\nodename{zipimport Examples}

Here is an example that imports a module from a ZIP archive - note that
the \module{zipimport} module is not explicitly used.

\begin{verbatim}
$ unzip -l /tmp/example.zip
Archive:  /tmp/example.zip
  Length     Date   Time    Name
 --------    ----   ----    ----
     8467  11-26-02 22:30   jwzthreading.py
 --------                   -------
     8467                   1 file
$ ./python
Python 2.3 (#1, Aug 1 2003, 19:54:32) 
>>> import sys
>>> sys.path.insert(0, '/tmp/example.zip')  # Add .zip file to front of path
>>> import jwzthreading
>>> jwzthreading.__file__
'/tmp/example.zip/jwzthreading.py'
\end{verbatim}
