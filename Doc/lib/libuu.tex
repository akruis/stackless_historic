\section{Standard Module \module{uu}}
\declaremodule{standard}{uu}

\modulesynopsis{Encode and decode files in uuencode format.}


This module encodes and decodes files in uuencode format, allowing
arbitrary binary data to be transferred over ascii-only connections.
Wherever a file argument is expected, the methods accept a file-like
object.  For backwards compatibility, a string containing a pathname
is also accepted, and the corresponding file will be opened for
reading and writing; the pathname \code{'-'} is understood to mean the
standard input or output.  However, this interface is deprecated; it's
better for the caller to open the file itself, and be sure that, when
required, the mode is \code{'rb'} or \code{'wb'} on Windows or DOS.

This code was contributed by Lance Ellinghouse, and modified by Jack
Jansen.
\index{Jansen, Jack}
\index{Ellinghouse, Lance}

The \module{uu} module defines the following functions:

\begin{funcdesc}{encode}{in_file, out_file\optional{, name\optional{, mode}}}
Uuencode file \var{in_file} into file \var{out_file}.  The uuencoded
file will have the header specifying \var{name} and \var{mode} as the
defaults for the results of decoding the file. The default defaults
are taken from \var{in_file}, or \code{'-'} and \code{0666}
respectively. 
\end{funcdesc}

\begin{funcdesc}{decode}{in_file\optional{, out_file\optional{, mode}}}
This call decodes uuencoded file \var{in_file} placing the result on
file \var{out_file}. If \var{out_file} is a pathname the \var{mode} is
also set. Defaults for \var{out_file} and \var{mode} are taken from
the uuencode header.
\end{funcdesc}
