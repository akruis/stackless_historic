\section{\module{xml.sax} ---
         Support for SAX2 parsers}

\declaremodule{standard}{xml.sax}
\modulesynopsis{Package containing SAX2 base classes and convenience
                functions.}
\moduleauthor{Lars Marius Garshol}{larsga@garshol.priv.no}
\sectionauthor{Fred L. Drake, Jr.}{fdrake@acm.org}
\sectionauthor{Martin v. L\"owis}{loewis@informatik.hu-berlin.de}

\versionadded{2.0}


The \module{xml.sax} package provides a number of modules which
implement the Simple API for XML (SAX) interface for Python.  The
package itself provides the SAX exceptions and the convenience
functions which will be most used by users of the SAX API.

The convenience functions are:

\begin{funcdesc}{make_parser}{\optional{parser_list}}
  Create and return a SAX \class{XMLReader} object.  The first parser
  found will be used.  If \var{parser_list} is provided, it must be a
  sequence of strings which name modules that have a function named
  \function{create_parser()}.  Modules listed in \var{parser_list}
  will be used before modules in the default list of parsers.
\end{funcdesc}

\begin{funcdesc}{parse}{filename_or_stream, handler\optional{, error_handler}}
  Create a SAX parser and use it to parse a document.  The document,
  passed in as \var{filename_or_stream}, can be a filename or a file
  object.  The \var{handler} parameter needs to be a SAX
  \class{ContentHandler} instance.  If \var{error_handler} is given,
  it must be a SAX \class{ErrorHandler} instance; if omitted, 
  \exception{SAXParseException} will be raised on all errors.  There
  is no return value; all work must be done by the \var{handler}
  passed in.
\end{funcdesc}

\begin{funcdesc}{parseString}{string, handler\optional{, error_handler}}
  Similar to \function{parse()}, but parses from a buffer \var{string}
  received as a parameter.
\end{funcdesc}

A typical SAX application uses three kinds of objects: readers,
handlers and input sources.  ``Reader'' in this context is another
term for parser, i.e.\ some piece of code that reads the bytes or
characters from the input source, and produces a sequence of events.
The events then get distributed to the handler objects, i.e.\ the
reader invokes a method on the handler.  A SAX application must
therefore obtain a reader object, create or open the input sources,
create the handlers, and connect these objects all together.  As the
final step of preparation, the reader is called to parse the input.
During parsing, methods on the handler objects are called based on
structural and syntactic events from the input data.

For these objects, only the interfaces are relevant; they are normally
not instantiated by the application itself.  Since Python does not have
an explicit notion of interface, they are formally introduced as
classes, but applications may use implementations which do not inherit
from the provided classes.  The \class{InputSource}, \class{Locator},
\class{Attributes}, \class{AttributesNS}, and
\class{XMLReader} interfaces are defined in the module
\refmodule{xml.sax.xmlreader}.  The handler interfaces are defined in
\refmodule{xml.sax.handler}.  For convenience, \class{InputSource}
(which is often instantiated directly) and the handler classes are
also available from \module{xml.sax}.  These interfaces are described
below.

In addition to these classes, \module{xml.sax} provides the following
exception classes.

\begin{excclassdesc}{SAXException}{msg\optional{, exception}}
  Encapsulate an XML error or warning.  This class can contain basic
  error or warning information from either the XML parser or the
  application: it can be subclassed to provide additional
  functionality or to add localization.  Note that although the
  handlers defined in the \class{ErrorHandler} interface receive
  instances of this exception, it is not required to actually raise
  the exception --- it is also useful as a container for information.

  When instantiated, \var{msg} should be a human-readable description
  of the error.  The optional \var{exception} parameter, if given,
  should be \code{None} or an exception that was caught by the parsing
  code and is being passed along as information.

  This is the base class for the other SAX exception classes.
\end{excclassdesc}

\begin{excclassdesc}{SAXParseException}{msg, exception, locator}
  Subclass of \exception{SAXException} raised on parse errors.
  Instances of this class are passed to the methods of the SAX
  \class{ErrorHandler} interface to provide information about the
  parse error.  This class supports the SAX \class{Locator} interface
  as well as the \class{SAXException} interface.
\end{excclassdesc}

\begin{excclassdesc}{SAXNotRecognizedException}{msg\optional{, exception}}
  Subclass of \exception{SAXException} raised when a SAX
  \class{XMLReader} is confronted with an unrecognized feature or
  property.  SAX applications and extensions may use this class for
  similar purposes.
\end{excclassdesc}

\begin{excclassdesc}{SAXNotSupportedException}{msg\optional{, exception}}
  Subclass of \exception{SAXException} raised when a SAX
  \class{XMLReader} is asked to enable a feature that is not
  supported, or to set a property to a value that the implementation
  does not support.  SAX applications and extensions may use this
  class for similar purposes.
\end{excclassdesc}


\begin{seealso}
  \seetitle[http://www.saxproject.org/]{SAX: The Simple API for
            XML}{This site is the focal point for the definition of
            the SAX API.  It provides a Java implementation and online
            documentation.  Links to implementations and historical
            information are also available.}
\end{seealso}


\subsection{SAXException Objects \label{sax-exception-objects}}

The \class{SAXException} exception class supports the following
methods:

\begin{methoddesc}[SAXException]{getMessage}{}
  Return a human-readable message describing the error condition.
\end{methoddesc}

\begin{methoddesc}[SAXException]{getException}{}
  Return an encapsulated exception object, or \code{None}.
\end{methoddesc}
