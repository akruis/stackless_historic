\section{\module{zipfile} ---
         Work with ZIP archives}

\modulesynopsis{Read and write ZIP-format archive files.}
\moduleauthor{James C. Ahlstrom}{jim@interet.com}
\sectionauthor{James C. Ahlstrom}{jim@interet.com}
% LaTeX markup by Fred L. Drake, Jr. <fdrake@acm.org>

\versionadded{1.6}

The ZIP file format is a common archive and compression standard.
This module provides tools to create, read, write, append, and list a
ZIP file.

The available attributes of this module are:

\begin{excdesc}{error}
  The error raised for bad ZIP files.
\end{excdesc}

\begin{datadesc}{_debug}
  Level of printing, defaults to \code{1}.
\end{datadesc}

\begin{classdesc}{ZipFile}{...}
  The class for reading and writing ZIP files.  See
  ``\citetitle{ZipFile Objects}'' (section \ref{zipfile-objects}) for
  constructor details.
\end{classdesc}

\begin{funcdesc}{is_zipfile}{path} 
  Returns true if \var{path} is a valid ZIP file based on its magic
  number, otherwise returns false.  This module does not currently
  handle ZIP files which have appended comments.
\end{funcdesc}

\begin{funcdesc}{zip2date}{zdate}
  Return \code{(\var{year}, \var{month}, \var{day})} for a ZIP date
  code.
\end{funcdesc}

\begin{funcdesc}{zip2time}{ztime}
  Return \code{(\var{hour}, \var{minute}, \var{second})} for a ZIP
  time code.
\end{funcdesc}

\begin{funcdesc}{date2zip}{year, month, day}
  Return a ZIP date code. 
\end{funcdesc}

\begin{funcdesc}{time2zip}{hour, minute, second}
  Return a ZIP time code.
\end{funcdesc}

\begin{datadesc}{ZIP_STORED}
  The numeric constant (\code{0}) for an uncompressed archive member.
\end{datadesc}

\begin{datadesc}{ZIP_DEFLATED}
  The numeric constant for the usual ZIP compression method.  This
  requires the zlib module.  No other compression methods are
  currently supported.
\end{datadesc}


\begin{seealso}
  \seetext{XXX point to ZIP format definition}
  \seetext{XXX point to Info-ZIP home page; mention WiZ}
\end{seealso}


\subsection{ZipFile Objects \label{zipfile-objects}}

\begin{classdesc}{ZipFile}{filename\optional{, mode\optional{, compression}}} 
  Open a ZIP file named \var{filename}.  The \var{mode} parameter
  should be \code{'r'} to read an existing file, \code{'w'} to
  truncate and write a new file, or \code{'a'} to append to an
  existing file.  For \var{mode} is \code{'a'} and \var{filename}
  refers to an existing ZIP file, then additional files are added to
  it.  If \var{filename} does not refer to a ZIP file, then a new ZIP
  archive is appended to the file.  This is meant for adding a ZIP
  archive to another file, such as \file{python.exe}.  Using
\begin{verbatim}
cat myzip.zip >> python.exe
\end{verbatim}
  also works, and at least \program{WinZip} can read such files.
  \var{compression} is the ZIP compression method to use when writing
  the archive, and should be \constant{ZIP_STORED} or
  \constant{ZIP_DEFLATED}; unrecognized values will cause
  \exception{ValueError} to be raised.  The default is
  \constant{ZIP_STORED}. 
\end{classdesc}

XXX explain the "extra" string for the ZIP format

\begin{memberdesc}{TOC}
  A read-only dictionary whose keys are the names in the archive, and
  whose values are tuples as follows:

\begin{tableii}{c|l}{code}{Index}{Meaning}
  \lineii{0}{File data seek offset}
  \lineii{1}{ZIP file "extra" data as a string}
  \lineii{2}{ZIP file bit flags}
  \lineii{3}{ZIP file compression type}
  \lineii{4}{File modification time in DOS format}
  \lineii{5}{File modification date in DOS format}
  \lineii{6}{The CRC-32 of the uncompressed data}
  \lineii{7}{The compressed size of the file}
  \lineii{8}{The uncompressed size of the file}
\end{tableii}
\end{memberdesc}

The class ZipFile has these methods: 

\begin{methoddesc}{listdir}{}
  Return a list of names in the archive.  Equivalent to
  \code{\var{zipfile}.TOC.keys()}.
\end{methoddesc}

\begin{methoddesc}{printdir}{}
  Print a table of contents for the archive to stdout. 
\end{methoddesc}

\begin{methoddesc}{read}{name}
  Return the bytes of the file in the archive.  The archive must be
  open for read or append.
\end{methoddesc}

\begin{methoddesc}{writestr}{bytes, arcname, year, month, day, hour,
                             minute, second\optional{, extra}}
  Write the string \var{bytes} and the other data to the archive, and
  give the archive member the name \var{arcname}.  \var{extra} is the
  ZIP extra data string.  The archive must be opened with mode
  \code{'w'} or \code{'a'}.
\end{methoddesc}

\begin{methoddesc}{write}{filename, arcname\optional{, extra}}
  Write the file named \var{filename} to the archive, giving it the
  archive name \var{arcname}.  \var{extra} is the ZIP extra data
  string.  The archive must be open with mode \code{'w'} or
  \code{'a'}.
\end{methoddesc}

\begin{methoddesc}{writepy}{pathname\optional{, basename}}
  Search for files \file{*.py} and add the corresponding file to the
  archive.  The corresponding file is a \file{*.pyo} file if
  available, else a \file{*.pyc} file, compiling if necessary.  If the
  pathname is a file, the filename must end with \file{.py}, and just
  the (corresponding \file{*.py[oc]}) file is added at the top level
  (no path information).  If it is a directory, and the directory is
  not a package directory, then all the files \file{*.py[oc]} are
  added at the top level.  If the directory is a package directory,
  then all \file{*.py[oc]} are added under the package name as a file
  path, and if any subdirectories are package directories, all of
  these are added recursively.  \var{basename} is intended for
  internal use only.  The \method{writepy()} method makes archives
  with file names like this:

\begin{verbatim}
    string.pyc                                # Top level name 
    test/__init__.pyc                         # Package directory 
    test/testall.pyc                          # Module test.testall
    test/bogus/__init__.pyc                   # Subpackage directory 
    test/bogus/myfile.pyc                     # Submodule test.bogus.myfile
\end{verbatim}
\end{methoddesc}

\begin{methoddesc}{close}{}
  Close the archive file.  You must call \method{close()} before
  exiting your program or essential records will not be written. 
\end{methoddesc}
