\section{\module{MimeWriter} ---
         Generic MIME file writer}

\declaremodule{standard}{MimeWriter}

\modulesynopsis{Generic MIME file writer.}
\sectionauthor{Christopher G. Petrilli}{petrilli@amber.org}

\deprecated{2.3}{The \refmodule{email} package should be used in
                 preference to the \module{MimeWriter} module.  This
                 module is present only to maintain backward
                 compatibility.}

This module defines the class \class{MimeWriter}.  The
\class{MimeWriter} class implements a basic formatter for creating
MIME multi-part files.  It doesn't seek around the output file nor
does it use large amounts of buffer space. You must write the parts
out in the order that they should occur in the final
file. \class{MimeWriter} does buffer the headers you add, allowing you 
to rearrange their order.

\begin{classdesc}{MimeWriter}{fp}
Return a new instance of the \class{MimeWriter} class.  The only
argument passed, \var{fp}, is a file object to be used for
writing. Note that a \class{StringIO} object could also be used.
\end{classdesc}


\subsection{MimeWriter Objects \label{MimeWriter-objects}}


\class{MimeWriter} instances have the following methods:

\begin{methoddesc}[MimeWriter]{addheader}{key, value\optional{, prefix}}
Add a header line to the MIME message. The \var{key} is the name of
the header, where the \var{value} obviously provides the value of the
header. The optional argument \var{prefix} determines where the header 
is inserted; \samp{0} means append at the end, \samp{1} is insert at
the start. The default is to append.
\end{methoddesc}

\begin{methoddesc}[MimeWriter]{flushheaders}{}
Causes all headers accumulated so far to be written out (and
forgotten). This is useful if you don't need a body part at all,
e.g.\ for a subpart of type \mimetype{message/rfc822} that's (mis)used
to store some header-like information.
\end{methoddesc}

\begin{methoddesc}[MimeWriter]{startbody}{ctype\optional{, plist\optional{, prefix}}}
Returns a file-like object which can be used to write to the
body of the message.  The content-type is set to the provided
\var{ctype}, and the optional parameter \var{plist} provides
additional parameters for the content-type declaration. \var{prefix}
functions as in \method{addheader()} except that the default is to
insert at the start.
\end{methoddesc}

\begin{methoddesc}[MimeWriter]{startmultipartbody}{subtype\optional{,
                               boundary\optional{, plist\optional{, prefix}}}}
Returns a file-like object which can be used to write to the
body of the message.  Additionally, this method initializes the
multi-part code, where \var{subtype} provides the multipart subtype,
\var{boundary} may provide a user-defined boundary specification, and
\var{plist} provides optional parameters for the subtype.
\var{prefix} functions as in \method{startbody()}.  Subparts should be
created using \method{nextpart()}.
\end{methoddesc}

\begin{methoddesc}[MimeWriter]{nextpart}{}
Returns a new instance of \class{MimeWriter} which represents an
individual part in a multipart message.  This may be used to write the 
part as well as used for creating recursively complex multipart
messages. The message must first be initialized with
\method{startmultipartbody()} before using \method{nextpart()}.
\end{methoddesc}

\begin{methoddesc}[MimeWriter]{lastpart}{}
This is used to designate the last part of a multipart message, and
should \emph{always} be used when writing multipart messages.
\end{methoddesc}
