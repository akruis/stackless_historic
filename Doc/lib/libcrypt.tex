\section{\module{crypt} ---
         Function used to check \UNIX{} passwords}

\declaremodule{builtin}{crypt}
  \platform{Unix}
\modulesynopsis{The \cfunction{crypt()} function used to check \UNIX{}
  passwords.}
\moduleauthor{Steven D. Majewski}{sdm7g@virginia.edu}
\sectionauthor{Steven D. Majewski}{sdm7g@virginia.edu}


This module implements an interface to the \manpage{crypt}{3} routine,
which is a one-way hash function based upon a modified DES algorithm;
see the \UNIX{} man page for further details.  Possible uses include
allowing Python scripts to accept typed passwords from the user, or
attempting to crack \UNIX{} passwords with a dictionary.
\index{crypt(3)}

\begin{funcdesc}{crypt}{word, salt} 
\var{word} will usually be a user's password.  \var{salt} is a
2-character string which will be used to select one of 4096 variations
of DES\indexii{cipher}{DES}.  The characters in \var{salt} must be
either \character{.}, \character{/}, or an alphanumeric character.
Returns the hashed password as a string, which will be composed of
characters from the same alphabet as the salt.
\end{funcdesc}

The module and documentation were written by Steve Majewski.
\index{Majewski, Steve}
