\section{Built-in Types}

The following sections describe the standard types that are built into
the interpreter.  These are the numeric types, sequence types, and
several others, including types themselves.  There is no explicit
Boolean type; use integers instead.
\indexii{built-in}{types}
\indexii{Boolean}{type}

Some operations are supported by several object types; in particular,
all objects can be compared, tested for truth value, and converted to
a string (with the \code{`{\rm \ldots}`} notation).  The latter conversion is
implicitly used when an object is written by the \code{print} statement.
\stindex{print}

\subsection{Truth Value Testing}

Any object can be tested for truth value, for use in an \code{if} or
\code{while} condition or as operand of the Boolean operations below.
The following values are considered false:
\stindex{if}
\stindex{while}
\indexii{truth}{value}
\indexii{Boolean}{operations}
\index{false}

\begin{itemize}
\renewcommand{\indexsubitem}{(Built-in object)}

\item	\code{None}
	\ttindex{None}

\item	zero of any numeric type, e.g., \code{0}, \code{0L}, \code{0.0}.

\item	any empty sequence, e.g., \code{''}, \code{()}, \code{[]}.

\item	any empty mapping, e.g., \code{\{\}}.

\item	instances of user-defined classes, if the class defines a
	\code{__nonzero__()} or \code{__len__()} method, when that
	method returns zero.

\end{itemize}

All other values are considered true --- so objects of many types are
always true.
\index{true}

Operations and built-in functions that have a Boolean result always
return \code{0} for false and \code{1} for true, unless otherwise
stated.  (Important exception: the Boolean operations \samp{or} and
\samp{and} always return one of their operands.)

\subsection{Boolean Operations}

These are the Boolean operations, ordered by ascending priority:
\indexii{Boolean}{operations}

\begin{tableiii}{|c|l|c|}{code}{Operation}{Result}{Notes}
  \lineiii{\var{x} or \var{y}}{if \var{x} is false, then \var{y}, else \var{x}}{(1)}
  \hline
  \lineiii{\var{x} and \var{y}}{if \var{x} is false, then \var{x}, else \var{y}}{(1)}
  \hline
  \lineiii{not \var{x}}{if \var{x} is false, then \code{1}, else \code{0}}{(2)}
\end{tableiii}
\opindex{and}
\opindex{or}
\opindex{not}

\noindent
Notes:

\begin{description}

\item[(1)]
These only evaluate their second argument if needed for their outcome.

\item[(2)]
\samp{not} has a lower priority than non-Boolean operators, so e.g.
\code{not a == b} is interpreted as \code{not(a == b)}, and
\code{a == not b} is a syntax error.

\end{description}

\subsection{Comparisons}

Comparison operations are supported by all objects.  They all have the
same priority (which is higher than that of the Boolean operations).
Comparisons can be chained arbitrarily, e.g. \code{x < y <= z} is
equivalent to \code{x < y and y <= z}, except that \code{y} is
evaluated only once (but in both cases \code{z} is not evaluated at
all when \code{x < y} is found to be false).
\indexii{chaining}{comparisons}

This table summarizes the comparison operations:

\begin{tableiii}{|c|l|c|}{code}{Operation}{Meaning}{Notes}
  \lineiii{<}{strictly less than}{}
  \lineiii{<=}{less than or equal}{}
  \lineiii{>}{strictly greater than}{}
  \lineiii{>=}{greater than or equal}{}
  \lineiii{==}{equal}{}
  \lineiii{<>}{not equal}{(1)}
  \lineiii{!=}{not equal}{(1)}
  \lineiii{is}{object identity}{}
  \lineiii{is not}{negated object identity}{}
\end{tableiii}
\indexii{operator}{comparison}
\opindex{==} % XXX *All* others have funny characters < ! >
\opindex{is}
\opindex{is not}

\noindent
Notes:

\begin{description}

\item[(1)]
\code{<>} and \code{!=} are alternate spellings for the same operator.
(I couldn't choose between \ABC{} and \C{}! :-)
\indexii{\ABC{}}{language}
\indexii{\C{}}{language}

\end{description}

Objects of different types, except different numeric types, never
compare equal; such objects are ordered consistently but arbitrarily
(so that sorting a heterogeneous array yields a consistent result).
Furthermore, some types (e.g., windows) support only a degenerate
notion of comparison where any two objects of that type are unequal.
Again, such objects are ordered arbitrarily but consistently.
\indexii{types}{numeric}
\indexii{objects}{comparing}

(Implementation note: objects of different types except numbers are
ordered by their type names; objects of the same types that don't
support proper comparison are ordered by their address.)

Two more operations with the same syntactic priority, \code{in} and
\code{not in}, are supported only by sequence types (below).
\opindex{in}
\opindex{not in}

\subsection{Numeric Types}

There are three numeric types: \dfn{plain integers}, \dfn{long integers}, and
\dfn{floating point numbers}.  Plain integers (also just called \dfn{integers})
are implemented using \code{long} in \C{}, which gives them at least 32
bits of precision.  Long integers have unlimited precision.  Floating
point numbers are implemented using \code{double} in \C{}.  All bets on
their precision are off unless you happen to know the machine you are
working with.
\indexii{numeric}{types}
\indexii{integer}{types}
\indexii{integer}{type}
\indexiii{long}{integer}{type}
\indexii{floating point}{type}
\indexii{\C{}}{language}

Numbers are created by numeric literals or as the result of built-in
functions and operators.  Unadorned integer literals (including hex
and octal numbers) yield plain integers.  Integer literals with an \samp{L}
or \samp{l} suffix yield long integers
(\samp{L} is preferred because \code{1l} looks too much like eleven!).
Numeric literals containing a decimal point or an exponent sign yield
floating point numbers.
\indexii{numeric}{literals}
\indexii{integer}{literals}
\indexiii{long}{integer}{literals}
\indexii{floating point}{literals}
\indexii{hexadecimal}{literals}
\indexii{octal}{literals}

Python fully supports mixed arithmetic: when a binary arithmetic
operator has operands of different numeric types, the operand with the
``smaller'' type is converted to that of the other, where plain
integer is smaller than long integer is smaller than floating point.
Comparisons between numbers of mixed type use the same rule.%
\footnote{As a consequence, the list \code{[1, 2]} is considered equal
	to \code{[1.0, 2.0]}, and similar for tuples.}
The functions \code{int()}, \code{long()} and \code{float()} can be used
to coerce numbers to a specific type.
\index{arithmetic}
\bifuncindex{int}
\bifuncindex{long}
\bifuncindex{float}

All numeric types support the following operations, sorted by
ascending priority (operations in the same box have the same
priority; all numeric operations have a higher priority than
comparison operations):

\begin{tableiii}{|c|l|c|}{code}{Operation}{Result}{Notes}
  \lineiii{\var{x} + \var{y}}{sum of \var{x} and \var{y}}{}
  \lineiii{\var{x} - \var{y}}{difference of \var{x} and \var{y}}{}
  \hline
  \lineiii{\var{x} * \var{y}}{product of \var{x} and \var{y}}{}
  \lineiii{\var{x} / \var{y}}{quotient of \var{x} and \var{y}}{(1)}
  \lineiii{\var{x} \%{} \var{y}}{remainder of \code{\var{x} / \var{y}}}{}
  \hline
  \lineiii{-\var{x}}{\var{x} negated}{}
  \lineiii{+\var{x}}{\var{x} unchanged}{}
  \hline
  \lineiii{abs(\var{x})}{absolute value of \var{x}}{}
  \lineiii{int(\var{x})}{\var{x} converted to integer}{(2)}
  \lineiii{long(\var{x})}{\var{x} converted to long integer}{(2)}
  \lineiii{float(\var{x})}{\var{x} converted to floating point}{}
  \lineiii{divmod(\var{x}, \var{y})}{the pair \code{(\var{x} / \var{y}, \var{x} \%{} \var{y})}}{(3)}
  \lineiii{pow(\var{x}, \var{y})}{\var{x} to the power \var{y}}{}
\end{tableiii}
\indexiii{operations on}{numeric}{types}

\noindent
Notes:
\begin{description}

\item[(1)]
For (plain or long) integer division, the result is an integer.
The result is always rounded towards minus infinity: 1/2 is 0, 
(-1)/2 is -1, 1/(-2) is -1, and (-1)/(-2) is 0.
\indexii{integer}{division}
\indexiii{long}{integer}{division}

\item[(2)]
Conversion from floating point to (long or plain) integer may round or
truncate as in \C{}; see functions \code{floor()} and \code{ceil()} in
module \code{math} for well-defined conversions.
\bifuncindex{floor}
\bifuncindex{ceil}
\indexii{numeric}{conversions}
\stmodindex{math}
\indexii{\C{}}{language}

\item[(3)]
See the section on built-in functions for an exact definition.

\end{description}
% XXXJH exceptions: overflow (when? what operations?) zerodivision

\subsubsection{Bit-string Operations on Integer Types}
\nodename{Bit-string Operations}

Plain and long integer types support additional operations that make
sense only for bit-strings.  Negative numbers are treated as their 2's
complement value (for long integers, this assumes a sufficiently large
number of bits that no overflow occurs during the operation).

The priorities of the binary bit-wise operations are all lower than
the numeric operations and higher than the comparisons; the unary
operation \samp{~} has the same priority as the other unary numeric
operations (\samp{+} and \samp{-}).

This table lists the bit-string operations sorted in ascending
priority (operations in the same box have the same priority):

\begin{tableiii}{|c|l|c|}{code}{Operation}{Result}{Notes}
  \lineiii{\var{x} | \var{y}}{bitwise \dfn{or} of \var{x} and \var{y}}{}
  \hline
  \lineiii{\var{x} \^{} \var{y}}{bitwise \dfn{exclusive or} of \var{x} and \var{y}}{}
  \hline
  \lineiii{\var{x} \&{} \var{y}}{bitwise \dfn{and} of \var{x} and \var{y}}{}
  \hline
  \lineiii{\var{x} << \var{n}}{\var{x} shifted left by \var{n} bits}{(1), (2)}
  \lineiii{\var{x} >> \var{n}}{\var{x} shifted right by \var{n} bits}{(1), (3)}
  \hline
  \hline
  \lineiii{\~\var{x}}{the bits of \var{x} inverted}{}
\end{tableiii}
\indexiii{operations on}{integer}{types}
\indexii{bit-string}{operations}
\indexii{shifting}{operations}
\indexii{masking}{operations}

\noindent
Notes:
\begin{description}
\item[(1)] Negative shift counts are illegal.
\item[(2)] A left shift by \var{n} bits is equivalent to
multiplication by \code{pow(2, \var{n})} without overflow check.
\item[(3)] A right shift by \var{n} bits is equivalent to
division by \code{pow(2, \var{n})} without overflow check.
\end{description}

\subsection{Sequence Types}

There are three sequence types: strings, lists and tuples.

Strings literals are written in single or double quotes:
\code{'xyzzy'}, \code{"frobozz"}.  See Chapter 2 of the Python
Reference Manual for more about string literals.  Lists are
constructed with square brackets, separating items with commas:
\code{[a, b, c]}.  Tuples are constructed by the comma operator (not
within square brackets), with or without enclosing parentheses, but an
empty tuple must have the enclosing parentheses, e.g.,
\code{a, b, c} or \code{()}.  A single item tuple must have a trailing
comma, e.g., \code{(d,)}.
\indexii{sequence}{types}
\indexii{string}{type}
\indexii{tuple}{type}
\indexii{list}{type}

Sequence types support the following operations.  The \samp{in} and
\samp{not\,in} operations have the same priorities as the comparison
operations.  The \samp{+} and \samp{*} operations have the same
priority as the corresponding numeric operations.\footnote{They must
have since the parser can't tell the type of the operands.}

This table lists the sequence operations sorted in ascending priority
(operations in the same box have the same priority).  In the table,
\var{s} and \var{t} are sequences of the same type; \var{n}, \var{i}
and \var{j} are integers:

\begin{tableiii}{|c|l|c|}{code}{Operation}{Result}{Notes}
  \lineiii{\var{x} in \var{s}}{\code{1} if an item of \var{s} is equal to \var{x}, else \code{0}}{}
  \lineiii{\var{x} not in \var{s}}{\code{0} if an item of \var{s} is
equal to \var{x}, else \code{1}}{}
  \hline
  \lineiii{\var{s} + \var{t}}{the concatenation of \var{s} and \var{t}}{}
  \hline
  \lineiii{\var{s} * \var{n}{\rm ,} \var{n} * \var{s}}{\var{n} copies of \var{s} concatenated}{}
  \hline
  \lineiii{\var{s}[\var{i}]}{\var{i}'th item of \var{s}, origin 0}{(1)}
  \lineiii{\var{s}[\var{i}:\var{j}]}{slice of \var{s} from \var{i} to \var{j}}{(1), (2)}
  \hline
  \lineiii{len(\var{s})}{length of \var{s}}{}
  \lineiii{min(\var{s})}{smallest item of \var{s}}{}
  \lineiii{max(\var{s})}{largest item of \var{s}}{}
\end{tableiii}
\indexiii{operations on}{sequence}{types}
\bifuncindex{len}
\bifuncindex{min}
\bifuncindex{max}
\indexii{concatenation}{operation}
\indexii{repetition}{operation}
\indexii{subscript}{operation}
\indexii{slice}{operation}
\opindex{in}
\opindex{not in}

\noindent
Notes:

\begin{description}
  
\item[(1)] If \var{i} or \var{j} is negative, the index is relative to
  the end of the string, i.e., \code{len(\var{s}) + \var{i}} or
  \code{len(\var{s}) + \var{j}} is substituted.  But note that \code{-0} is
  still \code{0}.
  
\item[(2)] The slice of \var{s} from \var{i} to \var{j} is defined as
  the sequence of items with index \var{k} such that \code{\var{i} <=
  \var{k} < \var{j}}.  If \var{i} or \var{j} is greater than
  \code{len(\var{s})}, use \code{len(\var{s})}.  If \var{i} is omitted,
  use \code{0}.  If \var{j} is omitted, use \code{len(\var{s})}.  If
  \var{i} is greater than or equal to \var{j}, the slice is empty.

\end{description}

\subsubsection{More String Operations}

String objects have one unique built-in operation: the \code{\%}
operator (modulo) with a string left argument interprets this string
as a C sprintf format string to be applied to the right argument, and
returns the string resulting from this formatting operation.

The right argument should be a tuple with one item for each argument
required by the format string; if the string requires a single
argument, the right argument may also be a single non-tuple object.%
\footnote{A tuple object in this case should be a singleton.}
The following format characters are understood:
\%, c, s, i, d, u, o, x, X, e, E, f, g, G.
Width and precision may be a * to specify that an integer argument
specifies the actual width or precision.  The flag characters -, +,
blank, \# and 0 are understood.  The size specifiers h, l or L may be
present but are ignored.  The \code{\%s} conversion takes any Python
object and converts it to a string using \code{str()} before
formatting it.  The ANSI features \code{\%p} and \code{\%n}
are not supported.  Since Python strings have an explicit length,
\code{\%s} conversions don't assume that \code{'\e0'} is the end of
the string.

For safety reasons, floating point precisions are clipped to 50;
\code{\%f} conversions for numbers whose absolute value is over 1e25
are replaced by \code{\%g} conversions.%
\footnote{These numbers are fairly arbitrary.  They are intended to
avoid printing endless strings of meaningless digits without hampering
correct use and without having to know the exact precision of floating
point values on a particular machine.}
All other errors raise exceptions.

If the right argument is a dictionary (or any kind of mapping), then
the formats in the string must have a parenthesized key into that
dictionary inserted immediately after the \code{\%} character, and
each format formats the corresponding entry from the mapping.  E.g.
\begin{verbatim}
    >>> count = 2
    >>> language = 'Python'
    >>> print '%(language)s has %(count)03d quote types.' % vars()
    Python has 002 quote types.
    >>> 
\end{verbatim}
In this case no * specifiers may occur in a format (since they
require a sequential parameter list).

Additional string operations are defined in standard module
\code{string} and in built-in module \code{regex}.
\index{string}
\index{regex}

\subsubsection{Mutable Sequence Types}

List objects support additional operations that allow in-place
modification of the object.
These operations would be supported by other mutable sequence types
(when added to the language) as well.
Strings and tuples are immutable sequence types and such objects cannot
be modified once created.
The following operations are defined on mutable sequence types (where
\var{x} is an arbitrary object):
\indexiii{mutable}{sequence}{types}
\indexii{list}{type}

\begin{tableiii}{|c|l|c|}{code}{Operation}{Result}{Notes}
  \lineiii{\var{s}[\var{i}] = \var{x}}
	{item \var{i} of \var{s} is replaced by \var{x}}{}
  \lineiii{\var{s}[\var{i}:\var{j}] = \var{t}}
  	{slice of \var{s} from \var{i} to \var{j} is replaced by \var{t}}{}
  \lineiii{del \var{s}[\var{i}:\var{j}]}
	{same as \code{\var{s}[\var{i}:\var{j}] = []}}{}
  \lineiii{\var{s}.append(\var{x})}
	{same as \code{\var{s}[len(\var{s}):len(\var{s})] = [\var{x}]}}{}
  \lineiii{\var{s}.count(\var{x})}
	{return number of \var{i}'s for which \code{\var{s}[\var{i}] == \var{x}}}{}
  \lineiii{\var{s}.index(\var{x})}
	{return smallest \var{i} such that \code{\var{s}[\var{i}] == \var{x}}}{(1)}
  \lineiii{\var{s}.insert(\var{i}, \var{x})}
	{same as \code{\var{s}[\var{i}:\var{i}] = [\var{x}]}
	  if \code{\var{i} >= 0}}{}
  \lineiii{\var{s}.remove(\var{x})}
	{same as \code{del \var{s}[\var{s}.index(\var{x})]}}{(1)}
  \lineiii{\var{s}.reverse()}
	{reverses the items of \var{s} in place}{}
  \lineiii{\var{s}.sort()}
	{permutes the items of \var{s} to satisfy
        \code{\var{s}[\var{i}] <= \var{s}[\var{j}]},
        for \code{\var{i} < \var{j}}}{(2)}
\end{tableiii}
\indexiv{operations on}{mutable}{sequence}{types}
\indexiii{operations on}{sequence}{types}
\indexiii{operations on}{list}{type}
\indexii{subscript}{assignment}
\indexii{slice}{assignment}
\stindex{del}
\renewcommand{\indexsubitem}{(list method)}
\ttindex{append}
\ttindex{count}
\ttindex{index}
\ttindex{insert}
\ttindex{remove}
\ttindex{reverse}
\ttindex{sort}

\noindent
Notes:
\begin{description}
\item[(1)] Raises an exception when \var{x} is not found in \var{s}.
  
\item[(2)] The \code{sort()} method takes an optional argument
  specifying a comparison function of two arguments (list items) which
  should return \code{-1}, \code{0} or \code{1} depending on whether the
  first argument is considered smaller than, equal to, or larger than the
  second argument.  Note that this slows the sorting process down
  considerably; e.g. to sort a list in reverse order it is much faster
  to use calls to \code{sort()} and \code{reverse()} than to use
  \code{sort()} with a comparison function that reverses the ordering of
  the elements.
\end{description}

\subsection{Mapping Types}

A \dfn{mapping} object maps values of one type (the key type) to
arbitrary objects.  Mappings are mutable objects.  There is currently
only one standard mapping type, the \dfn{dictionary}.  A dictionary's keys are
almost arbitrary values.  The only types of values not acceptable as
keys are values containing lists or dictionaries or other mutable
types that are compared by value rather than by object identity.
Numeric types used for keys obey the normal rules for numeric
comparison: if two numbers compare equal (e.g. 1 and 1.0) then they
can be used interchangeably to index the same dictionary entry.

\indexii{mapping}{types}
\indexii{dictionary}{type}

Dictionaries are created by placing a comma-separated list of
\code{\var{key}:\,\var{value}} pairs within braces, for example:
\code{\{'jack':\,4098, 'sjoerd':\,4127\}} or
\code{\{4098:\,'jack', 4127:\,'sjoerd'\}}.

The following operations are defined on mappings (where \var{a} is a
mapping, \var{k} is a key and \var{x} is an arbitrary object):

\begin{tableiii}{|c|l|c|}{code}{Operation}{Result}{Notes}
  \lineiii{len(\var{a})}{the number of items in \var{a}}{}
  \lineiii{\var{a}[\var{k}]}{the item of \var{a} with key \var{k}}{(1)}
  \lineiii{\var{a}[\var{k}] = \var{x}}{set \code{\var{a}[\var{k}]} to \var{x}}{}
  \lineiii{del \var{a}[\var{k}]}{remove \code{\var{a}[\var{k}]} from \var{a}}{(1)}
  \lineiii{\var{a}.items()}{a copy of \var{a}'s list of (key, item) pairs}{(2)}
  \lineiii{\var{a}.keys()}{a copy of \var{a}'s list of keys}{(2)}
  \lineiii{\var{a}.values()}{a copy of \var{a}'s list of values}{(2)}
  \lineiii{\var{a}.has_key(\var{k})}{\code{1} if \var{a} has a key \var{k}, else \code{0}}{}
\end{tableiii}
\indexiii{operations on}{mapping}{types}
\indexiii{operations on}{dictionary}{type}
\stindex{del}
\bifuncindex{len}
\renewcommand{\indexsubitem}{(dictionary method)}
\ttindex{keys}
\ttindex{has_key}

\noindent
Notes:
\begin{description}
\item[(1)] Raises an exception if \var{k} is not in the map.

\item[(2)] Keys and values are listed in random order.
\end{description}

\subsection{Other Built-in Types}

The interpreter supports several other kinds of objects.
Most of these support only one or two operations.

\subsubsection{Modules}

The only special operation on a module is attribute access:
\code{\var{m}.\var{name}}, where \var{m} is a module and \var{name} accesses
a name defined in \var{m}'s symbol table.  Module attributes can be
assigned to.  (Note that the \code{import} statement is not, strictly
spoken, an operation on a module object; \code{import \var{foo}} does not
require a module object named \var{foo} to exist, rather it requires
an (external) \emph{definition} for a module named \var{foo}
somewhere.)

A special member of every module is \code{__dict__}.
This is the dictionary containing the module's symbol table.
Modifying this dictionary will actually change the module's symbol
table, but direct assignment to the \code{__dict__} attribute is not
possible (i.e., you can write \code{\var{m}.__dict__['a'] = 1}, which
defines \code{\var{m}.a} to be \code{1}, but you can't write \code{\var{m}.__dict__ = \{\}}.

Modules are written like this: \code{<module 'sys'>}.

\subsubsection{Classes and Class Instances}
\nodename{Classes and Instances}

(See Chapters 3 and 7 of the Python Reference Manual for these.)

\subsubsection{Functions}

Function objects are created by function definitions.  The only
operation on a function object is to call it:
\code{\var{func}(\var{argument-list})}.

There are really two flavors of function objects: built-in functions
and user-defined functions.  Both support the same operation (to call
the function), but the implementation is different, hence the
different object types.

The implementation adds two special read-only attributes:
\code{\var{f}.func_code} is a function's \dfn{code object} (see below) and
\code{\var{f}.func_globals} is the dictionary used as the function's
global name space (this is the same as \code{\var{m}.__dict__} where
\var{m} is the module in which the function \var{f} was defined).

\subsubsection{Methods}
\obindex{method}

Methods are functions that are called using the attribute notation.
There are two flavors: built-in methods (such as \code{append()} on
lists) and class instance methods.  Built-in methods are described
with the types that support them.

The implementation adds two special read-only attributes to class
instance methods: \code{\var{m}.im_self} is the object whose method this
is, and \code{\var{m}.im_func} is the function implementing the method.
Calling \code{\var{m}(\var{arg-1}, \var{arg-2}, {\rm \ldots},
\var{arg-n})} is completely equivalent to calling
\code{\var{m}.im_func(\var{m}.im_self, \var{arg-1}, \var{arg-2}, {\rm
\ldots}, \var{arg-n})}.

(See the Python Reference Manual for more info.)

\subsubsection{Code Objects}
\obindex{code}

Code objects are used by the implementation to represent
``pseudo-compiled'' executable Python code such as a function body.
They differ from function objects because they don't contain a
reference to their global execution environment.  Code objects are
returned by the built-in \code{compile()} function and can be
extracted from function objects through their \code{func_code}
attribute.
\bifuncindex{compile}
\ttindex{func_code}

A code object can be executed or evaluated by passing it (instead of a
source string) to the \code{exec} statement or the built-in
\code{eval()} function.
\stindex{exec}
\bifuncindex{eval}

(See the Python Reference Manual for more info.)

\subsubsection{Type Objects}

Type objects represent the various object types.  An object's type is
accessed by the built-in function \code{type()}.  There are no special
operations on types.  The standard module \code{types} defines names
for all standard built-in types.
\bifuncindex{type}
\stmodindex{types}

Types are written like this: \code{<type 'int'>}.

\subsubsection{The Null Object}

This object is returned by functions that don't explicitly return a
value.  It supports no special operations.  There is exactly one null
object, named \code{None} (a built-in name).

It is written as \code{None}.

\subsubsection{File Objects}

File objects are implemented using \C{}'s \code{stdio} package and can be
created with the built-in function \code{open()} described under
Built-in Functions below.  They are also returned by some other
built-in functions and methods, e.g.\ \code{posix.popen()} and
\code{posix.fdopen()} and the \code{makefile()} method of socket
objects.
\bifuncindex{open}

When a file operation fails for an I/O-related reason, the exception
\code{IOError} is raised.  This includes situations where the
operation is not defined for some reason, like \code{seek()} on a tty
device or writing a file opened for reading.

Files have the following methods:


\renewcommand{\indexsubitem}{(file method)}

\begin{funcdesc}{close}{}
  Close the file.  A closed file cannot be read or written anymore.
\end{funcdesc}

\begin{funcdesc}{flush}{}
  Flush the internal buffer, like \code{stdio}'s \code{fflush()}.
\end{funcdesc}

\begin{funcdesc}{isatty}{}
  Return \code{1} if the file is connected to a tty(-like) device, else
  \code{0}.
\end{funcdesc}

\begin{funcdesc}{read}{\optional{size}}
  Read at most \var{size} bytes from the file (less if the read hits
  \EOF{} or no more data is immediately available on a pipe, tty or
  similar device).  If the \var{size} argument is negative or omitted,
  read all data until \EOF{} is reached.  The bytes are returned as a string
  object.  An empty string is returned when \EOF{} is encountered
  immediately.  (For certain files, like ttys, it makes sense to
  continue reading after an \EOF{} is hit.)
\end{funcdesc}

\begin{funcdesc}{readline}{\optional{size}}
  Read one entire line from the file.  A trailing newline character is
  kept in the string%
\footnote{The advantage of leaving the newline on is that an empty string 
	can be returned to mean \EOF{} without being ambiguous.  Another 
	advantage is that (in cases where it might matter, e.g. if you 
	want to make an exact copy of a file while scanning its lines) 
	you can tell whether the last line of a file ended in a newline
	or not (yes this happens!).}
  (but may be absent when a file ends with an
  incomplete line).  If the \var{size} argument is present and
  non-negative, it is a maximum byte count (including the trailing
  newline) and an incomplete line may be returned.
  An empty string is returned when \EOF{} is hit
  immediately.  Note: unlike \code{stdio}'s \code{fgets()}, the returned
  string contains null characters (\code{'\e 0'}) if they occurred in the
  input.
\end{funcdesc}

\begin{funcdesc}{readlines}{}
  Read until \EOF{} using \code{readline()} and return a list containing
  the lines thus read.
\end{funcdesc}

\begin{funcdesc}{seek}{offset\, whence}
  Set the file's current position, like \code{stdio}'s \code{fseek()}.
  The \var{whence} argument is optional and defaults to \code{0}
  (absolute file positioning); other values are \code{1} (seek
  relative to the current position) and \code{2} (seek relative to the
  file's end).  There is no return value.
\end{funcdesc}

\begin{funcdesc}{tell}{}
  Return the file's current position, like \code{stdio}'s \code{ftell()}.
\end{funcdesc}

\begin{funcdesc}{truncate}{\optional{size}}
Truncate the file's size.  If the optional size argument present, the
file is truncated to (at most) that size.  The size defaults to the
current position.  Availability of this function depends on the
operating system version (e.g., not all \UNIX{} versions support this
operation).
\end{funcdesc}

\begin{funcdesc}{write}{str}
Write a string to the file.  There is no return value.  Note: due to
buffering, the string may not actually show up in the file until
the \code{flush()} or \code{close()} method is called.
\end{funcdesc}

\begin{funcdesc}{writelines}{list}
Write a list of strings to the file.  There is no return value.
(The name is intended to match \code{readlines}; \code{writelines}
does not add line separators.)
\end{funcdesc}

\subsubsection{Internal Objects}

(See the Python Reference Manual for these.)

\subsection{Special Attributes}

The implementation adds a few special read-only attributes to several
object types, where they are relevant:

\begin{itemize}

\item
\code{\var{x}.__dict__} is a dictionary of some sort used to store an
object's (writable) attributes;

\item
\code{\var{x}.__methods__} lists the methods of many built-in object types,
e.g., \code{[].__methods__} yields
\code{['append', 'count', 'index', 'insert', 'remove', 'reverse', 'sort']};

\item
\code{\var{x}.__members__} lists data attributes;

\item
\code{\var{x}.__class__} is the class to which a class instance belongs;

\item
\code{\var{x}.__bases__} is the tuple of base classes of a class object.

\end{itemize}
