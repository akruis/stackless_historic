% This document is largely a stub used to allow the email package docs
% to be formatted separately from the rest of the Python
% documentation.  This allows the documentation to be released
% independently of the rest of Python since the email package is being
% maintained for multiple Python versions, and on an accelerated
% schedule.

\documentclass{howto}

\title{email Package Reference}
\author{Barry Warsaw}
\authoraddress{\email{barry@python.org}}

\date{\today}
\release{4.0}			% software release, not documentation
\setreleaseinfo{}		% empty for final release
\setshortversion{4.0}		% major.minor only for software

\begin{document}

\maketitle

\begin{abstract}
  The \module{email} package provides classes and utilities to create,
  parse, generate, and modify email messages, conforming to all the
  relevant email and MIME related RFCs.
\end{abstract}

% The ugly "%begin{latexonly}" pseudo-environment suppresses the table
% of contents for HTML generation.
%
%begin{latexonly}
\tableofcontents
%end{latexonly}

\section{Introduction}
The \module{email} package provides classes and utilities to create,
parse, generate, and modify email messages, conforming to all the
relevant email and MIME related RFCs.

This document describes version 4.0 of the \module{email} package, which is
distributed with Python 2.5 and is available as a standalone distutils-based
package for use with earlier Python versions.  \module{email} 4.0 is not
compatible with Python versions earlier than 2.3.  For more information about
the \module{email} package, including download links and mailing lists, see
\ulink{Python's email SIG}{http://www.python.org/sigs/email-sig}.

The documentation that follows was written for the Python project, so
if you're reading this as part of the standalone \module{email}
package documentation, there are a few notes to be aware of:

\begin{itemize}
\item Deprecation and ``version added'' notes are relative to the
      Python version a feature was added or deprecated.  See
      the package history in section \ref{email-pkg-history} for details.

\item If you're reading this documentation as part of the
      standalone \module{email} package, some of the internal links to
      other sections of the Python standard library may not resolve.

\end{itemize}

% Copyright (C) 2001,2002 Python Software Foundation
% Author: barry@zope.com (Barry Warsaw)

\section{\module{email} ---
	 An email and MIME handling package}

\declaremodule{standard}{email}
\modulesynopsis{Package supporting the parsing, manipulating, and
    generating email messages, including MIME documents.}
\moduleauthor{Barry A. Warsaw}{barry@zope.com}
\sectionauthor{Barry A. Warsaw}{barry@zope.com}

\versionadded{2.2}

The \module{email} package is a library for managing email messages,
including MIME and other \rfc{2822}-based message documents.  It
subsumes most of the functionality in several older standard modules
such as \refmodule{rfc822}, \refmodule{mimetools},
\refmodule{multifile}, and other non-standard packages such as
\module{mimecntl}.  It is specifically \emph{not} designed to do any
sending of email messages to SMTP (\rfc{2821}) servers; that is the
function of the \refmodule{smtplib} module.  The \module{email}
package attempts to be as RFC-compliant as possible, supporting in
addition to \rfc{2822}, such MIME-related RFCs as
\rfc{2045}-\rfc{2047}, and \rfc{2231}.

The primary distinguishing feature of the \module{email} package is
that it splits the parsing and generating of email messages from the
internal \emph{object model} representation of email.  Applications
using the \module{email} package deal primarily with objects; you can
add sub-objects to messages, remove sub-objects from messages,
completely re-arrange the contents, etc.  There is a separate parser
and a separate generator which handles the transformation from flat
text to the object model, and then back to flat text again.  There
are also handy subclasses for some common MIME object types, and a few
miscellaneous utilities that help with such common tasks as extracting
and parsing message field values, creating RFC-compliant dates, etc.

The following sections describe the functionality of the
\module{email} package.  The ordering follows a progression that
should be common in applications: an email message is read as flat
text from a file or other source, the text is parsed to produce the
object structure of the email message, this structure is manipulated,
and finally rendered back into flat text.

It is perfectly feasible to create the object structure out of whole
cloth --- i.e. completely from scratch.  From there, a similar
progression can be taken as above.

Also included are detailed specifications of all the classes and
modules that the \module{email} package provides, the exception
classes you might encounter while using the \module{email} package,
some auxiliary utilities, and a few examples.  For users of the older
\module{mimelib} package, or previous versions of the \module{email}
package, a section on differences and porting is provided.

\begin{seealso}
    \seemodule{smtplib}{SMTP protocol client}
\end{seealso}

\subsection{Representing an email message}
\declaremodule{standard}{email.Message}
\modulesynopsis{The base class representing email messages.}

The central class in the \module{email} package is the
\class{Message} class; it is the base class for the \module{email}
object model.  \class{Message} provides the core functionality for
setting and querying header fields, and for accessing message bodies.

Conceptually, a \class{Message} object consists of \emph{headers} and
\emph{payloads}.  Headers are \rfc{2822} style field names and
values where the field name and value are separated by a colon.  The
colon is not part of either the field name or the field value.

Headers are stored and returned in case-preserving form but are
matched case-insensitively.  There may also be a single envelope
header, also known as the \emph{Unix-From} header or the
\code{From_} header.  The payload is either a string in the case of
simple message objects or a list of \class{Message} objects for
MIME container documents (e.g. \mimetype{multipart/*} and
\mimetype{message/rfc822}).

\class{Message} objects provide a mapping style interface for
accessing the message headers, and an explicit interface for accessing
both the headers and the payload.  It provides convenience methods for
generating a flat text representation of the message object tree, for
accessing commonly used header parameters, and for recursively walking
over the object tree.

Here are the methods of the \class{Message} class:

\begin{classdesc}{Message}{}
The constructor takes no arguments.
\end{classdesc}

\begin{methoddesc}[Message]{as_string}{\optional{unixfrom}}
Return the entire message flatten as a string.  When optional
\var{unixfrom} is \code{True}, the envelope header is included in the
returned string.  \var{unixfrom} defaults to \code{False}.
\end{methoddesc}

\begin{methoddesc}[Message]{__str__}{}
Equivalent to \method{as_string(unixfrom=True)}.
\end{methoddesc}

\begin{methoddesc}[Message]{is_multipart}{}
Return \code{True} if the message's payload is a list of
sub-\class{Message} objects, otherwise return \code{False}.  When
\method{is_multipart()} returns False, the payload should be a string
object.
\end{methoddesc}

\begin{methoddesc}[Message]{set_unixfrom}{unixfrom}
Set the message's envelope header to \var{unixfrom}, which should be a string.
\end{methoddesc}

\begin{methoddesc}[Message]{get_unixfrom}{}
Return the message's envelope header.  Defaults to \code{None} if the
envelope header was never set.
\end{methoddesc}

\begin{methoddesc}[Message]{attach}{payload}
Add the given \var{payload} to the current payload, which must be
\code{None} or a list of \class{Message} objects before the call.
After the call, the payload will always be a list of \class{Message}
objects.  If you want to set the payload to a scalar object (e.g. a
string), use \method{set_payload()} instead.
\end{methoddesc}

\begin{methoddesc}[Message]{get_payload}{\optional{i\optional{, decode}}}
Return a reference the current payload, which will be a list of
\class{Message} objects when \method{is_multipart()} is \code{True}, or a
string when \method{is_multipart()} is \code{False}.  If the
payload is a list and you mutate the list object, you modify the
message's payload in place.

With optional argument \var{i}, \method{get_payload()} will return the
\var{i}-th element of the payload, counting from zero, if
\method{is_multipart()} is \code{True}.  An \exception{IndexError}
will be raised if \var{i} is less than 0 or greater than or equal to
the number of items in the payload.  If the payload is a string
(i.e. \method{is_multipart()} is \code{False}) and \var{i} is given, a
\exception{TypeError} is raised.

Optional \var{decode} is a flag indicating whether the payload should be
decoded or not, according to the \mailheader{Content-Transfer-Encoding} header.
When \code{True} and the message is not a multipart, the payload will be
decoded if this header's value is \samp{quoted-printable} or
\samp{base64}.  If some other encoding is used, or
\mailheader{Content-Transfer-Encoding} header is
missing, the payload is returned as-is (undecoded).  If the message is
a multipart and the \var{decode} flag is \code{True}, then \code{None} is
returned.  The default for \var{decode} is \code{False}.
\end{methoddesc}

\begin{methoddesc}[Message]{set_payload}{payload\optional{, charset}}
Set the entire message object's payload to \var{payload}.  It is the
client's responsibility to ensure the payload invariants.  Optional
\var{charset} sets the message's default character set; see
\method{set_charset()} for details.

\versionchanged[\var{charset} argument added]{2.2.2}
\end{methoddesc}

\begin{methoddesc}[Message]{set_charset}{charset}
Set the character set of the payload to \var{charset}, which can
either be a \class{Charset} instance (see \refmodule{email.Charset}), a
string naming a character set,
or \code{None}.  If it is a string, it will be converted to a
\class{Charset} instance.  If \var{charset} is \code{None}, the
\code{charset} parameter will be removed from the
\mailheader{Content-Type} header. Anything else will generate a
\exception{TypeError}.

The message will be assumed to be of type \mimetype{text/*} encoded with
\code{charset.input_charset}.  It will be converted to
\code{charset.output_charset}
and encoded properly, if needed, when generating the plain text
representation of the message.  MIME headers
(\mailheader{MIME-Version}, \mailheader{Content-Type},
\mailheader{Content-Transfer-Encoding}) will be added as needed.

\versionadded{2.2.2}
\end{methoddesc}

\begin{methoddesc}[Message]{get_charset}{}
Return the \class{Charset} instance associated with the message's payload.
\versionadded{2.2.2}
\end{methoddesc}

The following methods implement a mapping-like interface for accessing
the message's \rfc{2822} headers.  Note that there are some
semantic differences between these methods and a normal mapping
(i.e. dictionary) interface.  For example, in a dictionary there are
no duplicate keys, but here there may be duplicate message headers.  Also,
in dictionaries there is no guaranteed order to the keys returned by
\method{keys()}, but in a \class{Message} object, headers are always
returned in the order they appeared in the original message, or were
added to the message later.  Any header deleted and then re-added are
always appended to the end of the header list.

These semantic differences are intentional and are biased toward
maximal convenience.

Note that in all cases, any envelope header present in the message is
not included in the mapping interface.

\begin{methoddesc}[Message]{__len__}{}
Return the total number of headers, including duplicates.
\end{methoddesc}

\begin{methoddesc}[Message]{__contains__}{name}
Return true if the message object has a field named \var{name}.
Matching is done case-insensitively and \var{name} should not include the
trailing colon.  Used for the \code{in} operator,
e.g.:

\begin{verbatim}
if 'message-id' in myMessage:
    print 'Message-ID:', myMessage['message-id']
\end{verbatim}
\end{methoddesc}

\begin{methoddesc}[Message]{__getitem__}{name}
Return the value of the named header field.  \var{name} should not
include the colon field separator.  If the header is missing,
\code{None} is returned; a \exception{KeyError} is never raised.

Note that if the named field appears more than once in the message's
headers, exactly which of those field values will be returned is
undefined.  Use the \method{get_all()} method to get the values of all
the extant named headers.
\end{methoddesc}

\begin{methoddesc}[Message]{__setitem__}{name, val}
Add a header to the message with field name \var{name} and value
\var{val}.  The field is appended to the end of the message's existing
fields.

Note that this does \emph{not} overwrite or delete any existing header
with the same name.  If you want to ensure that the new header is the
only one present in the message with field name
\var{name}, delete the field first, e.g.:

\begin{verbatim}
del msg['subject']
msg['subject'] = 'Python roolz!'
\end{verbatim}
\end{methoddesc}

\begin{methoddesc}[Message]{__delitem__}{name}
Delete all occurrences of the field with name \var{name} from the
message's headers.  No exception is raised if the named field isn't
present in the headers.
\end{methoddesc}

\begin{methoddesc}[Message]{has_key}{name}
Return true if the message contains a header field named \var{name},
otherwise return false.
\end{methoddesc}

\begin{methoddesc}[Message]{keys}{}
Return a list of all the message's header field names.
\end{methoddesc}

\begin{methoddesc}[Message]{values}{}
Return a list of all the message's field values.
\end{methoddesc}

\begin{methoddesc}[Message]{items}{}
Return a list of 2-tuples containing all the message's field headers
and values.
\end{methoddesc}

\begin{methoddesc}[Message]{get}{name\optional{, failobj}}
Return the value of the named header field.  This is identical to
\method{__getitem__()} except that optional \var{failobj} is returned
if the named header is missing (defaults to \code{None}).
\end{methoddesc}

Here are some additional useful methods:

\begin{methoddesc}[Message]{get_all}{name\optional{, failobj}}
Return a list of all the values for the field named \var{name}.
If there are no such named headers in the message, \var{failobj} is
returned (defaults to \code{None}).
\end{methoddesc}

\begin{methoddesc}[Message]{add_header}{_name, _value, **_params}
Extended header setting.  This method is similar to
\method{__setitem__()} except that additional header parameters can be
provided as keyword arguments.  \var{_name} is the header field to add
and \var{_value} is the \emph{primary} value for the header.

For each item in the keyword argument dictionary \var{_params}, the
key is taken as the parameter name, with underscores converted to
dashes (since dashes are illegal in Python identifiers).  Normally,
the parameter will be added as \code{key="value"} unless the value is
\code{None}, in which case only the key will be added.

Here's an example:

\begin{verbatim}
msg.add_header('Content-Disposition', 'attachment', filename='bud.gif')
\end{verbatim}

This will add a header that looks like

\begin{verbatim}
Content-Disposition: attachment; filename="bud.gif"
\end{verbatim}
\end{methoddesc}

\begin{methoddesc}[Message]{replace_header}{_name, _value}
Replace a header.  Replace the first header found in the message that
matches \var{_name}, retaining header order and field name case.  If
no matching header was found, a \exception{KeyError} is raised.

\versionadded{2.2.2}
\end{methoddesc}

\begin{methoddesc}[Message]{get_content_type}{}
Return the message's content type.  The returned string is coerced to
lower case of the form \mimetype{maintype/subtype}.  If there was no
\mailheader{Content-Type} header in the message the default type as
given by \method{get_default_type()} will be returned.  Since
according to \rfc{2045}, messages always have a default type,
\method{get_content_type()} will always return a value.

\rfc{2045} defines a message's default type to be
\mimetype{text/plain} unless it appears inside a
\mimetype{multipart/digest} container, in which case it would be
\mimetype{message/rfc822}.  If the \mailheader{Content-Type} header
has an invalid type specification, \rfc{2045} mandates that the
default type be \mimetype{text/plain}.

\versionadded{2.2.2}
\end{methoddesc}

\begin{methoddesc}[Message]{get_content_maintype}{}
Return the message's main content type.  This is the
\mimetype{maintype} part of the string returned by
\method{get_content_type()}.

\versionadded{2.2.2}
\end{methoddesc}

\begin{methoddesc}[Message]{get_content_subtype}{}
Return the message's sub-content type.  This is the \mimetype{subtype}
part of the string returned by \method{get_content_type()}.

\versionadded{2.2.2}
\end{methoddesc}

\begin{methoddesc}[Message]{get_default_type}{}
Return the default content type.  Most messages have a default content
type of \mimetype{text/plain}, except for messages that are subparts
of \mimetype{multipart/digest} containers.  Such subparts have a
default content type of \mimetype{message/rfc822}.

\versionadded{2.2.2}
\end{methoddesc}

\begin{methoddesc}[Message]{set_default_type}{ctype}
Set the default content type.  \var{ctype} should either be
\mimetype{text/plain} or \mimetype{message/rfc822}, although this is
not enforced.  The default content type is not stored in the
\mailheader{Content-Type} header.

\versionadded{2.2.2}
\end{methoddesc}

\begin{methoddesc}[Message]{get_params}{\optional{failobj\optional{,
    header\optional{, unquote}}}}
Return the message's \mailheader{Content-Type} parameters, as a list.  The
elements of the returned list are 2-tuples of key/value pairs, as
split on the \character{=} sign.  The left hand side of the
\character{=} is the key, while the right hand side is the value.  If
there is no \character{=} sign in the parameter the value is the empty
string, otherwise the value is as described in \method{get_param()} and is
unquoted if optional \var{unquote} is \code{True} (the default).

Optional \var{failobj} is the object to return if there is no
\mailheader{Content-Type} header.  Optional \var{header} is the header to
search instead of \mailheader{Content-Type}.

\versionchanged[\var{unquote} argument added]{2.2.2}
\end{methoddesc}

\begin{methoddesc}[Message]{get_param}{param\optional{,
    failobj\optional{, header\optional{, unquote}}}}
Return the value of the \mailheader{Content-Type} header's parameter
\var{param} as a string.  If the message has no \mailheader{Content-Type}
header or if there is no such parameter, then \var{failobj} is
returned (defaults to \code{None}).

Optional \var{header} if given, specifies the message header to use
instead of \mailheader{Content-Type}.

Parameter keys are always compared case insensitively.  The return
value can either be a string, or a 3-tuple if the parameter was
\rfc{2231} encoded.  When it's a 3-tuple, the elements of the value are of
the form \code{(CHARSET, LANGUAGE, VALUE)}, where \code{LANGUAGE} may
be the empty string.  Your application should be prepared to deal with
3-tuple return values, which it can convert to a Unicode string like
so:

\begin{verbatim}
param = msg.get_param('foo')
if isinstance(param, tuple):
    param = unicode(param[2], param[0])
\end{verbatim}

In any case, the parameter value (either the returned string, or the
\code{VALUE} item in the 3-tuple) is always unquoted, unless
\var{unquote} is set to \code{False}.

\versionchanged[\var{unquote} argument added, and 3-tuple return value
possible]{2.2.2}
\end{methoddesc}

\begin{methoddesc}[Message]{set_param}{param, value\optional{,
    header\optional{, requote\optional{, charset\optional{, language}}}}}

Set a parameter in the \mailheader{Content-Type} header.  If the
parameter already exists in the header, its value will be replaced
with \var{value}.  If the \mailheader{Content-Type} header as not yet
been defined for this message, it will be set to \mimetype{text/plain}
and the new parameter value will be appended as per \rfc{2045}.

Optional \var{header} specifies an alternative header to
\mailheader{Content-Type}, and all parameters will be quoted as
necessary unless optional \var{requote} is \code{False} (the default
is \code{True}).

If optional \var{charset} is specified, the parameter will be encoded
according to \rfc{2231}. Optional \var{language} specifies the RFC
2231 language, defaulting to the empty string.  Both \var{charset} and
\var{language} should be strings.

\versionadded{2.2.2}
\end{methoddesc}

\begin{methoddesc}[Message]{del_param}{param\optional{, header\optional{,
    requote}}}
Remove the given parameter completely from the
\mailheader{Content-Type} header.  The header will be re-written in
place without the parameter or its value.  All values will be quoted
as necessary unless \var{requote} is \code{False} (the default is
\code{True}).  Optional \var{header} specifies an alternative to
\mailheader{Content-Type}.

\versionadded{2.2.2}
\end{methoddesc}

\begin{methoddesc}[Message]{set_type}{type\optional{, header}\optional{,
    requote}}
Set the main type and subtype for the \mailheader{Content-Type}
header. \var{type} must be a string in the form
\mimetype{maintype/subtype}, otherwise a \exception{ValueError} is
raised.

This method replaces the \mailheader{Content-Type} header, keeping all
the parameters in place.  If \var{requote} is \code{False}, this
leaves the existing header's quoting as is, otherwise the parameters
will be quoted (the default).

An alternative header can be specified in the \var{header} argument.
When the \mailheader{Content-Type} header is set a
\mailheader{MIME-Version} header is also added.

\versionadded{2.2.2}
\end{methoddesc}

\begin{methoddesc}[Message]{get_filename}{\optional{failobj}}
Return the value of the \code{filename} parameter of the
\mailheader{Content-Disposition} header of the message, or \var{failobj} if
either the header is missing, or has no \code{filename} parameter.
The returned string will always be unquoted as per
\method{Utils.unquote()}.
\end{methoddesc}

\begin{methoddesc}[Message]{get_boundary}{\optional{failobj}}
Return the value of the \code{boundary} parameter of the
\mailheader{Content-Type} header of the message, or \var{failobj} if either
the header is missing, or has no \code{boundary} parameter.  The
returned string will always be unquoted as per
\method{Utils.unquote()}.
\end{methoddesc}

\begin{methoddesc}[Message]{set_boundary}{boundary}
Set the \code{boundary} parameter of the \mailheader{Content-Type}
header to \var{boundary}.  \method{set_boundary()} will always quote
\var{boundary} if necessary.  A \exception{HeaderParseError} is raised
if the message object has no \mailheader{Content-Type} header.

Note that using this method is subtly different than deleting the old
\mailheader{Content-Type} header and adding a new one with the new boundary
via \method{add_header()}, because \method{set_boundary()} preserves the
order of the \mailheader{Content-Type} header in the list of headers.
However, it does \emph{not} preserve any continuation lines which may
have been present in the original \mailheader{Content-Type} header.
\end{methoddesc}

\begin{methoddesc}[Message]{get_content_charset}{\optional{failobj}}
Return the \code{charset} parameter of the \mailheader{Content-Type}
header, coerced to lower case.  If there is no
\mailheader{Content-Type} header, or if that header has no
\code{charset} parameter, \var{failobj} is returned.

Note that this method differs from \method{get_charset()} which
returns the \class{Charset} instance for the default encoding of the
message body.

\versionadded{2.2.2}
\end{methoddesc}

\begin{methoddesc}[Message]{get_charsets}{\optional{failobj}}
Return a list containing the character set names in the message.  If
the message is a \mimetype{multipart}, then the list will contain one
element for each subpart in the payload, otherwise, it will be a list
of length 1.

Each item in the list will be a string which is the value of the
\code{charset} parameter in the \mailheader{Content-Type} header for the
represented subpart.  However, if the subpart has no
\mailheader{Content-Type} header, no \code{charset} parameter, or is not of
the \mimetype{text} main MIME type, then that item in the returned list
will be \var{failobj}.
\end{methoddesc}

\begin{methoddesc}[Message]{walk}{}
The \method{walk()} method is an all-purpose generator which can be
used to iterate over all the parts and subparts of a message object
tree, in depth-first traversal order.  You will typically use
\method{walk()} as the iterator in a \code{for} loop; each
iteration returns the next subpart.

Here's an example that prints the MIME type of every part of a
multipart message structure:

\begin{verbatim}
>>> for part in msg.walk():
>>>     print part.get_content_type()
multipart/report
text/plain
message/delivery-status
text/plain
text/plain
message/rfc822
\end{verbatim}
\end{methoddesc}

\class{Message} objects can also optionally contain two instance
attributes, which can be used when generating the plain text of a MIME
message.

\begin{datadesc}{preamble}
The format of a MIME document allows for some text between the blank
line following the headers, and the first multipart boundary string.
Normally, this text is never visible in a MIME-aware mail reader
because it falls outside the standard MIME armor.  However, when
viewing the raw text of the message, or when viewing the message in a
non-MIME aware reader, this text can become visible.

The \var{preamble} attribute contains this leading extra-armor text
for MIME documents.  When the \class{Parser} discovers some text after
the headers but before the first boundary string, it assigns this text
to the message's \var{preamble} attribute.  When the \class{Generator}
is writing out the plain text representation of a MIME message, and it
finds the message has a \var{preamble} attribute, it will write this
text in the area between the headers and the first boundary.  See
\refmodule{email.Parser} and \refmodule{email.Generator} for details.

Note that if the message object has no preamble, the
\var{preamble} attribute will be \code{None}.
\end{datadesc}

\begin{datadesc}{epilogue}
The \var{epilogue} attribute acts the same way as the \var{preamble}
attribute, except that it contains text that appears between the last
boundary and the end of the message.

One note: when generating the flat text for a \mimetype{multipart}
message that has no \var{epilogue} (using the standard
\class{Generator} class), no newline is added after the closing
boundary line.  If the message object has an \var{epilogue} and its
value does not start with a newline, a newline is printed after the
closing boundary.  This seems a little clumsy, but it makes the most
practical sense.  The upshot is that if you want to ensure that a
newline get printed after your closing \mimetype{multipart} boundary,
set the \var{epilogue} to the empty string.
\end{datadesc}

\subsubsection{Deprecated methods}

The following methods are deprecated in \module{email} version 2.
They are documented here for completeness.

\begin{methoddesc}[Message]{add_payload}{payload}
Add \var{payload} to the message object's existing payload.  If, prior
to calling this method, the object's payload was \code{None}
(i.e. never before set), then after this method is called, the payload
will be the argument \var{payload}.

If the object's payload was already a list
(i.e. \method{is_multipart()} returns 1), then \var{payload} is
appended to the end of the existing payload list.

For any other type of existing payload, \method{add_payload()} will
transform the new payload into a list consisting of the old payload
and \var{payload}, but only if the document is already a MIME
multipart document.  This condition is satisfied if the message's
\mailheader{Content-Type} header's main type is either
\mimetype{multipart}, or there is no \mailheader{Content-Type}
header.  In any other situation,
\exception{MultipartConversionError} is raised.

\deprecated{2.2.2}{Use the \method{attach()} method instead.}
\end{methoddesc}

\begin{methoddesc}[Message]{get_type}{\optional{failobj}}
Return the message's content type, as a string of the form
\mimetype{maintype/subtype} as taken from the
\mailheader{Content-Type} header.
The returned string is coerced to lowercase.

If there is no \mailheader{Content-Type} header in the message,
\var{failobj} is returned (defaults to \code{None}).

\deprecated{2.2.2}{Use the \method{get_content_type()} method instead.}
\end{methoddesc}

\begin{methoddesc}[Message]{get_main_type}{\optional{failobj}}
Return the message's \emph{main} content type.  This essentially returns the
\var{maintype} part of the string returned by \method{get_type()}, with the
same semantics for \var{failobj}.

\deprecated{2.2.2}{Use the \method{get_content_maintype()} method instead.}
\end{methoddesc}

\begin{methoddesc}[Message]{get_subtype}{\optional{failobj}}
Return the message's sub-content type.  This essentially returns the
\var{subtype} part of the string returned by \method{get_type()}, with the
same semantics for \var{failobj}.

\deprecated{2.2.2}{Use the \method{get_content_subtype()} method instead.}
\end{methoddesc}



\subsection{Parsing email messages}
\declaremodule{standard}{email.parser}
\modulesynopsis{Parse flat text email messages to produce a message
	        object structure.}

Message object structures can be created in one of two ways: they can be
created from whole cloth by instantiating \class{Message} objects and
stringing them together via \method{attach()} and
\method{set_payload()} calls, or they can be created by parsing a flat text
representation of the email message.

The \module{email} package provides a standard parser that understands
most email document structures, including MIME documents.  You can
pass the parser a string or a file object, and the parser will return
to you the root \class{Message} instance of the object structure.  For
simple, non-MIME messages the payload of this root object will likely
be a string containing the text of the message.  For MIME
messages, the root object will return \code{True} from its
\method{is_multipart()} method, and the subparts can be accessed via
the \method{get_payload()} and \method{walk()} methods.

There are actually two parser interfaces available for use, the classic
\class{Parser} API and the incremental \class{FeedParser} API.  The classic
\class{Parser} API is fine if you have the entire text of the message in
memory as a string, or if the entire message lives in a file on the file
system.  \class{FeedParser} is more appropriate for when you're reading the
message from a stream which might block waiting for more input (e.g. reading
an email message from a socket).  The \class{FeedParser} can consume and parse
the message incrementally, and only returns the root object when you close the
parser\footnote{As of email package version 3.0, introduced in
Python 2.4, the classic \class{Parser} was re-implemented in terms of the
\class{FeedParser}, so the semantics and results are identical between the two
parsers.}.

Note that the parser can be extended in limited ways, and of course
you can implement your own parser completely from scratch.  There is
no magical connection between the \module{email} package's bundled
parser and the \class{Message} class, so your custom parser can create
message object trees any way it finds necessary.

\subsubsection{FeedParser API}

\versionadded{2.4}

The \class{FeedParser}, imported from the \module{email.feedparser} module,
provides an API that is conducive to incremental parsing of email messages,
such as would be necessary when reading the text of an email message from a
source that can block (e.g. a socket).  The
\class{FeedParser} can of course be used to parse an email message fully
contained in a string or a file, but the classic \class{Parser} API may be
more convenient for such use cases.  The semantics and results of the two
parser APIs are identical.

The \class{FeedParser}'s API is simple; you create an instance, feed it a
bunch of text until there's no more to feed it, then close the parser to
retrieve the root message object.  The \class{FeedParser} is extremely
accurate when parsing standards-compliant messages, and it does a very good
job of parsing non-compliant messages, providing information about how a
message was deemed broken.  It will populate a message object's \var{defects}
attribute with a list of any problems it found in a message.  See the
\refmodule{email.errors} module for the list of defects that it can find.

Here is the API for the \class{FeedParser}:

\begin{classdesc}{FeedParser}{\optional{_factory}}
Create a \class{FeedParser} instance.  Optional \var{_factory} is a
no-argument callable that will be called whenever a new message object is
needed.  It defaults to the \class{email.message.Message} class.
\end{classdesc}

\begin{methoddesc}[FeedParser]{feed}{data}
Feed the \class{FeedParser} some more data.  \var{data} should be a
string containing one or more lines.  The lines can be partial and the
\class{FeedParser} will stitch such partial lines together properly.  The
lines in the string can have any of the common three line endings, carriage
return, newline, or carriage return and newline (they can even be mixed).
\end{methoddesc}

\begin{methoddesc}[FeedParser]{close}{}
Closing a \class{FeedParser} completes the parsing of all previously fed data,
and returns the root message object.  It is undefined what happens if you feed
more data to a closed \class{FeedParser}.
\end{methoddesc}

\subsubsection{Parser class API}

The \class{Parser} class, imported from the \module{email.parser} module,
provides an API that can be used to parse a message when the complete contents
of the message are available in a string or file.  The
\module{email.parser} module also provides a second class, called
\class{HeaderParser} which can be used if you're only interested in
the headers of the message. \class{HeaderParser} can be much faster in
these situations, since it does not attempt to parse the message body,
instead setting the payload to the raw body as a string.
\class{HeaderParser} has the same API as the \class{Parser} class.

\begin{classdesc}{Parser}{\optional{_class}}
The constructor for the \class{Parser} class takes an optional
argument \var{_class}.  This must be a callable factory (such as a
function or a class), and it is used whenever a sub-message object
needs to be created.  It defaults to \class{Message} (see
\refmodule{email.message}).  The factory will be called without
arguments.

The optional \var{strict} flag is ignored.  \deprecated{2.4}{Because the
\class{Parser} class is a backward compatible API wrapper around the
new-in-Python 2.4 \class{FeedParser}, \emph{all} parsing is effectively
non-strict.  You should simply stop passing a \var{strict} flag to the
\class{Parser} constructor.}

\versionchanged[The \var{strict} flag was added]{2.2.2}
\versionchanged[The \var{strict} flag was deprecated]{2.4}
\end{classdesc}

The other public \class{Parser} methods are:

\begin{methoddesc}[Parser]{parse}{fp\optional{, headersonly}}
Read all the data from the file-like object \var{fp}, parse the
resulting text, and return the root message object.  \var{fp} must
support both the \method{readline()} and the \method{read()} methods
on file-like objects.

The text contained in \var{fp} must be formatted as a block of \rfc{2822}
style headers and header continuation lines, optionally preceded by a
envelope header.  The header block is terminated either by the
end of the data or by a blank line.  Following the header block is the
body of the message (which may contain MIME-encoded subparts).

Optional \var{headersonly} is as with the \method{parse()} method.

\versionchanged[The \var{headersonly} flag was added]{2.2.2}
\end{methoddesc}

\begin{methoddesc}[Parser]{parsestr}{text\optional{, headersonly}}
Similar to the \method{parse()} method, except it takes a string
object instead of a file-like object.  Calling this method on a string
is exactly equivalent to wrapping \var{text} in a \class{StringIO}
instance first and calling \method{parse()}.

Optional \var{headersonly} is a flag specifying whether to stop
parsing after reading the headers or not.  The default is \code{False},
meaning it parses the entire contents of the file.

\versionchanged[The \var{headersonly} flag was added]{2.2.2}
\end{methoddesc}

Since creating a message object structure from a string or a file
object is such a common task, two functions are provided as a
convenience.  They are available in the top-level \module{email}
package namespace.

\begin{funcdesc}{message_from_string}{s\optional{, _class\optional{, strict}}}
Return a message object structure from a string.  This is exactly
equivalent to \code{Parser().parsestr(s)}.  Optional \var{_class} and
\var{strict} are interpreted as with the \class{Parser} class constructor.

\versionchanged[The \var{strict} flag was added]{2.2.2}
\end{funcdesc}

\begin{funcdesc}{message_from_file}{fp\optional{, _class\optional{, strict}}}
Return a message object structure tree from an open file object.  This
is exactly equivalent to \code{Parser().parse(fp)}.  Optional
\var{_class} and \var{strict} are interpreted as with the
\class{Parser} class constructor.

\versionchanged[The \var{strict} flag was added]{2.2.2}
\end{funcdesc}

Here's an example of how you might use this at an interactive Python
prompt:

\begin{verbatim}
>>> import email
>>> msg = email.message_from_string(myString)
\end{verbatim}

\subsubsection{Additional notes}

Here are some notes on the parsing semantics:

\begin{itemize}
\item Most non-\mimetype{multipart} type messages are parsed as a single
      message object with a string payload.  These objects will return
      \code{False} for \method{is_multipart()}.  Their
      \method{get_payload()} method will return a string object.

\item All \mimetype{multipart} type messages will be parsed as a
      container message object with a list of sub-message objects for
      their payload.  The outer container message will return
      \code{True} for \method{is_multipart()} and their
      \method{get_payload()} method will return the list of
      \class{Message} subparts.

\item Most messages with a content type of \mimetype{message/*}
      (e.g. \mimetype{message/delivery-status} and
      \mimetype{message/rfc822}) will also be parsed as container
      object containing a list payload of length 1.  Their
      \method{is_multipart()} method will return \code{True}.  The
      single element in the list payload will be a sub-message object.

\item Some non-standards compliant messages may not be internally consistent
      about their \mimetype{multipart}-edness.  Such messages may have a
      \mailheader{Content-Type} header of type \mimetype{multipart}, but their
      \method{is_multipart()} method may return \code{False}.  If such
      messages were parsed with the \class{FeedParser}, they will have an
      instance of the \class{MultipartInvariantViolationDefect} class in their
      \var{defects} attribute list.  See \refmodule{email.errors} for
      details.
\end{itemize}


\subsection{Generating MIME documents}
\declaremodule{standard}{email.Generator}
\modulesynopsis{Generate flat text email messages from a message object tree.}

One of the most common tasks is to generate the flat text of the email
message represented by a message object tree.  You will need to do
this if you want to send your message via the \refmodule{smtplib}
module or the \refmodule{nntplib} module, or print the message on the
console.  Taking a message object tree and producing a flat text
document is the job of the \class{Generator} class.

Again, as with the \refmodule{email.Parser} module, you aren't limited
to the functionality of the bundled generator; you could write one
from scratch yourself.  However the bundled generator knows how to
generate most email in a standards-compliant way, should handle MIME
and non-MIME email messages just fine, and is designed so that the
transformation from flat text, to an object tree via the
\class{Parser} class,
and back to flat text, is idempotent (the input is identical to the
output).

Here are the public methods of the \class{Generator} class:

\begin{classdesc}{Generator}{outfp\optional{, mangle_from_\optional{,
    maxheaderlen}}}
The constructor for the \class{Generator} class takes a file-like
object called \var{outfp} for an argument.  \var{outfp} must support
the \method{write()} method and be usable as the output file in a
Python 2.0 extended print statement.

Optional \var{mangle_from_} is a flag that, when true, puts a \samp{>}
character in front of any line in the body that starts exactly as
\samp{From } (i.e. \code{From} followed by a space at the front of the
line).  This is the only guaranteed portable way to avoid having such
lines be mistaken for \emph{Unix-From} headers (see
\ulink{WHY THE CONTENT-LENGTH FORMAT IS BAD}
{http://home.netscape.com/eng/mozilla/2.0/relnotes/demo/content-length.html}
for details).

Optional \var{maxheaderlen} specifies the longest length for a
non-continued header.  When a header line is longer than
\var{maxheaderlen} (in characters, with tabs expanded to 8 spaces),
the header will be broken on semicolons and continued as per
\rfc{2822}.  If no semicolon is found, then the header is left alone.
Set to zero to disable wrapping headers.  Default is 78, as
recommended (but not required) by \rfc{2822}.
\end{classdesc}

The other public \class{Generator} methods are:

\begin{methoddesc}[Generator]{__call__}{msg\optional{, unixfrom}}
Print the textual representation of the message object tree rooted at
\var{msg} to the output file specified when the \class{Generator}
instance was created.  Sub-objects are visited depth-first and the
resulting text will be properly MIME encoded.

Optional \var{unixfrom} is a flag that forces the printing of the
\emph{Unix-From} (a.k.a. envelope header or \code{From_} header)
delimiter before the first \rfc{2822} header of the root message
object.  If the root object has no \emph{Unix-From} header, a standard
one is crafted.  By default, this is set to 0 to inhibit the printing
of the \emph{Unix-From} delimiter.

Note that for sub-objects, no \emph{Unix-From} header is ever printed.
\end{methoddesc}

\begin{methoddesc}[Generator]{write}{s}
Write the string \var{s} to the underlying file object,
i.e. \var{outfp} passed to \class{Generator}'s constructor.  This
provides just enough file-like API for \class{Generator} instances to
be used in extended print statements.
\end{methoddesc}

As a convenience, see the methods \method{Message.as_string()} and
\code{str(aMessage)}, a.k.a. \method{Message.__str__()}, which
simplify the generation of a formatted string representation of a
message object.  For more detail, see \refmodule{email.Message}.


\subsection{Creating email and MIME objects from scratch}
\declaremodule{standard}{email.mime}
\declaremodule{standard}{email.mime.base}
\declaremodule{standard}{email.mime.nonmultipart}
\declaremodule{standard}{email.mime.multipart}
\declaremodule{standard}{email.mime.audio}
\declaremodule{standard}{email.mime.image}
\declaremodule{standard}{email.mime.message}
\declaremodule{standard}{email.mime.text}
Ordinarily, you get a message object structure by passing a file or
some text to a parser, which parses the text and returns the root
message object.  However you can also build a complete message
structure from scratch, or even individual \class{Message} objects by
hand.  In fact, you can also take an existing structure and add new
\class{Message} objects, move them around, etc.  This makes a very
convenient interface for slicing-and-dicing MIME messages.

You can create a new object structure by creating \class{Message} instances,
adding attachments and all the appropriate headers manually.  For MIME
messages though, the \module{email} package provides some convenient
subclasses to make things easier.

Here are the classes:

\begin{classdesc}{MIMEBase}{_maintype, _subtype, **_params}
Module: \module{email.mime.base}

This is the base class for all the MIME-specific subclasses of
\class{Message}.  Ordinarily you won't create instances specifically
of \class{MIMEBase}, although you could.  \class{MIMEBase} is provided
primarily as a convenient base class for more specific MIME-aware
subclasses.

\var{_maintype} is the \mailheader{Content-Type} major type
(e.g. \mimetype{text} or \mimetype{image}), and \var{_subtype} is the
\mailheader{Content-Type} minor type 
(e.g. \mimetype{plain} or \mimetype{gif}).  \var{_params} is a parameter
key/value dictionary and is passed directly to
\method{Message.add_header()}.

The \class{MIMEBase} class always adds a \mailheader{Content-Type} header
(based on \var{_maintype}, \var{_subtype}, and \var{_params}), and a
\mailheader{MIME-Version} header (always set to \code{1.0}).
\end{classdesc}

\begin{classdesc}{MIMENonMultipart}{}
Module: \module{email.mime.nonmultipart}

A subclass of \class{MIMEBase}, this is an intermediate base class for
MIME messages that are not \mimetype{multipart}.  The primary purpose
of this class is to prevent the use of the \method{attach()} method,
which only makes sense for \mimetype{multipart} messages.  If
\method{attach()} is called, a \exception{MultipartConversionError}
exception is raised.

\versionadded{2.2.2}
\end{classdesc}

\begin{classdesc}{MIMEMultipart}{\optional{subtype\optional{,
    boundary\optional{, _subparts\optional{, _params}}}}}
Module: \module{email.mime.multipart}

A subclass of \class{MIMEBase}, this is an intermediate base class for
MIME messages that are \mimetype{multipart}.  Optional \var{_subtype}
defaults to \mimetype{mixed}, but can be used to specify the subtype
of the message.  A \mailheader{Content-Type} header of
\mimetype{multipart/}\var{_subtype} will be added to the message
object.  A \mailheader{MIME-Version} header will also be added.

Optional \var{boundary} is the multipart boundary string.  When
\code{None} (the default), the boundary is calculated when needed.

\var{_subparts} is a sequence of initial subparts for the payload.  It
must be possible to convert this sequence to a list.  You can always
attach new subparts to the message by using the
\method{Message.attach()} method.

Additional parameters for the \mailheader{Content-Type} header are
taken from the keyword arguments, or passed into the \var{_params}
argument, which is a keyword dictionary.

\versionadded{2.2.2}
\end{classdesc}

\begin{classdesc}{MIMEApplication}{_data\optional{, _subtype\optional{,
    _encoder\optional{, **_params}}}}
Module: \module{email.mime.application}

A subclass of \class{MIMENonMultipart}, the \class{MIMEApplication} class is
used to represent MIME message objects of major type \mimetype{application}.
\var{_data} is a string containing the raw byte data.  Optional \var{_subtype}
specifies the MIME subtype and defaults to \mimetype{octet-stream}.  

Optional \var{_encoder} is a callable (i.e. function) which will
perform the actual encoding of the data for transport.  This
callable takes one argument, which is the \class{MIMEApplication} instance.
It should use \method{get_payload()} and \method{set_payload()} to
change the payload to encoded form.  It should also add any
\mailheader{Content-Transfer-Encoding} or other headers to the message
object as necessary.  The default encoding is base64.  See the
\refmodule{email.encoders} module for a list of the built-in encoders.

\var{_params} are passed straight through to the base class constructor.
\versionadded{2.5}
\end{classdesc}

\begin{classdesc}{MIMEAudio}{_audiodata\optional{, _subtype\optional{,
    _encoder\optional{, **_params}}}}
Module: \module{email.mime.audio}

A subclass of \class{MIMENonMultipart}, the \class{MIMEAudio} class
is used to create MIME message objects of major type \mimetype{audio}.
\var{_audiodata} is a string containing the raw audio data.  If this
data can be decoded by the standard Python module \refmodule{sndhdr},
then the subtype will be automatically included in the
\mailheader{Content-Type} header.  Otherwise you can explicitly specify the
audio subtype via the \var{_subtype} parameter.  If the minor type could
not be guessed and \var{_subtype} was not given, then \exception{TypeError}
is raised.

Optional \var{_encoder} is a callable (i.e. function) which will
perform the actual encoding of the audio data for transport.  This
callable takes one argument, which is the \class{MIMEAudio} instance.
It should use \method{get_payload()} and \method{set_payload()} to
change the payload to encoded form.  It should also add any
\mailheader{Content-Transfer-Encoding} or other headers to the message
object as necessary.  The default encoding is base64.  See the
\refmodule{email.encoders} module for a list of the built-in encoders.

\var{_params} are passed straight through to the base class constructor.
\end{classdesc}

\begin{classdesc}{MIMEImage}{_imagedata\optional{, _subtype\optional{,
    _encoder\optional{, **_params}}}}
Module: \module{email.mime.image}

A subclass of \class{MIMENonMultipart}, the \class{MIMEImage} class is
used to create MIME message objects of major type \mimetype{image}.
\var{_imagedata} is a string containing the raw image data.  If this
data can be decoded by the standard Python module \refmodule{imghdr},
then the subtype will be automatically included in the
\mailheader{Content-Type} header.  Otherwise you can explicitly specify the
image subtype via the \var{_subtype} parameter.  If the minor type could
not be guessed and \var{_subtype} was not given, then \exception{TypeError}
is raised.

Optional \var{_encoder} is a callable (i.e. function) which will
perform the actual encoding of the image data for transport.  This
callable takes one argument, which is the \class{MIMEImage} instance.
It should use \method{get_payload()} and \method{set_payload()} to
change the payload to encoded form.  It should also add any
\mailheader{Content-Transfer-Encoding} or other headers to the message
object as necessary.  The default encoding is base64.  See the
\refmodule{email.encoders} module for a list of the built-in encoders.

\var{_params} are passed straight through to the \class{MIMEBase}
constructor.
\end{classdesc}

\begin{classdesc}{MIMEMessage}{_msg\optional{, _subtype}}
Module: \module{email.mime.message}

A subclass of \class{MIMENonMultipart}, the \class{MIMEMessage} class
is used to create MIME objects of main type \mimetype{message}.
\var{_msg} is used as the payload, and must be an instance of class
\class{Message} (or a subclass thereof), otherwise a
\exception{TypeError} is raised.

Optional \var{_subtype} sets the subtype of the message; it defaults
to \mimetype{rfc822}.
\end{classdesc}

\begin{classdesc}{MIMEText}{_text\optional{, _subtype\optional{, _charset}}}
Module: \module{email.mime.text}

A subclass of \class{MIMENonMultipart}, the \class{MIMEText} class is
used to create MIME objects of major type \mimetype{text}.
\var{_text} is the string for the payload.  \var{_subtype} is the
minor type and defaults to \mimetype{plain}.  \var{_charset} is the
character set of the text and is passed as a parameter to the
\class{MIMENonMultipart} constructor; it defaults to \code{us-ascii}.  No
guessing or encoding is performed on the text data.

\versionchanged[The previously deprecated \var{_encoding} argument has
been removed.  Encoding happens implicitly based on the \var{_charset}
argument]{2.4}
\end{classdesc}


\subsection{Internationalized headers}
\declaremodule{standard}{email.Header}
\modulesynopsis{Representing non-ASCII headers}

\rfc{2822} is the base standard that describes the format of email
messages.  It derives from the older \rfc{822} standard which came
into widespread use at a time when most email was composed of \ASCII{}
characters only.  \rfc{2822} is a specification written assuming email
contains only 7-bit \ASCII{} characters.

Of course, as email has been deployed worldwide, it has become
internationalized, such that language specific character sets can now
be used in email messages.  The base standard still requires email
messages to be transfered using only 7-bit \ASCII{} characters, so a
slew of RFCs have been written describing how to encode email
containing non-\ASCII{} characters into \rfc{2822}-compliant format.
These RFCs include \rfc{2045}, \rfc{2046}, \rfc{2047}, and \rfc{2231}.
The \module{email} package supports these standards in its
\module{email.Header} and \module{email.Charset} modules.

If you want to include non-\ASCII{} characters in your email headers,
say in the \mailheader{Subject} or \mailheader{To} fields, you should
use the \class{Header} class and assign the field in the
\class{Message} object to an instance of \class{Header} instead of
using a string for the header value.  For example:

\begin{verbatim}
>>> from email.Message import Message
>>> from email.Header import Header
>>> msg = Message()
>>> h = Header('p\xf6stal', 'iso-8859-1')
>>> msg['Subject'] = h
>>> print msg.as_string()
Subject: =?iso-8859-1?q?p=F6stal?=


\end{verbatim}

Notice here how we wanted the \mailheader{Subject} field to contain a
non-\ASCII{} character?  We did this by creating a \class{Header}
instance and passing in the character set that the byte string was
encoded in.  When the subsequent \class{Message} instance was
flattened, the \mailheader{Subject} field was properly \rfc{2047}
encoded.  MIME-aware mail readers would show this header using the
embedded ISO-8859-1 character.

\versionadded{2.2.2}

Here is the \class{Header} class description:

\begin{classdesc}{Header}{\optional{s\optional{, charset\optional{,
    maxlinelen\optional{, header_name\optional{, continuation_ws\optional{,
    errors}}}}}}}
Create a MIME-compliant header that can contain strings in different
character sets.

Optional \var{s} is the initial header value.  If \code{None} (the
default), the initial header value is not set.  You can later append
to the header with \method{append()} method calls.  \var{s} may be a
byte string or a Unicode string, but see the \method{append()}
documentation for semantics.

Optional \var{charset} serves two purposes: it has the same meaning as
the \var{charset} argument to the \method{append()} method.  It also
sets the default character set for all subsequent \method{append()}
calls that omit the \var{charset} argument.  If \var{charset} is not
provided in the constructor (the default), the \code{us-ascii}
character set is used both as \var{s}'s initial charset and as the
default for subsequent \method{append()} calls.

The maximum line length can be specified explicit via
\var{maxlinelen}.  For splitting the first line to a shorter value (to
account for the field header which isn't included in \var{s},
e.g. \mailheader{Subject}) pass in the name of the field in
\var{header_name}.  The default \var{maxlinelen} is 76, and the
default value for \var{header_name} is \code{None}, meaning it is not
taken into account for the first line of a long, split header.

Optional \var{continuation_ws} must be \rfc{2822}-compliant folding
whitespace, and is usually either a space or a hard tab character.
This character will be prepended to continuation lines.
\end{classdesc}

Optional \var{errors} is passed straight through to the
\method{append()} method.

\begin{methoddesc}[Header]{append}{s\optional{, charset\optional{, errors}}}
Append the string \var{s} to the MIME header.

Optional \var{charset}, if given, should be a \class{Charset} instance
(see \refmodule{email.Charset}) or the name of a character set, which
will be converted to a \class{Charset} instance.  A value of
\code{None} (the default) means that the \var{charset} given in the
constructor is used.

\var{s} may be a byte string or a Unicode string.  If it is a byte
string (i.e. \code{isinstance(s, str)} is true), then
\var{charset} is the encoding of that byte string, and a
\exception{UnicodeError} will be raised if the string cannot be
decoded with that character set.

If \var{s} is a Unicode string, then \var{charset} is a hint
specifying the character set of the characters in the string.  In this
case, when producing an \rfc{2822}-compliant header using \rfc{2047}
rules, the Unicode string will be encoded using the following charsets
in order: \code{us-ascii}, the \var{charset} hint, \code{utf-8}.  The
first character set to not provoke a \exception{UnicodeError} is used.

Optional \var{errors} is passed through to any \function{unicode()} or
\function{ustr.encode()} call, and defaults to ``strict''.
\end{methoddesc}

\begin{methoddesc}[Header]{encode}{}
Encode a message header into an RFC-compliant format, possibly
wrapping long lines and encapsulating non-\ASCII{} parts in base64 or
quoted-printable encodings.
\end{methoddesc}

The \class{Header} class also provides a number of methods to support
standard operators and built-in functions.

\begin{methoddesc}[Header]{__str__}{}
A synonym for \method{Header.encode()}.  Useful for
\code{str(aHeader)}.
\end{methoddesc}

\begin{methoddesc}[Header]{__unicode__}{}
A helper for the built-in \function{unicode()} function.  Returns the
header as a Unicode string.
\end{methoddesc}

\begin{methoddesc}[Header]{__eq__}{other}
This method allows you to compare two \class{Header} instances for equality.
\end{methoddesc}

\begin{methoddesc}[Header]{__ne__}{other}
This method allows you to compare two \class{Header} instances for inequality.
\end{methoddesc}

The \module{email.Header} module also provides the following
convenient functions.

\begin{funcdesc}{decode_header}{header}
Decode a message header value without converting the character set.
The header value is in \var{header}.

This function returns a list of \code{(decoded_string, charset)} pairs
containing each of the decoded parts of the header.  \var{charset} is
\code{None} for non-encoded parts of the header, otherwise a lower
case string containing the name of the character set specified in the
encoded string.

Here's an example:

\begin{verbatim}
>>> from email.Header import decode_header
>>> decode_header('=?iso-8859-1?q?p=F6stal?=')
[('p\\xf6stal', 'iso-8859-1')]
\end{verbatim}
\end{funcdesc}

\begin{funcdesc}{make_header}{decoded_seq\optional{, maxlinelen\optional{,
    header_name\optional{, continuation_ws}}}}
Create a \class{Header} instance from a sequence of pairs as returned
by \function{decode_header()}.

\function{decode_header()} takes a header value string and returns a
sequence of pairs of the format \code{(decoded_string, charset)} where
\var{charset} is the name of the character set.

This function takes one of those sequence of pairs and returns a
\class{Header} instance.  Optional \var{maxlinelen},
\var{header_name}, and \var{continuation_ws} are as in the
\class{Header} constructor.
\end{funcdesc}


\subsection{Representing character sets}
\declaremodule{standard}{email.charset}
\modulesynopsis{Character Sets}

This module provides a class \class{Charset} for representing
character sets and character set conversions in email messages, as
well as a character set registry and several convenience methods for
manipulating this registry.  Instances of \class{Charset} are used in
several other modules within the \module{email} package.

Import this class from the \module{email.charset} module.

\versionadded{2.2.2}

\begin{classdesc}{Charset}{\optional{input_charset}}
Map character sets to their email properties.

This class provides information about the requirements imposed on
email for a specific character set.  It also provides convenience
routines for converting between character sets, given the availability
of the applicable codecs.  Given a character set, it will do its best
to provide information on how to use that character set in an email
message in an RFC-compliant way.

Certain character sets must be encoded with quoted-printable or base64
when used in email headers or bodies.  Certain character sets must be
converted outright, and are not allowed in email.

Optional \var{input_charset} is as described below; it is always
coerced to lower case.  After being alias normalized it is also used
as a lookup into the registry of character sets to find out the header
encoding, body encoding, and output conversion codec to be used for
the character set.  For example, if
\var{input_charset} is \code{iso-8859-1}, then headers and bodies will
be encoded using quoted-printable and no output conversion codec is
necessary.  If \var{input_charset} is \code{euc-jp}, then headers will
be encoded with base64, bodies will not be encoded, but output text
will be converted from the \code{euc-jp} character set to the
\code{iso-2022-jp} character set.
\end{classdesc}

\class{Charset} instances have the following data attributes:

\begin{datadesc}{input_charset}
The initial character set specified.  Common aliases are converted to
their \emph{official} email names (e.g. \code{latin_1} is converted to
\code{iso-8859-1}).  Defaults to 7-bit \code{us-ascii}.
\end{datadesc}

\begin{datadesc}{header_encoding}
If the character set must be encoded before it can be used in an
email header, this attribute will be set to \code{Charset.QP} (for
quoted-printable), \code{Charset.BASE64} (for base64 encoding), or
\code{Charset.SHORTEST} for the shortest of QP or BASE64 encoding.
Otherwise, it will be \code{None}.
\end{datadesc}

\begin{datadesc}{body_encoding}
Same as \var{header_encoding}, but describes the encoding for the
mail message's body, which indeed may be different than the header
encoding.  \code{Charset.SHORTEST} is not allowed for
\var{body_encoding}.
\end{datadesc}

\begin{datadesc}{output_charset}
Some character sets must be converted before they can be used in
email headers or bodies.  If the \var{input_charset} is one of
them, this attribute will contain the name of the character set
output will be converted to.  Otherwise, it will be \code{None}.
\end{datadesc}

\begin{datadesc}{input_codec}
The name of the Python codec used to convert the \var{input_charset} to
Unicode.  If no conversion codec is necessary, this attribute will be
\code{None}.
\end{datadesc}

\begin{datadesc}{output_codec}
The name of the Python codec used to convert Unicode to the
\var{output_charset}.  If no conversion codec is necessary, this
attribute will have the same value as the \var{input_codec}.
\end{datadesc}

\class{Charset} instances also have the following methods:

\begin{methoddesc}[Charset]{get_body_encoding}{}
Return the content transfer encoding used for body encoding.

This is either the string \samp{quoted-printable} or \samp{base64}
depending on the encoding used, or it is a function, in which case you
should call the function with a single argument, the Message object
being encoded.  The function should then set the
\mailheader{Content-Transfer-Encoding} header itself to whatever is
appropriate.

Returns the string \samp{quoted-printable} if
\var{body_encoding} is \code{QP}, returns the string
\samp{base64} if \var{body_encoding} is \code{BASE64}, and returns the
string \samp{7bit} otherwise.
\end{methoddesc}

\begin{methoddesc}{convert}{s}
Convert the string \var{s} from the \var{input_codec} to the
\var{output_codec}.
\end{methoddesc}

\begin{methoddesc}{to_splittable}{s}
Convert a possibly multibyte string to a safely splittable format.
\var{s} is the string to split.

Uses the \var{input_codec} to try and convert the string to Unicode,
so it can be safely split on character boundaries (even for multibyte
characters).

Returns the string as-is if it isn't known how to convert \var{s} to
Unicode with the \var{input_charset}.

Characters that could not be converted to Unicode will be replaced
with the Unicode replacement character \character{U+FFFD}.
\end{methoddesc}

\begin{methoddesc}{from_splittable}{ustr\optional{, to_output}}
Convert a splittable string back into an encoded string.  \var{ustr}
is a Unicode string to ``unsplit''.

This method uses the proper codec to try and convert the string from
Unicode back into an encoded format.  Return the string as-is if it is
not Unicode, or if it could not be converted from Unicode.

Characters that could not be converted from Unicode will be replaced
with an appropriate character (usually \character{?}).

If \var{to_output} is \code{True} (the default), uses
\var{output_codec} to convert to an 
encoded format.  If \var{to_output} is \code{False}, it uses
\var{input_codec}.
\end{methoddesc}

\begin{methoddesc}{get_output_charset}{}
Return the output character set.

This is the \var{output_charset} attribute if that is not \code{None},
otherwise it is \var{input_charset}.
\end{methoddesc}

\begin{methoddesc}{encoded_header_len}{}
Return the length of the encoded header string, properly calculating
for quoted-printable or base64 encoding.
\end{methoddesc}

\begin{methoddesc}{header_encode}{s\optional{, convert}}
Header-encode the string \var{s}.

If \var{convert} is \code{True}, the string will be converted from the
input charset to the output charset automatically.  This is not useful
for multibyte character sets, which have line length issues (multibyte
characters must be split on a character, not a byte boundary); use the
higher-level \class{Header} class to deal with these issues (see
\refmodule{email.header}).  \var{convert} defaults to \code{False}.

The type of encoding (base64 or quoted-printable) will be based on
the \var{header_encoding} attribute.
\end{methoddesc}

\begin{methoddesc}{body_encode}{s\optional{, convert}}
Body-encode the string \var{s}.

If \var{convert} is \code{True} (the default), the string will be
converted from the input charset to output charset automatically.
Unlike \method{header_encode()}, there are no issues with byte
boundaries and multibyte charsets in email bodies, so this is usually
pretty safe.

The type of encoding (base64 or quoted-printable) will be based on
the \var{body_encoding} attribute.
\end{methoddesc}

The \class{Charset} class also provides a number of methods to support
standard operations and built-in functions.

\begin{methoddesc}[Charset]{__str__}{}
Returns \var{input_charset} as a string coerced to lower case.
\method{__repr__()} is an alias for \method{__str__()}.
\end{methoddesc}

\begin{methoddesc}[Charset]{__eq__}{other}
This method allows you to compare two \class{Charset} instances for equality.
\end{methoddesc}

\begin{methoddesc}[Header]{__ne__}{other}
This method allows you to compare two \class{Charset} instances for inequality.
\end{methoddesc}

The \module{email.charset} module also provides the following
functions for adding new entries to the global character set, alias,
and codec registries:

\begin{funcdesc}{add_charset}{charset\optional{, header_enc\optional{,
    body_enc\optional{, output_charset}}}}
Add character properties to the global registry.

\var{charset} is the input character set, and must be the canonical
name of a character set.

Optional \var{header_enc} and \var{body_enc} is either
\code{Charset.QP} for quoted-printable, \code{Charset.BASE64} for
base64 encoding, \code{Charset.SHORTEST} for the shortest of
quoted-printable or base64 encoding, or \code{None} for no encoding.
\code{SHORTEST} is only valid for \var{header_enc}. The default is
\code{None} for no encoding.

Optional \var{output_charset} is the character set that the output
should be in.  Conversions will proceed from input charset, to
Unicode, to the output charset when the method
\method{Charset.convert()} is called.  The default is to output in the
same character set as the input.

Both \var{input_charset} and \var{output_charset} must have Unicode
codec entries in the module's character set-to-codec mapping; use
\function{add_codec()} to add codecs the module does
not know about.  See the \refmodule{codecs} module's documentation for
more information.

The global character set registry is kept in the module global
dictionary \code{CHARSETS}.
\end{funcdesc}

\begin{funcdesc}{add_alias}{alias, canonical}
Add a character set alias.  \var{alias} is the alias name,
e.g. \code{latin-1}.  \var{canonical} is the character set's canonical
name, e.g. \code{iso-8859-1}.

The global charset alias registry is kept in the module global
dictionary \code{ALIASES}.
\end{funcdesc}

\begin{funcdesc}{add_codec}{charset, codecname}
Add a codec that map characters in the given character set to and from
Unicode.

\var{charset} is the canonical name of a character set.
\var{codecname} is the name of a Python codec, as appropriate for the
second argument to the \function{unicode()} built-in, or to the
\method{encode()} method of a Unicode string.
\end{funcdesc}


\subsection{Encoders}
\declaremodule{standard}{email.Encoders}
\modulesynopsis{Encoders for email message payloads.}

When creating \class{Message} objects from scratch, you often need to
encode the payloads for transport through compliant mail servers.
This is especially true for \mimetype{image/*} and \mimetype{text/*}
type messages containing binary data.

The \module{email} package provides some convenient encodings in its
\module{Encoders} module.  These encoders are actually used by the
\class{MIMEAudio} and \class{MIMEImage} class constructors to provide default
encodings.  All encoder functions take exactly one argument, the message
object to encode.  They usually extract the payload, encode it, and reset the
payload to this newly encoded value.  They should also set the
\mailheader{Content-Transfer-Encoding} header as appropriate.

Here are the encoding functions provided:

\begin{funcdesc}{encode_quopri}{msg}
Encodes the payload into quoted-printable form and sets the
\mailheader{Content-Transfer-Encoding} header to
\code{quoted-printable}\footnote{Note that encoding with
\method{encode_quopri()} also encodes all tabs and space characters in
the data.}.
This is a good encoding to use when most of your payload is normal
printable data, but contains a few unprintable characters.
\end{funcdesc}

\begin{funcdesc}{encode_base64}{msg}
Encodes the payload into base64 form and sets the
\mailheader{Content-Transfer-Encoding} header to
\code{base64}.  This is a good encoding to use when most of your payload
is unprintable data since it is a more compact form than
quoted-printable.  The drawback of base64 encoding is that it
renders the text non-human readable.
\end{funcdesc}

\begin{funcdesc}{encode_7or8bit}{msg}
This doesn't actually modify the message's payload, but it does set
the \mailheader{Content-Transfer-Encoding} header to either \code{7bit} or
\code{8bit} as appropriate, based on the payload data.
\end{funcdesc}

\begin{funcdesc}{encode_noop}{msg}
This does nothing; it doesn't even set the
\mailheader{Content-Transfer-Encoding} header.
\end{funcdesc}


\subsection{Exception classes}
\declaremodule{standard}{email.Errors}
\modulesynopsis{The exception classes used by the email package.}

The following exception classes are defined in the
\module{email.Errors} module:

\begin{excclassdesc}{MessageError}{}
This is the base class for all exceptions that the \module{email}
package can raise.  It is derived from the standard
\exception{Exception} class and defines no additional methods.
\end{excclassdesc}

\begin{excclassdesc}{MessageParseError}{}
This is the base class for exceptions thrown by the \class{Parser}
class.  It is derived from \exception{MessageError}.
\end{excclassdesc}

\begin{excclassdesc}{HeaderParseError}{}
Raised under some error conditions when parsing the \rfc{2822} headers of
a message, this class is derived from \exception{MessageParseError}.
It can be raised from the \method{Parser.parse()} or
\method{Parser.parsestr()} methods.

Situations where it can be raised include finding an envelope
header after the first \rfc{2822} header of the message, finding a
continuation line before the first \rfc{2822} header is found, or finding
a line in the headers which is neither a header or a continuation
line.
\end{excclassdesc}

\begin{excclassdesc}{BoundaryError}{}
Raised under some error conditions when parsing the \rfc{2822} headers of
a message, this class is derived from \exception{MessageParseError}.
It can be raised from the \method{Parser.parse()} or
\method{Parser.parsestr()} methods.

Situations where it can be raised include not being able to find the
starting or terminating boundary in a \mimetype{multipart/*} message
when strict parsing is used.
\end{excclassdesc}

\begin{excclassdesc}{MultipartConversionError}{}
Raised when a payload is added to a \class{Message} object using
\method{add_payload()}, but the payload is already a scalar and the
message's \mailheader{Content-Type} main type is not either
\mimetype{multipart} or missing.  \exception{MultipartConversionError}
multiply inherits from \exception{MessageError} and the built-in
\exception{TypeError}.

Since \method{Message.add_payload()} is deprecated, this exception is
rarely raised in practice.  However the exception may also be raised
if the \method{attach()} method is called on an instance of a class
derived from \class{MIMENonMultipart} (e.g. \class{MIMEImage}).
\end{excclassdesc}


\subsection{Miscellaneous utilities}
\declaremodule{standard}{email.Utils}
\modulesynopsis{Miscellaneous email package utilities.}

There are several useful utilities provided in the \module{email.Utils}
module:

\begin{funcdesc}{quote}{str}
Return a new string with backslashes in \var{str} replaced by two
backslashes, and double quotes replaced by backslash-double quote.
\end{funcdesc}

\begin{funcdesc}{unquote}{str}
Return a new string which is an \emph{unquoted} version of \var{str}.
If \var{str} ends and begins with double quotes, they are stripped
off.  Likewise if \var{str} ends and begins with angle brackets, they
are stripped off.
\end{funcdesc}

\begin{funcdesc}{parseaddr}{address}
Parse address -- which should be the value of some address-containing
field such as \mailheader{To} or \mailheader{Cc} -- into its constituent
\emph{realname} and \emph{email address} parts.  Returns a tuple of that
information, unless the parse fails, in which case a 2-tuple of
\code{('', '')} is returned.
\end{funcdesc}

\begin{funcdesc}{formataddr}{pair}
The inverse of \method{parseaddr()}, this takes a 2-tuple of the form
\code{(realname, email_address)} and returns the string value suitable
for a \mailheader{To} or \mailheader{Cc} header.  If the first element of
\var{pair} is false, then the second element is returned unmodified.
\end{funcdesc}

\begin{funcdesc}{getaddresses}{fieldvalues}
This method returns a list of 2-tuples of the form returned by
\code{parseaddr()}.  \var{fieldvalues} is a sequence of header field
values as might be returned by \method{Message.get_all()}.  Here's a
simple example that gets all the recipients of a message:

\begin{verbatim}
from email.Utils import getaddresses

tos = msg.get_all('to', [])
ccs = msg.get_all('cc', [])
resent_tos = msg.get_all('resent-to', [])
resent_ccs = msg.get_all('resent-cc', [])
all_recipients = getaddresses(tos + ccs + resent_tos + resent_ccs)
\end{verbatim}
\end{funcdesc}

\begin{funcdesc}{parsedate}{date}
Attempts to parse a date according to the rules in \rfc{2822}.
however, some mailers don't follow that format as specified, so
\function{parsedate()} tries to guess correctly in such cases. 
\var{date} is a string containing an \rfc{2822} date, such as 
\code{"Mon, 20 Nov 1995 19:12:08 -0500"}.  If it succeeds in parsing
the date, \function{parsedate()} returns a 9-tuple that can be passed
directly to \function{time.mktime()}; otherwise \code{None} will be
returned.  Note that fields 6, 7, and 8 of the result tuple are not
usable.
\end{funcdesc}

\begin{funcdesc}{parsedate_tz}{date}
Performs the same function as \function{parsedate()}, but returns
either \code{None} or a 10-tuple; the first 9 elements make up a tuple
that can be passed directly to \function{time.mktime()}, and the tenth
is the offset of the date's timezone from UTC (which is the official
term for Greenwich Mean Time)\footnote{Note that the sign of the timezone
offset is the opposite of the sign of the \code{time.timezone}
variable for the same timezone; the latter variable follows the
\POSIX{} standard while this module follows \rfc{2822}.}.  If the input
string has no timezone, the last element of the tuple returned is
\code{None}.  Note that fields 6, 7, and 8 of the result tuple are not
usable.
\end{funcdesc}

\begin{funcdesc}{mktime_tz}{tuple}
Turn a 10-tuple as returned by \function{parsedate_tz()} into a UTC
timestamp.  It the timezone item in the tuple is \code{None}, assume
local time.  Minor deficiency: \function{mktime_tz()} interprets the
first 8 elements of \var{tuple} as a local time and then compensates
for the timezone difference.  This may yield a slight error around
changes in daylight savings time, though not worth worrying about for
common use.
\end{funcdesc}

\begin{funcdesc}{formatdate}{\optional{timeval\optional{, localtime}\optional{, usegmt}}}
Returns a date string as per \rfc{2822}, e.g.:

\begin{verbatim}
Fri, 09 Nov 2001 01:08:47 -0000
\end{verbatim}

Optional \var{timeval} if given is a floating point time value as
accepted by \function{time.gmtime()} and \function{time.localtime()},
otherwise the current time is used.

Optional \var{localtime} is a flag that when \code{True}, interprets
\var{timeval}, and returns a date relative to the local timezone
instead of UTC, properly taking daylight savings time into account.
The default is \code{False} meaning UTC is used.

Optional \var{usegmt} is a flag that when \code{True}, outputs a 
date string with the timezone as an ascii string \code{GMT}, rather
than a numeric \code{-0000}. This is needed for some protocols (such
as HTTP). This only applies when \var{localtime} is \code{False}.
\versionadded{2.4}
\end{funcdesc}

\begin{funcdesc}{make_msgid}{\optional{idstring}}
Returns a string suitable for an \rfc{2822}-compliant
\mailheader{Message-ID} header.  Optional \var{idstring} if given, is
a string used to strengthen the uniqueness of the message id.
\end{funcdesc}

\begin{funcdesc}{decode_rfc2231}{s}
Decode the string \var{s} according to \rfc{2231}.
\end{funcdesc}

\begin{funcdesc}{encode_rfc2231}{s\optional{, charset\optional{, language}}}
Encode the string \var{s} according to \rfc{2231}.  Optional
\var{charset} and \var{language}, if given is the character set name
and language name to use.  If neither is given, \var{s} is returned
as-is.  If \var{charset} is given but \var{language} is not, the
string is encoded using the empty string for \var{language}.
\end{funcdesc}

\begin{funcdesc}{collapse_rfc2231_value}{value\optional{, errors\optional{,
    fallback_charset}}}
When a header parameter is encoded in \rfc{2231} format,
\method{Message.get_param()} may return a 3-tuple containing the character
set, language, and value.  \function{collapse_rfc2231_value()} turns this into
a unicode string.  Optional \var{errors} is passed to the \var{errors}
argument of the built-in \function{unicode()} function; it defaults to
\code{replace}.  Optional \var{fallback_charset} specifies the character set
to use if the one in the \rfc{2231} header is not known by Python; it defaults
to \code{us-ascii}.

For convenience, if the \var{value} passed to
\function{collapse_rfc2231_value()} is not a tuple, it should be a string and
it is returned unquoted.
\end{funcdesc}

\begin{funcdesc}{decode_params}{params}
Decode parameters list according to \rfc{2231}.  \var{params} is a
sequence of 2-tuples containing elements of the form
\code{(content-type, string-value)}.
\end{funcdesc}

\versionchanged[The \function{dump_address_pair()} function has been removed;
use \function{formataddr()} instead]{2.4}

\versionchanged[The \function{decode()} function has been removed; use the
\method{Header.decode_header()} method instead]{2.4}

\versionchanged[The \function{encode()} function has been removed; use the
\method{Header.encode()} method instead]{2.4}


\subsection{Iterators}
\section{\module{email.Iterators} ---
         Message object tree iterators}

\declaremodule{standard}{email.Iterators}
\modulesynopsis{Iterate over a  message object tree.}
\sectionauthor{Barry A. Warsaw}{barry@zope.com}

\versionadded{2.2}

Iterating over a message object tree is fairly easy with the
\method{Message.walk()} method.  The \module{email.Iterators} module
provides some useful higher level iterations over message object
trees.

\begin{funcdesc}{body_line_iterator}{msg}
This iterates over all the payloads in all the subparts of \var{msg},
returning the string payloads line-by-line.  It skips over all the
subpart headers, and it skips over any subpart with a payload that
isn't a Python string.  This is somewhat equivalent to reading the
flat text representation of the message from a file using
\method{readline()}, skipping over all the intervening headers.
\end{funcdesc}

\begin{funcdesc}{typed_subpart_iterator}{msg\optional{,
    maintype\optional{, subtype}}}
This iterates over all the subparts of \var{msg}, returning only those
subparts that match the MIME type specified by \var{maintype} and
\var{subtype}.

Note that \var{subtype} is optional; if omitted, then subpart MIME
type matching is done only with the main type.  \var{maintype} is
optional too; it defaults to \code{text}.

Thus, by default \function{typed_subpart_iterator()} returns each
subpart that has a MIME type of \code{text/*}.
\end{funcdesc}



\subsection{Differences from \module{email} v1 (up to Python 2.2.1)}

Version 1 of the \module{email} package was bundled with Python
releases up to Python 2.2.1.  Version 2 was developed for the Python
2.3 release, and backported to Python 2.2.2.  It was also available as
a separate distutils based package.  \module{email} version 2 is
almost entirely backward compatible with version 1, with the
following differences:

\begin{itemize}
\item The \module{email.Header} and \module{email.Charset} modules
      have been added.
\item The pickle format for \class{Message} instances has changed.
      Since this was never (and still isn't) formally defined, this
      isn't considered a backward incompatibility.  However if your
      application pickles and unpickles \class{Message} instances, be
      aware that in \module{email} version 2, \class{Message}
      instances now have private variables \var{_charset} and
      \var{_default_type}.
\item Several methods in the \class{Message} class have been
      deprecated, or their signatures changed.  Also, many new methods
      have been added.  See the documentation for the \class{Message}
      class for details.  The changes should be completely backward
      compatible.
\item The object structure has changed in the face of
      \mimetype{message/rfc822} content types.  In \module{email}
      version 1, such a type would be represented by a scalar payload,
      i.e. the container message's \method{is_multipart()} returned
      false, \method{get_payload()} was not a list object, but a single
      \class{Message} instance.

      This structure was inconsistent with the rest of the package, so
      the object representation for \mimetype{message/rfc822} content
      types was changed.  In \module{email} version 2, the container
      \emph{does} return \code{True} from \method{is_multipart()}, and
      \method{get_payload()} returns a list containing a single
      \class{Message} item.

      Note that this is one place that backward compatibility could
      not be completely maintained.  However, if you're already
      testing the return type of \method{get_payload()}, you should be
      fine.  You just need to make sure your code doesn't do a
      \method{set_payload()} with a \class{Message} instance on a
      container with a content type of \mimetype{message/rfc822}.
\item The \class{Parser} constructor's \var{strict} argument was
      added, and its \method{parse()} and \method{parsestr()} methods
      grew a \var{headersonly} argument.  The \var{strict} flag was
      also added to functions \function{email.message_from_file()}
      and \function{email.message_from_string()}.
\item \method{Generator.__call__()} is deprecated; use
      \method{Generator.flatten()} instead.  The \class{Generator}
      class has also grown the \method{clone()} method.
\item The \class{DecodedGenerator} class in the
      \module{email.Generator} module was added.
\item The intermediate base classes \class{MIMENonMultipart} and
      \class{MIMEMultipart} have been added, and interposed in the
      class hierarchy for most of the other MIME-related derived
      classes.
\item The \var{_encoder} argument to the \class{MIMEText} constructor
      has been deprecated.  Encoding  now happens implicitly based
      on the \var{_charset} argument.
\item The following functions in the \module{email.Utils} module have
      been deprecated: \function{dump_address_pairs()},
      \function{decode()}, and \function{encode()}.  The following
      functions have been added to the module:
      \function{make_msgid()}, \function{decode_rfc2231()},
      \function{encode_rfc2231()}, and \function{decode_params()}.
\item The non-public function \function{email.Iterators._structure()}
      was added.
\end{itemize}

\subsection{Differences from \module{mimelib}}

The \module{email} package was originally prototyped as a separate
library called
\ulink{\module{mimelib}}{http://mimelib.sf.net/}.
Changes have been made so that
method names are more consistent, and some methods or modules have
either been added or removed.  The semantics of some of the methods
have also changed.  For the most part, any functionality available in
\module{mimelib} is still available in the \refmodule{email} package,
albeit often in a different way.  Backward compatibility between
the \module{mimelib} package and the \module{email} package was not a
priority.

Here is a brief description of the differences between the
\module{mimelib} and the \refmodule{email} packages, along with hints on
how to port your applications.

Of course, the most visible difference between the two packages is
that the package name has been changed to \refmodule{email}.  In
addition, the top-level package has the following differences:

\begin{itemize}
\item \function{messageFromString()} has been renamed to
      \function{message_from_string()}.
\item \function{messageFromFile()} has been renamed to
      \function{message_from_file()}.
\end{itemize}

The \class{Message} class has the following differences:

\begin{itemize}
\item The method \method{asString()} was renamed to \method{as_string()}.
\item The method \method{ismultipart()} was renamed to
      \method{is_multipart()}.
\item The \method{get_payload()} method has grown a \var{decode}
      optional argument.
\item The method \method{getall()} was renamed to \method{get_all()}.
\item The method \method{addheader()} was renamed to \method{add_header()}.
\item The method \method{gettype()} was renamed to \method{get_type()}.
\item The method\method{getmaintype()} was renamed to
      \method{get_main_type()}.
\item The method \method{getsubtype()} was renamed to
      \method{get_subtype()}.
\item The method \method{getparams()} was renamed to
      \method{get_params()}.
      Also, whereas \method{getparams()} returned a list of strings,
      \method{get_params()} returns a list of 2-tuples, effectively
      the key/value pairs of the parameters, split on the \character{=}
      sign.
\item The method \method{getparam()} was renamed to \method{get_param()}.
\item The method \method{getcharsets()} was renamed to
      \method{get_charsets()}.
\item The method \method{getfilename()} was renamed to
      \method{get_filename()}.
\item The method \method{getboundary()} was renamed to
      \method{get_boundary()}.
\item The method \method{setboundary()} was renamed to
      \method{set_boundary()}.
\item The method \method{getdecodedpayload()} was removed.  To get
      similar functionality, pass the value 1 to the \var{decode} flag
      of the {get_payload()} method.
\item The method \method{getpayloadastext()} was removed.  Similar
      functionality
      is supported by the \class{DecodedGenerator} class in the
      \refmodule{email.Generator} module.
\item The method \method{getbodyastext()} was removed.  You can get
      similar functionality by creating an iterator with
      \function{typed_subpart_iterator()} in the
      \refmodule{email.Iterators} module.
\end{itemize}

The \class{Parser} class has no differences in its public interface.
It does have some additional smarts to recognize
\mimetype{message/delivery-status} type messages, which it represents as
a \class{Message} instance containing separate \class{Message}
subparts for each header block in the delivery status
notification\footnote{Delivery Status Notifications (DSN) are defined
in \rfc{1894}.}.

The \class{Generator} class has no differences in its public
interface.  There is a new class in the \refmodule{email.Generator}
module though, called \class{DecodedGenerator} which provides most of
the functionality previously available in the
\method{Message.getpayloadastext()} method.

The following modules and classes have been changed:

\begin{itemize}
\item The \class{MIMEBase} class constructor arguments \var{_major}
      and \var{_minor} have changed to \var{_maintype} and
      \var{_subtype} respectively.
\item The \code{Image} class/module has been renamed to
      \code{MIMEImage}.  The \var{_minor} argument has been renamed to
      \var{_subtype}.
\item The \code{Text} class/module has been renamed to
      \code{MIMEText}.  The \var{_minor} argument has been renamed to
      \var{_subtype}.
\item The \code{MessageRFC822} class/module has been renamed to
      \code{MIMEMessage}.  Note that an earlier version of
      \module{mimelib} called this class/module \code{RFC822}, but
      that clashed with the Python standard library module
      \refmodule{rfc822} on some case-insensitive file systems.

      Also, the \class{MIMEMessage} class now represents any kind of
      MIME message with main type \mimetype{message}.  It takes an
      optional argument \var{_subtype} which is used to set the MIME
      subtype.  \var{_subtype} defaults to \mimetype{rfc822}.
\end{itemize}

\module{mimelib} provided some utility functions in its
\module{address} and \module{date} modules.  All of these functions
have been moved to the \refmodule{email.Utils} module.

The \code{MsgReader} class/module has been removed.  Its functionality
is most closely supported in the \function{body_line_iterator()}
function in the \refmodule{email.Iterators} module.

\subsection{Examples}

Here are a few examples of how to use the \module{email} package to
read, write, and send simple email messages, as well as more complex
MIME messages.

First, let's see how to create and send a simple text message:

\begin{verbatim}
# Import smtplib for the actual sending function
import smtplib

# Here are the email pacakge modules we'll need
from email import Encoders
from email.MIMEText import MIMEText

# Open a plain text file for reading
fp = open(textfile)
# Create a text/plain message, using Quoted-Printable encoding for non-ASCII
# characters.
msg = MIMEText(fp.read(), _encoder=Encoders.encode_quopri)
fp.close()

# me == the sender's email address
# you == the recipient's email address
msg['Subject'] = 'The contents of %s' % textfile
msg['From'] = me
msg['To'] = you

# Send the message via our own SMTP server.  Use msg.as_string() with
# unixfrom=0 so as not to confuse SMTP.
s = smtplib.SMTP()
s.connect()
s.sendmail(me, [you], msg.as_string(0))
s.close()
\end{verbatim}

Here's an example of how to send a MIME message containing a bunch of
family pictures:

\begin{verbatim}
# Import smtplib for the actual sending function
import smtplib

# Here are the email pacakge modules we'll need
from email.MIMEImage import MIMEImage
from email.MIMEBase import MIMEBase

COMMASPACE = ', '

# Create the container (outer) email message.
# me == the sender's email address
# family = the list of all recipients' email addresses
msg = MIMEBase('multipart', 'mixed')
msg['Subject'] = 'Our family reunion'
msg['From'] = me
msg['To'] = COMMASPACE.join(family)
msg.preamble = 'Our family reunion'
# Guarantees the message ends in a newline
msg.epilogue = ''

# Assume we know that the image files are all in PNG format
for file in pngfiles:
    # Open the files in binary mode.  Let the MIMEIMage class automatically
    # guess the specific image type.
    fp = open(file, 'rb')
    img = MIMEImage(fp.read())
    fp.close()
    msg.attach(img)

# Send the email via our own SMTP server.
s = smtplib.SMTP()
s.connect()
s.sendmail(me, family, msg.as_string(unixfrom=0))
s.close()
\end{verbatim}

Here's an example\footnote{Thanks to Matthew Dixon Cowles for the
original inspiration and examples.} of how to send the entire contents
of a directory as an email message:

\begin{verbatim}
#!/usr/bin/env python

"""Send the contents of a directory as a MIME message.

Usage: dirmail [options] from to [to ...]*

Options:
    -h / --help
        Print this message and exit.

    -d directory
    --directory=directory
        Mail the contents of the specified directory, otherwise use the
        current directory.  Only the regular files in the directory are sent,
        and we don't recurse to subdirectories.

`from' is the email address of the sender of the message.

`to' is the email address of the recipient of the message, and multiple
recipients may be given.

The email is sent by forwarding to your local SMTP server, which then does the
normal delivery process.  Your local machine must be running an SMTP server.
"""

import sys
import os
import getopt
import smtplib
# For guessing MIME type based on file name extension
import mimetypes

from email import Encoders
from email.Message import Message
from email.MIMEAudio import MIMEAudio
from email.MIMEBase import MIMEBase
from email.MIMEImage import MIMEImage
from email.MIMEText import MIMEText

COMMASPACE = ', '


def usage(code, msg=''):
    print >> sys.stderr, __doc__
    if msg:
        print >> sys.stderr, msg
    sys.exit(code)


def main():
    try:
        opts, args = getopt.getopt(sys.argv[1:], 'hd:', ['help', 'directory='])
    except getopt.error, msg:
        usage(1, msg)

    dir = os.curdir
    for opt, arg in opts:
        if opt in ('-h', '--help'):
            usage(0)
        elif opt in ('-d', '--directory'):
            dir = arg

    if len(args) < 2:
        usage(1)

    sender = args[0]
    recips = args[1:]
    
    # Create the enclosing (outer) message
    outer = MIMEBase('multipart', 'mixed')
    outer['Subject'] = 'Contents of directory %s' % os.path.abspath(dir)
    outer['To'] = COMMASPACE.join(recips)
    outer['From'] = sender
    outer.preamble = 'You will not see this in a MIME-aware mail reader.\n'
    # To guarantee the message ends with a newline
    outer.epilogue = ''

    for filename in os.listdir(dir):
        path = os.path.join(dir, filename)
        if not os.path.isfile(path):
            continue
        # Guess the Content-Type: based on the file's extension.  Encoding
        # will be ignored, although we should check for simple things like
        # gzip'd or compressed files
        ctype, encoding = mimetypes.guess_type(path)
        if ctype is None or encoding is not None:
            # No guess could be made, or the file is encoded (compressed), so
            # use a generic bag-of-bits type.
            ctype = 'application/octet-stream'
        maintype, subtype = ctype.split('/', 1)
        if maintype == 'text':
            fp = open(path)
            # Note: we should handle calculating the charset
            msg = MIMEText(fp.read(), _subtype=subtype)
            fp.close()
        elif maintype == 'image':
            fp = open(path, 'rb')
            msg = MIMEImage(fp.read(), _subtype=subtype)
            fp.close()
        elif maintype == 'audio':
            fp = open(path, 'rb')
            msg = MIMEAudio(fp.read(), _subtype=subtype)
            fp.close()
        else:
            fp = open(path, 'rb')
            msg = MIMEBase(maintype, subtype)
            msg.add_payload(fp.read())
            fp.close()
            # Encode the payload using Base64
            Encoders.encode_base64(msg)
        # Set the filename parameter
        msg.add_header('Content-Disposition', 'attachment', filename=filename)
        outer.attach(msg)

    fp = open('/tmp/debug.pck', 'w')
    import cPickle
    cPickle.dump(outer, fp)
    fp.close()
    # Now send the message
    s = smtplib.SMTP()
    s.connect()
    s.sendmail(sender, recips, outer.as_string(0))
    s.close()


if __name__ == '__main__':
    main()
\end{verbatim}

And finally, here's an example of how to unpack a MIME message like
the one above, into a directory of files:

\begin{verbatim}
#!/usr/bin/env python

"""Unpack a MIME message into a directory of files.

Usage: unpackmail [options] msgfile

Options:
    -h / --help
        Print this message and exit.

    -d directory
    --directory=directory
        Unpack the MIME message into the named directory, which will be
        created if it doesn't already exist.

msgfile is the path to the file containing the MIME message.
"""

import sys
import os
import getopt
import errno
import mimetypes
import email


def usage(code, msg=''):
    print >> sys.stderr, __doc__
    if msg:
        print >> sys.stderr, msg
    sys.exit(code)


def main():
    try:
        opts, args = getopt.getopt(sys.argv[1:], 'hd:', ['help', 'directory='])
    except getopt.error, msg:
        usage(1, msg)

    dir = os.curdir
    for opt, arg in opts:
        if opt in ('-h', '--help'):
            usage(0)
        elif opt in ('-d', '--directory'):
            dir = arg

    try:
        msgfile = args[0]
    except IndexError:
        usage(1)

    try:
        os.mkdir(dir)
    except OSError, e:
        # Ignore directory exists error
        if e.errno <> errno.EEXIST: raise

    fp = open(msgfile)
    msg = email.message_from_file(fp)
    fp.close()

    counter = 1
    for part in msg.walk():
        # multipart/* are just containers
        if part.get_main_type() == 'multipart':
            continue
        # Applications should really sanitize the given filename so that an
        # email message can't be used to overwrite important files
        filename = part.get_filename()
        if not filename:
            ext = mimetypes.guess_extension(part.get_type())
            if not ext:
                # Use a generic bag-of-bits extension
                ext = '.bin'
            filename = 'part-%03d%s' % (counter, ext)
        counter += 1
        fp = open(os.path.join(dir, filename), 'wb')
        fp.write(part.get_payload(decode=1))
        fp.close()


if __name__ == '__main__':
    main()
\end{verbatim}


\end{document}
