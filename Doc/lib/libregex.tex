\section{Built-in Module \sectcode{regex}}

\bimodindex{regex}
This module provides regular expression matching operations similar to
those found in Emacs.  It is always available.

By default the patterns are Emacs-style regular expressions,
with one exception.  There is
a way to change the syntax to match that of several well-known
\UNIX{} utilities.  The exception is that Emacs' \samp{\e s}
pattern is not supported, since the original implementation references
the Emacs syntax tables.

This module is 8-bit clean: both patterns and strings may contain null
bytes and characters whose high bit is set.

\strong{Please note:} There is a little-known fact about Python string
literals which means that you don't usually have to worry about
doubling backslashes, even though they are used to escape special
characters in string literals as well as in regular expressions.  This
is because Python doesn't remove backslashes from string literals if
they are followed by an unrecognized escape character.
\emph{However}, if you want to include a literal \dfn{backslash} in a
regular expression represented as a string literal, you have to
\emph{quadruple} it.  E.g.\  to extract \LaTeX\ \samp{\e section\{{\rm
\ldots}\}} headers from a document, you can use this pattern:
\code{'\e \e \e\e section\{\e (.*\e )\}'}.

The module defines these functions, and an exception:

\renewcommand{\indexsubitem}{(in module regex)}

\begin{funcdesc}{match}{pattern\, string}
  Return how many characters at the beginning of \var{string} match
  the regular expression \var{pattern}.  Return \code{-1} if the
  string does not match the pattern (this is different from a
  zero-length match!).
\end{funcdesc}

\begin{funcdesc}{search}{pattern\, string}
  Return the first position in \var{string} that matches the regular
  expression \var{pattern}.  Return -1 if no position in the string
  matches the pattern (this is different from a zero-length match
  anywhere!).
\end{funcdesc}

\begin{funcdesc}{compile}{pattern\optional{\, translate}}
  Compile a regular expression pattern into a regular expression
  object, which can be used for matching using its \code{match} and
  \code{search} methods, described below.  The optional argument
  \var{translate}, if present, must be a 256-character string
  indicating how characters (both of the pattern and of the strings to
  be matched) are translated before comparing them; the \code{i}-th
  element of the string gives the translation for the character with
  \ASCII{} code \code{i}.  This can be used to implement
  case-insensitive matching; see the \code{casefold} data item below.

  The sequence

\bcode\begin{verbatim}
prog = regex.compile(pat)
result = prog.match(str)
\end{verbatim}\ecode

is equivalent to

\bcode\begin{verbatim}
result = regex.match(pat, str)
\end{verbatim}\ecode

but the version using \code{compile()} is more efficient when multiple
regular expressions are used concurrently in a single program.  (The
compiled version of the last pattern passed to \code{regex.match()} or
\code{regex.search()} is cached, so programs that use only a single
regular expression at a time needn't worry about compiling regular
expressions.)
\end{funcdesc}

\begin{funcdesc}{set_syntax}{flags}
  Set the syntax to be used by future calls to \code{compile},
  \code{match} and \code{search}.  (Already compiled expression objects
  are not affected.)  The argument is an integer which is the OR of
  several flag bits.  The return value is the previous value of
  the syntax flags.  Names for the flags are defined in the standard
  module \code{regex_syntax}; read the file \file{regex_syntax.py} for
  more information.
\end{funcdesc}

\begin{funcdesc}{symcomp}{pattern\optional{\, translate}}
This is like \code{compile}, but supports symbolic group names: if a
parenthesis-enclosed group begins with a group name in angular
brackets, e.g. \code{'\e(<id>[a-z][a-z0-9]*\e)'}, the group can
be referenced by its name in arguments to the \code{group} method of
the resulting compiled regular expression object, like this:
\code{p.group('id')}.  Group names may contain alphanumeric characters
and \code{'_'} only.
\end{funcdesc}

\begin{excdesc}{error}
  Exception raised when a string passed to one of the functions here
  is not a valid regular expression (e.g., unmatched parentheses) or
  when some other error occurs during compilation or matching.  (It is
  never an error if a string contains no match for a pattern.)
\end{excdesc}

\begin{datadesc}{casefold}
A string suitable to pass as \var{translate} argument to
\code{compile} to map all upper case characters to their lowercase
equivalents.
\end{datadesc}

\noindent
Compiled regular expression objects support these methods:

\renewcommand{\indexsubitem}{(regex method)}
\begin{funcdesc}{match}{string\optional{\, pos}}
  Return how many characters at the beginning of \var{string} match
  the compiled regular expression.  Return \code{-1} if the string
  does not match the pattern (this is different from a zero-length
  match!).
  
  The optional second parameter \var{pos} gives an index in the string
  where the search is to start; it defaults to \code{0}.  This is not
  completely equivalent to slicing the string; the \code{'\^'} pattern
  character matches at the real begin of the string and at positions
  just after a newline, not necessarily at the index where the search
  is to start.
\end{funcdesc}

\begin{funcdesc}{search}{string\optional{\, pos}}
  Return the first position in \var{string} that matches the regular
  expression \code{pattern}.  Return \code{-1} if no position in the
  string matches the pattern (this is different from a zero-length
  match anywhere!).
  
  The optional second parameter has the same meaning as for the
  \code{match} method.
\end{funcdesc}

\begin{funcdesc}{group}{index\, index\, ...}
This method is only valid when the last call to the \code{match}
or \code{search} method found a match.  It returns one or more
groups of the match.  If there is a single \var{index} argument,
the result is a single string; if there are multiple arguments, the
result is a tuple with one item per argument.  If the \var{index} is
zero, the corresponding return value is the entire matching string; if
it is in the inclusive range [1..99], it is the string matching the
the corresponding parenthesized group (using the default syntax,
groups are parenthesized using \code{\\(} and \code{\\)}).  If no
such group exists, the corresponding result is \code{None}.

If the regular expression was compiled by \code{symcomp} instead of
\code{compile}, the \var{index} arguments may also be strings
identifying groups by their group name.
\end{funcdesc}

\noindent
Compiled regular expressions support these data attributes:

\renewcommand{\indexsubitem}{(regex attribute)}

\begin{datadesc}{regs}
When the last call to the \code{match} or \code{search} method found a
match, this is a tuple of pairs of indices corresponding to the
beginning and end of all parenthesized groups in the pattern.  Indices
are relative to the string argument passed to \code{match} or
\code{search}.  The 0-th tuple gives the beginning and end or the
whole pattern.  When the last match or search failed, this is
\code{None}.
\end{datadesc}

\begin{datadesc}{last}
When the last call to the \code{match} or \code{search} method found a
match, this is the string argument passed to that method.  When the
last match or search failed, this is \code{None}.
\end{datadesc}

\begin{datadesc}{translate}
This is the value of the \var{translate} argument to
\code{regex.compile} that created this regular expression object.  If
the \var{translate} argument was omitted in the \code{regex.compile}
call, this is \code{None}.
\end{datadesc}

\begin{datadesc}{givenpat}
The regular expression pattern as passed to \code{compile} or
\code{symcomp}.
\end{datadesc}

\begin{datadesc}{realpat}
The regular expression after stripping the group names for regular
expressions compiled with \code{symcomp}.  Same as \code{givenpat}
otherwise.
\end{datadesc}

\begin{datadesc}{groupindex}
A dictionary giving the mapping from symbolic group names to numerical
group indices for regular expressions compiled with \code{symcomp}.
\code{None} otherwise.
\end{datadesc}
