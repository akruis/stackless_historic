\declaremodule{standard}{email.Encoders}
\modulesynopsis{Encoders for email message payloads.}

When creating \class{Message} objects from scratch, you often need to
encode the payloads for transport through compliant mail servers.
This is especially true for \mimetype{image/*} and \mimetype{text/*}
type messages containing binary data.

The \module{email} package provides some convenient encodings in its
\module{Encoders} module.  These encoders are actually used by the
\class{MIMEAudio} and \class{MIMEImage} class constructors to provide default
encodings.  All encoder functions take exactly one argument, the message
object to encode.  They usually extract the payload, encode it, and reset the
payload to this newly encoded value.  They should also set the
\mailheader{Content-Transfer-Encoding} header as appropriate.

Here are the encoding functions provided:

\begin{funcdesc}{encode_quopri}{msg}
Encodes the payload into quoted-printable form and sets the
\mailheader{Content-Transfer-Encoding} header to
\code{quoted-printable}\footnote{Note that encoding with
\method{encode_quopri()} also encodes all tabs and space characters in
the data.}.
This is a good encoding to use when most of your payload is normal
printable data, but contains a few unprintable characters.
\end{funcdesc}

\begin{funcdesc}{encode_base64}{msg}
Encodes the payload into base64 form and sets the
\mailheader{Content-Transfer-Encoding} header to
\code{base64}.  This is a good encoding to use when most of your payload
is unprintable data since it is a more compact form than
quoted-printable.  The drawback of base64 encoding is that it
renders the text non-human readable.
\end{funcdesc}

\begin{funcdesc}{encode_7or8bit}{msg}
This doesn't actually modify the message's payload, but it does set
the \mailheader{Content-Transfer-Encoding} header to either \code{7bit} or
\code{8bit} as appropriate, based on the payload data.
\end{funcdesc}

\begin{funcdesc}{encode_noop}{msg}
This does nothing; it doesn't even set the
\mailheader{Content-Transfer-Encoding} header.
\end{funcdesc}
