\section{\module{codecs} ---
         Python codec registry and base classes}

\declaremodule{standard}{codec}
\modulesynopsis{Encode and decode data and streams.}
\moduleauthor{Marc-Andre Lemburg}{mal@lemburg.com}
\sectionauthor{Marc-Andre Lemburg}{mal@lemburg.com}


\index{Unicode}
\index{Codecs}
\indexii{Codecs}{encode}
\indexii{Codecs}{decode}
\index{streams}
\indexii{stackable}{streams}


This module defines base classes for standard Python codecs (encoders
and decoders) and provides access to the internal Python codec
registry which manages the codec lookup process.

It defines the following functions:

\begin{funcdesc}{register}{search_function}
Register a codec search function. Search functions are expected to
take one argument, the encoding name in all lower case letters, and
return a tuple of functions \code{(\var{encoder}, \var{decoder}, \var{stream_reader},
\var{stream_writer})} taking the following arguments:

  \var{encoder} and \var{decoder}: These must be functions or methods
  which have the same interface as the .encode/.decode methods of
  Codec instances (see Codec Interface). The functions/methods are
  expected to work in a stateless mode.

  \var{stream_reader} and \var{stream_writer}: These have to be
  factory functions providing the following interface:

	\code{factory(\var{stream},\var{errors}='strict')}

  The factory functions must return objects providing the interfaces
  defined by the base classes
  \class{StreamWriter}/\class{StreamReader} resp. Stream codecs can
  maintain state.

  Possible values for errors are 'strict' (raise an exception in case
  of an encoding error), 'replace' (replace malformed data with a
  suitable replacement marker, e.g. '?') and 'ignore' (ignore
  malformed data and continue without further notice).

In case a search function cannot find a given encoding, it should
return None.
\end{funcdesc}

\begin{funcdesc}{lookup}{encoding}
Looks up a codec tuple in the Python codec registry and returns the
function tuple as defined above.

Encodings are first looked up in the registry's cache. If not found,
the list of registered search functions is scanned. If no codecs tuple
is found, a LookupError is raised. Otherwise, the codecs tuple is
stored in the cache and returned to the caller.
\end{funcdesc}

To simplify working with encoded files or stream, the module
also defines these utility functions:

\begin{funcdesc}{open}{filename, mode\optional{, encoding=None, errors='strict', buffering=1}}
Open an encoded file using the given \var{mode} and return
a wrapped version providing transparent encoding/decoding.

Note: The wrapped version will only accept the object format defined
by the codecs, i.e. Unicode objects for most builtin codecs. Output is
also codec dependent and will usually by Unicode as well.

\var{encoding} specifies the encoding which is to be used for the
the file.

\var{errors} may be given to define the error handling. It defaults
to 'strict' which causes a \exception{ValueError} to be raised in case
an encoding error occurs.

\var{buffering} has the same meaning as for the builtin open() API.
It defaults to line buffered.
\end{funcdesc}

\begin{funcdesc}{EncodedFile}{file, input\optional{, output=None, errors='strict'}}

Return a wrapped version of file which provides transparent
encoding translation.

Strings written to the wrapped file are interpreted according to the
given \var{input} encoding and then written to the original file as
string using the \var{output} encoding. The intermediate encoding will
usually be Unicode but depends on the specified codecs.

If \var{output} is not given, it defaults to input.

\var{errors} may be given to define the error handling. It defaults to
'strict' which causes \exception{ValueError} to be raised in case
an encoding error occurs.
\end{funcdesc}



...XXX document codec base classes...



The module also provides the following constants which are useful
for reading and writing to platform dependent files:

\begin{datadesc}{BOM}
\dataline{BOM_BE}
\dataline{BOM_LE}
\dataline{BOM32_BE}
\dataline{BOM32_LE}
\dataline{BOM64_BE}
\dataline{BOM64_LE}
These constants define the byte order marks (BOM) used in data
streams to indicate the byte order used in the stream or file.
\constant{BOM} is either \constant{BOM_BE} or \constant{BOM_LE}
depending on the platform's native byte order, while the others
represent big endian (\samp{_BE} suffix) and little endian
(\samp{_LE} suffix) byte order using 32-bit and 64-bit encodings.
\end{datadesc}

