\documentclass{howto}

% TODO:
% Document lookbehind assertions
% Better way of displaying a RE, a string, and what it matches
% Mention optional argument to match.groups()
% Unicode (at least a reference)

\title{Regular Expression HOWTO}

\release{0.05}

\author{A.M. Kuchling}
\authoraddress{\email{amk@amk.ca}}

\begin{document}
\maketitle

\begin{abstract}
\noindent
This document is an introductory tutorial to using regular expressions
in Python with the \module{re} module.  It provides a gentler
introduction than the corresponding section in the Library Reference.

This document is available from 
\url{http://www.amk.ca/python/howto}.

\end{abstract}

\tableofcontents

\section{Introduction}

The \module{re} module was added in Python 1.5, and provides
Perl-style regular expression patterns.  Earlier versions of Python
came with the \module{regex} module, which provided Emacs-style
patterns.  \module{regex} module was removed in Python 2.5.

Regular expressions (or REs) are essentially a tiny, highly
specialized programming language embedded inside Python and made
available through the \module{re} module.  Using this little language,
you specify the rules for the set of possible strings that you want to
match; this set might contain English sentences, or e-mail addresses,
or TeX commands, or anything you like.  You can then ask questions
such as ``Does this string match the pattern?'', or ``Is there a match
for the pattern anywhere in this string?''.  You can also use REs to
modify a string or to split it apart in various ways.

Regular expression patterns are compiled into a series of bytecodes
which are then executed by a matching engine written in C.  For
advanced use, it may be necessary to pay careful attention to how the
engine will execute a given RE, and write the RE in a certain way in
order to produce bytecode that runs faster.  Optimization isn't
covered in this document, because it requires that you have a good
understanding of the matching engine's internals.

The regular expression language is relatively small and restricted, so
not all possible string processing tasks can be done using regular
expressions.  There are also tasks that \emph{can} be done with
regular expressions, but the expressions turn out to be very
complicated.  In these cases, you may be better off writing Python
code to do the processing; while Python code will be slower than an
elaborate regular expression, it will also probably be more understandable.

\section{Simple Patterns}

We'll start by learning about the simplest possible regular
expressions.  Since regular expressions are used to operate on
strings, we'll begin with the most common task: matching characters.

For a detailed explanation of the computer science underlying regular
expressions (deterministic and non-deterministic finite automata), you
can refer to almost any textbook on writing compilers.

\subsection{Matching Characters}

Most letters and characters will simply match themselves.  For
example, the regular expression \regexp{test} will match the string
\samp{test} exactly.  (You can enable a case-insensitive mode that
would let this RE match \samp{Test} or \samp{TEST} as well; more
about this later.)  

There are exceptions to this rule; some characters are
special, and don't match themselves.  Instead, they signal that some
out-of-the-ordinary thing should be matched, or they affect other
portions of the RE by repeating them.  Much of this document is
devoted to discussing various metacharacters and what they do.

Here's a complete list of the metacharacters; their meanings will be
discussed in the rest of this HOWTO.

\begin{verbatim}
. ^ $ * + ? { [ ] \ | ( )
\end{verbatim}
% $

The first metacharacters we'll look at are \samp{[} and \samp{]}.
They're used for specifying a character class, which is a set of
characters that you wish to match.  Characters can be listed
individually, or a range of characters can be indicated by giving two
characters and separating them by a \character{-}.  For example,
\regexp{[abc]} will match any of the characters \samp{a}, \samp{b}, or
\samp{c}; this is the same as
\regexp{[a-c]}, which uses a range to express the same set of
characters.  If you wanted to match only lowercase letters, your
RE would be \regexp{[a-z]}.

Metacharacters are not active inside classes.  For example,
\regexp{[akm\$]} will match any of the characters \character{a},
\character{k}, \character{m}, or \character{\$}; \character{\$} is
usually a metacharacter, but inside a character class it's stripped of
its special nature.

You can match the characters not within a range by \dfn{complementing}
the set.  This is indicated by including a \character{\^} as the first
character of the class; \character{\^} elsewhere will simply match the
\character{\^} character.  For example, \verb|[^5]| will match any
character except \character{5}.

Perhaps the most important metacharacter is the backslash, \samp{\e}.  
As in Python string literals, the backslash can be followed by various
characters to signal various special sequences.  It's also used to escape
all the metacharacters so you can still match them in patterns; for
example, if you need to match a \samp{[} or 
\samp{\e}, you can precede them with a backslash to remove their
special meaning: \regexp{\e[} or \regexp{\e\e}.

Some of the special sequences beginning with \character{\e} represent
predefined sets of characters that are often useful, such as the set
of digits, the set of letters, or the set of anything that isn't
whitespace.  The following predefined special sequences are available:

\begin{itemize}
\item[\code{\e d}]Matches any decimal digit; this is
equivalent to the class \regexp{[0-9]}.

\item[\code{\e D}]Matches any non-digit character; this is
equivalent to the class \verb|[^0-9]|.

\item[\code{\e s}]Matches any whitespace character; this is
equivalent to the class \regexp{[ \e t\e n\e r\e f\e v]}.

\item[\code{\e S}]Matches any non-whitespace character; this is
equivalent to the class \verb|[^ \t\n\r\f\v]|.

\item[\code{\e w}]Matches any alphanumeric character; this is equivalent to the class
\regexp{[a-zA-Z0-9_]}.  

\item[\code{\e W}]Matches any non-alphanumeric character; this is equivalent to the class
\verb|[^a-zA-Z0-9_]|.   
\end{itemize}

These sequences can be included inside a character class.  For
example, \regexp{[\e s,.]} is a character class that will match any
whitespace character, or \character{,} or \character{.}.

The final metacharacter in this section is \regexp{.}.  It matches
anything except a newline character, and there's an alternate mode
(\code{re.DOTALL}) where it will match even a newline.  \character{.}
is often used where you want to match ``any character''.  

\subsection{Repeating Things}

Being able to match varying sets of characters is the first thing
regular expressions can do that isn't already possible with the
methods available on strings.  However, if that was the only
additional capability of regexes, they wouldn't be much of an advance.
Another capability is that you can specify that portions of the RE
must be repeated a certain number of times.

The first metacharacter for repeating things that we'll look at is
\regexp{*}.  \regexp{*} doesn't match the literal character \samp{*};
instead, it specifies that the previous character can be matched zero
or more times, instead of exactly once.

For example, \regexp{ca*t} will match \samp{ct} (0 \samp{a}
characters), \samp{cat} (1 \samp{a}), \samp{caaat} (3 \samp{a}
characters), and so forth.  The RE engine has various internal
limitations stemming from the size of C's \code{int} type, that will
prevent it from matching over 2 billion \samp{a} characters; you
probably don't have enough memory to construct a string that large, so
you shouldn't run into that limit.

Repetitions such as \regexp{*} are \dfn{greedy}; when repeating a RE,
the matching engine will try to repeat it as many times as possible.
If later portions of the pattern don't match, the matching engine will
then back up and try again with few repetitions.

A step-by-step example will make this more obvious.  Let's consider
the expression \regexp{a[bcd]*b}.  This matches the letter
\character{a}, zero or more letters from the class \code{[bcd]}, and
finally ends with a \character{b}.  Now imagine matching this RE
against the string \samp{abcbd}.  

\begin{tableiii}{c|l|l}{}{Step}{Matched}{Explanation}
\lineiii{1}{\code{a}}{The \regexp{a} in the RE matches.}
\lineiii{2}{\code{abcbd}}{The engine matches \regexp{[bcd]*}, going as far as
it can, which is to the end of the string.}
\lineiii{3}{\emph{Failure}}{The engine tries to match \regexp{b}, but the
current position is at the end of the string, so it fails.}
\lineiii{4}{\code{abcb}}{Back up, so that  \regexp{[bcd]*} matches
one less character.}
\lineiii{5}{\emph{Failure}}{Try \regexp{b} again, but the
current position is at the last character, which is a \character{d}.}
\lineiii{6}{\code{abc}}{Back up again, so that  \regexp{[bcd]*} is
only matching \samp{bc}.}
\lineiii{6}{\code{abcb}}{Try \regexp{b} again.  This time 
but the character at the current position is \character{b}, so it succeeds.}
\end{tableiii}

The end of the RE has now been reached, and it has matched
\samp{abcb}.  This demonstrates how the matching engine goes as far as
it can at first, and if no match is found it will then progressively
back up and retry the rest of the RE again and again.  It will back up
until it has tried zero matches for \regexp{[bcd]*}, and if that
subsequently fails, the engine will conclude that the string doesn't
match the RE at all.

Another repeating metacharacter is \regexp{+}, which matches one or
more times.  Pay careful attention to the difference between
\regexp{*} and \regexp{+}; \regexp{*} matches \emph{zero} or more
times, so whatever's being repeated may not be present at all, while
\regexp{+} requires at least \emph{one} occurrence.  To use a similar
example, \regexp{ca+t} will match \samp{cat} (1 \samp{a}),
\samp{caaat} (3 \samp{a}'s), but won't match \samp{ct}.

There are two more repeating qualifiers.  The question mark character,
\regexp{?}, matches either once or zero times; you can think of it as
marking something as being optional.  For example, \regexp{home-?brew}
matches either \samp{homebrew} or \samp{home-brew}.  

The most complicated repeated qualifier is
\regexp{\{\var{m},\var{n}\}}, where \var{m} and \var{n} are decimal
integers.  This qualifier means there must be at least \var{m}
repetitions, and at most \var{n}.  For example, \regexp{a/\{1,3\}b}
will match \samp{a/b}, \samp{a//b}, and \samp{a///b}.  It won't match
\samp{ab}, which has no slashes, or \samp{a////b}, which has four.

You can omit either \var{m} or \var{n}; in that case, a reasonable
value is assumed for the missing value.  Omitting \var{m} is
interpreted as a lower limit of 0, while omitting \var{n} results in  an
upper bound of infinity --- actually, the 2 billion limit mentioned
earlier, but that might as well be infinity.  

Readers of a reductionist bent may notice that the three other qualifiers
can all be expressed using this notation.  \regexp{\{0,\}} is the same
as \regexp{*}, \regexp{\{1,\}} is equivalent to \regexp{+}, and
\regexp{\{0,1\}} is the same as \regexp{?}.  It's better to use
\regexp{*}, \regexp{+}, or \regexp{?} when you can, simply because
they're shorter and easier to read.

\section{Using Regular Expressions}

Now that we've looked at some simple regular expressions, how do we
actually use them in Python?  The \module{re} module provides an
interface to the regular expression engine, allowing you to compile
REs into objects and then perform matches with them.

\subsection{Compiling Regular Expressions}

Regular expressions are compiled into \class{RegexObject} instances,
which have methods for various operations such as searching for
pattern matches or performing string substitutions.

\begin{verbatim}
>>> import re
>>> p = re.compile('ab*')
>>> print p
<re.RegexObject instance at 80b4150>
\end{verbatim}

\function{re.compile()} also accepts an optional \var{flags}
argument, used to enable various special features and syntax
variations.  We'll go over the available settings later, but for now a
single example will do:

\begin{verbatim}
>>> p = re.compile('ab*', re.IGNORECASE)
\end{verbatim}

The RE is passed to \function{re.compile()} as a string.  REs are
handled as strings because regular expressions aren't part of the core
Python language, and no special syntax was created for expressing
them.  (There are applications that don't need REs at all, so there's
no need to bloat the language specification by including them.)
Instead, the \module{re} module is simply a C extension module
included with Python, just like the \module{socket} or \module{zlib}
module.

Putting REs in strings keeps the Python language simpler, but has one
disadvantage which is the topic of the next section.

\subsection{The Backslash Plague}

As stated earlier, regular expressions use the backslash
character (\character{\e}) to indicate special forms or to allow
special characters to be used without invoking their special meaning.
This conflicts with Python's usage of the same character for the same
purpose in string literals.

Let's say you want to write a RE that matches the string
\samp{{\e}section}, which might be found in a \LaTeX\ file.  To figure
out what to write in the program code, start with the desired string
to be matched.  Next, you must escape any backslashes and other
metacharacters by preceding them with a backslash, resulting in the
string \samp{\e\e section}.  The resulting string that must be passed
to \function{re.compile()} must be \verb|\\section|.  However, to
express this as a Python string literal, both backslashes must be
escaped \emph{again}.

\begin{tableii}{c|l}{code}{Characters}{Stage}
  \lineii{\e section}{Text string to be matched}
  \lineii{\e\e section}{Escaped backslash for \function{re.compile}}
  \lineii{"\e\e\e\e section"}{Escaped backslashes for a string literal}
\end{tableii}

In short, to match a literal backslash, one has to write
\code{'\e\e\e\e'} as the RE string, because the regular expression
must be \samp{\e\e}, and each backslash must be expressed as
\samp{\e\e} inside a regular Python string literal.  In REs that
feature backslashes repeatedly, this leads to lots of repeated
backslashes and makes the resulting strings difficult to understand.

The solution is to use Python's raw string notation for regular
expressions; backslashes are not handled in any special way in
a string literal prefixed with \character{r}, so \code{r"\e n"} is a
two-character string containing \character{\e} and \character{n},
while \code{"\e n"} is a one-character string containing a newline.
Frequently regular expressions will be expressed in Python
code using this raw string notation.  

\begin{tableii}{c|c}{code}{Regular String}{Raw string}
  \lineii{"ab*"}{\code{r"ab*"}}
  \lineii{"\e\e\e\e section"}{\code{r"\e\e section"}}
  \lineii{"\e\e w+\e\e s+\e\e 1"}{\code{r"\e w+\e s+\e 1"}}
\end{tableii}

\subsection{Performing Matches}

Once you have an object representing a compiled regular expression,
what do you do with it?  \class{RegexObject} instances have several
methods and attributes.  Only the most significant ones will be
covered here; consult \ulink{the Library
Reference}{http://www.python.org/doc/lib/module-re.html} for a
complete listing.

\begin{tableii}{c|l}{code}{Method/Attribute}{Purpose}
  \lineii{match()}{Determine if the RE matches at the beginning of
  the string.}
  \lineii{search()}{Scan through a string, looking for any location
  where this RE matches.}
  \lineii{findall()}{Find all substrings where the RE matches,
and returns them as a list.}
  \lineii{finditer()}{Find all substrings where the RE matches,
and returns them as an iterator.}
\end{tableii}

\method{match()} and \method{search()} return \code{None} if no match
can be found.  If they're successful, a \code{MatchObject} instance is
returned, containing information about the match: where it starts and
ends, the substring it matched, and more.

You can learn about this by interactively experimenting with the
\module{re} module.  If you have Tkinter available, you may also want
to look at \file{Tools/scripts/redemo.py}, a demonstration program
included with the Python distribution.  It allows you to enter REs and
strings, and displays whether the RE matches or fails.
\file{redemo.py} can be quite useful when trying to debug a
complicated RE.  Phil Schwartz's
\ulink{Kodos}{http://www.phil-schwartz.com/kodos.spy} is also an interactive
tool for developing and testing RE patterns.  This HOWTO will use the
standard Python interpreter for its examples.

First, run the Python interpreter, import the \module{re} module, and
compile a RE:

\begin{verbatim}
Python 2.2.2 (#1, Feb 10 2003, 12:57:01)
>>> import re
>>> p = re.compile('[a-z]+')
>>> p
<_sre.SRE_Pattern object at 80c3c28>
\end{verbatim}

Now, you can try matching various strings against the RE
\regexp{[a-z]+}.  An empty string shouldn't match at all, since
\regexp{+} means 'one or more repetitions'.  \method{match()} should
return \code{None} in this case, which will cause the interpreter to
print no output.  You can explicitly print the result of
\method{match()} to make this clear.

\begin{verbatim}
>>> p.match("")
>>> print p.match("")
None
\end{verbatim}

Now, let's try it on a string that it should match, such as
\samp{tempo}.  In this case, \method{match()} will return a
\class{MatchObject}, so you should store the result in a variable for
later use.

\begin{verbatim}
>>> m = p.match( 'tempo')
>>> print m
<_sre.SRE_Match object at 80c4f68>
\end{verbatim}

Now you can query the \class{MatchObject} for information about the
matching string.   \class{MatchObject} instances also have several
methods and attributes; the most important ones are:

\begin{tableii}{c|l}{code}{Method/Attribute}{Purpose}
  \lineii{group()}{Return the string matched by the RE}
  \lineii{start()}{Return the starting position of the match}
  \lineii{end()}{Return the ending position of the match}
  \lineii{span()}{Return a tuple containing the (start, end) positions 
                  of the match}
\end{tableii}

Trying these methods will soon clarify their meaning:

\begin{verbatim}
>>> m.group()
'tempo'
>>> m.start(), m.end()
(0, 5)
>>> m.span()
(0, 5)
\end{verbatim}

\method{group()} returns the substring that was matched by the
RE.  \method{start()} and \method{end()} return the starting and
ending index of the match. \method{span()} returns both start and end
indexes in a single tuple.  Since the \method{match} method only
checks if the RE matches at the start of a string,
\method{start()} will always be zero.  However, the \method{search}
method of \class{RegexObject} instances scans through the string, so 
the match may not start at zero in that case.

\begin{verbatim}
>>> print p.match('::: message')
None
>>> m = p.search('::: message') ; print m
<re.MatchObject instance at 80c9650>
>>> m.group()
'message'
>>> m.span()
(4, 11)
\end{verbatim}

In actual programs, the most common style is to store the
\class{MatchObject} in a variable, and then check if it was
\code{None}.  This usually looks like:

\begin{verbatim}
p = re.compile( ... )
m = p.match( 'string goes here' )
if m:
    print 'Match found: ', m.group()
else:
    print 'No match'
\end{verbatim}

Two \class{RegexObject} methods return all of the matches for a pattern.
\method{findall()} returns a list of matching strings:

\begin{verbatim}
>>> p = re.compile('\d+')
>>> p.findall('12 drummers drumming, 11 pipers piping, 10 lords a-leaping')
['12', '11', '10']
\end{verbatim}

\method{findall()} has to create the entire list before it can be
returned as the result.  In Python 2.2, the \method{finditer()} method
is also available, returning a sequence of \class{MatchObject} instances 
as an iterator.

\begin{verbatim}
>>> iterator = p.finditer('12 drummers drumming, 11 ... 10 ...')
>>> iterator
<callable-iterator object at 0x401833ac>
>>> for match in iterator:
...     print match.span()
...
(0, 2)
(22, 24)
(29, 31)
\end{verbatim}


\subsection{Module-Level Functions}

You don't have to produce a \class{RegexObject} and call its methods;
the \module{re} module also provides top-level functions called
\function{match()}, \function{search()}, \function{sub()}, and so
forth.  These functions take the same arguments as the corresponding
\class{RegexObject} method, with the RE string added as the first
argument, and still return either \code{None} or a \class{MatchObject}
instance.

\begin{verbatim}
>>> print re.match(r'From\s+', 'Fromage amk')
None
>>> re.match(r'From\s+', 'From amk Thu May 14 19:12:10 1998')
<re.MatchObject instance at 80c5978>
\end{verbatim}

Under the hood, these functions simply produce a \class{RegexObject}
for you and call the appropriate method on it.  They also store the
compiled object in a cache, so future calls using the same
RE are faster.  

Should you use these module-level functions, or should you get the
\class{RegexObject} and call its methods yourself?  That choice
depends on how frequently the RE will be used, and on your personal
coding style.  If a RE is being used at only one point in the code,
then the module functions are probably more convenient.  If a program
contains a lot of regular expressions, or re-uses the same ones in
several locations, then it might be worthwhile to collect all the
definitions in one place, in a section of code that compiles all the
REs ahead of time.  To take an example from the standard library,
here's an extract from \file{xmllib.py}:

\begin{verbatim}
ref = re.compile( ... )
entityref = re.compile( ... )
charref = re.compile( ... )
starttagopen = re.compile( ... )
\end{verbatim}

I generally prefer to work with the compiled object, even for
one-time uses, but few people will be as much of a purist about this
as I am.

\subsection{Compilation Flags}

Compilation flags let you modify some aspects of how regular
expressions work.  Flags are available in the \module{re} module under
two names, a long name such as \constant{IGNORECASE}, and a short,
one-letter form such as \constant{I}.  (If you're familiar with Perl's
pattern modifiers, the one-letter forms use the same letters; the
short form of \constant{re.VERBOSE} is \constant{re.X}, for example.)
Multiple flags can be specified by bitwise OR-ing them; \code{re.I |
re.M} sets both the \constant{I} and \constant{M} flags, for example.

Here's a table of the available flags, followed by
a more detailed explanation of each one.

\begin{tableii}{c|l}{}{Flag}{Meaning}
  \lineii{\constant{DOTALL}, \constant{S}}{Make \regexp{.} match any
  character, including newlines}
  \lineii{\constant{IGNORECASE}, \constant{I}}{Do case-insensitive matches}
  \lineii{\constant{LOCALE}, \constant{L}}{Do a locale-aware match}
  \lineii{\constant{MULTILINE}, \constant{M}}{Multi-line matching,
  affecting \regexp{\^} and \regexp{\$}}
  \lineii{\constant{VERBOSE}, \constant{X}}{Enable verbose REs,
  which can be organized more cleanly and understandably.}
\end{tableii}

\begin{datadesc}{I}
\dataline{IGNORECASE}
Perform case-insensitive matching; character class and literal strings
will match
letters by ignoring case.  For example, \regexp{[A-Z]} will match
lowercase letters, too, and \regexp{Spam} will match \samp{Spam},
\samp{spam}, or \samp{spAM}.
This lowercasing doesn't take the current locale into account; it will
if you also set the \constant{LOCALE} flag.
\end{datadesc}

\begin{datadesc}{L}
\dataline{LOCALE}
Make \regexp{\e w}, \regexp{\e W}, \regexp{\e b},
and \regexp{\e B}, dependent on the current locale.  

Locales are a feature of the C library intended to help in writing
programs that take account of language differences.  For example, if
you're processing French text, you'd want to be able to write
\regexp{\e w+} to match words, but \regexp{\e w} only matches the
character class \regexp{[A-Za-z]}; it won't match \character{\'e} or
\character{\c c}.  If your system is configured properly and a French
locale is selected, certain C functions will tell the program that
\character{\'e} should also be considered a letter.  Setting the
\constant{LOCALE} flag when compiling a regular expression will cause the
resulting compiled object to use these C functions for \regexp{\e w};
this is slower, but also enables \regexp{\e w+} to match French words as
you'd expect.
\end{datadesc}

\begin{datadesc}{M}
\dataline{MULTILINE}
(\regexp{\^} and \regexp{\$} haven't been explained yet; 
they'll be introduced in section~\ref{more-metacharacters}.)

Usually \regexp{\^} matches only at the beginning of the string, and
\regexp{\$} matches only at the end of the string and immediately before the
newline (if any) at the end of the string. When this flag is
specified, \regexp{\^} matches at the beginning of the string and at
the beginning of each line within the string, immediately following
each newline.  Similarly, the \regexp{\$} metacharacter matches either at
the end of the string and at the end of each line (immediately
preceding each newline).

\end{datadesc}

\begin{datadesc}{S}
\dataline{DOTALL}
Makes the \character{.} special character match any character at all,
including a newline; without this flag, \character{.} will match
anything \emph{except} a newline.
\end{datadesc}

\begin{datadesc}{X}
\dataline{VERBOSE} This flag allows you to write regular expressions
that are more readable by granting you more flexibility in how you can
format them.  When this flag has been specified, whitespace within the
RE string is ignored, except when the whitespace is in a character
class or preceded by an unescaped backslash; this lets you organize
and indent the RE more clearly.  It also enables you to put comments
within a RE that will be ignored by the engine; comments are marked by
a \character{\#} that's neither in a character class or preceded by an
unescaped backslash.

For example, here's a RE that uses \constant{re.VERBOSE}; see how
much easier it is to read?

\begin{verbatim}
charref = re.compile(r"""
 &[#]		     # Start of a numeric entity reference
 (
   [0-9]+[^0-9]      # Decimal form
   | 0[0-7]+[^0-7]   # Octal form
   | x[0-9a-fA-F]+[^0-9a-fA-F] # Hexadecimal form
 )
""", re.VERBOSE)
\end{verbatim}

Without the verbose setting, the RE would look like this:
\begin{verbatim}
charref = re.compile("&#([0-9]+[^0-9]"
                     "|0[0-7]+[^0-7]"
                     "|x[0-9a-fA-F]+[^0-9a-fA-F])")
\end{verbatim}

In the above example, Python's automatic concatenation of string
literals has been used to break up the RE into smaller pieces, but
it's still more difficult to understand than the version using
\constant{re.VERBOSE}.

\end{datadesc}

\section{More Pattern Power}

So far we've only covered a part of the features of regular
expressions.  In this section, we'll cover some new metacharacters,
and how to use groups to retrieve portions of the text that was matched.

\subsection{More Metacharacters\label{more-metacharacters}}

There are some metacharacters that we haven't covered yet.  Most of
them will be covered in this section.

Some of the remaining metacharacters to be discussed are
\dfn{zero-width assertions}.  They don't cause the engine to advance
through the string; instead, they consume no characters at all,
and simply succeed or fail.  For example, \regexp{\e b} is an
assertion that the current position is located at a word boundary; the
position isn't changed by the \regexp{\e b} at all.  This means that
zero-width assertions should never be repeated, because if they match
once at a given location, they can obviously be matched an infinite
number of times.

\begin{list}{}{}

\item[\regexp{|}] 
Alternation, or the ``or'' operator.  
If A and B are regular expressions, 
\regexp{A|B} will match any string that matches either \samp{A} or \samp{B}.
\regexp{|} has very low precedence in order to make it work reasonably when
you're alternating multi-character strings.
\regexp{Crow|Servo} will match either \samp{Crow} or \samp{Servo}, not
\samp{Cro}, a \character{w} or an \character{S}, and \samp{ervo}.

To match a literal \character{|},
use \regexp{\e|}, or enclose it inside a character class, as in \regexp{[|]}.

\item[\regexp{\^}] Matches at the beginning of lines.  Unless the
\constant{MULTILINE} flag has been set, this will only match at the
beginning of the string.  In \constant{MULTILINE} mode, this also
matches immediately after each newline within the string.  

For example, if you wish to match the word \samp{From} only at the
beginning of a line, the RE to use is \verb|^From|.

\begin{verbatim}
>>> print re.search('^From', 'From Here to Eternity')
<re.MatchObject instance at 80c1520>
>>> print re.search('^From', 'Reciting From Memory')
None
\end{verbatim}

%To match a literal \character{\^}, use \regexp{\e\^} or enclose it
%inside a character class, as in \regexp{[{\e}\^]}.

\item[\regexp{\$}] Matches at the end of a line, which is defined as
either the end of the string, or any location followed by a newline
character.    

\begin{verbatim}
>>> print re.search('}$', '{block}')
<re.MatchObject instance at 80adfa8>
>>> print re.search('}$', '{block} ')
None
>>> print re.search('}$', '{block}\n')
<re.MatchObject instance at 80adfa8>
\end{verbatim}
% $

To match a literal \character{\$}, use \regexp{\e\$} or enclose it
inside a character class, as in  \regexp{[\$]}.

\item[\regexp{\e A}] Matches only at the start of the string.  When
not in \constant{MULTILINE} mode, \regexp{\e A} and \regexp{\^} are
effectively the same.  In \constant{MULTILINE} mode, however, they're
different; \regexp{\e A} still matches only at the beginning of the
string, but \regexp{\^} may match at any location inside the string
that follows a newline character.

\item[\regexp{\e Z}]Matches only at the end of the string.  

\item[\regexp{\e b}] Word boundary.  
This is a zero-width assertion that matches only at the
beginning or end of a word.  A word is defined as a sequence of
alphanumeric characters, so the end of a word is indicated by
whitespace or a non-alphanumeric character.  

The following example matches \samp{class} only when it's a complete
word; it won't match when it's contained inside another word.

\begin{verbatim}
>>> p = re.compile(r'\bclass\b')
>>> print p.search('no class at all')
<re.MatchObject instance at 80c8f28>
>>> print p.search('the declassified algorithm')
None
>>> print p.search('one subclass is')
None
\end{verbatim}

There are two subtleties you should remember when using this special
sequence.  First, this is the worst collision between Python's string
literals and regular expression sequences.  In Python's string
literals, \samp{\e b} is the backspace character, ASCII value 8.  If
you're not using raw strings, then Python will convert the \samp{\e b} to
a backspace, and your RE won't match as you expect it to.  The
following example looks the same as our previous RE, but omits
the \character{r} in front of the RE string.

\begin{verbatim}
>>> p = re.compile('\bclass\b')
>>> print p.search('no class at all')
None
>>> print p.search('\b' + 'class' + '\b')  
<re.MatchObject instance at 80c3ee0>
\end{verbatim}

Second, inside a character class, where there's no use for this
assertion, \regexp{\e b} represents the backspace character, for
compatibility with Python's string literals.

\item[\regexp{\e B}] Another zero-width assertion, this is the
opposite of \regexp{\e b}, only matching when the current
position is not at a word boundary.

\end{list}

\subsection{Grouping}

Frequently you need to obtain more information than just whether the
RE matched or not.  Regular expressions are often used to dissect
strings by writing a RE divided into several subgroups which
match different components of interest.  For example, an RFC-822
header line is divided into a header name and a value, separated by a
\character{:}.  This can be handled by writing a regular expression
which matches an entire header line, and has one group which matches the
header name, and another group which matches the header's value.

Groups are marked by the \character{(}, \character{)} metacharacters.
\character{(} and \character{)} have much the same meaning as they do
in mathematical expressions; they group together the expressions
contained inside them. For example, you can repeat the contents of a
group with a repeating qualifier, such as \regexp{*}, \regexp{+},
\regexp{?}, or \regexp{\{\var{m},\var{n}\}}.  For example,
\regexp{(ab)*} will match zero or more repetitions of \samp{ab}.

\begin{verbatim}
>>> p = re.compile('(ab)*')
>>> print p.match('ababababab').span()
(0, 10)
\end{verbatim}

Groups indicated with \character{(}, \character{)} also capture the
starting and ending index of the text that they match; this can be
retrieved by passing an argument to \method{group()},
\method{start()}, \method{end()}, and \method{span()}.  Groups are
numbered starting with 0.  Group 0 is always present; it's the whole
RE, so \class{MatchObject} methods all have group 0 as their default
argument.  Later we'll see how to express groups that don't capture
the span of text that they match.

\begin{verbatim}
>>> p = re.compile('(a)b')
>>> m = p.match('ab')
>>> m.group()
'ab'
>>> m.group(0)
'ab'
\end{verbatim}

Subgroups are numbered from left to right, from 1 upward.  Groups can
be nested; to determine the number, just count the opening parenthesis
characters, going from left to right.

\begin{verbatim}
>>> p = re.compile('(a(b)c)d')
>>> m = p.match('abcd')
>>> m.group(0)
'abcd'
>>> m.group(1)
'abc'
>>> m.group(2)
'b'
\end{verbatim}

\method{group()} can be passed multiple group numbers at a time, in
which case it will return a tuple containing the corresponding values
for those groups.

\begin{verbatim}  
>>> m.group(2,1,2)
('b', 'abc', 'b')
\end{verbatim}  

The \method{groups()} method returns a tuple containing the strings
for all the subgroups, from 1 up to however many there are.

\begin{verbatim}  
>>> m.groups()
('abc', 'b')
\end{verbatim}  

Backreferences in a pattern allow you to specify that the contents of
an earlier capturing group must also be found at the current location
in the string.  For example, \regexp{\e 1} will succeed if the exact
contents of group 1 can be found at the current position, and fails
otherwise.  Remember that Python's string literals also use a
backslash followed by numbers to allow including arbitrary characters
in a string, so be sure to use a raw string when incorporating
backreferences in a RE.

For example, the following RE detects doubled words in a string.

\begin{verbatim}
>>> p = re.compile(r'(\b\w+)\s+\1')
>>> p.search('Paris in the the spring').group()
'the the'
\end{verbatim}

Backreferences like this aren't often useful for just searching
through a string --- there are few text formats which repeat data in
this way --- but you'll soon find out that they're \emph{very} useful
when performing string substitutions.

\subsection{Non-capturing and Named Groups}

Elaborate REs may use many groups, both to capture substrings of
interest, and to group and structure the RE itself.  In complex REs,
it becomes difficult to keep track of the group numbers.  There are
two features which help with this problem.  Both of them use a common
syntax for regular expression extensions, so we'll look at that first.

Perl 5 added several additional features to standard regular
expressions, and the Python \module{re} module supports most of them.
It would have been difficult to choose new single-keystroke
metacharacters or new special sequences beginning with \samp{\e} to
represent the new features without making Perl's regular expressions
confusingly different from standard REs.  If you chose \samp{\&} as a
new metacharacter, for example, old expressions would be assuming that
\samp{\&} was a regular character and wouldn't have escaped it by
writing \regexp{\e \&} or \regexp{[\&]}.  

The solution chosen by the Perl developers was to use \regexp{(?...)}
as the extension syntax.  \samp{?} immediately after a parenthesis was
a syntax error because the \samp{?} would have nothing to repeat, so
this didn't introduce any compatibility problems.  The characters
immediately after the \samp{?}  indicate what extension is being used,
so \regexp{(?=foo)} is one thing (a positive lookahead assertion) and
\regexp{(?:foo)} is something else (a non-capturing group containing
the subexpression \regexp{foo}).

Python adds an extension syntax to Perl's extension syntax.  If the
first character after the question mark is a \samp{P}, you know that
it's an extension that's specific to Python.  Currently there are two
such extensions: \regexp{(?P<\var{name}>...)} defines a named group,
and \regexp{(?P=\var{name})} is a backreference to a named group.  If
future versions of Perl 5 add similar features using a different
syntax, the \module{re} module will be changed to support the new
syntax, while preserving the Python-specific syntax for
compatibility's sake.

Now that we've looked at the general extension syntax, we can return
to the features that simplify working with groups in complex REs.
Since groups are numbered from left to right and a complex expression
may use many groups, it can become difficult to keep track of the
correct numbering, and modifying such a complex RE is annoying.
Insert a new group near the beginning, and you change the numbers of
everything that follows it.

First, sometimes you'll want to use a group to collect a part of a
regular expression, but aren't interested in retrieving the group's
contents.  You can make this fact explicit by using a non-capturing
group: \regexp{(?:...)}, where you can put any other regular
expression inside the parentheses.  

\begin{verbatim}
>>> m = re.match("([abc])+", "abc")
>>> m.groups()
('c',)
>>> m = re.match("(?:[abc])+", "abc")
>>> m.groups()
()
\end{verbatim}

Except for the fact that you can't retrieve the contents of what the
group matched, a non-capturing group behaves exactly the same as a
capturing group; you can put anything inside it, repeat it with a
repetition metacharacter such as \samp{*}, and nest it within other
groups (capturing or non-capturing).  \regexp{(?:...)} is particularly
useful when modifying an existing group, since you can add new groups
without changing how all the other groups are numbered.  It should be
mentioned that there's no performance difference in searching between
capturing and non-capturing groups; neither form is any faster than
the other.

The second, and more significant, feature is named groups; instead of
referring to them by numbers, groups can be referenced by a name.

The syntax for a named group is one of the Python-specific extensions:
\regexp{(?P<\var{name}>...)}.  \var{name} is, obviously, the name of
the group.  Except for associating a name with a group, named groups
also behave identically to capturing groups.  The \class{MatchObject}
methods that deal with capturing groups all accept either integers, to
refer to groups by number, or a string containing the group name.
Named groups are still given numbers, so you can retrieve information
about a group in two ways:

\begin{verbatim}
>>> p = re.compile(r'(?P<word>\b\w+\b)')
>>> m = p.search( '(((( Lots of punctuation )))' )
>>> m.group('word')
'Lots'
>>> m.group(1)
'Lots'
\end{verbatim}

Named groups are handy because they let you use easily-remembered
names, instead of having to remember numbers.  Here's an example RE
from the \module{imaplib} module:

\begin{verbatim}
InternalDate = re.compile(r'INTERNALDATE "'
        r'(?P<day>[ 123][0-9])-(?P<mon>[A-Z][a-z][a-z])-'
	r'(?P<year>[0-9][0-9][0-9][0-9])'
        r' (?P<hour>[0-9][0-9]):(?P<min>[0-9][0-9]):(?P<sec>[0-9][0-9])'
        r' (?P<zonen>[-+])(?P<zoneh>[0-9][0-9])(?P<zonem>[0-9][0-9])'
        r'"')
\end{verbatim}

It's obviously much easier to retrieve \code{m.group('zonem')},
instead of having to remember to retrieve group 9.

Since the syntax for backreferences, in an expression like
\regexp{(...)\e 1}, refers to the number of the group there's
naturally a variant that uses the group name instead of the number.
This is also a Python extension: \regexp{(?P=\var{name})} indicates
that the contents of the group called \var{name} should again be found
at the current point.  The regular expression for finding doubled
words, \regexp{(\e b\e w+)\e s+\e 1} can also be written as
\regexp{(?P<word>\e b\e w+)\e s+(?P=word)}:

\begin{verbatim}
>>> p = re.compile(r'(?P<word>\b\w+)\s+(?P=word)')
>>> p.search('Paris in the the spring').group()
'the the'
\end{verbatim}

\subsection{Lookahead Assertions}

Another zero-width assertion is the lookahead assertion.  Lookahead
assertions are available in both positive and negative form, and 
look like this:

\begin{itemize}
\item[\regexp{(?=...)}] Positive lookahead assertion.  This succeeds
if the contained regular expression, represented here by \code{...},
successfully matches at the current location, and fails otherwise.
But, once the contained expression has been tried, the matching engine
doesn't advance at all; the rest of the pattern is tried right where
the assertion started.

\item[\regexp{(?!...)}] Negative lookahead assertion.  This is the
opposite of the positive assertion; it succeeds if the contained expression
\emph{doesn't} match at the current position in the string.
\end{itemize}

An example will help make this concrete by demonstrating a case
where a lookahead is useful.  Consider a simple pattern to match a
filename and split it apart into a base name and an extension,
separated by a \samp{.}.  For example, in \samp{news.rc}, \samp{news}
is the base name, and \samp{rc} is the filename's extension.  

The pattern to match this is quite simple: 

\regexp{.*[.].*\$}

Notice that the \samp{.} needs to be treated specially because it's a
metacharacter; I've put it inside a character class.  Also notice the
trailing \regexp{\$}; this is added to ensure that all the rest of the
string must be included in the extension.  This regular expression
matches \samp{foo.bar} and \samp{autoexec.bat} and \samp{sendmail.cf} and
\samp{printers.conf}.

Now, consider complicating the problem a bit; what if you want to
match filenames where the extension is not \samp{bat}?
Some incorrect attempts:

\verb|.*[.][^b].*$|
% $

The first attempt above tries to exclude \samp{bat} by requiring that
the first character of the extension is not a \samp{b}.  This is
wrong, because the pattern also doesn't match \samp{foo.bar}.

% Messes up the HTML without the curly braces around \^
\regexp{.*[.]([{\^}b]..|.[{\^}a].|..[{\^}t])\$}

The expression gets messier when you try to patch up the first
solution by requiring one of the following cases to match: the first
character of the extension isn't \samp{b}; the second character isn't
\samp{a}; or the third character isn't \samp{t}.  This accepts
\samp{foo.bar} and rejects \samp{autoexec.bat}, but it requires a
three-letter extension and won't accept a filename with a two-letter
extension such as \samp{sendmail.cf}.  We'll complicate the pattern
again in an effort to fix it.

\regexp{.*[.]([{\^}b].?.?|.[{\^}a]?.?|..?[{\^}t]?)\$}

In the third attempt, the second and third letters are all made
optional in order to allow matching extensions shorter than three
characters, such as \samp{sendmail.cf}.

The pattern's getting really complicated now, which makes it hard to
read and understand.  Worse, if the problem changes and you want to
exclude both \samp{bat} and \samp{exe} as extensions, the pattern
would get even more complicated and confusing.

A negative lookahead cuts through all this:

\regexp{.*[.](?!bat\$).*\$}
% $

The lookahead means: if the expression \regexp{bat} doesn't match at
this point, try the rest of the pattern; if \regexp{bat\$} does match,
the whole pattern will fail.  The trailing \regexp{\$} is required to
ensure that something like \samp{sample.batch}, where the extension
only starts with \samp{bat}, will be allowed.

Excluding another filename extension is now easy; simply add it as an
alternative inside the assertion.  The following pattern excludes
filenames that end in either \samp{bat} or \samp{exe}:

\regexp{.*[.](?!bat\$|exe\$).*\$}
% $


\section{Modifying Strings}

Up to this point, we've simply performed searches against a static
string.  Regular expressions are also commonly used to modify a string
in various ways, using the following \class{RegexObject} methods:

\begin{tableii}{c|l}{code}{Method/Attribute}{Purpose}
  \lineii{split()}{Split the string into a list, splitting it wherever the RE matches}
  \lineii{sub()}{Find all substrings where the RE matches, and replace them with a different string}
  \lineii{subn()}{Does the same thing as \method{sub()}, 
   but returns the new string and the number of replacements}
\end{tableii}


\subsection{Splitting Strings}

The \method{split()} method of a \class{RegexObject} splits a string
apart wherever the RE matches, returning a list of the pieces.
It's similar to the \method{split()} method of strings but
provides much more
generality in the delimiters that you can split by;
\method{split()} only supports splitting by whitespace or by
a fixed string.  As you'd expect, there's a module-level
\function{re.split()} function, too.

\begin{methoddesc}{split}{string \optional{, maxsplit\code{ = 0}}}
  Split \var{string} by the matches of the regular expression.  If
  capturing parentheses are used in the RE, then their contents will
  also be returned as part of the resulting list.  If \var{maxsplit}
  is nonzero, at most \var{maxsplit} splits are performed.
\end{methoddesc}

You can limit the number of splits made, by passing a value for
\var{maxsplit}.  When \var{maxsplit} is nonzero, at most
\var{maxsplit} splits will be made, and the remainder of the string is
returned as the final element of the list.  In the following example,
the delimiter is any sequence of non-alphanumeric characters.

\begin{verbatim}
>>> p = re.compile(r'\W+')
>>> p.split('This is a test, short and sweet, of split().')
['This', 'is', 'a', 'test', 'short', 'and', 'sweet', 'of', 'split', '']
>>> p.split('This is a test, short and sweet, of split().', 3)
['This', 'is', 'a', 'test, short and sweet, of split().']
\end{verbatim}

Sometimes you're not only interested in what the text between
delimiters is, but also need to know what the delimiter was.  If
capturing parentheses are used in the RE, then their values are also
returned as part of the list.  Compare the following calls:

\begin{verbatim}
>>> p = re.compile(r'\W+')
>>> p2 = re.compile(r'(\W+)')
>>> p.split('This... is a test.')
['This', 'is', 'a', 'test', '']
>>> p2.split('This... is a test.')
['This', '... ', 'is', ' ', 'a', ' ', 'test', '.', '']
\end{verbatim}

The module-level function \function{re.split()} adds the RE to be
used as the first argument, but is otherwise the same.  

\begin{verbatim}
>>> re.split('[\W]+', 'Words, words, words.')
['Words', 'words', 'words', '']
>>> re.split('([\W]+)', 'Words, words, words.')
['Words', ', ', 'words', ', ', 'words', '.', '']
>>> re.split('[\W]+', 'Words, words, words.', 1)
['Words', 'words, words.']
\end{verbatim}

\subsection{Search and Replace}

Another common task is to find all the matches for a pattern, and
replace them with a different string.  The \method{sub()} method takes
a replacement value, which can be either a string or a function, and
the string to be processed.

\begin{methoddesc}{sub}{replacement, string\optional{, count\code{ = 0}}}
Returns the string obtained by replacing the leftmost non-overlapping
occurrences of the RE in \var{string} by the replacement
\var{replacement}.  If the pattern isn't found, \var{string} is returned
unchanged.  

The optional argument \var{count} is the maximum number of pattern
occurrences to be replaced; \var{count} must be a non-negative
integer.  The default value of 0 means to replace all occurrences.
\end{methoddesc}

Here's a simple example of using the \method{sub()} method.  It
replaces colour names with the word \samp{colour}:

\begin{verbatim}
>>> p = re.compile( '(blue|white|red)')
>>> p.sub( 'colour', 'blue socks and red shoes')
'colour socks and colour shoes'
>>> p.sub( 'colour', 'blue socks and red shoes', count=1)
'colour socks and red shoes'
\end{verbatim}

The \method{subn()} method does the same work, but returns a 2-tuple
containing the new string value and the number of replacements 
that were performed:

\begin{verbatim}
>>> p = re.compile( '(blue|white|red)')
>>> p.subn( 'colour', 'blue socks and red shoes')
('colour socks and colour shoes', 2)
>>> p.subn( 'colour', 'no colours at all')
('no colours at all', 0)
\end{verbatim}

Empty matches are replaced only when they're not
adjacent to a previous match.  

\begin{verbatim}
>>> p = re.compile('x*')
>>> p.sub('-', 'abxd')
'-a-b-d-'
\end{verbatim}

If \var{replacement} is a string, any backslash escapes in it are
processed.  That is, \samp{\e n} is converted to a single newline
character, \samp{\e r} is converted to a carriage return, and so forth.
Unknown escapes such as \samp{\e j} are left alone.  Backreferences,
such as \samp{\e 6}, are replaced with the substring matched by the
corresponding group in the RE.  This lets you incorporate
portions of the original text in the resulting
replacement string.

This example matches the word \samp{section} followed by a string
enclosed in \samp{\{}, \samp{\}}, and changes \samp{section} to
\samp{subsection}:

\begin{verbatim}
>>> p = re.compile('section{ ( [^}]* ) }', re.VERBOSE)
>>> p.sub(r'subsection{\1}','section{First} section{second}')
'subsection{First} subsection{second}'
\end{verbatim}

There's also a syntax for referring to named groups as defined by the
\regexp{(?P<name>...)} syntax.  \samp{\e g<name>} will use the
substring matched by the group named \samp{name}, and 
\samp{\e g<\var{number}>} 
uses the corresponding group number.  
\samp{\e g<2>} is therefore equivalent to \samp{\e 2}, 
but isn't ambiguous in a
replacement string such as \samp{\e g<2>0}.  (\samp{\e 20} would be
interpreted as a reference to group 20, not a reference to group 2
followed by the literal character \character{0}.)  The following
substitutions are all equivalent, but use all three variations of the
replacement string.

\begin{verbatim}
>>> p = re.compile('section{ (?P<name> [^}]* ) }', re.VERBOSE)
>>> p.sub(r'subsection{\1}','section{First}')
'subsection{First}'
>>> p.sub(r'subsection{\g<1>}','section{First}')
'subsection{First}'
>>> p.sub(r'subsection{\g<name>}','section{First}')
'subsection{First}'
\end{verbatim}

\var{replacement} can also be a function, which gives you even more
control.  If \var{replacement} is a function, the function is
called for every non-overlapping occurrence of \var{pattern}.  On each
call, the function is 
passed a \class{MatchObject} argument for the match
and can use this information to compute the desired replacement string and return it.

In the following example, the replacement function translates 
decimals into hexadecimal:

\begin{verbatim}
>>> def hexrepl( match ):
...     "Return the hex string for a decimal number"
...     value = int( match.group() )
...     return hex(value)
...
>>> p = re.compile(r'\d+')
>>> p.sub(hexrepl, 'Call 65490 for printing, 49152 for user code.')
'Call 0xffd2 for printing, 0xc000 for user code.'
\end{verbatim}

When using the module-level \function{re.sub()} function, the pattern
is passed as the first argument.  The pattern may be a string or a
\class{RegexObject}; if you need to specify regular expression flags,
you must either use a \class{RegexObject} as the first parameter, or use
embedded modifiers in the pattern, e.g.  \code{sub("(?i)b+", "x", "bbbb
BBBB")} returns \code{'x x'}.

\section{Common Problems}

Regular expressions are a powerful tool for some applications, but in
some ways their behaviour isn't intuitive and at times they don't
behave the way you may expect them to.  This section will point out
some of the most common pitfalls.

\subsection{Use String Methods}

Sometimes using the \module{re} module is a mistake.  If you're
matching a fixed string, or a single character class, and you're not
using any \module{re} features such as the \constant{IGNORECASE} flag,
then the full power of regular expressions may not be required.
Strings have several methods for performing operations with fixed
strings and they're usually much faster, because the implementation is
a single small C loop that's been optimized for the purpose, instead
of the large, more generalized regular expression engine.

One example might be replacing a single fixed string with another
one; for example, you might replace \samp{word}
with \samp{deed}.  \code{re.sub()} seems like the function to use for
this, but consider the \method{replace()} method.  Note that 
\function{replace()} will also replace \samp{word} inside
words, turning \samp{swordfish} into \samp{sdeedfish}, but the 
na{\"\i}ve RE \regexp{word} would have done that, too.  (To avoid performing
the substitution on parts of words, the pattern would have to be
\regexp{\e bword\e b}, in order to require that \samp{word} have a
word boundary on either side.  This takes the job beyond 
\method{replace}'s abilities.)

Another common task is deleting every occurrence of a single character
from a string or replacing it with another single character.  You
might do this with something like \code{re.sub('\e n', ' ', S)}, but
\method{translate()} is capable of doing both tasks
and will be faster than any regular expression operation can be.

In short, before turning to the \module{re} module, consider whether
your problem can be solved with a faster and simpler string method.

\subsection{match() versus search()}

The \function{match()} function only checks if the RE matches at
the beginning of the string while \function{search()} will scan
forward through the string for a match.
It's important to keep this distinction in mind.  Remember, 
\function{match()} will only report a successful match which
will start at 0; if the match wouldn't start at zero, 
\function{match()} will \emph{not} report it.

\begin{verbatim}
>>> print re.match('super', 'superstition').span()  
(0, 5)
>>> print re.match('super', 'insuperable')    
None
\end{verbatim}

On the other hand, \function{search()} will scan forward through the
string, reporting the first match it finds.

\begin{verbatim}
>>> print re.search('super', 'superstition').span()
(0, 5)
>>> print re.search('super', 'insuperable').span()
(2, 7)
\end{verbatim}

Sometimes you'll be tempted to keep using \function{re.match()}, and
just add \regexp{.*} to the front of your RE.  Resist this temptation
and use \function{re.search()} instead.  The regular expression
compiler does some analysis of REs in order to speed up the process of
looking for a match.  One such analysis figures out what the first
character of a match must be; for example, a pattern starting with
\regexp{Crow} must match starting with a \character{C}.  The analysis
lets the engine quickly scan through the string looking for the
starting character, only trying the full match if a \character{C} is found.

Adding \regexp{.*} defeats this optimization, requiring scanning to
the end of the string and then backtracking to find a match for the
rest of the RE.  Use \function{re.search()} instead.

\subsection{Greedy versus Non-Greedy}

When repeating a regular expression, as in \regexp{a*}, the resulting
action is to consume as much of the pattern as possible.  This
fact often bites you when you're trying to match a pair of
balanced delimiters, such as the angle brackets surrounding an HTML
tag.  The na{\"\i}ve pattern for matching a single HTML tag doesn't
work because of the greedy nature of \regexp{.*}.

\begin{verbatim}
>>> s = '<html><head><title>Title</title>'
>>> len(s)
32
>>> print re.match('<.*>', s).span()
(0, 32)
>>> print re.match('<.*>', s).group()
<html><head><title>Title</title>
\end{verbatim}

The RE matches the \character{<} in \samp{<html>}, and the
\regexp{.*} consumes the rest of the string.  There's still more left
in the RE, though, and the \regexp{>} can't match at the end of
the string, so the regular expression engine has to backtrack
character by character until it finds a match for the \regexp{>}.  
The final match extends from the \character{<} in \samp{<html>}
to the \character{>} in \samp{</title>}, which isn't what you want.

In this case, the solution is to use the non-greedy qualifiers
\regexp{*?}, \regexp{+?}, \regexp{??}, or
\regexp{\{\var{m},\var{n}\}?}, which match as \emph{little} text as
possible.  In the above example, the \character{>} is tried
immediately after the first \character{<} matches, and when it fails,
the engine advances a character at a time, retrying the \character{>}
at every step.  This produces just the right result:

\begin{verbatim}
>>> print re.match('<.*?>', s).group()
<html>
\end{verbatim}

(Note that parsing HTML or XML with regular expressions is painful.
Quick-and-dirty patterns will handle common cases, but HTML and XML
have special cases that will break the obvious regular expression; by
the time you've written a regular expression that handles all of the
possible cases, the patterns will be \emph{very} complicated.  Use an
HTML or XML parser module for such tasks.)

\subsection{Not Using re.VERBOSE}

By now you've probably noticed that regular expressions are a very
compact notation, but they're not terribly readable.  REs of
moderate complexity can become lengthy collections of backslashes,
parentheses, and metacharacters, making them difficult to read and
understand.  

For such REs, specifying the \code{re.VERBOSE} flag when
compiling the regular expression can be helpful, because it allows
you to format the regular expression more clearly.

The \code{re.VERBOSE} flag has several effects.  Whitespace in the
regular expression that \emph{isn't} inside a character class is
ignored.  This means that an expression such as \regexp{dog | cat} is
equivalent to the less readable \regexp{dog|cat}, but \regexp{[a b]}
will still match the characters \character{a}, \character{b}, or a
space.  In addition, you can also put comments inside a RE; comments
extend from a \samp{\#} character to the next newline.  When used with
triple-quoted strings, this enables REs to be formatted more neatly:

\begin{verbatim}
pat = re.compile(r"""
 \s*                 # Skip leading whitespace
 (?P<header>[^:]+)   # Header name
 \s* :               # Whitespace, and a colon
 (?P<value>.*?)      # The header's value -- *? used to
                     # lose the following trailing whitespace
 \s*$                # Trailing whitespace to end-of-line
""", re.VERBOSE)
\end{verbatim}
% $

This is far more readable than:

\begin{verbatim}
pat = re.compile(r"\s*(?P<header>[^:]+)\s*:(?P<value>.*?)\s*$")
\end{verbatim}
% $

\section{Feedback}

Regular expressions are a complicated topic.  Did this document help
you understand them?  Were there parts that were unclear, or Problems
you encountered that weren't covered here?  If so, please send
suggestions for improvements to the author.

The most complete book on regular expressions is almost certainly
Jeffrey Friedl's \citetitle{Mastering Regular Expressions}, published
by O'Reilly.  Unfortunately, it exclusively concentrates on Perl and
Java's flavours of regular expressions, and doesn't contain any Python
material at all, so it won't be useful as a reference for programming
in Python.  (The first edition covered Python's now-removed
\module{regex} module, which won't help you much.)  Consider checking
it out from your library.

\end{document}

