\documentclass{manual}

% XXX PM explain how to add new types to Python

\title{Extending and Embedding the Python Interpreter}

\author{Guido van Rossum\\
	Fred L. Drake, Jr., editor}
\authoraddress{
	PythonLabs\\
	E-mail: \email{python-docs@python.org}
}

\date{\today}		% XXX update before release!
\release{2.1 alpha}		% software release, not documentation
\setshortversion{2.1}		% major.minor only for software


% Tell \index to actually write the .idx file
\makeindex

\begin{document}

\maketitle

\ifhtml
\chapter*{Front Matter\label{front}}
\fi

Copyright \copyright{} 2001-2006 Python Software Foundation.
All rights reserved.

Copyright \copyright{} 2000 BeOpen.com.
All rights reserved.

Copyright \copyright{} 1995-2000 Corporation for National Research Initiatives.
All rights reserved.

Copyright \copyright{} 1991-1995 Stichting Mathematisch Centrum.
All rights reserved.

See the end of this document for complete license and permissions
information.



\begin{abstract}

\noindent
Python is an interpreted, object-oriented programming language.  This
document describes how to write modules in C or \Cpp{} to extend the
Python interpreter with new modules.  Those modules can define new
functions but also new object types and their methods.  The document
also describes how to embed the Python interpreter in another
application, for use as an extension language.  Finally, it shows how
to compile and link extension modules so that they can be loaded
dynamically (at run time) into the interpreter, if the underlying
operating system supports this feature.

This document assumes basic knowledge about Python.  For an informal
introduction to the language, see the
\citetitle[../tut/tut.html]{Python Tutorial}.  The
\citetitle[../ref/ref.html]{Python Reference Manual} gives a more
formal definition of the language.  The
\citetitle[../lib/lib.html]{Python Library Reference} documents the
existing object types, functions and modules (both built-in and
written in Python) that give the language its wide application range.

For a detailed description of the whole Python/C API, see the separate
\citetitle[../api/api.html]{Python/C API Reference Manual}.

\end{abstract}

\tableofcontents


\chapter{Extending Python with \C{} or \Cpp{} \label{intro}}


It is quite easy to add new built-in modules to Python, if you know
how to program in C.  Such \dfn{extension modules} can do two things
that can't be done directly in Python: they can implement new built-in
object types, and they can call C library functions and system calls.

To support extensions, the Python API (Application Programmers
Interface) defines a set of functions, macros and variables that
provide access to most aspects of the Python run-time system.  The
Python API is incorporated in a C source file by including the header
\code{"Python.h"}.

The compilation of an extension module depends on its intended use as
well as on your system setup; details are given in later chapters.


\section{A Simple Example
         \label{simpleExample}}

Let's create an extension module called \samp{spam} (the favorite food
of Monty Python fans...) and let's say we want to create a Python
interface to the C library function \cfunction{system()}.\footnote{An
interface for this function already exists in the standard module
\module{os} --- it was chosen as a simple and straightforward example.}
This function takes a null-terminated character string as argument and
returns an integer.  We want this function to be callable from Python
as follows:

\begin{verbatim}
>>> import spam
>>> status = spam.system("ls -l")
\end{verbatim}

Begin by creating a file \file{spammodule.c}.  (Historically, if a
module is called \samp{spam}, the C file containing its implementation
is called \file{spammodule.c}; if the module name is very long, like
\samp{spammify}, the module name can be just \file{spammify.c}.)

The first line of our file can be:

\begin{verbatim}
#include <Python.h>
\end{verbatim}

which pulls in the Python API (you can add a comment describing the
purpose of the module and a copyright notice if you like).

\begin{notice}[warning]
  Since Python may define some pre-processor definitions which affect
  the standard headers on some systems, you \emph{must} include
  \file{Python.h} before any standard headers are included.
\end{notice}

All user-visible symbols defined by \file{Python.h} have a prefix of
\samp{Py} or \samp{PY}, except those defined in standard header files.
For convenience, and since they are used extensively by the Python
interpreter, \code{"Python.h"} includes a few standard header files:
\code{<stdio.h>}, \code{<string.h>}, \code{<errno.h>}, and
\code{<stdlib.h>}.  If the latter header file does not exist on your
system, it declares the functions \cfunction{malloc()},
\cfunction{free()} and \cfunction{realloc()} directly.

The next thing we add to our module file is the C function that will
be called when the Python expression \samp{spam.system(\var{string})}
is evaluated (we'll see shortly how it ends up being called):

\begin{verbatim}
static PyObject *
spam_system(PyObject *self, PyObject *args)
{
    const char *command;
    int sts;

    if (!PyArg_ParseTuple(args, "s", &command))
        return NULL;
    sts = system(command);
    return Py_BuildValue("i", sts);
}
\end{verbatim}

There is a straightforward translation from the argument list in
Python (for example, the single expression \code{"ls -l"}) to the
arguments passed to the C function.  The C function always has two
arguments, conventionally named \var{self} and \var{args}.

The \var{self} argument is only used when the C function implements a
built-in method, not a function. In the example, \var{self} will
always be a \NULL{} pointer, since we are defining a function, not a
method.  (This is done so that the interpreter doesn't have to
understand two different types of C functions.)

The \var{args} argument will be a pointer to a Python tuple object
containing the arguments.  Each item of the tuple corresponds to an
argument in the call's argument list.  The arguments are Python
objects --- in order to do anything with them in our C function we have
to convert them to C values.  The function \cfunction{PyArg_ParseTuple()}
in the Python API checks the argument types and converts them to C
values.  It uses a template string to determine the required types of
the arguments as well as the types of the C variables into which to
store the converted values.  More about this later.

\cfunction{PyArg_ParseTuple()} returns true (nonzero) if all arguments have
the right type and its components have been stored in the variables
whose addresses are passed.  It returns false (zero) if an invalid
argument list was passed.  In the latter case it also raises an
appropriate exception so the calling function can return
\NULL{} immediately (as we saw in the example).


\section{Intermezzo: Errors and Exceptions
         \label{errors}}

An important convention throughout the Python interpreter is the
following: when a function fails, it should set an exception condition
and return an error value (usually a \NULL{} pointer).  Exceptions
are stored in a static global variable inside the interpreter; if this
variable is \NULL{} no exception has occurred.  A second global
variable stores the ``associated value'' of the exception (the second
argument to \keyword{raise}).  A third variable contains the stack
traceback in case the error originated in Python code.  These three
variables are the C equivalents of the result in Python of 
\method{sys.exc_info()} (see the section on module \module{sys} in the
\citetitle[../lib/lib.html]{Python Library Reference}).  It is
important to know about them to understand how errors are passed
around.

The Python API defines a number of functions to set various types of
exceptions.

The most common one is \cfunction{PyErr_SetString()}.  Its arguments
are an exception object and a C string.  The exception object is
usually a predefined object like \cdata{PyExc_ZeroDivisionError}.  The
C string indicates the cause of the error and is converted to a
Python string object and stored as the ``associated value'' of the
exception.

Another useful function is \cfunction{PyErr_SetFromErrno()}, which only
takes an exception argument and constructs the associated value by
inspection of the global variable \cdata{errno}.  The most
general function is \cfunction{PyErr_SetObject()}, which takes two object
arguments, the exception and its associated value.  You don't need to
\cfunction{Py_INCREF()} the objects passed to any of these functions.

You can test non-destructively whether an exception has been set with
\cfunction{PyErr_Occurred()}.  This returns the current exception object,
or \NULL{} if no exception has occurred.  You normally don't need
to call \cfunction{PyErr_Occurred()} to see whether an error occurred in a
function call, since you should be able to tell from the return value.

When a function \var{f} that calls another function \var{g} detects
that the latter fails, \var{f} should itself return an error value
(usually \NULL{} or \code{-1}).  It should \emph{not} call one of the
\cfunction{PyErr_*()} functions --- one has already been called by \var{g}.
\var{f}'s caller is then supposed to also return an error indication
to \emph{its} caller, again \emph{without} calling \cfunction{PyErr_*()},
and so on --- the most detailed cause of the error was already
reported by the function that first detected it.  Once the error
reaches the Python interpreter's main loop, this aborts the currently
executing Python code and tries to find an exception handler specified
by the Python programmer.

(There are situations where a module can actually give a more detailed
error message by calling another \cfunction{PyErr_*()} function, and in
such cases it is fine to do so.  As a general rule, however, this is
not necessary, and can cause information about the cause of the error
to be lost: most operations can fail for a variety of reasons.)

To ignore an exception set by a function call that failed, the exception
condition must be cleared explicitly by calling \cfunction{PyErr_Clear()}. 
The only time C code should call \cfunction{PyErr_Clear()} is if it doesn't
want to pass the error on to the interpreter but wants to handle it
completely by itself (possibly by trying something else, or pretending
nothing went wrong).

Every failing \cfunction{malloc()} call must be turned into an
exception --- the direct caller of \cfunction{malloc()} (or
\cfunction{realloc()}) must call \cfunction{PyErr_NoMemory()} and
return a failure indicator itself.  All the object-creating functions
(for example, \cfunction{PyInt_FromLong()}) already do this, so this
note is only relevant to those who call \cfunction{malloc()} directly.

Also note that, with the important exception of
\cfunction{PyArg_ParseTuple()} and friends, functions that return an
integer status usually return a positive value or zero for success and
\code{-1} for failure, like \UNIX{} system calls.

Finally, be careful to clean up garbage (by making
\cfunction{Py_XDECREF()} or \cfunction{Py_DECREF()} calls for objects
you have already created) when you return an error indicator!

The choice of which exception to raise is entirely yours.  There are
predeclared C objects corresponding to all built-in Python exceptions,
such as \cdata{PyExc_ZeroDivisionError}, which you can use directly.
Of course, you should choose exceptions wisely --- don't use
\cdata{PyExc_TypeError} to mean that a file couldn't be opened (that
should probably be \cdata{PyExc_IOError}).  If something's wrong with
the argument list, the \cfunction{PyArg_ParseTuple()} function usually
raises \cdata{PyExc_TypeError}.  If you have an argument whose value
must be in a particular range or must satisfy other conditions,
\cdata{PyExc_ValueError} is appropriate.

You can also define a new exception that is unique to your module.
For this, you usually declare a static object variable at the
beginning of your file:

\begin{verbatim}
static PyObject *SpamError;
\end{verbatim}

and initialize it in your module's initialization function
(\cfunction{initspam()}) with an exception object (leaving out
the error checking for now):

\begin{verbatim}
PyMODINIT_FUNC
initspam(void)
{
    PyObject *m;

    m = Py_InitModule("spam", SpamMethods);
    if (m == NULL)
        return;

    SpamError = PyErr_NewException("spam.error", NULL, NULL);
    Py_INCREF(SpamError);
    PyModule_AddObject(m, "error", SpamError);
}
\end{verbatim}

Note that the Python name for the exception object is
\exception{spam.error}.  The \cfunction{PyErr_NewException()} function
may create a class with the base class being \exception{Exception}
(unless another class is passed in instead of \NULL), described in the
\citetitle[../lib/lib.html]{Python Library Reference} under ``Built-in
Exceptions.''

Note also that the \cdata{SpamError} variable retains a reference to
the newly created exception class; this is intentional!  Since the
exception could be removed from the module by external code, an owned
reference to the class is needed to ensure that it will not be
discarded, causing \cdata{SpamError} to become a dangling pointer.
Should it become a dangling pointer, C code which raises the exception
could cause a core dump or other unintended side effects.

We discuss the use of PyMODINIT_FUNC as a function return type later in this
sample.

\section{Back to the Example
         \label{backToExample}}

Going back to our example function, you should now be able to
understand this statement:

\begin{verbatim}
    if (!PyArg_ParseTuple(args, "s", &command))
        return NULL;
\end{verbatim}

It returns \NULL{} (the error indicator for functions returning
object pointers) if an error is detected in the argument list, relying
on the exception set by \cfunction{PyArg_ParseTuple()}.  Otherwise the
string value of the argument has been copied to the local variable
\cdata{command}.  This is a pointer assignment and you are not supposed
to modify the string to which it points (so in Standard C, the variable
\cdata{command} should properly be declared as \samp{const char
*command}).

The next statement is a call to the \UNIX{} function
\cfunction{system()}, passing it the string we just got from
\cfunction{PyArg_ParseTuple()}:

\begin{verbatim}
    sts = system(command);
\end{verbatim}

Our \function{spam.system()} function must return the value of
\cdata{sts} as a Python object.  This is done using the function
\cfunction{Py_BuildValue()}, which is something like the inverse of
\cfunction{PyArg_ParseTuple()}: it takes a format string and an
arbitrary number of C values, and returns a new Python object.
More info on \cfunction{Py_BuildValue()} is given later.

\begin{verbatim}
    return Py_BuildValue("i", sts);
\end{verbatim}

In this case, it will return an integer object.  (Yes, even integers
are objects on the heap in Python!)

If you have a C function that returns no useful argument (a function
returning \ctype{void}), the corresponding Python function must return
\code{None}.   You need this idiom to do so (which is implemented by the
\csimplemacro{Py_RETURN_NONE} macro):

\begin{verbatim}
    Py_INCREF(Py_None);
    return Py_None;
\end{verbatim}

\cdata{Py_None} is the C name for the special Python object
\code{None}.  It is a genuine Python object rather than a \NULL{}
pointer, which means ``error'' in most contexts, as we have seen.


\section{The Module's Method Table and Initialization Function
         \label{methodTable}}

I promised to show how \cfunction{spam_system()} is called from Python
programs.  First, we need to list its name and address in a ``method
table'':

\begin{verbatim}
static PyMethodDef SpamMethods[] = {
    ...
    {"system",  spam_system, METH_VARARGS,
     "Execute a shell command."},
    ...
    {NULL, NULL, 0, NULL}        /* Sentinel */
};
\end{verbatim}

Note the third entry (\samp{METH_VARARGS}).  This is a flag telling
the interpreter the calling convention to be used for the C
function.  It should normally always be \samp{METH_VARARGS} or
\samp{METH_VARARGS | METH_KEYWORDS}; a value of \code{0} means that an
obsolete variant of \cfunction{PyArg_ParseTuple()} is used.

When using only \samp{METH_VARARGS}, the function should expect
the Python-level parameters to be passed in as a tuple acceptable for
parsing via \cfunction{PyArg_ParseTuple()}; more information on this
function is provided below.

The \constant{METH_KEYWORDS} bit may be set in the third field if
keyword arguments should be passed to the function.  In this case, the
C function should accept a third \samp{PyObject *} parameter which
will be a dictionary of keywords.  Use
\cfunction{PyArg_ParseTupleAndKeywords()} to parse the arguments to
such a function.

The method table must be passed to the interpreter in the module's
initialization function.  The initialization function must be named
\cfunction{init\var{name}()}, where \var{name} is the name of the
module, and should be the only non-\keyword{static} item defined in
the module file:

\begin{verbatim}
PyMODINIT_FUNC
initspam(void)
{
    (void) Py_InitModule("spam", SpamMethods);
}
\end{verbatim}

Note that PyMODINIT_FUNC declares the function as \code{void} return type, 
declares any special linkage declarations required by the platform, and for 
\Cpp{} declares the function as \code{extern "C"}.

When the Python program imports module \module{spam} for the first
time, \cfunction{initspam()} is called. (See below for comments about
embedding Python.)  It calls
\cfunction{Py_InitModule()}, which creates a ``module object'' (which
is inserted in the dictionary \code{sys.modules} under the key
\code{"spam"}), and inserts built-in function objects into the newly
created module based upon the table (an array of \ctype{PyMethodDef}
structures) that was passed as its second argument.
\cfunction{Py_InitModule()} returns a pointer to the module object
that it creates (which is unused here).  It may abort with a fatal error
for certain errors, or return \NULL{} if the module could not be
initialized satisfactorily.

When embedding Python, the \cfunction{initspam()} function is not
called automatically unless there's an entry in the
\cdata{_PyImport_Inittab} table.  The easiest way to handle this is to 
statically initialize your statically-linked modules by directly
calling \cfunction{initspam()} after the call to
\cfunction{Py_Initialize()}:

\begin{verbatim}
int
main(int argc, char *argv[])
{
    /* Pass argv[0] to the Python interpreter */
    Py_SetProgramName(argv[0]);

    /* Initialize the Python interpreter.  Required. */
    Py_Initialize();

    /* Add a static module */
    initspam();
\end{verbatim}

An example may be found in the file \file{Demo/embed/demo.c} in the
Python source distribution.

\note{Removing entries from \code{sys.modules} or importing
compiled modules into multiple interpreters within a process (or
following a \cfunction{fork()} without an intervening
\cfunction{exec()}) can create problems for some extension modules.
Extension module authors should exercise caution when initializing
internal data structures.}

A more substantial example module is included in the Python source
distribution as \file{Modules/xxmodule.c}.  This file may be used as a 
template or simply read as an example.  The \program{modulator.py}
script included in the source distribution or Windows install provides 
a simple graphical user interface for declaring the functions and
objects which a module should implement, and can generate a template
which can be filled in.  The script lives in the
\file{Tools/modulator/} directory; see the \file{README} file there
for more information.


\section{Compilation and Linkage
         \label{compilation}}

There are two more things to do before you can use your new extension:
compiling and linking it with the Python system.  If you use dynamic
loading, the details may depend on the style of dynamic loading your
system uses; see the chapters about building extension modules
(chapter \ref{building}) and additional information that pertains only
to building on Windows (chapter \ref{building-on-windows}) for more
information about this.

If you can't use dynamic loading, or if you want to make your module a
permanent part of the Python interpreter, you will have to change the
configuration setup and rebuild the interpreter.  Luckily, this is
very simple on \UNIX: just place your file (\file{spammodule.c} for
example) in the \file{Modules/} directory of an unpacked source
distribution, add a line to the file \file{Modules/Setup.local}
describing your file:

\begin{verbatim}
spam spammodule.o
\end{verbatim}

and rebuild the interpreter by running \program{make} in the toplevel
directory.  You can also run \program{make} in the \file{Modules/}
subdirectory, but then you must first rebuild \file{Makefile}
there by running `\program{make} Makefile'.  (This is necessary each
time you change the \file{Setup} file.)

If your module requires additional libraries to link with, these can
be listed on the line in the configuration file as well, for instance:

\begin{verbatim}
spam spammodule.o -lX11
\end{verbatim}

\section{Calling Python Functions from C
         \label{callingPython}}

So far we have concentrated on making C functions callable from
Python.  The reverse is also useful: calling Python functions from C.
This is especially the case for libraries that support so-called
``callback'' functions.  If a C interface makes use of callbacks, the
equivalent Python often needs to provide a callback mechanism to the
Python programmer; the implementation will require calling the Python
callback functions from a C callback.  Other uses are also imaginable.

Fortunately, the Python interpreter is easily called recursively, and
there is a standard interface to call a Python function.  (I won't
dwell on how to call the Python parser with a particular string as
input --- if you're interested, have a look at the implementation of
the \programopt{-c} command line option in \file{Python/pythonmain.c}
from the Python source code.)

Calling a Python function is easy.  First, the Python program must
somehow pass you the Python function object.  You should provide a
function (or some other interface) to do this.  When this function is
called, save a pointer to the Python function object (be careful to
\cfunction{Py_INCREF()} it!) in a global variable --- or wherever you
see fit. For example, the following function might be part of a module
definition:

\begin{verbatim}
static PyObject *my_callback = NULL;

static PyObject *
my_set_callback(PyObject *dummy, PyObject *args)
{
    PyObject *result = NULL;
    PyObject *temp;

    if (PyArg_ParseTuple(args, "O:set_callback", &temp)) {
        if (!PyCallable_Check(temp)) {
            PyErr_SetString(PyExc_TypeError, "parameter must be callable");
            return NULL;
        }
        Py_XINCREF(temp);         /* Add a reference to new callback */
        Py_XDECREF(my_callback);  /* Dispose of previous callback */
        my_callback = temp;       /* Remember new callback */
        /* Boilerplate to return "None" */
        Py_INCREF(Py_None);
        result = Py_None;
    }
    return result;
}
\end{verbatim}

This function must be registered with the interpreter using the
\constant{METH_VARARGS} flag; this is described in section
\ref{methodTable}, ``The Module's Method Table and Initialization
Function.''  The \cfunction{PyArg_ParseTuple()} function and its
arguments are documented in section~\ref{parseTuple}, ``Extracting
Parameters in Extension Functions.''

The macros \cfunction{Py_XINCREF()} and \cfunction{Py_XDECREF()}
increment/decrement the reference count of an object and are safe in
the presence of \NULL{} pointers (but note that \var{temp} will not be 
\NULL{} in this context).  More info on them in
section~\ref{refcounts}, ``Reference Counts.''

Later, when it is time to call the function, you call the C function
\cfunction{PyEval_CallObject()}.\ttindex{PyEval_CallObject()}  This
function has two arguments, both pointers to arbitrary Python objects:
the Python function, and the argument list.  The argument list must
always be a tuple object, whose length is the number of arguments.  To
call the Python function with no arguments, pass an empty tuple; to
call it with one argument, pass a singleton tuple.
\cfunction{Py_BuildValue()} returns a tuple when its format string
consists of zero or more format codes between parentheses.  For
example:

\begin{verbatim}
    int arg;
    PyObject *arglist;
    PyObject *result;
    ...
    arg = 123;
    ...
    /* Time to call the callback */
    arglist = Py_BuildValue("(i)", arg);
    result = PyEval_CallObject(my_callback, arglist);
    Py_DECREF(arglist);
\end{verbatim}

\cfunction{PyEval_CallObject()} returns a Python object pointer: this is
the return value of the Python function.  \cfunction{PyEval_CallObject()} is
``reference-count-neutral'' with respect to its arguments.  In the
example a new tuple was created to serve as the argument list, which
is \cfunction{Py_DECREF()}-ed immediately after the call.

The return value of \cfunction{PyEval_CallObject()} is ``new'': either it
is a brand new object, or it is an existing object whose reference
count has been incremented.  So, unless you want to save it in a
global variable, you should somehow \cfunction{Py_DECREF()} the result,
even (especially!) if you are not interested in its value.

Before you do this, however, it is important to check that the return
value isn't \NULL.  If it is, the Python function terminated by
raising an exception.  If the C code that called
\cfunction{PyEval_CallObject()} is called from Python, it should now
return an error indication to its Python caller, so the interpreter
can print a stack trace, or the calling Python code can handle the
exception.  If this is not possible or desirable, the exception should
be cleared by calling \cfunction{PyErr_Clear()}.  For example:

\begin{verbatim}
    if (result == NULL)
        return NULL; /* Pass error back */
    ...use result...
    Py_DECREF(result); 
\end{verbatim}

Depending on the desired interface to the Python callback function,
you may also have to provide an argument list to
\cfunction{PyEval_CallObject()}.  In some cases the argument list is
also provided by the Python program, through the same interface that
specified the callback function.  It can then be saved and used in the
same manner as the function object.  In other cases, you may have to
construct a new tuple to pass as the argument list.  The simplest way
to do this is to call \cfunction{Py_BuildValue()}.  For example, if
you want to pass an integral event code, you might use the following
code:

\begin{verbatim}
    PyObject *arglist;
    ...
    arglist = Py_BuildValue("(l)", eventcode);
    result = PyEval_CallObject(my_callback, arglist);
    Py_DECREF(arglist);
    if (result == NULL)
        return NULL; /* Pass error back */
    /* Here maybe use the result */
    Py_DECREF(result);
\end{verbatim}

Note the placement of \samp{Py_DECREF(arglist)} immediately after the
call, before the error check!  Also note that strictly spoken this
code is not complete: \cfunction{Py_BuildValue()} may run out of
memory, and this should be checked.


\section{Extracting Parameters in Extension Functions
         \label{parseTuple}}

\ttindex{PyArg_ParseTuple()}

The \cfunction{PyArg_ParseTuple()} function is declared as follows:

\begin{verbatim}
int PyArg_ParseTuple(PyObject *arg, char *format, ...);
\end{verbatim}

The \var{arg} argument must be a tuple object containing an argument
list passed from Python to a C function.  The \var{format} argument
must be a format string, whose syntax is explained in
``\ulink{Parsing arguments and building
values}{../api/arg-parsing.html}'' in the
\citetitle[../api/api.html]{Python/C API Reference Manual}.  The
remaining arguments must be addresses of variables whose type is
determined by the format string.

Note that while \cfunction{PyArg_ParseTuple()} checks that the Python
arguments have the required types, it cannot check the validity of the
addresses of C variables passed to the call: if you make mistakes
there, your code will probably crash or at least overwrite random bits
in memory.  So be careful!

Note that any Python object references which are provided to the
caller are \emph{borrowed} references; do not decrement their
reference count!

Some example calls:

\begin{verbatim}
    int ok;
    int i, j;
    long k, l;
    const char *s;
    int size;

    ok = PyArg_ParseTuple(args, ""); /* No arguments */
        /* Python call: f() */
\end{verbatim}

\begin{verbatim}
    ok = PyArg_ParseTuple(args, "s", &s); /* A string */
        /* Possible Python call: f('whoops!') */
\end{verbatim}

\begin{verbatim}
    ok = PyArg_ParseTuple(args, "lls", &k, &l, &s); /* Two longs and a string */
        /* Possible Python call: f(1, 2, 'three') */
\end{verbatim}

\begin{verbatim}
    ok = PyArg_ParseTuple(args, "(ii)s#", &i, &j, &s, &size);
        /* A pair of ints and a string, whose size is also returned */
        /* Possible Python call: f((1, 2), 'three') */
\end{verbatim}

\begin{verbatim}
    {
        const char *file;
        const char *mode = "r";
        int bufsize = 0;
        ok = PyArg_ParseTuple(args, "s|si", &file, &mode, &bufsize);
        /* A string, and optionally another string and an integer */
        /* Possible Python calls:
           f('spam')
           f('spam', 'w')
           f('spam', 'wb', 100000) */
    }
\end{verbatim}

\begin{verbatim}
    {
        int left, top, right, bottom, h, v;
        ok = PyArg_ParseTuple(args, "((ii)(ii))(ii)",
                 &left, &top, &right, &bottom, &h, &v);
        /* A rectangle and a point */
        /* Possible Python call:
           f(((0, 0), (400, 300)), (10, 10)) */
    }
\end{verbatim}

\begin{verbatim}
    {
        Py_complex c;
        ok = PyArg_ParseTuple(args, "D:myfunction", &c);
        /* a complex, also providing a function name for errors */
        /* Possible Python call: myfunction(1+2j) */
    }
\end{verbatim}


\section{Keyword Parameters for Extension Functions
         \label{parseTupleAndKeywords}}

\ttindex{PyArg_ParseTupleAndKeywords()}

The \cfunction{PyArg_ParseTupleAndKeywords()} function is declared as
follows:

\begin{verbatim}
int PyArg_ParseTupleAndKeywords(PyObject *arg, PyObject *kwdict,
                                char *format, char *kwlist[], ...);
\end{verbatim}

The \var{arg} and \var{format} parameters are identical to those of the
\cfunction{PyArg_ParseTuple()} function.  The \var{kwdict} parameter
is the dictionary of keywords received as the third parameter from the
Python runtime.  The \var{kwlist} parameter is a \NULL-terminated
list of strings which identify the parameters; the names are matched
with the type information from \var{format} from left to right.  On
success, \cfunction{PyArg_ParseTupleAndKeywords()} returns true,
otherwise it returns false and raises an appropriate exception.

\note{Nested tuples cannot be parsed when using keyword
arguments!  Keyword parameters passed in which are not present in the
\var{kwlist} will cause \exception{TypeError} to be raised.}

Here is an example module which uses keywords, based on an example by
Geoff Philbrick (\email{philbrick@hks.com}):%
\index{Philbrick, Geoff}

\begin{verbatim}
#include "Python.h"

static PyObject *
keywdarg_parrot(PyObject *self, PyObject *args, PyObject *keywds)
{  
    int voltage;
    char *state = "a stiff";
    char *action = "voom";
    char *type = "Norwegian Blue";

    static char *kwlist[] = {"voltage", "state", "action", "type", NULL};

    if (!PyArg_ParseTupleAndKeywords(args, keywds, "i|sss", kwlist, 
                                     &voltage, &state, &action, &type))
        return NULL; 
  
    printf("-- This parrot wouldn't %s if you put %i Volts through it.\n", 
           action, voltage);
    printf("-- Lovely plumage, the %s -- It's %s!\n", type, state);

    Py_INCREF(Py_None);

    return Py_None;
}

static PyMethodDef keywdarg_methods[] = {
    /* The cast of the function is necessary since PyCFunction values
     * only take two PyObject* parameters, and keywdarg_parrot() takes
     * three.
     */
    {"parrot", (PyCFunction)keywdarg_parrot, METH_VARARGS | METH_KEYWORDS,
     "Print a lovely skit to standard output."},
    {NULL, NULL, 0, NULL}   /* sentinel */
};
\end{verbatim}

\begin{verbatim}
void
initkeywdarg(void)
{
  /* Create the module and add the functions */
  Py_InitModule("keywdarg", keywdarg_methods);
}
\end{verbatim}


\section{Building Arbitrary Values
         \label{buildValue}}

This function is the counterpart to \cfunction{PyArg_ParseTuple()}.  It is
declared as follows:

\begin{verbatim}
PyObject *Py_BuildValue(char *format, ...);
\end{verbatim}

It recognizes a set of format units similar to the ones recognized by
\cfunction{PyArg_ParseTuple()}, but the arguments (which are input to the
function, not output) must not be pointers, just values.  It returns a
new Python object, suitable for returning from a C function called
from Python.

One difference with \cfunction{PyArg_ParseTuple()}: while the latter
requires its first argument to be a tuple (since Python argument lists
are always represented as tuples internally),
\cfunction{Py_BuildValue()} does not always build a tuple.  It builds
a tuple only if its format string contains two or more format units.
If the format string is empty, it returns \code{None}; if it contains
exactly one format unit, it returns whatever object is described by
that format unit.  To force it to return a tuple of size 0 or one,
parenthesize the format string.

Examples (to the left the call, to the right the resulting Python value):

\begin{verbatim}
    Py_BuildValue("")                        None
    Py_BuildValue("i", 123)                  123
    Py_BuildValue("iii", 123, 456, 789)      (123, 456, 789)
    Py_BuildValue("s", "hello")              'hello'
    Py_BuildValue("y", "hello")              b'hello'
    Py_BuildValue("ss", "hello", "world")    ('hello', 'world')
    Py_BuildValue("s#", "hello", 4)          'hell'
    Py_BuildValue("y#", "hello", 4)          b'hell'
    Py_BuildValue("()")                      ()
    Py_BuildValue("(i)", 123)                (123,)
    Py_BuildValue("(ii)", 123, 456)          (123, 456)
    Py_BuildValue("(i,i)", 123, 456)         (123, 456)
    Py_BuildValue("[i,i]", 123, 456)         [123, 456]
    Py_BuildValue("{s:i,s:i}",
                  "abc", 123, "def", 456)    {'abc': 123, 'def': 456}
    Py_BuildValue("((ii)(ii)) (ii)",
                  1, 2, 3, 4, 5, 6)          (((1, 2), (3, 4)), (5, 6))
\end{verbatim}


\section{Reference Counts
         \label{refcounts}}

In languages like C or \Cpp, the programmer is responsible for
dynamic allocation and deallocation of memory on the heap.  In C,
this is done using the functions \cfunction{malloc()} and
\cfunction{free()}.  In \Cpp, the operators \keyword{new} and
\keyword{delete} are used with essentially the same meaning and
we'll restrict the following discussion to the C case.

Every block of memory allocated with \cfunction{malloc()} should
eventually be returned to the pool of available memory by exactly one
call to \cfunction{free()}.  It is important to call
\cfunction{free()} at the right time.  If a block's address is
forgotten but \cfunction{free()} is not called for it, the memory it
occupies cannot be reused until the program terminates.  This is
called a \dfn{memory leak}.  On the other hand, if a program calls
\cfunction{free()} for a block and then continues to use the block, it
creates a conflict with re-use of the block through another
\cfunction{malloc()} call.  This is called \dfn{using freed memory}.
It has the same bad consequences as referencing uninitialized data ---
core dumps, wrong results, mysterious crashes.

Common causes of memory leaks are unusual paths through the code.  For
instance, a function may allocate a block of memory, do some
calculation, and then free the block again.  Now a change in the
requirements for the function may add a test to the calculation that
detects an error condition and can return prematurely from the
function.  It's easy to forget to free the allocated memory block when
taking this premature exit, especially when it is added later to the
code.  Such leaks, once introduced, often go undetected for a long
time: the error exit is taken only in a small fraction of all calls,
and most modern machines have plenty of virtual memory, so the leak
only becomes apparent in a long-running process that uses the leaking
function frequently.  Therefore, it's important to prevent leaks from
happening by having a coding convention or strategy that minimizes
this kind of errors.

Since Python makes heavy use of \cfunction{malloc()} and
\cfunction{free()}, it needs a strategy to avoid memory leaks as well
as the use of freed memory.  The chosen method is called
\dfn{reference counting}.  The principle is simple: every object
contains a counter, which is incremented when a reference to the
object is stored somewhere, and which is decremented when a reference
to it is deleted.  When the counter reaches zero, the last reference
to the object has been deleted and the object is freed.

An alternative strategy is called \dfn{automatic garbage collection}.
(Sometimes, reference counting is also referred to as a garbage
collection strategy, hence my use of ``automatic'' to distinguish the
two.)  The big advantage of automatic garbage collection is that the
user doesn't need to call \cfunction{free()} explicitly.  (Another claimed
advantage is an improvement in speed or memory usage --- this is no
hard fact however.)  The disadvantage is that for C, there is no
truly portable automatic garbage collector, while reference counting
can be implemented portably (as long as the functions \cfunction{malloc()}
and \cfunction{free()} are available --- which the C Standard guarantees).
Maybe some day a sufficiently portable automatic garbage collector
will be available for C.  Until then, we'll have to live with
reference counts.

While Python uses the traditional reference counting implementation,
it also offers a cycle detector that works to detect reference
cycles.  This allows applications to not worry about creating direct
or indirect circular references; these are the weakness of garbage
collection implemented using only reference counting.  Reference
cycles consist of objects which contain (possibly indirect) references
to themselves, so that each object in the cycle has a reference count
which is non-zero.  Typical reference counting implementations are not
able to reclaim the memory belonging to any objects in a reference
cycle, or referenced from the objects in the cycle, even though there
are no further references to the cycle itself.

The cycle detector is able to detect garbage cycles and can reclaim
them so long as there are no finalizers implemented in Python
(\method{__del__()} methods).  When there are such finalizers, the
detector exposes the cycles through the \ulink{\module{gc}
module}{../lib/module-gc.html} (specifically, the \code{garbage}
variable in that module).  The \module{gc} module also exposes a way
to run the detector (the \function{collect()} function), as well as
configuration interfaces and the ability to disable the detector at
runtime.  The cycle detector is considered an optional component;
though it is included by default, it can be disabled at build time
using the \longprogramopt{without-cycle-gc} option to the
\program{configure} script on \UNIX{} platforms (including Mac OS X)
or by removing the definition of \code{WITH_CYCLE_GC} in the
\file{pyconfig.h} header on other platforms.  If the cycle detector is
disabled in this way, the \module{gc} module will not be available.


\subsection{Reference Counting in Python
            \label{refcountsInPython}}

There are two macros, \code{Py_INCREF(x)} and \code{Py_DECREF(x)},
which handle the incrementing and decrementing of the reference count.
\cfunction{Py_DECREF()} also frees the object when the count reaches zero.
For flexibility, it doesn't call \cfunction{free()} directly --- rather, it
makes a call through a function pointer in the object's \dfn{type
object}.  For this purpose (and others), every object also contains a
pointer to its type object.

The big question now remains: when to use \code{Py_INCREF(x)} and
\code{Py_DECREF(x)}?  Let's first introduce some terms.  Nobody
``owns'' an object; however, you can \dfn{own a reference} to an
object.  An object's reference count is now defined as the number of
owned references to it.  The owner of a reference is responsible for
calling \cfunction{Py_DECREF()} when the reference is no longer
needed.  Ownership of a reference can be transferred.  There are three
ways to dispose of an owned reference: pass it on, store it, or call
\cfunction{Py_DECREF()}.  Forgetting to dispose of an owned reference
creates a memory leak.

It is also possible to \dfn{borrow}\footnote{The metaphor of
``borrowing'' a reference is not completely correct: the owner still
has a copy of the reference.} a reference to an object.  The borrower
of a reference should not call \cfunction{Py_DECREF()}.  The borrower must
not hold on to the object longer than the owner from which it was
borrowed.  Using a borrowed reference after the owner has disposed of
it risks using freed memory and should be avoided
completely.\footnote{Checking that the reference count is at least 1
\strong{does not work} --- the reference count itself could be in
freed memory and may thus be reused for another object!}

The advantage of borrowing over owning a reference is that you don't
need to take care of disposing of the reference on all possible paths
through the code --- in other words, with a borrowed reference you
don't run the risk of leaking when a premature exit is taken.  The
disadvantage of borrowing over leaking is that there are some subtle
situations where in seemingly correct code a borrowed reference can be
used after the owner from which it was borrowed has in fact disposed
of it.

A borrowed reference can be changed into an owned reference by calling
\cfunction{Py_INCREF()}.  This does not affect the status of the owner from
which the reference was borrowed --- it creates a new owned reference,
and gives full owner responsibilities (the new owner must
dispose of the reference properly, as well as the previous owner).


\subsection{Ownership Rules
            \label{ownershipRules}}

Whenever an object reference is passed into or out of a function, it
is part of the function's interface specification whether ownership is
transferred with the reference or not.

Most functions that return a reference to an object pass on ownership
with the reference.  In particular, all functions whose function it is
to create a new object, such as \cfunction{PyInt_FromLong()} and
\cfunction{Py_BuildValue()}, pass ownership to the receiver.  Even if
the object is not actually new, you still receive ownership of a new
reference to that object.  For instance, \cfunction{PyInt_FromLong()}
maintains a cache of popular values and can return a reference to a
cached item.

Many functions that extract objects from other objects also transfer
ownership with the reference, for instance
\cfunction{PyObject_GetAttrString()}.  The picture is less clear, here,
however, since a few common routines are exceptions:
\cfunction{PyTuple_GetItem()}, \cfunction{PyList_GetItem()},
\cfunction{PyDict_GetItem()}, and \cfunction{PyDict_GetItemString()}
all return references that you borrow from the tuple, list or
dictionary.

The function \cfunction{PyImport_AddModule()} also returns a borrowed
reference, even though it may actually create the object it returns:
this is possible because an owned reference to the object is stored in
\code{sys.modules}.

When you pass an object reference into another function, in general,
the function borrows the reference from you --- if it needs to store
it, it will use \cfunction{Py_INCREF()} to become an independent
owner.  There are exactly two important exceptions to this rule:
\cfunction{PyTuple_SetItem()} and \cfunction{PyList_SetItem()}.  These
functions take over ownership of the item passed to them --- even if
they fail!  (Note that \cfunction{PyDict_SetItem()} and friends don't
take over ownership --- they are ``normal.'')

When a C function is called from Python, it borrows references to its
arguments from the caller.  The caller owns a reference to the object,
so the borrowed reference's lifetime is guaranteed until the function
returns.  Only when such a borrowed reference must be stored or passed
on, it must be turned into an owned reference by calling
\cfunction{Py_INCREF()}.

The object reference returned from a C function that is called from
Python must be an owned reference --- ownership is transferred from
the function to its caller.


\subsection{Thin Ice
            \label{thinIce}}

There are a few situations where seemingly harmless use of a borrowed
reference can lead to problems.  These all have to do with implicit
invocations of the interpreter, which can cause the owner of a
reference to dispose of it.

The first and most important case to know about is using
\cfunction{Py_DECREF()} on an unrelated object while borrowing a
reference to a list item.  For instance:

\begin{verbatim}
void
bug(PyObject *list)
{
    PyObject *item = PyList_GetItem(list, 0);

    PyList_SetItem(list, 1, PyInt_FromLong(0L));
    PyObject_Print(item, stdout, 0); /* BUG! */
}
\end{verbatim}

This function first borrows a reference to \code{list[0]}, then
replaces \code{list[1]} with the value \code{0}, and finally prints
the borrowed reference.  Looks harmless, right?  But it's not!

Let's follow the control flow into \cfunction{PyList_SetItem()}.  The list
owns references to all its items, so when item 1 is replaced, it has
to dispose of the original item 1.  Now let's suppose the original
item 1 was an instance of a user-defined class, and let's further
suppose that the class defined a \method{__del__()} method.  If this
class instance has a reference count of 1, disposing of it will call
its \method{__del__()} method.

Since it is written in Python, the \method{__del__()} method can execute
arbitrary Python code.  Could it perhaps do something to invalidate
the reference to \code{item} in \cfunction{bug()}?  You bet!  Assuming
that the list passed into \cfunction{bug()} is accessible to the
\method{__del__()} method, it could execute a statement to the effect of
\samp{del list[0]}, and assuming this was the last reference to that
object, it would free the memory associated with it, thereby
invalidating \code{item}.

The solution, once you know the source of the problem, is easy:
temporarily increment the reference count.  The correct version of the
function reads:

\begin{verbatim}
void
no_bug(PyObject *list)
{
    PyObject *item = PyList_GetItem(list, 0);

    Py_INCREF(item);
    PyList_SetItem(list, 1, PyInt_FromLong(0L));
    PyObject_Print(item, stdout, 0);
    Py_DECREF(item);
}
\end{verbatim}

This is a true story.  An older version of Python contained variants
of this bug and someone spent a considerable amount of time in a C
debugger to figure out why his \method{__del__()} methods would fail...

The second case of problems with a borrowed reference is a variant
involving threads.  Normally, multiple threads in the Python
interpreter can't get in each other's way, because there is a global
lock protecting Python's entire object space.  However, it is possible
to temporarily release this lock using the macro
\csimplemacro{Py_BEGIN_ALLOW_THREADS}, and to re-acquire it using
\csimplemacro{Py_END_ALLOW_THREADS}.  This is common around blocking
I/O calls, to let other threads use the processor while waiting for
the I/O to complete.  Obviously, the following function has the same
problem as the previous one:

\begin{verbatim}
void
bug(PyObject *list)
{
    PyObject *item = PyList_GetItem(list, 0);
    Py_BEGIN_ALLOW_THREADS
    ...some blocking I/O call...
    Py_END_ALLOW_THREADS
    PyObject_Print(item, stdout, 0); /* BUG! */
}
\end{verbatim}


\subsection{NULL Pointers
            \label{nullPointers}}

In general, functions that take object references as arguments do not
expect you to pass them \NULL{} pointers, and will dump core (or
cause later core dumps) if you do so.  Functions that return object
references generally return \NULL{} only to indicate that an
exception occurred.  The reason for not testing for \NULL{}
arguments is that functions often pass the objects they receive on to
other function --- if each function were to test for \NULL,
there would be a lot of redundant tests and the code would run more
slowly.

It is better to test for \NULL{} only at the ``source:'' when a
pointer that may be \NULL{} is received, for example, from
\cfunction{malloc()} or from a function that may raise an exception.

The macros \cfunction{Py_INCREF()} and \cfunction{Py_DECREF()}
do not check for \NULL{} pointers --- however, their variants
\cfunction{Py_XINCREF()} and \cfunction{Py_XDECREF()} do.

The macros for checking for a particular object type
(\code{Py\var{type}_Check()}) don't check for \NULL{} pointers ---
again, there is much code that calls several of these in a row to test
an object against various different expected types, and this would
generate redundant tests.  There are no variants with \NULL{}
checking.

The C function calling mechanism guarantees that the argument list
passed to C functions (\code{args} in the examples) is never
\NULL{} --- in fact it guarantees that it is always a tuple.\footnote{
These guarantees don't hold when you use the ``old'' style
calling convention --- this is still found in much existing code.}

It is a severe error to ever let a \NULL{} pointer ``escape'' to
the Python user.

% Frank Stajano:
% A pedagogically buggy example, along the lines of the previous listing, 
% would be helpful here -- showing in more concrete terms what sort of 
% actions could cause the problem. I can't very well imagine it from the 
% description.


\section{Writing Extensions in \Cpp
         \label{cplusplus}}

It is possible to write extension modules in \Cpp.  Some restrictions
apply.  If the main program (the Python interpreter) is compiled and
linked by the C compiler, global or static objects with constructors
cannot be used.  This is not a problem if the main program is linked
by the \Cpp{} compiler.  Functions that will be called by the
Python interpreter (in particular, module initialization functions)
have to be declared using \code{extern "C"}.
It is unnecessary to enclose the Python header files in
\code{extern "C" \{...\}} --- they use this form already if the symbol
\samp{__cplusplus} is defined (all recent \Cpp{} compilers define this
symbol).


\section{Providing a C API for an Extension Module
         \label{using-cobjects}}
\sectionauthor{Konrad Hinsen}{hinsen@cnrs-orleans.fr}

Many extension modules just provide new functions and types to be
used from Python, but sometimes the code in an extension module can
be useful for other extension modules. For example, an extension
module could implement a type ``collection'' which works like lists
without order. Just like the standard Python list type has a C API
which permits extension modules to create and manipulate lists, this
new collection type should have a set of C functions for direct
manipulation from other extension modules.

At first sight this seems easy: just write the functions (without
declaring them \keyword{static}, of course), provide an appropriate
header file, and document the C API. And in fact this would work if
all extension modules were always linked statically with the Python
interpreter. When modules are used as shared libraries, however, the
symbols defined in one module may not be visible to another module.
The details of visibility depend on the operating system; some systems
use one global namespace for the Python interpreter and all extension
modules (Windows, for example), whereas others require an explicit
list of imported symbols at module link time (AIX is one example), or
offer a choice of different strategies (most Unices). And even if
symbols are globally visible, the module whose functions one wishes to
call might not have been loaded yet!

Portability therefore requires not to make any assumptions about
symbol visibility. This means that all symbols in extension modules
should be declared \keyword{static}, except for the module's
initialization function, in order to avoid name clashes with other
extension modules (as discussed in section~\ref{methodTable}). And it
means that symbols that \emph{should} be accessible from other
extension modules must be exported in a different way.

Python provides a special mechanism to pass C-level information
(pointers) from one extension module to another one: CObjects.
A CObject is a Python data type which stores a pointer (\ctype{void
*}).  CObjects can only be created and accessed via their C API, but
they can be passed around like any other Python object. In particular, 
they can be assigned to a name in an extension module's namespace.
Other extension modules can then import this module, retrieve the
value of this name, and then retrieve the pointer from the CObject.

There are many ways in which CObjects can be used to export the C API
of an extension module. Each name could get its own CObject, or all C
API pointers could be stored in an array whose address is published in
a CObject. And the various tasks of storing and retrieving the pointers
can be distributed in different ways between the module providing the
code and the client modules.

The following example demonstrates an approach that puts most of the
burden on the writer of the exporting module, which is appropriate
for commonly used library modules. It stores all C API pointers
(just one in the example!) in an array of \ctype{void} pointers which
becomes the value of a CObject. The header file corresponding to
the module provides a macro that takes care of importing the module
and retrieving its C API pointers; client modules only have to call
this macro before accessing the C API.

The exporting module is a modification of the \module{spam} module from
section~\ref{simpleExample}. The function \function{spam.system()}
does not call the C library function \cfunction{system()} directly,
but a function \cfunction{PySpam_System()}, which would of course do
something more complicated in reality (such as adding ``spam'' to
every command). This function \cfunction{PySpam_System()} is also
exported to other extension modules.

The function \cfunction{PySpam_System()} is a plain C function,
declared \keyword{static} like everything else:

\begin{verbatim}
static int
PySpam_System(const char *command)
{
    return system(command);
}
\end{verbatim}

The function \cfunction{spam_system()} is modified in a trivial way:

\begin{verbatim}
static PyObject *
spam_system(PyObject *self, PyObject *args)
{
    const char *command;
    int sts;

    if (!PyArg_ParseTuple(args, "s", &command))
        return NULL;
    sts = PySpam_System(command);
    return Py_BuildValue("i", sts);
}
\end{verbatim}

In the beginning of the module, right after the line

\begin{verbatim}
#include "Python.h"
\end{verbatim}

two more lines must be added:

\begin{verbatim}
#define SPAM_MODULE
#include "spammodule.h"
\end{verbatim}

The \code{\#define} is used to tell the header file that it is being
included in the exporting module, not a client module. Finally,
the module's initialization function must take care of initializing
the C API pointer array:

\begin{verbatim}
PyMODINIT_FUNC
initspam(void)
{
    PyObject *m;
    static void *PySpam_API[PySpam_API_pointers];
    PyObject *c_api_object;

    m = Py_InitModule("spam", SpamMethods);
    if (m == NULL)
        return;

    /* Initialize the C API pointer array */
    PySpam_API[PySpam_System_NUM] = (void *)PySpam_System;

    /* Create a CObject containing the API pointer array's address */
    c_api_object = PyCObject_FromVoidPtr((void *)PySpam_API, NULL);

    if (c_api_object != NULL)
        PyModule_AddObject(m, "_C_API", c_api_object);
}
\end{verbatim}

Note that \code{PySpam_API} is declared \keyword{static}; otherwise
the pointer array would disappear when \function{initspam()} terminates!

The bulk of the work is in the header file \file{spammodule.h},
which looks like this:

\begin{verbatim}
#ifndef Py_SPAMMODULE_H
#define Py_SPAMMODULE_H
#ifdef __cplusplus
extern "C" {
#endif

/* Header file for spammodule */

/* C API functions */
#define PySpam_System_NUM 0
#define PySpam_System_RETURN int
#define PySpam_System_PROTO (const char *command)

/* Total number of C API pointers */
#define PySpam_API_pointers 1


#ifdef SPAM_MODULE
/* This section is used when compiling spammodule.c */

static PySpam_System_RETURN PySpam_System PySpam_System_PROTO;

#else
/* This section is used in modules that use spammodule's API */

static void **PySpam_API;

#define PySpam_System \
 (*(PySpam_System_RETURN (*)PySpam_System_PROTO) PySpam_API[PySpam_System_NUM])

/* Return -1 and set exception on error, 0 on success. */
static int
import_spam(void)
{
    PyObject *module = PyImport_ImportModule("spam");

    if (module != NULL) {
        PyObject *c_api_object = PyObject_GetAttrString(module, "_C_API");
        if (c_api_object == NULL)
            return -1;
        if (PyCObject_Check(c_api_object))
            PySpam_API = (void **)PyCObject_AsVoidPtr(c_api_object);
        Py_DECREF(c_api_object);
    }
    return 0;
}

#endif

#ifdef __cplusplus
}
#endif

#endif /* !defined(Py_SPAMMODULE_H) */
\end{verbatim}

All that a client module must do in order to have access to the
function \cfunction{PySpam_System()} is to call the function (or
rather macro) \cfunction{import_spam()} in its initialization
function:

\begin{verbatim}
PyMODINIT_FUNC
initclient(void)
{
    PyObject *m;

    m = Py_InitModule("client", ClientMethods);
    if (m == NULL)
        return;
    if (import_spam() < 0)
        return;
    /* additional initialization can happen here */
}
\end{verbatim}

The main disadvantage of this approach is that the file
\file{spammodule.h} is rather complicated. However, the
basic structure is the same for each function that is
exported, so it has to be learned only once.

Finally it should be mentioned that CObjects offer additional
functionality, which is especially useful for memory allocation and
deallocation of the pointer stored in a CObject. The details
are described in the \citetitle[../api/api.html]{Python/C API
Reference Manual} in the section
``\ulink{CObjects}{../api/cObjects.html}'' and in the implementation
of CObjects (files \file{Include/cobject.h} and
\file{Objects/cobject.c} in the Python source code distribution).

\chapter{Defining New Types
        \label{defining-new-types}}
\sectionauthor{Michael Hudson}{mwh@python.net}
\sectionauthor{Dave Kuhlman}{dkuhlman@rexx.com}
\sectionauthor{Jim Fulton}{jim@zope.com}

As mentioned in the last chapter, Python allows the writer of an
extension module to define new types that can be manipulated from
Python code, much like strings and lists in core Python.

This is not hard; the code for all extension types follows a pattern,
but there are some details that you need to understand before you can
get started.

\begin{notice}
The way new types are defined changed dramatically (and for the
better) in Python 2.2.  This document documents how to define new
types for Python 2.2 and later.  If you need to support older
versions of Python, you will need to refer to older versions of this
documentation.
\end{notice}

\section{The Basics
    \label{dnt-basics}}

The Python runtime sees all Python objects as variables of type
\ctype{PyObject*}.  A \ctype{PyObject} is not a very magnificent
object - it just contains the refcount and a pointer to the object's
``type object''.  This is where the action is; the type object
determines which (C) functions get called when, for instance, an
attribute gets looked up on an object or it is multiplied by another
object.  These C functions are called ``type methods'' to distinguish
them from things like \code{[].append} (which we call ``object
methods'').

So, if you want to define a new object type, you need to create a new
type object.

This sort of thing can only be explained by example, so here's a
minimal, but complete, module that defines a new type:

\verbatiminput{noddy.c}

Now that's quite a bit to take in at once, but hopefully bits will
seem familiar from the last chapter.

The first bit that will be new is:

\begin{verbatim}
typedef struct {
    PyObject_HEAD
} noddy_NoddyObject;
\end{verbatim}

This is what a Noddy object will contain---in this case, nothing more
than every Python object contains, namely a refcount and a pointer to a type
object.  These are the fields the \code{PyObject_HEAD} macro brings
in.  The reason for the macro is to standardize the layout and to
enable special debugging fields in debug builds.  Note that there is
no semicolon after the \code{PyObject_HEAD} macro; one is included in
the macro definition.  Be wary of adding one by accident; it's easy to
do from habit, and your compiler might not complain, but someone
else's probably will!  (On Windows, MSVC is known to call this an
error and refuse to compile the code.)

For contrast, let's take a look at the corresponding definition for
standard Python integers:

\begin{verbatim}
typedef struct {
    PyObject_HEAD
    long ob_ival;
} PyIntObject;
\end{verbatim}

Moving on, we come to the crunch --- the type object.

\begin{verbatim}
static PyTypeObject noddy_NoddyType = {
    PyObject_HEAD_INIT(NULL)
    0,                         /*ob_size*/
    "noddy.Noddy",             /*tp_name*/
    sizeof(noddy_NoddyObject), /*tp_basicsize*/
    0,                         /*tp_itemsize*/
    0,                         /*tp_dealloc*/
    0,                         /*tp_print*/
    0,                         /*tp_getattr*/
    0,                         /*tp_setattr*/
    0,                         /*tp_compare*/
    0,                         /*tp_repr*/
    0,                         /*tp_as_number*/
    0,                         /*tp_as_sequence*/
    0,                         /*tp_as_mapping*/
    0,                         /*tp_hash */
    0,                         /*tp_call*/
    0,                         /*tp_str*/
    0,                         /*tp_getattro*/
    0,                         /*tp_setattro*/
    0,                         /*tp_as_buffer*/
    Py_TPFLAGS_DEFAULT,        /*tp_flags*/
    "Noddy objects",           /* tp_doc */
};
\end{verbatim}

Now if you go and look up the definition of \ctype{PyTypeObject} in
\file{object.h} you'll see that it has many more fields that the
definition above.  The remaining fields will be filled with zeros by
the C compiler, and it's common practice to not specify them
explicitly unless you need them.

This is so important that we're going to pick the top of it apart still
further:

\begin{verbatim}
    PyObject_HEAD_INIT(NULL)
\end{verbatim}

This line is a bit of a wart; what we'd like to write is:

\begin{verbatim}
    PyObject_HEAD_INIT(&PyType_Type)
\end{verbatim}

as the type of a type object is ``type'', but this isn't strictly
conforming C and some compilers complain.  Fortunately, this member
will be filled in for us by \cfunction{PyType_Ready()}.

\begin{verbatim}
    0,                          /* ob_size */
\end{verbatim}

The \member{ob_size} field of the header is not used; its presence in
the type structure is a historical artifact that is maintained for
binary compatibility with extension modules compiled for older
versions of Python.  Always set this field to zero.

\begin{verbatim}
    "noddy.Noddy",              /* tp_name */
\end{verbatim}

The name of our type.  This will appear in the default textual
representation of our objects and in some error messages, for example:

\begin{verbatim}
>>> "" + noddy.new_noddy()
Traceback (most recent call last):
  File "<stdin>", line 1, in ?
TypeError: cannot add type "noddy.Noddy" to string
\end{verbatim}

Note that the name is a dotted name that includes both the module name
and the name of the type within the module. The module in this case is
\module{noddy} and the type is \class{Noddy}, so we set the type name
to \class{noddy.Noddy}.

\begin{verbatim}
    sizeof(noddy_NoddyObject),  /* tp_basicsize */
\end{verbatim}

This is so that Python knows how much memory to allocate when you call
\cfunction{PyObject_New()}.

\note{If you want your type to be subclassable from Python, and your
type has the same \member{tp_basicsize} as its base type, you may
have problems with multiple inheritance.  A Python subclass of your
type will have to list your type first in its \member{__bases__}, or
else it will not be able to call your type's \method{__new__} method
without getting an error.  You can avoid this problem by ensuring
that your type has a larger value for \member{tp_basicsize} than
its base type does.  Most of the time, this will be true anyway,
because either your base type will be \class{object}, or else you will
be adding data members to your base type, and therefore increasing its
size.}

\begin{verbatim}
    0,                          /* tp_itemsize */
\end{verbatim}

This has to do with variable length objects like lists and strings.
Ignore this for now.

Skipping a number of type methods that we don't provide, we set the
class flags to \constant{Py_TPFLAGS_DEFAULT}.

\begin{verbatim}
    Py_TPFLAGS_DEFAULT,        /*tp_flags*/
\end{verbatim}

All types should include this constant in their flags.  It enables all
of the members defined by the current version of Python.

We provide a doc string for the type in \member{tp_doc}.

\begin{verbatim}
    "Noddy objects",           /* tp_doc */
\end{verbatim}

Now we get into the type methods, the things that make your objects
different from the others.  We aren't going to implement any of these
in this version of the module.  We'll expand this example later to
have more interesting behavior.

For now, all we want to be able to do is to create new \class{Noddy}
objects. To enable object creation, we have to provide a
\member{tp_new} implementation. In this case, we can just use the
default implementation provided by the API function
\cfunction{PyType_GenericNew()}.  We'd like to just assign this to the
\member{tp_new} slot, but we can't, for portability sake, On some
platforms or compilers, we can't statically initialize a structure
member with a function defined in another C module, so, instead, we'll
assign the \member{tp_new} slot in the module initialization function
just before calling \cfunction{PyType_Ready()}:

\begin{verbatim}
    noddy_NoddyType.tp_new = PyType_GenericNew;
    if (PyType_Ready(&noddy_NoddyType) < 0)
        return;
\end{verbatim}

All the other type methods are \NULL, so we'll go over them later
--- that's for a later section!

Everything else in the file should be familiar, except for some code
in \cfunction{initnoddy()}:

\begin{verbatim}
    if (PyType_Ready(&noddy_NoddyType) < 0)
        return;
\end{verbatim}

This initializes the \class{Noddy} type, filing in a number of
members, including \member{ob_type} that we initially set to \NULL.

\begin{verbatim}
    PyModule_AddObject(m, "Noddy", (PyObject *)&noddy_NoddyType);
\end{verbatim}

This adds the type to the module dictionary.  This allows us to create
\class{Noddy} instances by calling the \class{Noddy} class:

\begin{verbatim}
>>> import noddy
>>> mynoddy = noddy.Noddy()
\end{verbatim}

That's it!  All that remains is to build it; put the above code in a
file called \file{noddy.c} and

\begin{verbatim}
from distutils.core import setup, Extension
setup(name="noddy", version="1.0",
      ext_modules=[Extension("noddy", ["noddy.c"])])
\end{verbatim}

in a file called \file{setup.py}; then typing

\begin{verbatim}
$ python setup.py build
\end{verbatim} %$ <-- bow to font-lock  ;-(

at a shell should produce a file \file{noddy.so} in a subdirectory;
move to that directory and fire up Python --- you should be able to
\code{import noddy} and play around with Noddy objects.

That wasn't so hard, was it?

Of course, the current Noddy type is pretty uninteresting. It has no
data and doesn't do anything. It can't even be subclassed.

\subsection{Adding data and methods to the Basic example}

Let's expend the basic example to add some data and methods.  Let's
also make the type usable as a base class. We'll create
a new module, \module{noddy2} that adds these capabilities:

\verbatiminput{noddy2.c}

This version of the module has a number of changes.

We've added an extra include:

\begin{verbatim}
#include "structmember.h"
\end{verbatim}

This include provides declarations that we use to handle attributes,
as described a bit later.

The name of the \class{Noddy} object structure has been shortened to
\class{Noddy}.  The type object name has been shortened to
\class{NoddyType}.

The  \class{Noddy} type now has three data attributes, \var{first},
\var{last}, and \var{number}.  The \var{first} and \var{last}
variables are Python strings containing first and last names. The
\var{number} attribute is an integer.

The object structure is updated accordingly:

\begin{verbatim}
typedef struct {
    PyObject_HEAD
    PyObject *first;
    PyObject *last;
    int number;
} Noddy;
\end{verbatim}

Because we now have data to manage, we have to be more careful about
object allocation and deallocation.  At a minimum, we need a
deallocation method:

\begin{verbatim}
static void
Noddy_dealloc(Noddy* self)
{
    Py_XDECREF(self->first);
    Py_XDECREF(self->last);
    self->ob_type->tp_free((PyObject*)self);
}
\end{verbatim}

which is assigned to the \member{tp_dealloc} member:

\begin{verbatim}
    (destructor)Noddy_dealloc, /*tp_dealloc*/
\end{verbatim}

This method decrements the reference counts of the two Python
attributes. We use \cfunction{Py_XDECREF()} here because the
\member{first} and \member{last} members could be \NULL.  It then
calls the \member{tp_free} member of the object's type to free the
object's memory.  Note that the object's type might not be
\class{NoddyType}, because the object may be an instance of a
subclass.

We want to make sure that the first and last names are initialized to
empty strings, so we provide a new method:

\begin{verbatim}
static PyObject *
Noddy_new(PyTypeObject *type, PyObject *args, PyObject *kwds)
{
    Noddy *self;

    self = (Noddy *)type->tp_alloc(type, 0);
    if (self != NULL) {
        self->first = PyString_FromString("");
        if (self->first == NULL)
          {
            Py_DECREF(self);
            return NULL;
          }

        self->last = PyString_FromString("");
        if (self->last == NULL)
          {
            Py_DECREF(self);
            return NULL;
          }

        self->number = 0;
    }

    return (PyObject *)self;
}
\end{verbatim}

and install it in the \member{tp_new} member:

\begin{verbatim}
    Noddy_new,                 /* tp_new */
\end{verbatim}

The new member is responsible for creating (as opposed to
initializing) objects of the type.  It is exposed in Python as the
\method{__new__()} method.  See the paper titled ``Unifying types and
classes in Python'' for a detailed discussion of the \method{__new__()}
method.  One reason to implement a new method is to assure the initial
values of instance variables.  In this case, we use the new method to
make sure that the initial values of the members \member{first} and
\member{last} are not \NULL. If we didn't care whether the initial
values were \NULL, we could have used \cfunction{PyType_GenericNew()} as
our new method, as we did before.  \cfunction{PyType_GenericNew()}
initializes all of the instance variable members to \NULL.

The new method is a static method that is passed the type being
instantiated and any arguments passed when the type was called,
and that returns the new object created. New methods always accept
positional and keyword arguments, but they often ignore the arguments,
leaving the argument handling to initializer methods. Note that if the
type supports subclassing, the type passed may not be the type being
defined.  The new method calls the tp_alloc slot to allocate memory.
We don't fill the \member{tp_alloc} slot ourselves. Rather
\cfunction{PyType_Ready()} fills it for us by inheriting it from our
base class, which is \class{object} by default.  Most types use the
default allocation.

\note{If you are creating a co-operative \member{tp_new} (one that
calls a base type's \member{tp_new} or \method{__new__}), you
must \emph{not} try to determine what method to call using
method resolution order at runtime.  Always statically determine
what type you are going to call, and call its \member{tp_new}
directly, or via \code{type->tp_base->tp_new}.  If you do
not do this, Python subclasses of your type that also inherit
from other Python-defined classes may not work correctly.
(Specifically, you may not be able to create instances of
such subclasses without getting a \exception{TypeError}.)}

We provide an initialization function:

\begin{verbatim}
static int
Noddy_init(Noddy *self, PyObject *args, PyObject *kwds)
{
    PyObject *first=NULL, *last=NULL, *tmp;

    static char *kwlist[] = {"first", "last", "number", NULL};

    if (! PyArg_ParseTupleAndKeywords(args, kwds, "|OOi", kwlist,
                                      &first, &last,
                                      &self->number))
        return -1;

    if (first) {
        tmp = self->first;
        Py_INCREF(first);
        self->first = first;
        Py_XDECREF(tmp);
    }

    if (last) {
        tmp = self->last;
        Py_INCREF(last);
        self->last = last;
        Py_XDECREF(tmp);
    }

    return 0;
}
\end{verbatim}

by filling the \member{tp_init} slot.

\begin{verbatim}
    (initproc)Noddy_init,         /* tp_init */
\end{verbatim}

The \member{tp_init} slot is exposed in Python as the
\method{__init__()} method. It is used to initialize an object after
it's created. Unlike the new method, we can't guarantee that the
initializer is called.  The initializer isn't called when unpickling
objects and it can be overridden.  Our initializer accepts arguments
to provide initial values for our instance. Initializers always accept
positional and keyword arguments.

Initializers can be called multiple times.  Anyone can call the
\method{__init__()} method on our objects.  For this reason, we have
to be extra careful when assigning the new values.  We might be
tempted, for example to assign the \member{first} member like this:

\begin{verbatim}
    if (first) {
        Py_XDECREF(self->first);
        Py_INCREF(first);
        self->first = first;
    }
\end{verbatim}

But this would be risky.  Our type doesn't restrict the type of the
\member{first} member, so it could be any kind of object.  It could
have a destructor that causes code to be executed that tries to
access the \member{first} member.  To be paranoid and protect
ourselves against this possibility, we almost always reassign members
before decrementing their reference counts.  When don't we have to do
this?
\begin{itemize}
\item when we absolutely know that the reference count is greater than
  1
\item when we know that deallocation of the object\footnote{This is
  true when we know that the object is a basic type, like a string or
  a float} will not cause any
  calls back into our type's code
\item when decrementing a reference count in a \member{tp_dealloc}
  handler when garbage-collections is not supported\footnote{We relied
  on this in the \member{tp_dealloc} handler in this example, because
  our type doesn't support garbage collection. Even if a type supports
  garbage collection, there are calls that can be made to ``untrack''
  the object from garbage collection, however, these calls are
  advanced and not covered here.}
\item
\end{itemize}


We want to want to expose our instance variables as attributes. There
are a number of ways to do that. The simplest way is to define member
definitions:

\begin{verbatim}
static PyMemberDef Noddy_members[] = {
    {"first", T_OBJECT_EX, offsetof(Noddy, first), 0,
     "first name"},
    {"last", T_OBJECT_EX, offsetof(Noddy, last), 0,
     "last name"},
    {"number", T_INT, offsetof(Noddy, number), 0,
     "noddy number"},
    {NULL}  /* Sentinel */
};
\end{verbatim}

and put the definitions in the \member{tp_members} slot:

\begin{verbatim}
    Noddy_members,             /* tp_members */
\end{verbatim}

Each member definition has a member name, type, offset, access flags
and documentation string. See the ``Generic Attribute Management''
section below for details.

A disadvantage of this approach is that it doesn't provide a way to
restrict the types of objects that can be assigned to the Python
attributes.  We expect the first and last names to be strings, but any
Python objects can be assigned.  Further, the attributes can be
deleted, setting the C pointers to \NULL.  Even though we can make
sure the members are initialized to non-\NULL{} values, the members can
be set to \NULL{} if the attributes are deleted.

We define a single method, \method{name}, that outputs the objects
name as the concatenation of the first and last names.

\begin{verbatim}
static PyObject *
Noddy_name(Noddy* self)
{
    static PyObject *format = NULL;
    PyObject *args, *result;

    if (format == NULL) {
        format = PyString_FromString("%s %s");
        if (format == NULL)
            return NULL;
    }

    if (self->first == NULL) {
        PyErr_SetString(PyExc_AttributeError, "first");
        return NULL;
    }

    if (self->last == NULL) {
        PyErr_SetString(PyExc_AttributeError, "last");
        return NULL;
    }

    args = Py_BuildValue("OO", self->first, self->last);
    if (args == NULL)
        return NULL;

    result = PyString_Format(format, args);
    Py_DECREF(args);

    return result;
}
\end{verbatim}

The method is implemented as a C function that takes a \class{Noddy} (or
\class{Noddy} subclass) instance as the first argument.  Methods
always take an instance as the first argument. Methods often take
positional and keyword arguments as well, but in this cased we don't
take any and don't need to accept a positional argument tuple or
keyword argument dictionary. This method is equivalent to the Python
method:

\begin{verbatim}
    def name(self):
       return "%s %s" % (self.first, self.last)
\end{verbatim}

Note that we have to check for the possibility that our \member{first}
and \member{last} members are \NULL.  This is because they can be
deleted, in which case they are set to \NULL.  It would be better to
prevent deletion of these attributes and to restrict the attribute
values to be strings.  We'll see how to do that in the next section.

Now that we've defined the method, we need to create an array of
method definitions:

\begin{verbatim}
static PyMethodDef Noddy_methods[] = {
    {"name", (PyCFunction)Noddy_name, METH_NOARGS,
     "Return the name, combining the first and last name"
    },
    {NULL}  /* Sentinel */
};
\end{verbatim}

and assign them to the \member{tp_methods} slot:

\begin{verbatim}
    Noddy_methods,             /* tp_methods */
\end{verbatim}

Note that we used the \constant{METH_NOARGS} flag to indicate that the
method is passed no arguments.

Finally, we'll make our type usable as a base class.  We've written
our methods carefully so far so that they don't make any assumptions
about the type of the object being created or used, so all we need to
do is to add the \constant{Py_TPFLAGS_BASETYPE} to our class flag
definition:

\begin{verbatim}
    Py_TPFLAGS_DEFAULT | Py_TPFLAGS_BASETYPE, /*tp_flags*/
\end{verbatim}

We rename \cfunction{initnoddy()} to \cfunction{initnoddy2()}
and update the module name passed to \cfunction{Py_InitModule3()}.

Finally, we update our \file{setup.py} file to build the new module:

\begin{verbatim}
from distutils.core import setup, Extension
setup(name="noddy", version="1.0",
      ext_modules=[
         Extension("noddy", ["noddy.c"]),
         Extension("noddy2", ["noddy2.c"]),
         ])
\end{verbatim}

\subsection{Providing finer control over data attributes}

In this section, we'll provide finer control over how the
\member{first} and \member{last} attributes are set in the
\class{Noddy} example. In the previous version of our module, the
instance variables \member{first} and \member{last} could be set to
non-string values or even deleted. We want to make sure that these
attributes always contain strings.

\verbatiminput{noddy3.c}

To provide greater control, over the \member{first} and \member{last}
attributes, we'll use custom getter and setter functions.  Here are
the functions for getting and setting the \member{first} attribute:

\begin{verbatim}
Noddy_getfirst(Noddy *self, void *closure)
{
    Py_INCREF(self->first);
    return self->first;
}

static int
Noddy_setfirst(Noddy *self, PyObject *value, void *closure)
{
  if (value == NULL) {
    PyErr_SetString(PyExc_TypeError, "Cannot delete the first attribute");
    return -1;
  }

  if (! PyString_Check(value)) {
    PyErr_SetString(PyExc_TypeError,
                    "The first attribute value must be a string");
    return -1;
  }

  Py_DECREF(self->first);
  Py_INCREF(value);
  self->first = value;

  return 0;
}
\end{verbatim}

The getter function is passed a \class{Noddy} object and a
``closure'', which is void pointer. In this case, the closure is
ignored. (The closure supports an advanced usage in which definition
data is passed to the getter and setter. This could, for example, be
used to allow a single set of getter and setter functions that decide
the attribute to get or set based on data in the closure.)

The setter function is passed the \class{Noddy} object, the new value,
and the closure. The new value may be \NULL, in which case the
attribute is being deleted.  In our setter, we raise an error if the
attribute is deleted or if the attribute value is not a string.

We create an array of \ctype{PyGetSetDef} structures:

\begin{verbatim}
static PyGetSetDef Noddy_getseters[] = {
    {"first",
     (getter)Noddy_getfirst, (setter)Noddy_setfirst,
     "first name",
     NULL},
    {"last",
     (getter)Noddy_getlast, (setter)Noddy_setlast,
     "last name",
     NULL},
    {NULL}  /* Sentinel */
};
\end{verbatim}

and register it in the \member{tp_getset} slot:

\begin{verbatim}
    Noddy_getseters,           /* tp_getset */
\end{verbatim}

to register out attribute getters and setters.

The last item in a \ctype{PyGetSetDef} structure is the closure
mentioned above. In this case, we aren't using the closure, so we just
pass \NULL.

We also remove the member definitions for these attributes:

\begin{verbatim}
static PyMemberDef Noddy_members[] = {
    {"number", T_INT, offsetof(Noddy, number), 0,
     "noddy number"},
    {NULL}  /* Sentinel */
};
\end{verbatim}

We also need to update the \member{tp_init} handler to only allow
strings\footnote{We now know that the first and last members are strings,
so perhaps we could be less careful about decrementing their
reference counts, however, we accept instances of string subclasses.
Even though deallocating normal strings won't call back into our
objects, we can't guarantee that deallocating an instance of a string
subclass won't. call back into out objects.} to be passed:

\begin{verbatim}
static int
Noddy_init(Noddy *self, PyObject *args, PyObject *kwds)
{
    PyObject *first=NULL, *last=NULL, *tmp;

    static char *kwlist[] = {"first", "last", "number", NULL};

    if (! PyArg_ParseTupleAndKeywords(args, kwds, "|SSi", kwlist,
                                      &first, &last,
                                      &self->number))
        return -1;

    if (first) {
        tmp = self->first;
        Py_INCREF(first);
        self->first = first;
        Py_DECREF(tmp);
    }

    if (last) {
        tmp = self->last;
        Py_INCREF(last);
        self->last = last;
        Py_DECREF(tmp);
    }

    return 0;
}
\end{verbatim}

With these changes, we can assure that the \member{first} and
\member{last} members are never NULL so we can remove checks for \NULL
values in almost all cases. This means that most of the
\cfunction{Py_XDECREF()} calls can be converted to \cfunction{Py_DECREF()}
calls. The only place we can't change these calls is in the
deallocator, where there is the possibility that the initialization of
these members failed in the constructor.

We also rename the module initialization function and module name in
the initialization function, as we did before, and we add an extra
definition to the \file{setup.py} file.

\subsection{Supporting cyclic garbage collection}

Python has a cyclic-garbage collector that can identify unneeded
objects even when their reference counts are not zero. This can happen
when objects are involved in cycles.  For example, consider:

\begin{verbatim}
>>> l = []
>>> l.append(l)
>>> del l
\end{verbatim}

In this example, we create a list that contains itself. When we delete
it, it still has a reference from itself. Its reference count doesn't
drop to zero.  Fortunately, Python's cyclic-garbage collector will
eventually figure out that the list is garbage and free it.

In the second version of the \class{Noddy} example, we allowed any
kind of object to be stored in the \member{first} or \member{last}
attributes\footnote{Even in the third version, we aren't guaranteed to
avoid cycles.  Instances of string subclasses are allowed and string
subclasses could allow cycles even if normal strings don't.}. This
means that \class{Noddy} objects can participate in cycles:

\begin{verbatim}
>>> import noddy2
>>> n = noddy2.Noddy()
>>> l = [n]
>>> n.first = l
\end{verbatim}

This is pretty silly, but it gives us an excuse to add support for the
cyclic-garbage collector to the \class{Noddy} example.  To support
cyclic garbage collection, types need to fill two slots and set a
class flag that enables these slots:

\verbatiminput{noddy4.c}

The traversal method provides access to subobjects that
could participate in cycles:

\begin{verbatim}
static int
Noddy_traverse(Noddy *self, visitproc visit, void *arg)
{
    int vret;

    if (self->first) {
        vret = visit(self->first, arg);
        if (vret != 0)
            return vret;
    }
    if (self->last) {
        vret = visit(self->last, arg);
        if (vret != 0)
            return vret;
    }

    return 0;
}
\end{verbatim}

For each subobject that can participate in cycles, we need to call the
\cfunction{visit()} function, which is passed to the traversal method.
The \cfunction{visit()} function takes as arguments the subobject and
the extra argument \var{arg} passed to the traversal method.  It
returns an integer value that must be returned if it is non-zero.


Python 2.4 and higher provide a \cfunction{Py_VISIT()} macro that automates
calling visit functions.  With \cfunction{Py_VISIT()},
\cfunction{Noddy_traverse()} can be simplified:


\begin{verbatim}
static int
Noddy_traverse(Noddy *self, visitproc visit, void *arg)
{
    Py_VISIT(self->first);
    Py_VISIT(self->last);
    return 0;
}
\end{verbatim}

\note{Note that the \member{tp_traverse} implementation must name its
    arguments exactly \var{visit} and \var{arg} in order to use
    \cfunction{Py_VISIT()}.  This is to encourage uniformity
    across these boring implementations.}

We also need to provide a method for clearing any subobjects that can
participate in cycles.  We implement the method and reimplement the
deallocator to use it:

\begin{verbatim}
static int
Noddy_clear(Noddy *self)
{
    PyObject *tmp;

    tmp = self->first;
    self->first = NULL;
    Py_XDECREF(tmp);

    tmp = self->last;
    self->last = NULL;
    Py_XDECREF(tmp);

    return 0;
}

static void
Noddy_dealloc(Noddy* self)
{
    Noddy_clear(self);
    self->ob_type->tp_free((PyObject*)self);
}
\end{verbatim}

Notice the use of a temporary variable in \cfunction{Noddy_clear()}.
We use the temporary variable so that we can set each member to \NULL
before decrementing it's reference count.  We do this because, as was
discussed earlier, if the reference count drops to zero, we might
cause code to run that calls back into the object.  In addition,
because we now support garbage collection, we also have to worry about
code being run that triggers garbage collection.  If garbage
collection is run, our \member{tp_traverse} handler could get called.
We can't take a chance of having \cfunction{Noddy_traverse()} called
when a member's reference count has dropped to zero and it's value
hasn't been set to \NULL.

Python 2.4 and higher provide a \cfunction{Py_CLEAR()} that automates
the careful decrementing of reference counts.  With
\cfunction{Py_CLEAR()}, the \cfunction{Noddy_clear()} function can be
simplified:

\begin{verbatim}
static int
Noddy_clear(Noddy *self)
{
    Py_CLEAR(self->first);
    Py_CLEAR(self->last);
    return 0;
}
\end{verbatim}

Finally, we add the \constant{Py_TPFLAGS_HAVE_GC} flag to the class
flags:

\begin{verbatim}
    Py_TPFLAGS_DEFAULT | Py_TPFLAGS_BASETYPE | Py_TPFLAGS_HAVE_GC, /*tp_flags*/
\end{verbatim}

That's pretty much it.  If we had written custom \member{tp_alloc} or
\member{tp_free} slots, we'd need to modify them for cyclic-garbage
collection. Most extensions will use the versions automatically
provided.

\section{Type Methods
         \label{dnt-type-methods}}

This section aims to give a quick fly-by on the various type methods
you can implement and what they do.

Here is the definition of \ctype{PyTypeObject}, with some fields only
used in debug builds omitted:

\verbatiminput{typestruct.h}

Now that's a \emph{lot} of methods.  Don't worry too much though - if
you have a type you want to define, the chances are very good that you
will only implement a handful of these.

As you probably expect by now, we're going to go over this and give
more information about the various handlers.  We won't go in the order
they are defined in the structure, because there is a lot of
historical baggage that impacts the ordering of the fields; be sure
your type initialization keeps the fields in the right order!  It's
often easiest to find an example that includes all the fields you need
(even if they're initialized to \code{0}) and then change the values
to suit your new type.

\begin{verbatim}
    char *tp_name; /* For printing */
\end{verbatim}

The name of the type - as mentioned in the last section, this will
appear in various places, almost entirely for diagnostic purposes.
Try to choose something that will be helpful in such a situation!

\begin{verbatim}
    int tp_basicsize, tp_itemsize; /* For allocation */
\end{verbatim}

These fields tell the runtime how much memory to allocate when new
objects of this type are created.  Python has some built-in support
for variable length structures (think: strings, lists) which is where
the \member{tp_itemsize} field comes in.  This will be dealt with
later.

\begin{verbatim}
    char *tp_doc;
\end{verbatim}

Here you can put a string (or its address) that you want returned when
the Python script references \code{obj.__doc__} to retrieve the
doc string.

Now we come to the basic type methods---the ones most extension types
will implement.


\subsection{Finalization and De-allocation}

\index{object!deallocation}
\index{deallocation, object}
\index{object!finalization}
\index{finalization, of objects}

\begin{verbatim}
    destructor tp_dealloc;
\end{verbatim}

This function is called when the reference count of the instance of
your type is reduced to zero and the Python interpreter wants to
reclaim it.  If your type has memory to free or other clean-up to
perform, put it here.  The object itself needs to be freed here as
well.  Here is an example of this function:

\begin{verbatim}
static void
newdatatype_dealloc(newdatatypeobject * obj)
{
    free(obj->obj_UnderlyingDatatypePtr);
    obj->ob_type->tp_free(obj);
}
\end{verbatim}

One important requirement of the deallocator function is that it
leaves any pending exceptions alone.  This is important since
deallocators are frequently called as the interpreter unwinds the
Python stack; when the stack is unwound due to an exception (rather
than normal returns), nothing is done to protect the deallocators from
seeing that an exception has already been set.  Any actions which a
deallocator performs which may cause additional Python code to be
executed may detect that an exception has been set.  This can lead to
misleading errors from the interpreter.  The proper way to protect
against this is to save a pending exception before performing the
unsafe action, and restoring it when done.  This can be done using the
\cfunction{PyErr_Fetch()}\ttindex{PyErr_Fetch()} and
\cfunction{PyErr_Restore()}\ttindex{PyErr_Restore()} functions:

\begin{verbatim}
static void
my_dealloc(PyObject *obj)
{
    MyObject *self = (MyObject *) obj;
    PyObject *cbresult;

    if (self->my_callback != NULL) {
        PyObject *err_type, *err_value, *err_traceback;
        int have_error = PyErr_Occurred() ? 1 : 0;

        if (have_error)
            PyErr_Fetch(&err_type, &err_value, &err_traceback);

        cbresult = PyObject_CallObject(self->my_callback, NULL);
        if (cbresult == NULL)
            PyErr_WriteUnraisable();
        else
            Py_DECREF(cbresult);

        if (have_error)
            PyErr_Restore(err_type, err_value, err_traceback);

        Py_DECREF(self->my_callback);
    }
    obj->ob_type->tp_free((PyObject*)self);
}
\end{verbatim}


\subsection{Object Presentation}

In Python, there are three ways to generate a textual representation
of an object: the \function{repr()}\bifuncindex{repr} function (or
equivalent back-tick syntax), the \function{str()}\bifuncindex{str}
function, and the \keyword{print} statement.  For most objects, the
\keyword{print} statement is equivalent to the \function{str()}
function, but it is possible to special-case printing to a
\ctype{FILE*} if necessary; this should only be done if efficiency is
identified as a problem and profiling suggests that creating a
temporary string object to be written to a file is too expensive.

These handlers are all optional, and most types at most need to
implement the \member{tp_str} and \member{tp_repr} handlers.

\begin{verbatim}
    reprfunc tp_repr;
    reprfunc tp_str;
    printfunc tp_print;
\end{verbatim}

The \member{tp_repr} handler should return a string object containing
a representation of the instance for which it is called.  Here is a
simple example:

\begin{verbatim}
static PyObject *
newdatatype_repr(newdatatypeobject * obj)
{
    return PyString_FromFormat("Repr-ified_newdatatype{{size:\%d}}",
                               obj->obj_UnderlyingDatatypePtr->size);
}
\end{verbatim}

If no \member{tp_repr} handler is specified, the interpreter will
supply a representation that uses the type's \member{tp_name} and a
uniquely-identifying value for the object.

The \member{tp_str} handler is to \function{str()} what the
\member{tp_repr} handler described above is to \function{repr()}; that
is, it is called when Python code calls \function{str()} on an
instance of your object.  Its implementation is very similar to the
\member{tp_repr} function, but the resulting string is intended for
human consumption.  If \member{tp_str} is not specified, the
\member{tp_repr} handler is used instead.

Here is a simple example:

\begin{verbatim}
static PyObject *
newdatatype_str(newdatatypeobject * obj)
{
    return PyString_FromFormat("Stringified_newdatatype{{size:\%d}}",
                               obj->obj_UnderlyingDatatypePtr->size);
}
\end{verbatim}

The print function will be called whenever Python needs to "print" an
instance of the type.  For example, if 'node' is an instance of type
TreeNode, then the print function is called when Python code calls:

\begin{verbatim}
print node
\end{verbatim}

There is a flags argument and one flag, \constant{Py_PRINT_RAW}, and
it suggests that you print without string quotes and possibly without
interpreting escape sequences.

The print function receives a file object as an argument. You will
likely want to write to that file object.

Here is a sample print function:

\begin{verbatim}
static int
newdatatype_print(newdatatypeobject *obj, FILE *fp, int flags)
{
    if (flags & Py_PRINT_RAW) {
        fprintf(fp, "<{newdatatype object--size: %d}>",
                obj->obj_UnderlyingDatatypePtr->size);
    }
    else {
        fprintf(fp, "\"<{newdatatype object--size: %d}>\"",
                obj->obj_UnderlyingDatatypePtr->size);
    }
    return 0;
}
\end{verbatim}


\subsection{Attribute Management}

For every object which can support attributes, the corresponding type
must provide the functions that control how the attributes are
resolved.  There needs to be a function which can retrieve attributes
(if any are defined), and another to set attributes (if setting
attributes is allowed).  Removing an attribute is a special case, for
which the new value passed to the handler is \NULL.

Python supports two pairs of attribute handlers; a type that supports
attributes only needs to implement the functions for one pair.  The
difference is that one pair takes the name of the attribute as a
\ctype{char*}, while the other accepts a \ctype{PyObject*}.  Each type
can use whichever pair makes more sense for the implementation's
convenience.

\begin{verbatim}
    getattrfunc  tp_getattr;        /* char * version */
    setattrfunc  tp_setattr;
    /* ... */
    getattrofunc tp_getattrofunc;   /* PyObject * version */
    setattrofunc tp_setattrofunc;
\end{verbatim}

If accessing attributes of an object is always a simple operation
(this will be explained shortly), there are generic implementations
which can be used to provide the \ctype{PyObject*} version of the
attribute management functions.  The actual need for type-specific
attribute handlers almost completely disappeared starting with Python
2.2, though there are many examples which have not been updated to use
some of the new generic mechanism that is available.


\subsubsection{Generic Attribute Management}

\versionadded{2.2}

Most extension types only use \emph{simple} attributes.  So, what
makes the attributes simple?  There are only a couple of conditions
that must be met:

\begin{enumerate}
  \item   The name of the attributes must be known when
          \cfunction{PyType_Ready()} is called.

  \item   No special processing is needed to record that an attribute
          was looked up or set, nor do actions need to be taken based
          on the value.
\end{enumerate}

Note that this list does not place any restrictions on the values of
the attributes, when the values are computed, or how relevant data is
stored.

When \cfunction{PyType_Ready()} is called, it uses three tables
referenced by the type object to create \emph{descriptors} which are
placed in the dictionary of the type object.  Each descriptor controls
access to one attribute of the instance object.  Each of the tables is
optional; if all three are \NULL, instances of the type will only have
attributes that are inherited from their base type, and should leave
the \member{tp_getattro} and \member{tp_setattro} fields \NULL{} as
well, allowing the base type to handle attributes.

The tables are declared as three fields of the type object:

\begin{verbatim}
    struct PyMethodDef *tp_methods;
    struct PyMemberDef *tp_members;
    struct PyGetSetDef *tp_getset;
\end{verbatim}

If \member{tp_methods} is not \NULL, it must refer to an array of
\ctype{PyMethodDef} structures.  Each entry in the table is an
instance of this structure:

\begin{verbatim}
typedef struct PyMethodDef {
    char        *ml_name;       /* method name */
    PyCFunction  ml_meth;       /* implementation function */
    int	         ml_flags;      /* flags */
    char        *ml_doc;        /* docstring */
} PyMethodDef;
\end{verbatim}

One entry should be defined for each method provided by the type; no
entries are needed for methods inherited from a base type.  One
additional entry is needed at the end; it is a sentinel that marks the
end of the array.  The \member{ml_name} field of the sentinel must be
\NULL.

XXX Need to refer to some unified discussion of the structure fields,
shared with the next section.

The second table is used to define attributes which map directly to
data stored in the instance.  A variety of primitive C types are
supported, and access may be read-only or read-write.  The structures
in the table are defined as:

\begin{verbatim}
typedef struct PyMemberDef {
    char *name;
    int   type;
    int   offset;
    int   flags;
    char *doc;
} PyMemberDef;
\end{verbatim}

For each entry in the table, a descriptor will be constructed and
added to the type which will be able to extract a value from the
instance structure.  The \member{type} field should contain one of the
type codes defined in the \file{structmember.h} header; the value will
be used to determine how to convert Python values to and from C
values.  The \member{flags} field is used to store flags which control
how the attribute can be accessed.

XXX Need to move some of this to a shared section!

The following flag constants are defined in \file{structmember.h};
they may be combined using bitwise-OR.

\begin{tableii}{l|l}{constant}{Constant}{Meaning}
  \lineii{READONLY \ttindex{READONLY}}
         {Never writable.}
  \lineii{RO \ttindex{RO}}
         {Shorthand for \constant{READONLY}.}
  \lineii{READ_RESTRICTED \ttindex{READ_RESTRICTED}}
         {Not readable in restricted mode.}
  \lineii{WRITE_RESTRICTED \ttindex{WRITE_RESTRICTED}}
         {Not writable in restricted mode.}
  \lineii{RESTRICTED \ttindex{RESTRICTED}}
         {Not readable or writable in restricted mode.}
\end{tableii}

An interesting advantage of using the \member{tp_members} table to
build descriptors that are used at runtime is that any attribute
defined this way can have an associated doc string simply by providing
the text in the table.  An application can use the introspection API
to retrieve the descriptor from the class object, and get the
doc string using its \member{__doc__} attribute.

As with the \member{tp_methods} table, a sentinel entry with a
\member{name} value of \NULL{} is required.


% XXX Descriptors need to be explained in more detail somewhere, but
% not here.
%
% Descriptor objects have two handler functions which correspond to
% the \member{tp_getattro} and \member{tp_setattro} handlers.  The
% \method{__get__()} handler is a function which is passed the
% descriptor, instance, and type objects, and returns the value of the
% attribute, or it returns \NULL{} and sets an exception.  The
% \method{__set__()} handler is passed the descriptor, instance, type,
% and new value;


\subsubsection{Type-specific Attribute Management}

For simplicity, only the \ctype{char*} version will be demonstrated
here; the type of the name parameter is the only difference between
the \ctype{char*} and \ctype{PyObject*} flavors of the interface.
This example effectively does the same thing as the generic example
above, but does not use the generic support added in Python 2.2.  The
value in showing this is two-fold: it demonstrates how basic attribute
management can be done in a way that is portable to older versions of
Python, and explains how the handler functions are called, so that if
you do need to extend their functionality, you'll understand what
needs to be done.

The \member{tp_getattr} handler is called when the object requires an
attribute look-up.  It is called in the same situations where the
\method{__getattr__()} method of a class would be called.

A likely way to handle this is (1) to implement a set of functions
(such as \cfunction{newdatatype_getSize()} and
\cfunction{newdatatype_setSize()} in the example below), (2) provide a
method table listing these functions, and (3) provide a getattr
function that returns the result of a lookup in that table.  The
method table uses the same structure as the \member{tp_methods} field
of the type object.

Here is an example:

\begin{verbatim}
static PyMethodDef newdatatype_methods[] = {
    {"getSize", (PyCFunction)newdatatype_getSize, METH_VARARGS,
     "Return the current size."},
    {"setSize", (PyCFunction)newdatatype_setSize, METH_VARARGS,
     "Set the size."},
    {NULL, NULL, 0, NULL}           /* sentinel */
};

static PyObject *
newdatatype_getattr(newdatatypeobject *obj, char *name)
{
    return Py_FindMethod(newdatatype_methods, (PyObject *)obj, name);
}
\end{verbatim}

The \member{tp_setattr} handler is called when the
\method{__setattr__()} or \method{__delattr__()} method of a class
instance would be called.  When an attribute should be deleted, the
third parameter will be \NULL.  Here is an example that simply raises
an exception; if this were really all you wanted, the
\member{tp_setattr} handler should be set to \NULL.

\begin{verbatim}
static int
newdatatype_setattr(newdatatypeobject *obj, char *name, PyObject *v)
{
    (void)PyErr_Format(PyExc_RuntimeError, "Read-only attribute: \%s", name);
    return -1;
}
\end{verbatim}


\subsection{Object Comparison}

\begin{verbatim}
    cmpfunc tp_compare;
\end{verbatim}

The \member{tp_compare} handler is called when comparisons are needed
and the object does not implement the specific rich comparison method
which matches the requested comparison.  (It is always used if defined
and the \cfunction{PyObject_Compare()} or \cfunction{PyObject_Cmp()}
functions are used, or if \function{cmp()} is used from Python.)
It is analogous to the \method{__cmp__()} method.  This function
should return \code{-1} if \var{obj1} is less than
\var{obj2}, \code{0} if they are equal, and \code{1} if
\var{obj1} is greater than
\var{obj2}.
(It was previously allowed to return arbitrary negative or positive
integers for less than and greater than, respectively; as of Python
2.2, this is no longer allowed.  In the future, other return values
may be assigned a different meaning.)

A \member{tp_compare} handler may raise an exception.  In this case it
should return a negative value.  The caller has to test for the
exception using \cfunction{PyErr_Occurred()}.


Here is a sample implementation:

\begin{verbatim}
static int
newdatatype_compare(newdatatypeobject * obj1, newdatatypeobject * obj2)
{
    long result;

    if (obj1->obj_UnderlyingDatatypePtr->size <
        obj2->obj_UnderlyingDatatypePtr->size) {
        result = -1;
    }
    else if (obj1->obj_UnderlyingDatatypePtr->size >
             obj2->obj_UnderlyingDatatypePtr->size) {
        result = 1;
    }
    else {
        result = 0;
    }
    return result;
}
\end{verbatim}


\subsection{Abstract Protocol Support}

Python supports a variety of \emph{abstract} `protocols;' the specific
interfaces provided to use these interfaces are documented in the
\citetitle[../api/api.html]{Python/C API Reference Manual} in the
chapter ``\ulink{Abstract Objects Layer}{../api/abstract.html}.''

A number of these abstract interfaces were defined early in the
development of the Python implementation.  In particular, the number,
mapping, and sequence protocols have been part of Python since the
beginning.  Other protocols have been added over time.  For protocols
which depend on several handler routines from the type implementation,
the older protocols have been defined as optional blocks of handlers
referenced by the type object.  For newer protocols there are
additional slots in the main type object, with a flag bit being set to
indicate that the slots are present and should be checked by the
interpreter.  (The flag bit does not indicate that the slot values are
non-\NULL. The flag may be set to indicate the presence of a slot,
but a slot may still be unfilled.)

\begin{verbatim}
    PyNumberMethods   tp_as_number;
    PySequenceMethods tp_as_sequence;
    PyMappingMethods  tp_as_mapping;
\end{verbatim}

If you wish your object to be able to act like a number, a sequence,
or a mapping object, then you place the address of a structure that
implements the C type \ctype{PyNumberMethods},
\ctype{PySequenceMethods}, or \ctype{PyMappingMethods}, respectively.
It is up to you to fill in this structure with appropriate values. You
can find examples of the use of each of these in the \file{Objects}
directory of the Python source distribution.


\begin{verbatim}
    hashfunc tp_hash;
\end{verbatim}

This function, if you choose to provide it, should return a hash
number for an instance of your data type. Here is a moderately
pointless example:

\begin{verbatim}
static long
newdatatype_hash(newdatatypeobject *obj)
{
    long result;
    result = obj->obj_UnderlyingDatatypePtr->size;
    result = result * 3;
    return result;
}
\end{verbatim}

\begin{verbatim}
    ternaryfunc tp_call;
\end{verbatim}

This function is called when an instance of your data type is "called",
for example, if \code{obj1} is an instance of your data type and the Python
script contains \code{obj1('hello')}, the \member{tp_call} handler is
invoked.

This function takes three arguments:

\begin{enumerate}
  \item
    \var{arg1} is the instance of the data type which is the subject of
    the call. If the call is \code{obj1('hello')}, then \var{arg1} is
    \code{obj1}.

  \item
    \var{arg2} is a tuple containing the arguments to the call.  You
    can use \cfunction{PyArg_ParseTuple()} to extract the arguments.

  \item
    \var{arg3} is a dictionary of keyword arguments that were passed.
    If this is non-\NULL{} and you support keyword arguments, use
    \cfunction{PyArg_ParseTupleAndKeywords()} to extract the
    arguments.  If you do not want to support keyword arguments and
    this is non-\NULL, raise a \exception{TypeError} with a message
    saying that keyword arguments are not supported.
\end{enumerate}

Here is a desultory example of the implementation of the call function.

\begin{verbatim}
/* Implement the call function.
 *    obj1 is the instance receiving the call.
 *    obj2 is a tuple containing the arguments to the call, in this
 *         case 3 strings.
 */
static PyObject *
newdatatype_call(newdatatypeobject *obj, PyObject *args, PyObject *other)
{
    PyObject *result;
    char *arg1;
    char *arg2;
    char *arg3;

    if (!PyArg_ParseTuple(args, "sss:call", &arg1, &arg2, &arg3)) {
        return NULL;
    }
    result = PyString_FromFormat(
        "Returning -- value: [\%d] arg1: [\%s] arg2: [\%s] arg3: [\%s]\n",
        obj->obj_UnderlyingDatatypePtr->size,
        arg1, arg2, arg3);
    printf("\%s", PyString_AS_STRING(result));
    return result;
}
\end{verbatim}

XXX some fields need to be added here...


\begin{verbatim}
    /* Added in release 2.2 */
    /* Iterators */
    getiterfunc tp_iter;
    iternextfunc tp_iternext;
\end{verbatim}

These functions provide support for the iterator protocol.  Any object
which wishes to support iteration over its contents (which may be
generated during iteration) must implement the \code{tp_iter}
handler.  Objects which are returned by a \code{tp_iter} handler must
implement both the \code{tp_iter} and \code{tp_iternext} handlers.
Both handlers take exactly one parameter, the instance for which they
are being called, and return a new reference.  In the case of an
error, they should set an exception and return \NULL.

For an object which represents an iterable collection, the
\code{tp_iter} handler must return an iterator object.  The iterator
object is responsible for maintaining the state of the iteration.  For
collections which can support multiple iterators which do not
interfere with each other (as lists and tuples do), a new iterator
should be created and returned.  Objects which can only be iterated
over once (usually due to side effects of iteration) should implement
this handler by returning a new reference to themselves, and should
also implement the \code{tp_iternext} handler.  File objects are an
example of such an iterator.

Iterator objects should implement both handlers.  The \code{tp_iter}
handler should return a new reference to the iterator (this is the
same as the \code{tp_iter} handler for objects which can only be
iterated over destructively).  The \code{tp_iternext} handler should
return a new reference to the next object in the iteration if there is
one.  If the iteration has reached the end, it may return \NULL{}
without setting an exception or it may set \exception{StopIteration};
avoiding the exception can yield slightly better performance.  If an
actual error occurs, it should set an exception and return \NULL.

\subsection{More Suggestions}

Remember that you can omit most of these functions, in which case you
provide \code{0} as a value.  There are type definitions for each of
the functions you must provide.  They are in \file{object.h} in the
Python include directory that comes with the source distribution of
Python.

In order to learn how to implement any specific method for your new
data type, do the following: Download and unpack the Python source
distribution.  Go the \file{Objects} directory, then search the
C source files for \code{tp_} plus the function you want (for
example, \code{tp_print} or \code{tp_compare}).  You will find
examples of the function you want to implement.

When you need to verify that an object is an instance of the type
you are implementing, use the \cfunction{PyObject_TypeCheck} function.
A sample of its use might be something like the following:

\begin{verbatim}
    if (! PyObject_TypeCheck(some_object, &MyType)) {
        PyErr_SetString(PyExc_TypeError, "arg #1 not a mything");
        return NULL;
    }
\end{verbatim}

\chapter{Building C and \Cpp{} Extensions with distutils
     \label{building}}

\sectionauthor{Martin v. L\"owis}{martin@v.loewis.de}

Starting in Python 1.4, Python provides, on \UNIX{}, a special make
file for building make files for building dynamically-linked
extensions and custom interpreters.  Starting with Python 2.0, this
mechanism (known as related to Makefile.pre.in, and Setup files) is no
longer supported. Building custom interpreters was rarely used, and
extensions modules can be build using distutils.

Building an extension module using distutils requires that distutils
is installed on the build machine, which is included in Python 2.x and
available separately for Python 1.5. Since distutils also supports
creation of binary packages, users don't necessarily need a compiler
and distutils to install the extension.

A distutils package contains a driver script, \file{setup.py}. This is
a plain Python file, which, in the most simple case, could look like
this:

\begin{verbatim}
from distutils.core import setup, Extension

module1 = Extension('demo',
                    sources = ['demo.c'])

setup (name = 'PackageName',
       version = '1.0',
       description = 'This is a demo package',
       ext_modules = [module1])

\end{verbatim}

With this \file{setup.py}, and a file \file{demo.c}, running

\begin{verbatim}
python setup.py build 
\end{verbatim}

will compile \file{demo.c}, and produce an extension module named
\samp{demo} in the \file{build} directory. Depending on the system,
the module file will end up in a subdirectory \file{build/lib.system},
and may have a name like \file{demo.so} or \file{demo.pyd}.

In the \file{setup.py}, all execution is performed by calling the
\samp{setup} function. This takes a variable number of keyword 
arguments, of which the example above uses only a
subset. Specifically, the example specifies meta-information to build
packages, and it specifies the contents of the package.  Normally, a
package will contain of addition modules, like Python source modules,
documentation, subpackages, etc. Please refer to the distutils
documentation in \citetitle[../dist/dist.html]{Distributing Python
Modules} to learn more about the features of distutils; this section
explains building extension modules only.

It is common to pre-compute arguments to \function{setup}, to better
structure the driver script. In the example above,
the\samp{ext_modules} argument to \function{setup} is a list of
extension modules, each of which is an instance of the
\class{Extension}. In the example, the instance defines an extension
named \samp{demo} which is build by compiling a single source file,
\file{demo.c}.

In many cases, building an extension is more complex, since additional
preprocessor defines and libraries may be needed. This is demonstrated
in the example below.

\begin{verbatim}
from distutils.core import setup, Extension

module1 = Extension('demo',
                    define_macros = [('MAJOR_VERSION', '1'),
                                     ('MINOR_VERSION', '0')],
                    include_dirs = ['/usr/local/include'],
                    libraries = ['tcl83'],
                    library_dirs = ['/usr/local/lib'],
                    sources = ['demo.c'])

setup (name = 'PackageName',
       version = '1.0',
       description = 'This is a demo package',
       author = 'Martin v. Loewis',
       author_email = 'martin@v.loewis.de',
       url = 'http://www.python.org/doc/current/ext/building.html',
       long_description = '''
This is really just a demo package.
''',
       ext_modules = [module1])

\end{verbatim}

In this example, \function{setup} is called with additional
meta-information, which is recommended when distribution packages have
to be built. For the extension itself, it specifies preprocessor
defines, include directories, library directories, and libraries.
Depending on the compiler, distutils passes this information in
different ways to the compiler. For example, on \UNIX{}, this may
result in the compilation commands

\begin{verbatim}
gcc -DNDEBUG -g -O3 -Wall -Wstrict-prototypes -fPIC -DMAJOR_VERSION=1 -DMINOR_VERSION=0 -I/usr/local/include -I/usr/local/include/python2.2 -c demo.c -o build/temp.linux-i686-2.2/demo.o

gcc -shared build/temp.linux-i686-2.2/demo.o -L/usr/local/lib -ltcl83 -o build/lib.linux-i686-2.2/demo.so
\end{verbatim}

These lines are for demonstration purposes only; distutils users
should trust that distutils gets the invocations right.

\section{Distributing your extension modules
     \label{distributing}}

When an extension has been successfully build, there are three ways to
use it.

End-users will typically want to install the module, they do so by
running

\begin{verbatim}
python setup.py install
\end{verbatim}

Module maintainers should produce source packages; to do so, they run

\begin{verbatim}
python setup.py sdist
\end{verbatim}

In some cases, additional files need to be included in a source
distribution; this is done through a \file{MANIFEST.in} file; see the
distutils documentation for details.

If the source distribution has been build successfully, maintainers
can also create binary distributions. Depending on the platform, one
of the following commands can be used to do so.

\begin{verbatim}
python setup.py bdist_wininst
python setup.py bdist_rpm
python setup.py bdist_dumb
\end{verbatim}


\chapter{Building C and \Cpp{} Extensions on Windows%
     \label{building-on-windows}}


This chapter briefly explains how to create a Windows extension module
for Python using Microsoft Visual \Cpp, and follows with more
detailed background information on how it works.  The explanatory
material is useful for both the Windows programmer learning to build
Python extensions and the \UNIX{} programmer interested in producing
software which can be successfully built on both \UNIX{} and Windows.

Module authors are encouraged to use the distutils approach for
building extension modules, instead of the one described in this
section. You will still need the C compiler that was used to build
Python; typically Microsoft Visual \Cpp.

\begin{notice}
  This chapter mentions a number of filenames that include an encoded
  Python version number.  These filenames are represented with the
  version number shown as \samp{XY}; in practive, \character{X} will
  be the major version number and \character{Y} will be the minor
  version number of the Python release you're working with.  For
  example, if you are using Python 2.2.1, \samp{XY} will actually be
  \samp{22}.
\end{notice}


\section{A Cookbook Approach \label{win-cookbook}}

There are two approaches to building extension modules on Windows,
just as there are on \UNIX: use the
\ulink{\module{distutils}}{../lib/module-distutils.html} package to
control the build process, or do things manually.  The distutils
approach works well for most extensions; documentation on using
\ulink{\module{distutils}}{../lib/module-distutils.html} to build and
package extension modules is available in
\citetitle[../dist/dist.html]{Distributing Python Modules}.  This
section describes the manual approach to building Python extensions
written in C or \Cpp.

To build extensions using these instructions, you need to have a copy
of the Python sources of the same version as your installed Python.
You will need Microsoft Visual \Cpp{} ``Developer Studio''; project
files are supplied for V\Cpp{} version 7.1, but you can use older
versions of V\Cpp.  Notice that you should use the same version of
V\Cpp that was used to build Python itself. The example files
described here are distributed with the Python sources in the
\file{PC\textbackslash example_nt\textbackslash} directory.

\begin{enumerate}
  \item
  \strong{Copy the example files}\\
    The \file{example_nt} directory is a subdirectory of the \file{PC}
    directory, in order to keep all the PC-specific files under the
    same directory in the source distribution.  However, the
    \file{example_nt} directory can't actually be used from this
    location.  You first need to copy or move it up one level, so that
    \file{example_nt} is a sibling of the \file{PC} and \file{Include}
    directories.  Do all your work from within this new location.

  \item
  \strong{Open the project}\\
    From V\Cpp, use the \menuselection{File \sub Open Solution}
    dialog (not \menuselection{File \sub Open}!).  Navigate to and
    select the file \file{example.sln}, in the \emph{copy} of the
    \file{example_nt} directory you made above.  Click Open.

  \item
  \strong{Build the example DLL}\\
    In order to check that everything is set up right, try building:

    \begin{enumerate}
      \item
        Select a configuration.  This step is optional.  Choose
        \menuselection{Build \sub Configuration Manager \sub Active 
        Solution Configuration} and select either \guilabel{Release} 
        or\guilabel{Debug}.  If you skip this step,
        V\Cpp{} will use the Debug configuration by default.

      \item
        Build the DLL.  Choose \menuselection{Build \sub Build
        Solution}.  This creates all intermediate and result files in
        a subdirectory called either \file{Debug} or \file{Release},
        depending on which configuration you selected in the preceding
        step.
    \end{enumerate}

  \item
  \strong{Testing the debug-mode DLL}\\
    Once the Debug build has succeeded, bring up a DOS box, and change
    to the \file{example_nt\textbackslash Debug} directory.  You
    should now be able to repeat the following session (\code{C>} is
    the DOS prompt, \code{>\code{>}>} is the Python prompt; note that
    build information and various debug output from Python may not
    match this screen dump exactly):

\begin{verbatim}
C>..\..\PCbuild\python_d
Adding parser accelerators ...
Done.
Python 2.2 (#28, Dec 19 2001, 23:26:37) [MSC 32 bit (Intel)] on win32
Type "copyright", "credits" or "license" for more information.
>>> import example
[4897 refs]
>>> example.foo()
Hello, world
[4903 refs]
>>>
\end{verbatim}

    Congratulations!  You've successfully built your first Python
    extension module.

  \item
  \strong{Creating your own project}\\
    Choose a name and create a directory for it.  Copy your C sources
    into it.  Note that the module source file name does not
    necessarily have to match the module name, but the name of the
    initialization function should match the module name --- you can
    only import a module \module{spam} if its initialization function
    is called \cfunction{initspam()}, and it should call
    \cfunction{Py_InitModule()} with the string \code{"spam"} as its
    first argument (use the minimal \file{example.c} in this directory
    as a guide).  By convention, it lives in a file called
    \file{spam.c} or \file{spammodule.c}.  The output file should be
    called \file{spam.dll} or \file{spam.pyd} (the latter is supported
    to avoid confusion with a system library \file{spam.dll} to which
    your module could be a Python interface) in Release mode, or
    \file{spam_d.dll} or \file{spam_d.pyd} in Debug mode.

    Now your options are:

    \begin{enumerate}
      \item  Copy \file{example.sln} and \file{example.vcproj}, rename
             them to \file{spam.*}, and edit them by hand, or
      \item  Create a brand new project; instructions are below.
    \end{enumerate}

    In either case, copy \file{example_nt\textbackslash example.def}
    to \file{spam\textbackslash spam.def}, and edit the new
    \file{spam.def} so its second line contains the string
    `\code{initspam}'.  If you created a new project yourself, add the
    file \file{spam.def} to the project now.  (This is an annoying
    little file with only two lines.  An alternative approach is to
    forget about the \file{.def} file, and add the option
    \programopt{/export:initspam} somewhere to the Link settings, by
    manually editing the setting in Project Properties dialog).

  \item
  \strong{Creating a brand new project}\\
    Use the \menuselection{File \sub New \sub Project} dialog to
    create a new Project Workspace.  Select \guilabel{Visual C++
    Projects/Win32/ Win32 Project}, enter the name (\samp{spam}), and
    make sure the Location is set to parent of the \file{spam}
    directory you have created (which should be a direct subdirectory
    of the Python build tree, a sibling of \file{Include} and
    \file{PC}).  Select Win32 as the platform (in my version, this is
    the only choice).  Make sure the Create new workspace radio button
    is selected.  Click OK.

    You should now create the file \file{spam.def} as instructed in
    the previous section. Add the source files to the project, using
    \menuselection{Project \sub Add Existing Item}. Set the pattern to
    \code{*.*} and select both \file{spam.c} and \file{spam.def} and
    click OK.  (Inserting them one by one is fine too.)

    Now open the \menuselection{Project \sub spam properties} dialog.
    You only need to change a few settings.  Make sure \guilabel{All
    Configurations} is selected from the \guilabel{Settings for:}
    dropdown list.  Select the C/\Cpp{} tab.  Choose the General
    category in the popup menu at the top.  Type the following text in
    the entry box labeled \guilabel{Additional Include Directories}:

\begin{verbatim}
..\Include,..\PC
\end{verbatim}

    Then, choose the General category in the Linker tab, and enter

\begin{verbatim}
..\PCbuild
\end{verbatim}

    in the text box labelled \guilabel{Additional library Directories}.

    Now you need to add some mode-specific settings:

    Select \guilabel{Release} in the \guilabel{Configuration}
    dropdown list.  Choose the \guilabel{Link} tab, choose the
    \guilabel{Input} category, and append \code{pythonXY.lib} to the
    list in the \guilabel{Additional Dependencies} box.

    Select \guilabel{Debug} in the \guilabel{Configuration} dropdown
    list, and append \code{pythonXY_d.lib} to the list in the
    \guilabel{Additional Dependencies} box.  Then click the C/\Cpp{}
    tab, select \guilabel{Code Generation}, and select
    \guilabel{Multi-threaded Debug DLL} from the \guilabel{Runtime
    library} dropdown list.

    Select \guilabel{Release} again from the \guilabel{Configuration}
    dropdown list.  Select \guilabel{Multi-threaded DLL} from the
    \guilabel{Runtime library} dropdown list.
\end{enumerate}


If your module creates a new type, you may have trouble with this line:

\begin{verbatim}
    PyObject_HEAD_INIT(&PyType_Type)
\end{verbatim}

Change it to:

\begin{verbatim}
    PyObject_HEAD_INIT(NULL)
\end{verbatim}

and add the following to the module initialization function:

\begin{verbatim}
    MyObject_Type.ob_type = &PyType_Type;
\end{verbatim}

Refer to section~3 of the
\citetitle[http://www.python.org/doc/FAQ.html]{Python FAQ} for details
on why you must do this.


\section{Differences Between \UNIX{} and Windows
     \label{dynamic-linking}}
\sectionauthor{Chris Phoenix}{cphoenix@best.com}


\UNIX{} and Windows use completely different paradigms for run-time
loading of code.  Before you try to build a module that can be
dynamically loaded, be aware of how your system works.

In \UNIX, a shared object (\file{.so}) file contains code to be used by the
program, and also the names of functions and data that it expects to
find in the program.  When the file is joined to the program, all
references to those functions and data in the file's code are changed
to point to the actual locations in the program where the functions
and data are placed in memory.  This is basically a link operation.

In Windows, a dynamic-link library (\file{.dll}) file has no dangling
references.  Instead, an access to functions or data goes through a
lookup table.  So the DLL code does not have to be fixed up at runtime
to refer to the program's memory; instead, the code already uses the
DLL's lookup table, and the lookup table is modified at runtime to
point to the functions and data.

In \UNIX, there is only one type of library file (\file{.a}) which
contains code from several object files (\file{.o}).  During the link
step to create a shared object file (\file{.so}), the linker may find
that it doesn't know where an identifier is defined.  The linker will
look for it in the object files in the libraries; if it finds it, it
will include all the code from that object file.

In Windows, there are two types of library, a static library and an
import library (both called \file{.lib}).  A static library is like a
\UNIX{} \file{.a} file; it contains code to be included as necessary.
An import library is basically used only to reassure the linker that a
certain identifier is legal, and will be present in the program when
the DLL is loaded.  So the linker uses the information from the
import library to build the lookup table for using identifiers that
are not included in the DLL.  When an application or a DLL is linked,
an import library may be generated, which will need to be used for all
future DLLs that depend on the symbols in the application or DLL.

Suppose you are building two dynamic-load modules, B and C, which should
share another block of code A.  On \UNIX, you would \emph{not} pass
\file{A.a} to the linker for \file{B.so} and \file{C.so}; that would
cause it to be included twice, so that B and C would each have their
own copy.  In Windows, building \file{A.dll} will also build
\file{A.lib}.  You \emph{do} pass \file{A.lib} to the linker for B and
C.  \file{A.lib} does not contain code; it just contains information
which will be used at runtime to access A's code.  

In Windows, using an import library is sort of like using \samp{import
spam}; it gives you access to spam's names, but does not create a
separate copy.  On \UNIX, linking with a library is more like
\samp{from spam import *}; it does create a separate copy.


\section{Using DLLs in Practice \label{win-dlls}}
\sectionauthor{Chris Phoenix}{cphoenix@best.com}

Windows Python is built in Microsoft Visual \Cpp; using other
compilers may or may not work (though Borland seems to).  The rest of
this section is MSV\Cpp{} specific.

When creating DLLs in Windows, you must pass \file{pythonXY.lib} to
the linker.  To build two DLLs, spam and ni (which uses C functions
found in spam), you could use these commands:

\begin{verbatim}
cl /LD /I/python/include spam.c ../libs/pythonXY.lib
cl /LD /I/python/include ni.c spam.lib ../libs/pythonXY.lib
\end{verbatim}

The first command created three files: \file{spam.obj},
\file{spam.dll} and \file{spam.lib}.  \file{Spam.dll} does not contain
any Python functions (such as \cfunction{PyArg_ParseTuple()}), but it
does know how to find the Python code thanks to \file{pythonXY.lib}.

The second command created \file{ni.dll} (and \file{.obj} and
\file{.lib}), which knows how to find the necessary functions from
spam, and also from the Python executable.

Not every identifier is exported to the lookup table.  If you want any
other modules (including Python) to be able to see your identifiers,
you have to say \samp{_declspec(dllexport)}, as in \samp{void
_declspec(dllexport) initspam(void)} or \samp{PyObject
_declspec(dllexport) *NiGetSpamData(void)}.

Developer Studio will throw in a lot of import libraries that you do
not really need, adding about 100K to your executable.  To get rid of
them, use the Project Settings dialog, Link tab, to specify
\emph{ignore default libraries}.  Add the correct
\file{msvcrt\var{xx}.lib} to the list of libraries.

\chapter{Embedding Python in Another Application
     \label{embedding}}

The previous chapters discussed how to extend Python, that is, how to
extend the functionality of Python by attaching a library of C
functions to it.  It is also possible to do it the other way around:
enrich your C/\Cpp{} application by embedding Python in it.  Embedding
provides your application with the ability to implement some of the
functionality of your application in Python rather than C or \Cpp.
This can be used for many purposes; one example would be to allow
users to tailor the application to their needs by writing some scripts
in Python.  You can also use it yourself if some of the functionality
can be written in Python more easily.

Embedding Python is similar to extending it, but not quite.  The
difference is that when you extend Python, the main program of the
application is still the Python interpreter, while if you embed
Python, the main program may have nothing to do with Python ---
instead, some parts of the application occasionally call the Python
interpreter to run some Python code.

So if you are embedding Python, you are providing your own main
program.  One of the things this main program has to do is initialize
the Python interpreter.  At the very least, you have to call the
function \cfunction{Py_Initialize()} (on Mac OS, call
\cfunction{PyMac_Initialize()} instead).  There are optional calls to
pass command line arguments to Python.  Then later you can call the
interpreter from any part of the application.

There are several different ways to call the interpreter: you can pass
a string containing Python statements to
\cfunction{PyRun_SimpleString()}, or you can pass a stdio file pointer
and a file name (for identification in error messages only) to
\cfunction{PyRun_SimpleFile()}.  You can also call the lower-level
operations described in the previous chapters to construct and use
Python objects.

A simple demo of embedding Python can be found in the directory
\file{Demo/embed/} of the source distribution.


\begin{seealso}
  \seetitle[../api/api.html]{Python/C API Reference Manual}{The
            details of Python's C interface are given in this manual.
            A great deal of necessary information can be found here.}
\end{seealso}


\section{Very High Level Embedding
         \label{high-level-embedding}}

The simplest form of embedding Python is the use of the very
high level interface. This interface is intended to execute a
Python script without needing to interact with the application
directly. This can for example be used to perform some operation
on a file.

\begin{verbatim}
#include <Python.h>

int
main(int argc, char *argv[])
{
  Py_Initialize();
  PyRun_SimpleString("from time import time,ctime\n"
                     "print 'Today is',ctime(time())\n");
  Py_Finalize();
  return 0;
}
\end{verbatim}

The above code first initializes the Python interpreter with
\cfunction{Py_Initialize()}, followed by the execution of a hard-coded
Python script that print the date and time.  Afterwards, the
\cfunction{Py_Finalize()} call shuts the interpreter down, followed by
the end of the program.  In a real program, you may want to get the
Python script from another source, perhaps a text-editor routine, a
file, or a database.  Getting the Python code from a file can better
be done by using the \cfunction{PyRun_SimpleFile()} function, which
saves you the trouble of allocating memory space and loading the file
contents.


\section{Beyond Very High Level Embedding: An overview
         \label{lower-level-embedding}}

The high level interface gives you the ability to execute
arbitrary pieces of Python code from your application, but
exchanging data values is quite cumbersome to say the least. If
you want that, you should use lower level calls. At the cost of
having to write more C code, you can achieve almost anything.

It should be noted that extending Python and embedding Python
is quite the same activity, despite the different intent. Most
topics discussed in the previous chapters are still valid. To
show this, consider what the extension code from Python to C
really does:

\begin{enumerate}
    \item Convert data values from Python to C,
    \item Perform a function call to a C routine using the
        converted values, and
    \item Convert the data values from the call from C to Python.
\end{enumerate}

When embedding Python, the interface code does:

\begin{enumerate}
    \item Convert data values from C to Python,
    \item Perform a function call to a Python interface routine
        using the converted values, and
    \item Convert the data values from the call from Python to C.
\end{enumerate}

As you can see, the data conversion steps are simply swapped to
accomodate the different direction of the cross-language transfer.
The only difference is the routine that you call between both
data conversions. When extending, you call a C routine, when
embedding, you call a Python routine.

This chapter will not discuss how to convert data from Python
to C and vice versa.  Also, proper use of references and dealing
with errors is assumed to be understood.  Since these aspects do not
differ from extending the interpreter, you can refer to earlier
chapters for the required information.


\section{Pure Embedding
         \label{pure-embedding}}

The first program aims to execute a function in a Python
script. Like in the section about the very high level interface,
the Python interpreter does not directly interact with the
application (but that will change in th next section).

The code to run a function defined in a Python script is:

\verbatiminput{run-func.c}

This code loads a Python script using \code{argv[1]}, and calls the
function named in \code{argv[2]}.  Its integer arguments are the other
values of the \code{argv} array.  If you compile and link this
program (let's call the finished executable \program{call}), and use
it to execute a Python script, such as:

\begin{verbatim}
def multiply(a,b):
    print "Will compute", a, "times", b
    c = 0
    for i in range(0, a):
        c = c + b
    return c
\end{verbatim}

then the result should be:

\begin{verbatim}
$ call multiply 3 2
Will compute 3 times 2
Result of call: 6
\end{verbatim} % $

Although the program is quite large for its functionality, most of the
code is for data conversion between Python and C, and for error
reporting.  The interesting part with respect to embedding Python
starts with

\begin{verbatim}
    Py_Initialize();
    pName = PyString_FromString(argv[1]);
    /* Error checking of pName left out */
    pModule = PyImport_Import(pName);
\end{verbatim}

After initializing the interpreter, the script is loaded using
\cfunction{PyImport_Import()}.  This routine needs a Python string
as its argument, which is constructed using the
\cfunction{PyString_FromString()} data conversion routine.

\begin{verbatim}
    pFunc = PyObject_GetAttrString(pModule, argv[2]);
    /* pFunc is a new reference */

    if (pFunc && PyCallable_Check(pFunc)) {
        ...
    }
    Py_XDECREF(pFunc);
\end{verbatim}

Once the script is loaded, the name we're looking for is retrieved
using \cfunction{PyObject_GetAttrString()}.  If the name exists, and
the object returned is callable, you can safely assume that it is a
function.  The program then proceeds by constructing a tuple of
arguments as normal.  The call to the Python function is then made
with:

\begin{verbatim}
    pValue = PyObject_CallObject(pFunc, pArgs);
\end{verbatim}

Upon return of the function, \code{pValue} is either \NULL{} or it
contains a reference to the return value of the function.  Be sure to
release the reference after examining the value.


\section{Extending Embedded Python
         \label{extending-with-embedding}}

Until now, the embedded Python interpreter had no access to
functionality from the application itself.  The Python API allows this
by extending the embedded interpreter.  That is, the embedded
interpreter gets extended with routines provided by the application.
While it sounds complex, it is not so bad.  Simply forget for a while
that the application starts the Python interpreter.  Instead, consider
the application to be a set of subroutines, and write some glue code
that gives Python access to those routines, just like you would write
a normal Python extension.  For example:

\begin{verbatim}
static int numargs=0;

/* Return the number of arguments of the application command line */
static PyObject*
emb_numargs(PyObject *self, PyObject *args)
{
    if(!PyArg_ParseTuple(args, ":numargs"))
        return NULL;
    return Py_BuildValue("i", numargs);
}

static PyMethodDef EmbMethods[] = {
    {"numargs", emb_numargs, METH_VARARGS,
     "Return the number of arguments received by the process."},
    {NULL, NULL, 0, NULL}
};
\end{verbatim}

Insert the above code just above the \cfunction{main()} function.
Also, insert the following two statements directly after
\cfunction{Py_Initialize()}:

\begin{verbatim}
    numargs = argc;
    Py_InitModule("emb", EmbMethods);
\end{verbatim}

These two lines initialize the \code{numargs} variable, and make the
\function{emb.numargs()} function accessible to the embedded Python
interpreter.  With these extensions, the Python script can do things
like

\begin{verbatim}
import emb
print "Number of arguments", emb.numargs()
\end{verbatim}

In a real application, the methods will expose an API of the
application to Python.


%\section{For the future}
%
%You don't happen to have a nice library to get textual
%equivalents of numeric values do you :-) ?
%Callbacks here ? (I may be using information from that section
%?!)
%threads
%code examples do not really behave well if errors happen
% (what to watch out for)


\section{Embedding Python in \Cpp
     \label{embeddingInCplusplus}}

It is also possible to embed Python in a \Cpp{} program; precisely how this
is done will depend on the details of the \Cpp{} system used; in general you
will need to write the main program in \Cpp, and use the \Cpp{} compiler
to compile and link your program.  There is no need to recompile Python
itself using \Cpp.


\section{Linking Requirements
         \label{link-reqs}}

While the \program{configure} script shipped with the Python sources
will correctly build Python to export the symbols needed by
dynamically linked extensions, this is not automatically inherited by
applications which embed the Python library statically, at least on
\UNIX.  This is an issue when the application is linked to the static
runtime library (\file{libpython.a}) and needs to load dynamic
extensions (implemented as \file{.so} files).

The problem is that some entry points are defined by the Python
runtime solely for extension modules to use.  If the embedding
application does not use any of these entry points, some linkers will
not include those entries in the symbol table of the finished
executable.  Some additional options are needed to inform the linker
not to remove these symbols.

Determining the right options to use for any given platform can be
quite difficult, but fortunately the Python configuration already has
those values.  To retrieve them from an installed Python interpreter,
start an interactive interpreter and have a short session like this:

\begin{verbatim}
>>> import distutils.sysconfig
>>> distutils.sysconfig.get_config_var('LINKFORSHARED')
'-Xlinker -export-dynamic'
\end{verbatim}
\refstmodindex{distutils.sysconfig}

The contents of the string presented will be the options that should
be used.  If the string is empty, there's no need to add any
additional options.  The \constant{LINKFORSHARED} definition
corresponds to the variable of the same name in Python's top-level
\file{Makefile}.



\appendix
\chapter{Reporting Bugs}
\label{reporting-bugs}

Python is a mature programming language which has established a
reputation for stability.  In order to maintain this reputation, the
developers would like to know of any deficiencies you find in Python
or its documentation.

Before submitting a report, you will be required to log into SourceForge;
this will make it possible for the developers to contact you
for additional information if needed.  It is not possible to submit a
bug report anonymously.

All bug reports should be submitted via the Python Bug Tracker on
SourceForge (\url{http://sourceforge.net/bugs/?group_id=5470}).  The
bug tracker offers a Web form which allows pertinent information to be
entered and submitted to the developers.

The first step in filing a report is to determine whether the problem
has already been reported.  The advantage in doing so, aside from
saving the developers time, is that you learn what has been done to
fix it; it may be that the problem has already been fixed for the next
release, or additional information is needed (in which case you are
welcome to provide it if you can!).  To do this, search the bug
database using the search box near the bottom of the page.

If the problem you're reporting is not already in the bug tracker, go
back to the Python Bug Tracker
(\url{http://sourceforge.net/bugs/?group_id=5470}).  Select the
``Submit a Bug'' link at the top of the page to open the bug reporting
form.

The submission form has a number of fields.  The only fields that are
required are the ``Summary'' and ``Details'' fields.  For the summary,
enter a \emph{very} short description of the problem; less than ten
words is good.  In the Details field, describe the problem in detail,
including what you expected to happen and what did happen.  Be sure to
include the version of Python you used, whether any extension modules
were involved, and what hardware and software platform you were using
(including version information as appropriate).

The only other field that you may want to set is the ``Category''
field, which allows you to place the bug report into a broad category
(such as ``Documentation'' or ``Library'').

Each bug report will be assigned to a developer who will determine
what needs to be done to correct the problem.  You will
receive an update each time action is taken on the bug.


\begin{seealso}
  \seetitle[http://www-mice.cs.ucl.ac.uk/multimedia/software/documentation/ReportingBugs.html]{How
        to Report Bugs Effectively}{Article which goes into some
        detail about how to create a useful bug report.  This
        describes what kind of information is useful and why it is
        useful.}

  \seetitle[http://www.mozilla.org/quality/bug-writing-guidelines.html]{Bug
        Writing Guidelines}{Information about writing a good bug
        report.  Some of this is specific to the Mozilla project, but
        describes general good practices.}
\end{seealso}


\chapter{History and License}
\section{History of the software}

Python was created in the early 1990s by Guido van Rossum at Stichting
Mathematisch Centrum (CWI, see \url{http://www.cwi.nl/}) in the Netherlands
as a successor of a language called ABC.  Guido remains Python's
principal author, although it includes many contributions from others.

In 1995, Guido continued his work on Python at the Corporation for
National Research Initiatives (CNRI, see \url{http://www.cnri.reston.va.us/})
in Reston, Virginia where he released several versions of the
software.

In May 2000, Guido and the Python core development team moved to
BeOpen.com to form the BeOpen PythonLabs team.  In October of the same
year, the PythonLabs team moved to Digital Creations (now Zope
Corporation; see \url{http://www.zope.com/}).  In 2001, the Python
Software Foundation (PSF, see \url{http://www.python.org/psf/}) was
formed, a non-profit organization created specifically to own
Python-related Intellectual Property.  Zope Corporation is a
sponsoring member of the PSF.

All Python releases are Open Source (see
\url{http://www.opensource.org/} for the Open Source Definition).
Historically, most, but not all, Python releases have also been
GPL-compatible; the table below summarizes the various releases.

\begin{tablev}{c|c|c|c|c}{textrm}%
  {Release}{Derived from}{Year}{Owner}{GPL compatible?}
  \linev{0.9.0 thru 1.2}{n/a}{1991-1995}{CWI}{yes}
  \linev{1.3 thru 1.5.2}{1.2}{1995-1999}{CNRI}{yes}
  \linev{1.6}{1.5.2}{2000}{CNRI}{no}
  \linev{2.0}{1.6}{2000}{BeOpen.com}{no}
  \linev{1.6.1}{1.6}{2001}{CNRI}{no}
  \linev{2.1}{2.0+1.6.1}{2001}{PSF}{no}
  \linev{2.0.1}{2.0+1.6.1}{2001}{PSF}{yes}
  \linev{2.1.1}{2.1+2.0.1}{2001}{PSF}{yes}
  \linev{2.2}{2.1.1}{2001}{PSF}{yes}
  \linev{2.1.2}{2.1.1}{2002}{PSF}{yes}
  \linev{2.1.3}{2.1.2}{2002}{PSF}{yes}
  \linev{2.2.1}{2.2}{2002}{PSF}{yes}
  \linev{2.2.2}{2.2.1}{2002}{PSF}{yes}
  \linev{2.2.3}{2.2.2}{2002-2003}{PSF}{yes}
  \linev{2.3}{2.2.2}{2002-2003}{PSF}{yes}
  \linev{2.3.1}{2.3}{2002-2003}{PSF}{yes}
  \linev{2.3.2}{2.3.1}{2003}{PSF}{yes}
  \linev{2.3.3}{2.3.2}{2003}{PSF}{yes}
  \linev{2.3.4}{2.3.3}{2004}{PSF}{yes}
  \linev{2.3.5}{2.3.4}{2005}{PSF}{yes}
  \linev{2.4}{2.3}{2004}{PSF}{yes}
  \linev{2.4.1}{2.4}{2005}{PSF}{yes}
  \linev{2.4.2}{2.4.1}{2005}{PSF}{yes}
  \linev{2.4.3}{2.4.2}{2006}{PSF}{yes}
  \linev{2.5}{2.4}{2006}{PSF}{yes}
  \linev{2.5.1}{2.5}{2007}{PSF}{yes}
\end{tablev}

\note{GPL-compatible doesn't mean that we're distributing
Python under the GPL.  All Python licenses, unlike the GPL, let you
distribute a modified version without making your changes open source.
The GPL-compatible licenses make it possible to combine Python with
other software that is released under the GPL; the others don't.}

Thanks to the many outside volunteers who have worked under Guido's
direction to make these releases possible.


\section{Terms and conditions for accessing or otherwise using Python}

\centerline{\strong{PSF LICENSE AGREEMENT FOR PYTHON \version}}

\begin{enumerate}
\item
This LICENSE AGREEMENT is between the Python Software Foundation
(``PSF''), and the Individual or Organization (``Licensee'') accessing
and otherwise using Python \version{} software in source or binary
form and its associated documentation.

\item
Subject to the terms and conditions of this License Agreement, PSF
hereby grants Licensee a nonexclusive, royalty-free, world-wide
license to reproduce, analyze, test, perform and/or display publicly,
prepare derivative works, distribute, and otherwise use Python
\version{} alone or in any derivative version, provided, however, that
PSF's License Agreement and PSF's notice of copyright, i.e.,
``Copyright \copyright{} 2001-2006 Python Software Foundation; All
Rights Reserved'' are retained in Python \version{} alone or in any
derivative version prepared by Licensee.

\item
In the event Licensee prepares a derivative work that is based on
or incorporates Python \version{} or any part thereof, and wants to
make the derivative work available to others as provided herein, then
Licensee hereby agrees to include in any such work a brief summary of
the changes made to Python \version.

\item
PSF is making Python \version{} available to Licensee on an ``AS IS''
basis.  PSF MAKES NO REPRESENTATIONS OR WARRANTIES, EXPRESS OR
IMPLIED.  BY WAY OF EXAMPLE, BUT NOT LIMITATION, PSF MAKES NO AND
DISCLAIMS ANY REPRESENTATION OR WARRANTY OF MERCHANTABILITY OR FITNESS
FOR ANY PARTICULAR PURPOSE OR THAT THE USE OF PYTHON \version{} WILL
NOT INFRINGE ANY THIRD PARTY RIGHTS.

\item
PSF SHALL NOT BE LIABLE TO LICENSEE OR ANY OTHER USERS OF PYTHON
\version{} FOR ANY INCIDENTAL, SPECIAL, OR CONSEQUENTIAL DAMAGES OR
LOSS AS A RESULT OF MODIFYING, DISTRIBUTING, OR OTHERWISE USING PYTHON
\version, OR ANY DERIVATIVE THEREOF, EVEN IF ADVISED OF THE
POSSIBILITY THEREOF.

\item
This License Agreement will automatically terminate upon a material
breach of its terms and conditions.

\item
Nothing in this License Agreement shall be deemed to create any
relationship of agency, partnership, or joint venture between PSF and
Licensee.  This License Agreement does not grant permission to use PSF
trademarks or trade name in a trademark sense to endorse or promote
products or services of Licensee, or any third party.

\item
By copying, installing or otherwise using Python \version, Licensee
agrees to be bound by the terms and conditions of this License
Agreement.
\end{enumerate}


\centerline{\strong{BEOPEN.COM LICENSE AGREEMENT FOR PYTHON 2.0}}

\centerline{\strong{BEOPEN PYTHON OPEN SOURCE LICENSE AGREEMENT VERSION 1}}

\begin{enumerate}
\item
This LICENSE AGREEMENT is between BeOpen.com (``BeOpen''), having an
office at 160 Saratoga Avenue, Santa Clara, CA 95051, and the
Individual or Organization (``Licensee'') accessing and otherwise
using this software in source or binary form and its associated
documentation (``the Software'').

\item
Subject to the terms and conditions of this BeOpen Python License
Agreement, BeOpen hereby grants Licensee a non-exclusive,
royalty-free, world-wide license to reproduce, analyze, test, perform
and/or display publicly, prepare derivative works, distribute, and
otherwise use the Software alone or in any derivative version,
provided, however, that the BeOpen Python License is retained in the
Software, alone or in any derivative version prepared by Licensee.

\item
BeOpen is making the Software available to Licensee on an ``AS IS''
basis.  BEOPEN MAKES NO REPRESENTATIONS OR WARRANTIES, EXPRESS OR
IMPLIED.  BY WAY OF EXAMPLE, BUT NOT LIMITATION, BEOPEN MAKES NO AND
DISCLAIMS ANY REPRESENTATION OR WARRANTY OF MERCHANTABILITY OR FITNESS
FOR ANY PARTICULAR PURPOSE OR THAT THE USE OF THE SOFTWARE WILL NOT
INFRINGE ANY THIRD PARTY RIGHTS.

\item
BEOPEN SHALL NOT BE LIABLE TO LICENSEE OR ANY OTHER USERS OF THE
SOFTWARE FOR ANY INCIDENTAL, SPECIAL, OR CONSEQUENTIAL DAMAGES OR LOSS
AS A RESULT OF USING, MODIFYING OR DISTRIBUTING THE SOFTWARE, OR ANY
DERIVATIVE THEREOF, EVEN IF ADVISED OF THE POSSIBILITY THEREOF.

\item
This License Agreement will automatically terminate upon a material
breach of its terms and conditions.

\item
This License Agreement shall be governed by and interpreted in all
respects by the law of the State of California, excluding conflict of
law provisions.  Nothing in this License Agreement shall be deemed to
create any relationship of agency, partnership, or joint venture
between BeOpen and Licensee.  This License Agreement does not grant
permission to use BeOpen trademarks or trade names in a trademark
sense to endorse or promote products or services of Licensee, or any
third party.  As an exception, the ``BeOpen Python'' logos available
at http://www.pythonlabs.com/logos.html may be used according to the
permissions granted on that web page.

\item
By copying, installing or otherwise using the software, Licensee
agrees to be bound by the terms and conditions of this License
Agreement.
\end{enumerate}


\centerline{\strong{CNRI LICENSE AGREEMENT FOR PYTHON 1.6.1}}

\begin{enumerate}
\item
This LICENSE AGREEMENT is between the Corporation for National
Research Initiatives, having an office at 1895 Preston White Drive,
Reston, VA 20191 (``CNRI''), and the Individual or Organization
(``Licensee'') accessing and otherwise using Python 1.6.1 software in
source or binary form and its associated documentation.

\item
Subject to the terms and conditions of this License Agreement, CNRI
hereby grants Licensee a nonexclusive, royalty-free, world-wide
license to reproduce, analyze, test, perform and/or display publicly,
prepare derivative works, distribute, and otherwise use Python 1.6.1
alone or in any derivative version, provided, however, that CNRI's
License Agreement and CNRI's notice of copyright, i.e., ``Copyright
\copyright{} 1995-2001 Corporation for National Research Initiatives;
All Rights Reserved'' are retained in Python 1.6.1 alone or in any
derivative version prepared by Licensee.  Alternately, in lieu of
CNRI's License Agreement, Licensee may substitute the following text
(omitting the quotes): ``Python 1.6.1 is made available subject to the
terms and conditions in CNRI's License Agreement.  This Agreement
together with Python 1.6.1 may be located on the Internet using the
following unique, persistent identifier (known as a handle):
1895.22/1013.  This Agreement may also be obtained from a proxy server
on the Internet using the following URL:
\url{http://hdl.handle.net/1895.22/1013}.''

\item
In the event Licensee prepares a derivative work that is based on
or incorporates Python 1.6.1 or any part thereof, and wants to make
the derivative work available to others as provided herein, then
Licensee hereby agrees to include in any such work a brief summary of
the changes made to Python 1.6.1.

\item
CNRI is making Python 1.6.1 available to Licensee on an ``AS IS''
basis.  CNRI MAKES NO REPRESENTATIONS OR WARRANTIES, EXPRESS OR
IMPLIED.  BY WAY OF EXAMPLE, BUT NOT LIMITATION, CNRI MAKES NO AND
DISCLAIMS ANY REPRESENTATION OR WARRANTY OF MERCHANTABILITY OR FITNESS
FOR ANY PARTICULAR PURPOSE OR THAT THE USE OF PYTHON 1.6.1 WILL NOT
INFRINGE ANY THIRD PARTY RIGHTS.

\item
CNRI SHALL NOT BE LIABLE TO LICENSEE OR ANY OTHER USERS OF PYTHON
1.6.1 FOR ANY INCIDENTAL, SPECIAL, OR CONSEQUENTIAL DAMAGES OR LOSS AS
A RESULT OF MODIFYING, DISTRIBUTING, OR OTHERWISE USING PYTHON 1.6.1,
OR ANY DERIVATIVE THEREOF, EVEN IF ADVISED OF THE POSSIBILITY THEREOF.

\item
This License Agreement will automatically terminate upon a material
breach of its terms and conditions.

\item
This License Agreement shall be governed by the federal
intellectual property law of the United States, including without
limitation the federal copyright law, and, to the extent such
U.S. federal law does not apply, by the law of the Commonwealth of
Virginia, excluding Virginia's conflict of law provisions.
Notwithstanding the foregoing, with regard to derivative works based
on Python 1.6.1 that incorporate non-separable material that was
previously distributed under the GNU General Public License (GPL), the
law of the Commonwealth of Virginia shall govern this License
Agreement only as to issues arising under or with respect to
Paragraphs 4, 5, and 7 of this License Agreement.  Nothing in this
License Agreement shall be deemed to create any relationship of
agency, partnership, or joint venture between CNRI and Licensee.  This
License Agreement does not grant permission to use CNRI trademarks or
trade name in a trademark sense to endorse or promote products or
services of Licensee, or any third party.

\item
By clicking on the ``ACCEPT'' button where indicated, or by copying,
installing or otherwise using Python 1.6.1, Licensee agrees to be
bound by the terms and conditions of this License Agreement.
\end{enumerate}

\centerline{ACCEPT}



\centerline{\strong{CWI LICENSE AGREEMENT FOR PYTHON 0.9.0 THROUGH 1.2}}

Copyright \copyright{} 1991 - 1995, Stichting Mathematisch Centrum
Amsterdam, The Netherlands.  All rights reserved.

Permission to use, copy, modify, and distribute this software and its
documentation for any purpose and without fee is hereby granted,
provided that the above copyright notice appear in all copies and that
both that copyright notice and this permission notice appear in
supporting documentation, and that the name of Stichting Mathematisch
Centrum or CWI not be used in advertising or publicity pertaining to
distribution of the software without specific, written prior
permission.

STICHTING MATHEMATISCH CENTRUM DISCLAIMS ALL WARRANTIES WITH REGARD TO
THIS SOFTWARE, INCLUDING ALL IMPLIED WARRANTIES OF MERCHANTABILITY AND
FITNESS, IN NO EVENT SHALL STICHTING MATHEMATISCH CENTRUM BE LIABLE
FOR ANY SPECIAL, INDIRECT OR CONSEQUENTIAL DAMAGES OR ANY DAMAGES
WHATSOEVER RESULTING FROM LOSS OF USE, DATA OR PROFITS, WHETHER IN AN
ACTION OF CONTRACT, NEGLIGENCE OR OTHER TORTIOUS ACTION, ARISING OUT
OF OR IN CONNECTION WITH THE USE OR PERFORMANCE OF THIS SOFTWARE.


\section{Licenses and Acknowledgements for Incorporated Software}

This section is an incomplete, but growing list of licenses and
acknowledgements for third-party software incorporated in the
Python distribution.


\subsection{Mersenne Twister}

The \module{_random} module includes code based on a download from
\url{http://www.math.keio.ac.jp/~matumoto/MT2002/emt19937ar.html}.
The following are the verbatim comments from the original code:

\begin{verbatim}
A C-program for MT19937, with initialization improved 2002/1/26.
Coded by Takuji Nishimura and Makoto Matsumoto.

Before using, initialize the state by using init_genrand(seed)
or init_by_array(init_key, key_length).

Copyright (C) 1997 - 2002, Makoto Matsumoto and Takuji Nishimura,
All rights reserved.

Redistribution and use in source and binary forms, with or without
modification, are permitted provided that the following conditions
are met:

 1. Redistributions of source code must retain the above copyright
    notice, this list of conditions and the following disclaimer.

 2. Redistributions in binary form must reproduce the above copyright
    notice, this list of conditions and the following disclaimer in the
    documentation and/or other materials provided with the distribution.

 3. The names of its contributors may not be used to endorse or promote
    products derived from this software without specific prior written
    permission.

THIS SOFTWARE IS PROVIDED BY THE COPYRIGHT HOLDERS AND CONTRIBUTORS
"AS IS" AND ANY EXPRESS OR IMPLIED WARRANTIES, INCLUDING, BUT NOT
LIMITED TO, THE IMPLIED WARRANTIES OF MERCHANTABILITY AND FITNESS FOR
A PARTICULAR PURPOSE ARE DISCLAIMED.  IN NO EVENT SHALL THE COPYRIGHT OWNER OR
CONTRIBUTORS BE LIABLE FOR ANY DIRECT, INDIRECT, INCIDENTAL, SPECIAL,
EXEMPLARY, OR CONSEQUENTIAL DAMAGES (INCLUDING, BUT NOT LIMITED TO,
PROCUREMENT OF SUBSTITUTE GOODS OR SERVICES; LOSS OF USE, DATA, OR
PROFITS; OR BUSINESS INTERRUPTION) HOWEVER CAUSED AND ON ANY THEORY OF
LIABILITY, WHETHER IN CONTRACT, STRICT LIABILITY, OR TORT (INCLUDING
NEGLIGENCE OR OTHERWISE) ARISING IN ANY WAY OUT OF THE USE OF THIS
SOFTWARE, EVEN IF ADVISED OF THE POSSIBILITY OF SUCH DAMAGE.


Any feedback is very welcome.
http://www.math.keio.ac.jp/matumoto/emt.html
email: matumoto@math.keio.ac.jp
\end{verbatim}



\subsection{Sockets}

The \module{socket} module uses the functions, \function{getaddrinfo},
and \function{getnameinfo}, which are coded in separate source files
from the WIDE Project, \url{http://www.wide.ad.jp/about/index.html}.

\begin{verbatim}      
Copyright (C) 1995, 1996, 1997, and 1998 WIDE Project.
All rights reserved.
 
Redistribution and use in source and binary forms, with or without
modification, are permitted provided that the following conditions
are met:
1. Redistributions of source code must retain the above copyright
   notice, this list of conditions and the following disclaimer.
2. Redistributions in binary form must reproduce the above copyright
   notice, this list of conditions and the following disclaimer in the
   documentation and/or other materials provided with the distribution.
3. Neither the name of the project nor the names of its contributors
   may be used to endorse or promote products derived from this software
   without specific prior written permission.

THIS SOFTWARE IS PROVIDED BY THE PROJECT AND CONTRIBUTORS ``AS IS'' AND
GAI_ANY EXPRESS OR IMPLIED WARRANTIES, INCLUDING, BUT NOT LIMITED TO, THE
IMPLIED WARRANTIES OF MERCHANTABILITY AND FITNESS FOR A PARTICULAR PURPOSE
ARE DISCLAIMED.  IN NO EVENT SHALL THE PROJECT OR CONTRIBUTORS BE LIABLE
FOR GAI_ANY DIRECT, INDIRECT, INCIDENTAL, SPECIAL, EXEMPLARY, OR CONSEQUENTIAL
DAMAGES (INCLUDING, BUT NOT LIMITED TO, PROCUREMENT OF SUBSTITUTE GOODS
OR SERVICES; LOSS OF USE, DATA, OR PROFITS; OR BUSINESS INTERRUPTION)
HOWEVER CAUSED AND ON GAI_ANY THEORY OF LIABILITY, WHETHER IN CONTRACT, STRICT
LIABILITY, OR TORT (INCLUDING NEGLIGENCE OR OTHERWISE) ARISING IN GAI_ANY WAY
OUT OF THE USE OF THIS SOFTWARE, EVEN IF ADVISED OF THE POSSIBILITY OF
SUCH DAMAGE.
\end{verbatim}



\subsection{Floating point exception control}

The source for the \module{fpectl} module includes the following notice:

\begin{verbatim}
     ---------------------------------------------------------------------  
    /                       Copyright (c) 1996.                           \ 
   |          The Regents of the University of California.                 |
   |                        All rights reserved.                           |
   |                                                                       |
   |   Permission to use, copy, modify, and distribute this software for   |
   |   any purpose without fee is hereby granted, provided that this en-   |
   |   tire notice is included in all copies of any software which is or   |
   |   includes  a  copy  or  modification  of  this software and in all   |
   |   copies of the supporting documentation for such software.           |
   |                                                                       |
   |   This  work was produced at the University of California, Lawrence   |
   |   Livermore National Laboratory under  contract  no.  W-7405-ENG-48   |
   |   between  the  U.S.  Department  of  Energy and The Regents of the   |
   |   University of California for the operation of UC LLNL.              |
   |                                                                       |
   |                              DISCLAIMER                               |
   |                                                                       |
   |   This  software was prepared as an account of work sponsored by an   |
   |   agency of the United States Government. Neither the United States   |
   |   Government  nor the University of California nor any of their em-   |
   |   ployees, makes any warranty, express or implied, or  assumes  any   |
   |   liability  or  responsibility  for the accuracy, completeness, or   |
   |   usefulness of any information,  apparatus,  product,  or  process   |
   |   disclosed,   or  represents  that  its  use  would  not  infringe   |
   |   privately-owned rights. Reference herein to any specific  commer-   |
   |   cial  products,  process,  or  service  by trade name, trademark,   |
   |   manufacturer, or otherwise, does not  necessarily  constitute  or   |
   |   imply  its endorsement, recommendation, or favoring by the United   |
   |   States Government or the University of California. The views  and   |
   |   opinions  of authors expressed herein do not necessarily state or   |
   |   reflect those of the United States Government or  the  University   |
   |   of  California,  and shall not be used for advertising or product   |
    \  endorsement purposes.                                              / 
     ---------------------------------------------------------------------
\end{verbatim}



\subsection{MD5 message digest algorithm}

The source code for the \module{md5} module contains the following notice:

\begin{verbatim}
  Copyright (C) 1999, 2002 Aladdin Enterprises.  All rights reserved.

  This software is provided 'as-is', without any express or implied
  warranty.  In no event will the authors be held liable for any damages
  arising from the use of this software.

  Permission is granted to anyone to use this software for any purpose,
  including commercial applications, and to alter it and redistribute it
  freely, subject to the following restrictions:

  1. The origin of this software must not be misrepresented; you must not
     claim that you wrote the original software. If you use this software
     in a product, an acknowledgment in the product documentation would be
     appreciated but is not required.
  2. Altered source versions must be plainly marked as such, and must not be
     misrepresented as being the original software.
  3. This notice may not be removed or altered from any source distribution.

  L. Peter Deutsch
  ghost@aladdin.com

  Independent implementation of MD5 (RFC 1321).

  This code implements the MD5 Algorithm defined in RFC 1321, whose
  text is available at
	http://www.ietf.org/rfc/rfc1321.txt
  The code is derived from the text of the RFC, including the test suite
  (section A.5) but excluding the rest of Appendix A.  It does not include
  any code or documentation that is identified in the RFC as being
  copyrighted.

  The original and principal author of md5.h is L. Peter Deutsch
  <ghost@aladdin.com>.  Other authors are noted in the change history
  that follows (in reverse chronological order):

  2002-04-13 lpd Removed support for non-ANSI compilers; removed
	references to Ghostscript; clarified derivation from RFC 1321;
	now handles byte order either statically or dynamically.
  1999-11-04 lpd Edited comments slightly for automatic TOC extraction.
  1999-10-18 lpd Fixed typo in header comment (ansi2knr rather than md5);
	added conditionalization for C++ compilation from Martin
	Purschke <purschke@bnl.gov>.
  1999-05-03 lpd Original version.
\end{verbatim}



\subsection{Asynchronous socket services}

The \module{asynchat} and \module{asyncore} modules contain the
following notice:

\begin{verbatim}      
 Copyright 1996 by Sam Rushing

                         All Rights Reserved

 Permission to use, copy, modify, and distribute this software and
 its documentation for any purpose and without fee is hereby
 granted, provided that the above copyright notice appear in all
 copies and that both that copyright notice and this permission
 notice appear in supporting documentation, and that the name of Sam
 Rushing not be used in advertising or publicity pertaining to
 distribution of the software without specific, written prior
 permission.

 SAM RUSHING DISCLAIMS ALL WARRANTIES WITH REGARD TO THIS SOFTWARE,
 INCLUDING ALL IMPLIED WARRANTIES OF MERCHANTABILITY AND FITNESS, IN
 NO EVENT SHALL SAM RUSHING BE LIABLE FOR ANY SPECIAL, INDIRECT OR
 CONSEQUENTIAL DAMAGES OR ANY DAMAGES WHATSOEVER RESULTING FROM LOSS
 OF USE, DATA OR PROFITS, WHETHER IN AN ACTION OF CONTRACT,
 NEGLIGENCE OR OTHER TORTIOUS ACTION, ARISING OUT OF OR IN
 CONNECTION WITH THE USE OR PERFORMANCE OF THIS SOFTWARE.
\end{verbatim}


\subsection{Cookie management}

The \module{Cookie} module contains the following notice:

\begin{verbatim}
 Copyright 2000 by Timothy O'Malley <timo@alum.mit.edu>

                All Rights Reserved

 Permission to use, copy, modify, and distribute this software
 and its documentation for any purpose and without fee is hereby
 granted, provided that the above copyright notice appear in all
 copies and that both that copyright notice and this permission
 notice appear in supporting documentation, and that the name of
 Timothy O'Malley  not be used in advertising or publicity
 pertaining to distribution of the software without specific, written
 prior permission.

 Timothy O'Malley DISCLAIMS ALL WARRANTIES WITH REGARD TO THIS
 SOFTWARE, INCLUDING ALL IMPLIED WARRANTIES OF MERCHANTABILITY
 AND FITNESS, IN NO EVENT SHALL Timothy O'Malley BE LIABLE FOR
 ANY SPECIAL, INDIRECT OR CONSEQUENTIAL DAMAGES OR ANY DAMAGES
 WHATSOEVER RESULTING FROM LOSS OF USE, DATA OR PROFITS,
 WHETHER IN AN ACTION OF CONTRACT, NEGLIGENCE OR OTHER TORTIOUS
 ACTION, ARISING OUT OF OR IN CONNECTION WITH THE USE OR
 PERFORMANCE OF THIS SOFTWARE.
\end{verbatim}      



\subsection{Profiling}

The \module{profile} and \module{pstats} modules contain
the following notice:

\begin{verbatim}
 Copyright 1994, by InfoSeek Corporation, all rights reserved.
 Written by James Roskind

 Permission to use, copy, modify, and distribute this Python software
 and its associated documentation for any purpose (subject to the
 restriction in the following sentence) without fee is hereby granted,
 provided that the above copyright notice appears in all copies, and
 that both that copyright notice and this permission notice appear in
 supporting documentation, and that the name of InfoSeek not be used in
 advertising or publicity pertaining to distribution of the software
 without specific, written prior permission.  This permission is
 explicitly restricted to the copying and modification of the software
 to remain in Python, compiled Python, or other languages (such as C)
 wherein the modified or derived code is exclusively imported into a
 Python module.

 INFOSEEK CORPORATION DISCLAIMS ALL WARRANTIES WITH REGARD TO THIS
 SOFTWARE, INCLUDING ALL IMPLIED WARRANTIES OF MERCHANTABILITY AND
 FITNESS. IN NO EVENT SHALL INFOSEEK CORPORATION BE LIABLE FOR ANY
 SPECIAL, INDIRECT OR CONSEQUENTIAL DAMAGES OR ANY DAMAGES WHATSOEVER
 RESULTING FROM LOSS OF USE, DATA OR PROFITS, WHETHER IN AN ACTION OF
 CONTRACT, NEGLIGENCE OR OTHER TORTIOUS ACTION, ARISING OUT OF OR IN
 CONNECTION WITH THE USE OR PERFORMANCE OF THIS SOFTWARE.
\end{verbatim}



\subsection{Execution tracing}

The \module{trace} module contains the following notice:

\begin{verbatim}
 portions copyright 2001, Autonomous Zones Industries, Inc., all rights...
 err...  reserved and offered to the public under the terms of the
 Python 2.2 license.
 Author: Zooko O'Whielacronx
 http://zooko.com/
 mailto:zooko@zooko.com

 Copyright 2000, Mojam Media, Inc., all rights reserved.
 Author: Skip Montanaro

 Copyright 1999, Bioreason, Inc., all rights reserved.
 Author: Andrew Dalke

 Copyright 1995-1997, Automatrix, Inc., all rights reserved.
 Author: Skip Montanaro

 Copyright 1991-1995, Stichting Mathematisch Centrum, all rights reserved.


 Permission to use, copy, modify, and distribute this Python software and
 its associated documentation for any purpose without fee is hereby
 granted, provided that the above copyright notice appears in all copies,
 and that both that copyright notice and this permission notice appear in
 supporting documentation, and that the name of neither Automatrix,
 Bioreason or Mojam Media be used in advertising or publicity pertaining to
 distribution of the software without specific, written prior permission.
\end{verbatim} 



\subsection{UUencode and UUdecode functions}

The \module{uu} module contains the following notice:

\begin{verbatim}
 Copyright 1994 by Lance Ellinghouse
 Cathedral City, California Republic, United States of America.
                        All Rights Reserved
 Permission to use, copy, modify, and distribute this software and its
 documentation for any purpose and without fee is hereby granted,
 provided that the above copyright notice appear in all copies and that
 both that copyright notice and this permission notice appear in
 supporting documentation, and that the name of Lance Ellinghouse
 not be used in advertising or publicity pertaining to distribution
 of the software without specific, written prior permission.
 LANCE ELLINGHOUSE DISCLAIMS ALL WARRANTIES WITH REGARD TO
 THIS SOFTWARE, INCLUDING ALL IMPLIED WARRANTIES OF MERCHANTABILITY AND
 FITNESS, IN NO EVENT SHALL LANCE ELLINGHOUSE CENTRUM BE LIABLE
 FOR ANY SPECIAL, INDIRECT OR CONSEQUENTIAL DAMAGES OR ANY DAMAGES
 WHATSOEVER RESULTING FROM LOSS OF USE, DATA OR PROFITS, WHETHER IN AN
 ACTION OF CONTRACT, NEGLIGENCE OR OTHER TORTIOUS ACTION, ARISING OUT
 OF OR IN CONNECTION WITH THE USE OR PERFORMANCE OF THIS SOFTWARE.

 Modified by Jack Jansen, CWI, July 1995:
 - Use binascii module to do the actual line-by-line conversion
   between ascii and binary. This results in a 1000-fold speedup. The C
   version is still 5 times faster, though.
 - Arguments more compliant with python standard
\end{verbatim}



\subsection{XML Remote Procedure Calls}

The \module{xmlrpclib} module contains the following notice:

\begin{verbatim}
     The XML-RPC client interface is

 Copyright (c) 1999-2002 by Secret Labs AB
 Copyright (c) 1999-2002 by Fredrik Lundh

 By obtaining, using, and/or copying this software and/or its
 associated documentation, you agree that you have read, understood,
 and will comply with the following terms and conditions:

 Permission to use, copy, modify, and distribute this software and
 its associated documentation for any purpose and without fee is
 hereby granted, provided that the above copyright notice appears in
 all copies, and that both that copyright notice and this permission
 notice appear in supporting documentation, and that the name of
 Secret Labs AB or the author not be used in advertising or publicity
 pertaining to distribution of the software without specific, written
 prior permission.

 SECRET LABS AB AND THE AUTHOR DISCLAIMS ALL WARRANTIES WITH REGARD
 TO THIS SOFTWARE, INCLUDING ALL IMPLIED WARRANTIES OF MERCHANT-
 ABILITY AND FITNESS.  IN NO EVENT SHALL SECRET LABS AB OR THE AUTHOR
 BE LIABLE FOR ANY SPECIAL, INDIRECT OR CONSEQUENTIAL DAMAGES OR ANY
 DAMAGES WHATSOEVER RESULTING FROM LOSS OF USE, DATA OR PROFITS,
 WHETHER IN AN ACTION OF CONTRACT, NEGLIGENCE OR OTHER TORTIOUS
 ACTION, ARISING OUT OF OR IN CONNECTION WITH THE USE OR PERFORMANCE
 OF THIS SOFTWARE.
\end{verbatim}


\end{document}
