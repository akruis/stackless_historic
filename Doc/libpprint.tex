%%  Author:  Fred L. Drake, Jr.		<fdrake@acm.org>

\section{Standard module \sectcode{pprint}}
\stmodindex{pprint}

The \code{pprint} module provides a capability to ``pretty-print''
arbitrary Python data structures in a form which can be used as input
to the interpreter.  If the formatted structures include objects which
are not fundamental Python types, the representation may not be
loadable.  This may be the case if objects such as files, sockets,
classes, or instances are included, as well as many other builtin
objects which are not representable as Python constants.

The formatted representation keeps objects on a single line if it can,
and breaks them out onto multiple lines if they won't fit within the
width allowed width.  Construct PrettyPrinter objects explicitly if
you need to adjust the width constraint.

The \code{pprint} module defines the following functions:

\renewcommand{\indexsubitem}{(in module pprint)}

\begin{funcdesc}{pformat}{object}
Return the formatted representation of \var{object} as a string.  The
default parameters for formatting are used.
\end{funcdesc}

\begin{funcdesc}{pprint}{object\optional{, stream}}
Prints the formatted representation of \var{object} on \var{stream},
followed by a newline.  If \var{stream} is omitted, \code{sys.stdout}
is used.  This may be used in the interactive interpreter instead of a
\code{print} command for inspecting values.  The default parameters
for formatting are used.
\end{funcdesc}

\begin{funcdesc}{isreadable}{object}
Determine if the formatted representation of \var{object} is
``readable,'' or can be used to reconstruct the value using
\code{eval()}.  Note that this returns false for recursive objects.
\end{funcdesc}

\begin{funcdesc}{isrecursive}{object}
Determine if \var{object} requires a recursive representation.
\end{funcdesc}

\begin{funcdesc}{saferepr}{object}
Return a string representation of \var{object}, protected against
recursive data structures.  If the representation of \var{object}
exposes a recursive entry, the recursive reference will be represented
as \samp{$<$Recursion on \var{typename} with id=\var{number}$>$}.
\end{funcdesc}


% Now for the implementation class:

\begin{funcdesc}{PrettyPrinter}{...}
Construct a PrettyPrinter instance.  This constructor understands
several keyword parameters.  An output stream may be set using the
\var{stream} keyword; the only method used on the stream object is the
file protocol's \code{write()} method.  If not specified, the
PrettyPrinter adopts \code{sys.stdout}.  Three additional parameters
may be used to control the formatted representation.  The keywords are
\var{indent}, \var{depth}, and \var{width}.  The amount of indentation
added for each recursive level is specified by \var{indent}; the
default is one.  Other values can cause output to look a little odd,
but can make nesting easier to spot.  The number of levels which may
be printed is controlled by \var{depth}; if the data structure being
printed is too deep, the next contained level is replaced by
\samp{...}.  By default, there is no constraint on the depth of the
objects being formatted.  The desired output width is constrained
using the \var{width} parameter; the default is eighty characters.  If
a structure cannot be formatted within the constrained width, a best
effort will be made.
\end{funcdesc}


% Guido marked this as a good spot for an example in the template,
% but I think this needs a better location in this module.  Not sure where.

Example:

\begin{verbatim}
>>> import pprint
>>> stuff = sys.path[:]
>>> stuff.insert(0, stuff)
>>> pprint.pprint(stuff)
[<Recursion on list with id=869440>,
 '',
 '/usr/local/lib/python1.4',
 '/usr/local/lib/python1.4/test',
 '/usr/local/lib/python1.4/sunos5',
 '/usr/local/lib/python1.4/sharedmodules',
 '/usr/local/lib/python1.4/tkinter']
>>> 
>>> stuff[0] = stuff[1:]
>>> pp = pprint.PrettyPrinter(indent=4)
>>> pp.pprint(stuff)
[   [   '',
        '/usr/local/lib/python1.4',
        '/usr/local/lib/python1.4/test',
        '/usr/local/lib/python1.4/sunos5',
        '/usr/local/lib/python1.4/sharedmodules',
        '/usr/local/lib/python1.4/tkinter'],
    '',
    '/usr/local/lib/python1.4',
    '/usr/local/lib/python1.4/test',
    '/usr/local/lib/python1.4/sunos5',
    '/usr/local/lib/python1.4/sharedmodules',
    '/usr/local/lib/python1.4/tkinter']
>>>
>>> import parser
>>> tup = parser.ast2tuple(
...     parser.suite(open('pprint.py').read()))[1][1][1]
>>> pp = pprint.PrettyPrinter(depth=6)
>>> pp.pprint(tup)
(266, (267, (307, (287, (288, (...))))))
\end{verbatim}


\subsection{PrettyPrinter Objects}

PrettyPrinter instances (returned by \code{PrettyPrinter()} above)
have the following methods.

\renewcommand{\indexsubitem}{(PrettyPrinter method)}

\begin{funcdesc}{pformat}{object}
Return the formatted representation of \var{object}.  This takes into
account the options passed to the PrettyPrinter constructor.
\end{funcdesc}

\begin{funcdesc}{pprint}{object}
Print the formatted representation of \var{object} on the configured
stream, followed by a newline.
\end{funcdesc}

The following methods provide the implementations for the
corresponding functions of the same names.  Using these methods on an
instance is slightly more efficient since new PrettyPrinter objects
don't need to be created.

\begin{funcdesc}{isreadable}{object}
Determine if the formatted representation of the object is
``readable,'' or can be used to reconstruct the value using
\code{eval()}.  Note that this returns false for recursive objects.
If the \var{depth} parameter of the PrettyPrinter is set and the
object is deeper than allowed, this returns false.
\end{funcdesc}

\begin{funcdesc}{isrecursive}{object}
Determine if the object requires a recursive representation.
\end{funcdesc}
