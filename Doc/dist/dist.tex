\documentclass{howto}
\usepackage{distutils}

% $Id$

\title{Distributing Python Modules}

\author{Greg Ward}
\authoraddress{E-mail: \email{gward@python.net}}

\makeindex

\begin{document}

\maketitle
\begin{abstract}
  \noindent
  This document describes the Python Distribution Utilities
  (``Distutils'') from the module developer's point-of-view, describing
  how to use the Distutils to make Python modules and extensions easily
  available to a wider audience with very little overhead for
  build/release/install mechanics.
\end{abstract}

\tableofcontents

\section{Introduction}
\label{intro}

In the past, Python module developers have not had much infrastructure
support for distributing modules, nor have Python users had much support
for installing and maintaining third-party modules.  With the
introduction of the Python Distribution Utilities (Distutils for short)
in Python 1.6, this situation should start to improve.

This document only covers using the Distutils to distribute your Python
modules.  Using the Distutils does not tie you to Python 1.6, though:
the Distutils work just fine with Python 1.5.2, and it is reasonable
(and expected to become commonplace) to expect users of Python 1.5.2 to
download and install the Distutils separately before they can install
your modules.  Python 1.6 (or later) users, of course, won't have to add
anything to their Python installation in order to use the Distutils to
install third-party modules.

This document concentrates on the role of developer/distributor: if
you're looking for information on installing Python modules, you
should refer to the \citetitle[../inst/inst.html]{Installing Python
Modules} manual.


\section{Concepts \& Terminology}
\label{concepts}

Using the Distutils is quite simple, both for module developers and for
users/administrators installing third-party modules.  As a developer,
your responsibilites (apart from writing solid, well-documented and
well-tested code, of course!) are:
\begin{itemize}
\item write a setup script (\file{setup.py} by convention)
\item (optional) write a setup configuration file
\item create a source distribution
\item (optional) create one or more built (binary) distributions
\end{itemize}
Each of these tasks is covered in this document.

Not all module developers have access to a multitude of platforms, so
it's not always feasible to expect them to create a multitude of built
distributions.  It is hoped that a class of intermediaries, called
\emph{packagers}, will arise to address this need.  Packagers will take
source distributions released by module developers, build them on one or
more platforms, and release the resulting built distributions.  Thus,
users on the most popular platforms will be able to install most popular
Python module distributions in the most natural way for their platform,
without having to run a single setup script or compile a line of code.


\subsection{A simple example}
\label{simple-example}

The setup script is usually quite simple, although since it's written in
Python, there are no arbitrary limits to what you can do with
it.\footnote{But be careful about putting arbitrarily expensive
  operations in your setup script; unlike, say, Autoconf-style configure
  scripts, the setup script may be run multiple times in the course of
  building and installing your module distribution.  If you need to
  insert potentially expensive processing steps into the Distutils
  chain, see section~\ref{extending} on extending the Distutils.}  If
all you want to do is distribute a module called \module{foo}, contained
in a file \file{foo.py}, then your setup script can be as little as
this:
\begin{verbatim}
from distutils.core import setup
setup (name = "foo",
       version = "1.0",
       py_modules = ["foo"])
\end{verbatim}

Some observations:
\begin{itemize}
\item most information that you supply to the Distutils is supplied as
  keyword arguments to the \function{setup()} function
\item those keyword arguments fall into two categories: package
  meta-data (name, version number) and information about what's in the
  package (a list of pure Python modules, in this case)
\item modules are specified by module name, not filename (the same will
  hold true for packages and extensions)
\item it's recommended that you supply a little more meta-data, in
  particular your name, email address and a URL for the project
  (see section~\ref{setup-script} for an example)
\end{itemize}

To create a source distribution for this module, you would create a
setup script, \file{setup.py}, containing the above code, and run:
\begin{verbatim}
python setup.py sdist
\end{verbatim}
which will create an archive file (e.g., tarball on \UNIX, ZIP file on
Windows) containing your setup script, \file{setup.py}, and your module,
\file{foo.py}.  The archive file will be named \file{Foo-1.0.tar.gz} (or
\file{.zip}), and will unpack into a directory \file{Foo-1.0}.

If an end-user wishes to install your \module{foo} module, all she has
to do is download \file{Foo-1.0.tar.gz} (or \file{.zip}), unpack it,
and---from the \file{Foo-1.0} directory---run
\begin{verbatim}
python setup.py install
\end{verbatim}
which will ultimately copy \file{foo.py} to the appropriate directory
for third-party modules in their Python installation.

This simple example demonstrates some fundamental concepts of the
Distutils: first, both developers and installers have the same basic
user interface, i.e. the setup script.  The difference is which
Distutils \emph{commands} they use: the \command{sdist} command is
almost exclusively for module developers, while \command{install} is
more often for installers (although most developers will want to install
their own code occasionally).

If you want to make things really easy for your users, you can create
one or more built distributions for them.  For instance, if you are
running on a Windows machine, and want to make things easy for other
Windows users, you can create an executable installer (the most
appropriate type of built distribution for this platform) with the
\command{bdist\_wininst} command.  For example:
\begin{verbatim}
python setup.py bdist_wininst
\end{verbatim}
will create an executable installer, \file{Foo-1.0.win32.exe}, in the
current directory.

\XXX{not implemented yet}
(Another way to create executable installers for Windows is with the
\command{bdist\_wise} command, which uses Wise---the commercial
installer-generator used to create Python's own installer---to create
the installer.  Wise-based installers are more appropriate for large,
industrial-strength applications that need the full capabilities of a
``real'' installer.  \command{bdist\_wininst} creates a self-extracting
zip file with a minimal user interface, which is enough for small- to
medium-sized module collections.  You'll need to have version XXX of
Wise installed on your system for the \command{bdist\_wise} command to
work; it's available from \url{http://foo/bar/baz}.)

Currently (Distutils 0.9.2), the are only other useful built
distribution format is RPM, implemented by the \command{bdist\_rpm}
command.  For example, the following command will create an RPM file
called \file{Foo-1.0.noarch.rpm}:
\begin{verbatim}
python setup.py bdist_rpm
\end{verbatim}
(This uses the \command{rpm} command, so has to be run on an RPM-based
system such as Red Hat Linux, SuSE Linux, or Mandrake Linux.)

You can find out what distribution formats are available at any time by
running
\begin{verbatim}
python setup.py bdist --help-formats
\end{verbatim}


\subsection{General Python terminology}
\label{python-terms}

If you're reading this document, you probably have a good idea of what
modules, extensions, and so forth are.  Nevertheless, just to be sure
that everyone is operating from a common starting point, we offer the
following glossary of common Python terms:
\begin{description}
\item[module] the basic unit of code reusability in Python: a block of
  code imported by some other code.  Three types of modules concern us
  here: pure Python modules, extension modules, and packages.
\item[pure Python module] a module written in Python and contained in a
  single \file{.py} file (and possibly associated \file{.pyc} and/or
  \file{.pyo} files).  Sometimes referred to as a ``pure module.''
\item[extension module] a module written in the low-level language of
  the Python implemention: C/C++ for Python, Java for JPython.
  Typically contained in a single dynamically loadable pre-compiled
  file, e.g. a shared object (\file{.so}) file for Python extensions on
  \UNIX, a DLL (given the \file{.pyd} extension) for Python extensions
  on Windows, or a Java class file for JPython extensions.  (Note that
  currently, the Distutils only handles C/C++ extensions for Python.)
\item[package] a module that contains other modules; typically contained
  in a directory in the filesystem and distinguished from other
  directories by the presence of a file \file{\_\_init\_\_.py}.
\item[root package] the root of the hierarchy of packages.  (This isn't
  really a package, since it doesn't have an \file{\_\_init\_\_.py}
  file.  But we have to call it something.)  The vast majority of the
  standard library is in the root package, as are many small, standalone
  third-party modules that don't belong to a larger module collection.
  Unlike regular packages, modules in the root package can be found in
  many directories: in fact, every directory listed in \code{sys.path}
  can contribute modules to the root package.
\end{description}


\subsection{Distutils-specific terminology}
\label{distutils-term}

The following terms apply more specifically to the domain of
distributing Python modules using the Distutils:
\begin{description}
\item[module distribution] a collection of Python modules distributed
  together as a single downloadable resource and meant to be installed
  \emph{en masse}.  Examples of some well-known module distributions are
  Numeric Python, PyXML, PIL (the Python Imaging Library), or
  mxDateTime.  (This would be called a \emph{package}, except that term
  is already taken in the Python context: a single module distribution
  may contain zero, one, or many Python packages.)
\item[pure module distribution] a module distribution that contains only
  pure Python modules and packages.  Sometimes referred to as a ``pure
  distribution.''
\item[non-pure module distribution] a module distribution that contains
  at least one extension module.  Sometimes referred to as a ``non-pure
  distribution.''
\item[distribution root] the top-level directory of your source tree (or 
  source distribution); the directory where \file{setup.py} exists and
  is run from
\end{description}


\section{Writing the Setup Script}
\label{setup-script}

The setup script is the centre of all activity in building,
distributing, and installing modules using the Distutils.  The main
purpose of the setup script is to describe your module distribution to
the Distutils, so that the various commands that operate on your modules
do the right thing.  As we saw in section~\ref{simple-example} above,
the setup script consists mainly of a call to \function{setup()}, and
most information supplied to the Distutils by the module developer is
supplied as keyword arguments to \function{setup()}.

Here's a slightly more involved example, which we'll follow for the next
couple of sections: the Distutils' own setup script.  (Keep in mind that
although the Distutils are included with Python 1.6 and later, they also
have an independent existence so that Python 1.5.2 users can use them to
install other module distributions.  The Distutils' own setup script,
shown here, is used to install the package into Python 1.5.2.)

\begin{verbatim}
#!/usr/bin/env python

from distutils.core import setup

setup (name = "Distutils",
       version = "1.0",
       description = "Python Distribution Utilities",
       author = "Greg Ward",
       author_email = "gward@python.net",
       url = "http://www.python.org/sigs/distutils-sig/",

       packages = ['distutils', 'distutils.command'],
      )
\end{verbatim}
There are only two differences between this and the trivial one-file
distribution presented in section~\ref{simple-example}: more
meta-data, and the specification of pure Python modules by package,
rather than by module.  This is important since the Distutils consist of
a couple of dozen modules split into (so far) two packages; an explicit
list of every module would be tedious to generate and difficult to
maintain.

Note that any pathnames (files or directories) supplied in the setup
script should be written using the \UNIX{} convention, i.e.
slash-separated.  The Distutils will take care of converting this
platform-neutral representation into whatever is appropriate on your
current platform before actually using the pathname.  This makes your
setup script portable across operating systems, which of course is one
of the major goals of the Distutils.  In this spirit, all pathnames in
this document are slash-separated (MacOS programmers should keep in
mind that the \emph{absence} of a leading slash indicates a relative
path, the opposite of the MacOS convention with colons).


\subsection{Listing whole packages}
\label{listing-packages}

The \option{packages} option tells the Distutils to process (build,
distribute, install, etc.) all pure Python modules found in each package
mentioned in the \option{packages} list.  In order to do this, of
course, there has to be a correspondence between package names and
directories in the filesystem.  The default correspondence is the most
obvious one, i.e. package \module{distutils} is found in the directory
\file{distutils} relative to the distribution root.  Thus, when you say
\code{packages = ['foo']} in your setup script, you are promising that
the Distutils will find a file \file{foo/\_\_init\_\_.py} (which might
be spelled differently on your system, but you get the idea) relative to
the directory where your setup script lives.  (If you break this
promise, the Distutils will issue a warning but process the broken
package anyways.)

If you use a different convention to lay out your source directory,
that's no problem: you just have to supply the \option{package\_dir}
option to tell the Distutils about your convention.  For example, say
you keep all Python source under \file{lib}, so that modules in the
``root package'' (i.e., not in any package at all) are right in
\file{lib}, modules in the \module{foo} package are in \file{lib/foo},
and so forth.  Then you would put
\begin{verbatim}
package_dir = {'': 'lib'}
\end{verbatim}
in your setup script.  (The keys to this dictionary are package names,
and an empty package name stands for the root package.  The values are
directory names relative to your distribution root.)  In this case, when
you say \code{packages = ['foo']}, you are promising that the file
\file{lib/foo/\_\_init\_\_.py} exists.

Another possible convention is to put the \module{foo} package right in 
\file{lib}, the \module{foo.bar} package in \file{lib/bar}, etc.  This
would be written in the setup script as
\begin{verbatim}
package_dir = {'foo': 'lib'}
\end{verbatim}
A \code{\var{package}: \var{dir}} entry in the \option{package\_dir}
dictionary implicitly applies to all packages below \var{package}, so
the \module{foo.bar} case is automatically handled here.  In this
example, having \code{packages = ['foo', 'foo.bar']} tells the Distutils
to look for \file{lib/\_\_init\_\_.py} and
\file{lib/bar/\_\_init\_\_.py}.  (Keep in mind that although
\option{package\_dir} applies recursively, you must explicitly list all
packages in \option{packages}: the Distutils will \emph{not} recursively
scan your source tree looking for any directory with an
\file{\_\_init\_\_.py} file.)


\subsection{Listing individual modules}
\label{listing-modules}

For a small module distribution, you might prefer to list all modules
rather than listing packages---especially the case of a single module
that goes in the ``root package'' (i.e., no package at all).  This
simplest case was shown in section~\ref{simple-example}; here is a
slightly more involved example:
\begin{verbatim}
py_modules = ['mod1', 'pkg.mod2']
\end{verbatim}
This describes two modules, one of them in the ``root'' package, the
other in the \module{pkg} package.  Again, the default package/directory
layout implies that these two modules can be found in \file{mod1.py} and
\file{pkg/mod2.py}, and that \file{pkg/\_\_init\_\_.py} exists as well.
And again, you can override the package/directory correspondence using
the \option{package\_dir} option.


\subsection{Describing extension modules}
\label{describing-extensions}

Just as writing Python extension modules is a bit more complicated than
writing pure Python modules, describing them to the Distutils is a bit
more complicated.  Unlike pure modules, it's not enough just to list
modules or packages and expect the Distutils to go out and find the
right files; you have to specify the extension name, source file(s), and
any compile/link requirements (include directories, libraries to link
with, etc.).

All of this is done through another keyword argument to
\function{setup()}, the \option{extensions} option.  \option{extensions}
is just a list of \class{Extension} instances, each of which describes a
single extension module.  Suppose your distribution includes a single
extension, called \module{foo} and implemented by \file{foo.c}.  If no
additional instructions to the compiler/linker are needed, describing
this extension is quite simple:
\begin{verbatim}
Extension("foo", ["foo.c"])
\end{verbatim}
The \class{Extension} class can be imported from
\module{distutils.core}, along with \function{setup()}.  Thus, the setup
script for a module distribution that contains only this one extension
and nothing else might be:
\begin{verbatim}
from distutils.core import setup, Extension
setup(name = "foo", version = "1.0",
      ext_modules = [Extension("foo", ["foo.c"])])
\end{verbatim}

The \class{Extension} class (actually, the underlying extension-building
machinery implemented by the \command{built\_ext} command) supports a
great deal of flexibility in describing Python extensions, which is
explained in the following sections.  


\subsubsection{Extension names and packages}

The first argument to the \class{Extension} constructor is always the
name of the extension, including any package names.  For example,
\begin{verbatim}
Extension("foo", ["src/foo1.c", "src/foo2.c"])
\end{verbatim}
describes an extension that lives in the root package, while
\begin{verbatim}
Extension("pkg.foo", ["src/foo1.c", "src/foo2.c"])
\end{verbatim}
describes the same extension in the \module{pkg} package.  The source
files and resulting object code are identical in both cases; the only
difference is where in the filesystem (and therefore where in Python's
namespace hierarchy) the resulting extension lives.

If you have a number of extensions all in the same package (or all under
the same base package), use the \option{ext\_package} keyword argument
to \function{setup()}.  For example,
\begin{verbatim}
setup(...
      ext_package = "pkg",
      ext_modules = [Extension("foo", ["foo.c"]),
                     Extension("subpkg.bar", ["bar.c"])]
     )
\end{verbatim}
will compile \file{foo.c} to the extension \module{pkg.foo}, and
\file{bar.c} to \module{pkg.subpkg.bar}.


\subsubsection{Extension source files}

The second argument to the \class{Extension} constructor is a list of
source files.  Since the Distutils currently only support C/C++
extensions, these are normally C/C++ source files.  (Be sure to use
appropriate extensions to distinguish C++ source files: \file{.cc} and
\file{.cpp} seem to be recognized by both \UNIX{} and Windows compilers.)

However, you can also include SWIG interface (\file{.i}) files in the
list; the \command{build\_ext} command knows how to deal with SWIG
extensions: it will run SWIG on the interface file and compile the
resulting C/C++ file into your extension.

\XXX{SWIG support is rough around the edges and largely untested;
  especially SWIG support of C++ extensions!  Explain in more detail
  here when the interface firms up.}

On some platforms, you can include non-source files that are processed
by the compiler and included in your extension.  Currently, this just
means Windows resource files for Visual C++.  \XXX{get more detail on
  this feature from Thomas Heller!}


\subsubsection{Preprocessor options}

Three optional arguments to \class{Extension} will help if you need to
specify include directories to search or preprocessor macros to
define/undefine: \code{include\_dirs}, \code{define\_macros}, and
\code{undef\_macros}.

For example, if your extension requires header files in the
\file{include} directory under your distribution root, use the
\code{include\_dirs} option:
\begin{verbatim}
Extension("foo", ["foo.c"], include_dirs=["include"])
\end{verbatim}

You can specify absolute directories there; if you know that your
extension will only be built on \UNIX{} systems with X11R6 installed to
\file{/usr}, you can get away with
\begin{verbatim}
Extension("foo", ["foo.c"], include_dirs=["/usr/include/X11"])
\end{verbatim}
You should avoid this sort of non-portable usage if you plan to
distribute your code: it's probably better to write your code to include
(e.g.) \code{<X11/Xlib.h>}.

If you need to include header files from some other Python extension,
you can take advantage of the fact that the Distutils install extension
header files in a consistent way.  For example, the Numerical Python
header files are installed (on a standard \UNIX{} installation) to
\file{/usr/local/include/python1.5/Numerical}.  (The exact location will
differ according to your platform and Python installation.)  Since the
Python include directory---\file{/usr/local/include/python1.5} in this
case---is always included in the search path when building Python
extensions, the best approach is to include (e.g.)
\code{<Numerical/arrayobject.h>}.  If you insist on putting the
\file{Numerical} include directory right into your header search path,
though, you can find that directory using the Distutils
\module{sysconfig} module:
\begin{verbatim}
from distutils.sysconfig import get_python_inc
incdir = os.path.join(get_python_inc(plat_specific=1), "Numerical")
setup(...,
      Extension(..., include_dirs=[incdir]))
\end{verbatim}
Even though this is quite portable---it will work on any Python
installation, regardless of platform---it's probably easier to just
write your C code in the sensible way.

You can define and undefine pre-processor macros with the
\code{define\_macros} and \code{undef\_macros} options.
\code{define\_macros} takes a list of \code{(name, value)} tuples, where
\code{name} is the name of the macro to define (a string) and
\code{value} is its value: either a string or \code{None}.  (Defining a
macro \code{FOO} to \code{None} is the equivalent of a bare
\code{\#define FOO} in your C source: with most compilers, this sets
\code{FOO} to the string \code{1}.)  \code{undef\_macros} is just
a list of macros to undefine.

For example:
\begin{verbatim}
Extension(...,
          define_macros=[('NDEBUG', '1')],
                         ('HAVE_STRFTIME', None),
          undef_macros=['HAVE_FOO', 'HAVE_BAR'])
\end{verbatim}
is the equivalent of having this at the top of every C source file:
\begin{verbatim}
#define NDEBUG 1
#define HAVE_STRFTIME
#undef HAVE_FOO
#undef HAVE_BAR
\end{verbatim}


\subsubsection{Library options}

You can also specify the libraries to link against when building your
extension, and the directories to search for those libraries.  The
\code{libraries} option is a list of libraries to link against,
\code{library\_dirs} is a list of directories to search for libraries at 
link-time, and \code{runtime\_library\_dirs} is a list of directories to 
search for shared (dynamically loaded) libraries at run-time.

For example, if you need to link against libraries known to be in the
standard library search path on target systems
\begin{verbatim}
Extension(...,
          libraries=["gdbm", "readline"])
\end{verbatim}

If you need to link with libraries in a non-standard location, you'll
have to include the location in \code{library\_dirs}:
\begin{verbatim}
Extension(...,
          library_dirs=["/usr/X11R6/lib"],
          libraries=["X11", "Xt"])
\end{verbatim}
(Again, this sort of non-portable construct should be avoided if you
intend to distribute your code.)

\XXX{still undocumented: extra\_objects, extra\_compile\_args,
  extra\_link\_args, export\_symbols---none of which are frequently
  needed, some of which might be completely unnecessary!}


\section{Writing the Setup Configuration File}
\label{setup-config}

Often, it's not possible to write down everything needed to build a
distribution \emph{a priori}: you may need to get some information from
the user, or from the user's system, in order to proceed.  As long as
that information is fairly simple---a list of directories to search for
C header files or libraries, for example---then providing a
configuration file, \file{setup.cfg}, for users to edit is a cheap and
easy way to solicit it.  Configuration files also let you provide
default values for any command option, which the installer can then
override either on the command-line or by editing the config file.

(If you have more advanced needs, such as determining which extensions
to build based on what capabilities are present on the target system,
then you need the Distutils ``auto-configuration'' facility.  This
started to appear in Distutils 0.9 but, as of this writing, isn't mature 
or stable enough yet for real-world use.)

\XXX{should reference description of distutils config files in
  ``Installing'' manual here}

The setup configuration file is a useful middle-ground between the setup
script---which, ideally, would be opaque to installers\footnote{This
  ideal probably won't be achieved until auto-configuration is fully
  supported by the Distutils.}---and the command-line to the setup
script, which is outside of your control and entirely up to the
installer.  In fact, \file{setup.cfg} (and any other Distutils
configuration files present on the target system) are processed after
the contents of the setup script, but before the command-line.  This has 
several useful consequences:
\begin{itemize}
\item installers can override some of what you put in \file{setup.py} by
  editing \file{setup.cfg}
\item you can provide non-standard defaults for options that are not
  easily set in \file{setup.py}
\item installers can override anything in \file{setup.cfg} using the
  command-line options to \file{setup.py}
\end{itemize}

The basic syntax of the configuration file is simple:
\begin{verbatim}
[command]
option=value
...
\end{verbatim}
where \var{command} is one of the Distutils commands (e.g.
\command{build\_py}, \command{install}), and \var{option} is one of the
options that command supports.  Any number of options can be supplied
for each command, and any number of command sections can be included in
the file.  Blank lines are ignored, as are comments (from a
\character{\#} character to end-of-line).  Long option values can be
split across multiple lines simply by indenting the continuation lines.

You can find out the list of options supported by a particular command
with the universal \longprogramopt{help} option, e.g.
\begin{verbatim}
> python setup.py --help build_ext
[...]
Options for 'build_ext' command:
  --build-lib (-b)     directory for compiled extension modules
  --build-temp (-t)    directory for temporary files (build by-products)
  --inplace (-i)       ignore build-lib and put compiled extensions into the
                       source directory alongside your pure Python modules
  --include-dirs (-I)  list of directories to search for header files
  --define (-D)        C preprocessor macros to define
  --undef (-U)         C preprocessor macros to undefine
[...]
\end{verbatim}
Or consult section \ref{reference} of this document (the command
reference).

Note that an option spelled \longprogramopt{foo-bar} on the command-line 
is spelled \option{foo\_bar} in configuration files.

For example, say you want your extensions to be built
``in-place''---that is, you have an extension \module{pkg.ext}, and you
want the compiled extension file (\file{ext.so} on \UNIX, say) to be put
in the same source directory as your pure Python modules
\module{pkg.mod1} and \module{pkg.mod2}.  You can always use the
\longprogramopt{inplace} option on the command-line to ensure this:
\begin{verbatim}
python setup.py build_ext --inplace
\end{verbatim}
But this requires that you always specify the \command{build\_ext}
command explicitly, and remember to provide \longprogramopt{inplace}.
An easier way is to ``set and forget'' this option, by encoding it in
\file{setup.cfg}, the configuration file for this distribution:
\begin{verbatim}
[build_ext]
inplace=1
\end{verbatim}
This will affect all builds of this module distribution, whether or not
you explcitly specify \command{build\_ext}.  If you include
\file{setup.cfg} in your source distribution, it will also affect
end-user builds---which is probably a bad idea for this option, since
always building extensions in-place would break installation of the
module distribution.  In certain peculiar cases, though, modules are
built right in their installation directory, so this is conceivably a
useful ability.  (Distributing extensions that expect to be built in
their installation directory is almost always a bad idea, though.)

Another example: certain commands take a lot of options that don't
change from run-to-run; for example, \command{bdist\_rpm} needs to know
everything required to generate a ``spec'' file for creating an RPM
distribution.  Some of this information comes from the setup script, and
some is automatically generated by the Distutils (such as the list of
files installed).  But some of it has to be supplied as options to
\command{bdist\_rpm}, which would be very tedious to do on the
command-line for every run.  Hence, here is a snippet from the
Distutils' own \file{setup.cfg}:
\begin{verbatim}
[bdist_rpm]
release = 1
packager = Greg Ward <gward@python.net>
doc_files = CHANGES.txt
            README.txt
            USAGE.txt
            doc/
            examples/
\end{verbatim}
Note that the \option{doc\_files} option is simply a
whitespace-separated string split across multiple lines for readability.


\section{Creating a Source Distribution}
\label{source-dist}

As shown in section~\ref{simple-example}, you use the
\command{sdist} command to create a source distribution.  In the
simplest case,
\begin{verbatim}
python setup.py sdist
\end{verbatim}
(assuming you haven't specified any \command{sdist} options in the setup
script or config file), \command{sdist} creates the archive of the
default format for the current platform.  The default format is gzip'ed
tar file (\file{.tar.gz}) on \UNIX, and ZIP file on Windows.
\XXX{no MacOS support here}

You can specify as many formats as you like using the
\longprogramopt{formats} option, for example:
\begin{verbatim}
python setup.py sdist --formats=gztar,zip
\end{verbatim}
to create a gzipped tarball and a zip file.  The available formats are:
\begin{tableiii}{l|l|c}{code}%
  {Format}{Description}{Notes}
  \lineiii{zip}{zip file (\file{.zip})}{(1),(3)}
  \lineiii{gztar}{gzip'ed tar file (\file{.tar.gz})}{(2),(4)}
  \lineiii{bztar}{bzip2'ed tar file (\file{.tar.gz})}{(4)}
  \lineiii{ztar}{compressed tar file (\file{.tar.Z})}{(4)}
  \lineiii{tar}{tar file (\file{.tar})}{(4)}
\end{tableiii}

\noindent Notes:
\begin{description}
\item[(1)] default on Windows
\item[(2)] default on \UNIX
\item[(3)] requires either external \program{zip} utility or
  \module{zipfile} module (not part of the standard Python library)
\item[(4)] requires external utilities: \program{tar} and possibly one
  of \program{gzip}, \program{bzip2}, or \program{compress}
\end{description}



\subsection{Specifying the files to distribute}
\label{manifest}

If you don't supply an explicit list of files (or instructions on how to
generate one), the \command{sdist} command puts a minimal default set
into the source distribution:
\begin{itemize}
\item all Python source files implied by the \option{py\_modules} and
  \option{packages} options
\item all C source files mentioned in the \option{ext\_modules} or
  \option{libraries} options (\XXX{getting C library sources currently
    broken -- no get\_source\_files() method in build\_clib.py!})
\item anything that looks like a test script: \file{test/test*.py}
  (currently, the Distutils don't do anything with test scripts except
  include them in source distributions, but in the future there will be
  a standard for testing Python module distributions)
\item \file{README.txt} (or \file{README}), \file{setup.py} (or whatever 
  you called your setup script), and \file{setup.cfg}
\end{itemize}
Sometimes this is enough, but usually you will want to specify
additional files to distribute.  The typical way to do this is to write
a \emph{manifest template}, called \file{MANIFEST.in} by default.  The
manifest template is just a list of instructions for how to generate
your manifest file, \file{MANIFEST}, which is the exact list of files to
include in your source distribution.  The \command{sdist} command
processes this template and generates a manifest based on its
instructions and what it finds in the filesystem.

If you prefer to roll your own manifest file, the format is simple: one
filename per line, regular files (or symlinks to them) only.  If you do
supply your own \file{MANIFEST}, you must specify everything: the
default set of files described above does not apply in this case.

The manifest template has one command per line, where each command
specifies a set of files to include or exclude from the source
distribution.  For an example, again we turn to the Distutils' own
manifest template:
\begin{verbatim}
include *.txt
recursive-include examples *.txt *.py
prune examples/sample?/build
\end{verbatim}
The meanings should be fairly clear: include all files in the
distribution root matching \code{*.txt}, all files anywhere under the
\file{examples} directory matching \code{*.txt} or \code{*.py}, and
exclude all directories matching \code{examples/sample?/build}.  All of
this is done \emph{after} the standard include set, so you can exclude
files from the standard set with explicit instructions in the manifest
template.  (Or, you can use the \longprogramopt{no-defaults} option to
disable the standard set entirely.)  There are several other commands
available in the manifest template mini-language; see
section~\ref{sdist-cmd}.

The order of commands in the manifest template matters: initially, we
have the list of default files as described above, and each command in
the template adds to or removes from that list of files.  Once we have
fully processed the manifest template, we remove files that should not
be included in the source distribution:
\begin{itemize}
\item all files in the Distutils ``build'' tree (default \file{build/})
\item all files in directories named \file{RCS} or \file{CVS}
\end{itemize}
Now we have our complete list of files, which is written to the manifest
for future reference, and then used to build the source distribution
archive(s).

You can disable the default set of included files with the
\longprogramopt{no-defaults} option, and you can disable the standard
exclude set with \longprogramopt{no-prune}.

Following the Distutils' own manifest template, let's trace how the
\command{sdist} command builds the list of files to include in the
Distutils source distribution:
\begin{enumerate}
\item include all Python source files in the \file{distutils} and
  \file{distutils/command} subdirectories (because packages
  corresponding to those two directories were mentioned in the
  \option{packages} option in the setup script---see
  section~\ref{setup-script})
\item include \file{README.txt}, \file{setup.py}, and \file{setup.cfg}
  (standard files)
\item include \file{test/test*.py} (standard files)
\item include \file{*.txt} in the distribution root (this will find
  \file{README.txt} a second time, but such redundancies are weeded out
  later)
\item include anything matching \file{*.txt} or \file{*.py} in the
  sub-tree under \file{examples},
\item exclude all files in the sub-trees starting at directories
  matching \file{examples/sample?/build}---this may exclude files
  included by the previous two steps, so it's important that the
  \code{prune} command in the manifest template comes after the
  \code{recursive-include} command
\item exclude the entire \file{build} tree, and any \file{RCS} or
  \file{CVS} directories
\end{enumerate}
Just like in the setup script, file and directory names in the manifest
template should always be slash-separated; the Distutils will take care
of converting them to the standard representation on your platform.
That way, the manifest template is portable across operating systems.


\subsection{Manifest-related options}
\label{manifest-options}

The normal course of operations for the \command{sdist} command is as
follows:
\begin{itemize}
\item if the manifest file, \file{MANIFEST} doesn't exist, read
  \file{MANIFEST.in} and create the manifest
\item if neither \file{MANIFEST} nor \file{MANIFEST.in} exist, create a
  manifest with just the default file set\footnote{In versions of the
    Distutils up to and including 0.9.2 (Python 2.0b1), this feature was
    broken; use the \programopt{-f} (\longprogramopt{force-manifest})
    option to work around the bug.}
\item if either \file{MANIFEST.in} or the setup script (\file{setup.py})
  are more recent than \file{MANIFEST}, recreate \file{MANIFEST} by
  reading \file{MANIFEST.in}
\item use the list of files now in \file{MANIFEST} (either just
  generated or read in) to create the source distribution archive(s)
\end{itemize}
There are a couple of options that modify this behaviour.  First, use
the \longprogramopt{no-defaults} and \longprogramopt{no-prune} to
disable the standard ``include'' and ``exclude'' sets.\footnote{Note
  that if you have no manifest template, no manifest, and use the
  \longprogramopt{no-defaults}, you will get an empty manifest.  Another
  bug in Distutils 0.9.2 and earlier causes an uncaught exception in
  this case.  The workaround is: Don't Do That.}

Second, you might want to force the manifest to be regenerated---for
example, if you have added or removed files or directories that match an
existing pattern in the manifest template, you should regenerate the
manifest:
\begin{verbatim}
python setup.py sdist --force-manifest
\end{verbatim}

Or, you might just want to (re)generate the manifest, but not create a
source distribution:
\begin{verbatim}
python setup.py sdist --manifest-only
\end{verbatim}
\longprogramopt{manifest-only} implies \longprogramopt{force-manifest}.
\programopt{-o} is a shortcut for \longprogramopt{manifest-only}, and
\programopt{-f} for \longprogramopt{force-manifest}.


\section{Creating Built Distributions}
\label{built-dist}

A ``built distribution'' is what you're probably used to thinking of
either as a ``binary package'' or an ``installer'' (depending on your
background).  It's not necessarily binary, though, because it might
contain only Python source code and/or byte-code; and we don't call it a
package, because that word is already spoken for in Python.  (And
``installer'' is a term specific to the Windows world.  \XXX{do Mac
  people use it?})

A built distribution is how you make life as easy as possible for
installers of your module distribution: for users of RPM-based Linux
systems, it's a binary RPM; for Windows users, it's an executable
installer; for Debian-based Linux users, it's a Debian package; and so
forth.  Obviously, no one person will be able to create built
distributions for every platform under the sun, so the Distutils are
designed to enable module developers to concentrate on their
specialty---writing code and creating source distributions---while an
intermediary species of \emph{packager} springs up to turn source
distributions into built distributions for as many platforms as there
are packagers.

Of course, the module developer could be his own packager; or the
packager could be a volunteer ``out there'' somewhere who has access to
a platform which the original developer does not; or it could be
software periodically grabbing new source distributions and turning them
into built distributions for as many platforms as the software has
access to.  Regardless of the nature of the beast, a packager uses the
setup script and the \command{bdist} command family to generate built
distributions.

As a simple example, if I run the following command in the Distutils
source tree:
\begin{verbatim}
python setup.py bdist
\end{verbatim}
then the Distutils builds my module distribution (the Distutils itself
in this case), does a ``fake'' installation (also in the \file{build}
directory), and creates the default type of built distribution for my
platform.  The default format for built distributions is a ``dumb'' tar
file on \UNIX, and an simple executable installer on Windows.  (That tar
file is considered ``dumb'' because it has to be unpacked in a specific
location to work.)

Thus, the above command on a \UNIX{} system creates
\file{Distutils-0.9.1.\filevar{plat}.tar.gz}; unpacking this tarball
from the right place installs the Distutils just as though you had
downloaded the source distribution and run \code{python setup.py
  install}.  (The ``right place'' is either the root of the filesystem or 
Python's \filevar{prefix} directory, depending on the options given to
the \command{bdist\_dumb} command; the default is to make dumb
distributions relative to \filevar{prefix}.)  

Obviously, for pure Python distributions, this isn't a huge win---but
for non-pure distributions, which include extensions that would need to
be compiled, it can mean the difference between someone being able to
use your extensions or not.  And creating ``smart'' built distributions,
such as an RPM package or an executable installer for Windows, is a big
win for users even if your distribution doesn't include any extensions.

The \command{bdist} command has a \longprogramopt{formats} option,
similar to the \command{sdist} command, which you can use to select the
types of built distribution to generate: for example,
\begin{verbatim}
python setup.py bdist --format=zip
\end{verbatim}
would, when run on a \UNIX{} system, create
\file{Distutils-0.8.\filevar{plat}.zip}---again, this archive would be
unpacked from the root directory to install the Distutils.

The available formats for built distributions are:
\begin{tableiii}{l|l|c}{code}%
  {Format}{Description}{Notes}
  \lineiii{gztar}{gzipped tar file (\file{.tar.gz})}{(1),(3)}
  \lineiii{ztar}{compressed tar file (\file{.tar.Z})}{(3)}
  \lineiii{tar}{tar file (\file{.tar})}{(3)}
  \lineiii{zip}{zip file (\file{.zip})}{(4)}
  \lineiii{rpm}{RPM}{(5)}
  \lineiii{srpm}{source RPM}{(5) \XXX{to do!}}
  \lineiii{wininst}{self-extracting ZIP file for Windows}{(2),(6)}
  %\lineiii{wise}{Wise installer for Windows}{(3)}
\end{tableiii}

\noindent Notes:
\begin{description}
\item[(1)] default on \UNIX
\item[(2)] default on Windows \XXX{to-do!}
\item[(3)] requires external utilities: \program{tar} and possibly one
  of \program{gzip}, \program{bzip2}, or \program{compress}
\item[(4)] requires either external \program{zip} utility or
  \module{zipfile} module (not part of the standard Python library)
\item[(5)] requires external \program{rpm} utility, version 3.0.4 or
  better (use \code{rpm --version} to find out which version you have)
\item[(6)] \XXX{requirements for \command{bdist\_wininst}?}
%\item[(3)] not implemented yet
\end{description}

You don't have to use the \command{bdist} command with the
\longprogramopt{formats} option; you can also use the command that
directly implements the format you're interested in.  Some of these
\command{bdist} ``sub-commands'' actually generate several similar
formats; for instance, the \command{bdist\_dumb} command generates all
the ``dumb'' archive formats (\code{tar}, \code{ztar}, \code{gztar}, and
\code{zip}), and \command{bdist\_rpm} generates both binary and source
RPMs.  The \command{bdist} sub-commands, and the formats generated by
each, are:
\begin{tableii}{l|l}{command}%
  {Command}{Formats}
  \lineii{bdist\_dumb}{tar, ztar, gztar, zip}
  \lineii{bdist\_rpm}{rpm, srpm}
  \lineii{bdist\_wininst}{wininst}
  %\lineii{bdist\_wise}{wise}
\end{tableii}

The following sections give details on the individual \command{bdist\_*}
commands.


\subsection{Creating dumb built distributions}
\label{creating-dumb}

\XXX{Need to document absolute vs. prefix-relative packages here, but
  first I have to implement it!}


\subsection{Creating RPM packages}
\label{creating-rpms}

The RPM format is used by many of popular Linux distributions, including
Red Hat, SuSE, and Mandrake.  If one of these (or any of the other
RPM-based Linux distributions) is your usual environment, creating RPM
packages for other users of that same distribution is trivial.
Depending on the complexity of your module distribution and differences
between Linux distributions, you may also be able to create RPMs that
work on different RPM-based distributions.

The usual way to create an RPM of your module distribution is to run the 
\command{bdist\_rpm} command:
\begin{verbatim}
python setup.py bdist_rpm
\end{verbatim}
or the \command{bdist} command with the \longprogramopt{format} option:
\begin{verbatim}
python setup.py bdist --formats=rpm
\end{verbatim}
The former allows you to specify RPM-specific options; the latter allows 
you to easily specify multiple formats in one run.  If you need to do
both, you can explicitly specify multiple \command{bdist\_*} commands
and their options:
\begin{verbatim}
python setup.py bdist_rpm --packager="John Doe <jdoe@python.net>" \
                bdist_wininst --target_version="2.0"
\end{verbatim}

Creating RPM packages is driven by a \file{.spec} file, much as using
the Distutils is driven by the setup script.  To make your life easier,
the \command{bdist\_rpm} command normally creates a \file{.spec} file
based on the information you supply in the setup script, on the command
line, and in any Distutils configuration files.  Various options and
section in the \file{.spec} file are derived from options in the setup
script as follows:
\begin{tableii}{l|l}{textrm}%
  {RPM \file{.spec} file option or section}{Distutils setup script option}
  \lineii{Name}{\option{name}}
  \lineii{Summary (in preamble)}{\option{description}}
  \lineii{Version}{\option{version}}
  \lineii{Vendor}{\option{author} and \option{author\_email}, or \\&
                  \option{maintainer} and \option{maintainer\_email}}
  \lineii{Copyright}{\option{licence}}
  \lineii{Url}{\option{url}}
  \lineii{\%description (section)}{\option{long\_description}}
\end{tableii}

Additionally, there many options in \file{.spec} files that don't have
corresponding options in the setup script.  Most of these are handled
through options to the \command{bdist\_rpm} command as follows:
\begin{tableiii}{l|l|l}{textrm}%
  {RPM \file{.spec} file option or section}%
  {\command{bdist\_rpm} option}%
  {default value}
  \lineiii{Release}{\option{release}}{``1''}
  \lineiii{Group}{\option{group}}{``Development/Libraries''}
  \lineiii{Vendor}{\option{vendor}}{(see above)}
  \lineiii{Packager}{packager}{(none)}
  \lineiii{Provides}{provides}{(none)}
  \lineiii{Requires}{requires}{(none)}
  \lineiii{Conflicts}{conflicts}{(none)}
  \lineiii{Obsoletes}{obsoletes}{(none)}
  \lineiii{Distribution}{\option{distribution\_name}}{(none)}
  \lineiii{BuildRequires}{\option{build\_requires}}{(none)}
  \lineiii{Icon}{\option{icon}}{(none)}
\end{tableiii}
Obviously, supplying even a few of these options on the command-line
would be tedious and error-prone, so it's usually best to put them in
the setup configuration file, \file{setup.cfg}---see
section~\ref{setup-config}.  If you distribute or package many Python
module distributions, you might want to put options that apply to all of
them in your personal Distutils configuration file
(\file{\textasciitilde/.pydistutils.cfg}).

There are three steps to building a binary RPM package, all of which are 
handled automatically by the Distutils:
\begin{enumerate}
\item create a \file{.spec} file, which describes the package (analogous 
  to the Distutils setup script; in fact, much of the information in the 
  setup script winds up in the \file{.spec} file)
\item create the source RPM
\item create the ``binary'' RPM (which may or may not contain binary
  code, depending on whether your module distribution contains Python
  extensions)
\end{enumerate}
Normally, RPM bundles the last two steps together; when you use the
Distutils, all three steps are typically bundled together.

If you wish, you can separate these three steps.  You can use the
\longprogramopt{spec-only} option to make \command{bdist\_rpm} just
create the \file{.spec} file and exit; in this case, the \file{.spec}
file will be written to the ``distribution directory''---normally
\file{dist/}, but customizable with the \longprogramopt{dist-dir}
option.  (Normally, the \file{.spec} file winds up deep in the ``build
tree,'' in a temporary directory created by \command{bdist\_rpm}.)

\XXX{this isn't implemented yet---is it needed?!}
You can also specify a custom \file{.spec} file with the
\longprogramopt{spec-file} option; used in conjunctin with
\longprogramopt{spec-only}, this gives you an opportunity to customize
the \file{.spec} file manually:
\begin{verbatim}
> python setup.py bdist_rpm --spec-only
# ...edit dist/FooBar-1.0.spec
> python setup.py bdist_rpm --spec-file=dist/FooBar-1.0.spec
\end{verbatim}
(Although a better way to do this is probably to override the standard
\command{bdist\_rpm} command with one that writes whatever else you want
to the \file{.spec} file; see section~\ref{extending} for information on
extending the Distutils.)


\subsection{Creating Windows installers}
\label{creating-wininst}



\section{Examples}
\label{examples}


\subsection{Pure Python distribution (by module)}
\label{pure-mod}


\subsection{Pure Python distribution (by package)}
\label{pure-pkg}


\subsection{Single extension module}
\label{single-ext}


\subsection{Multiple extension modules}
\label{multiple-ext}


\subsection{Putting it all together}



\section{Extending the Distutils}
\label{extending}


\subsection{Extending existing commands}
\label{extend-existing}


\subsection{Writing new commands}
\label{new-commands}



\section{Reference}
\label{reference}


\subsection{Building modules: the \protect\command{build} command family}
\label{build-cmds}

\subsubsection{\protect\command{build}}
\label{build-cmd}

\subsubsection{\protect\command{build\_py}}
\label{build-py-cmd}

\subsubsection{\protect\command{build\_ext}}
\label{build-ext-cmd}

\subsubsection{\protect\command{build\_clib}}
\label{build-clib-cmd}


\subsection{Installing modules: the \protect\command{install} command family}
\label{install-cmd}

The install command ensures that the build commands have been run and then
runs the subcommands \command{install\_lib},
\command{install\_data} and
\command{install\_scripts}.

\subsubsection{\protect\command{install\_lib}}
\label{install-lib-cmd}

\subsubsection{\protect\command{install\_data}}
\label{install-data-cmd}
This command installs all data files provided with the distribution.

\subsubsection{\protect\command{install\_scripts}}
\label{install-scripts-cmd}
This command installs all (Python) scripts in the distribution.


\subsection{Cleaning up: the \protect\command{clean} command}
\label{clean-cmd}


\subsection{Creating a source distribution: the
            \protect\command{sdist} command}
\label{sdist-cmd}


\XXX{fragment moved down from above: needs context!}

The manifest template commands are:
\begin{tableii}{ll}{command}{Command}{Description}
  \lineii{include \var{pat1} \var{pat2} ... }
    {include all files matching any of the listed patterns}
  \lineii{exclude \var{pat1} \var{pat2} ... }
    {exclude all files matching any of the listed patterns}
  \lineii{recursive-include \var{dir} \var{pat1} \var{pat2} ... }
    {include all files under \var{dir} matching any of the listed patterns}
  \lineii{recursive-exclude \var{dir} \var{pat1} \var{pat2} ...}
    {exclude all files under \var{dir} matching any of the listed patterns}
  \lineii{global-include \var{pat1} \var{pat2} ...}
    {include all files anywhere in the source tree matching\\&
     any of the listed patterns}
  \lineii{global-exclude \var{pat1} \var{pat2} ...}
    {exclude all files anywhere in the source tree matching\\&
     any of the listed patterns}
  \lineii{prune \var{dir}}{exclude all files under \var{dir}}
  \lineii{graft \var{dir}}{include all files under \var{dir}}
\end{tableii}
The patterns here are \UNIX-style ``glob'' patterns: \code{*} matches any
sequence of regular filename characters, \code{?} matches any single
regular filename character, and \code{[\var{range}]} matches any of the
characters in \var{range} (e.g., \code{a-z}, \code{a-zA-Z},
\code{a-f0-9\_.}).  The definition of ``regular filename character'' is
platform-specific: on \UNIX{} it is anything except slash; on Windows
anything except backslash or colon; on MacOS anything except colon.

\XXX{Windows and MacOS support not there yet}


\subsection{Creating a ``built'' distribution: the
  \protect\command{bdist} command family}
\label{bdist-cmds}


\subsubsection{\protect\command{blib}}

\subsubsection{\protect\command{blib\_dumb}}

\subsubsection{\protect\command{blib\_rpm}}

\subsubsection{\protect\command{blib\_wise}}








\end{document}
