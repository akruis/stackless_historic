\chapter{Data model\label{datamodel}}


\section{Objects, values and types\label{objects}}

\dfn{Objects} are Python's abstraction for data.  All data in a Python
program is represented by objects or by relations between objects.
(In a sense, and in conformance to Von Neumann's model of a
``stored program computer,'' code is also represented by objects.)
\index{object}
\index{data}

Every object has an identity, a type and a value.  An object's
\emph{identity} never changes once it has been created; you may think
of it as the object's address in memory.  The `\keyword{is}' operator
compares the identity of two objects; the
\function{id()}\bifuncindex{id} function returns an integer
representing its identity (currently implemented as its address).
An object's \dfn{type} is
also unchangeable.\footnote{Since Python 2.2, a gradual merging of
types and classes has been started that makes this and a few other
assertions made in this manual not 100\% accurate and complete:
for example, it \emph{is} now possible in some cases to change an
object's type, under certain controlled conditions.  Until this manual
undergoes extensive revision, it must now be taken as authoritative
only regarding ``classic classes'', that are still the default, for
compatibility purposes, in Python 2.2 and 2.3.}
An object's type determines the operations that the object
supports (e.g., ``does it have a length?'') and also defines the
possible values for objects of that type.  The
\function{type()}\bifuncindex{type} function returns an object's type
(which is an object itself).  The \emph{value} of some
objects can change.  Objects whose value can change are said to be
\emph{mutable}; objects whose value is unchangeable once they are
created are called \emph{immutable}.
(The value of an immutable container object that contains a reference
to a mutable object can change when the latter's value is changed;
however the container is still considered immutable, because the
collection of objects it contains cannot be changed.  So, immutability
is not strictly the same as having an unchangeable value, it is more
subtle.)
An object's mutability is determined by its type; for instance,
numbers, strings and tuples are immutable, while dictionaries and
lists are mutable.
\index{identity of an object}
\index{value of an object}
\index{type of an object}
\index{mutable object}
\index{immutable object}

Objects are never explicitly destroyed; however, when they become
unreachable they may be garbage-collected.  An implementation is
allowed to postpone garbage collection or omit it altogether --- it is
a matter of implementation quality how garbage collection is
implemented, as long as no objects are collected that are still
reachable.  (Implementation note: the current implementation uses a
reference-counting scheme with (optional) delayed detection of
cyclically linked garbage, which collects most objects as soon as they
become unreachable, but is not guaranteed to collect garbage
containing circular references.  See the
\citetitle[../lib/module-gc.html]{Python Library Reference} for
information on controlling the collection of cyclic garbage.)
\index{garbage collection}
\index{reference counting}
\index{unreachable object}

Note that the use of the implementation's tracing or debugging
facilities may keep objects alive that would normally be collectable.
Also note that catching an exception with a
`\keyword{try}...\keyword{except}' statement may keep objects alive.

Some objects contain references to ``external'' resources such as open
files or windows.  It is understood that these resources are freed
when the object is garbage-collected, but since garbage collection is
not guaranteed to happen, such objects also provide an explicit way to
release the external resource, usually a \method{close()} method.
Programs are strongly recommended to explicitly close such
objects.  The `\keyword{try}...\keyword{finally}' statement provides
a convenient way to do this.

Some objects contain references to other objects; these are called
\emph{containers}.  Examples of containers are tuples, lists and
dictionaries.  The references are part of a container's value.  In
most cases, when we talk about the value of a container, we imply the
values, not the identities of the contained objects; however, when we
talk about the mutability of a container, only the identities of
the immediately contained objects are implied.  So, if an immutable
container (like a tuple)
contains a reference to a mutable object, its value changes
if that mutable object is changed.
\index{container}

Types affect almost all aspects of object behavior.  Even the importance
of object identity is affected in some sense: for immutable types,
operations that compute new values may actually return a reference to
any existing object with the same type and value, while for mutable
objects this is not allowed.  E.g., after
\samp{a = 1; b = 1},
\code{a} and \code{b} may or may not refer to the same object with the
value one, depending on the implementation, but after
\samp{c = []; d = []}, \code{c} and \code{d}
are guaranteed to refer to two different, unique, newly created empty
lists.
(Note that \samp{c = d = []} assigns the same object to both
\code{c} and \code{d}.)


\section{The standard type hierarchy\label{types}}

Below is a list of the types that are built into Python.  Extension
modules (written in C, Java, or other languages, depending on
the implementation) can define additional types.  Future versions of
Python may add types to the type hierarchy (e.g., rational
numbers, efficiently stored arrays of integers, etc.).
\index{type}
\indexii{data}{type}
\indexii{type}{hierarchy}
\indexii{extension}{module}
\indexii{C}{language}

Some of the type descriptions below contain a paragraph listing
`special attributes.'  These are attributes that provide access to the
implementation and are not intended for general use.  Their definition
may change in the future.
\index{attribute}
\indexii{special}{attribute}
\indexiii{generic}{special}{attribute}

\begin{description}

\item[None]
This type has a single value.  There is a single object with this value.
This object is accessed through the built-in name \code{None}.
It is used to signify the absence of a value in many situations, e.g.,
it is returned from functions that don't explicitly return anything.
Its truth value is false.
\obindex{None}

\item[NotImplemented]
This type has a single value.  There is a single object with this value.
This object is accessed through the built-in name \code{NotImplemented}.
Numeric methods and rich comparison methods may return this value if
they do not implement the operation for the operands provided.  (The
interpreter will then try the reflected operation, or some other
fallback, depending on the operator.)  Its truth value is true.
\obindex{NotImplemented}

\item[Ellipsis]
This type has a single value.  There is a single object with this value.
This object is accessed through the built-in name \code{Ellipsis}.
It is used to indicate the presence of the \samp{...} syntax in a
slice.  Its truth value is true.
\obindex{Ellipsis}

\item[Numbers]
These are created by numeric literals and returned as results by
arithmetic operators and arithmetic built-in functions.  Numeric
objects are immutable; once created their value never changes.  Python
numbers are of course strongly related to mathematical numbers, but
subject to the limitations of numerical representation in computers.
\obindex{numeric}

Python distinguishes between integers, floating point numbers, and
complex numbers:

\begin{description}
\item[Integers]
These represent elements from the mathematical set of whole numbers.
\obindex{integer}

There are three types of integers:

\begin{description}

\item[Plain integers]
These represent numbers in the range -2147483648 through 2147483647.
(The range may be larger on machines with a larger natural word
size, but not smaller.)
When the result of an operation would fall outside this range, the
result is normally returned as a long integer (in some cases, the
exception \exception{OverflowError} is raised instead).
For the purpose of shift and mask operations, integers are assumed to
have a binary, 2's complement notation using 32 or more bits, and
hiding no bits from the user (i.e., all 4294967296 different bit
patterns correspond to different values).
\obindex{plain integer}
\withsubitem{(built-in exception)}{\ttindex{OverflowError}}

\item[Long integers]
These represent numbers in an unlimited range, subject to available
(virtual) memory only.  For the purpose of shift and mask operations,
a binary representation is assumed, and negative numbers are
represented in a variant of 2's complement which gives the illusion of
an infinite string of sign bits extending to the left.
\obindex{long integer}

\item[Booleans]
These represent the truth values False and True.  The two objects
representing the values False and True are the only Boolean objects.
The Boolean type is a subtype of plain integers, and Boolean values
behave like the values 0 and 1, respectively, in almost all contexts,
the exception being that when converted to a string, the strings
\code{"False"} or \code{"True"} are returned, respectively.
\obindex{Boolean}
\ttindex{False}
\ttindex{True}

\end{description} % Integers

The rules for integer representation are intended to give the most
meaningful interpretation of shift and mask operations involving
negative integers and the least surprises when switching between the
plain and long integer domains.  Any operation except left shift,
if it yields a result in the plain integer domain without causing
overflow, will yield the same result in the long integer domain or
when using mixed operands.
\indexii{integer}{representation}

\item[Floating point numbers]
These represent machine-level double precision floating point numbers.  
You are at the mercy of the underlying machine architecture (and
C or Java implementation) for the accepted range and handling of overflow.
Python does not support single-precision floating point numbers; the
savings in processor and memory usage that are usually the reason for using
these is dwarfed by the overhead of using objects in Python, so there
is no reason to complicate the language with two kinds of floating
point numbers.
\obindex{floating point}
\indexii{floating point}{number}
\indexii{C}{language}
\indexii{Java}{language}

\item[Complex numbers]
These represent complex numbers as a pair of machine-level double
precision floating point numbers.  The same caveats apply as for
floating point numbers.  The real and imaginary parts of a complex
number \code{z} can be retrieved through the read-only attributes
\code{z.real} and \code{z.imag}.
\obindex{complex}
\indexii{complex}{number}

\end{description} % Numbers


\item[Sequences]
These represent finite ordered sets indexed by non-negative numbers.
The built-in function \function{len()}\bifuncindex{len} returns the
number of items of a sequence.
When the length of a sequence is \var{n}, the
index set contains the numbers 0, 1, \ldots, \var{n}-1.  Item
\var{i} of sequence \var{a} is selected by \code{\var{a}[\var{i}]}.
\obindex{sequence}
\index{index operation}
\index{item selection}
\index{subscription}

Sequences also support slicing: \code{\var{a}[\var{i}:\var{j}]}
selects all items with index \var{k} such that \var{i} \code{<=}
\var{k} \code{<} \var{j}.  When used as an expression, a slice is a
sequence of the same type.  This implies that the index set is
renumbered so that it starts at 0.
\index{slicing}

Some sequences also support ``extended slicing'' with a third ``step''
parameter: \code{\var{a}[\var{i}:\var{j}:\var{k}]} selects all items
of \var{a} with index \var{x} where \code{\var{x} = \var{i} +
\var{n}*\var{k}}, \var{n} \code{>=} \code{0} and \var{i} \code{<=}
\var{x} \code{<} \var{j}.
\index{extended slicing}

Sequences are distinguished according to their mutability:

\begin{description}

\item[Immutable sequences]
An object of an immutable sequence type cannot change once it is
created.  (If the object contains references to other objects,
these other objects may be mutable and may be changed; however,
the collection of objects directly referenced by an immutable object
cannot change.)
\obindex{immutable sequence}
\obindex{immutable}

The following types are immutable sequences:

\begin{description}

\item[Strings]
The items of a string are characters.  There is no separate
character type; a character is represented by a string of one item.
Characters represent (at least) 8-bit bytes.  The built-in
functions \function{chr()}\bifuncindex{chr} and
\function{ord()}\bifuncindex{ord} convert between characters and
nonnegative integers representing the byte values.  Bytes with the
values 0-127 usually represent the corresponding \ASCII{} values, but
the interpretation of values is up to the program.  The string
data type is also used to represent arrays of bytes, e.g., to hold data
read from a file.
\obindex{string}
\index{character}
\index{byte}
\index{ASCII@\ASCII}

(On systems whose native character set is not \ASCII, strings may use
EBCDIC in their internal representation, provided the functions
\function{chr()} and \function{ord()} implement a mapping between \ASCII{} and
EBCDIC, and string comparison preserves the \ASCII{} order.
Or perhaps someone can propose a better rule?)
\index{ASCII@\ASCII}
\index{EBCDIC}
\index{character set}
\indexii{string}{comparison}
\bifuncindex{chr}
\bifuncindex{ord}

\item[Unicode]
The items of a Unicode object are Unicode code units.  A Unicode code
unit is represented by a Unicode object of one item and can hold
either a 16-bit or 32-bit value representing a Unicode ordinal (the
maximum value for the ordinal is given in \code{sys.maxunicode}, and
depends on how Python is configured at compile time).  Surrogate pairs
may be present in the Unicode object, and will be reported as two
separate items.  The built-in functions
\function{unichr()}\bifuncindex{unichr} and
\function{ord()}\bifuncindex{ord} convert between code units and
nonnegative integers representing the Unicode ordinals as defined in
the Unicode Standard 3.0. Conversion from and to other encodings are
possible through the Unicode method \method{encode} and the built-in
function \function{unicode()}.\bifuncindex{unicode}
\obindex{unicode}
\index{character}
\index{integer}
\index{Unicode}

\item[Tuples]
The items of a tuple are arbitrary Python objects.
Tuples of two or more items are formed by comma-separated lists
of expressions.  A tuple of one item (a `singleton') can be formed
by affixing a comma to an expression (an expression by itself does
not create a tuple, since parentheses must be usable for grouping of
expressions).  An empty tuple can be formed by an empty pair of
parentheses.
\obindex{tuple}
\indexii{singleton}{tuple}
\indexii{empty}{tuple}

\end{description} % Immutable sequences

\item[Mutable sequences]
Mutable sequences can be changed after they are created.  The
subscription and slicing notations can be used as the target of
assignment and \keyword{del} (delete) statements.
\obindex{mutable sequence}
\obindex{mutable}
\indexii{assignment}{statement}
\index{delete}
\stindex{del}
\index{subscription}
\index{slicing}

There is currently a single intrinsic mutable sequence type:

\begin{description}

\item[Lists]
The items of a list are arbitrary Python objects.  Lists are formed
by placing a comma-separated list of expressions in square brackets.
(Note that there are no special cases needed to form lists of length 0
or 1.)
\obindex{list}

\end{description} % Mutable sequences

The extension module \module{array}\refstmodindex{array} provides an
additional example of a mutable sequence type.


\end{description} % Sequences

\item[Mappings]
These represent finite sets of objects indexed by arbitrary index sets.
The subscript notation \code{a[k]} selects the item indexed
by \code{k} from the mapping \code{a}; this can be used in
expressions and as the target of assignments or \keyword{del} statements.
The built-in function \function{len()} returns the number of items
in a mapping.
\bifuncindex{len}
\index{subscription}
\obindex{mapping}

There is currently a single intrinsic mapping type:

\begin{description}

\item[Dictionaries]
These\obindex{dictionary} represent finite sets of objects indexed by
nearly arbitrary values.  The only types of values not acceptable as
keys are values containing lists or dictionaries or other mutable
types that are compared by value rather than by object identity, the
reason being that the efficient implementation of dictionaries
requires a key's hash value to remain constant.
Numeric types used for keys obey the normal rules for numeric
comparison: if two numbers compare equal (e.g., \code{1} and
\code{1.0}) then they can be used interchangeably to index the same
dictionary entry.

Dictionaries are mutable; they can be created by the
\code{\{...\}} notation (see section~\ref{dict}, ``Dictionary
Displays'').

The extension modules \module{dbm}\refstmodindex{dbm},
\module{gdbm}\refstmodindex{gdbm}, \module{bsddb}\refstmodindex{bsddb}
provide additional examples of mapping types.

\end{description} % Mapping types

\item[Callable types]
These\obindex{callable} are the types to which the function call
operation (see section~\ref{calls}, ``Calls'') can be applied:
\indexii{function}{call}
\index{invocation}
\indexii{function}{argument}

\begin{description}

\item[User-defined functions]
A user-defined function object is created by a function definition
(see section~\ref{function}, ``Function definitions'').  It should be
called with an argument
list containing the same number of items as the function's formal
parameter list.
\indexii{user-defined}{function}
\obindex{function}
\obindex{user-defined function}

Special attributes: 

\begin{tableiii}{lll}{member}{Attribute}{Meaning}{}
  \lineiii{func_doc}{The function's documentation string, or
    \code{None} if unavailable}{Writable}

  \lineiii{__doc__}{Another way of spelling
    \member{func_doc}}{Writable}

  \lineiii{func_name}{The function's name}{Writable}

  \lineiii{__name__}{Another way of spelling
    \member{func_name}}{Writable}

  \lineiii{__module__}{The name of the module the function was defined
    in, or \code{None} if unavailable.}{Writable}

  \lineiii{func_defaults}{Atuple containing default argument values
    for those arguments that have defaults, or \code{None} if no
    arguments have a default value}{Writable}

  \lineiii{func_code}{The code object representing the compiled
    function body.}{Writable}

  \lineiii{func_globals}{A reference to the dictionary that holds the
    function's global variables --- the global namespace of the module
    in which the function was defined.}{Read-only}

  \lineiii{func_dict}{The namespace supporting arbitrary function
    attributes.}{Writable}

  \lineiii{func_closure}{\code{None} or a tuple of cells that contain
    bindings for the function's free variables.}{Read-only}
\end{tableiii}

Most of the attributes labelled ``Writable'' check the type of the
assigned value.

\versionchanged[\code{func_name} is now writable]{2.4}

Function objects also support getting and setting arbitrary
attributes, which can be used, for example, to attach metadata to
functions.  Regular attribute dot-notation is used to get and set such
attributes. \emph{Note that the current implementation only supports
function attributes on user-defined functions.  Function attributes on
built-in functions may be supported in the future.}

Additional information about a function's definition can be retrieved
from its code object; see the description of internal types below.

\withsubitem{(function attribute)}{
  \ttindex{func_doc}
  \ttindex{__doc__}
  \ttindex{__name__}
  \ttindex{__module__}
  \ttindex{__dict__}
  \ttindex{func_defaults}
  \ttindex{func_closure}
  \ttindex{func_code}
  \ttindex{func_globals}
  \ttindex{func_dict}}
\indexii{global}{namespace}

\item[User-defined methods]
A user-defined method object combines a class, a class instance (or
\code{None}) and any callable object (normally a user-defined
function).
\obindex{method}
\obindex{user-defined method}
\indexii{user-defined}{method}

Special read-only attributes: \member{im_self} is the class instance
object, \member{im_func} is the function object;
\member{im_class} is the class of \member{im_self} for bound methods
or the class that asked for the method for unbound methods;
\member{__doc__} is the method's documentation (same as
\code{im_func.__doc__}); \member{__name__} is the method name (same as
\code{im_func.__name__}); \member{__module__} is the name of the
module the method was defined in, or \code{None} if unavailable.
\versionchanged[\member{im_self} used to refer to the class that
                defined the method]{2.2}
\withsubitem{(method attribute)}{
  \ttindex{__doc__}
  \ttindex{__name__}
  \ttindex{__module__}
  \ttindex{im_func}
  \ttindex{im_self}}

Methods also support accessing (but not setting) the arbitrary
function attributes on the underlying function object.

User-defined method objects may be created when getting an attribute
of a class (perhaps via an instance of that class), if that attribute
is a user-defined function object, an unbound user-defined method object,
or a class method object.
When the attribute is a user-defined method object, a new
method object is only created if the class from which it is being
retrieved is the same as, or a derived class of, the class stored
in the original method object; otherwise, the original method object
is used as it is.

When a user-defined method object is created by retrieving
a user-defined function object from a class, its \member{im_self}
attribute is \code{None} and the method object is said to be unbound.
When one is created by retrieving a user-defined function object
from a class via one of its instances, its \member{im_self} attribute
is the instance, and the method object is said to be bound.
In either case, the new method's \member{im_class} attribute
is the class from which the retrieval takes place, and
its \member{im_func} attribute is the original function object.
\withsubitem{(method attribute)}{
  \ttindex{im_class}\ttindex{im_func}\ttindex{im_self}}

When a user-defined method object is created by retrieving another
method object from a class or instance, the behaviour is the same
as for a function object, except that the \member{im_func} attribute
of the new instance is not the original method object but its
\member{im_func} attribute.
\withsubitem{(method attribute)}{
  \ttindex{im_func}}

When a user-defined method object is created by retrieving a
class method object from a class or instance, its \member{im_self}
attribute is the class itself (the same as the \member{im_class}
attribute), and its \member{im_func} attribute is the function
object underlying the class method.
\withsubitem{(method attribute)}{
  \ttindex{im_class}\ttindex{im_func}\ttindex{im_self}}

When an unbound user-defined method object is called, the underlying
function (\member{im_func}) is called, with the restriction that the
first argument must be an instance of the proper class
(\member{im_class}) or of a derived class thereof.

When a bound user-defined method object is called, the underlying
function (\member{im_func}) is called, inserting the class instance
(\member{im_self}) in front of the argument list.  For instance, when
\class{C} is a class which contains a definition for a function
\method{f()}, and \code{x} is an instance of \class{C}, calling
\code{x.f(1)} is equivalent to calling \code{C.f(x, 1)}.

When a user-defined method object is derived from a class method object,
the ``class instance'' stored in \member{im_self} will actually be the
class itself, so that calling either \code{x.f(1)} or \code{C.f(1)} is
equivalent to calling \code{f(C,1)} where \code{f} is the underlying
function.

Note that the transformation from function object to (unbound or
bound) method object happens each time the attribute is retrieved from
the class or instance.  In some cases, a fruitful optimization is to
assign the attribute to a local variable and call that local variable.
Also notice that this transformation only happens for user-defined
functions; other callable objects (and all non-callable objects) are
retrieved without transformation.  It is also important to note that
user-defined functions which are attributes of a class instance are
not converted to bound methods; this \emph{only} happens when the
function is an attribute of the class.

\item[Generator functions\index{generator!function}\index{generator!iterator}]
A function or method which uses the \keyword{yield} statement (see
section~\ref{yield}, ``The \keyword{yield} statement'') is called a
\dfn{generator function}.  Such a function, when called, always
returns an iterator object which can be used to execute the body of
the function:  calling the iterator's \method{next()} method will
cause the function to execute until it provides a value using the
\keyword{yield} statement.  When the function executes a
\keyword{return} statement or falls off the end, a
\exception{StopIteration} exception is raised and the iterator will
have reached the end of the set of values to be returned.

\item[Built-in functions]
A built-in function object is a wrapper around a \C{} function.  Examples
of built-in functions are \function{len()} and \function{math.sin()}
(\module{math} is a standard built-in module).
The number and type of the arguments are
determined by the C function.
Special read-only attributes: \member{__doc__} is the function's
documentation string, or \code{None} if unavailable; \member{__name__}
is the function's name; \member{__self__} is set to \code{None} (but see
the next item); \member{__module__} is the name of the module the
function was defined in or \code{None} if unavailable.
\obindex{built-in function}
\obindex{function}
\indexii{C}{language}

\item[Built-in methods]
This is really a different disguise of a built-in function, this time
containing an object passed to the C function as an implicit extra
argument.  An example of a built-in method is
\code{\var{alist}.append()}, assuming
\var{alist} is a list object.
In this case, the special read-only attribute \member{__self__} is set
to the object denoted by \var{list}.
\obindex{built-in method}
\obindex{method}
\indexii{built-in}{method}

\item[Class Types]
Class types, or ``new-style classes,'' are callable.  These objects
normally act as factories for new instances of themselves, but
variations are possible for class types that override
\method{__new__()}.  The arguments of the call are passed to
\method{__new__()} and, in the typical case, to \method{__init__()} to
initialize the new instance.

\item[Classic Classes]
Class objects are described below.  When a class object is called,
a new class instance (also described below) is created and
returned.  This implies a call to the class's \method{__init__()} method
if it has one.  Any arguments are passed on to the \method{__init__()}
method.  If there is no \method{__init__()} method, the class must be called
without arguments.
\withsubitem{(object method)}{\ttindex{__init__()}}
\obindex{class}
\obindex{class instance}
\obindex{instance}
\indexii{class object}{call}

\item[Class instances]
Class instances are described below.  Class instances are callable
only when the class has a \method{__call__()} method; \code{x(arguments)}
is a shorthand for \code{x.__call__(arguments)}.

\end{description}

\item[Modules]
Modules are imported by the \keyword{import} statement (see
section~\ref{import}, ``The \keyword{import} statement'').%
\stindex{import}\obindex{module}
A module object has a namespace implemented by a dictionary object
(this is the dictionary referenced by the func_globals attribute of
functions defined in the module).  Attribute references are translated
to lookups in this dictionary, e.g., \code{m.x} is equivalent to
\code{m.__dict__["x"]}.
A module object does not contain the code object used to
initialize the module (since it isn't needed once the initialization
is done).

Attribute assignment updates the module's namespace dictionary,
e.g., \samp{m.x = 1} is equivalent to \samp{m.__dict__["x"] = 1}.

Special read-only attribute: \member{__dict__} is the module's
namespace as a dictionary object.
\withsubitem{(module attribute)}{\ttindex{__dict__}}

Predefined (writable) attributes: \member{__name__}
is the module's name; \member{__doc__} is the
module's documentation string, or
\code{None} if unavailable; \member{__file__} is the pathname of the
file from which the module was loaded, if it was loaded from a file.
The \member{__file__} attribute is not present for C{} modules that are
statically linked into the interpreter; for extension modules loaded
dynamically from a shared library, it is the pathname of the shared
library file.
\withsubitem{(module attribute)}{
  \ttindex{__name__}
  \ttindex{__doc__}
  \ttindex{__file__}}
\indexii{module}{namespace}

\item[Classes]
Class objects are created by class definitions (see
section~\ref{class}, ``Class definitions'').
A class has a namespace implemented by a dictionary object.
Class attribute references are translated to
lookups in this dictionary,
e.g., \samp{C.x} is translated to \samp{C.__dict__["x"]}.
When the attribute name is not found
there, the attribute search continues in the base classes.  The search
is depth-first, left-to-right in the order of occurrence in the
base class list.

When a class attribute reference (for class \class{C}, say)
would yield a user-defined function object or
an unbound user-defined method object whose associated class is either
\class{C} or one of its base classes, it is transformed into an unbound
user-defined method object whose \member{im_class} attribute is~\class{C}.
When it would yield a class method object, it is transformed into
a bound user-defined method object whose \member{im_class} and
\member{im_self} attributes are both~\class{C}.  When it would yield
a static method object, it is transformed into the object wrapped
by the static method object. See section~\ref{descriptors} for another
way in which attributes retrieved from a class may differ from those
actually contained in its \member{__dict__}.
\obindex{class}
\obindex{class instance}
\obindex{instance}
\indexii{class object}{call}
\index{container}
\obindex{dictionary}
\indexii{class}{attribute}

Class attribute assignments update the class's dictionary, never the
dictionary of a base class.
\indexiii{class}{attribute}{assignment}

A class object can be called (see above) to yield a class instance (see
below).
\indexii{class object}{call}

Special attributes: \member{__name__} is the class name;
\member{__module__} is the module name in which the class was defined;
\member{__dict__} is the dictionary containing the class's namespace;
\member{__bases__} is a tuple (possibly empty or a singleton)
containing the base classes, in the order of their occurrence in the
base class list; \member{__doc__} is the class's documentation string,
or None if undefined.
\withsubitem{(class attribute)}{
  \ttindex{__name__}
  \ttindex{__module__}
  \ttindex{__dict__}
  \ttindex{__bases__}
  \ttindex{__doc__}}

\item[Class instances]
A class instance is created by calling a class object (see above).
A class instance has a namespace implemented as a dictionary which
is the first place in which
attribute references are searched.  When an attribute is not found
there, and the instance's class has an attribute by that name,
the search continues with the class attributes.  If a class attribute
is found that is a user-defined function object or an unbound
user-defined method object whose associated class is the class
(call it~\class{C}) of the instance for which the attribute reference
was initiated or one of its bases,
it is transformed into a bound user-defined method object whose
\member{im_class} attribute is~\class{C} whose \member{im_self} attribute
is the instance. Static method and class method objects are also
transformed, as if they had been retrieved from class~\class{C};
see above under ``Classes''. See section~\ref{descriptors} for
another way in which attributes of a class retrieved via its
instances may differ from the objects actually stored in the
class's \member{__dict__}.
If no class attribute is found, and the object's class has a
\method{__getattr__()} method, that is called to satisfy the lookup.
\obindex{class instance}
\obindex{instance}
\indexii{class}{instance}
\indexii{class instance}{attribute}

Attribute assignments and deletions update the instance's dictionary,
never a class's dictionary.  If the class has a \method{__setattr__()} or
\method{__delattr__()} method, this is called instead of updating the
instance dictionary directly.
\indexiii{class instance}{attribute}{assignment}

Class instances can pretend to be numbers, sequences, or mappings if
they have methods with certain special names.  See
section~\ref{specialnames}, ``Special method names.''
\obindex{numeric}
\obindex{sequence}
\obindex{mapping}

Special attributes: \member{__dict__} is the attribute
dictionary; \member{__class__} is the instance's class.
\withsubitem{(instance attribute)}{
  \ttindex{__dict__}
  \ttindex{__class__}}

\item[Files]
A file\obindex{file} object represents an open file.  File objects are
created by the \function{open()}\bifuncindex{open} built-in function,
and also by
\withsubitem{(in module os)}{\ttindex{popen()}}\function{os.popen()},
\function{os.fdopen()}, and the
\method{makefile()}\withsubitem{(socket method)}{\ttindex{makefile()}}
method of socket objects (and perhaps by other functions or methods
provided by extension modules).  The objects
\ttindex{sys.stdin}\code{sys.stdin},
\ttindex{sys.stdout}\code{sys.stdout} and
\ttindex{sys.stderr}\code{sys.stderr} are initialized to file objects
corresponding to the interpreter's standard\index{stdio} input, output
and error streams.  See the \citetitle[../lib/lib.html]{Python Library
Reference} for complete documentation of file objects.
\withsubitem{(in module sys)}{
  \ttindex{stdin}
  \ttindex{stdout}
  \ttindex{stderr}}


\item[Internal types]
A few types used internally by the interpreter are exposed to the user.
Their definitions may change with future versions of the interpreter,
but they are mentioned here for completeness.
\index{internal type}
\index{types, internal}

\begin{description}

\item[Code objects]
Code objects represent \emph{byte-compiled} executable Python code, or 
\emph{bytecode}.
The difference between a code
object and a function object is that the function object contains an
explicit reference to the function's globals (the module in which it
was defined), while a code object contains no context; 
also the default argument values are stored in the function object,
not in the code object (because they represent values calculated at
run-time).  Unlike function objects, code objects are immutable and
contain no references (directly or indirectly) to mutable objects.
\index{bytecode}
\obindex{code}

Special read-only attributes: \member{co_name} gives the function
name; \member{co_argcount} is the number of positional arguments
(including arguments with default values); \member{co_nlocals} is the
number of local variables used by the function (including arguments);
\member{co_varnames} is a tuple containing the names of the local
variables (starting with the argument names); \member{co_cellvars} is
a tuple containing the names of local variables that are referenced by
nested functions; \member{co_freevars} is a tuple containing the names
of free variables; \member{co_code} is a string representing the
sequence of bytecode instructions;
\member{co_consts} is a tuple containing the literals used by the
bytecode; \member{co_names} is a tuple containing the names used by
the bytecode; \member{co_filename} is the filename from which the code
was compiled; \member{co_firstlineno} is the first line number of the
function; \member{co_lnotab} is a string encoding the mapping from
byte code offsets to line numbers (for details see the source code of
the interpreter); \member{co_stacksize} is the required stack size
(including local variables); \member{co_flags} is an integer encoding
a number of flags for the interpreter.

\withsubitem{(code object attribute)}{
  \ttindex{co_argcount}
  \ttindex{co_code}
  \ttindex{co_consts}
  \ttindex{co_filename}
  \ttindex{co_firstlineno}
  \ttindex{co_flags}
  \ttindex{co_lnotab}
  \ttindex{co_name}
  \ttindex{co_names}
  \ttindex{co_nlocals}
  \ttindex{co_stacksize}
  \ttindex{co_varnames}
  \ttindex{co_cellvars}
  \ttindex{co_freevars}}

The following flag bits are defined for \member{co_flags}: bit
\code{0x04} is set if the function uses the \samp{*arguments} syntax
to accept an arbitrary number of positional arguments; bit
\code{0x08} is set if the function uses the \samp{**keywords} syntax
to accept arbitrary keyword arguments; bit \code{0x20} is set if the
function is a generator.
\obindex{generator}

Future feature declarations (\samp{from __future__ import division})
also use bits in \member{co_flags} to indicate whether a code object
was compiled with a particular feature enabled: bit \code{0x2000} is
set if the function was compiled with future division enabled; bits
\code{0x10} and \code{0x1000} were used in earlier versions of Python.

Other bits in \member{co_flags} are reserved for internal use.

If\index{documentation string} a code object represents a function,
the first item in
\member{co_consts} is the documentation string of the function, or
\code{None} if undefined.

\item[Frame objects]
Frame objects represent execution frames.  They may occur in traceback
objects (see below).
\obindex{frame}

Special read-only attributes: \member{f_back} is to the previous
stack frame (towards the caller), or \code{None} if this is the bottom
stack frame; \member{f_code} is the code object being executed in this
frame; \member{f_locals} is the dictionary used to look up local
variables; \member{f_globals} is used for global variables;
\member{f_builtins} is used for built-in (intrinsic) names;
\member{f_restricted} is a flag indicating whether the function is
executing in restricted execution mode; \member{f_lasti} gives the
precise instruction (this is an index into the bytecode string of
the code object).
\withsubitem{(frame attribute)}{
  \ttindex{f_back}
  \ttindex{f_code}
  \ttindex{f_globals}
  \ttindex{f_locals}
  \ttindex{f_lasti}
  \ttindex{f_builtins}
  \ttindex{f_restricted}}

Special writable attributes: \member{f_trace}, if not \code{None}, is a
function called at the start of each source code line (this is used by
the debugger); \member{f_exc_type}, \member{f_exc_value},
\member{f_exc_traceback} represent the most recent exception caught in
this frame; \member{f_lineno} is the current line number of the frame
--- writing to this from within a trace function jumps to the given line
(only for the bottom-most frame).  A debugger can implement a Jump
command (aka Set Next Statement) by writing to f_lineno.
\withsubitem{(frame attribute)}{
  \ttindex{f_trace}
  \ttindex{f_exc_type}
  \ttindex{f_exc_value}
  \ttindex{f_exc_traceback}
  \ttindex{f_lineno}}

\item[Traceback objects] \label{traceback}
Traceback objects represent a stack trace of an exception.  A
traceback object is created when an exception occurs.  When the search
for an exception handler unwinds the execution stack, at each unwound
level a traceback object is inserted in front of the current
traceback.  When an exception handler is entered, the stack trace is
made available to the program.
(See section~\ref{try}, ``The \code{try} statement.'')
It is accessible as \code{sys.exc_traceback}, and also as the third
item of the tuple returned by \code{sys.exc_info()}.  The latter is
the preferred interface, since it works correctly when the program is
using multiple threads.
When the program contains no suitable handler, the stack trace is written
(nicely formatted) to the standard error stream; if the interpreter is
interactive, it is also made available to the user as
\code{sys.last_traceback}.
\obindex{traceback}
\indexii{stack}{trace}
\indexii{exception}{handler}
\indexii{execution}{stack}
\withsubitem{(in module sys)}{
  \ttindex{exc_info}
  \ttindex{exc_traceback}
  \ttindex{last_traceback}}
\ttindex{sys.exc_info}
\ttindex{sys.exc_traceback}
\ttindex{sys.last_traceback}

Special read-only attributes: \member{tb_next} is the next level in the
stack trace (towards the frame where the exception occurred), or
\code{None} if there is no next level; \member{tb_frame} points to the
execution frame of the current level; \member{tb_lineno} gives the line
number where the exception occurred; \member{tb_lasti} indicates the
precise instruction.  The line number and last instruction in the
traceback may differ from the line number of its frame object if the
exception occurred in a \keyword{try} statement with no matching
except clause or with a finally clause.
\withsubitem{(traceback attribute)}{
  \ttindex{tb_next}
  \ttindex{tb_frame}
  \ttindex{tb_lineno}
  \ttindex{tb_lasti}}
\stindex{try}

\item[Slice objects]
Slice objects are used to represent slices when \emph{extended slice
syntax} is used.  This is a slice using two colons, or multiple slices
or ellipses separated by commas, e.g., \code{a[i:j:step]}, \code{a[i:j,
k:l]}, or \code{a[..., i:j]}.  They are also created by the built-in
\function{slice()}\bifuncindex{slice} function.

Special read-only attributes: \member{start} is the lower bound;
\member{stop} is the upper bound; \member{step} is the step value; each is
\code{None} if omitted. These attributes can have any type.
\withsubitem{(slice object attribute)}{
  \ttindex{start}
  \ttindex{stop}
  \ttindex{step}}

Slice objects support one method:

\begin{methoddesc}[slice]{indices}{self, length}
This method takes a single integer argument \var{length} and computes
information about the extended slice that the slice object would
describe if applied to a sequence of \var{length} items.  It returns a
tuple of three integers; respectively these are the \var{start} and
\var{stop} indices and the \var{step} or stride length of the slice.
Missing or out-of-bounds indices are handled in a manner consistent
with regular slices.
\versionadded{2.3}
\end{methoddesc}

\item[Static method objects]
Static method objects provide a way of defeating the transformation
of function objects to method objects described above. A static method
object is a wrapper around any other object, usually a user-defined
method object. When a static method object is retrieved from a class
or a class instance, the object actually returned is the wrapped object,
which is not subject to any further transformation. Static method
objects are not themselves callable, although the objects they
wrap usually are. Static method objects are created by the built-in
\function{staticmethod()} constructor.

\item[Class method objects]
A class method object, like a static method object, is a wrapper
around another object that alters the way in which that object
is retrieved from classes and class instances. The behaviour of
class method objects upon such retrieval is described above,
under ``User-defined methods''. Class method objects are created
by the built-in \function{classmethod()} constructor.

\end{description} % Internal types

\end{description} % Types


\section{Special method names\label{specialnames}}

A class can implement certain operations that are invoked by special
syntax (such as arithmetic operations or subscripting and slicing) by
defining methods with special names.\indexii{operator}{overloading}
This is Python's approach to \dfn{operator overloading}, allowing
classes to define their own behavior with respect to language
operators.  For instance, if a class defines
a method named \method{__getitem__()}, and \code{x} is an instance of
this class, then \code{x[i]} is equivalent to
\code{x.__getitem__(i)}.  Except where mentioned, attempts to execute
an operation raise an exception when no appropriate method is defined.
\withsubitem{(mapping object method)}{\ttindex{__getitem__()}}

When implementing a class that emulates any built-in type, it is
important that the emulation only be implemented to the degree that it
makes sense for the object being modelled.  For example, some
sequences may work well with retrieval of individual elements, but
extracting a slice may not make sense.  (One example of this is the
\class{NodeList} interface in the W3C's Document Object Model.)


\subsection{Basic customization\label{customization}}

\begin{methoddesc}[object]{__new__}{cls\optional{, \moreargs}}
Called to create a new instance of class \var{cls}.  \method{__new__()}
is a static method (special-cased so you need not declare it as such)
that takes the class of which an instance was requested as its first
argument.  The remaining arguments are those passed to the object
constructor expression (the call to the class).  The return value of
\method{__new__()} should be the new object instance (usually an
instance of \var{cls}).

Typical implementations create a new instance of the class by invoking
the superclass's \method{__new__()} method using
\samp{super(\var{currentclass}, \var{cls}).__new__(\var{cls}[, ...])}
with appropriate arguments and then modifying the newly-created instance
as necessary before returning it.

If \method{__new__()} returns an instance of \var{cls}, then the new
instance's \method{__init__()} method will be invoked like
\samp{__init__(\var{self}[, ...])}, where \var{self} is the new instance
and the remaining arguments are the same as were passed to
\method{__new__()}.

If \method{__new__()} does not return an instance of \var{cls}, then the
new instance's \method{__init__()} method will not be invoked.

\method{__new__()} is intended mainly to allow subclasses of
immutable types (like int, str, or tuple) to customize instance
creation.
\end{methoddesc}

\begin{methoddesc}[object]{__init__}{self\optional{, \moreargs}}
Called\indexii{class}{constructor} when the instance is created.  The
arguments are those passed to the class constructor expression.  If a
base class has an \method{__init__()} method, the derived class's
\method{__init__()} method, if any, must explicitly call it to ensure proper
initialization of the base class part of the instance; for example:
\samp{BaseClass.__init__(\var{self}, [\var{args}...])}.  As a special
constraint on constructors, no value may be returned; doing so will
cause a \exception{TypeError} to be raised at runtime.
\end{methoddesc}


\begin{methoddesc}[object]{__del__}{self}
Called when the instance is about to be destroyed.  This is also
called a destructor\index{destructor}.  If a base class
has a \method{__del__()} method, the derived class's \method{__del__()}
method, if any,
must explicitly call it to ensure proper deletion of the base class
part of the instance.  Note that it is possible (though not recommended!)
for the \method{__del__()}
method to postpone destruction of the instance by creating a new
reference to it.  It may then be called at a later time when this new
reference is deleted.  It is not guaranteed that
\method{__del__()} methods are called for objects that still exist when
the interpreter exits.
\stindex{del}

\begin{notice}
\samp{del x} doesn't directly call
\code{x.__del__()} --- the former decrements the reference count for
\code{x} by one, and the latter is only called when \code{x}'s reference
count reaches zero.  Some common situations that may prevent the
reference count of an object from going to zero include: circular
references between objects (e.g., a doubly-linked list or a tree data
structure with parent and child pointers); a reference to the object
on the stack frame of a function that caught an exception (the
traceback stored in \code{sys.exc_traceback} keeps the stack frame
alive); or a reference to the object on the stack frame that raised an
unhandled exception in interactive mode (the traceback stored in
\code{sys.last_traceback} keeps the stack frame alive).  The first
situation can only be remedied by explicitly breaking the cycles; the
latter two situations can be resolved by storing \code{None} in
\code{sys.exc_traceback} or \code{sys.last_traceback}.  Circular
references which are garbage are detected when the option cycle
detector is enabled (it's on by default), but can only be cleaned up
if there are no Python-level \method{__del__()} methods involved.
Refer to the documentation for the \ulink{\module{gc}
module}{../lib/module-gc.html} for more information about how
\method{__del__()} methods are handled by the cycle detector,
particularly the description of the \code{garbage} value.
\end{notice}

\begin{notice}[warning]
Due to the precarious circumstances under which
\method{__del__()} methods are invoked, exceptions that occur during their
execution are ignored, and a warning is printed to \code{sys.stderr}
instead.  Also, when \method{__del__()} is invoked in response to a module
being deleted (e.g., when execution of the program is done), other
globals referenced by the \method{__del__()} method may already have been
deleted.  For this reason, \method{__del__()} methods should do the
absolute minimum needed to maintain external invariants.  Starting with
version 1.5, Python guarantees that globals whose name begins with a single
underscore are deleted from their module before other globals are deleted;
if no other references to such globals exist, this may help in assuring that
imported modules are still available at the time when the
\method{__del__()} method is called.
\end{notice}
\end{methoddesc}

\begin{methoddesc}[object]{__repr__}{self}
Called by the \function{repr()}\bifuncindex{repr} built-in function
and by string conversions (reverse quotes) to compute the ``official''
string representation of an object.  If at all possible, this should
look like a valid Python expression that could be used to recreate an
object with the same value (given an appropriate environment).  If
this is not possible, a string of the form \samp{<\var{...some useful
description...}>} should be returned.  The return value must be a
string object.
If a class defines \method{__repr__()} but not \method{__str__()},
then \method{__repr__()} is also used when an ``informal'' string
representation of instances of that class is required.		     

This is typically used for debugging, so it is important that the
representation is information-rich and unambiguous.
\indexii{string}{conversion}
\indexii{reverse}{quotes}
\indexii{backward}{quotes}
\index{back-quotes}
\end{methoddesc}

\begin{methoddesc}[object]{__str__}{self}
Called by the \function{str()}\bifuncindex{str} built-in function and
by the \keyword{print}\stindex{print} statement to compute the
``informal'' string representation of an object.  This differs from
\method{__repr__()} in that it does not have to be a valid Python
expression: a more convenient or concise representation may be used
instead.  The return value must be a string object.
\end{methoddesc}

\begin{methoddesc}[object]{__lt__}{self, other}
\methodline[object]{__le__}{self, other}
\methodline[object]{__eq__}{self, other}
\methodline[object]{__ne__}{self, other}
\methodline[object]{__gt__}{self, other}
\methodline[object]{__ge__}{self, other}
\versionadded{2.1}
These are the so-called ``rich comparison'' methods, and are called
for comparison operators in preference to \method{__cmp__()} below.
The correspondence between operator symbols and method names is as
follows:
\code{\var{x}<\var{y}} calls \code{\var{x}.__lt__(\var{y})},
\code{\var{x}<=\var{y}} calls \code{\var{x}.__le__(\var{y})},
\code{\var{x}==\var{y}} calls \code{\var{x}.__eq__(\var{y})},
\code{\var{x}!=\var{y}} and \code{\var{x}<>\var{y}} call
\code{\var{x}.__ne__(\var{y})},
\code{\var{x}>\var{y}} calls \code{\var{x}.__gt__(\var{y})}, and
\code{\var{x}>=\var{y}} calls \code{\var{x}.__ge__(\var{y})}.
These methods can return any value, but if the comparison operator is
used in a Boolean context, the return value should be interpretable as
a Boolean value, else a \exception{TypeError} will be raised.
By convention, \code{False} is used for false and \code{True} for true.

There are no implied relationships among the comparison operators.
The truth of \code{\var{x}==\var{y}} does not imply that \code{\var{x}!=\var{y}}
is false.  Accordingly, when defining \method{__eq__}, one should also
define \method{__ne__} so that the operators will behave as expected.

There are no reflected (swapped-argument) versions of these methods
(to be used when the left argument does not support the operation but
the right argument does); rather, \method{__lt__()} and
\method{__gt__()} are each other's reflection, \method{__le__()} and
\method{__ge__()} are each other's reflection, and \method{__eq__()}
and \method{__ne__()} are their own reflection.

Arguments to rich comparison methods are never coerced.  A rich
comparison method may return \code{NotImplemented} if it does not
implement the operation for a given pair of arguments.
\end{methoddesc}

\begin{methoddesc}[object]{__cmp__}{self, other}
Called by comparison operations if rich comparison (see above) is not
defined.  Should return a negative integer if \code{self < other},
zero if \code{self == other}, a positive integer if \code{self >
other}.  If no \method{__cmp__()}, \method{__eq__()} or
\method{__ne__()} operation is defined, class instances are compared
by object identity (``address'').  See also the description of
\method{__hash__()} for some important notes on creating objects which
support custom comparison operations and are usable as dictionary
keys.
(Note: the restriction that exceptions are not propagated by
\method{__cmp__()} has been removed since Python 1.5.)
\bifuncindex{cmp}
\index{comparisons}
\end{methoddesc}

\begin{methoddesc}[object]{__rcmp__}{self, other}
  \versionchanged[No longer supported]{2.1}
\end{methoddesc}

\begin{methoddesc}[object]{__hash__}{self}
Called for the key object for dictionary \obindex{dictionary}
operations, and by the built-in function
\function{hash()}\bifuncindex{hash}.  Should return a 32-bit integer
usable as a hash value
for dictionary operations.  The only required property is that objects
which compare equal have the same hash value; it is advised to somehow
mix together (e.g., using exclusive or) the hash values for the
components of the object that also play a part in comparison of
objects.  If a class does not define a \method{__cmp__()} method it should
not define a \method{__hash__()} operation either; if it defines
\method{__cmp__()} or \method{__eq__()} but not \method{__hash__()},
its instances will not be usable as dictionary keys.  If a class
defines mutable objects and implements a \method{__cmp__()} or
\method{__eq__()} method, it should not implement \method{__hash__()},
since the dictionary implementation requires that a key's hash value
is immutable (if the object's hash value changes, it will be in the
wrong hash bucket).
\withsubitem{(object method)}{\ttindex{__cmp__()}}
\end{methoddesc}

\begin{methoddesc}[object]{__nonzero__}{self}
Called to implement truth value testing, and the built-in operation
\code{bool()}; should return \code{False} or \code{True}, or their
integer equivalents \code{0} or \code{1}.
When this method is not defined, \method{__len__()} is
called, if it is defined (see below).  If a class defines neither
\method{__len__()} nor \method{__nonzero__()}, all its instances are
considered true.
\withsubitem{(mapping object method)}{\ttindex{__len__()}}
\end{methoddesc}

\begin{methoddesc}[object]{__unicode__}{self}
Called to implement \function{unicode()}\bifuncindex{unicode} builtin;
should return a Unicode object. When this method is not defined, string
conversion is attempted, and the result of string conversion is converted
to Unicode using the system default encoding.
\end{methoddesc}


\subsection{Customizing attribute access\label{attribute-access}}

The following methods can be defined to customize the meaning of
attribute access (use of, assignment to, or deletion of \code{x.name})
for class instances.

\begin{methoddesc}[object]{__getattr__}{self, name}
Called when an attribute lookup has not found the attribute in the
usual places (i.e. it is not an instance attribute nor is it found in
the class tree for \code{self}).  \code{name} is the attribute name.
This method should return the (computed) attribute value or raise an
\exception{AttributeError} exception.

Note that if the attribute is found through the normal mechanism,
\method{__getattr__()} is not called.  (This is an intentional
asymmetry between \method{__getattr__()} and \method{__setattr__()}.)
This is done both for efficiency reasons and because otherwise
\method{__setattr__()} would have no way to access other attributes of
the instance.  Note that at least for instance variables, you can fake
total control by not inserting any values in the instance attribute
dictionary (but instead inserting them in another object).  See the
\method{__getattribute__()} method below for a way to actually get
total control in new-style classes.
\withsubitem{(object method)}{\ttindex{__setattr__()}}
\end{methoddesc}

\begin{methoddesc}[object]{__setattr__}{self, name, value}
Called when an attribute assignment is attempted.  This is called
instead of the normal mechanism (i.e.\ store the value in the instance
dictionary).  \var{name} is the attribute name, \var{value} is the
value to be assigned to it.

If \method{__setattr__()} wants to assign to an instance attribute, it 
should not simply execute \samp{self.\var{name} = value} --- this
would cause a recursive call to itself.  Instead, it should insert the
value in the dictionary of instance attributes, e.g.,
\samp{self.__dict__[\var{name}] = value}.  For new-style classes,
rather than accessing the instance dictionary, it should call the base
class method with the same name, for example,
\samp{object.__setattr__(self, name, value)}.
\withsubitem{(instance attribute)}{\ttindex{__dict__}}
\end{methoddesc}

\begin{methoddesc}[object]{__delattr__}{self, name}
Like \method{__setattr__()} but for attribute deletion instead of
assignment.  This should only be implemented if \samp{del
obj.\var{name}} is meaningful for the object.
\end{methoddesc}

\subsubsection{More attribute access for new-style classes \label{new-style-attribute-access}}

The following methods only apply to new-style classes.

\begin{methoddesc}[object]{__getattribute__}{self, name}
Called unconditionally to implement attribute accesses for instances
of the class. If the class also defines \method{__getattr__}, it will
never be called (unless called explicitly).
This method should return the (computed) attribute
value or raise an \exception{AttributeError} exception.
In order to avoid infinite recursion in this method, its
implementation should always call the base class method with the same
name to access any attributes it needs, for example,
\samp{object.__getattribute__(self, name)}.
\end{methoddesc}

\subsubsection{Implementing Descriptors \label{descriptors}}

The following methods only apply when an instance of the class
containing the method (a so-called \emph{descriptor} class) appears in
the class dictionary of another new-style class, known as the
\emph{owner} class. In the examples below, ``the attribute'' refers to
the attribute whose name is the key of the property in the owner
class' \code{__dict__}.  Descriptors can only be implemented as
new-style classes themselves.

\begin{methoddesc}[object]{__get__}{self, instance, owner}
Called to get the attribute of the owner class (class attribute access)
or of an instance of that class (instance attribute access).
\var{owner} is always the owner class, while \var{instance} is the
instance that the attribute was accessed through, or \code{None} when
the attribute is accessed through the \var{owner}.  This method should
return the (computed) attribute value or raise an
\exception{AttributeError} exception.
\end{methoddesc}

\begin{methoddesc}[object]{__set__}{self, instance, value}
Called to set the attribute on an instance \var{instance} of the owner
class to a new value, \var{value}.
\end{methoddesc}

\begin{methoddesc}[object]{__delete__}{self, instance}
Called to delete the attribute on an instance \var{instance} of the
owner class.
\end{methoddesc}


\subsubsection{Invoking Descriptors \label{descriptor-invocation}}

In general, a descriptor is an object attribute with ``binding behavior'',
one whose attribute access has been overridden by methods in the descriptor
protocol:  \method{__get__()}, \method{__set__()}, and \method{__delete__()}.
If any of those methods are defined for an object, it is said to be a
descriptor.

The default behavior for attribute access is to get, set, or delete the
attribute from an object's dictionary. For instance, \code{a.x} has a
lookup chain starting with \code{a.__dict__['x']}, then
\code{type(a).__dict__['x']}, and continuing 
through the base classes of \code{type(a)} excluding metaclasses.

However, if the looked-up value is an object defining one of the descriptor
methods, then Python may override the default behavior and invoke the
descriptor method instead.  Where this occurs in the precedence chain depends
on which descriptor methods were defined and how they were called.  Note that
descriptors are only invoked for new style objects or classes
(ones that subclass \class{object()} or \class{type()}).

The starting point for descriptor invocation is a binding, \code{a.x}.
How the arguments are assembled depends on \code{a}:

\begin{itemize}
                      
  \item[Direct Call] The simplest and least common call is when user code
    directly invokes a descriptor method:    \code{x.__get__(a)}.

  \item[Instance Binding]  If binding to a new-style object instance,
    \code{a.x} is transformed into the call:
    \code{type(a).__dict__['x'].__get__(a, type(a))}.
                     
  \item[Class Binding]  If binding to a new-style class, \code{A.x}
    is transformed into the call: \code{A.__dict__['x'].__get__(None, A)}.

  \item[Super Binding] If \code{a} is an instance of \class{super},
    then the binding \code{super(B, obj).m()} searches
    \code{obj.__class__.__mro__} for the base class \code{A} immediately
    preceding \code{B} and then invokes the descriptor with the call:
    \code{A.__dict__['m'].__get__(obj, A)}.
                     
\end{itemize}

For instance bindings, the precedence of descriptor invocation depends
on the which descriptor methods are defined.  Data descriptors define
both \method{__get__()} and \method{__set__()}.  Non-data descriptors have
just the \method{__get__()} method.  Data descriptors always override
a redefinition in an instance dictionary.  In contrast, non-data
descriptors can be overridden by instances.

Python methods (including \function{staticmethod()} and \function{classmethod()})
are implemented as non-data descriptors.  Accordingly, instances can
redefine and override methods.  This allows individual instances to acquire
behaviors that differ from other instances of the same class.                     

The \function{property()} function is implemented as a data descriptor.
Accordingly, instances cannot override the behavior of a property.


\subsubsection{__slots__\label{slots}}

By default, instances of both old and new-style classes have a dictionary
for attribute storage.  This wastes space for objects having very few instance
variables.  The space consumption can become acute when creating large numbers
of instances.

The default can be overridden by defining \var{__slots__} in a new-style class
definition.  The \var{__slots__} declaration takes a sequence of instance
variables and reserves just enough space in each instance to hold a value
for each variable.  Space is saved because \var{__dict__} is not created for
each instance.
    
\begin{datadesc}{__slots__}
This class variable can be assigned a string, iterable, or sequence of strings
with variable names used by instances.  If defined in a new-style class,
\var{__slots__} reserves space for the declared variables
and prevents the automatic creation of \var{__dict__} and \var{__weakref__}
for each instance.
\versionadded{2.2}                     
\end{datadesc}

\noindent
Notes on using \var{__slots__}

\begin{itemize}

\item Without a \var{__dict__} variable, instances cannot be assigned new
variables not listed in the \var{__slots__} definition.  Attempts to assign
to an unlisted variable name raises \exception{AttributeError}. If dynamic
assignment of new variables is desired, then add \code{'__dict__'} to the
sequence of strings in the \var{__slots__} declaration.                                     
\versionchanged[Previously, adding \code{'__dict__'} to the \var{__slots__}
declaration would not enable the assignment of new attributes not
specifically listed in the sequence of instance variable names]{2.3}                     

\item Without a \var{__weakref__} variable for each instance, classes
defining \var{__slots__} do not support weak references to its instances.
If weak reference support is needed, then add \code{'__weakref__'} to the
sequence of strings in the \var{__slots__} declaration.                    
\versionchanged[Previously, adding \code{'__weakref__'} to the \var{__slots__}
declaration would not enable support for weak references]{2.3}                                            

\item \var{__slots__} are implemented at the class level by creating
descriptors (\ref{descriptors}) for each variable name.  As a result,
class attributes cannot be used to set default values for instance
variables defined by \var{__slots__}; otherwise, the class attribute would
overwrite the descriptor assignment. 

\item If a class defines a slot also defined in a base class, the instance
variable defined by the base class slot is inaccessible (except by retrieving
its descriptor directly from the base class). This renders the meaning of the
program undefined.  In the future, a check may be added to prevent this.

\item The action of a \var{__slots__} declaration is limited to the class
where it is defined.  As a result, subclasses will have a \var{__dict__}
unless they also define  \var{__slots__}.                     

\item \var{__slots__} do not work for classes derived from ``variable-length''
built-in types such as \class{long}, \class{str} and \class{tuple}. 

\item Any non-string iterable may be assigned to \var{__slots__}.
Mappings may also be used; however, in the future, special meaning may
be assigned to the values corresponding to each key.                      

\end{itemize}


\subsection{Customizing class creation\label{metaclasses}}

By default, new-style classes are constructed using \function{type()}.
A class definition is read into a separate namespace and the value
of class name is bound to the result of \code{type(name, bases, dict)}.

When the class definition is read, if \var{__metaclass__} is defined
then the callable assigned to it will be called instead of \function{type()}.
The allows classes or functions to be written which monitor or alter the class
creation process:

\begin{itemize}
\item Modifying the class dictionary prior to the class being created.
\item Returning an instance of another class -- essentially performing
the role of a factory function.
\end{itemize}

\begin{datadesc}{__metaclass__}
This variable can be any callable accepting arguments for \code{name},
\code{bases}, and \code{dict}.  Upon class creation, the callable is
used instead of the built-in \function{type()}.
\versionadded{2.2}                     
\end{datadesc}

The appropriate metaclass is determined by the following precedence rules:

\begin{itemize}

\item If \code{dict['__metaclass__']} exists, it is used.

\item Otherwise, if there is at least one base class, its metaclass is used
(this looks for a \var{__class__} attribute first and if not found, uses its
type).

\item Otherwise, if a global variable named __metaclass__ exists, it is used.

\item Otherwise, the old-style, classic metaclass (types.ClassType) is used.

\end{itemize}      

The potential uses for metaclasses are boundless. Some ideas that have
been explored including logging, interface checking, automatic delegation,
automatic property creation, proxies, frameworks, and automatic resource
locking/synchronization.


\subsection{Emulating callable objects\label{callable-types}}

\begin{methoddesc}[object]{__call__}{self\optional{, args...}}
Called when the instance is ``called'' as a function; if this method
is defined, \code{\var{x}(arg1, arg2, ...)} is a shorthand for
\code{\var{x}.__call__(arg1, arg2, ...)}.
\indexii{call}{instance}
\end{methoddesc}


\subsection{Emulating container types\label{sequence-types}}

The following methods can be defined to implement container
objects.  Containers usually are sequences (such as lists or tuples)
or mappings (like dictionaries), but can represent other containers as
well.  The first set of methods is used either to emulate a
sequence or to emulate a mapping; the difference is that for a
sequence, the allowable keys should be the integers \var{k} for which
\code{0 <= \var{k} < \var{N}} where \var{N} is the length of the
sequence, or slice objects, which define a range of items. (For backwards
compatibility, the method \method{__getslice__()} (see below) can also be
defined to handle simple, but not extended slices.) It is also recommended
that mappings provide the methods \method{keys()}, \method{values()},
\method{items()}, \method{has_key()}, \method{get()}, \method{clear()},
\method{setdefault()}, \method{iterkeys()}, \method{itervalues()},
\method{iteritems()}, \method{pop()}, \method{popitem()},		     
\method{copy()}, and \method{update()} behaving similar to those for
Python's standard dictionary objects.  The \module{UserDict} module
provides a \class{DictMixin} class to help create those methods
from a base set of \method{__getitem__()}, \method{__setitem__()},
\method{__delitem__()}, and \method{keys()}.		     
Mutable sequences should provide
methods \method{append()}, \method{count()}, \method{index()},
\method{extend()},		     
\method{insert()}, \method{pop()}, \method{remove()}, \method{reverse()}
and \method{sort()}, like Python standard list objects.  Finally,
sequence types should implement addition (meaning concatenation) and
multiplication (meaning repetition) by defining the methods
\method{__add__()}, \method{__radd__()}, \method{__iadd__()},
\method{__mul__()}, \method{__rmul__()} and \method{__imul__()} described
below; they should not define \method{__coerce__()} or other numerical
operators.  It is recommended that both mappings and sequences
implement the \method{__contains__()} method to allow efficient use of
the \code{in} operator; for mappings, \code{in} should be equivalent
of \method{has_key()}; for sequences, it should search through the
values.  It is further recommended that both mappings and sequences
implement the \method{__iter__()} method to allow efficient iteration
through the container; for mappings, \method{__iter__()} should be
the same as \method{iterkeys()}; for sequences, it should iterate
through the values.
\withsubitem{(mapping object method)}{
  \ttindex{keys()}
  \ttindex{values()}
  \ttindex{items()}
  \ttindex{iterkeys()}
  \ttindex{itervalues()}
  \ttindex{iteritems()}    
  \ttindex{has_key()}
  \ttindex{get()}
  \ttindex{setdefault()}
  \ttindex{pop()}      
  \ttindex{popitem()}    
  \ttindex{clear()}
  \ttindex{copy()}
  \ttindex{update()}
  \ttindex{__contains__()}}
\withsubitem{(sequence object method)}{
  \ttindex{append()}
  \ttindex{count()}
  \ttindex{extend()}    
  \ttindex{index()}
  \ttindex{insert()}
  \ttindex{pop()}
  \ttindex{remove()}
  \ttindex{reverse()}
  \ttindex{sort()}
  \ttindex{__add__()}
  \ttindex{__radd__()}
  \ttindex{__iadd__()}
  \ttindex{__mul__()}
  \ttindex{__rmul__()}
  \ttindex{__imul__()}
  \ttindex{__contains__()}
  \ttindex{__iter__()}}		     
\withsubitem{(numeric object method)}{\ttindex{__coerce__()}}

\begin{methoddesc}[container object]{__len__}{self}
Called to implement the built-in function
\function{len()}\bifuncindex{len}.  Should return the length of the
object, an integer \code{>=} 0.  Also, an object that doesn't define a
\method{__nonzero__()} method and whose \method{__len__()} method
returns zero is considered to be false in a Boolean context.
\withsubitem{(object method)}{\ttindex{__nonzero__()}}
\end{methoddesc}

\begin{methoddesc}[container object]{__getitem__}{self, key}
Called to implement evaluation of \code{\var{self}[\var{key}]}.
For sequence types, the accepted keys should be integers and slice
objects.\obindex{slice}  Note that
the special interpretation of negative indexes (if the class wishes to
emulate a sequence type) is up to the \method{__getitem__()} method.
If \var{key} is of an inappropriate type, \exception{TypeError} may be
raised; if of a value outside the set of indexes for the sequence
(after any special interpretation of negative values),
\exception{IndexError} should be raised.
\note{\keyword{for} loops expect that an
\exception{IndexError} will be raised for illegal indexes to allow
proper detection of the end of the sequence.}
\end{methoddesc}

\begin{methoddesc}[container object]{__setitem__}{self, key, value}
Called to implement assignment to \code{\var{self}[\var{key}]}.  Same
note as for \method{__getitem__()}.  This should only be implemented
for mappings if the objects support changes to the values for keys, or
if new keys can be added, or for sequences if elements can be
replaced.  The same exceptions should be raised for improper
\var{key} values as for the \method{__getitem__()} method.
\end{methoddesc}

\begin{methoddesc}[container object]{__delitem__}{self, key}
Called to implement deletion of \code{\var{self}[\var{key}]}.  Same
note as for \method{__getitem__()}.  This should only be implemented
for mappings if the objects support removal of keys, or for sequences
if elements can be removed from the sequence.  The same exceptions
should be raised for improper \var{key} values as for the
\method{__getitem__()} method.
\end{methoddesc}

\begin{methoddesc}[container object]{__iter__}{self}
This method is called when an iterator is required for a container.
This method should return a new iterator object that can iterate over
all the objects in the container.  For mappings, it should iterate
over the keys of the container, and should also be made available as
the method \method{iterkeys()}.

Iterator objects also need to implement this method; they are required
to return themselves.  For more information on iterator objects, see
``\ulink{Iterator Types}{../lib/typeiter.html}'' in the
\citetitle[../lib/lib.html]{Python Library Reference}.
\end{methoddesc}

The membership test operators (\keyword{in} and \keyword{not in}) are
normally implemented as an iteration through a sequence.  However,
container objects can supply the following special method with a more
efficient implementation, which also does not require the object be a
sequence.

\begin{methoddesc}[container object]{__contains__}{self, item}
Called to implement membership test operators.  Should return true if
\var{item} is in \var{self}, false otherwise.  For mapping objects,
this should consider the keys of the mapping rather than the values or
the key-item pairs.
\end{methoddesc}


\subsection{Additional methods for emulation of sequence types
  \label{sequence-methods}}

The following optional methods can be defined to further emulate sequence
objects.  Immutable sequences methods should at most only define
\method{__getslice__()}; mutable sequences might define all three
methods.

\begin{methoddesc}[sequence object]{__getslice__}{self, i, j}
\deprecated{2.0}{Support slice objects as parameters to the
\method{__getitem__()} method.}
Called to implement evaluation of \code{\var{self}[\var{i}:\var{j}]}.
The returned object should be of the same type as \var{self}.  Note
that missing \var{i} or \var{j} in the slice expression are replaced
by zero or \code{sys.maxint}, respectively.  If negative indexes are
used in the slice, the length of the sequence is added to that index.
If the instance does not implement the \method{__len__()} method, an
\exception{AttributeError} is raised.
No guarantee is made that indexes adjusted this way are not still
negative.  Indexes which are greater than the length of the sequence
are not modified.
If no \method{__getslice__()} is found, a slice
object is created instead, and passed to \method{__getitem__()} instead.
\end{methoddesc}

\begin{methoddesc}[sequence object]{__setslice__}{self, i, j, sequence}
Called to implement assignment to \code{\var{self}[\var{i}:\var{j}]}.
Same notes for \var{i} and \var{j} as for \method{__getslice__()}.

This method is deprecated. If no \method{__setslice__()} is found,
or for extended slicing of the form
\code{\var{self}[\var{i}:\var{j}:\var{k}]}, a
slice object is created, and passed to \method{__setitem__()},
instead of \method{__setslice__()} being called.
\end{methoddesc}

\begin{methoddesc}[sequence object]{__delslice__}{self, i, j}
Called to implement deletion of \code{\var{self}[\var{i}:\var{j}]}.
Same notes for \var{i} and \var{j} as for \method{__getslice__()}.
This method is deprecated. If no \method{__delslice__()} is found,
or for extended slicing of the form
\code{\var{self}[\var{i}:\var{j}:\var{k}]}, a
slice object is created, and passed to \method{__delitem__()},
instead of \method{__delslice__()} being called.
\end{methoddesc}

Notice that these methods are only invoked when a single slice with a
single colon is used, and the slice method is available.  For slice
operations involving extended slice notation, or in absence of the
slice methods, \method{__getitem__()}, \method{__setitem__()} or
\method{__delitem__()} is called with a slice object as argument.

The following example demonstrate how to make your program or module
compatible with earlier versions of Python (assuming that methods
\method{__getitem__()}, \method{__setitem__()} and \method{__delitem__()}
support slice objects as arguments):

\begin{verbatim}
class MyClass:
    ...
    def __getitem__(self, index):
        ...
    def __setitem__(self, index, value):
        ...
    def __delitem__(self, index):
        ...

    if sys.version_info < (2, 0):
        # They won't be defined if version is at least 2.0 final

        def __getslice__(self, i, j):
            return self[max(0, i):max(0, j):]
        def __setslice__(self, i, j, seq):
            self[max(0, i):max(0, j):] = seq
        def __delslice__(self, i, j):
            del self[max(0, i):max(0, j):]
    ...
\end{verbatim}

Note the calls to \function{max()}; these are necessary because of
the handling of negative indices before the
\method{__*slice__()} methods are called.  When negative indexes are
used, the \method{__*item__()} methods receive them as provided, but
the \method{__*slice__()} methods get a ``cooked'' form of the index
values.  For each negative index value, the length of the sequence is
added to the index before calling the method (which may still result
in a negative index); this is the customary handling of negative
indexes by the built-in sequence types, and the \method{__*item__()}
methods are expected to do this as well.  However, since they should
already be doing that, negative indexes cannot be passed in; they must
be constrained to the bounds of the sequence before being passed to
the \method{__*item__()} methods.
Calling \code{max(0, i)} conveniently returns the proper value.


\subsection{Emulating numeric types\label{numeric-types}}

The following methods can be defined to emulate numeric objects.
Methods corresponding to operations that are not supported by the
particular kind of number implemented (e.g., bitwise operations for
non-integral numbers) should be left undefined.

\begin{methoddesc}[numeric object]{__add__}{self, other}
\methodline[numeric object]{__sub__}{self, other}
\methodline[numeric object]{__mul__}{self, other}
\methodline[numeric object]{__floordiv__}{self, other}
\methodline[numeric object]{__mod__}{self, other}
\methodline[numeric object]{__divmod__}{self, other}
\methodline[numeric object]{__pow__}{self, other\optional{, modulo}}
\methodline[numeric object]{__lshift__}{self, other}
\methodline[numeric object]{__rshift__}{self, other}
\methodline[numeric object]{__and__}{self, other}
\methodline[numeric object]{__xor__}{self, other}
\methodline[numeric object]{__or__}{self, other}
These methods are
called to implement the binary arithmetic operations (\code{+},
\code{-}, \code{*}, \code{//}, \code{\%},
\function{divmod()}\bifuncindex{divmod},
\function{pow()}\bifuncindex{pow}, \code{**}, \code{<}\code{<},
\code{>}\code{>}, \code{\&}, \code{\^}, \code{|}).  For instance, to
evaluate the expression \var{x}\code{+}\var{y}, where \var{x} is an
instance of a class that has an \method{__add__()} method,
\code{\var{x}.__add__(\var{y})} is called.  The \method{__divmod__()}
method should be the equivalent to using \method{__floordiv__()} and
\method{__mod__()}; it should not be related to \method{__truediv__()}
(described below).  Note that
\method{__pow__()} should be defined to accept an optional third
argument if the ternary version of the built-in
\function{pow()}\bifuncindex{pow} function is to be supported.
\end{methoddesc}

\begin{methoddesc}[numeric object]{__div__}{self, other}
\methodline[numeric object]{__truediv__}{self, other}
The division operator (\code{/}) is implemented by these methods.  The
\method{__truediv__()} method is used when \code{__future__.division}
is in effect, otherwise \method{__div__()} is used.  If only one of
these two methods is defined, the object will not support division in
the alternate context; \exception{TypeError} will be raised instead.
\end{methoddesc}

\begin{methoddesc}[numeric object]{__radd__}{self, other}
\methodline[numeric object]{__rsub__}{self, other}
\methodline[numeric object]{__rmul__}{self, other}
\methodline[numeric object]{__rdiv__}{self, other}
\methodline[numeric object]{__rtruediv__}{self, other}
\methodline[numeric object]{__rfloordiv__}{self, other}	     
\methodline[numeric object]{__rmod__}{self, other}
\methodline[numeric object]{__rdivmod__}{self, other}
\methodline[numeric object]{__rpow__}{self, other}
\methodline[numeric object]{__rlshift__}{self, other}
\methodline[numeric object]{__rrshift__}{self, other}
\methodline[numeric object]{__rand__}{self, other}
\methodline[numeric object]{__rxor__}{self, other}
\methodline[numeric object]{__ror__}{self, other}
These methods are
called to implement the binary arithmetic operations (\code{+},
\code{-}, \code{*}, \code{/}, \code{\%},
\function{divmod()}\bifuncindex{divmod},
\function{pow()}\bifuncindex{pow}, \code{**}, \code{<}\code{<},
\code{>}\code{>}, \code{\&}, \code{\^}, \code{|}) with reflected
(swapped) operands.  These functions are only called if the left
operand does not support the corresponding operation.  For instance,
to evaluate the expression \var{x}\code{-}\var{y}, where \var{y} is an
instance of a class that has an \method{__rsub__()} method,
\code{\var{y}.__rsub__(\var{x})} is called.  Note that ternary
\function{pow()}\bifuncindex{pow} will not try calling
\method{__rpow__()} (the coercion rules would become too
complicated).
\end{methoddesc}

\begin{methoddesc}[numeric object]{__iadd__}{self, other}
\methodline[numeric object]{__isub__}{self, other}
\methodline[numeric object]{__imul__}{self, other}
\methodline[numeric object]{__idiv__}{self, other}
\methodline[numeric object]{__itruediv__}{self, other}
\methodline[numeric object]{__ifloordiv__}{self, other}
\methodline[numeric object]{__imod__}{self, other}		     
\methodline[numeric object]{__ipow__}{self, other\optional{, modulo}}
\methodline[numeric object]{__ilshift__}{self, other}
\methodline[numeric object]{__irshift__}{self, other}
\methodline[numeric object]{__iand__}{self, other}
\methodline[numeric object]{__ixor__}{self, other}
\methodline[numeric object]{__ior__}{self, other}
These methods are called to implement the augmented arithmetic
operations (\code{+=}, \code{-=}, \code{*=}, \code{/=}, \code{\%=},
\code{**=}, \code{<}\code{<=}, \code{>}\code{>=}, \code{\&=},
\code{\textasciicircum=}, \code{|=}).  These methods should attempt to do the
operation in-place (modifying \var{self}) and return the result (which
could be, but does not have to be, \var{self}).  If a specific method
is not defined, the augmented operation falls back to the normal
methods.  For instance, to evaluate the expression
\var{x}\code{+=}\var{y}, where \var{x} is an instance of a class that
has an \method{__iadd__()} method, \code{\var{x}.__iadd__(\var{y})} is
called.  If \var{x} is an instance of a class that does not define a
\method{__iadd()} method, \code{\var{x}.__add__(\var{y})} and
\code{\var{y}.__radd__(\var{x})} are considered, as with the
evaluation of \var{x}\code{+}\var{y}.
\end{methoddesc}

\begin{methoddesc}[numeric object]{__neg__}{self}
\methodline[numeric object]{__pos__}{self}
\methodline[numeric object]{__abs__}{self}
\methodline[numeric object]{__invert__}{self}
Called to implement the unary arithmetic operations (\code{-},
\code{+}, \function{abs()}\bifuncindex{abs} and \code{\~{}}).
\end{methoddesc}

\begin{methoddesc}[numeric object]{__complex__}{self}
\methodline[numeric object]{__int__}{self}
\methodline[numeric object]{__long__}{self}
\methodline[numeric object]{__float__}{self}
Called to implement the built-in functions
\function{complex()}\bifuncindex{complex},
\function{int()}\bifuncindex{int}, \function{long()}\bifuncindex{long},
and \function{float()}\bifuncindex{float}.  Should return a value of
the appropriate type.
\end{methoddesc}

\begin{methoddesc}[numeric object]{__oct__}{self}
\methodline[numeric object]{__hex__}{self}
Called to implement the built-in functions
\function{oct()}\bifuncindex{oct} and
\function{hex()}\bifuncindex{hex}.  Should return a string value.
\end{methoddesc}

\begin{methoddesc}[numeric object]{__coerce__}{self, other}
Called to implement ``mixed-mode'' numeric arithmetic.  Should either
return a 2-tuple containing \var{self} and \var{other} converted to
a common numeric type, or \code{None} if conversion is impossible.  When
the common type would be the type of \code{other}, it is sufficient to
return \code{None}, since the interpreter will also ask the other
object to attempt a coercion (but sometimes, if the implementation of
the other type cannot be changed, it is useful to do the conversion to
the other type here).  A return value of \code{NotImplemented} is
equivalent to returning \code{None}.
\end{methoddesc}

\subsection{Coercion rules\label{coercion-rules}}

This section used to document the rules for coercion.  As the language
has evolved, the coercion rules have become hard to document
precisely; documenting what one version of one particular
implementation does is undesirable.  Instead, here are some informal
guidelines regarding coercion.  In Python 3.0, coercion will not be
supported.

\begin{itemize}

\item

If the left operand of a \% operator is a string or Unicode object, no
coercion takes place and the string formatting operation is invoked
instead.

\item

It is no longer recommended to define a coercion operation.
Mixed-mode operations on types that don't define coercion pass the
original arguments to the operation.

\item

New-style classes (those derived from \class{object}) never invoke the
\method{__coerce__()} method in response to a binary operator; the only
time \method{__coerce__()} is invoked is when the built-in function
\function{coerce()} is called.

\item

For most intents and purposes, an operator that returns
\code{NotImplemented} is treated the same as one that is not
implemented at all.

\item

Below, \method{__op__()} and \method{__rop__()} are used to signify
the generic method names corresponding to an operator;
\method{__iop__} is used for the corresponding in-place operator.  For
example, for the operator `\code{+}', \method{__add__()} and
\method{__radd__()} are used for the left and right variant of the
binary operator, and \method{__iadd__} for the in-place variant.

\item

For objects \var{x} and \var{y}, first \code{\var{x}.__op__(\var{y})}
is tried.  If this is not implemented or returns \code{NotImplemented},
\code{\var{y}.__rop__(\var{x})} is tried.  If this is also not
implemented or returns \code{NotImplemented}, a \exception{TypeError}
exception is raised.  But see the following exception:

\item

Exception to the previous item: if the left operand is an instance of
a built-in type or a new-style class, and the right operand is an
instance of a proper subclass of that type or class, the right
operand's \method{__rop__()} method is tried \emph{before} the left
operand's \method{__op__()} method.  This is done so that a subclass can
completely override binary operators.  Otherwise, the left operand's
__op__ method would always accept the right operand: when an instance
of a given class is expected, an instance of a subclass of that class
is always acceptable.

\item

When either operand type defines a coercion, this coercion is called
before that type's \method{__op__()} or \method{__rop__()} method is
called, but no sooner.  If the coercion returns an object of a
different type for the operand whose coercion is invoked, part of the
process is redone using the new object.

\item

When an in-place operator (like `\code{+=}') is used, if the left
operand implements \method{__iop__()}, it is invoked without any
coercion.  When the operation falls back to \method{__op__()} and/or
\method{__rop__()}, the normal coercion rules apply.

\item

In \var{x}\code{+}\var{y}, if \var{x} is a sequence that implements
sequence concatenation, sequence concatenation is invoked.

\item

In \var{x}\code{*}\var{y}, if one operator is a sequence that
implements sequence repetition, and the other is an integer
(\class{int} or \class{long}), sequence repetition is invoked.

\item

Rich comparisons (implemented by methods \method{__eq__()} and so on)
never use coercion.  Three-way comparison (implemented by
\method{__cmp__()}) does use coercion under the same conditions as
other binary operations use it.

\item

In the current implementation, the built-in numeric types \class{int},
\class{long} and \class{float} do not use coercion; the type
\class{complex} however does use it.  The difference can become
apparent when subclassing these types.  Over time, the type
\class{complex} may be fixed to avoid coercion.  All these types
implement a \method{__coerce__()} method, for use by the built-in
\function{coerce()} function.

\end{itemize}
