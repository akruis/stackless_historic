\chapter{Data model\label{datamodel}}

\section{Objects, values and types\label{objects}}

\dfn{Objects} are Python's abstraction for data.  All data in a Python
program is represented by objects or by relations between objects.
(In a sense, and in conformance to Von Neumann's model of a
``stored program computer,'' code is also represented by objects.)
\index{object}
\index{data}

Every object has an identity, a type and a value.  An object's
\emph{identity} never changes once it has been created; you may think
of it as the object's address in memory.  The `\code{is}' operator
compares the identity of two objects; the
\function{id()}\bifuncindex{id} function returns an integer
representing its identity (currently implemented as its address).
An object's \dfn{type} is
also unchangeable.  It determines the operations that an object
supports (e.g., ``does it have a length?'') and also defines the
possible values for objects of that type.  The
\function{type()}\bifuncindex{type} function returns an object's type
(which is an object itself).  The \emph{value} of some
objects can change.  Objects whose value can change are said to be
\emph{mutable}; objects whose value is unchangeable once they are
created are called \emph{immutable}.
An object's mutability is determined by its type; for instance,
numbers, strings and tuples are immutable, while dictionaries and
lists are mutable.
\index{identity of an object}
\index{value of an object}
\index{type of an object}
\index{mutable object}
\index{immutable object}

Objects are never explicitly destroyed; however, when they become
unreachable they may be garbage-collected.  An implementation is
allowed to postpone garbage collection or omit it altogether --- it is
a matter of implementation quality how garbage collection is
implemented, as long as no objects are collected that are still
reachable.  (Implementation note: the current implementation uses a
reference-counting scheme which collects most objects as soon as they
become unreachable, but never collects garbage containing circular
references.)
\index{garbage collection}
\index{reference counting}
\index{unreachable object}

Note that the use of the implementation's tracing or debugging
facilities may keep objects alive that would normally be collectable.
Also note that catching an exception with a
`\code{try}...\code{except}' statement may keep objects alive.

Some objects contain references to ``external'' resources such as open
files or windows.  It is understood that these resources are freed
when the object is garbage-collected, but since garbage collection is
not guaranteed to happen, such objects also provide an explicit way to
release the external resource, usually a \method{close()} method.
Programs are strongly recommended to explicitly close such
objects.
The `\code{try}...\code{finally}' statement provides a convenient way
to do this.

Some objects contain references to other objects; these are called
\emph{containers}.  Examples of containers are tuples, lists and
dictionaries.  The references are part of a container's value.  In
most cases, when we talk about the value of a container, we imply the
values, not the identities of the contained objects; however, when we
talk about the mutability of a container, only the identities of
the immediately contained objects are implied.  So, if an immutable
container (like a tuple)
contains a reference to a mutable object, its value changes
if that mutable object is changed.
\index{container}

Types affect almost all aspects of object behavior.  Even the importance
of object identity is affected in some sense: for immutable types,
operations that compute new values may actually return a reference to
any existing object with the same type and value, while for mutable
objects this is not allowed.  E.g., after
\samp{a = 1; b = 1},
\code{a} and \code{b} may or may not refer to the same object with the
value one, depending on the implementation, but after
\samp{c = []; d = []}, \code{c} and \code{d}
are guaranteed to refer to two different, unique, newly created empty
lists.
(Note that \samp{c = d = []} assigns the same object to both
\code{c} and \code{d}.)

\section{The standard type hierarchy\label{types}}

Below is a list of the types that are built into Python.  Extension
modules written in \C{} can define additional types.  Future versions of
Python may add types to the type hierarchy (e.g., rational
numbers, efficiently stored arrays of integers, etc.).
\index{type}
\indexii{data}{type}
\indexii{type}{hierarchy}
\indexii{extension}{module}
\indexii{C}{language}

Some of the type descriptions below contain a paragraph listing
`special attributes.'  These are attributes that provide access to the
implementation and are not intended for general use.  Their definition
may change in the future.  There are also some `generic' special
attributes, not listed with the individual objects: \member{__methods__}
is a list of the method names of a built-in object, if it has any;
\member{__members__} is a list of the data attribute names of a built-in
object, if it has any.
\index{attribute}
\indexii{special}{attribute}
\indexiii{generic}{special}{attribute}
\withsubitem{(built-in object attribute)}{%
  \ttindex{__methods__}
  \ttindex{__members__}}

\begin{description}

\item[None]
This type has a single value.  There is a single object with this value.
This object is accessed through the built-in name \code{None}.
It is used to signify the absence of a value in many situations, e.g.,
it is returned from functions that don't explicitly return anything.
Its truth value is false.
\ttindex{None}
\obindex{None@{\tt None}}

\item[Ellipsis]
This type has a single value.  There is a single object with this value.
This object is accessed through the built-in name \code{Ellipsis}.
It is used to indicate the presence of the \samp{...} syntax in a
slice.  Its truth value is true.
\ttindex{Ellipsis}
\obindex{Ellipsis@{\tt Ellipsis}}

\item[Numbers]
These are created by numeric literals and returned as results by
arithmetic operators and arithmetic built-in functions.  Numeric
objects are immutable; once created their value never changes.  Python
numbers are of course strongly related to mathematical numbers, but
subject to the limitations of numerical representation in computers.
\obindex{number}
\obindex{numeric}

Python distinguishes between integers and floating point numbers:

\begin{description}
\item[Integers]
These represent elements from the mathematical set of whole numbers.
\obindex{integer}

There are two types of integers:

\begin{description}

\item[Plain integers]
These represent numbers in the range -2147483648 through 2147483647.
(The range may be larger on machines with a larger natural word
size, but not smaller.)
When the result of an operation falls outside this range, the
exception \exception{OverflowError} is raised.
For the purpose of shift and mask operations, integers are assumed to
have a binary, 2's complement notation using 32 or more bits, and
hiding no bits from the user (i.e., all 4294967296 different bit
patterns correspond to different values).
\obindex{plain integer}
\withsubitem{(built-in exception)}{\ttindex{OverflowError}}

\item[Long integers]
These represent numbers in an unlimited range, subject to available
(virtual) memory only.  For the purpose of shift and mask operations,
a binary representation is assumed, and negative numbers are
represented in a variant of 2's complement which gives the illusion of
an infinite string of sign bits extending to the left.
\obindex{long integer}

\end{description} % Integers

The rules for integer representation are intended to give the most
meaningful interpretation of shift and mask operations involving
negative integers and the least surprises when switching between the
plain and long integer domains.  For any operation except left shift,
if it yields a result in the plain integer domain without causing
overflow, it will yield the same result in the long integer domain or
when using mixed operands.
\indexii{integer}{representation}

\item[Floating point numbers]
These represent machine-level double precision floating point numbers.  
You are at the mercy of the underlying machine architecture and
\C{} implementation for the accepted range and handling of overflow.
Python does not support single-precision floating point numbers; the
savings in CPU and memory usage that are usually the reason for using
these is dwarfed by the overhead of using objects in Python, so there
is no reason to complicate the language with two kinds of floating
point numbers.
\obindex{floating point}
\indexii{floating point}{number}
\indexii{C}{language}

\item[Complex numbers]
These represent complex numbers as a pair of machine-level double
precision floating point numbers.  The same caveats apply as for
floating point numbers.  The real and imaginary value of a complex
number \code{z} can be retrieved through the attributes \code{z.real}
and \code{z.imag}.
\obindex{complex}
\indexii{complex}{number}

\end{description} % Numbers

\item[Sequences]
These represent finite ordered sets indexed by natural numbers.
The built-in function \function{len()}\bifuncindex{len} returns the
number of items of a sequence.
When the lenth of a sequence is \var{n}, the
index set contains the numbers 0, 1, \ldots, \var{n}-1.  Item
\var{i} of sequence \var{a} is selected by \code{\var{a}[\var{i}]}.
\obindex{seqence}
\index{index operation}
\index{item selection}
\index{subscription}

Sequences also support slicing: \code{\var{a}[\var{i}:\var{j}]}
selects all items with index \var{k} such that \var{i} \code{<=}
\var{k} \code{<} \var{j}.  When used as an expression, a slice is a
sequence of the same type.  This implies that the index set is
renumbered so that it starts at 0.
\index{slicing}

Sequences are distinguished according to their mutability:

\begin{description}
%
\item[Immutable sequences]
An object of an immutable sequence type cannot change once it is
created.  (If the object contains references to other objects,
these other objects may be mutable and may be changed; however,
the collection of objects directly referenced by an immutable object
cannot change.)
\obindex{immutable sequence}
\obindex{immutable}

The following types are immutable sequences:

\begin{description}

\item[Strings]
The items of a string are characters.  There is no separate
character type; a character is represented by a string of one item.
Characters represent (at least) 8-bit bytes.  The built-in
functions \function{chr()}\bifuncindex{chr} and
\function{ord()}\bifuncindex{ord} convert between characters and
nonnegative integers representing the byte values.  Bytes with the
values 0-127 usually represent the corresponding \ASCII{} values, but
the interpretation of values is up to the program.  The string
data type is also used to represent arrays of bytes, e.g., to hold data
read from a file.
\obindex{string}
\index{character}
\index{byte}
\index{ASCII@\ASCII{}}

(On systems whose native character set is not \ASCII{}, strings may use
EBCDIC in their internal representation, provided the functions
\function{chr()} and \function{ord()} implement a mapping between \ASCII{} and
EBCDIC, and string comparison preserves the \ASCII{} order.
Or perhaps someone can propose a better rule?)
\index{ASCII@\ASCII{}}
\index{EBCDIC}
\index{character set}
\indexii{string}{comparison}
\bifuncindex{chr}
\bifuncindex{ord}

\item[Tuples]
The items of a tuple are arbitrary Python objects.
Tuples of two or more items are formed by comma-separated lists
of expressions.  A tuple of one item (a `singleton') can be formed
by affixing a comma to an expression (an expression by itself does
not create a tuple, since parentheses must be usable for grouping of
expressions).  An empty tuple can be formed by an empty pair of
parentheses.
\obindex{tuple}
\indexii{singleton}{tuple}
\indexii{empty}{tuple}

\end{description} % Immutable sequences

\item[Mutable sequences]
Mutable sequences can be changed after they are created.  The
subscription and slicing notations can be used as the target of
assignment and \keyword{del} (delete) statements.
\obindex{mutable sequece}
\obindex{mutable}
\indexii{assignment}{statement}
\index{delete}
\stindex{del}
\index{subscription}
\index{slicing}

There is currently a single mutable sequence type:

\begin{description}

\item[Lists]
The items of a list are arbitrary Python objects.  Lists are formed
by placing a comma-separated list of expressions in square brackets.
(Note that there are no special cases needed to form lists of length 0
or 1.)
\obindex{list}

\end{description} % Mutable sequences

The extension module \module{array}\refstmodindex{array} provides an
additional example of a mutable sequence type.


\end{description} % Sequences

\item[Mappings]
These represent finite sets of objects indexed by arbitrary index sets.
The subscript notation \code{a[k]} selects the item indexed
by \code{k} from the mapping \code{a}; this can be used in
expressions and as the target of assignments or \keyword{del} statements.
The built-in function \function{len()} returns the number of items
in a mapping.
\bifuncindex{len}
\index{subscription}
\obindex{mapping}

There is currently a single intrinsic mapping type:

\begin{description}

\item[Dictionaries]
These represent finite sets of objects indexed by nearly arbitrary
values.  The only types of values not acceptable as keys are values
containing lists or dictionaries or other mutable types that are
compared by value rather than by object identity, the reason being
that the efficient implementation of dictionaries requires a key's
hash value to remain constant.
Numeric types used for keys obey the normal rules for numeric
comparison: if two numbers compare equal (e.g., \code{1} and
\code{1.0}) then they can be used interchangeably to index the same
dictionary entry.

Dictionaries are mutable; they are created by the \code{...}
notation (see section \ref{dict}, ``Dictionary Displays'').
\obindex{dictionary}
\obindex{mutable}

The extension modules \module{dbm}\refstmodindex{dbm},
\module{gdbm}\refstmodindex{gdbm}, \module{bsddb}\refstmodindex{bsddb}
provide additional examples of mapping types.

\end{description} % Mapping types

\item[Callable types]
These are the types to which the function call operation (see section
\ref{calls}, ``Calls'') can be applied:
\indexii{function}{call}
\index{invocation}
\indexii{function}{argument}
\obindex{callable}

\begin{description}

\item[User-defined functions]
A user-defined function object is created by a function definition
(see section \ref{function}, ``Function definitions'').  It should be
called with an argument
list containing the same number of items as the function's formal
parameter list.
\indexii{user-defined}{function}
\obindex{function}
\obindex{user-defined function}

Special read-only attributes: \member{func_doc} or \member{__doc__} is the
function's documentation string, or None if unavailable;
\member{func_name} or \member{__name__} is the function's name;
\member{func_defaults} is a tuple containing default argument values for
those arguments that have defaults, or \code{None} if no arguments
have a default value; \member{func_code} is the code object representing
the compiled function body; \member{func_globals} is (a reference to)
the dictionary that holds the function's global variables --- it
defines the global namespace of the module in which the function was
defined.  Additional information about a function's definition can be
retrieved from its code object; see the description of internal types
below.
\withsubitem{(function attribute)}{%
  \ttindex{func_doc}%
  \ttindex{__doc__}%
  \ttindex{__name__}%
  \ttindex{func_defaults}%
  \ttindex{func_code}%
  \ttindex{func_globals}}
\indexii{global}{namespace}

\item[User-defined methods]
A user-defined method object combines a class, a class instance (or
\code{None}) and a user-defined function.
\obindex{method}
\obindex{user-defined method}
\indexii{user-defined}{method}

Special read-only attributes: \member{im_self} is the class instance
object, \member{im_func} is the function object;
\member{im_class} is the class that defined the method (which may be a
base class of the class of which \member{im_self} is an instance);
\member{__doc__} is the method's documentation (same as
\code{im_func.__doc__}); \member{__name__} is the method name (same as
\code{im_func.__name__}).
\withsubitem{(method attribute)}{%
  \ttindex{im_func}%
  \ttindex{im_self}}

User-defined method objects are created in two ways: when getting an
attribute of a class that is a user-defined function object, or when
getting an attributes of a class instance that is a user-defined
function object.  In the former case (class attribute), the
\member{im_self} attribute is \code{None}, and the method object is said
to be unbound; in the latter case (instance attribute), \method{im_self}
is the instance, and the method object is said to be bound.  For
instance, when \class{C} is a class which contains a definition for a
function \method{f()}, \code{C.f} does not yield the function object
\code{f}; rather, it yields an unbound method object \code{m} where
\code{m.im_class} is \class{C}, \code{m.im_func} is \method{f()}, and
\code{m.im_self} is \code{None}.  When \code{x} is a \class{C}
instance, \code{x.f} yields a bound method object \code{m} where
\code{m.im_class} is \code{C}, \code{m.im_func} is \method{f()}, and
\code{m.im_self} is \code{x}.
\withsubitem{(method attribute)}{%
  \ttindex{im_class}%
  \ttindex{im_func}%
  \ttindex{im_self}}

When an unbound user-defined method object is called, the underlying
function (\member{im_func}) is called, with the restriction that the
first argument must be an instance of the proper class
(\member{im_class}) or of a derived class thereof.

When a bound user-defined method object is called, the underlying
function (\member{im_func}) is called, inserting the class instance
(\member{im_self}) in front of the argument list.  For instance, when
\class{C} is a class which contains a definition for a function
\method{f()}, and \code{x} is an instance of \class{C}, calling
\code{x.f(1)} is equivalent to calling \code{C.f(x, 1)}.

Note that the transformation from function object to (unbound or
bound) method object happens each time the attribute is retrieved from
the class or instance.  In some cases, a fruitful optimization is to
assign the attribute to a local variable and call that local variable.
Also notice that this transformation only happens for user-defined
functions; other callable objects (and all non-callable objects) are
retrieved without transformation.

\item[Built-in functions]
A built-in function object is a wrapper around a \C{} function.  Examples
of built-in functions are \function{len()} and \function{math.sin()}
(\module{math} is a standard built-in module).
The number and type of the arguments are
determined by the C function.
Special read-only attributes: \member{__doc__} is the function's
documentation string, or \code{None} if unavailable; \member{__name__}
is the function's name; \member{__self__} is set to \code{None} (but see
the next item).
\obindex{built-in function}
\obindex{function}
\indexii{C}{language}

\item[Built-in methods]
This is really a different disguise of a built-in function, this time
containing an object passed to the \C{} function as an implicit extra
argument.  An example of a built-in method is
\code{\var{list}.append()}, assuming
\var{list} is a list object.
In this case, the special read-only attribute \member{__self__} is set
to the object denoted by \code{list}.
\obindex{built-in method}
\obindex{method}
\indexii{built-in}{method}

\item[Classes]
Class objects are described below.  When a class object is called,
a new class instance (also described below) is created and
returned.  This implies a call to the class's \method{__init__()} method
if it has one.  Any arguments are passed on to the \method{__init__()}
method.  If there is no \method{__init__()} method, the class must be called
without arguments.
\withsubitem{(object method)}{\ttindex{__init__()}}
\obindex{class}
\obindex{class instance}
\obindex{instance}
\indexii{class object}{call}

\item[Class instances]
Class instances are described below.  Class instances are callable
only when the class has a \method{__call__()} method; \code{x(arguments)}
is a shorthand for \code{x.__call__(arguments)}.

\end{description}

\item[Modules]
Modules are imported by the \keyword{import} statement (see section
\ref{import}, ``The \keyword{import} statement'').
A module object has a namespace implemented by a dictionary object
(this is the dictionary referenced by the func_globals attribute of
functions defined in the module).  Attribute references are translated
to lookups in this dictionary, e.g., \code{m.x} is equivalent to
\code{m.__dict__["x"]}.
A module object does not contain the code object used to
initialize the module (since it isn't needed once the initialization
is done).
\stindex{import}
\obindex{module}

Attribute assignment updates the module's namespace dictionary,
e.g., \samp{m.x = 1} is equivalent to \samp{m.__dict__["x"] = 1}.

Special read-only attribute: \member{__dict__} is the module's
namespace as a dictionary object.
\withsubitem{(module attribute)}{\ttindex{__dict__}}

Predefined (writable) attributes: \member{__name__}
is the module's name; \member{__doc__} is the
module's documentation string, or
\code{None} if unavailable; \member{__file__} is the pathname of the
file from which the module was loaded, if it was loaded from a file.
The \member{__file__} attribute is not present for C{} modules that are
statically linked into the interpreter; for extension modules loaded
dynamically from a shared library, it is the pathname of the shared
library file.
\withsubitem{(module attribute)}{%
  \ttindex{__name__}%
  \ttindex{__doc__}%
  \ttindex{__file__}}
\indexii{module}{namespace}

\item[Classes]
Class objects are created by class definitions (see section
\ref{class}, ``Class definitions'').
A class has a namespace implemented by a dictionary object.
Class attribute references are translated to
lookups in this dictionary,
e.g., \samp{C.x} is translated to \samp{C.__dict__["x"]}.
When the attribute name is not found
there, the attribute search continues in the base classes.  The search
is depth-first, left-to-right in the order of occurrence in the
base class list.
When a class attribute reference would yield a user-defined function
object, it is transformed into an unbound user-defined method object
(see above).  The \member{im_class} attribute of this method object is the
class in which the function object was found, not necessarily the
class for which the attribute reference was initiated.
\obindex{class}
\obindex{class instance}
\obindex{instance}
\indexii{class object}{call}
\index{container}
\obindex{dictionary}
\indexii{class}{attribute}

Class attribute assignments update the class's dictionary, never the
dictionary of a base class.
\indexiii{class}{attribute}{assignment}

A class object can be called (see above) to yield a class instance (see
below).
\indexii{class object}{call}

Special attributes: \member{__name__} is the class name;
\member{__module__} is the module name in which the class was defined;
\member{__dict__} is the dictionary containing the class's namespace;
\member{__bases__} is a tuple (possibly empty or a singleton)
containing the base classes, in the order of their occurrence in the
base class list; \member{__doc__} is the class's documentation string,
or None if undefined.
\withsubitem{(class attribute)}{%
  \ttindex{__name__}%
  \ttindex{__module__}%
  \ttindex{__dict__}%
  \ttindex{__bases__}%
  \ttindex{__doc__}}

\item[Class instances]
A class instance is created by calling a class object (see above).
A class instance has a namespace implemented as a dictionary which
is the first place in which
attribute references are searched.  When an attribute is not found
there, and the instance's class has an attribute by that name,
the search continues with the class attributes.  If a class attribute
is found that is a user-defined function object (and in no other
case), it is transformed into an unbound user-defined method object
(see above).  The \member{im_class} attribute of this method object is
the class in which the function object was found, not necessarily the
class of the instance for which the attribute reference was initiated.
If no class attribute is found, and the object's class has a
\method{__getattr__()} method, that is called to satisfy the lookup.
\obindex{class instance}
\obindex{instance}
\indexii{class}{instance}
\indexii{class instance}{attribute}

Attribute assignments and deletions update the instance's dictionary,
never a class's dictionary.  If the class has a \method{__setattr__()} or
\method{__delattr__()} method, this is called instead of updating the
instance dictionary directly.
\indexiii{class instance}{attribute}{assignment}

Class instances can pretend to be numbers, sequences, or mappings if
they have methods with certain special names.  See
section \ref{specialnames}, ``Special method names.''
\obindex{number}
\obindex{sequence}
\obindex{mapping}

Special attributes: \member{__dict__} is the attribute
dictionary; \member{__class__} is the instance's class.
\withsubitem{(instance attribute)}{%
  \ttindex{__dict__}%
  \ttindex{__class__}}

\item[Files]
A file object represents an open file.  File objects are created by the
\function{open()} built-in function, and also by
\function{os.popen()}, \function{os.fdopen()}, and the
\method{makefile()} method of socket objects (and perhaps by other
functions or methods provided by extension modules).  The objects
\code{sys.stdin}, \code{sys.stdout} and \code{sys.stderr} are
initialized to file objects corresponding to the interpreter's
standard input, output and error streams.  See the \emph{Python
Library Reference} for complete documentation of file objects.
\obindex{file}
\indexii{C}{language}
\index{stdio}
\bifuncindex{open}
\withsubitem{(in module os)}{\ttindex{popen()}}
\withsubitem{(socket method)}{\ttindex{makefile()}}
\withsubitem{(in module sys)}{%
  \ttindex{stdin}%
  \ttindex{stdout}%
  \ttindex{stderr}}
\ttindex{sys.stdin}
\ttindex{sys.stdout}
\ttindex{sys.stderr}

\item[Internal types]
A few types used internally by the interpreter are exposed to the user.
Their definitions may change with future versions of the interpreter,
but they are mentioned here for completeness.
\index{internal type}
\index{types, internal}

\begin{description}

\item[Code objects]
Code objects represent \emph{byte-compiled} executable Python code, or 
\emph{bytecode}.
The difference between a code
object and a function object is that the function object contains an
explicit reference to the function's globals (the module in which it
was defined), while a code object contains no context; 
also the default argument values are stored in the function object,
not in the code object (because they represent values calculated at
run-time).  Unlike function objects, code objects are immutable and
contain no references (directly or indirectly) to mutable objects.
\index{bytecode}
\obindex{code}

Special read-only attributes: \member{co_name} gives the function
name; \member{co_argcount} is the number of positional arguments
(including arguments with default values); \member{co_nlocals} is the
number of local variables used by the function (including arguments);
\member{co_varnames} is a tuple containing the names of the local
variables (starting with the argument names); \member{co_code} is a
string representing the sequence of bytecode instructions;
\member{co_consts} is a tuple containing the literals used by the
bytecode; \member{co_names} is a tuple containing the names used by
the bytecode; \member{co_filename} is the filename from which the code
was compiled; \member{co_firstlineno} is the first line number of the
function; \member{co_lnotab} is a string encoding the mapping from
byte code offsets to line numbers (for detais see the source code of
the interpreter); \member{co_stacksize} is the required stack size
(including local variables); \member{co_flags} is an integer encoding
a number of flags for the interpreter.
\withsubitem{(code object attribute)}{%
  \ttindex{co_argcount}%
  \ttindex{co_code}%
  \ttindex{co_consts}%
  \ttindex{co_filename}%
  \ttindex{co_firstlineno}%
  \ttindex{co_flags}%
  \ttindex{co_lnotab}%
  \ttindex{co_name}%
  \ttindex{co_names}%
  \ttindex{co_nlocals}%
  \ttindex{co_stacksize}%
  \ttindex{co_varnames}}

The following flag bits are defined for \member{co_flags}: bit 2 is set
if the function uses the \samp{*arguments} syntax to accept an
arbitrary number of positional arguments; bit 3 is set if the function
uses the \samp{**keywords} syntax to accept arbitrary keyword
arguments; other bits are used internally or reserved for future use.
If a code object represents a function, the first item in
\member{co_consts} is the documentation string of the
function, or \code{None} if undefined.
\index{documentation string}

\item[Frame objects]
Frame objects represent execution frames.  They may occur in traceback
objects (see below).
\obindex{frame}

Special read-only attributes: \member{f_back} is to the previous
stack frame (towards the caller), or \code{None} if this is the bottom
stack frame; \member{f_code} is the code object being executed in this
frame; \member{f_locals} is the dictionary used to look up local
variables; \member{f_globals} is used for global variables;
\member{f_builtins} is used for built-in (intrinsic) names;
\member{f_restricted} is a flag indicating whether the function is
executing in restricted execution mode;
\member{f_lineno} gives the line number and \member{f_lasti} gives the
precise instruction (this is an index into the bytecode string of
the code object).
\withsubitem{(frame attribute)}{%
  \ttindex{f_back}%
  \ttindex{f_code}%
  \ttindex{f_globals}%
  \ttindex{f_locals}%
  \ttindex{f_lineno}%
  \ttindex{f_lasti}%
  \ttindex{f_builtins}%
  \ttindex{f_restricted}}

Special writable attributes: \member{f_trace}, if not \code{None}, is a
function called at the start of each source code line (this is used by
the debugger); \member{f_exc_type}, \member{f_exc_value},
\member{f_exc_traceback} represent the most recent exception caught in
this frame.
\withsubitem{(frame attribute)}{%
  \ttindex{f_trace}%
  \ttindex{f_exc_type}%
  \ttindex{f_exc_value}%
  \ttindex{f_exc_traceback}}

\item[Traceback objects] \label{traceback}
Traceback objects represent a stack trace of an exception.  A
traceback object is created when an exception occurs.  When the search
for an exception handler unwinds the execution stack, at each unwound
level a traceback object is inserted in front of the current
traceback.  When an exception handler is entered, the stack trace is
made available to the program.
(See section \ref{try}, ``The \code{try} statement.'')
It is accessible as \code{sys.exc_traceback}, and also as the third
item of the tuple returned by \code{sys.exc_info()}.  The latter is
the preferred interface, since it works correctly when the program is
using multiple threads.
When the program contains no suitable handler, the stack trace is written
(nicely formatted) to the standard error stream; if the interpreter is
interactive, it is also made available to the user as
\code{sys.last_traceback}.
\obindex{traceback}
\indexii{stack}{trace}
\indexii{exception}{handler}
\indexii{execution}{stack}
\withsubitem{(in module sys)}{%
  \ttindex{exc_info}%
  \ttindex{exc_traceback}%
  \ttindex{last_traceback}}
\ttindex{sys.exc_info}
\ttindex{sys.exc_traceback}
\ttindex{sys.last_traceback}

Special read-only attributes: \member{tb_next} is the next level in the
stack trace (towards the frame where the exception occurred), or
\code{None} if there is no next level; \member{tb_frame} points to the
execution frame of the current level; \member{tb_lineno} gives the line
number where the exception occurred; \member{tb_lasti} indicates the
precise instruction.  The line number and last instruction in the
traceback may differ from the line number of its frame object if the
exception occurred in a \keyword{try} statement with no matching
except clause or with a finally clause.
\withsubitem{(traceback attribute)}{%
  \ttindex{tb_next}%
  \ttindex{tb_frame}%
  \ttindex{tb_lineno}%
  \ttindex{tb_lasti}}
\stindex{try}

\item[Slice objects]
Slice objects are used to represent slices when \emph{extended slice
syntax} is used.  This is a slice using two colons, or multiple slices
or ellipses separated by commas, e.g., \code{a[i:j:step]}, \code{a[i:j,
k:l]}, or \code{a[..., i:j])}.  They are also created by the built-in
\function{slice()}\bifuncindex{slice} function.

Special read-only attributes: \member{start} is the lowerbound;
\member{stop} is the upperbound; \member{step} is the step value; each is
\code{None} if omitted. These attributes can have any type.
\withsubitem{(slice object attribute)}{%
  \ttindex{start}%
  \ttindex{stop}%
  \ttindex{step}}

\end{description} % Internal types

\end{description} % Types


\section{Special method names\label{specialnames}}

A class can implement certain operations that are invoked by special
syntax (such as arithmetic operations or subscripting and slicing) by
defining methods with special names.  For instance, if a class defines
a method named \method{__getitem__()}, and \code{x} is an instance of
this class, then \code{x[i]} is equivalent to
\code{x.__getitem__(i)}.  (The reverse is not true --- if \code{x} is
a list object, \code{x.__getitem__(i)} is not equivalent to
\code{x[i]}.)  Except where mentioned, attempts to execute an
operation raise an exception when no appropriate method is defined.
\withsubitem{(mapping object method)}{\ttindex{__getitem__()}}


\subsection{Basic customization\label{customization}}

\begin{methoddesc}[object]{__init__}{self\optional{, args...}}
Called when the instance is created.  The arguments are those passed
to the class constructor expression.  If a base class has an
\method{__init__()} method the derived class's \method{__init__()} method must
explicitly call it to ensure proper initialization of the base class
part of the instance, e.g., \samp{BaseClass.__init__(\var{self},
[\var{args}...])}.
\indexii{class}{constructor}
\end{methoddesc}


\begin{methoddesc}[object]{__del__}{self}
Called when the instance is about to be destroyed.  This is also
called a destructor\index{destructor}.  If a base class
has a \method{__del__()} method, the derived class's \method{__del__()} method
must explicitly call it to ensure proper deletion of the base class
part of the instance.  Note that it is possible (though not recommended!)
for the \method{__del__()}
method to postpone destruction of the instance by creating a new
reference to it.  It may then be called at a later time when this new
reference is deleted.  It is not guaranteed that
\method{__del__()} methods are called for objects that still exist when
the interpreter exits.
\stindex{del}

\strong{Programmer's note:} \samp{del x} doesn't directly call
\code{x.__del__()} --- the former decrements the reference count for
\code{x} by one, and the latter is only called when its reference
count reaches zero.  Some common situations that may prevent the
reference count of an object to go to zero include: circular
references between objects (e.g., a doubly-linked list or a tree data
structure with parent and child pointers); a reference to the object
on the stack frame of a function that caught an exception (the
traceback stored in \code{sys.exc_traceback} keeps the stack frame
alive); or a reference to the object on the stack frame that raised an
unhandled exception in interactive mode (the traceback stored in
\code{sys.last_traceback} keeps the stack frame alive).  The first
situation can only be remedied by explicitly breaking the cycles; the
latter two situations can be resolved by storing None in
\code{sys.exc_traceback} or \code{sys.last_traceback}.

\strong{Warning:} due to the precarious circumstances under which
\method{__del__()} methods are invoked, exceptions that occur during their
execution are ignored, and a warning is printed to \code{sys.stderr}
instead.  Also, when \method{__del__()} is invoked is response to a module
being deleted (e.g., when execution of the program is done), other
globals referenced by the \method{__del__()} method may already have been
deleted.  For this reason, \method{__del__()} methods should do the
absolute minimum needed to maintain external invariants.  Python 1.5
guarantees that globals whose name begins with a single underscore are
deleted from their module before other globals are deleted; if no
other references to such globals exist, this may help in assuring that
imported modules are still available at the time when the
\method{__del__()} method is called. 
\end{methoddesc}

\begin{methoddesc}[object]{__repr__}{self}
Called by the \function{repr()}\bifuncindex{repr} built-in function
and by string conversions (reverse quotes) to compute the ``official''
string representation of an object.  This should normally look like a
valid Python expression that can be used to recreate an object with
the same value.  By convention, objects which cannot be trivially
converted to strings which can be used to create a similar object
produce a string of the form \samp{<\var{...some useful
description...}>}.
\indexii{string}{conversion}
\indexii{reverse}{quotes}
\indexii{backward}{quotes}
\index{back-quotes}
\end{methoddesc}

\begin{methoddesc}[object]{__str__}{self}
Called by the \function{str()}\bifuncindex{str} built-in function and
by the \keyword{print}\stindex{print} statement to compute the
``informal'' string representation of an object.  This differs from
\method{__repr__()} in that it does not have to be a valid Python
expression: a more convenient or concise representation may be used
instead.
\end{methoddesc}

\begin{methoddesc}[object]{__cmp__}{self, other}
Called by all comparison operations.  Should return a negative integer if
\code{self < other},  zero if \code{self == other}, a positive integer if
\code{self > other}.  If no \method{__cmp__()} operation is defined, class
instances are compared by object identity (``address'').
(Note: the restriction that exceptions are not propagated by
\method{__cmp__()} has been removed in Python 1.5.)
\bifuncindex{cmp}
\index{comparisons}
\end{methoddesc}

\begin{methoddesc}[object]{__hash__}{self}
Called for the key object for dictionary\obindex{dictionary}
operations, and by the built-in function
\function{hash()}\bifuncindex{hash}.  Should return a 32-bit integer
usable as a hash value
for dictionary operations.  The only required property is that objects
which compare equal have the same hash value; it is advised to somehow
mix together (e.g., using exclusive or) the hash values for the
components of the object that also play a part in comparison of
objects.  If a class does not define a \method{__cmp__()} method it should
not define a \method{__hash__()} operation either; if it defines
\method{__cmp__()} but not \method{__hash__()} its instances will not be
usable as dictionary keys.  If a class defines mutable objects and
implements a \method{__cmp__()} method it should not implement
\method{__hash__()}, since the dictionary implementation requires that
a key's hash value is immutable (if the object's hash value changes, it
will be in the wrong hash bucket).
\withsubitem{(object method)}{\ttindex{__cmp__()}}
\end{methoddesc}

\begin{methoddesc}[object]{__nonzero__}{self}
Called to implement truth value testing; should return \code{0} or
\code{1}.  When this method is not defined, \method{__len__()} is
called, if it is defined (see below).  If a class defines neither
\method{__len__()} nor \method{__nonzero__()}, all its instances are
considered true.
\withsubitem{(mapping object method)}{\ttindex{__len__()}}
\end{methoddesc}


\subsection{Customizing attribute access\label{attribute-access}}

The following methods can be defined to customize the meaning of
attribute access (use of, assignment to, or deletion of \code{x.name})
for class instances.
For performance reasons, these methods are cached in the class object
at class definition time; therefore, they cannot be changed after the
class definition is executed.

\begin{methoddesc}[object]{__getattr__}{self, name}
Called when an attribute lookup has not found the attribute in the
usual places (i.e. it is not an instance attribute nor is it found in
the class tree for \code{self}).  \code{name} is the attribute name.
This method should return the (computed) attribute value or raise an
\exception{AttributeError} exception.

Note that if the attribute is found through the normal mechanism,
\method{__getattr__()} is not called.  (This is an intentional
asymmetry between \method{__getattr__()} and \method{__setattr__()}.)
This is done both for efficiency reasons and because otherwise
\method{__setattr__()} would have no way to access other attributes of
the instance.
Note that at least for instance variables, you can fake
total control by not inserting any values in the instance
attribute dictionary (but instead inserting them in another object).
\withsubitem{(object method)}{\ttindex{__setattr__()}}
\end{methoddesc}

\begin{methoddesc}[object]{__setattr__}{self, name, value}
Called when an attribute assignment is attempted.  This is called
instead of the normal mechanism (i.e.\ store the value in the instance
dictionary).  \var{name} is the attribute name, \var{value} is the
value to be assigned to it.

If \method{__setattr__()} wants to assign to an instance attribute, it 
should not simply execute \samp{self.\var{name} = value} --- this
would cause a recursive call to itself.  Instead, it should insert the
value in the dictionary of instance attributes, e.g.,
\samp{self.__dict__[\var{name}] = value}.
\withsubitem{(instance attribute)}{\ttindex{__dict__}}
\end{methoddesc}

\begin{methoddesc}[object]{__delattr__}{self, name}
Like \method{__setattr__()} but for attribute deletion instead of
assignment.  This should only be implemented if \samp{del
obj.\var{name}} is meaningful for the object.
\end{methoddesc}


\subsection{Emulating callable objects\label{callable-types}}

\begin{methoddesc}[object]{__call__}{self\optional{, args...}}
Called when the instance is ``called'' as a function; if this method
is defined, \code{\var{x}(arg1, arg2, ...)} is a shorthand for
\code{\var{x}.__call__(arg1, arg2, ...)}.
\indexii{call}{instance}
\end{methoddesc}


\subsection{Emulating sequence and mapping types\label{sequence-types}}

The following methods can be defined to emulate sequence or mapping
objects.  The first set of methods is used either to emulate a
sequence or to emulate a mapping; the difference is that for a
sequence, the allowable keys should be the integers \var{k} for which
\code{0 <= \var{k} < \var{N}} where \var{N} is the length of the
sequence, and the method \method{__getslice__()} (see below) should be
defined.  It is also recommended that mappings provide methods
\method{keys()}, \method{values()}, \method{items()},
\method{has_key()}, \method{get()}, \method{clear()}, \method{copy()},
and \method{update()} behaving similar to those for
Python's standard dictionary objects; mutable sequences should provide
methods \method{append()}, \method{count()}, \method{index()},
\method{insert()}, \method{pop()}, \method{remove()}, \method{reverse()}
and \method{sort()}, like Python standard list objects.  Finally,
sequence types should implement addition (meaning concatenation) and
multiplication (meaning repetition) by defining the methods
\method{__add__()}, \method{__radd__()}, \method{__mul__()} and
\method{__rmul__()} described below; they should not define
\method{__coerce__()} or other numerical operators.
\withsubitem{(mapping object method)}{%
  \ttindex{keys()}%
  \ttindex{values()}%
  \ttindex{items()}%
  \ttindex{has_key()}%
  \ttindex{get()}%
  \ttindex{clear()}%
  \ttindex{copy()}%
  \ttindex{update()}}
\withsubitem{(sequence object method)}{%
  \ttindex{append()}%
  \ttindex{count()}%
  \ttindex{index()}%
  \ttindex{insert()}%
  \ttindex{pop()}%
  \ttindex{remove()}%
  \ttindex{reverse()}%
  \ttindex{sort()}%
  \ttindex{__add__()}%
  \ttindex{__radd__()}%
  \ttindex{__mul__()}%
  \ttindex{__rmul__()}}
\withsubitem{(numberic object method)}{\ttindex{__coerce__()}}

\begin{methoddesc}[mapping object]{__len__}{self}
Called to implement the built-in function
\function{len()}\bifuncindex{len}.  Should return the length of the
object, an integer \code{>=} 0.  Also, an object that doesn't define a
\method{__nonzero__()} method and whose \method{__len__()} method
returns zero is considered to be false in a Boolean context.
\withsubitem{(object method)}{\ttindex{__nonzero__()}}
\end{methoddesc}

\begin{methoddesc}[mapping object]{__getitem__}{self, key}
Called to implement evaluation of \code{\var{self}[\var{key}]}.
For a sequence types, the accepted keys should be integers.  Note that the
special interpretation of negative indices (if the class wishes to
emulate a sequence type) is up to the \method{__getitem__()} method.
\end{methoddesc}

\begin{methoddesc}[mapping object]{__setitem__}{self, key, value}
Called to implement assignment to \code{\var{self}[\var{key}]}.  Same
note as for \method{__getitem__()}.  This should only be implemented
for mappings if the objects support changes to the values for keys, or
if new keys can be added, or for sequences if elements can be
replaced.
\end{methoddesc}

\begin{methoddesc}[mapping object]{__delitem__}{self, key}
Called to implement deletion of \code{\var{self}[\var{key}]}.  Same
note as for \method{__getitem__()}.  This should only be implemented
for mappings if the objects support removal of keys, or for sequences
if elements can be removed from the sequence.
\end{methoddesc}


\subsection{Additional methods for emulation of sequence types
  \label{sequence-methods}}

The following methods can be defined to further emulate sequence
objects.  Immutable sequences methods should only define
\method{__getslice__()}; mutable sequences, should define all three
three methods.

\begin{methoddesc}[sequence object]{__getslice__}{self, i, j}
Called to implement evaluation of \code{\var{self}[\var{i}:\var{j}]}.
The returned object should be of the same type as \var{self}.  Note
that missing \var{i} or \var{j} in the slice expression are replaced
by zero or \code{sys.maxint}, respectively, and no further
transformations on the indices is performed.  The interpretation of
negative indices and indices larger than the length of the sequence is
up to the method.
\end{methoddesc}

\begin{methoddesc}[sequence object]{__setslice__}{self, i, j, sequence}
Called to implement assignment to \code{\var{self}[\var{i}:\var{j}]}.
Same notes for \var{i} and \var{j} as for \method{__getslice__()}.
\end{methoddesc}

\begin{methoddesc}[sequence object]{__delslice__}{self, i, j}
Called to implement deletion of \code{\var{self}[\var{i}:\var{j}]}.
Same notes for \var{i} and \var{j} as for \method{__getslice__()}.
\end{methoddesc}

Notice that these methods are only invoked when a single slice with a
single colon is used.  For slice operations involving extended slice
notation, \method{__getitem__()}, \method{__setitem__()}
or\method{__delitem__()} is called.

\subsection{Emulating numeric types\label{numeric-types}}

The following methods can be defined to emulate numeric objects.
Methods corresponding to operations that are not supported by the
particular kind of number implemented (e.g., bitwise operations for
non-integral numbers) should be left undefined.

\begin{methoddesc}[numberic interface]{__add__}{self, other}
\methodline{__sub__}{self, other}
\methodline{__mul__}{self, other}
\methodline{__div__}{self, other}
\methodline{__mod__}{self, other}
\methodline{__divmod__}{self, other}
\methodline{__pow__}{self, other\optional{, modulo}}
\methodline{__lshift__}{self, other}
\methodline{__rshift__}{self, other}
\methodline{__and__}{self, other}
\methodline{__xor__}{self, other}
\methodline{__or__}{self, other}
These functions are
called to implement the binary arithmetic operations (\code{+},
\code{-}, \code{*}, \code{/}, \code{\%},
\function{divmod()}\bifuncindex{divmod},
\function{pow()}\bifuncindex{pow}, \code{**}, \code{<<}, \code{>>},
\code{\&}, \code{\^}, \code{|}).  For instance, to evaluate the
expression \var{x}\code{+}\var{y}, where \var{x} is an instance of a
class that has an \method{__add__()} method,
\code{\var{x}.__add__(\var{y})} is called.  Note that
\method{__pow__()} should be defined to accept an optional third
argument if the ternary version of the built-in
\function{pow()}\bifuncindex{pow} function is to be supported.
\end{methoddesc}

\begin{methoddesc}[numeric interface]{__radd__}{self, other}
\methodline{__rsub__}{self, other}
\methodline{__rmul__}{self, other}
\methodline{__rdiv__}{self, other}
\methodline{__rmod__}{self, other}
\methodline{__rdivmod__}{self, other}
\methodline{__rpow__}{self, other}
\methodline{__rlshift__}{self, other}
\methodline{__rrshift__}{self, other}
\methodline{__rand__}{self, other}
\methodline{__rxor__}{self, other}
\methodline{__ror__}{self, other}
These functions are
called to implement the binary arithmetic operations (\code{+},
\code{-}, \code{*}, \code{/}, \code{\%},
\function{divmod()}\bifuncindex{divmod},
\function{pow()}\bifuncindex{pow}, \code{**}, \code{<<}, \code{>>},
\code{\&}, \code{\^}, \code{|}) with reversed operands.  These
functions are only called if the left operand does not support the
corresponding operation.  For instance, to evaluate the expression
\var{x}\code{-}\var{y}, where \var{y} is an instance of a class that
has an \method{__rsub__()} method, \code{\var{y}.__rsub__(\var{x})} is
called.  Note that ternary \function{pow()}\bifuncindex{pow} will not
try calling \method{__rpow__()} (the coercion rules would become too
complicated).
\end{methoddesc}

\begin{methoddesc}[numeric interface]{__neg__}{self}
\methodline{__pos__}{self}
\methodline{__abs__}{self}
\methodline{__invert__}{self}
Called to implement the unary arithmetic operations (\code{-}, \code{+},
\function{abs()}\bifuncindex{abs} and \code{~}).
\end{methoddesc}

\begin{methoddesc}[numeric interface]{__int__}{self}
\methodlineni{__long__}{self}
\methodlineni{__float__}{self}
Called to implement the built-in functions
\function{int()}\bifuncindex{int}, \function{long()}\bifuncindex{long} 
and \function{float()}\bifuncindex{float}.  Should return a value of
the appropriate type.
\end{methoddesc}

\begin{methoddesc}[numeric interface]{__oct__}{self}
\methodlineni{__hex__}{self}
Called to implement the built-in functions
\function{oct()}\bifuncindex{oct} and
\function{hex()}\bifuncindex{hex}.  Should return a string value.
\end{methoddesc}

\begin{methoddesc}[numeric interface]{__coerce__}{self, other}
Called to implement ``mixed-mode'' numeric arithmetic.  Should either
return a 2-tuple containing \var{self} and \var{other} converted to
a common numeric type, or \code{None} if conversion is possible.  When
the common type would be the type of \code{other}, it is sufficient to
return \code{None}, since the interpreter will also ask the other
object to attempt a coercion (but sometimes, if the implementation of
the other type cannot be changed, it is useful to do the conversion to
the other type here).
\end{methoddesc}

\strong{Coercion rules}: to evaluate \var{x} \var{op} \var{y}, the
following steps are taken (where \method{__op__()} and
\method{__rop__()} are the method names corresponding to \var{op},
e.g., if var{op} is `\code{+}', \method{__add__()} and
\method{__radd__()} are used).  If an exception occurs at any point,
the evaluation is abandoned and exception handling takes over.

\begin{itemize}

\item[0.] If \var{x} is a string object and op is the modulo operator (\%),
the string formatting operation is invoked and the remaining steps are
skipped.

\item[1.] If \var{x} is a class instance:

	\begin{itemize}

	\item[1a.] If \var{x} has a \method{__coerce__()} method:
	replace \var{x} and \var{y} with the 2-tuple returned by
	\code{\var{x}.__coerce__(\var{y})}; skip to step 2 if the
	coercion returns \code{None}.

	\item[1b.] If neither \var{x} nor \var{y} is a class instance
	after coercion, go to step 3.

	\item[1c.] If \var{x} has a method \method{__op__()}, return
	\code{\var{x}.__op__(\var{y})}; otherwise, restore \var{x} and
	\var{y} to their value before step 1a.

	\end{itemize}

\item[2.] If \var{y} is a class instance:

	\begin{itemize}

	\item[2a.] If \var{y} has a \method{__coerce__()} method:
	replace \var{y} and \var{x} with the 2-tuple returned by
	\code{\var{y}.__coerce__(\var{x})}; skip to step 3 if the
	coercion returns \code{None}.

	\item[2b.] If neither \var{x} nor \var{y} is a class instance
	after coercion, go to step 3.

	\item[2b.] If \var{y} has a method \method{__rop__()}, return
	\code{\var{y}.__rop__(\var{x})}; otherwise, restore \var{x}
	and \var{y} to their value before step 2a.

	\end{itemize}

\item[3.] We only get here if neither \var{x} nor \var{y} is a class
instance.

	\begin{itemize}

	\item[3a.] If op is `\code{+}' and \var{x} is a sequence,
	sequence concatenation is invoked.

	\item[3b.] If op is `\code{*}' and one operand is a sequence
	and the other an integer, sequence repetition is invoked.

	\item[3c.] Otherwise, both operands must be numbers; they are
	coerced to a common type if possible, and the numeric
	operation is invoked for that type.

	\end{itemize}

\end{itemize}
