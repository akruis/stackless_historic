\documentclass{howto}
% $Id$

% TODO:
%   Go through and get the contributor's name for all the various changes

\title{What's New in Python 2.3}
\release{0.01}
\author{A.M. Kuchling}
\authoraddress{\email{akuchlin@mems-exchange.org}}

\begin{document}
\maketitle
\tableofcontents

%\section{Introduction \label{intro}}

{\large This article is a draft, and is currently up to date for some
random version of the CVS tree around May 26 2002.  Please send any
additions, comments or errata to the author.}

This article explains the new features in Python 2.3.  The tentative
release date of Python 2.3 is currently scheduled for August 30 2002.

This article doesn't attempt to provide a complete specification of
the new features, but instead provides a convenient overview.  For
full details, you should refer to the documentation for Python 2.3,
such as the
\citetitle[http://www.python.org/doc/2.3/lib/lib.html]{Python Library
Reference} and the
\citetitle[http://www.python.org/doc/2.3/ref/ref.html]{Python
Reference Manual}.  If you want to understand the complete
implementation and design rationale for a change, refer to the PEP for
a particular new feature.


%======================================================================
\section{PEP 255: Simple Generators\label{section-generators}}

In Python 2.2, generators were added as an optional feature, to be
enabled by a \code{from __future__ import generators} directive.  In
2.3 generators no longer need to be specially enabled, and are now
always present; this means that \keyword{yield} is now always a
keyword.  The rest of this section is a copy of the description of
generators from the ``What's New in Python 2.2'' document; if you read
it when 2.2 came out, you can skip the rest of this section.

You're doubtless familiar with how function calls work in Python or C.
When you call a function, it gets a private namespace where its local
variables are created.  When the function reaches a \keyword{return}
statement, the local variables are destroyed and the resulting value
is returned to the caller.  A later call to the same function will get
a fresh new set of local variables. But, what if the local variables
weren't thrown away on exiting a function?  What if you could later
resume the function where it left off?  This is what generators
provide; they can be thought of as resumable functions.

Here's the simplest example of a generator function:

\begin{verbatim}
def generate_ints(N):
    for i in range(N):
        yield i
\end{verbatim}

A new keyword, \keyword{yield}, was introduced for generators.  Any
function containing a \keyword{yield} statement is a generator
function; this is detected by Python's bytecode compiler which
compiles the function specially as a result.  

When you call a generator function, it doesn't return a single value;
instead it returns a generator object that supports the iterator
protocol.  On executing the \keyword{yield} statement, the generator
outputs the value of \code{i}, similar to a \keyword{return}
statement.  The big difference between \keyword{yield} and a
\keyword{return} statement is that on reaching a \keyword{yield} the
generator's state of execution is suspended and local variables are
preserved.  On the next call to the generator's \code{.next()} method,
the function will resume executing immediately after the
\keyword{yield} statement.  (For complicated reasons, the
\keyword{yield} statement isn't allowed inside the \keyword{try} block
of a \code{try...finally} statement; read \pep{255} for a full
explanation of the interaction between \keyword{yield} and
exceptions.)

Here's a sample usage of the \function{generate_ints} generator:

\begin{verbatim}
>>> gen = generate_ints(3)
>>> gen
<generator object at 0x8117f90>
>>> gen.next()
0
>>> gen.next()
1
>>> gen.next()
2
>>> gen.next()
Traceback (most recent call last):
  File "<stdin>", line 1, in ?
  File "<stdin>", line 2, in generate_ints
StopIteration
\end{verbatim}

You could equally write \code{for i in generate_ints(5)}, or
\code{a,b,c = generate_ints(3)}.

Inside a generator function, the \keyword{return} statement can only
be used without a value, and signals the end of the procession of
values; afterwards the generator cannot return any further values.
\keyword{return} with a value, such as \code{return 5}, is a syntax
error inside a generator function.  The end of the generator's results
can also be indicated by raising \exception{StopIteration} manually,
or by just letting the flow of execution fall off the bottom of the
function.

You could achieve the effect of generators manually by writing your
own class and storing all the local variables of the generator as
instance variables.  For example, returning a list of integers could
be done by setting \code{self.count} to 0, and having the
\method{next()} method increment \code{self.count} and return it.
However, for a moderately complicated generator, writing a
corresponding class would be much messier.
\file{Lib/test/test_generators.py} contains a number of more
interesting examples.  The simplest one implements an in-order
traversal of a tree using generators recursively.

\begin{verbatim}
# A recursive generator that generates Tree leaves in in-order.
def inorder(t):
    if t:
        for x in inorder(t.left):
            yield x
        yield t.label
        for x in inorder(t.right):
            yield x
\end{verbatim}

Two other examples in \file{Lib/test/test_generators.py} produce
solutions for the N-Queens problem (placing $N$ queens on an $NxN$
chess board so that no queen threatens another) and the Knight's Tour
(a route that takes a knight to every square of an $NxN$ chessboard
without visiting any square twice). 

The idea of generators comes from other programming languages,
especially Icon (\url{http://www.cs.arizona.edu/icon/}), where the
idea of generators is central.  In Icon, every
expression and function call behaves like a generator.  One example
from ``An Overview of the Icon Programming Language'' at
\url{http://www.cs.arizona.edu/icon/docs/ipd266.htm} gives an idea of
what this looks like:

\begin{verbatim}
sentence := "Store it in the neighboring harbor"
if (i := find("or", sentence)) > 5 then write(i)
\end{verbatim}

In Icon the \function{find()} function returns the indexes at which the
substring ``or'' is found: 3, 23, 33.  In the \keyword{if} statement,
\code{i} is first assigned a value of 3, but 3 is less than 5, so the
comparison fails, and Icon retries it with the second value of 23.  23
is greater than 5, so the comparison now succeeds, and the code prints
the value 23 to the screen.

Python doesn't go nearly as far as Icon in adopting generators as a
central concept.  Generators are considered a new part of the core
Python language, but learning or using them isn't compulsory; if they
don't solve any problems that you have, feel free to ignore them.
One novel feature of Python's interface as compared to
Icon's is that a generator's state is represented as a concrete object
(the iterator) that can be passed around to other functions or stored
in a data structure.

\begin{seealso}

\seepep{255}{Simple Generators}{Written by Neil Schemenauer, Tim
Peters, Magnus Lie Hetland.  Implemented mostly by Neil Schemenauer
and Tim Peters, with other fixes from the Python Labs crew.}

\end{seealso}


%======================================================================
\section{PEP 278: Universal Newline Support}

The three major operating systems used today are Microsoft Windows,
Apple's Macintosh OS, and the various \UNIX\ derivatives.  A minor
irritation is that these three platforms all use different characters
to mark the ends of lines in text files.  \UNIX\ uses character 10,
the ASCII linefeed, while MacOS uses character 13, the ASCII carriage
return, and Windows uses a two-character sequence of a carriage return
plus a newline.

Python's file objects can now support end of line conventions other
than the one followed by the platform on which Python is running.
Opening a file with the mode \samp{U} or \samp{rU} will open a file
for reading in universal newline mode.  All three line ending
conventions will be translated to a \samp{\e n} in the strings
returned by the various file methods such as \method{read()} and
\method{readline()}. 

Universal newline support is also used when importing modules and when
executing a file with the \function{execfile()} function.  This means
that Python modules can be shared between all three operating systems
without needing to convert the line-endings.

This feature can be disabled at compile-time by specifying 
\longprogramopt{without-universal-newlines} when running Python's
\file{configure} script.

\begin{seealso}

\seepep{278}{Universal Newline Support}{Written 
and implemented by Jack Jansen.}

\end{seealso}


%======================================================================
\section{PEP 279: The \function{enumerate()} Built-in Function}

A new built-in function, \function{enumerate()}, will make
certain loops a bit clearer.  \code{enumerate(thing)}, where
\var{thing} is either an iterator or a sequence, returns a iterator
that will return \code{(0, \var{thing[0]})}, \code{(1,
\var{thing[1]})}, \code{(2, \var{thing[2]})}, and so forth.  Fairly
often you'll see code to change every element of a list that looks
like this:

\begin{verbatim}
for i in range(len(L)):
    item = L[i]
    # ... compute some result based on item ...
    L[i] = result
\end{verbatim}

This can be rewritten using \function{enumerate()} as:

\begin{verbatim}
for i, item in enumerate(L):
    # ... compute some result based on item ...
    L[i] = result
\end{verbatim}


\begin{seealso}

\seepep{279}{The enumerate() built-in function}{Written 
by Raymond D. Hettinger.}

\end{seealso}


%======================================================================
\section{PEP 285: The \class{bool} Type\label{section-bool}}

A Boolean type was added to Python 2.3.  Two new constants were added
to the \module{__builtin__} module, \constant{True} and
\constant{False}.  The type object for this new type is named
\class{bool}; the constructor for it takes any Python value and
converts it to \constant{True} or \constant{False}.

\begin{verbatim}
>>> bool(1)
True
>>> bool(0)
False
>>> bool([])
False
>>> bool( (1,) )
True
\end{verbatim}

Most of the standard library modules and built-in functions have been
changed to return Booleans.

\begin{verbatim}
>>> obj = []
>>> hasattr(obj, 'append')
True
>>> isinstance(obj, list)
True
>>> isinstance(obj, tuple)
False
\end{verbatim}

Python's Booleans were added with the primary goal of making code
clearer.  For example, if you're reading a function and encounter the
statement \code{return 1}, you might wonder whether the \samp{1}
represents a truth value, or whether it's an index, or whether it's a
coefficient that multiplies some other quantity.  If the statement is
\code{return True}, however, the meaning of the return value is quite
clearly a truth value.

Python's Booleans were not added for the sake of strict type-checking.
A very strict language such as Pascal would also prevent you
performing arithmetic with Booleans, and would require that the
expression in an \keyword{if} statement always evaluate to a Boolean.
Python is not this strict, and it never will be.  (\pep{285}
explicitly says so.)  So you can still use any expression in an
\keyword{if}, even ones that evaluate to a list or tuple or some
random object, and the Boolean type is a subclass of the
\class{int} class, so arithmetic using a Boolean still works.

\begin{verbatim}
>>> True + 1
2
>>> False + 1
1
>>> False * 75
0
>>> True * 75
75
\end{verbatim}

To sum up \constant{True} and \constant{False} in a sentence: they're
alternative ways to spell the integer values 1 and 0, with the single
difference that \function{str()} and \function{repr()} return the
strings \samp{True} and \samp{False} instead of \samp{1} and \samp{0}.

\begin{seealso}

\seepep{285}{Adding a bool type}{Written and implemented by GvR.}

\end{seealso}


%======================================================================
%\section{Other Language Changes}

%Here are the changes that Python 2.3 makes to the core language.

%\begin{itemize}
%\item The \keyword{yield} statement is now always a keyword, as
%described in section~\ref{section-generators}.

%\item Two new constants, \constant{True} and \constant{False} were
%added along with the built-in \class{bool} type, as described in
%section~\ref{section-bool}.

%\item 
%\end{itemize}


%\begin{PendingDeprecationWarning}
A new warning PendingDeprecationWarning was added to provide
direction on features which are in the process of being deprecated.
The warning will not be printed by default.  To see the pending
deprecations, use -Walways::PendingDeprecationWarning:: on the command line
or warnings.filterwarnings().
%\end{PendingDeprecationWarning}


%======================================================================
\section{Specialized Object Allocator (pymalloc)\label{section-pymalloc}}

An experimental feature added to Python 2.1 was a specialized object
allocator called pymalloc, written by Vladimir Marangozov.  Pymalloc
was intended to be faster than the system \function{malloc()} and have
less memory overhead.  The allocator uses C's \function{malloc()}
function to get large pools of memory, and then fulfills smaller
memory requests from these pools.  

In 2.1 and 2.2, pymalloc was an experimental feature and wasn't
enabled by default; you had to explicitly turn it on by providing the
\longprogramopt{with-pymalloc} option to the \program{configure}
script.  In 2.3, pymalloc has had further enhancements and is now
enabled by default; you'll have to supply
\longprogramopt{without-pymalloc} to disable it.

This change is transparent to code written in Python; however,
pymalloc may expose bugs in C extensions.  Authors of C extension
modules should test their code with the object allocator enabled,
because some incorrect code may cause core dumps at runtime.  There
are a bunch of memory allocation functions in Python's C API that have
previously been just aliases for the C library's \function{malloc()}
and \function{free()}, meaning that if you accidentally called
mismatched functions, the error wouldn't be noticeable.  When the
object allocator is enabled, these functions aren't aliases of
\function{malloc()} and \function{free()} any more, and calling the
wrong function to free memory will get you a core dump.  For example,
if memory was allocated using \function{PyMem_New()}, it has to be
freed using \function{PyMem_Del()}, not \function{free()}.  A few
modules included with Python fell afoul of this and had to be fixed;
doubtless there are more third-party modules that will have the same
problem.

As part of this change, the confusing multiple interfaces for
allocating memory have been consolidated down into two APIs.  
Memory allocated with one API must not be freed with the other API.

\begin{itemize}
  \item To allocate and free an undistinguished chunk of memory using 
  Python's allocator, use
  \cfunction{PyMem_Malloc()}, \cfunction{PyMem_Realloc()}, and
  \cfunction{PyMem_Free()}.

  \item In rare cases you may want to avoid using Python's allocator 
  in order to allocate a chunk of memory; 
  use \cfunction{PyObject_Malloc}, \cfunction{PyObject_Realloc}, 
  and \cfunction{PyObject_Free}.

  \item To allocate and free Python objects,  
  use \cfunction{PyObject_New()}, \cfunction{PyObject_NewVar()}, and 
  \cfunction{PyObject_Del()}.

\end{itemize}

Thanks to lots of work by Tim Peters, pymalloc in 2.3 also provides
debugging features to catch memory overwrites and doubled frees in
both extension modules and in the interpreter itself.  To enable this
support, turn on the Python interpreter's debugging code by running
\program{configure} with \longprogramopt{with-pydebug}.  

\begin{seealso}

\seeurl{http://cvs.sourceforge.net/cgi-bin/viewcvs.cgi/python/python/dist/src/Objects/obmalloc.c}
{For the full details of the pymalloc implementation, see
the comments at the top of the file \file{Objects/obmalloc.c} in the
Python source code.  The above link points to the file within the
SourceForge CVS browser.}

\end{seealso}

%======================================================================
\section{New and Improved Modules}

As usual, Python's standard modules had a number of enhancements and
bug fixes.  Here's a partial list; consult the \file{Misc/NEWS} file
in the source tree, or the CVS logs, for a more complete list.

\begin{itemize}

\item One minor but far-reaching change is that the names of extension
types defined by the modules included with Python now contain the
module and a \samp{.} in front of the type name.  For example, in
Python 2.2, if you created a socket and printed its
\member{__class__}, you'd get this output:

\begin{verbatim}
>>> s = socket.socket()
>>> s.__class__
<type 'socket'>
\end{verbatim}

In 2.3, you get this:
\begin{verbatim}
>>> s.__class__
<type '_socket.socket'>
\end{verbatim}

\item The \method{strip()}, \method{lstrip()}, and \method{rstrip()}
string methods now have an optional argument for specifying the
characters to strip.  The default is still to remove all whitespace
characters:

\begin{verbatim}
>>> '   abc '.strip()
'abc'
>>> '><><abc<><><>'.strip('<>')
'abc'
>>> '><><abc<><><>\n'.strip('<>')
'abc<><><>\n'
>>> u'\u4000\u4001abc\u4000'.strip(u'\u4000')
u'\u4001abc'
>>>
\end{verbatim}

\item Another new string method is \method{zfill()}, originally a
function in the \module{string} module.  \method{zfill()} pads a
numeric string with zeros on the left until it's the specified width.
Note that the \code{\%} operator is still more flexible and powerful
than \method{zfill()}.

\begin{verbatim}
>>> '45'.zfill(4)
'0045'
>>> '12345'.zfill(4)
'12345'
>>> 'goofy'.zfill(6)
'0goofy'
\end{verbatim}

\item Dictionaries have a new method, method{pop(\var{key})}, that
returns the value corresponding to \var{key} and removes that
key/value pair from the dictionary.  \method{pop()} will raise a
\exception{KeyError} if the requsted key isn't present in the
dictionary:

\begin{verbatim}
>>> d = {1:2}
>>> d
{1: 2}
>>> d.pop(4)
Traceback (most recent call last):
  File ``<stdin>'', line 1, in ?
KeyError: 4
>>> d.pop(1)
2
>>> d.pop(1)
Traceback (most recent call last):
  File ``<stdin>'', line 1, in ?
KeyError: pop(): dictionary is empty
>>> d
{}
>>>
\end{verbatim}

\item Two new functions in the \module{math} module, 
\function{degrees(\var{rads})} and \function{radians(\var{degs})},
convert between radians and degrees.  Other functions in the 
\module{math} module such as
\function{math.sin()} and \function{math.cos()} have always required
input values measured in radians. (Contributed by Raymond Hettinger.)

\item Two new functions, \function{killpg()} and \function{mknod()},
were added to the \module{posix} module that underlies the \module{os}
module.

\item Two new binary packagers were added to the Distutils.
\code{bdist_pkgtool} builds \file{.pkg} files to use with Solaris
\program{pkgtool}, and \code{bdist_sdux} builds \program{swinstall}
packages for use on HP-UX.  (Contributed by Mark Alexander.)

\item The \module{array} module now supports arrays of Unicode
characters using the \samp{u} format character.  Arrays also 
now support using the \code{+=} assignment operator to add another array's
contents, and the \code{*=} assignment operator to repeat an array.
(Contributed by Jason Orendorff.)

\item The \module{grp} module now returns enhanced tuples:

\begin{verbatim}
>>> import grp
>>> g = grp.getgrnam('amk')
>>> g.gr_name, g.gr_gid
('amk', 500)
\end{verbatim}

\item The \module{readline} module also gained a number of new
functions: \function{get_history_item()},
\function{get_current_history_length()}, and \function{redisplay()}.

\item Support for more advanced POSIX signal handling was added
to the \module{signal} module by adding the \function{sigpending},
\function{sigprocmask} and \function{sigsuspend} functions, where supported
by the platform.  These functions make it possible to avoid some previously
unavoidable race conditions.

\end{itemize}


% ======================================================================
\section{Build and C API Changes}

Changes to Python's build process, and to the C API, include:

\begin{itemize}

\item Python can now optionally be built as a shared library
(\file{libpython2.3.so}) by supplying \longprogramopt{enable-shared}
when running Python's \file{configure} script.  (Contributed by Ondrej
Palkovsky.)

\item The \cfunction{PyArg_NoArgs()} macro is now deprecated, and code
that
uses it should be changed to use \code{PyArg_ParseTuple(args, "")}
instead.

\item A new function, \cfunction{PyObject_DelItemString(\var{mapping},
char *\var{key})} was added
as shorthand for 
\code{PyObject_DelItem(\var{mapping}, PyString_New(\var{key})}.

\item The source code for the Expat XML parser is now included with
the Python source, so the \module{pyexpat} module is no longer
dependent on having a system library containing Expat.

\item File objects now manage their internal string buffer
differently by increasing it exponentially when needed.  
This results in the benchmark tests in \file{Lib/test/test_bufio.py} 
speeding up from 57 seconds to 1.7 seconds, according to one
measurement.

\item XXX Introduce two new flag bits that can be set in a PyMethodDef method
descriptor, as used for the tp_methods slot of a type.  These new flag
bits are both optional, and mutually exclusive.  Most methods will not
use either.  These flags are used to create special method types which
exist in the same namespace as normal methods without having to use
tedious construction code to insert the new special method objects in
the type's tp_dict after PyType_Ready() has been called.

If METH_CLASS is specified, the method will represent a class method
like that returned by the classmethod() built-in.

If METH_STATIC is specified, the method will represent a static method
like that returned by the staticmethod() built-in.

These flags may not be used in the PyMethodDef table for modules since
these special method types are not meaningful in that case; a
ValueError will be raised if these flags are found in that context.

\end{itemize}

\subsection{Port-Specific Changes}

XXX write this

XXX OS/2 EMX port

XXX MacOS: Weaklink most toolbox modules, improving backward
compatibility. Modules will no longer fail to load if a single routine
is missing on the curent OS version, in stead calling the missing
routine will raise an exception.  Should finally fix 531398. 2.2.1
candidate.  Also blacklisted some constants with definitions that
were not Python-compatible.

XXX Checked in Sean Reifschneider's RPM spec file and patches.


%======================================================================
\section{Other Changes and Fixes}

Finally, there are various miscellaneous fixes:

\begin{itemize}

\item The tools used to build the documentation now work under Cygwin
as well as \UNIX.

\end{itemize}

%======================================================================
\section{Acknowledgements \label{acks}}

The author would like to thank the following people for offering
suggestions, corrections and assistance with various drafts of this
article: Fred~L. Drake, Jr., Detlef Lannert.

\end{document}
