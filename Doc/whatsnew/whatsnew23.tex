\documentclass{howto}
% $Id$

\title{What's New in Python 2.3}
\release{0.03}
\author{A.M. Kuchling}
\authoraddress{\email{akuchlin@mems-exchange.org}}

\begin{document}
\maketitle
\tableofcontents

% Optik (or whatever it gets called)
%
% MacOS framework-related changes (section of its own, probably)
%
% New sorting code
%
% xreadlines obsolete; files are their own iterator

%\section{Introduction \label{intro}}

{\large This article is a draft, and is currently up to date for some
random version of the CVS tree around mid-July 2002.  Please send any
additions, comments or errata to the author.}

This article explains the new features in Python 2.3.  The tentative
release date of Python 2.3 is currently scheduled for some undefined
time before the end of 2002.

This article doesn't attempt to provide a complete specification of
the new features, but instead provides a convenient overview.  For
full details, you should refer to the documentation for Python 2.3,
such as the
\citetitle[http://www.python.org/doc/2.3/lib/lib.html]{Python Library
Reference} and the
\citetitle[http://www.python.org/doc/2.3/ref/ref.html]{Python
Reference Manual}.  If you want to understand the complete
implementation and design rationale for a change, refer to the PEP for
a particular new feature.


%======================================================================
\section{PEP 218: A Standard Set Datatype}

The new \module{sets} module contains an implementation of a set
datatype.  The \class{Set} class is for mutable sets, sets that can
have members added and removed.  The \class{ImmutableSet} class is for
sets that can't be modified, and can be used as dictionary keys.  Sets
are built on top of dictionaries, so the elements within a set must be
hashable.

As a simple example, 

\begin{verbatim}
>>> import sets
>>> S = sets.Set([1,2,3])
>>> S
Set([1, 2, 3])
>>> 1 in S
True
>>> 0 in S
False
>>> S.add(5)
>>> S.remove(3)
>>> S
Set([1, 2, 5])
>>> 
\end{verbatim}

The union and intersection of sets can be computed with the
\method{union()} and \method{intersection()} methods, or,
alternatively, using the bitwise operators \samp{\&} and \samp{|}.
Mutable sets also have in-place versions of these methods,
\method{union_update()} and \method{intersection_update()}.

\begin{verbatim}
>>> S1 = sets.Set([1,2,3])
>>> S2 = sets.Set([4,5,6])
>>> S1.union(S2)
Set([1, 2, 3, 4, 5, 6])
>>> S1 | S2                  # Alternative notation
Set([1, 2, 3, 4, 5, 6])
>>> S1.intersection(S2)  
Set([])
>>> S1 & S2                  # Alternative notation
Set([])
>>> S1.union_update(S2)
Set([1, 2, 3, 4, 5, 6])
>>> S1
Set([1, 2, 3, 4, 5, 6])
>>> 
\end{verbatim}

It's also possible to take the symmetric difference of two sets.  This
is the set of all elements in the union that aren't in the
intersection.  An alternative way of expressing the symmetric
difference is that it contains all elements that are in exactly one
set.  Again, there's an in-place version, with the ungainly name
\method{symmetric_difference_update()}.

\begin{verbatim}
>>> S1 = sets.Set([1,2,3,4])
>>> S2 = sets.Set([3,4,5,6])
>>> S1.symmetric_difference(S2)
Set([1, 2, 5, 6])
>>> S1 ^ S2
Set([1, 2, 5, 6])
>>>
\end{verbatim}

There are also methods, \method{issubset()} and \method{issuperset()},
for checking whether one set is a strict subset or superset of
another:

\begin{verbatim}
>>> S1 = sets.Set([1,2,3])
>>> S2 = sets.Set([2,3])
>>> S2.issubset(S1)
True
>>> S1.issubset(S2)
False
>>> S1.issuperset(S2)
True
>>>
\end{verbatim}


\begin{seealso}

\seepep{218}{Adding a Built-In Set Object Type}{PEP written by Greg V. Wilson.
Implemented by Greg V. Wilson, Alex Martelli, and GvR.}

\end{seealso}



%======================================================================
\section{PEP 255: Simple Generators\label{section-generators}}

In Python 2.2, generators were added as an optional feature, to be
enabled by a \code{from __future__ import generators} directive.  In
2.3 generators no longer need to be specially enabled, and are now
always present; this means that \keyword{yield} is now always a
keyword.  The rest of this section is a copy of the description of
generators from the ``What's New in Python 2.2'' document; if you read
it when 2.2 came out, you can skip the rest of this section.

You're doubtless familiar with how function calls work in Python or C.
When you call a function, it gets a private namespace where its local
variables are created.  When the function reaches a \keyword{return}
statement, the local variables are destroyed and the resulting value
is returned to the caller.  A later call to the same function will get
a fresh new set of local variables. But, what if the local variables
weren't thrown away on exiting a function?  What if you could later
resume the function where it left off?  This is what generators
provide; they can be thought of as resumable functions.

Here's the simplest example of a generator function:

\begin{verbatim}
def generate_ints(N):
    for i in range(N):
        yield i
\end{verbatim}

A new keyword, \keyword{yield}, was introduced for generators.  Any
function containing a \keyword{yield} statement is a generator
function; this is detected by Python's bytecode compiler which
compiles the function specially as a result.  

When you call a generator function, it doesn't return a single value;
instead it returns a generator object that supports the iterator
protocol.  On executing the \keyword{yield} statement, the generator
outputs the value of \code{i}, similar to a \keyword{return}
statement.  The big difference between \keyword{yield} and a
\keyword{return} statement is that on reaching a \keyword{yield} the
generator's state of execution is suspended and local variables are
preserved.  On the next call to the generator's \code{.next()} method,
the function will resume executing immediately after the
\keyword{yield} statement.  (For complicated reasons, the
\keyword{yield} statement isn't allowed inside the \keyword{try} block
of a \code{try...finally} statement; read \pep{255} for a full
explanation of the interaction between \keyword{yield} and
exceptions.)

Here's a sample usage of the \function{generate_ints} generator:

\begin{verbatim}
>>> gen = generate_ints(3)
>>> gen
<generator object at 0x8117f90>
>>> gen.next()
0
>>> gen.next()
1
>>> gen.next()
2
>>> gen.next()
Traceback (most recent call last):
  File "stdin", line 1, in ?
  File "stdin", line 2, in generate_ints
StopIteration
\end{verbatim}

You could equally write \code{for i in generate_ints(5)}, or
\code{a,b,c = generate_ints(3)}.

Inside a generator function, the \keyword{return} statement can only
be used without a value, and signals the end of the procession of
values; afterwards the generator cannot return any further values.
\keyword{return} with a value, such as \code{return 5}, is a syntax
error inside a generator function.  The end of the generator's results
can also be indicated by raising \exception{StopIteration} manually,
or by just letting the flow of execution fall off the bottom of the
function.

You could achieve the effect of generators manually by writing your
own class and storing all the local variables of the generator as
instance variables.  For example, returning a list of integers could
be done by setting \code{self.count} to 0, and having the
\method{next()} method increment \code{self.count} and return it.
However, for a moderately complicated generator, writing a
corresponding class would be much messier.
\file{Lib/test/test_generators.py} contains a number of more
interesting examples.  The simplest one implements an in-order
traversal of a tree using generators recursively.

\begin{verbatim}
# A recursive generator that generates Tree leaves in in-order.
def inorder(t):
    if t:
        for x in inorder(t.left):
            yield x
        yield t.label
        for x in inorder(t.right):
            yield x
\end{verbatim}

Two other examples in \file{Lib/test/test_generators.py} produce
solutions for the N-Queens problem (placing $N$ queens on an $NxN$
chess board so that no queen threatens another) and the Knight's Tour
(a route that takes a knight to every square of an $NxN$ chessboard
without visiting any square twice). 

The idea of generators comes from other programming languages,
especially Icon (\url{http://www.cs.arizona.edu/icon/}), where the
idea of generators is central.  In Icon, every
expression and function call behaves like a generator.  One example
from ``An Overview of the Icon Programming Language'' at
\url{http://www.cs.arizona.edu/icon/docs/ipd266.htm} gives an idea of
what this looks like:

\begin{verbatim}
sentence := "Store it in the neighboring harbor"
if (i := find("or", sentence)) > 5 then write(i)
\end{verbatim}

In Icon the \function{find()} function returns the indexes at which the
substring ``or'' is found: 3, 23, 33.  In the \keyword{if} statement,
\code{i} is first assigned a value of 3, but 3 is less than 5, so the
comparison fails, and Icon retries it with the second value of 23.  23
is greater than 5, so the comparison now succeeds, and the code prints
the value 23 to the screen.

Python doesn't go nearly as far as Icon in adopting generators as a
central concept.  Generators are considered a new part of the core
Python language, but learning or using them isn't compulsory; if they
don't solve any problems that you have, feel free to ignore them.
One novel feature of Python's interface as compared to
Icon's is that a generator's state is represented as a concrete object
(the iterator) that can be passed around to other functions or stored
in a data structure.

\begin{seealso}

\seepep{255}{Simple Generators}{Written by Neil Schemenauer, Tim
Peters, Magnus Lie Hetland.  Implemented mostly by Neil Schemenauer
and Tim Peters, with other fixes from the Python Labs crew.}

\end{seealso}


%======================================================================
\section{PEP 263: Source Code Encodings \label{section-encodings}}

Python source files can now be declared as being in different
character set encodings.  Encodings are declared by including a
specially formatted comment in the first or second line of the source
file.  For example, a UTF-8 file can be declared with:

\begin{verbatim}
#!/usr/bin/env python
# -*- coding: UTF-8 -*-
\end{verbatim}

Without such an encoding declaration, the default encoding used is
ISO-8859-1, also known as Latin1.  

The encoding declaration only affects Unicode string literals; the
text in the source code will be converted to Unicode using the
specified encoding.  Note that Python identifiers are still restricted
to ASCII characters, so you can't have variable names that use
characters outside of the usual alphanumerics.

\begin{seealso}

\seepep{263}{Defining Python Source Code Encodings}{Written by
Marc-Andr\'e Lemburg and Martin von L\"owis; implemented by SUZUKI
Hisao and Martin von L\"owis.}

\end{seealso}


%======================================================================
\section{PEP 277: Unicode file name support for Windows NT}

On Windows NT, 2000, and XP, the system stores file names as Unicode
strings. Traditionally, Python has represented file names as byte
strings, which is inadequate because it renders some file names
inaccessible.

Python now allows using arbitrary Unicode strings (within the
limitations of the file system) for all functions that expect file
names, in particular the \function{open()} built-in. If a Unicode
string is passed to \function{os.listdir}, Python now returns a list
of Unicode strings.  A new function, \function{os.getcwdu()}, returns
the current directory as a Unicode string.

Byte strings still work as file names, and Python will transparently
convert them to Unicode using the \code{mbcs} encoding.

Other systems also allow Unicode strings as file names, but convert
them to byte strings before passing them to the system which may cause
a \exception{UnicodeError} to be raised. Applications can test whether
arbitrary Unicode strings are supported as file names by checking
\member{os.path.unicode_file_names}, a Boolean value.

\begin{seealso}

\seepep{277}{Unicode file name support for Windows NT}{Written by Neil
Hodgson; implemented by Neil Hodgson, Martin von L\"owis, and Mark
Hammond.}

\end{seealso}


%======================================================================
\section{PEP 278: Universal Newline Support}

The three major operating systems used today are Microsoft Windows,
Apple's Macintosh OS, and the various \UNIX\ derivatives.  A minor
irritation is that these three platforms all use different characters
to mark the ends of lines in text files.  \UNIX\ uses character 10,
the ASCII linefeed, while MacOS uses character 13, the ASCII carriage
return, and Windows uses a two-character sequence of a carriage return
plus a newline.

Python's file objects can now support end of line conventions other
than the one followed by the platform on which Python is running.
Opening a file with the mode \samp{U} or \samp{rU} will open a file
for reading in universal newline mode.  All three line ending
conventions will be translated to a \samp{\e n} in the strings
returned by the various file methods such as \method{read()} and
\method{readline()}. 

Universal newline support is also used when importing modules and when
executing a file with the \function{execfile()} function.  This means
that Python modules can be shared between all three operating systems
without needing to convert the line-endings.

This feature can be disabled at compile-time by specifying 
\longprogramopt{without-universal-newlines} when running Python's
\file{configure} script.

\begin{seealso}

\seepep{278}{Universal Newline Support}{Written 
and implemented by Jack Jansen.}

\end{seealso}


%======================================================================
\section{PEP 279: The \function{enumerate()} Built-in Function\label{section-enumerate}}

A new built-in function, \function{enumerate()}, will make
certain loops a bit clearer.  \code{enumerate(thing)}, where
\var{thing} is either an iterator or a sequence, returns a iterator
that will return \code{(0, \var{thing[0]})}, \code{(1,
\var{thing[1]})}, \code{(2, \var{thing[2]})}, and so forth.  Fairly
often you'll see code to change every element of a list that looks
like this:

\begin{verbatim}
for i in range(len(L)):
    item = L[i]
    # ... compute some result based on item ...
    L[i] = result
\end{verbatim}

This can be rewritten using \function{enumerate()} as:

\begin{verbatim}
for i, item in enumerate(L):
    # ... compute some result based on item ...
    L[i] = result
\end{verbatim}


\begin{seealso}

\seepep{279}{The enumerate() built-in function}{Written 
by Raymond D. Hettinger.}

\end{seealso}


%======================================================================
\section{PEP 285: The \class{bool} Type\label{section-bool}}

A Boolean type was added to Python 2.3.  Two new constants were added
to the \module{__builtin__} module, \constant{True} and
\constant{False}.  The type object for this new type is named
\class{bool}; the constructor for it takes any Python value and
converts it to \constant{True} or \constant{False}.

\begin{verbatim}
>>> bool(1)
True
>>> bool(0)
False
>>> bool([])
False
>>> bool( (1,) )
True
\end{verbatim}

Most of the standard library modules and built-in functions have been
changed to return Booleans.

\begin{verbatim}
>>> obj = []
>>> hasattr(obj, 'append')
True
>>> isinstance(obj, list)
True
>>> isinstance(obj, tuple)
False
\end{verbatim}

Python's Booleans were added with the primary goal of making code
clearer.  For example, if you're reading a function and encounter the
statement \code{return 1}, you might wonder whether the \samp{1}
represents a truth value, or whether it's an index, or whether it's a
coefficient that multiplies some other quantity.  If the statement is
\code{return True}, however, the meaning of the return value is quite
clearly a truth value.

Python's Booleans were not added for the sake of strict type-checking.
A very strict language such as Pascal would also prevent you
performing arithmetic with Booleans, and would require that the
expression in an \keyword{if} statement always evaluate to a Boolean.
Python is not this strict, and it never will be.  (\pep{285}
explicitly says so.)  So you can still use any expression in an
\keyword{if}, even ones that evaluate to a list or tuple or some
random object, and the Boolean type is a subclass of the
\class{int} class, so arithmetic using a Boolean still works.

\begin{verbatim}
>>> True + 1
2
>>> False + 1
1
>>> False * 75
0
>>> True * 75
75
\end{verbatim}

To sum up \constant{True} and \constant{False} in a sentence: they're
alternative ways to spell the integer values 1 and 0, with the single
difference that \function{str()} and \function{repr()} return the
strings \samp{True} and \samp{False} instead of \samp{1} and \samp{0}.

\begin{seealso}

\seepep{285}{Adding a bool type}{Written and implemented by GvR.}

\end{seealso}


%======================================================================
\section{PEP 293: Codec Error Handling Callbacks}

When encoding a Unicode string into a byte string, unencodable
characters may be encountered.  So far, Python has allowed specifying
the error processing as either ``strict'' (raising
\exception{UnicodeError}), ``ignore'' (skip the character), or
``replace'' (with question mark), defaulting to ``strict''. It may be
desirable to specify an alternative processing of the error, e.g. by
inserting an XML character reference or HTML entity reference into the
converted string.

Python now has a flexible framework to add additional processing
strategies.  New error handlers can be added with
\function{codecs.register_error}. Codecs then can access the error
handler with \function{codecs.lookup_error}. An equivalent C API has
been added for codecs written in C. The error handler gets the
necessary state information, such as the string being converted, the
position in the string where the error was detected, and the target
encoding.  The handler can then either raise an exception, or return a
replacement string.

Two additional error handlers have been implemented using this
framework: ``backslashreplace'' uses Python backslash quoting to
represent the unencodable character, and ``xmlcharrefreplace'' emits
XML character references.

\begin{seealso}

\seepep{293}{Codec Error Handling Callbacks}{Written and implemented by 
Walter D\"orwald.}

\end{seealso}


%======================================================================
\section{Extended Slices\label{section-slices}}

Ever since Python 1.4, the slicing syntax has supported an optional
third ``step'' or ``stride'' argument.  For example, these are all
legal Python syntax: \code{L[1:10:2]}, \code{L[:-1:1]},
\code{L[::-1]}.  This was added to Python included at the request of
the developers of Numerical Python.  However, the built-in sequence
types of lists, tuples, and strings have never supported this feature,
and you got a \exception{TypeError} if you tried it.  Michael Hudson
contributed a patch that was applied to Python 2.3 and fixed this 
shortcoming.

For example, you can now easily extract the elements of a list that
have even indexes:

\begin{verbatim}
>>> L = range(10)
>>> L[::2]
[0, 2, 4, 6, 8]
\end{verbatim}

Negative values also work, so you can make a copy of the same list in
reverse order:

\begin{verbatim}
>>> L[::-1]
[9, 8, 7, 6, 5, 4, 3, 2, 1, 0]
\end{verbatim}

This also works for strings:

\begin{verbatim}
>>> s='abcd'
>>> s[::2]
'ac'
>>> s[::-1]
'dcba'
\end{verbatim}

as well as tuples and arrays.

If you have a mutable sequence (i.e. a list or an array) you can
assign to or delete an extended slice, but there are some differences
in assignment to extended and regular slices.  Assignment to a regular
slice can be used to change the length of the sequence:

\begin{verbatim}
>>> a = range(3)
>>> a
[0, 1, 2]
>>> a[1:3] = [4, 5, 6]
>>> a
[0, 4, 5, 6]
\end{verbatim}

but when assigning to an extended slice the list on the right hand
side of the statement must contain the same number of items as the
slice it is replacing:

\begin{verbatim}
>>> a = range(4)
>>> a
[0, 1, 2, 3]
>>> a[::2]
[0, 2]
>>> a[::2] = range(0, -2, -1)
>>> a
[0, 1, -1, 3]
>>> a[::2] = range(3)
Traceback (most recent call last):
  File "<stdin>", line 1, in ?
ValueError: attempt to assign list of size 3 to extended slice of size 2
\end{verbatim}

Deletion is more straightforward:

\begin{verbatim}
>>> a = range(4)
>>> a[::2]
[0, 2]
>>> del a[::2]
>>> a
[1, 3]
\end{verbatim}

One can also now pass slice objects to builtin sequences
\method{__getitem__} methods:

\begin{verbatim}
>>> range(10).__getitem__(slice(0, 5, 2))
[0, 2, 4]
\end{verbatim}

or use them directly in subscripts:

\begin{verbatim}
>>> range(10)[slice(0, 5, 2)]
[0, 2, 4]
\end{verbatim}

To make implementing sequences that support extended slicing in Python
easier, slice ojects now have a method \method{indices} which given
the length of a sequence returns \code{(start, stop, step)} handling
omitted and out-of-bounds indices in a manner consistent with regular
slices (and this innocuous phrase hides a welter of confusing
details!).  The method is intended to be used like this:

\begin{verbatim}
class FakeSeq:
    ...
    def calc_item(self, i):
        ...
    def __getitem__(self, item):
        if isinstance(item, slice):
            return FakeSeq([self.calc_item(i) 
                            in range(*item.indices(len(self)))])
	else:
            return self.calc_item(i)
\end{verbatim}

From this example you can also see that the builtin ``\class{slice}''
object is now the type object for the slice type, and is no longer a
function.  This is consistent with Python 2.2, where \class{int},
\class{str}, etc., underwent the same change.


%======================================================================
\section{Other Language Changes}

Here are all of the changes that Python 2.3 makes to the core Python
language.

\begin{itemize}
\item The \keyword{yield} statement is now always a keyword, as
described in section~\ref{section-generators} of this document.

\item A new built-in function \function{enumerate()} 
was added, as described in section~\ref{section-enumerate} of this
document.

\item Two new constants, \constant{True} and \constant{False} were
added along with the built-in \class{bool} type, as described in
section~\ref{section-bool} of this document.

\item Built-in types now support the extended slicing syntax, 
as described in section~\ref{section-slices} of this document.

\item Dictionaries have a new method, \method{pop(\var{key})}, that
returns the value corresponding to \var{key} and removes that
key/value pair from the dictionary.  \method{pop()} will raise a
\exception{KeyError} if the requested key isn't present in the
dictionary:

\begin{verbatim}
>>> d = {1:2}
>>> d
{1: 2}
>>> d.pop(4)
Traceback (most recent call last):
  File ``stdin'', line 1, in ?
KeyError: 4
>>> d.pop(1)
2
>>> d.pop(1)
Traceback (most recent call last):
  File ``stdin'', line 1, in ?
KeyError: pop(): dictionary is empty
>>> d
{}
>>>
\end{verbatim}

(Patch contributed by Raymond Hettinger.)

\item The \keyword{assert} statement no longer  checks the \code{__debug__}
flag, so you can no longer disable assertions by assigning to \code{__debug__}.
Running Python with the \programopt{-O} switch will still generate 
code that doesn't execute any assertions.

\item Most type objects are now callable, so you can use them
to create new objects such as functions, classes, and modules.  (This
means that the \module{new} module can be deprecated in a future
Python version, because you can now use the type objects available
in the \module{types} module.)
% XXX should new.py use PendingDeprecationWarning?
For example, you can create a new module object with the following code:

\begin{verbatim}
>>> import types
>>> m = types.ModuleType('abc','docstring')
>>> m
<module 'abc' (built-in)>
>>> m.__doc__
'docstring'
\end{verbatim}

\item 
A new warning, \exception{PendingDeprecationWarning} was added to
indicate features which are in the process of being
deprecated.  The warning will \emph{not} be printed by default.  To
check for use of features that will be deprecated in the future,
supply \programopt{-Walways::PendingDeprecationWarning::} on the
command line or use \function{warnings.filterwarnings()}.

\item Using \code{None} as a variable name will now result in a
\exception{SyntaxWarning} warning.  In a future version of Python,
\code{None} may finally become a keyword.

\item Python runs multithreaded programs by switching between threads
after executing N bytecodes.  The default value for N has been
increased from 10 to 100 bytecodes, speeding up single-threaded
applications by reducing the switching overhead.  Some multithreaded
applications may suffer slower response time, but that's easily fixed
by setting the limit back to a lower number by calling
\function{sys.setcheckinterval(\var{N})}.

\item One minor but far-reaching change is that the names of extension
types defined by the modules included with Python now contain the
module and a \samp{.} in front of the type name.  For example, in
Python 2.2, if you created a socket and printed its
\member{__class__}, you'd get this output:

\begin{verbatim}
>>> s = socket.socket()
>>> s.__class__
<type 'socket'>
\end{verbatim}

In 2.3, you get this:
\begin{verbatim}
>>> s.__class__
<type '_socket.socket'>
\end{verbatim}

\end{itemize}


\subsection{String Changes}

\begin{itemize}

\item The \code{in} operator now works differently for strings.
Previously, when evaluating \code{\var{X} in \var{Y}} where \var{X}
and \var{Y} are strings, \var{X} could only be a single character.
That's now changed; \var{X} can be a string of any length, and
\code{\var{X} in \var{Y}} will return \constant{True} if \var{X} is a
substring of \var{Y}.  If \var{X} is the empty string, the result is
always \constant{True}.

\begin{verbatim}
>>> 'ab' in 'abcd'
True
>>> 'ad' in 'abcd'
False
>>> '' in 'abcd'
True
\end{verbatim}

Note that this doesn't tell you where the substring starts; the
\method{find()} method is still necessary to figure that out.

\item The \method{strip()}, \method{lstrip()}, and \method{rstrip()}
string methods now have an optional argument for specifying the
characters to strip.  The default is still to remove all whitespace
characters:

\begin{verbatim}
>>> '   abc '.strip()
'abc'
>>> '><><abc<><><>'.strip('<>')
'abc'
>>> '><><abc<><><>\n'.strip('<>')
'abc<><><>\n'
>>> u'\u4000\u4001abc\u4000'.strip(u'\u4000')
u'\u4001abc'
>>>
\end{verbatim}

(Contributed by Simon Brunning.)

\item The \method{startswith()} and \method{endswith()}
string methods now accept negative numbers for the start and end
parameters.

\item Another new string method is \method{zfill()}, originally a
function in the \module{string} module.  \method{zfill()} pads a
numeric string with zeros on the left until it's the specified width.
Note that the \code{\%} operator is still more flexible and powerful
than \method{zfill()}.

\begin{verbatim}
>>> '45'.zfill(4)
'0045'
>>> '12345'.zfill(4)
'12345'
>>> 'goofy'.zfill(6)
'0goofy'
\end{verbatim}

(Contributed by Walter D\"orwald.)

\item A new type object, \class{basestring}, has been added.  
   Both 8-bit strings and Unicode strings inherit from this type, so
   \code{isinstance(obj, basestring)} will return \constant{True} for
   either kind of string.  It's a completely abstract type, so you
   can't create \class{basestring} instances.

\item Interned strings are no longer immortal.  Interned will now be
garbage-collected in the usual way when the only reference to them is
from the internal dictionary of interned strings.  (Implemented by
Oren Tirosh.)

\end{itemize}


\subsection{Optimizations}

\begin{itemize}

\item The \method{sort()} method of list objects has been extensively
rewritten by Tim Peters, and the implementation is significantly
faster.

\item Multiplication of large long integers is now much faster thanks
to an implementation of Karatsuba multiplication, an algorithm that
scales better than the O(n*n) required for the grade-school
multiplication algorithm.  (Original patch by Christopher A. Craig,
and significantly reworked by Tim Peters.)

\item The \code{SET_LINENO} opcode is now gone.  This may provide a
small speed increase, subject to your compiler's idiosyncrasies.
(Removed by Michael Hudson.)

\item A number of small rearrangements have been made in various
hotspots to improve performance, inlining a function here, removing
some code there.  (Implemented mostly by GvR, but lots of people have
contributed to one change or another.)

\end{itemize}


%======================================================================
\section{New and Improved Modules}

As usual, Python's standard modules had a number of enhancements and
bug fixes.  Here's a partial list of the most notable changes, sorted
alphabetically by module name. Consult the
\file{Misc/NEWS} file in the source tree for a more
complete list of changes, or look through the CVS logs for all the
details.

\begin{itemize}

\item The \module{array} module now supports arrays of Unicode
characters using the \samp{u} format character.  Arrays also now
support using the \code{+=} assignment operator to add another array's
contents, and the \code{*=} assignment operator to repeat an array.
(Contributed by Jason Orendorff.)

\item The Distutils \class{Extension} class now supports 
an extra constructor argument named \samp{depends} for listing
additional source files that an extension depends on.  This lets
Distutils recompile the module if any of the dependency files are
modified.  For example, if \samp{sampmodule.c} includes the header
file \file{sample.h}, you would create the \class{Extension} object like
this:

\begin{verbatim}
ext = Extension("samp",
                sources=["sampmodule.c"],
                depends=["sample.h"])
\end{verbatim}

Modifying \file{sample.h} would then cause the module to be recompiled.
(Contributed by Jeremy Hylton.)

\item The \module{getopt} module gained a new function,
\function{gnu_getopt()}, that supports the same arguments as the existing
\function{getopt()} function but uses GNU-style scanning mode. 
The existing \function{getopt()} stops processing options as soon as a
non-option argument is encountered, but in GNU-style mode processing
continues, meaning that options and arguments can be mixed.  For
example:

\begin{verbatim}
>>> getopt.getopt(['-f', 'filename', 'output', '-v'], 'f:v')
([('-f', 'filename')], ['output', '-v'])
>>> getopt.gnu_getopt(['-f', 'filename', 'output', '-v'], 'f:v')
([('-f', 'filename'), ('-v', '')], ['output'])
\end{verbatim}

(Contributed by Peter \AA{strand}.)

\item The \module{grp}, \module{pwd}, and \module{resource} modules
now return enhanced tuples: 

\begin{verbatim}
>>> import grp
>>> g = grp.getgrnam('amk')
>>> g.gr_name, g.gr_gid
('amk', 500)
\end{verbatim}

\item The new \module{heapq} module contains an implementation of a
heap queue algorithm.  A heap is an array-like data structure that
keeps items in a sorted order such that, for every index k, heap[k] <=
heap[2*k+1] and heap[k] <= heap[2*k+2].  This makes it quick to remove
the smallest item, and inserting a new item while maintaining the heap
property is O(lg~n).  (See
\url{http://www.nist.gov/dads/HTML/priorityque.html} for more
information about the priority queue data structure.)

The Python \module{heapq} module provides \function{heappush()} and
\function{heappop()} functions for adding and removing items while
maintaining the heap property on top of some other mutable Python
sequence type.  For example:

\begin{verbatim}
>>> import heapq
>>> heap = []
>>> for item in [3, 7, 5, 11, 1]:
...    heapq.heappush(heap, item)
...
>>> heap
[1, 3, 5, 11, 7]
>>> heapq.heappop(heap)
1
>>> heapq.heappop(heap)
3
>>> heap
[5, 7, 11]
>>>
>>> heapq.heappush(heap, 5)
>>> heap = []
>>> for item in [3, 7, 5, 11, 1]:
...    heapq.heappush(heap, item)
...
>>> heap
[1, 3, 5, 11, 7]
>>> heapq.heappop(heap)
1
>>> heapq.heappop(heap)
3
>>> heap
[5, 7, 11]
>>>
\end{verbatim}

(Contributed by Kevin O'Connor.)

\item Two new functions in the \module{math} module, 
\function{degrees(\var{rads})} and \function{radians(\var{degs})},
convert between radians and degrees.  Other functions in the 
\module{math} module such as
\function{math.sin()} and \function{math.cos()} have always required
input values measured in radians. (Contributed by Raymond Hettinger.)

\item Seven new functions, \function{getpgid()}, \function{killpg()},
\function{lchown()}, \function{major()}, \function{makedev()},
\function{minor()}, and \function{mknod()}, were added to the
\module{posix} module that underlies the \module{os} module.
(Contributed by Gustavo Niemeyer and Geert Jansen.)

\item The parser objects provided by the \module{pyexpat} module 
can now optionally buffer character data, resulting in fewer calls to
your character data handler and therefore faster performance.  Setting
the parser object's \member{buffer_text} attribute to \constant{True} 
will enable buffering.

\item The \module{readline} module also gained a number of new
functions: \function{get_history_item()},
\function{get_current_history_length()}, and \function{redisplay()}.

\item Support for more advanced POSIX signal handling was added
to the \module{signal} module by adding the \function{sigpending},
\function{sigprocmask} and \function{sigsuspend} functions, where supported
by the platform.  These functions make it possible to avoid some previously
unavoidable race conditions.

\item The \module{socket} module now supports timeouts.  You
can call the \method{settimeout(\var{t})} method on a socket object to
set a timeout of \var{t} seconds.  Subsequent socket operations that
take longer than \var{t} seconds to complete will abort and raise a
\exception{socket.error} exception.  

The original timeout implementation was by Tim O'Malley.  Michael
Gilfix integrated it into the Python \module{socket} module, after the
patch had undergone a lengthy review.  After it was checked in, Guido
van~Rossum rewrote parts of it.  This is a good example of the free
software development process in action.

\item The value of the C \constant{PYTHON_API_VERSION} macro is now exposed 
at the Python level as \code{sys.api_version}.

\item The new \module{textwrap} module contains functions for wrapping
strings containing paragraphs of text.  The \function{wrap(\var{text},
\var{width})} function takes a string and returns a list containing
the text split into lines of no more than the chosen width.  The
\function{fill(\var{text}, \var{width})} function returns a single
string, reformatted to fit into lines no longer than the chosen width.
(As you can guess, \function{fill()} is built on top of
\function{wrap()}.  For example:

\begin{verbatim}
>>> import textwrap
>>> paragraph = "Not a whit, we defy augury: ... more text ..."
>>> textwrap.wrap(paragraph, 60)
["Not a whit, we defy augury: there's a special providence in", 
 "the fall of a sparrow. If it be now, 'tis not to come; if it", 
 ...]
>>> print textwrap.fill(paragraph, 35)
Not a whit, we defy augury: there's
a special providence in the fall of
a sparrow. If it be now, 'tis not
to come; if it be not to come, it
will be now; if it be not now, yet
it will come: the readiness is all.
>>> 
\end{verbatim}

The module also contains a \class{TextWrapper} class that actually
implements the text wrapping strategy.   Both the 
\class{TextWrapper} class and the \function{wrap()} and
\function{fill()} functions support a number of additional keyword
arguments for fine-tuning the formatting; consult the module's
documentation for details. 
% XXX add a link to the module docs?
(Contributed by Greg Ward.)

\item The \module{time} module's \function{strptime()} function has
long been an annoyance because it uses the platform C library's 
\function{strptime()} implementation, and different platforms
sometimes have odd bugs.  Brett Cannon contributed a portable
implementation that's written in pure Python, which should behave
identically on all platforms.

\item The DOM implementation
in \module{xml.dom.minidom} can now generate XML output in a
particular encoding, by specifying an optional encoding argument to
the \method{toxml()} and \method{toprettyxml()} methods of DOM nodes.

\item The \function{stat} family of functions can now report fractions
of a second in a time stamp. Similar to \function{time.time}, such
time stamps are represented as floats.

During testing, it was found that some applications break if time
stamps are floats. For compatibility, when using the tuple interface
of the \class{stat_result}, time stamps are represented as integers.
When using named fields (first introduced in Python 2.2), time stamps
are still represented as ints, unless \function{os.stat_float_times}
is invoked:

\begin{verbatim}
>>> os.stat_float_times(True)
>>> os.stat("/tmp").st_mtime
1034791200.6335014
\end{verbatim}

In Python 2.4, the default will change to return floats.

Application developers should use this feature only if all their
libraries work properly when confronted with floating point time
stamps (or use the tuple API). If used, the feature should be
activated on application level, instead of trying to activate it on a
per-use basis.

\end{itemize}


%======================================================================
\section{Specialized Object Allocator (pymalloc)\label{section-pymalloc}}

An experimental feature added to Python 2.1 was a specialized object
allocator called pymalloc, written by Vladimir Marangozov.  Pymalloc
was intended to be faster than the system \cfunction{malloc()} and have
less memory overhead for typical allocation patterns of Python
programs.  The allocator uses C's \cfunction{malloc()} function to get
large pools of memory, and then fulfills smaller memory requests from
these pools.

In 2.1 and 2.2, pymalloc was an experimental feature and wasn't
enabled by default; you had to explicitly turn it on by providing the
\longprogramopt{with-pymalloc} option to the \program{configure}
script.  In 2.3, pymalloc has had further enhancements and is now
enabled by default; you'll have to supply
\longprogramopt{without-pymalloc} to disable it.

This change is transparent to code written in Python; however,
pymalloc may expose bugs in C extensions.  Authors of C extension
modules should test their code with the object allocator enabled,
because some incorrect code may cause core dumps at runtime.  There
are a bunch of memory allocation functions in Python's C API that have
previously been just aliases for the C library's \cfunction{malloc()}
and \cfunction{free()}, meaning that if you accidentally called
mismatched functions, the error wouldn't be noticeable.  When the
object allocator is enabled, these functions aren't aliases of
\cfunction{malloc()} and \cfunction{free()} any more, and calling the
wrong function to free memory may get you a core dump.  For example,
if memory was allocated using \cfunction{PyObject_Malloc()}, it has to
be freed using \cfunction{PyObject_Free()}, not \cfunction{free()}.  A
few modules included with Python fell afoul of this and had to be
fixed; doubtless there are more third-party modules that will have the
same problem.

As part of this change, the confusing multiple interfaces for
allocating memory have been consolidated down into two API families.
Memory allocated with one family must not be manipulated with
functions from the other family.

There is another family of functions specifically for allocating
Python \emph{objects} (as opposed to memory).

\begin{itemize}
  \item To allocate and free an undistinguished chunk of memory use
  the ``raw memory'' family: \cfunction{PyMem_Malloc()},
  \cfunction{PyMem_Realloc()}, and \cfunction{PyMem_Free()}.

  \item The ``object memory'' family is the interface to the pymalloc
  facility described above and is biased towards a large number of
  ``small'' allocations: \cfunction{PyObject_Malloc},
  \cfunction{PyObject_Realloc}, and \cfunction{PyObject_Free}.

  \item To allocate and free Python objects, use the ``object'' family
  \cfunction{PyObject_New()}, \cfunction{PyObject_NewVar()}, and
  \cfunction{PyObject_Del()}.
\end{itemize}

Thanks to lots of work by Tim Peters, pymalloc in 2.3 also provides
debugging features to catch memory overwrites and doubled frees in
both extension modules and in the interpreter itself.  To enable this
support, turn on the Python interpreter's debugging code by running
\program{configure} with \longprogramopt{with-pydebug}.  

To aid extension writers, a header file \file{Misc/pymemcompat.h} is
distributed with the source to Python 2.3 that allows Python
extensions to use the 2.3 interfaces to memory allocation and compile
against any version of Python since 1.5.2.  You would copy the file
from Python's source distribution and bundle it with the source of
your extension.

\begin{seealso}

\seeurl{http://cvs.sourceforge.net/cgi-bin/viewcvs.cgi/python/python/dist/src/Objects/obmalloc.c}
{For the full details of the pymalloc implementation, see
the comments at the top of the file \file{Objects/obmalloc.c} in the
Python source code.  The above link points to the file within the
SourceForge CVS browser.}

\end{seealso}


% ======================================================================
\section{Build and C API Changes}

Changes to Python's build process and to the C API include:

\begin{itemize}

\item The C-level interface to the garbage collector has been changed,
to make it easier to write extension types that support garbage
collection, and to make it easier to debug misuses of the functions.
Various functions have slightly different semantics, so a bunch of
functions had to be renamed.  Extensions that use the old API will
still compile but will \emph{not} participate in garbage collection,
so updating them for 2.3 should be considered fairly high priority.

To upgrade an extension module to the new API, perform the following
steps:

\begin{itemize}

\item Rename \cfunction{Py_TPFLAGS_GC} to \cfunction{PyTPFLAGS_HAVE_GC}.

\item Use \cfunction{PyObject_GC_New} or \cfunction{PyObject_GC_NewVar} to
allocate objects, and \cfunction{PyObject_GC_Del} to deallocate them.

\item Rename \cfunction{PyObject_GC_Init} to \cfunction{PyObject_GC_Track} and
\cfunction{PyObject_GC_Fini} to \cfunction{PyObject_GC_UnTrack}.

\item Remove \cfunction{PyGC_HEAD_SIZE} from object size calculations.

\item Remove calls to \cfunction{PyObject_AS_GC} and \cfunction{PyObject_FROM_GC}.

\end{itemize}

\item Python can now optionally be built as a shared library
(\file{libpython2.3.so}) by supplying \longprogramopt{enable-shared}
when running Python's \file{configure} script.  (Contributed by Ondrej
Palkovsky.)

\item The \csimplemacro{DL_EXPORT} and \csimplemacro{DL_IMPORT} macros
are now deprecated.  Initialization functions for Python extension
modules should now be declared using the new macro
\csimplemacro{PyMODINIT_FUNC}, while the Python core will generally
use the \csimplemacro{PyAPI_FUNC} and \csimplemacro{PyAPI_DATA}
macros.

\item The interpreter can be compiled without any docstrings for 
the built-in functions and modules by supplying
\longprogramopt{without-doc-strings} to the \file{configure} script.
This makes the Python executable about 10\% smaller, but will also
mean that you can't get help for Python's built-ins.  (Contributed by
Gustavo Niemeyer.)

\item The cycle detection implementation used by the garbage collection
has proven to be stable, so it's now being made mandatory; you can no
longer compile Python without it, and the
\longprogramopt{with-cycle-gc} switch to \file{configure} has been removed.

\item The \cfunction{PyArg_NoArgs()} macro is now deprecated, and code
that uses it should be changed.  For Python 2.2 and later, the method
definition table can specify the
\constant{METH_NOARGS} flag, signalling that there are no arguments, and 
the argument checking can then be removed.  If compatibility with
pre-2.2 versions of Python is important, the code could use
\code{PyArg_ParseTuple(args, "")} instead, but this will be slower 
than using \constant{METH_NOARGS}.

\item A new function, \cfunction{PyObject_DelItemString(\var{mapping},
char *\var{key})} was added
as shorthand for 
\code{PyObject_DelItem(\var{mapping}, PyString_New(\var{key})}.

\item File objects now manage their internal string buffer
differently by increasing it exponentially when needed.  
This results in the benchmark tests in \file{Lib/test/test_bufio.py} 
speeding up from 57 seconds to 1.7 seconds, according to one
measurement.

\item It's now possible to define class and static methods for a C
extension type by setting either the \constant{METH_CLASS} or
\constant{METH_STATIC} flags in a method's \ctype{PyMethodDef}
structure.

\item Python now includes a copy of the Expat XML parser's source code,
removing any dependence on a system version or local installation of
Expat.  

\end{itemize}

\subsection{Port-Specific Changes}

Support for a port to IBM's OS/2 using the EMX runtime environment was
merged into the main Python source tree.  EMX is a POSIX emulation
layer over the OS/2 system APIs.  The Python port for EMX tries to
support all the POSIX-like capability exposed by the EMX runtime, and
mostly succeeds; \function{fork()} and \function{fcntl()} are
restricted by the limitations of the underlying emulation layer.  The
standard OS/2 port, which uses IBM's Visual Age compiler, also gained
support for case-sensitive import semantics as part of the integration
of the EMX port into CVS.  (Contributed by Andrew MacIntyre.)

On MacOS, most toolbox modules have been weaklinked to improve
backward compatibility.  This means that modules will no longer fail
to load if a single routine is missing on the curent OS version.
Instead calling the missing routine will raise an exception.
(Contributed by Jack Jansen.)

The RPM spec files, found in the \file{Misc/RPM/} directory in the
Python source distribution, were updated for 2.3.  (Contributed by
Sean Reifschneider.)

Python now supports AtheOS (\url{http://www.atheos.cx}) and GNU/Hurd.


%======================================================================
\section{Other Changes and Fixes}

Finally, there are various miscellaneous fixes:

\begin{itemize}

\item The tools used to build the documentation now work under Cygwin
as well as \UNIX.

\item The \code{SET_LINENO} opcode has been removed.  Back in the
mists of time, this opcode was needed to produce line numbers in
tracebacks and support trace functions (for, e.g., \module{pdb}).
Since Python 1.5, the line numbers in tracebacks have been computed
using a different mechanism that works with ``python -O''.  For Python
2.3 Michael Hudson implemented a similar scheme to determine when to
call the trace function, removing the need for \code{SET_LINENO}
entirely.

Python code will be hard pushed to notice a difference from this
change, apart from a slight speed up when python is run without
\programopt{-O}.

C extensions that access the \member{f_lineno} field of frame objects
should instead call \code{PyCode_Addr2Line(f->f_code, f->f_lasti)}.
This will have the added effect of making the code work as desired
under ``python -O'' in earlier versions of Python.

\end{itemize}


%======================================================================
\section{Porting to Python 2.3}

XXX write this


%======================================================================
\section{Acknowledgements \label{acks}}

The author would like to thank the following people for offering
suggestions, corrections and assistance with various drafts of this
article: Michael Chermside, Scott David Daniels, Fred~L. Drake, Jr.,
Michael Hudson, Detlef Lannert, Martin von L\"owis, Andrew MacIntyre,
Lalo Martins, Gustavo Niemeyer, Neal Norwitz, Jason Tishler.

\end{document}
