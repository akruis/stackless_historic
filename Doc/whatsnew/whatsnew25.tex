\documentclass{howto}
\usepackage{distutils}
% $Id$

% Fix XXX comments
% Distutils upload (PEP 243)
% The easy_install stuff

\title{What's New in Python 2.5}
\release{0.1}
\author{A.M. Kuchling}
\authoraddress{\email{amk@amk.ca}}

\begin{document}
\maketitle
\tableofcontents

This article explains the new features in Python 2.5.  No release date
for Python 2.5 has been set; it will probably be released in the
autumn of 2006.  \pep{356} describes the planned release schedule.

(This is still an early draft, and some sections are still skeletal or
completely missing.  Comments on the present material will still be
welcomed.)

% XXX Compare with previous release in 2 - 3 sentences here.

This article doesn't attempt to provide a complete specification of
the new features, but instead provides a convenient overview.  For
full details, you should refer to the documentation for Python 2.5.
% XXX add hyperlink when the documentation becomes available online.
If you want to understand the complete implementation and design
rationale, refer to the PEP for a particular new feature.


%======================================================================
\section{PEP 308: Conditional Expressions}

For a long time, people have been requesting a way to write
conditional expressions, expressions that return value A or value B
depending on whether a Boolean value is true or false.  A conditional
expression lets you write a single assignment statement that has the
same effect as the following:

\begin{verbatim}
if condition:
    x = true_value
else:
    x = false_value
\end{verbatim}

There have been endless tedious discussions of syntax on both
python-dev and comp.lang.python.  A vote was even held that found the
majority of voters wanted conditional expressions in some form,
but there was no syntax that was preferred by a clear majority.
Candidates included C's \code{cond ? true_v : false_v},
\code{if cond then true_v else false_v}, and 16 other variations.

GvR eventually chose a surprising syntax:

\begin{verbatim}
x = true_value if condition else false_value
\end{verbatim}

Evaluation is still lazy as in existing Boolean expressions, so the
order of evaluation jumps around a bit.  The \var{condition}
expression in the middle is evaluated first, and the \var{true_value}
expression is evaluated only if the condition was true.  Similarly,
the \var{false_value} expression is only evaluated when the condition
is false.

This syntax may seem strange and backwards; why does the condition go
in the \emph{middle} of the expression, and not in the front as in C's
\code{c ? x : y}?  The decision was checked by applying the new syntax
to the modules in the standard library and seeing how the resulting
code read.  In many cases where a conditional expression is used, one
value seems to be the 'common case' and one value is an 'exceptional
case', used only on rarer occasions when the condition isn't met.  The
conditional syntax makes this pattern a bit more obvious:

\begin{verbatim}
contents = ((doc + '\n') if doc else '')
\end{verbatim}

I read the above statement as meaning ``here \var{contents} is 
usually assigned a value of \code{doc+'\e n'}; sometimes 
\var{doc} is empty, in which special case an empty string is returned.''  
I doubt I will use conditional expressions very often where there 
isn't a clear common and uncommon case.

There was some discussion of whether the language should require
surrounding conditional expressions with parentheses.  The decision
was made to \emph{not} require parentheses in the Python language's
grammar, but as a matter of style I think you should always use them.
Consider these two statements:

\begin{verbatim}
# First version -- no parens
level = 1 if logging else 0

# Second version -- with parens
level = (1 if logging else 0)
\end{verbatim}

In the first version, I think a reader's eye might group the statement
into 'level = 1', 'if logging', 'else 0', and think that the condition
decides whether the assignment to \var{level} is performed.  The
second version reads better, in my opinion, because it makes it clear
that the assignment is always performed and the choice is being made
between two values.

Another reason for including the brackets: a few odd combinations of
list comprehensions and lambdas could look like incorrect conditional
expressions. See \pep{308} for some examples.  If you put parentheses
around your conditional expressions, you won't run into this case.


\begin{seealso}

\seepep{308}{Conditional Expressions}{PEP written by
Guido van Rossum and Raymond D. Hettinger; implemented by Thomas
Wouters.}

\end{seealso}


%======================================================================
\section{PEP 309: Partial Function Application}

The \module{functional} module is intended to contain tools for
functional-style programming.  Currently it only contains a
\class{partial()} function, but new functions will probably be added
in future versions of Python.

For programs written in a functional style, it can be useful to
construct variants of existing functions that have some of the
parameters filled in.  Consider a Python function \code{f(a, b, c)};
you could create a new function \code{g(b, c)} that was equivalent to
\code{f(1, b, c)}.  This is called ``partial function application'',
and is provided by the \class{partial} class in the new
\module{functional} module.

The constructor for \class{partial} takes the arguments
\code{(\var{function}, \var{arg1}, \var{arg2}, ...
\var{kwarg1}=\var{value1}, \var{kwarg2}=\var{value2})}.  The resulting
object is callable, so you can just call it to invoke \var{function}
with the filled-in arguments.

Here's a small but realistic example:

\begin{verbatim}
import functional

def log (message, subsystem):
    "Write the contents of 'message' to the specified subsystem."
    print '%s: %s' % (subsystem, message)
    ...

server_log = functional.partial(log, subsystem='server')
server_log('Unable to open socket')
\end{verbatim}

Here's another example, from a program that uses PyGTk.  Here a
context-sensitive pop-up menu is being constructed dynamically.  The
callback provided for the menu option is a partially applied version
of the \method{open_item()} method, where the first argument has been
provided.

\begin{verbatim}
...
class Application:
    def open_item(self, path):
       ...
    def init (self):
        open_func = functional.partial(self.open_item, item_path)
        popup_menu.append( ("Open", open_func, 1) )
\end{verbatim}


\begin{seealso}

\seepep{309}{Partial Function Application}{PEP proposed and written by
Peter Harris; implemented by Hye-Shik Chang, with adaptations by
Raymond Hettinger.}

\end{seealso}


%======================================================================
\section{PEP 314: Metadata for Python Software Packages v1.1}

Some simple dependency support was added to Distutils.  The
\function{setup()} function now has \code{requires}, \code{provides},
and \code{obsoletes} keyword parameters.  When you build a source
distribution using the \code{sdist} command, the dependency
information will be recorded in the \file{PKG-INFO} file.

Another new keyword parameter is \code{download_url}, which should be
set to a URL for the package's source code.  This means it's now
possible to look up an entry in the package index, determine the
dependencies for a package, and download the required packages.

% XXX put example here
 
\begin{seealso}

\seepep{314}{Metadata for Python Software Packages v1.1}{PEP proposed
and written by A.M. Kuchling, Richard Jones, and Fred Drake; 
implemented by Richard Jones and Fred Drake.}

\end{seealso}


%======================================================================
\section{PEP 328: Absolute and Relative Imports}

The simpler part of PEP 328 was implemented in Python 2.4: parentheses
could now be used to enclose the names imported from a module using
the \code{from ... import ...} statement, making it easier to import
many different names.

The more complicated part has been implemented in Python 2.5:
importing a module can be specified to use absolute or
package-relative imports.  The plan is to move toward making absolute
imports the default in future versions of Python.

Let's say you have a package directory like this:
\begin{verbatim}
pkg/
pkg/__init__.py
pkg/main.py
pkg/string.py
\end{verbatim}

This defines a package named \module{pkg} containing the
\module{pkg.main} and \module{pkg.string} submodules.  

Consider the code in the \file{main.py} module.  What happens if it
executes the statement \code{import string}?  In Python 2.4 and
earlier, it will first look in the package's directory to perform a
relative import, finds \file{pkg/string.py}, imports the contents of
that file as the \module{pkg.string} module, and that module is bound
to the name \samp{string} in the \module{pkg.main} module's namespace.

That's fine if \module{pkg.string} was what you wanted.  But what if
you wanted Python's standard \module{string} module?  There's no clean
way to ignore \module{pkg.string} and look for the standard module;
generally you had to look at the contents of \code{sys.modules}, which
is slightly unclean.   
Holger Krekel's \module{py.std} package provides a tidier way to perform
imports from the standard library, \code{import py ; py.std.string.join()},
but that package isn't available on all Python installations.

Reading code which relies on relative imports is also less clear,
because a reader may be confused about which module, \module{string}
or \module{pkg.string}, is intended to be used.  Python users soon
learned not to duplicate the names of standard library modules in the
names of their packages' submodules, but you can't protect against
having your submodule's name being used for a new module added in a
future version of Python.

In Python 2.5, you can switch \keyword{import}'s behaviour to 
absolute imports using a \code{from __future__ import absolute_import}
directive.  This absolute-import behaviour will become the default in
a future version (probably Python 2.7).  Once absolute imports 
are the default, \code{import string} will
always find the standard library's version.
It's suggested that users should begin using absolute imports as much
as possible, so it's preferable to begin writing \code{from pkg import
string} in your code.  

Relative imports are still possible by adding a leading period 
to the module name when using the \code{from ... import} form:

\begin{verbatim}
# Import names from pkg.string
from .string import name1, name2
# Import pkg.string
from . import string
\end{verbatim}

This imports the \module{string} module relative to the current
package, so in \module{pkg.main} this will import \var{name1} and
\var{name2} from \module{pkg.string}.  Additional leading periods
perform the relative import starting from the parent of the current
package.  For example, code in the \module{A.B.C} module can do:

\begin{verbatim}
from . import D   		# Imports A.B.D
from .. import E                # Imports A.E
from ..F import G               # Imports A.F.G
\end{verbatim}

Leading periods cannot be used with the \code{import \var{modname}} 
form of the import statement, only the \code{from ... import} form.

\begin{seealso}

\seepep{328}{Imports: Multi-Line and Absolute/Relative}
{PEP written by Aahz; implemented by Thomas Wouters.}

\seeurl{http://codespeak.net/py/current/doc/index.html}
{The py library by Holger Krekel, which contains the \module{py.std} package.}

\end{seealso}


%======================================================================
\section{PEP 338: Executing Modules as Scripts}

The \programopt{-m} switch added in Python 2.4 to execute a module as
a script gained a few more abilities.  Instead of being implemented in
C code inside the Python interpreter, the switch now uses an
implementation in a new module, \module{runpy}.

The \module{runpy} module implements a more sophisticated import
mechanism so that it's now possible to run modules in a package such
as \module{pychecker.checker}.  The module also supports alternative
import mechanisms such as the \module{zipimport} module.  (This means
you can add a .zip archive's path to \code{sys.path} and then use the
\programopt{-m} switch to execute code from the archive.


\begin{seealso}

\seepep{338}{Executing modules as scripts}{PEP written and 
implemented by Nick Coghlan.}

\end{seealso}


%======================================================================
\section{PEP 341: Unified try/except/finally}

Until Python 2.5, the \keyword{try} statement came in two
flavours. You could use a \keyword{finally} block to ensure that code
is always executed, or a number of \keyword{except} blocks to catch an
exception.  You couldn't combine both \keyword{except} blocks and a
\keyword{finally} block, because generating the right bytecode for the
combined version was complicated and it wasn't clear what the
semantics of the combined should be.  

GvR spent some time working with Java, which does support the
equivalent of combining \keyword{except} blocks and a
\keyword{finally} block, and this clarified what the statement should
mean.  In Python 2.5, you can now write:

\begin{verbatim}
try:
    block-1 ...
except Exception1:
    handler-1 ...
except Exception2:
    handler-2 ...
else:
    else-block
finally:
    final-block 
\end{verbatim}

The code in \var{block-1} is executed.  If the code raises an
exception, the handlers are tried in order: \var{handler-1},
\var{handler-2}, ...  If no exception is raised, the \var{else-block}
is executed.  No matter what happened previously, the
\var{final-block} is executed once the code block is complete and any
raised exceptions handled.  Even if there's an error in an exception
handler or the \var{else-block} and a new exception is raised, the
\var{final-block} is still executed.

\begin{seealso}

\seepep{341}{Unifying try-except and try-finally}{PEP written by Georg Brandl; 
implementation by Thomas Lee.}

\end{seealso}


%======================================================================
\section{PEP 342: New Generator Features}

Python 2.5 adds a simple way to pass values \emph{into} a generator.
As introduced in Python 2.3, generators only produce output; once a
generator's code is invoked to create an iterator, there's no way to
pass any new information into the function when its execution is
resumed.  Sometimes the ability to pass in some information would be
useful.  Hackish solutions to this include making the generator's code
look at a global variable and then changing the global variable's
value, or passing in some mutable object that callers then modify.

To refresh your memory of basic generators, here's a simple example:

\begin{verbatim}
def counter (maximum):
    i = 0
    while i < maximum:
        yield i
	i += 1
\end{verbatim}

When you call \code{counter(10)}, the result is an iterator that
returns the values from 0 up to 9.  On encountering the
\keyword{yield} statement, the iterator returns the provided value and
suspends the function's execution, preserving the local variables.
Execution resumes on the following call to the iterator's 
\method{next()} method, picking up after the \keyword{yield} statement.

In Python 2.3, \keyword{yield} was a statement; it didn't return any
value.  In 2.5, \keyword{yield} is now an expression, returning a
value that can be assigned to a variable or otherwise operated on:

\begin{verbatim}
val = (yield i)
\end{verbatim}

I recommend that you always put parentheses around a \keyword{yield}
expression when you're doing something with the returned value, as in
the above example.  The parentheses aren't always necessary, but it's
easier to always add them instead of having to remember when they're
needed.\footnote{The exact rules are that a \keyword{yield}-expression must
always be parenthesized except when it occurs at the top-level
expression on the right-hand side of an assignment, meaning you can
write \code{val = yield i} but have to use parentheses when there's an
operation, as in \code{val = (yield i) + 12}.}

Values are sent into a generator by calling its
\method{send(\var{value})} method.  The generator's code is then
resumed and the \keyword{yield} expression returns the specified
\var{value}.  If the regular \method{next()} method is called, the
\keyword{yield} returns \constant{None}.

Here's the previous example, modified to allow changing the value of
the internal counter.

\begin{verbatim}
def counter (maximum):
    i = 0
    while i < maximum:
        val = (yield i)
	# If value provided, change counter
        if val is not None:
            i = val
	else:
  	    i += 1
\end{verbatim}

And here's an example of changing the counter:

\begin{verbatim}
>>> it = counter(10)
>>> print it.next()
0
>>> print it.next()
1
>>> print it.send(8)
8
>>> print it.next()
9
>>> print it.next()
Traceback (most recent call last):
  File ``t.py'', line 15, in ?
    print it.next()
StopIteration
\end{verbatim}

Because \keyword{yield} will often be returning \constant{None}, you
should always check for this case.  Don't just use its value in
expressions unless you're sure that the \method{send()} method
will be the only method used resume your generator function.

In addition to \method{send()}, there are two other new methods on
generators:

\begin{itemize}

  \item \method{throw(\var{type}, \var{value}=None,
  \var{traceback}=None)} is used to raise an exception inside the
  generator; the exception is raised by the \keyword{yield} expression
  where the generator's execution is paused.

  \item \method{close()} raises a new \exception{GeneratorExit}
  exception inside the generator to terminate the iteration.  
  On receiving this
  exception, the generator's code must either raise
  \exception{GeneratorExit} or \exception{StopIteration}; catching the 
  exception and doing anything else is illegal and will trigger
  a \exception{RuntimeError}.  \method{close()} will also be called by 
  Python's garbage collection when the generator is garbage-collected.

  If you need to run cleanup code in case of a \exception{GeneratorExit},
  I suggest using a \code{try: ... finally:} suite instead of 
  catching \exception{GeneratorExit}.

\end{itemize}

The cumulative effect of these changes is to turn generators from
one-way producers of information into both producers and consumers.

Generators also become \emph{coroutines}, a more generalized form of
subroutines.  Subroutines are entered at one point and exited at
another point (the top of the function, and a \keyword{return
statement}), but coroutines can be entered, exited, and resumed at
many different points (the \keyword{yield} statements).  We'll have to
figure out patterns for using coroutines effectively in Python.

The addition of the \method{close()} method has one side effect that
isn't obvious.  \method{close()} is called when a generator is
garbage-collected, so this means the generator's code gets one last
chance to run before the generator is destroyed, and this last chance
means that \code{try...finally} statements in generators can now be
guaranteed to work; the \keyword{finally} clause will now always get a
chance to run.  The syntactic restriction that you couldn't mix
\keyword{yield} statements with a \code{try...finally} suite has
therefore been removed.  This seems like a minor bit of language
trivia, but using generators and \code{try...finally} is actually
necessary in order to implement the  \keyword{with} statement
described by PEP 343.  We'll look at this new statement in the following 
section.

\begin{seealso}

\seepep{342}{Coroutines via Enhanced Generators}{PEP written by 
Guido van Rossum and Phillip J. Eby;
implemented by Phillip J. Eby.  Includes examples of 
some fancier uses of generators as coroutines.}

\seeurl{http://en.wikipedia.org/wiki/Coroutine}{The Wikipedia entry for 
coroutines.}

\seeurl{http://www.sidhe.org/\~{}dan/blog/archives/000178.html}{An
explanation of coroutines from a Perl point of view, written by Dan
Sugalski.}

\end{seealso}


%======================================================================
\section{PEP 343: The 'with' statement}

The \keyword{with} statement allows a clearer 
version of code that uses \code{try...finally} blocks

First, I'll discuss the statement as it will commonly be used, and
then I'll discuss the detailed implementation and how to write objects
(called ``context managers'') that can be used with this statement.
Most people, who will only use \keyword{with} in company with an
existing object, don't need to know these details and can 
just use objects that are documented to work as context managers.
Authors of new context managers will need to understand the details of
the underlying implementation.

The \keyword{with} statement is a new control-flow structure whose
basic structure is:

\begin{verbatim}
with expression as variable:
    with-block
\end{verbatim}

The expression is evaluated, and it should result in a type of object
that's called a context manager.  The context manager can return a
value that will be bound to the name \var{variable}.  (Note carefully:
\var{variable} is \emph{not} assigned the result of \var{expression}.
One method of the context manager is run before \var{with-block} is
executed, and another method is run after the block is done, even if
the block raised an exception.

To enable the statement in Python 2.5, you need 
to add the following directive to your module:

\begin{verbatim}
from __future__ import with_statement
\end{verbatim}

Some standard Python objects can now behave as context managers.  For
example, file objects:

\begin{verbatim}
with open('/etc/passwd', 'r') as f:
    for line in f:
        print line

# f has been automatically closed at this point.
\end{verbatim}

The \module{threading} module's locks and condition variables 
also support the \keyword{with} statement:

\begin{verbatim}
lock = threading.Lock()
with lock:
    # Critical section of code
    ...
\end{verbatim}

The lock is acquired before the block is executed, and released once 
the block is complete.

The \module{decimal} module's contexts, which encapsulate the desired
precision and rounding characteristics for computations, can also be
used as context managers.

\begin{verbatim}
import decimal

v1 = decimal.Decimal('578')

# Displays with default precision of 28 digits
print v1.sqrt()

with decimal.Context(prec=16):
    # All code in this block uses a precision of 16 digits.
    # The original context is restored on exiting the block.
    print v1.sqrt()
\end{verbatim}

\subsection{Writing Context Managers}

% XXX write this

This section still needs to be written.

The new \module{contextlib} module provides some functions and a
decorator that are useful for writing context managers.  
Future versions will go into more detail.

% XXX describe further

\begin{seealso}

\seepep{343}{The ``with'' statement}{PEP written by 
Guido van Rossum and Nick Coghlan. }

\end{seealso}


%======================================================================
\section{PEP 352: Exceptions as New-Style Classes}

Exception classes can now be new-style classes, not just classic
classes, and the built-in \exception{Exception} class and all the
standard built-in exceptions (\exception{NameError},
\exception{ValueError}, etc.) are now new-style classes.

The inheritance hierarchy for exceptions has been rearranged a bit.
In 2.5, the inheritance relationships are:

\begin{verbatim}
BaseException       # New in Python 2.5
|- KeyboardInterrupt
|- SystemExit
|- Exception
   |- (all other current built-in exceptions)
\end{verbatim}

This rearrangement was done because people often want to catch all
exceptions that indicate program errors.  \exception{KeyboardInterrupt} and
\exception{SystemExit} aren't errors, though, and usually represent an explicit
action such as the user hitting Control-C or code calling
\function{sys.exit()}.  A bare \code{except:} will catch all exceptions,
so you commonly need to list \exception{KeyboardInterrupt} and
\exception{SystemExit} in order to re-raise them.  The usual pattern is:

\begin{verbatim}
try:
    ...
except (KeyboardInterrupt, SystemExit):
    raise
except: 
    # Log error...  
    # Continue running program...
\end{verbatim}

In Python 2.5, you can now write \code{except Exception} to achieve
the same result, catching all the exceptions that usually indicate errors 
but leaving \exception{KeyboardInterrupt} and
\exception{SystemExit} alone.  As in previous versions,
a bare \code{except:} still catches all exceptions.

The goal for Python 3.0 is to require any class raised as an exception
to derive from \exception{BaseException} or some descendant of
\exception{BaseException}, and future releases in the
Python 2.x series may begin to enforce this constraint.  Therefore, I
suggest you begin making all your exception classes derive from
\exception{Exception} now.  It's been suggested that the bare
\code{except:} form should be removed in Python 3.0, but Guido van~Rossum
hasn't decided whether to do this or not.

Raising of strings as exceptions, as in the statement \code{raise
"Error occurred"}, is deprecated in Python 2.5 and will trigger a
warning.  The aim is to be able to remove the string-exception feature
in a few releases.


\begin{seealso}

\seepep{352}{Required Superclass for Exceptions}{PEP written by 
Brett Cannon and Guido van Rossum; implemented by Brett Cannon.}

\end{seealso}


%======================================================================
\section{PEP 353: Using ssize_t as the index type\label{section-353}}

A wide-ranging change to Python's C API, using a new 
\ctype{Py_ssize_t} type definition instead of \ctype{int}, 
will permit the interpreter to handle more data on 64-bit platforms.
This change doesn't affect Python's capacity on 32-bit platforms.

Various pieces of the Python interpreter used C's \ctype{int} type to
store sizes or counts; for example, the number of items in a list or
tuple were stored in an \ctype{int}.  The C compilers for most 64-bit
platforms still define \ctype{int} as a 32-bit type, so that meant
that lists could only hold up to \code{2**31 - 1} = 2147483647 items.
(There are actually a few different programming models that 64-bit C
compilers can use -- see
\url{http://www.unix.org/version2/whatsnew/lp64_wp.html} for a
discussion -- but the most commonly available model leaves \ctype{int}
as 32 bits.)

A limit of 2147483647 items doesn't really matter on a 32-bit platform
because you'll run out of memory before hitting the length limit.
Each list item requires space for a pointer, which is 4 bytes, plus
space for a \ctype{PyObject} representing the item.  2147483647*4 is
already more bytes than a 32-bit address space can contain.

It's possible to address that much memory on a 64-bit platform,
however.  The pointers for a list that size would only require 16GiB
of space, so it's not unreasonable that Python programmers might
construct lists that large.  Therefore, the Python interpreter had to
be changed to use some type other than \ctype{int}, and this will be a
64-bit type on 64-bit platforms.  The change will cause
incompatibilities on 64-bit machines, so it was deemed worth making
the transition now, while the number of 64-bit users is still
relatively small.  (In 5 or 10 years, we may \emph{all} be on 64-bit
machines, and the transition would be more painful then.)

This change most strongly affects authors of C extension modules.  
Python strings and container types such as lists and tuples 
now use \ctype{Py_ssize_t} to store their size.  
Functions such as \cfunction{PyList_Size()} 
now return \ctype{Py_ssize_t}.  Code in extension modules
may therefore need to have some variables changed to
\ctype{Py_ssize_t}.  

The \cfunction{PyArg_ParseTuple()} and \cfunction{Py_BuildValue()} functions
have a new conversion code, \samp{n}, for \ctype{Py_ssize_t}.  
\cfunction{PyArg_ParseTuple()}'s \samp{s\#} and \samp{t\#} still output
\ctype{int} by default, but you can define the macro 
\csimplemacro{PY_SSIZE_T_CLEAN} before including \file{Python.h} 
to make them return \ctype{Py_ssize_t}.

\pep{353} has a section on conversion guidelines that 
extension authors should read to learn about supporting 64-bit
platforms.

\begin{seealso}

\seepep{353}{Using ssize_t as the index type}{PEP written and implemented by Martin von L\"owis.}

\end{seealso}


%======================================================================
\section{PEP 357: The '__index__' method}

The NumPy developers had a problem that could only be solved by adding
a new special method, \method{__index__}.  When using slice notation,
as in \code{[\var{start}:\var{stop}:\var{step}]}, the values of the
\var{start}, \var{stop}, and \var{step} indexes must all be either
integers or long integers.  NumPy defines a variety of specialized
integer types corresponding to unsigned and signed integers of 8, 16,
32, and 64 bits, but there was no way to signal that these types could
be used as slice indexes.

Slicing can't just use the existing \method{__int__} method because
that method is also used to implement coercion to integers.  If
slicing used \method{__int__}, floating-point numbers would also
become legal slice indexes and that's clearly an undesirable
behaviour.

Instead, a new special method called \method{__index__} was added.  It
takes no arguments and returns an integer giving the slice index to
use.  For example:

\begin{verbatim}
class C:
    def __index__ (self):
        return self.value  
\end{verbatim}

The return value must be either a Python integer or long integer.
The interpreter will check that the type returned is correct, and
raises a \exception{TypeError} if this requirement isn't met.

A corresponding \member{nb_index} slot was added to the C-level
\ctype{PyNumberMethods} structure to let C extensions implement this
protocol.  \cfunction{PyNumber_Index(\var{obj})} can be used in
extension code to call the \method{__index__} function and retrieve
its result.

\begin{seealso}

\seepep{357}{Allowing Any Object to be Used for Slicing}{PEP written 
and implemented by Travis Oliphant.}

\end{seealso}


%======================================================================
\section{Other Language Changes}

Here are all of the changes that Python 2.5 makes to the core Python
language.

\begin{itemize}

\item The \function{min()} and \function{max()} built-in functions
gained a \code{key} keyword argument analogous to the \code{key}
argument for \method{sort()}.  This argument supplies a function
that takes a single argument and is called for every value in the list; 
\function{min()}/\function{max()} will return the element with the 
smallest/largest return value from this function.
For example, to find the longest string in a list, you can do:

\begin{verbatim}
L = ['medium', 'longest', 'short']
# Prints 'longest'
print max(L, key=len)		   
# Prints 'short', because lexicographically 'short' has the largest value
print max(L)         
\end{verbatim}

(Contributed by Steven Bethard and Raymond Hettinger.)

\item Two new built-in functions, \function{any()} and
\function{all()}, evaluate whether an iterator contains any true or
false values.  \function{any()} returns \constant{True} if any value
returned by the iterator is true; otherwise it will return
\constant{False}.  \function{all()} returns \constant{True} only if
all of the values returned by the iterator evaluate as being true.
(Suggested by GvR, and implemented by Raymond Hettinger.)

\item The list of base classes in a class definition can now be empty.  
As an example, this is now legal:

\begin{verbatim}
class C():
    pass
\end{verbatim}
(Implemented by Brett Cannon.)

% XXX __missing__ hook in dictionaries

\end{itemize}


%======================================================================
\subsection{Interactive Interpreter Changes}

In the interactive interpreter, \code{quit} and \code{exit} 
have long been strings so that new users get a somewhat helpful message
when they try to quit:

\begin{verbatim}
>>> quit
'Use Ctrl-D (i.e. EOF) to exit.'
\end{verbatim}

In Python 2.5, \code{quit} and \code{exit} are now objects that still
produce string representations of themselves, but are also callable.
Newbies who try \code{quit()} or \code{exit()} will now exit the
interpreter as they expect.  (Implemented by Georg Brandl.)


%======================================================================
\subsection{Optimizations}

\begin{itemize}

\item When they were introduced 
in Python 2.4, the built-in \class{set} and \class{frozenset} types
were built on top of Python's dictionary type.  
In 2.5 the internal data structure has been customized for implementing sets,
and as a result sets will use a third less memory and are somewhat faster.
(Implemented by Raymond Hettinger.)

\item The performance of some Unicode operations has been improved.
% XXX provide details?

\item The code generator's peephole optimizer now performs
simple constant folding in expressions.  If you write something like
\code{a = 2+3}, the code generator will do the arithmetic and produce
code corresponding to \code{a = 5}.

\end{itemize}

The net result of the 2.5 optimizations is that Python 2.5 runs the
pystone benchmark around XXX\% faster than Python 2.4.


%======================================================================
\section{New, Improved, and Deprecated Modules}

As usual, Python's standard library received a number of enhancements and
bug fixes.  Here's a partial list of the most notable changes, sorted
alphabetically by module name. Consult the
\file{Misc/NEWS} file in the source tree for a more
complete list of changes, or look through the SVN logs for all the
details.

\begin{itemize}

% collections.deque now has .remove()
% collections.defaultdict

% the cPickle module no longer accepts the deprecated None option in the
% args tuple returned by __reduce__().

% csv module improvements

% datetime.datetime() now has a strptime class method which can be used to
% create datetime object using a string and format.

% fileinput: opening hook used to control how files are opened.
% .input() now has a mode parameter
% now has a fileno() function
% accepts Unicode filenames

\item In the \module{gc} module, the new \function{get_count()} function
returns a 3-tuple containing the current collection counts for the
three GC generations.  This is accounting information for the garbage
collector; when these counts reach a specified threshold, a garbage
collection sweep will be made.  The existing \function{gc.collect()}
function now takes an optional \var{generation} argument of 0, 1, or 2
to specify which generation to collect.

\item The \function{nsmallest()} and 
\function{nlargest()} functions in the \module{heapq} module 
now support a \code{key} keyword argument similar to the one
provided by the \function{min()}/\function{max()} functions
and the \method{sort()} methods.  For example:
Example:

\begin{verbatim}
>>> import heapq
>>> L = ["short", 'medium', 'longest', 'longer still']
>>> heapq.nsmallest(2, L)  # Return two lowest elements, lexicographically
['longer still', 'longest']
>>> heapq.nsmallest(2, L, key=len)   # Return two shortest elements
['short', 'medium']
\end{verbatim}

(Contributed by Raymond Hettinger.)

\item The \function{itertools.islice()} function now accepts
\code{None} for the start and step arguments.  This makes it more
compatible with the attributes of slice objects, so that you can now write
the following:

\begin{verbatim}
s = slice(5)     # Create slice object
itertools.islice(iterable, s.start, s.stop, s.step)
\end{verbatim}

(Contributed by Raymond Hettinger.)

\item The \module{operator} module's \function{itemgetter()} 
and \function{attrgetter()} functions now support multiple fields.  
A call such as \code{operator.attrgetter('a', 'b')}
will return a function 
that retrieves the \member{a} and \member{b} attributes.  Combining 
this new feature with the \method{sort()} method's \code{key} parameter 
lets you easily sort lists using multiple fields.
(Contributed by Raymond Hettinger.)


\item The \module{os} module underwent a number of changes.  The
\member{stat_float_times} variable now defaults to true, meaning that
\function{os.stat()} will now return time values as floats.  (This
doesn't necessarily mean that \function{os.stat()} will return times
that are precise to fractions of a second; not all systems support
such precision.)

Constants named \member{os.SEEK_SET}, \member{os.SEEK_CUR}, and
\member{os.SEEK_END} have been added; these are the parameters to the
\function{os.lseek()} function.  Two new constants for locking are
\member{os.O_SHLOCK} and \member{os.O_EXLOCK}.

Two new functions, \function{wait3()} and \function{wait4()}, were
added.  They're similar the \function{waitpid()} function which waits
for a child process to exit and returns a tuple of the process ID and
its exit status, but \function{wait3()} and \function{wait4()} return
additional information.  \function{wait3()} doesn't take a process ID
as input, so it waits for any child process to exit and returns a
3-tuple of \var{process-id}, \var{exit-status}, \var{resource-usage}
as returned from the \function{resource.getrusage()} function.
\function{wait4(\var{pid})} does take a process ID.
(Contributed by Chad J. Schroeder.)

On FreeBSD, the \function{os.stat()} function now returns 
times with nanosecond resolution, and the returned object
now has \member{st_gen} and \member{st_birthtime}.
The \member{st_flags} member is also available, if the platform supports it.
(Contributed by Antti Louko and  Diego Petten\`o.)
% (Patch 1180695, 1212117)

\item The old \module{regex} and \module{regsub} modules, which have been 
deprecated ever since Python 2.0, have finally been deleted.  
Other deleted modules: \module{statcache}, \module{tzparse},
\module{whrandom}.

\item The \file{lib-old} directory,
which includes ancient modules such as \module{dircmp} and
\module{ni}, was also deleted.  \file{lib-old} wasn't on the default
\code{sys.path}, so unless your programs explicitly added the directory to 
\code{sys.path}, this removal shouldn't affect your code.

\item The \module{socket} module now supports \constant{AF_NETLINK}
sockets on Linux, thanks to a patch from Philippe Biondi.  
Netlink sockets are a Linux-specific mechanism for communications
between a user-space process and kernel code; an introductory 
article about them is at \url{http://www.linuxjournal.com/article/7356}.
In Python code, netlink addresses are represented as a tuple of 2 integers, 
\code{(\var{pid}, \var{group_mask})}.

Socket objects also gained accessor methods \method{getfamily()}, 
\method{gettype()}, and \method{getproto()} methods to retrieve the
family, type, and protocol values for the socket.

\item New module: \module{spwd} provides functions for accessing the
shadow password database on systems that support it.  
% XXX give example

% XXX patch #1382163: sys.subversion,  Py_GetBuildNumber()

\item The \class{TarFile} class in the \module{tarfile} module now has
an \method{extractall()} method that extracts all members from the
archive into the current working directory.  It's also possible to set
a different directory as the extraction target, and to unpack only a
subset of the archive's members.  

A tarfile's compression can be autodetected by 
using the mode \code{'r|*'}.
% patch 918101
(Contributed by Lars Gust\"abel.)

\item The \module{unicodedata} module has been updated to use version 4.1.0
of the Unicode character database.  Version 3.2.0 is required 
by some specifications, so it's still available as 
\member{unicodedata.db_3_2_0}.

% patch #754022: Greatly enhanced webbrowser.py (by Oleg Broytmann).


\item The \module{xmlrpclib} module now supports returning 
      \class{datetime} objects for the XML-RPC date type.  Supply 
      \code{use_datetime=True} to the \function{loads()} function
      or the \class{Unmarshaller} class to enable this feature.
      (Contributed by Skip Montanaro.)
% Patch 1120353


\end{itemize}



%======================================================================
% whole new modules get described in subsections here

% XXX new distutils features: upload

\subsection{The ctypes package}

The \module{ctypes} package, written by Thomas Heller, has been added 
to the standard library.  \module{ctypes} lets you call arbitrary functions 
in shared libraries or DLLs.

In subsequent alpha releases of Python 2.5, I'll add a brief 
introduction that shows some basic usage of the module.

% XXX write introduction


\subsection{The ElementTree package}

A subset of Fredrik Lundh's ElementTree library for processing XML has
been added to the standard library as \module{xmlcore.etree}.  The
available modules are
\module{ElementTree}, \module{ElementPath}, and
\module{ElementInclude} from ElementTree 1.2.6.   
The \module{cElementTree} accelerator module is also included. 

The rest of this section will provide a brief overview of using
ElementTree.  Full documentation for ElementTree is available at
\url{http://effbot.org/zone/element-index.htm}.

ElementTree represents an XML document as a tree of element nodes.
The text content of the document is stored as the \member{.text}
and \member{.tail} attributes of 
(This is one of the major differences between ElementTree and 
the Document Object Model; in the DOM there are many different
types of node, including \class{TextNode}.)

The most commonly used parsing function is \function{parse()}, that
takes either a string (assumed to contain a filename) or a file-like
object and returns an \class{ElementTree} instance:

\begin{verbatim}
from xmlcore.etree import ElementTree as ET

tree = ET.parse('ex-1.xml')

feed = urllib.urlopen(
          'http://planet.python.org/rss10.xml')
tree = ET.parse(feed)
\end{verbatim}

Once you have an \class{ElementTree} instance, you
can call its \method{getroot()} method to get the root \class{Element} node.

There's also an \function{XML()} function that takes a string literal
and returns an \class{Element} node (not an \class{ElementTree}).  
This function provides a tidy way to incorporate XML fragments,
approaching the convenience of an XML literal:

\begin{verbatim}
svg = et.XML("""<svg width="10px" version="1.0">
             </svg>""")
svg.set('height', '320px')
svg.append(elem1)
\end{verbatim}

Each XML element supports some dictionary-like and some list-like
access methods.  Dictionary-like methods are used to access attribute
values, and list-like methods are used to access child nodes.

% XXX finish this

To generate XML output, you should call the
\method{ElementTree.write()} method.  Like \function{parse()},
it can take either a string or a file-like object:

\begin{verbatim}
# Encoding is US-ASCII
tree.write('output.xml')

# Encoding is UTF-8
f = open('output.xml', 'w')
tree.write(f, 'utf-8')
\end{verbatim}

(Caution: the default encoding used for output is ASCII, which isn't
very useful for general XML work, raising an exception if there are
any characters with values greater than 127.  You should always
specify a different encoding such as UTF-8 that can handle any Unicode
character.)


% XXX write introduction

\begin{seealso}

\seeurl{http://effbot.org/zone/element-index.htm}
{Official documentation for ElementTree.}


\end{seealso}


\subsection{The hashlib package}

A new \module{hashlib} module has been added to replace the
\module{md5} and \module{sha} modules.  \module{hashlib} adds support
for additional secure hashes (SHA-224, SHA-256, SHA-384, and SHA-512).
When available, the module uses OpenSSL for fast platform optimized
implementations of algorithms.  

The old \module{md5} and \module{sha} modules still exist as wrappers
around hashlib to preserve backwards compatibility.  The new module's
interface is very close to that of the old modules, but not identical.
The most significant difference is that the constructor functions
for creating new hashing objects are named differently.

\begin{verbatim}
# Old versions
h = md5.md5()	
h = md5.new()	

# New version 
h = hashlib.md5()

# Old versions
h = sha.sha()	
h = sha.new()	

# New version 
h = hashlib.sha1()

# Hash that weren't previously available
h = hashlib.sha224()
h = hashlib.sha256()
h = hashlib.sha384()
h = hashlib.sha512()

# Alternative form
h = hashlib.new('md5')		# Provide algorithm as a string
\end{verbatim}

Once a hash object has been created, its methods are the same as before:
\method{update(\var{string})} hashes the specified string into the 
current digest state, \method{digest()} and \method{hexdigest()}
return the digest value as a binary string or a string of hex digits,
and \method{copy()} returns a new hashing object with the same digest state.

This module was contributed by Gregory P. Smith.


\subsection{The sqlite3 package}

The pysqlite module (\url{http://www.pysqlite.org}), a wrapper for the
SQLite embedded database, has been added to the standard library under
the package name \module{sqlite3}.  SQLite is a C library that
provides a SQL-language database that stores data in disk files
without requiring a separate server process.  pysqlite was written by
Gerhard H\"aring, and provides a SQL interface that complies with the
DB-API 2.0 specification described by \pep{249}. This means that it
should be possible to write the first version of your applications
using SQLite for data storage and, if switching to a larger database
such as PostgreSQL or Oracle is necessary, the switch should be
relatively easy.

If you're compiling the Python source yourself, note that the source
tree doesn't include the SQLite code itself, only the wrapper module.
You'll need to have the SQLite libraries and headers installed before
compiling Python, and the build process will compile the module when
the necessary headers are available.

To use the module, you must first create a \class{Connection} object
that represents the database.  Here the data will be stored in the 
\file{/tmp/example} file:

\begin{verbatim}
conn = sqlite3.connect('/tmp/example')
\end{verbatim}

You can also supply the special name \samp{:memory:} to create
a database in RAM.

Once you have a \class{Connection}, you can create a \class{Cursor} 
object and call its \method{execute()} method to perform SQL commands:

\begin{verbatim}
c = conn.cursor()

# Create table
c.execute('''create table stocks
(date timestamp, trans varchar, symbol varchar,
 qty decimal, price decimal)''')

# Insert a row of data
c.execute("""insert into stocks
          values ('2006-01-05','BUY','RHAT',100, 35.14)""")
\end{verbatim}    

Usually your SQL queries will need to reflect the value of Python
variables.  You shouldn't assemble your query using Python's string
operations because doing so is insecure; it makes your program
vulnerable to what's called an SQL injection attack.  Instead, use
SQLite's parameter substitution, putting \samp{?} as a placeholder
wherever you want to use a value, and then provide a tuple of values
as the second argument to the cursor's \method{execute()} method.  For
example:

\begin{verbatim}    
# Never do this -- insecure!
symbol = 'IBM'
c.execute("... where symbol = '%s'" % symbol)

# Do this instead
t = (symbol,)
c.execute("... where symbol = '?'", t)

# Larger example
for t in (('2006-03-28', 'BUY', 'IBM', 1000, 45.00),
	  ('2006-04-05', 'BUY', 'MSOFT', 1000, 72.00),
	  ('2006-04-06', 'SELL', 'IBM', 500, 53.00),
	 ):
    c.execute('insert into stocks values (?,?,?,?,?)', t)
\end{verbatim}

To retrieve data after executing a SELECT statement, you can either 
treat the cursor as an iterator, call the cursor's \method{fetchone()}
method to retrieve a single matching row, 
or call \method{fetchall()} to get a list of the matching rows.

This example uses the iterator form:

\begin{verbatim}
>>> c = conn.cursor()
>>> c.execute('select * from stocks order by price')
>>> for row in c:
...    print row
...
(u'2006-01-05', u'BUY', u'RHAT', 100, 35.140000000000001)
(u'2006-03-28', u'BUY', u'IBM', 1000, 45.0)
(u'2006-04-06', u'SELL', u'IBM', 500, 53.0)
(u'2006-04-05', u'BUY', u'MSOFT', 1000, 72.0)
>>>
\end{verbatim}

You should also use parameter substitution with SELECT statements:

\begin{verbatim}
>>> c.execute('select * from stocks where symbol=?', ('IBM',))
>>> print c.fetchall()
[(u'2006-03-28', u'BUY', u'IBM', 1000, 45.0), 
 (u'2006-04-06', u'SELL', u'IBM', 500, 53.0)]
\end{verbatim}

For more information about the SQL dialect supported by SQLite, see 
\url{http://www.sqlite.org}.

\begin{seealso}

\seeurl{http://www.pysqlite.org}
{The pysqlite web page.}

\seeurl{http://www.sqlite.org}
{The SQLite web page; the documentation describes the syntax and the
available data types for the supported SQL dialect.}

\seepep{249}{Database API Specification 2.0}{PEP written by
Marc-Andr\'e Lemburg.}

\end{seealso}


% ======================================================================
\section{Build and C API Changes}

Changes to Python's build process and to the C API include:

\begin{itemize}

\item The largest change to the C API came from \pep{353},
which modifies the interpreter to use a \ctype{Py_ssize_t} type
definition instead of \ctype{int}.  See the earlier
section~ref{section-353} for a discussion of this change.

\item The design of the bytecode compiler has changed a great deal, to
no longer generate bytecode by traversing the parse tree.  Instead
the parse tree is converted to an abstract syntax tree (or AST), and it is 
the abstract syntax tree that's traversed to produce the bytecode.

No documentation has been written for the AST code yet.  To start
learning about it, read the definition of the various AST nodes in
\file{Parser/Python.asdl}.  A Python script reads this file and
generates a set of C structure definitions in
\file{Include/Python-ast.h}.  The \cfunction{PyParser_ASTFromString()}
and \cfunction{PyParser_ASTFromFile()}, defined in
\file{Include/pythonrun.h}, take Python source as input and return the
root of an AST representing the contents.  This AST can then be turned
into a code object by \cfunction{PyAST_Compile()}.  For more
information, read the source code, and then ask questions on
python-dev.

% List of names taken from Jeremy's python-dev post at 
% http://mail.python.org/pipermail/python-dev/2005-October/057500.html
The AST code was developed under Jeremy Hylton's management, and
implemented by (in alphabetical order) Brett Cannon, Nick Coghlan,
Grant Edwards, John Ehresman, Kurt Kaiser, Neal Norwitz, Tim Peters,
Armin Rigo, and Neil Schemenauer, plus the participants in a number of
AST sprints at conferences such as PyCon.
 
\item The built-in set types now have an official C API.  Call
\cfunction{PySet_New()} and \cfunction{PyFrozenSet_New()} to create a
new set, \cfunction{PySet_Add()} and \cfunction{PySet_Discard()} to
add and remove elements, and \cfunction{PySet_Contains} and
\cfunction{PySet_Size} to examine the set's state.

\item The \cfunction{PyRange_New()} function was removed.  It was
never documented, never used in the core code, and had dangerously lax
error checking.

\end{itemize}


%======================================================================
%\subsection{Port-Specific Changes}

%Platform-specific changes go here.


%======================================================================
\section{Other Changes and Fixes \label{section-other}}

As usual, there were a bunch of other improvements and bugfixes
scattered throughout the source tree.  A search through the SVN change
logs finds there were XXX patches applied and YYY bugs fixed between
Python 2.4 and 2.5.  Both figures are likely to be underestimates.

Some of the more notable changes are:

\begin{itemize}

\item Evan Jones's patch to obmalloc, first described in a talk
at PyCon DC 2005, was applied.  Python 2.4 allocated small objects in
256K-sized arenas, but never freed arenas.  With this patch, Python
will free arenas when they're empty.  The net effect is that on some
platforms, when you allocate many objects, Python's memory usage may
actually drop when you delete them, and the memory may be returned to
the operating system.  (Implemented by Evan Jones, and reworked by Tim
Peters.)

\item Coverity, a company that markets a source code analysis tool
  called Prevent, provided the results of their examination of the Python
  source code.  The analysis found a number of refcounting bugs, often
  in error-handling code.  These bugs have been fixed.
  % XXX provide reference?

\end{itemize}


%======================================================================
\section{Porting to Python 2.5}

This section lists previously described changes that may require
changes to your code:

\begin{itemize}

\item The \module{pickle} module no longer uses the deprecated \var{bin} parameter.

\end{itemize}


%======================================================================
\section{Acknowledgements \label{acks}}

The author would like to thank the following people for offering
suggestions, corrections and assistance with various drafts of this
article: Mike Rovner, Thomas Wouters.

\end{document}
