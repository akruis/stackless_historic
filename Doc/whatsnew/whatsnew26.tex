\documentclass{howto}
\usepackage{distutils}
% $Id$

% Rules for maintenance:
% 
%  * Anyone can add text to this document.  Do not spend very much time
%    on the wording of your changes, because your text will probably
%    get rewritten to some degree.
%
%  * The maintainer will go through Misc/NEWS periodically and add
%    changes; it's therefore more important to add your changes to 
%    Misc/NEWS than to this file.
%
%  * This is not a complete list of every single change; completeness
%    is the purpose of Misc/NEWS.  Some changes I consider too small
%    or esoteric to include.  If such a change is added to the text,
%    I'll just remove it.  (This is another reason you shouldn't spend
%    too much time on writing your addition.)
%
%  * If you want to draw your new text to the attention of the
%    maintainer, add 'XXX' to the beginning of the paragraph or
%    section.
%
%  * It's OK to just add a fragmentary note about a change.  For
%    example: "XXX Describe the transmogrify() function added to the
%    socket module."  The maintainer will research the change and
%    write the necessary text.
%
%  * You can comment out your additions if you like, but it's not
%    necessary (especially when a final release is some months away).
%
%  * Credit the author of a patch or bugfix.   Just the name is
%    sufficient; the e-mail address isn't necessary.
%
%  * It's helpful to add the bug/patch number as a comment:
%
%       % Patch 12345
%       XXX Describe the transmogrify() function added to the socket
%       module.
%       (Contributed by P.Y. Developer.)
%
%    This saves the maintainer the effort of going through the SVN log
%    when researching a change.

\title{What's New in Python 2.6}
\release{0.0}
\author{A.M. Kuchling}
\authoraddress{\email{amk@amk.ca}}

\begin{document}
\maketitle
\tableofcontents

This article explains the new features in Python 2.6.  No release date
for Python 2.6 has been set; it will probably be released in mid 2008.

% Compare with previous release in 2 - 3 sentences here.

This article doesn't attempt to provide a complete specification of
the new features, but instead provides a convenient overview.  For
full details, you should refer to the documentation for Python 2.6.
% add hyperlink when the documentation becomes available online.
If you want to understand the complete implementation and design
rationale, refer to the PEP for a particular new feature.


%======================================================================

% Large, PEP-level features and changes should be described here.

% Should there be a new section here for 3k migration?
% Or perhaps a more general section describing module changes/deprecation?
% sets module deprecated

%======================================================================
\section{Other Language Changes}

Here are all of the changes that Python 2.6 makes to the core Python
language.

\begin{itemize}

% Bug 1569356
\item An obscure change: when you use the the \function{locals()}
function inside a \keyword{class} statement, the resulting dictionary
no longer returns free variables.  (Free variables, in this case, are
variables referred to in the \keyword{class} statement 
that aren't attributes of the class.)

\end{itemize}


%======================================================================
\subsection{Optimizations}

\begin{itemize}

% Patch 1624059
\item Internally, a bit is now set in type objects to indicate some of
the standard built-in types.  This speeds up checking if an object is
a subclass of one of these types.  (Contributed by Neal Norwitz.)

\end{itemize}

The net result of the 2.6 optimizations is that Python 2.6 runs the
pystone benchmark around XX\% faster than Python 2.5.


%======================================================================
\section{New, Improved, and Deprecated Modules}

As usual, Python's standard library received a number of enhancements and
bug fixes.  Here's a partial list of the most notable changes, sorted
alphabetically by module name. Consult the
\file{Misc/NEWS} file in the source tree for a more
complete list of changes, or look through the CVS logs for all the
details.

\begin{itemize}

\item New data type in the \module{collections} module:
\class{NamedTuple(\var{typename}, \var{fieldnames})} is a factory function that
creates subclasses of the standard tuple whose fields are accessible
by name as well as index.  For example:

\begin{verbatim}
var_type = collections.NamedTuple('variable', 
             'id name type size')
var = var_type(1, 'frequency', 'int', 4)

print var[0], var.id		# Equivalent
print var[2], var.type          # Equivalent
\end{verbatim}

(Contributed by Raymond Hettinger.)

\item New method in the \module{curses} module:
for a window, \method{chgat()} changes the display characters for a 
certain number of characters on a single line.

\begin{verbatim}
# Boldface text starting at y=0,x=21 
# and affecting the rest of the line.
stdscr.chgat(0,21, curses.A_BOLD)  
\end{verbatim}

(Contributed by Fabian Kreutz.)

\item The \module{gopherlib} module has been removed.

\item New function in the \module{heapq} module:
\function{merge(iter1, iter2, ...)} 
takes any number of iterables that return data 
\emph{in sorted order}, 
and 
returns a new iterator that returns the contents of
all the iterators, also in sorted order.  For example:

\begin{verbatim}
heapq.merge([1, 3, 5, 9], [2, 8, 16]) ->
  [1, 2, 3, 5, 8, 9, 16]
\end{verbatim}

(Contributed by Raymond Hettinger.)

\item New function in the \module{itertools} module:
\function{izip_longest(iter1, iter2, ...\optional{, fillvalue})}
makes tuples from each of the elements; if some of the iterables
are shorter than others, the missing values 
are set to \var{fillvalue}.  For example:

\begin{verbatim}
itertools.izip_longest([1,2,3], [1,2,3,4,5]) ->
  [(1, 1), (2, 2), (3, 3), (None, 4), (None, 5)]
\end{verbatim}

(Contributed by Raymond Hettinger.)

\item The \module{macfs} module has been removed.  This in turn
required the \function{macostools.touched()} function to be removed
because it depended on the \module{macfs} module.

% Patch #1490190
\item New functions in the \module{posix} module: \function{chflags()}
and \function{lchflags()} are wrappers for the corresponding system
calls (where they're available).  Constants for the flag values are
defined in the \module{stat} module; some possible values include
\constant{UF_IMMUTABLE} to signal the file may not be changed and
\constant{UF_APPEND} to indicate that data can only be appended to the
file.  (Contributed by M. Levinson.)

\item The \module{rgbimg} module has been removed.

\item The \module{smtplib} module now supports SMTP over 
SSL thanks to the addition of the \class{SMTP_SSL} class.
This class supports an interface identical to the existing \class{SMTP} 
class. (Contributed by Monty Taylor.)

\item The \module{test.test_support} module now contains a
\function{EnvironmentVarGuard} context manager that 
supports temporarily changing environment variables and 
automatically restores them to their old values.
(Contributed by Brett Cannon.)

\end{itemize}


%======================================================================
% whole new modules get described in \subsections here


% ======================================================================
\section{Build and C API Changes}

Changes to Python's build process and to the C API include:

\begin{itemize}

\item Detailed changes are listed here.

\end{itemize}


%======================================================================
\subsection{Port-Specific Changes}

Platform-specific changes go here.


%======================================================================
\section{Other Changes and Fixes \label{section-other}}

As usual, there were a bunch of other improvements and bugfixes
scattered throughout the source tree.  A search through the change
logs finds there were XXX patches applied and YYY bugs fixed between
Python 2.5 and 2.6.  Both figures are likely to be underestimates.

Some of the more notable changes are:

\begin{itemize}

\item Details go here.

\end{itemize}


%======================================================================
\section{Porting to Python 2.6}

This section lists previously described changes that may require
changes to your code:

\begin{itemize}

\item Everything is all in the details!

\end{itemize}


%======================================================================
\section{Acknowledgements \label{acks}}

The author would like to thank the following people for offering
suggestions, corrections and assistance with various drafts of this
article: .

\end{document}
