\section{Introduction}
\label{intro}

The modules in this manual are available on the Apple Macintosh only.

Aside from the modules described here there are also interfaces to
various MacOS toolboxes, which are currently not extensively
described. The toolboxes for which modules exist are:
\module{AE} (Apple Events),
\module{Cm} (Component Manager),
\module{Ctl} (Control Manager),
\module{Dlg} (Dialog Manager),
\module{Evt} (Event Manager),
\module{Fm} (Font Manager),
\module{List} (List Manager),
\module{Menu} (Moenu Manager),
\module{Qd} (QuickDraw),
\module{Qt} (QuickTime),
\module{Res} (Resource Manager and Handles),
\module{Scrap} (Scrap Manager),
\module{Snd} (Sound Manager),
\module{TE} (TextEdit),
\module{Waste} (non-Apple \program{TextEdit} replacement) and
\module{Win} (Window Manager).

If applicable the module will define a number of Python objects for
the various structures declared by the toolbox, and operations will be
implemented as methods of the object. Other operations will be
implemented as functions in the module. Not all operations possible in
\C{} will also be possible in Python (callbacks are often a problem), and
parameters will occasionally be different in Python (input and output
buffers, especially). All methods and functions have a \code{__doc__}
string describing their arguments and return values, and for
additional description you are referred to \emph{Inside Macintosh} or
similar works.

The following modules are documented here:

\localmoduletable


\section{\module{mac} ---
         Implementations for the \module{os} module}
\declaremodule{builtin}{mac}

\modulesynopsis{Implementations for the \module{os} module.}


This module implements the operating system dependent functionality
provided by the standard module \module{os}\refstmodindex{os}.  It is
best accessed through the \module{os} module.

The following functions are available in this module:
\function{chdir()},
\function{close()},
\function{dup()},
\function{fdopen()},
\function{getcwd()},
\function{lseek()},
\function{listdir()},
\function{mkdir()},
\function{open()},
\function{read()},
\function{rename()},
\function{rmdir()},
\function{stat()},
\function{sync()},
\function{unlink()},
\function{write()},
as well as the exception \exception{error}. Note that the times
returned by \function{stat()} are floating-point values, like all time
values in MacPython.

One additional function is available:

\begin{funcdesc}{xstat}{path}
  This function returns the same information as \function{stat()}, but
  with three additional values appended: the size of the resource fork
  of the file and its 4-character creator and type.
\end{funcdesc}


\section{\module{macpath} ---
         MacOS path manipulation functions}
\declaremodule{standard}{macpath}

\modulesynopsis{MacOS path manipulation functions.}


This module is the Macintosh implementation of the \module{os.path}
module.  It is most portably accessed as \module{os.path}.
\refstmodindex{os.path}

The following functions are available in this module:
\function{normcase()},
\function{normpath()},
\function{isabs()},
\function{join()},
\function{split()},
\function{isdir()},
\function{isfile()},
\function{walk()},
\function{exists()}.
For other functions available in \module{os.path} dummy counterparts
are available.
