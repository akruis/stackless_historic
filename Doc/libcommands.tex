\section{Standard Module \sectcode{commands}}
\stmodindex{commands}
\label{module-commands}

The \code{commands} module contains wrapper functions for \code{os.popen()} 
which take a system command as a string and return any output generated by 
the command and, optionally, the exit status.

The \code{commands} module is only usable on systems which support 
\code{popen()} (currently \UNIX{}).

The \code{commands} module defines the following functions:

\renewcommand{\indexsubitem}{(in module commands)}
\begin{funcdesc}{getstatusoutput}{cmd}
Execute the string \var{cmd} in a shell with \code{os.popen()} and return
a 2-tuple (status, output).  \var{cmd} is actually run as
\code{\{ cmd ; \} 2>\&1}, so that the returned output will contain output
or error messages. A trailing newline is stripped from the output.
The exit status for the  command can be interpreted according to the
rules for the \C{} function \code{wait()}.  
\end{funcdesc}

\begin{funcdesc}{getoutput}{cmd}
Like \code{getstatusoutput()}, except the exit status is ignored and
the return value is a string containing the command's output.  
\end{funcdesc}

\begin{funcdesc}{getstatus}{file}
Return the output of \samp{ls -ld \var{file}} as a string.  This
function uses the \code{getoutput()} function, and properly escapes
backslashes and dollar signs in the argument.
\end{funcdesc}

Example:

\begin{verbatim}
>>> import commands
>>> commands.getstatusoutput('ls /bin/ls')
(0, '/bin/ls')
>>> commands.getstatusoutput('cat /bin/junk')
(256, 'cat: /bin/junk: No such file or directory')
>>> commands.getstatusoutput('/bin/junk')
(256, 'sh: /bin/junk: not found')
>>> commands.getoutput('ls /bin/ls')
'/bin/ls'
>>> commands.getstatus('/bin/ls')
'-rwxr-xr-x  1 root        13352 Oct 14  1994 /bin/ls'
\end{verbatim}
