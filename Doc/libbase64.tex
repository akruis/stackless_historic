\section{Standard Module \module{base64}}
\label{module-base64}
\stmodindex{base64}
\indexii{base64}{encoding}
\index{MIME!base64 encoding}

This module perform base64 encoding and decoding of arbitrary binary
strings into text strings that can be safely emailed or posted.  The
encoding scheme is defined in \rfc{1421} (``Privacy Enhancement for
Internet Electronic Mail: Part I: Message Encryption and
Authentication Procedures'', section 4.3.2.4, ``Step 4: Printable
Encoding'') and is used for MIME email and
various other Internet-related applications; it is not the same as the
output produced by the \program{uuencode} program.  For example, the
string \code{'www.python.org'} is encoded as the string
\code{'d3d3LnB5dGhvbi5vcmc=\e n'}.  


\begin{funcdesc}{decode}{input, output}
Decode the contents of the \var{input} file and write the resulting
binary data to the \var{output} file.
\var{input} and \var{output} must either be file objects or objects that
mimic the file object interface. \var{input} will be read until
\code{\var{input}.read()} returns an empty string.
\end{funcdesc}

\begin{funcdesc}{decodestring}{s}
Decode the string \var{s}, which must contain one or more lines of
base64 encoded data, and return a string containing the resulting
binary data.
\end{funcdesc}

\begin{funcdesc}{encode}{input, output}
Encode the contents of the \var{input} file and write the resulting
base64 encoded data to the \var{output} file.
\var{input} and \var{output} must either be file objects or objects that
mimic the file object interface. \var{input} will be read until
\code{\var{input}.read()} returns an empty string.
\end{funcdesc}

\begin{funcdesc}{encodestring}{s}
Encode the string \var{s}, which can contain arbitrary binary data,
and return a string containing one or more lines of
base64 encoded data.
\end{funcdesc}
