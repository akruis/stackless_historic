\section{Standard module \sectcode{EasyDialogs}}
\stmodindex{EasyDialogs}

The \code{EasyDialogs} module contains some simple dialogs for
the Macintosh, modelled after the \code{stdwin} dialogs with similar
names.

The \code{EasyDialogs} module defines the following functions:

\renewcommand{\indexsubitem}{(in module EasyDialogs)}

\begin{funcdesc}{Message}{str}
A modal dialog with the message text \var{str}, which should be at
most 255 characters long, is displayed. Control is returned when the
user clicks ``OK''.
\end{funcdesc}

\begin{funcdesc}{AskString}{prompt\optional{\, default}}
Ask the user to input a string value, in a modal dialog. \var{Prompt}
is the promt message, the optional \var{default} arg is the initial
value for the string. All strings can be at most 255 bytes
long. \var{AskString} returns the string entered or \code{None} in
case the user cancelled.
\end{funcdesc}

\begin{funcdesc}{AskYesNoCancel}{question\optional{\, default}}
Present a dialog with text \var{question} and three buttons labelled
``yes'', ``no'' and ``cancel''. Return \code{1} for yes, \code{0} for
no and \code{-1} for cancel. The default return value chosen by
hitting return is \code{0}. This can be changed with the optional
\var{default} argument.
\end{funcdesc}

Note that \code{EasyDialogs} does not currently use the notification
manager. This means that displaying dialogs while the program is in
the background will need to unexpected results and possibly crashes.


\section{Standard module \sectcode{FrameWork}}
\stmodindex{FrameWork}

The \code{FrameWork} module contains classes that together provide a
framework for an interactive Macintosh application. The programmer
builds an application by creating subclasses that override various
methods of the bases classes, thereby implementing the functionality
wanted. Overriding functionality can often be done on various
different levels, i.e. to handle clicks in a single dialog window in a
non-standard way it is not necessary to override the complete event
handling.

The \code{FrameWork} is still very much work-in-progress, and the
documentation describes only the most important functionality, and not
in the most logical manner at that. Examine the source for more
esoteric needs. 

The \code{EasyDialogs} module defines the following functions:

\renewcommand{\indexsubitem}{(in module FrameWork)}

\begin{funcdesc}{Application}{}
An object representing the complete application. See below for a
description of the methods. The default \code{__init__} routine
creates an empty window dictionary and a menu bar with an apple menu.
\end{funcdesc}

\begin{funcdesc}{MenuBar}{}
An object representing the menubar. This object is usually not created
by the user.
\end{funcdesc}

\begin{funcdesc}{Menu}{bar\, title\optional{\, after}}
An object representing a menu. Upon creation you pass the
\code{MenuBar} the menu appears in, the \var{title} string and a
position (1-based) \var{after} where the menu should appear (default:
at the end).
\end{funcdesc}

\begin{funcdesc}{MenuItem}{menu\, title\optional{\, shortcut\, callback}}
Create a menu item object. The arguments are the menu to crate the
item it, the item title string and optionally the keyboard shortcut
and a callback routine. The callback is called with the arguments
menu-id, item number within menu (1-based), current front window and
the event record.
\end{funcdesc}

\begin{funcdesc}{Separator}{menu}
Add a separator to the end of a menu.
\end{funcdesc}

\begin{funcdesc}{SubMenu}{menu\, label}
Create a submenu named \var{label} under menu \var{menu}. The menu
object is returned.
\end{funcdesc}

\begin{funcdesc}{Window}{parent}
Creates a (modeless) window. \var{Parent} is the application object to
which the window belongs. The window is not displayed until later.
\end{funcdesc}

\begin{funcdesc}{DialogWindow}{parent}
Creates a modeless dialog window.
\end{funcdesc}


\subsection{Application objects}
Application objects have the following methods, among others:

\renewcommand{\indexsubitem}{(Application method)}

\begin{funcdesc}{makeusermenus}{}
Override this method if you need menus in your application. Append the
menus to \code{self.menubar}.
\end{funcdesc}

\begin{funcdesc}{getabouttext}{}
Override this method to return a text string describing your
application. Alternatively, override the \code{do_about} method for
more elaborate about messages.
\end{funcdesc}

\begin{funcdesc}{mainloop}{\optional{mask\, wait}}
This routine is the main event loop, call it to set your application
rolling. \var{Mask} is the mask of events you want to handle,
\var{wait} is the number of ticks you want to leave to other
concurrent application (default 0, which is probably not a good
idea). This method does not return until \code{self} is raised.

The event loop is split into many small parts, each of which can be
overridden. The default methods take care of dispatching events to
windows and dialogs, handling drags and resizes, Apple Events, events
for non-FrameWork windows, etc.
\end{funcdesc}

\begin{funcdesc}{do_char}{c\, event}
The user typed character \var{c}. The complete details of the event
can be found in the \var{event} structure. This method can also be
provided in a \code{Window} object, which overrides the
application-wide handler if the window is frontmost.
\end{funcdesc}

\begin{funcdesc}{do_dialogevent}{event}
Called early in the event loop to handle modeless dialog events. The
default method simply dispatches the event to the relevant dialog (not
through the the \code{DialogWindow} object involved). Override if you
need special handling of dialog events (keyboard shortcuts, etc).
\end{funcdesc}

\subsection{Window Objects}

Window objects have the following methods, among others:

\renewcommand{\indexsubitem}{(Window method)}

\begin{funcdesc}{open}{}
Override this method to open a window. Store the MacOS window-id in
\code{self.wid} and call \code{self.do_postopen} to register the
window with the parent application.
\end{funcdesc}

\begin{funcdesc}{close}{}
Override this method to do any special processing on window
close. Call \code{self.do_postclose} to cleanup the parent state.
\end{funcdesc}

\begin{funcdesc}{do_postresize}{width\, height\, macoswindowid}
Called after the window is resized. Override if more needs to be done
than calling \code{InvalRect}.
\end{funcdesc}

\begin{funcdesc}{do_contentclick}{local\, modifiers\, event}
The user clicked in the content part of a window. The arguments are
the coordinates (window-relative), the key modifiers and the raw
event.
\end{funcdesc}

\begin{funcdesc}{do_update}{macoswindowid\, event}
An update event for the window was received. Redraw the window.
\end{funcdesc}

\begin{funcdesc}{do_activate}{activate\, event}
The window was activated (\code{activate==1}) or deactivated
(\code{activate==0}). Handle things like focus highlighting, etc.
\end{funcdesc}

\subsection{DialogWindow Objects}

DialogWindow objects have the following methods besides those of
\code{Window} objects:

\renewcommand{\indexsubitem}{(DialogWindow method)}

\begin{funcdesc}{open}{resid}
Create the dialog window, from the DLOG resource with id
\var{resid}. The dialog object is stored in \code{self.wid}.
\end{funcdesc}

\begin{funcdesc}{do_itemhit}{item\, event}
Item number \var{item} was hit. You are responsible for redrawing
toggle buttons, etc.
\end{funcdesc}

